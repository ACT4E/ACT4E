\documentclass[10pt, article, one side]{memoir}

\usepackage{multicol} % multiple columns
\usepackage{textcomp}
\usepackage{titlesec}
\usepackage{fullpage}
\usepackage[T1]{fontenc}
\usepackage[utf8]{inputenc}
\usepackage{beamerarticle}
\usepackage{dutchcal}
\usepackage{newpxtext}
\usepackage{verbatim}
\usepackage{fontawesome}
\usepackage{booktabs}
\usepackage{enumerate}
\usepackage{longtable}
\usepackage{todonotes}
\usepackage{pgfplots}
%\usepackage{ltablex}
\usepackage[english]{babel} % language
\definecolor{darkblue}{rgb}{0.0, 0.0, 0.55}
\usepackage{hyperref}
\hypersetup{
    colorlinks,
    linkcolor={darkblue},
    citecolor={darkblue},
    urlcolor={darkblue}}

\thispagestyle{empty}
\usepackage{xltabular}



\begin{document}

    \begin{minipage}{0.45\linewidth}
        \includegraphics[width=0.9\linewidth]{pics/ethlogo.pdf}
        \vfill
    \end{minipage}
    \begin{minipage}{0.5\linewidth}
        \begin{flushright}
            \begin{tabular}{l}
                Andrea Censi, Jonathan Lorand, Gioele Zardini \\
                ETH Zurich \\
                Department of Mechanical and Process Engineering \\
                Institute for Dynamic Systems and Control \\
                Sonneggstrasse 3, ML K42.2, \\
                8092 Zurich, Switzerland
            \end{tabular}
        \end{flushright}
    \end{minipage}

    \vspace{1cm}

    \noindent MIT Press\\
    \noindent One Broadway\\
    \noindent 12th Floor \\
    \noindent Cambridge, MA 02142\\[+7pt]

    \todo[inline]{Fill cover letter}

    \noindent Yours sincerely,

    \noindent \hspace{10cm} Andrea Censi, Jonathan Lorand, Gioele Zardini

    \newpage

    \chapter{The prospectus}
    Title: Applied Category Theory for Engineering
    \section{Brief description}
    In many domains of engineering and applied sciences, it is advantageous to think explicitly about abstraction and compositionality, to improve both the understanding of problems, and the design of solutions.
    However, the type of math which could be useful to applications is not traditionally taught.
    Applied Category Theory is a recent field of mathematics which could help a lot, but it is quite unreachable by non-mathematicians.
    While several valuable resources for learning applied category theory have emerged recently, they fail to meet the two essential criteria of a) being approachable from an engineering background and b) highlighting how applied category theory can be used to formalize and solve concrete problems.

    This book will fill this gap, by illustrating how applied category theory can be effectively leveraged to address the design and analysis of complex systems, through the “compositionality way of thinking”.
    The book adopts a computational and constructive approach: each concept is accompanied by pen-and-paper exercises, as well as computer-based (Python-based) ones.
    Furthermore, each section provides exhaustive video content (in the style of MOOCs), explaining concepts and providing further examples.

    \section{Outstanding features}
    The book is unique in at least five aspects.
    \begin{itemize}
        \item It is specifically written \textbf{for engineers and practitioners} (unlike existing efforts, as described in \todo{add ref}), and applied category theory is just the means toward practical tools for complex system analysis and design.
              This implies that the presentation of materials sometimes diverges from the usual way category theory is taught, and some common concepts will be de-emphasized in favor of more obscure concepts that find extensive use in applications.
              Furthermore, the book is \textbf{self-contained} for readers with a basic background in engineering.
              The book culminates presenting a success story of the aforementioned “compositionality way of thinking”, presenting a monotone theory of co-design and its application to engineering design optimization problems.
        \item It uses an \textbf{inductive exposition} style, rather than a deductive explanation style.
              Indeed, in a typical mathematical exposition of category theory, one usually defines a general mathematical structure, and then gives several specific examples.
              Instead, in this book, we first build up the examples as something that is interesting per se, and then we show how they can all be instances of the same general concept.
              This way, the general concept is well motivated.
              The path laid by the book is one of \emph{spiral learning}.
        \item The book follows a \textbf{computational approach}: in addition to classic pen-and-paper exercises, the reader can implement various constructions in Python, and leverage an auto-grading system to get feedback.
              Once more, this allows one to fully grasp the practical advantages promoted by the presented ``compositionality way of thinking''.
        \item The book uses a \textbf{new pedagogy of colors} to aid in the parsing of formulas and diagrams (see attached example).
              We also color the various composition operations.
              Typically, applied category theory features many different quantities, which are used in different contexts: our approach, which is colorblind-friendly (one of the authors is colorblind), helps the reader understanding the various concepts introduced.

              \begin{center}
                  \todo[inline]{add use of colors table}
              \end{center}
        \item Every section of the book features \textbf{dedicated video content} (e.g., explanations of concepts and step-by-step solutions of exercises) in the style of MOOCs.
              We are in the process of creating exhaustive content for a MOOC.
    \end{itemize}

    \section{Competition}
    \todo[inline]{Re-elaborate the full section}
    One can categorize books on the subject in at least three groups.
    First, there are regular category theory books, including:
    \begin{itemize}
        \item \emph{Categories for the Working Mathematician, Mac Lane, Graduate Texts in Mathematics, Springer, 1978};
        \item \emph{Categories and Sheaves, Kashiwara and Schapira, Grundlehren der mathematischen Wissenschaften, Springer, 2006};
        \item \emph{Category Theory in Context, Riehl, Aurora: Dover Modern Math Originals, 2016};
        \item \emph{Category Theory, Awodey, Oxford University Press, 2010};
        \item \emph{The Joy of Abstraction: An Exploration of Math, Category Theory, and Life, Cheng, Cambridge University Press, 2022};
    \end{itemize}
    These books are typically hard to approach for mathematicians (from which the joke: “What’s the best category theory book?”, “The third”), and out of reach for engineers.
    Furthermore, they typically do not feature practical applications.

    Second, there are books on applied category theory, including:
    \begin{itemize}
        \item \emph{Picturing Quantum Processes: A First Course in Quantum Theory and Diagrammatic Reasoning, Coecke and Kissinger, Cambridge University Press, 2017};
        \item \emph{Category Theory for the Sciences, Spivak, MIT Press, 2014};
        \item \emph{An Invitation to Applied Category Theory: Seven Sketches in Compositionality, Fond and Spivak, Cambridge University Press, 2019};
    \end{itemize}

    While definitely offering a more approachable alternative, such books are still (intentionally or not) inherently targeted to mathematicians, lacking practical applications, and, most importantly, an application-driven selection of topics.

    Finally, our book is specifically written \textbf{for engineers and practitioners}, and applied category theory is just the means toward practical tools for complex system analysis and design.
    Motivated by practical engineering examples, the presentation of materials sometimes diverges from the usual way category theory is taught, and some common concepts are de-emphasized in favor of more obscure concepts that find extensive use in applications.
    Furthermore, the book is self-contained for readers with a basic background in engineering.

    \section{Apparatus}
    The book will include examples, pen-and-paper exercises, and computational (Python-based) exercises.
    Solutions to pen-and-paper exercises will be provided, and the computational exercises will be automatically graded through a system we have developed during dedicated classes at ETH Zurich.
    The book will include glossaries, bibliography, and appendices.
    Furthermore, we have created a “developer” version of the book, for instructors to use in their classes.

    \section{Audience}
    The book is intended for engineers, and we assume basic knowledge of algebra at the level of a bachelor’s degree in engineering/computer science and is self-contained.
    In the past, we have organized both ETH Zurich lectures, as well as international, online ones.
    From our records, we highlight that professionals were particularly interested in the subject presented in this book.
    The book has a rigorous character, and provides both descriptive and quantitative explanations of various ideas.

    \section{Authors}
    \todo[inline]{See if we need more exhaustive bios}
    \noindent \textbf{Dr.
        Andrea Censi}, Senior Research Scientist\\
    Institute for Dynamic Systems and Control, ETH Zurich\\
    E-mail: \href{mailto:acensi@ethz.ch}{acensi@ethz.ch}\\
    Website: \href{https://censi.science}{censi.science}\\
    \emph{Bio:} Andrea Censi is deputy director of the Dynamic Systems and Control chair at ETH Zurich.
    He obtained a Ph.D.
    from Caltech.
    Previously, he has been a research scientist at MIT and Director of Research at Aptiv Mobility (now Motional).
    He is president of the Duckietown Foundation.
    \\

    \noindent \textbf{Dr.
        Jonathan Lorand}, Postdoctoral researcher\\
    Institute for Dynamic Systems and Control, ETH Zurich\\
    E-mail: \href{mailto:jlorand@ethz.ch}{jlorand@ethz.ch}.
    \\
    Website: \href{http://lorand.earth/math/}{lorand.earth/math/}\\
    \emph{Bio:}
    Jonathan is a postdoctoral researcher working as a mathematician at the Institute for Dynamic Systems and Control at ETH Zurich.
    He obtained his PhD in mathematics from the University of Zurich in 2020, working in symplectic geometry and category theory.
    Since then he is focused on applied category theory, as well on the topic of transdisciplinarity, in which he will obtain an MA in 2024 from the Zurich University of the Arts.
    \\

    \noindent \textbf{Gioele Zardini}, Ph.
    D.
    Candidate\\
    Institute for Dynamic Systems and Control, ETH Zurich\\
    E-mail: \href{mailto:gzardini@ethz.ch}{gzardini@ethz.ch}.
    \\
    Website: \href{https://gioele.science}{gioele.science}\\
    \emph{Bio:}
    Gioele is a Ph.D.
    candidate at ETH Zurich, and incoming faculty at Massachusetts Institute of Technology.
    He received his BSc.
    and MSc. in Mechanical Engineering with focus in Robotics, Systems and Control from ETH Zurich in 2017 and 2019, respectively.
    He spent time in Singapore as a researcher at nuTonomy (now Motional), at Stanford University and at MIT.
    \\

    All three authors have experience in teaching various courses (both at the undergraduate and graduate level) at ETH Zurich and University of Zurich.

    \section{Market considerations}
    One of the great engineering challenges of this century is dealing with the design of complex systems, and, as explained above, the math traditionally thought is not practical, and does not help reasoning about them.
    The interest for applied category theory has been growing dramatically over the past few years, taking a small community of mathematicians and bringing it to a worldwide movement, with representatives in academia and in industry.

    We believe that the course presented by this book will be taught in at least 50\% of the major universities in the next 7 years, as part of every engineering program.

    Over the past few years, we organized online classes, as well as dissemination workshops at large international conferences (e.g., in robotics, systems and control, and intelligent transportation systems), and we have created a large, diverse network of interested people.
    We are curating a dedicated mailing list, with people following our steps in this journey.

    \section{Status of the book}
    The first version of the book is complete, and needs most probably adaptation to a common format from the publisher.
    We expect the complete manuscript to be completed within the next six months or less.
    The manuscript will reach around 200’000 words.

    The book features at least two figures on each page (vector figures, typically mathematical diagrams), and one full color photograph per chapter.

    You can find the current version of the book at \href{https://z7-stage-act4e-book.zuper.ai/sync/ACT4E/ACT4E/alphubel-prod/build/last/build-public/ACT4E-public.pdf}{https://z7-stage-act4e-book.zuper.ai/sync/ACT4E/ACT4E/alphubel-prod/build/last/build-public/ACT4E-public.pdf}.

    \section{Reviewers}
    \todo[inline]{choose order wisely}
    We suggest the following list of reviewers:
    \begin{itemize}
        \item Prof.
              Michael Johnson, Macquarie University, \href{mailto:Michael.Johnson@mq.edu.au}{Michael.Johnson@mq.edu.au};
        \item Prof.
              John Baez, University of California, Riverside and Topos Institute, \href{mailto:john.baez@ucr.edu}{john.baez@ucr.edu};
        \item Prof.
              Daniel Koditschek, University of Pennsylvania, \href{mailto:kod@seas.upenn.edu}{kod@seas.upenn.edu};
        \item Prof.
              Aaron Ames, California Institute of Technology, \href{mailto:ames@cds.caltech.edu}{ames@cds.caltech.edu};
        \item Prof.
              Paulo Tabuada, University of California, Los Angeles, \href{mailto:tabuada@ee.ucla.edu}{tabuada@ee.ucla.edu};
        \item Prof.
              Hadas Kress-Gazit, Cornell University, \href{mailto:hadaskg@cornell.edu}{hadaskg@cornell.edu};
        \item Prof.
              Gregory Chirikjian, National University of Singapore, \href{mailto:mpegre@nus.edu.sg}{mpegre@nus.edu.sg};
        \item Dr.
              Paolo Perrone, University of Oxford, \href{mailto:paolo.perrone@cs.ox.ac.uk}{paolo.perrone@cs.ox.ac.uk};
        \item Dr.
              David Spivak, Topos Institute, \href{mailto:dspivak@gmail.com}{dspivak@gmail.com};
        \item Dr.
              Brendan Fond, Topos Institute, \href{mailto:brendan@topos.institute}{brendan@topos.institute};
        \item \todo{others?}
    \end{itemize}

    \chapter{Table of contents}
    Attached, you find the current table of contents.

    \chapter{Sample chapters}
    Attached, you find sample chapters.
\end{document}
