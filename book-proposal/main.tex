\documentclass[10pt, article, one side]{memoir}

\usepackage{multicol} % multiple columns
\usepackage{textcomp}
\usepackage{titlesec}
\usepackage{fullpage}
\usepackage[T1]{fontenc}
\usepackage[utf8]{inputenc}
\usepackage{beamerarticle}
\usepackage{dutchcal}
\usepackage{newpxtext}
\usepackage{verbatim}
\usepackage{fontawesome}
\usepackage{booktabs}
\usepackage{enumerate}
\usepackage{longtable}
\usepackage{todonotes}
\usepackage{pgfplots}
%\usepackage{ltablex}
\usepackage[english]{babel} % language
\definecolor{darkblue}{rgb}{0.0, 0.0, 0.55}
\usepackage{hyperref}
\hypersetup{
    colorlinks,
    linkcolor={darkblue},
    citecolor={darkblue},
    urlcolor={darkblue}}
\usepackage{cleveref}

\thispagestyle{empty}
\usepackage{xltabular}

\begin{document}

    \begin{minipage}{0.45\linewidth}
        \includegraphics[width=0.9\linewidth]{pics/ethlogo.pdf}
        \vfill
    \end{minipage}
    \begin{minipage}{0.5\linewidth}
        \begin{flushright}
            \begin{tabular}{l}
                Andrea Censi, Jonathan Lorand, Gioele Zardini \\
                ETH Z\"urich \\
                Department of Mechanical and Process Engineering \\
                Institute for Dynamic Systems and Control \\
                Sonneggstrasse 3, ML K42.2, \\
                8092 Z\"urich, Switzerland
            \end{tabular}
        \end{flushright}
    \end{minipage}

    \vspace{1cm}

    \noindent MIT Press\\
    \noindent One Broadway\\
    \noindent 12th Floor \\
    \noindent Cambridge, MA 02142\\[+7pt]

    \todo[inline]{Fill cover letter}
    Dear editor,\\[-5pt]

    We are pleased to submit a proposal for the book titled ``Applied Category Theory for Engineering'', authored by \text{Dr.}
    Andrea Censi, \text{Dr.}
    Jonathan Lorand, and Gioele Zardini, currently employed at the Institute for Dynamic Systems and Control, ETH Zurich.
    We believe that this book presents a unique and valuable contribution to the fields of applied category theory and engineering, providing an accessible resource for a broad audience from engineering and applied mathematics, both in academia and in industry.
    \\[-7pt]
    
    The book aims to bridge the gap between the need for abstract, compositional mathematical tools to address the complex challenges of contemporary engineering, and the lack of accessible, comprehensive resources which teach such tools in a way that is tailored to engineering. Besides providing a self-contained, pedagogical exposition of the essentials of category theory, our text covers various more advanced topics which are particularly relevant for engineering, as well as extended examples of applications, particularly in the area of complex systems analysis and design. \\[-7pt]
     
    Features of our book include the innovative use of colors in mathematical notation, a wealth of examples, exercises and quizzes, companion code exercises in python to highlight the computational side of the theory, and dedicated external multimedial resources which are linked with the book via QR-codes. A first-draft version of the manuscript is currently complete, and has a length of approximately 200,000 words. Considering the growing demand for applied category theory resources in engineering, we believe this book will be highly valuable both in university engineering programs and as a resource for practitioners.
    \\[-7pt]
    
    We look forward to the opportunity to discuss this project further with you and explore a potential collaboration with your publishing house.
    Please do not hesitate to reach out if you require any additional information or have any questions.

\


    \noindent Yours sincerely,\\

    \noindent \hspace{8cm}
    Andrea Censi, Jonathan Lorand, Gioele Zardini

    \newpage

    \chapter{The prospectus}
    Title of the book: \textbf{Applied Category Theory for Engineering}
    \section{Brief description}
    In many engineering and applied science domains, it is advantageous to think explicitly about abstraction and compositionality, to improve both the understanding of problems, and the design of solutions.
    However, the mathematics which could be useful to such applications is not traditionally taught.
    Applied Category Theory, a relatively new field of mathematics, has the potential to provide significant support in these areas.
    Unfortunately, it often remains out of reach for non-mathematicians.
    While valuable resources for learning applied category theory have emerged recently, they fall short in meeting the two essential criteria of a) being accessible for those with an engineering background, and b) demonstrating of applied category theory can be employed to formalize and solve real-world problems.

    This book aims to bridge this gap by illustrating the effective utilization of applied category theory in addressing the design and analysis of complex systems through the “compositionality way of thinking”.
    It adopts a computational and constructive approach: each concept is accompanied by pen-and-paper exercises, as well as computer-based ones using Python.
    To further enhance the learning experience, each section is accompanied by comprehensive video content in the style of Massive Open Online Courses (MOOCs).
    These videos provide clear explanations of concepts and offer additional examples to deepen understanding.

    By combining a practical approach, diverse exercises, and video content, this book strives to make applied category theory accessible and applicable to engineers, empowering them to tackle complex problems and foster innovative solutions.

    \section{Outstanding features}
    Here are five outstanding features of our book that we wish to highlight.
    \begin{enumerate}
        \item It is specifically \textbf{tailored for engineers and practitioners}, setting it apart from existing efforts in the field (as described in \cref{sec:competition}).
              In this context, applied category theory is just the means toward practical tools for complex system analysis and design.
              This implies that the presentation of materials sometimes diverges from the usual way category theory is taught, and some common concepts will be de-emphasized in favor of more obscure concepts that find extensive use in applications.
              Moreover, the book is deisgned to be \textbf{self-contained}, catering to readers who possess a basic engineering background.
              The book culminates with a success story of the aforementioned “compositionality way of thinking”, presenting a monotone theory of co-design and its application to engineering design optimization problems.
        \item It adopts an \textbf{inductive exposition} style, which differs from the deductive explanation style commonly used in mathematical expositions of the subject.
              Indeed, in a typical mathematical exposition of category theory, one usually defines a general mathematical structure, and then gives several specific examples.
              Instead, in this book, the examples are presented initially as intriguing and valuable in their own right.
              The focus is on building up these examples to demonstrate their individual significance, and then connecting them to a broader, overarching general concept.
              The book facilitates the path of \textbf{spiral learning}.
              As readers progress through the content, they continuously revisit and deepen their understanding of the examples, gradually uncovering the underlying general concepts.
              This approach allows for a comprehensive and interconnected understanding of the subject matter.
        \item The book emphasizes not only theory but also \textbf{computation}. In particular, the book discusses how to implement various constructions in Python and provides extensive programming exercises which reinforce readers' understanding of the concepts discussed and offers them an opportunity to gain hands-on, practical experience. The programming exercises are accompanies by an auto-grading system, which offers valuable feedback on the implemented code, allowing readers to assess their progress and ensure their solutions meet the intended objectives. By combining theory with practical implementation, the book fosters a \textbf{comprehensive understanding} of the practical benefits that arise from adopting the ``compositional way of thinking''.

        \item   The book employs an \textbf{innovative pedagogy}, which utilizes \textbf{colors} to facilitate the parsing of formulas and diagrams.
              This visual approach enhances the reader's comprehension by applying disting colors to represent different elements and composition operations within applied category theory.
              In traditional treatments of this subject, a wide range of quantities is introduced, each with specific contexts of use.
              To address this potential complexity, our color scheme is thoughtfully designed to ensure accessibility for all readers, including those with colorblindness (as one of the authors is colorblind, special care has been taken to create a colorblind-friendly system).
              This visual representation, exemplified in \cref{fig:use-colors}, adds clarity and supports a more intuitive grap of the subject matter.

              \begin{figure}[h]
                  \begin{center}
                      \includegraphics[width=0.3\linewidth]{pics/use_colors}
                  \end{center}
                  \caption{Colors used to represent different concepts. \label{fig:use-colors}}
              \end{figure}
        \item The book goes beyond traditional written content by incorporating \textbf{dedicated video resources} into every section.
              These videos serve as supplementary materials, providing additional support and guidance to the readers.
              In the style of Massive Open Online Courses (MOOCs), these videos offer dynamic explanations of concepts and step-by-step solutions to exercises.
              Recognizing the value of multimedia in enhancing the learning experience, the authors have invested significant effort into creating such content.
              This ongoing endeavor aims to provide learners with an extensive and multimodal collection of resources that complement and expand upon the material covered in the book
    \end{enumerate}

    \section{Competition}
    \label{sec:competition}

    Our book stands, to some degree, in competition with books which are general introductions to category theory or to applied category theory.
    Below we discuss the main titles of this kind which exist today, in order to compare and contrast them to our book.
    Our book is unique among these existing books because it has the following two qualities simultaneously:
    \begin{enumerate}
        \item Approachable and self-contained: Our book takes a deliberate, slow-paced approach that is accessible to readers without formal training in pure mathematics.
              It is written in a clear and engaging style, catering specifically to engineering students and practitioners.
              In particular, we often use engineering examples to motivate and introduce concepts, and our choices in topics and exposition sometimes diverge from more classical mathematical treatments, in order to address the particular relevance of certain topics for engineering applications
              By presenting the material in a self-contained manner, we ensure that readers can follow along without the need of extensive mathematical background.

        \item Application-oriented: Our book is distinct in its focus on applying category theory to engineering.
              We go beyond abstract concepts by incorporating real-world engineering examples to motivate and illustrate the theory.
    \end{enumerate}

    \paragraph{Categories for the Working Mathematician, \emph{S.
            MacLane, Graduate Texts in Mathematics, Springer, 1978}~\cite{mac2013categories}}
    \

    This was the early major expository monograph on category theory and for a very long time the go-to book for mathematicians to learn category theory.
    Written in a terse and dense style, it is comprehensive, insightful, and a slow and challenging read, even for mathematicians.
    As far as introductory category theory texts go, this is likely the least accessible one.
    It focuses on only on pure mathematics, and assumes a good deal of prior mathematical experience.

    \paragraph{Conceptual Mathematics, 2nd Ed., \emph{S.
            H.
            Schanuel, F.
            W.
            Lawvere, Cambridge University Press, 2009}~\cite{lawvere2009conceptual}}
    \

    This is one of the early expository monographs on category theory which aims to be broadly accessible and convey not only definitions and theorems, but also the spirit and ways of thinking used in category theory.
    It uses a non-traditional, conversational, lecture-based style, and focuses on basic conceptual insights rather than covering a large amount of technical material in a formally structured way.
    The book uses simple and insightful examples, both of a `pure' and `applied' flavor.
    However, it is still not a text which is mainly focused on \emph{applied} category theory, let alone one that centers engineering.
    Nor does it cover the amount and scope of material that our book does.

    \paragraph{Category Theory, 2nd Ed., \emph{S.
            Awodey, Oxford University Press, 2010}~\cite{awodey2010category}}
    \

    This was the first major `mathematics-style' textbook after MacLane's \emph{Categories for the working mathematician}.
    It covers a good deal of standard category theory material, and it aims to be much much more accessible than MacLane's book, both in terms of prerequisites needed and in terms of its pedagogical style.
    It is addressed not only to mathematics students, but also expressly to readers with a background in computer science, logic, linguistics, or philosophy.
    For this reason, the book covers material for example on cartesian closed categories and the lambda calculus.
    Despite its broader audience, the book remains, however, one written in a mathematical style.
    Furthermore, the applied areas that it does address are not in engineering.
    Our book differs both in scope and in style.

    \paragraph{Basic Category Theory, \emph{T.
            Leinster,  Cambridge University Press, 2014}~\cite{leinster2014basic}}
    \

    This is an excellently-written, introductory book on category theory.
    Relatively short in length and covering less material than, for example, Riehl's \emph{Category theory in context} (see below), this book is aimed at mathematicians and provides a concise, pedagogical, and elegant exposition of core category theory topics for such an audience.
    It does not however cover some theoretical topics which are essential for applied category theory (for example, monoidal categories), nor does it aim to discuss examples and applications outside of mathematics.

    \paragraph{Category Theory for the Sciences, \emph{D.
            I.
            Spivak, MIT Press, 2014}~\cite{spivak2014category}}
    \

    This text is arguably the first book on applied category theory proper and a pioneering contribution to the field.
    It is aimed both at scientists and at mathematicians interested in using category theory in applied domains.
    It is written to be accessible to a broad audience, and does so by teaching via examples and exercises, rather than theorems and proofs.
    The main differences with our book are scope -- our book aims to cover a much larger amount of category theory material and applications thereof -- and the fact that our book is focused expressly on engineering.
    Furthermore, although we provide many examples and exercises, our book has theorems and proofs, too.
    This is because some of our intended readers will want to understand the theory in this level of formal depth, and it is important for their work to learn to do so, to some degree.
    Nevertheless, our book remains accessible for those who wish to skip over proofs.

    \paragraph{Category Theory in Context, \emph{E.
            Riehl, Aurora: Dover Modern Math Originals, 2016}~\cite{riehl2017category}}
    \

    This is another well-known textbook for (pure) category theory, written in a modern style and conveying some of the categorical `way of thinking' alongside the formal mathematics.
    Although certainly more accessible than MacLane, it is nevertheless aimed at readers with a moderate level of mathematical training and maturity, and is also comparatively dense, covering a good deal of material relative to its length.
    It is a wonderful book for mathematicians which emphasizes how to use category theory fruitfully inside of mathematics; however it is not realistically accessible for readers without mathematical training, nor does it emphasize applications outside of mathematics.

    \paragraph{An Invitation to Applied Category Theory: Seven Sketches in Compositionality, \emph{B.
            Fong, D.
            I.
            Spivak, Cambridge University Press, 2019}~\cite{fong2019invitation}}
    \

    This book is, in two respects, very close to fulfilling our aims: it is written in an accessible style intended for a very broad audience and it is focused on applications of category theory outside of mathematics.
    The two main differences to our book are that this one is not focused mainly on engineering applications, and it is not as comprehensive as ours.
    As the word `sketch' suggests, some of the topics in Fong and Spivak's book are only touched upon briefly, in order to introduce an important idea and whet the readers appetite to learn more through other resources.
    In contrast, our book offers more thorough treatments of many important topics, illustrating and spelling out details in depth so that readers without much mathematical training are able to follow even technical aspects step-by-step and connect the technical theory with examples and applications.

    \paragraph{The Joy of Abstraction: An Exploration of Math, Category Theory, and Life, \emph{E.
            Cheng, Cambridge University Press, 2022}~\cite{cheng2022joy}}
    \

    This book is aimed at a very broad audience -- it does not assume any university-level mathematical training -- yet it builds up to introducing the core essential material of category theory, including technical details, so that readers might then be able to read other introductory books on category theory without struggling too much.
    It is accessible and pedagogical, and conveys also a conceptual and intuitive understanding of the categorical way of thinking.
    A main difference to our book is that, while Cheng's book uses examples that include various aspects of daily life, it is not focused on applying category theory in the sciences or in engineering specifically.
    Our book, in contrast, covers more -- and other -- kinds of material, both in terms of theory and in terms of examples and exercises.

    \

    We are aware of a few further books (\cite{Grandis}, \cite{Roman}, \cite{Simmons}) which are also mathematics-focused introductions to basic category theory.
    Because they are subsumed in content and expository nature by the books discussed above, we do not see the need to describe these in detail.
    However, we mention them for completeness' sake.

    Other books that are related to ours, but do not stand in competition, are books which focus explicitly on applications of category theory in specific domains other than engineering, and include a somewhat accessible introduction to category theory (most specialized books do not).
    The following books, for example, cover topics of category theory in computer science (\cite{BarrWells},  \cite{Milewski}, \cite{Pierce}) and physics (\cite{CoeckeKisssinger}).

    \section{Apparatus}
    The book encompasses a comprehensive set of learning resources to enhance understanding and engagement.
    It features a range of examples, pen-and-paper exercises, and computational exercises implemented in Python.
    Solution to the pen-and-paper exercises will be readily available, while the computational exercises will be automatically graded using a system that we have developed and refined during dedicated classes at ETH Zurich.

    In addition to these valuable learning components, the book includes glossaries, a bibliography, and appendices to further support readers in their exploration of the subject matter.

    To cater to instructors and their teaching needs, we have also developed a ``developer'' version of the book.
    This version is specifically designed for instructors to incorporate into their classes, providing them with additional resources and tools to effectively teach the material.

    Through the combination of examples, exercises, automated grading, supplementary resources, and a dedicated version for instructors, our book aims to create a comprehensive and supportive learning experience that fosters engagement and facilitates effective instruction.

    \section{Audience}
    The book is specifically tailored for engineers, assuming a foundational understanding of algebra equivalent to a bachelor's degree in engineering or computer science.
    It is designed to be self-contained, ensuring that readers can follow along without the need for additional references.

    In the past, we have successfully organized lectures at ETH Zurich, as well as online sessions that reached an international audience.
    Through these experiences, we have observed a strong interest from students and professionals in the subject matter covered in this book.
    This highlights the practical relevance and applicability of the concepts discussed.

    Maintaining a rigorous approach, the book combines descriptive and quantitative explanations to elucidate various ideas.
    It strikes a balance between providing clear, intuitive descriptions of concepts and offering a quantitative analysis to deepen understanding.
    This multifaceted approach ensures that readers gain a comprehensive and well-rounded grasp of the material.

    With its focus on engineering, self-contained nature, track record of successfull lectures, and balanced approach to explanations, this books serves as an invaluable resource for professionals and engineers seeking a rigourous yet accessible exploration of applied category theory.

    \section{Authors}
    \todo[inline]{See if we need more exhaustive bios}
    \noindent \textbf{Dr.
        Andrea Censi}, Senior Research Scientist\\
    Institute for Dynamic Systems and Control, ETH Zurich\\
    E-mail: \href{mailto:acensi@ethz.ch}{acensi@ethz.ch}\\
    Website: \href{https://censi.science}{censi.science}\\
    \emph{Bio:} Andrea Censi is deputy director of the Dynamic Systems and Control chair at ETH Zurich.
    He obtained a Ph.D.
    from Caltech.
    Previously, he has been a research scientist at MIT and Director of Research at Aptiv Mobility (now Motional).
    He is president of the Duckietown Foundation.
    \\

    \noindent \textbf{Dr.
        Jonathan Lorand}, Postdoctoral researcher\\
    Institute for Dynamic Systems and Control, ETH Zurich\\
    E-mail: \href{mailto:jlorand@ethz.ch}{jlorand@ethz.ch}.
    \\
    Website: \href{http://lorand.earth/math/}{lorand.earth/math/}\\
    \emph{Bio:}
    Jonathan is a postdoctoral researcher working as a mathematician at the Institute for Dynamic Systems and Control at ETH Zurich.
    He obtained his PhD in mathematics from the University of Zurich in 2020, working in symplectic geometry and category theory.
    Since then he is focused on applied category theory, as well on the topic of transdisciplinarity, in which he will obtain an MA in 2024 from the Zurich University of the Arts.
    \\

    \noindent \textbf{Gioele Zardini}, Ph.
    D.
    Candidate\\
    Institute for Dynamic Systems and Control, ETH Zurich\\
    E-mail: \href{mailto:gzardini@ethz.ch}{gzardini@ethz.ch}.
    \\
    Website: \href{https://gioele.science}{gioele.science}\\
    \emph{Bio:}
    Gioele is a Ph.D.
    candidate at ETH Zurich, and incoming faculty at Massachusetts Institute of Technology.
    He received his BSc.
    and MSc. in Mechanical Engineering with focus in Robotics, Systems and Control from ETH Zurich in 2017 and 2019, respectively.
    He spent time in Singapore as a researcher at nuTonomy (now Motional), at Stanford University and at MIT.
    \\

    All three authors have experience in teaching various courses (both at the undergraduate and graduate level) at ETH Zurich and University of Zurich.

    \section{Market considerations}
    One of the great engineering challenges of our time revolves around the design of complex systems.
    However, the conventional mathematics taught traditionally falls short in providing practical solutions and aiding reasoning in this context.
    In recent years, there has been a remarkable surge of interest in applied category theory.
    What started as a small community of mathematicians has rapidly evolved into a global movement, with representatives from academia and industry alike.
    This growing interest underscores the recognition of the potential that applied category theory holds in addressing complex system design challenges.

    We firmly believe that the course presented in this book will become a staple in at least 50\% of the major universities within the next seven years, becoming an integral part of every engineering program.
    This projection reflects the growing recognition and importance of applied category theory in engineering disciplines.

    Throughout this journey, we have organized online classes and conducted dissemination workshops ar prominent international conferences, focusing on domains such as robotics, systems and contorl, and intelligent transportation systems.
    These efforts have enabled us to establish a vast and diverse network of individuals who share a keen interest in this subject.
    To further nurture this community, we are actively curating a dedicated mailing list, connecting individuals who are following our progress and developments closely.

    By fostering collaboration, disseminating knowledge, and nurturing a vibrant community, we aim to drive the widespread adoption of applied category theory in engineering, transforming the way complex systems are approached and designed.

    \section{Status of the book}
    The initial version of the book is close to completion and requires adaptation to a standard format by the publisher.
    We anticipate that the final manuscript will be fully completed within the next nine to twelve months, if not sooner.
    It is expected to encompass approximately 200,000 words, offering a comprehensive exploration of the subject matter.

    Visual aids play a prominent role in the book, with a minimum of two figures included on each page.
    These figures are vector-based, typically comprising mathematical diagrams which enhance understanding and visualization.
    Additionally, each chapter is enriched with a captivating full-color photograph, adding an engaging visual element to complement the textual content.

    The inclusion of abundant visual elements serves to enhance the reader's comprehension and create an aesthetically pleasing reading experience.
    By effectively utilizing figures and photographs, the book provides both clarity and visual appeal, contributing to an immersive learning journey.

    You can find the current version of the book \href{https://z7-stage-act4e-book.zuper.ai/sync/ACT4E/ACT4E/alphubel-prod/build/last/build-public/ACT4E-public.pdf}{here}.

    \section{Reviewers}
    \todo[inline]{choose order wisely}
    We suggest the following list of reviewers:
    \begin{itemize}
        \item Prof.
              Michael Johnson, Macquarie University, \href{mailto:Michael.Johnson@mq.edu.au}{Michael.Johnson@mq.edu.au};
        \item Prof.
              John Baez, University of California, Riverside and Topos Institute, \href{mailto:john.baez@ucr.edu}{john.baez@ucr.edu};
        \item Prof.
              Daniel Koditschek, University of Pennsylvania, \href{mailto:kod@seas.upenn.edu}{kod@seas.upenn.edu};
        \item Prof.
              Aaron Ames, California Institute of Technology, \href{mailto:ames@cds.caltech.edu}{ames@cds.caltech.edu};
        \item Prof.
              Paulo Tabuada, University of California, Los Angeles, \href{mailto:tabuada@ee.ucla.edu}{tabuada@ee.ucla.edu};
        \item Prof.
              Hadas Kress-Gazit, Cornell University, \href{mailto:hadaskg@cornell.edu}{hadaskg@cornell.edu};
        \item Prof.
              Gregory Chirikjian, National University of Singapore, \href{mailto:mpegre@nus.edu.sg}{mpegre@nus.edu.sg};
        \item Dr.
              Paolo Perrone, University of Oxford, \href{mailto:paolo.perrone@cs.ox.ac.uk}{paolo.perrone@cs.ox.ac.uk};
        \item Dr.
              David Spivak, Topos Institute, \href{mailto:dspivak@gmail.com}{dspivak@gmail.com};
        \item Dr.
              Brendan Fong, Topos Institute, \href{mailto:brendan@topos.institute}{brendan@topos.institute};
        \item \todo{others?}
    \end{itemize}

    \chapter{Table of contents}
    Attached, you find the current table of contents.

    \chapter{Sample chapters}
    Attached, you find sample chapters.

    \bibliographystyle{ieeetr}
    \bibliography{references}



    \begin{thebibliography}{00}

        
        \bibitem{cheng2022joy}
        E. Cheng, ... 
        
        \bibitem{CoeckeKisssinger}
        B.
        Coecke, A.
        Kissinger,
        \emph{Picturing Quantum Processes: A First Course in Quantum Theory and Diagrammatic Reasoning},
        Cambridge University Press, 2017.

        \bibitem{Grandis}
        M.
        Grandis, Category theory and applications.

        \bibitem{Roman}
        S.
        Roman, An introduction to the language of category theory.

        \bibitem{Simmons}
        H.
        Simmons, An introduction to category theory.

        \bibitem{Milewski}
        B.
        Milewski, Category theory for programmers

        \bibitem{Pierce} Pierce, Basic Category Theory for Computer Scientists

        \bibitem{BarrWells} Barr, Wells, Category Theory for Computing Science

    \end{thebibliography}

\end{document}
