%!TEX root = ../main.tex
\section{Example Section }

\begin{frame}[fragile]{Example: "distribution networks"}

\textbf{General situation}: something is distributed via a network

\

\textbf{Specific example}: electric power distributed via a power grid. 

\begin{itemize}
\item Toy model: 
\end{itemize}

\vfill
\vfill
\vfill
\vfill
\vfill
\vfill


\end{frame}

\begin{frame}[fragile]{}


To model \textbf{connectivity}: arrows

\textbf{Direction} of arrows: flow of distribution

\vfill \vfill \vfill \vfill \vfill \vfill \vfill \vfill \vfill 

\end{frame}

\begin{frame}[fragile]{}

Which consumers are connected to which power plants? Look at \textbf{paths}:


\vfill \vfill \vfill \vfill \vfill \vfill \vfill \vfill \vfill 
\end{frame}


\begin{frame}[fragile]{}

We also might want to show which high voltage are connected to each other:

\vfill \vfill \vfill \vfill \vfill \vfill \vfill \vfill \vfill 

Note: there is also a way to make these relationships symmetric.

\end{frame}


\begin{frame}[fragile]{}

The information above can also be represented as directed graph:

\vfill \vfill \vfill \vfill \vfill \vfill \vfill \vfill \vfill 

\end{frame}


\begin{frame}[fragile]{}

For comparison, a representation of a power grid taken from Wikipedia: 

\vfill \vfill \vfill \vfill \vfill \vfill \vfill \vfill \vfill 

\end{frame}



\begin{frame}[fragile]{Binary relations}

\textbf{Definition:} A (binary) \textbf{relation} from a set $X$ to a set $Y$ is a subset of $X \times Y$

\

\textbf{Example:}  $X = \{ x_1, x_2, x_3 \}$, \  $Y = \{ y_1, y_2, y_3, y_4 \}$

\

$R \subseteq X \times Y$ given by  $$R = \{ \tup{x_1, y_1}, \tup{x_2, y_3}, \tup{x_2, y_4}  \}$$


\vfill \vfill \vfill \vfill \vfill \vfill \vfill \vfill \vfill \vfill \vfill \vfill 


\end{frame}




\begin{frame}[fragile]




 \vfill 

\textbf{Notation:} If $R \subseteq X \times Y$ is a relation, we write $R: X \rightarrow Y$ or $X \overset{R}{\rightarrow} Y$. 

\

Sometimes the notation $R: X \tickar Y$ is used to emphasize that $R$ is a relation, and not a function. 


\vfill  \vfill \vfill 
 
We will see: we can think of a relation as a type of \textbf{morphism}.

\vfill \vfill \vfill \vfill 




\end{frame}




\begin{frame}[fragile]

\textbf{Example:} 
In the power grid example, we had 

\vfill \vfill \vfill \vfill \vfill \vfill 

This represents a relation 
$$X = \{ \text{plant1}, \text{plant2}, \text{plant3} \} \quad \longrightarrow \quad Y = \{ \text{HVN1}, \text{HVN2}, \text{HVN3}, \text{HVN4},  \text{HVN5} \}$$

\end{frame}







\begin{frame}[fragile]{Relations can be composed}

Suppose we have relations $R \subseteq X \times Y$ and $S \subseteq Y \times Z$, \text{i.e.} relations
$$
X \overset{R}{\longrightarrow} Y \overset{S}{\longrightarrow} Z.
$$

\

How might we compose $R$ and $S$ to obtain a relation $X \overset{R \then S}{\xrightarrow{\hspace*{1cm}}} Z$ \ ?

\vfill \vfill \vfill \vfill \vfill \vfill \vfill \vfill 

\end{frame}



\begin{frame}[fragile]

\textbf{Example:} 


\vfill \vfill \vfill \vfill \vfill \vfill 

What is the composition $R \then S$ ? 

\

Look at \textbf{paths} from $X$ to $Z$. 
\end{frame}




\begin{frame}[fragile]

\vfill \vfill \vfill \vfill \vfill \vfill 

So, $R \then S$ is this relation:

\vfill \vfill \vfill \vfill \vfill \vfill 

\end{frame}





\begin{frame}[fragile]

\textbf{Defintion:}  Let $R \subseteq X \times Y$ and $S \subseteq Y \times Z$ be relations. Their \textbf{composition} is
$$
R \then S := \{ \}
$$
which is a relation $X \rightarrow Z$. 

\

\textbf{Definition:} The category $\cat{Rel}$ of sets and relations:
\begin{itemize}
\item Objects: 
\item Homsets: given sets $X$ and $Y$, 
$$
\hom_{\cat{Rel}}(X,Y) := \mathcal{P}(X \times Y) = \text{ all subsets of $X \times Y$}
$$
\end{itemize}


\vfill \vfill \vfill \vfill \vfill \vfill 
\end{frame}






\begin{frame}[fragile]

\textbf{Example:} 


\vfill \vfill \vfill \vfill \vfill \vfill 
\end{frame}







\

\


\begin{frame}[fragile] (reflexive relations)

\textbf{Example:} In the power grid example, we also had 

\vfill \vfill \vfill \vfill \vfill \vfill \vfill 

This represents a relation 
$$Y = \{ \text{HVN1}, \text{HVN2}, \text{HVN3}, \text{HVN4},  \text{HVN5} \} \quad \longrightarrow \quad Y = \{ \text{HVN1}, \text{HVN2}, \text{HVN3}, \text{HVN4},  \text{HVN5} \}$$

\end{frame}




\begin{frame}[fragile]{}


...

\end{frame}



