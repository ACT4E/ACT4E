%%%%%%%%%%
%\begin{comment}
%\includeonly{
%chapters/005_title_page,
%chapters/010_introduction,
%chapters/015_basic_concepts,
%chapters/020_things_transforming,
%chapters/030_things_connecting,
%chapters/040_specialization,
%chapters/041_sameness,
%chapters/050_resource_composition,
%chapters/060_alternatives
%chapters/070_tradeoffs,
%chapters/075_posets_with_more,
%chapters/080_monotonicity,
%chapters/095_functors,
%chapters/090_duality_design,
%chapters/100_composing_no_structure,
%chapters/110_composing_structure,
%chapters/150_restrictions,
%chapters/115_parallel_theory,
%chapters/120_parallel_composition,
%chapters/130_design_problems,
%chapters/135_companion_and_conjoint,
%chapters/143_categorical_trace,
%chapters/146_closing_loop_codesign,
%chapters/160_ordering_dps,
%chapters/165_order_and_compositions,
%chapters/155_relations_facts,
%chapters/170_higher_order_design,
%chapters/097_staring_pareto,
%chapters/180_uncertainty,
%chapters/190_enrichments,
%chapters/200_computations,
%}
%\end{comment}




% for some reason, the normal author thanks does not work, looking at reason

% \author{Andrea Censi\thanks{\footnotesize{Institute for Dynamic Systems and Control (IDSC), Department of Mechanical and Process Engineering (D-MAVT), ETH Z\"urich, Z\"urich, Switzerland
%         acensi,jlorand,gzardini@ethz.ch}}  ,
% Jonathan Lorand$^1$,
% David I. Spivak\thanks{\footnotesize{Department of Mathematics, Massachusetts Institute of Technology, Cambridge, MA, USA.\href{dspivak@gmail.com}{dspivak@gmail.com}}},\\
% Joshua Tan\thanks{\footnotesize{Department of Computer Science, University of Oxford, Oxford, UK. \href{joshua.tan@cs.ox.ac.uk}{joshua.tan@cs.ox.ac.uk}}}  ,
%         Gioele Zardini$^1$}
% %\thanks{\footnotesize{Institute for Dynamic Systems and Control, Department of Mechanical and Process Engineering, ETH Z\"urich, Z\"urich, Switzerland        acensi,jlorand,gzardini@ethz.ch}}
% %\thanks{\footnotesize{Department of Mathematics, Massachusetts Institute of Technology, Cambridge, MA, USA.\href{dspivak@gmail.com}{dspivak@gmail.com}}}
% %\thanks{\footnotesize{Department of Computer Science, University of Oxford, Oxford, UK. \href{joshua.tan@cs.ox.ac.uk}{joshua.tan@cs.ox.ac.uk}}}}

% % \email{acensi@idsc.mavt.ethz.ch}
% % \homepage{https://censi.science}
% % % \orcid{TODO}
% % % \thanks{You can use the \texttt{\textbackslash{}email}, \texttt{\textbackslash{}homepage}, and \texttt{\textbackslash{}thanks} commands to add additional information for the preceding \texttt{\textbackslash{}author}. If applicable, this can also be used to indicate that a work has previously been published in conference proceedings.}
% % \author{Jonathan Lorand}
% % \email{jlorand@ethz.ch}
% % \homepage{http://lorand.earth/math}
% % \affiliation{Institute of Dynamic Systems and Control (IDSC),
% %     Department of Mechanical and Process Engineering (D-MAVT), ETH Z\"urich, Z\"urich,
% %     Switzerland}
% % \author{David I. Spivak}
% % \email{dspivak@gmail.com}
% % \homepage{http://math.mit.edu/$\sim$dspivak/}
% % \affiliation{Department of Mathematics, Massachusetts Institute of Technology,
% %     Cambridge, MA, USA.}
% % \author{Joshua Tan}
% % \email{joshua.tan@cs.ox.ac.uk}
% % \homepage{http://joshuatan.com/research}
% % \affiliation{Department of Computer Science, University of Oxford, Oxford, UK.}
% % \author{Gioele Zardini}
% % \email{gzardini@ethz.ch}
% % \homepage{http://gioele.science}
% % \maketitle
% \maketitle
% % \begin{abstract}
%     This paper is about engineering design and
%     category theory: an odd couple, consisting of one of the most concrete and
%     one of the most abstract fields of human knowledge.
%     We will show how category theory can help in defining, understanding, and
%     solving formal engineering design problems.
%     This theory is especially well-suited for
%     heterogeneous components with complex
%   recursive co-design constraints.


%     This paper is written for two audiences: we
%     want to explain to engineers the power and the necessity of the
%     category-theoretical language to describe complex design problems; and we
%     want to call the mathematicians' attention to the correspondence between
%     very abstract category theory and extremely concrete engineering design
%     choices that are observable in any artefact.


%     For the reader who already knows category theory: We introduce a category~$\DP$ of
%     \emph{design problems} and study its properties. We show that~$\DP$ is a
%     generalization of the category~$\textbf{Rel}$ of relations.  Diagrams
%     in~$\DP$ describe the co-design constraints among different subsystems of a
%     large system. We show that~$\DP$ is \emph{compact closed}, which makes the
%     diagrams easy to manipulate as they behave similarly to other formalisms
%     such as signal flow diagrams. Finally, we describe the 2-category structure
%     of~$\DP$, which models naturally the dominance between alternative designs.
% % \end{abstract}

\include*{chapters/005_title_page}
\setcounter{tocdepth}{1}
\tableofcontents

% % Introduction
% \chapter{Introduction}
% \include*{chapters/010_introduction}
% \clearpage

\part{Category Theory Basics}

% How things transform
\chapter{\red{Transmutation}}
\include*{chapters/020_things_transforming}
\clearpage

\chapter{\red{Connection}}
\include*{chapters/030_things_connecting}
\clearpage

\chapter{\red{Specialization}}
\include*{chapters/040_specialization}
\clearpage
%
\chapter{\red{Sameness}}
\include*{chapters/041_sameness}
\clearpage
%
%
\book{
\chapter{Motion}
\include*{chapters/043_motion}
\clearpage}
%
%\chapter{Concepts of  Formal Engineering Design}
%\include*{chapters/015_basic_concepts}
%\clearpage
%
\chapter{\red{Thinking about trade-offs}}
\include*{chapters/070_tradeoffs}
\include*{chapters/097_staring_pareto}
\clearpage
%

% Posets with more structire
\chapter{Posets with more structure}
\include*{chapters/075_posets_with_more}
\clearpage
%
%% Monotonicity
\chapter{Monotonicity is a fact of life}
\include*{chapters/080_monotonicity}
\clearpage
%
%% Functors (TBD)
\chapter{Functors}
\include*{chapters/095_functors}
\clearpage

\book{
\chapter{Naturality}
\include*{chapters/096_naturality}
\clearpage

\chapter{Adjunctions}
\include*{chapters/091_adjunctions}
\clearpage

%% Duality of design
\chapter{Duality}
\include*{chapters/090_duality_design}
\clearpage
}

\chapter{\red{Combining}}
\include*{chapters/050_resource_composition}
\clearpage
\include*{chapters/060_alternatives}
\clearpage

\book{
\chapter{Parallelism}
\include*{chapters/115_parallel_theory}
}

\book{
\part{Co-Design}
%
\book{
\chapter{Design without categories}
\include*{papers/arxiv_submission_v6/merged_noclass}
\clearpage
%

\chapter{Profunctors}
\include*{chapters/100_profunctors}
\clearpage
%
%% Composing recognizing structure
\chapter{Composing recognizing structure}
\include*{chapters/110_composing_structure}
\clearpage
%
%% Restrictions and alternatives
\chapter{Thinking about restrictions and alternatives}
\include*{chapters/150_restrictions}
\clearpage
%
%% Parallel composition theory
%\chapter{Parallel composition}
%\include*{chapters/115_parallel_theory}
\clearpage
%
%%% Parallel composition
\chapter{Parallel composition in DP}
\include*{chapters/120_parallel_composition}
\clearpage
%
%%% Defining design problems
\chapter{Creating design problems}
\include*{chapters/130_design_problems}
\clearpage
%
%% Companion and Conjoint
\chapter{Lifting}
\include*{chapters/135_companion_and_conjoint}
\clearpage
%
%% Cartegorical Trace
\chapter{Trace}
\include*{chapters/143_categorical_trace}
\clearpage
%
%%% Closing the loop in co-design problems
\chapter{Closing the loop in co-design problems}
\include*{chapters/146_closing_loop_codesign}
\clearpage
%
%%% Ordering design problems
\chapter{Ordering design problems}
\include*{chapters/160_ordering_dps}
\clearpage
%
%% Orders and composition
\chapter{Order and composition}
\include*{chapters/165_order_and_compositions}
\clearpage
%
%%% Relations, facts
\chapter{Relationship between products}
\include*{chapters/155_relations_facts}
\clearpage
%
%
}

\part{Case Studies}
\part{Higher order stuff}

%% Uncertainty
\book{
\chapter{Thinking about uncertainty with intervals}
\include*{chapters/180_uncertainty}
\clearpage
%
%% Enrichment
\chapter{Enrichments}
\include*{chapters/190_enrichments}
\clearpage

\chapter{Higher order design}
\include*{chapters/170_higher_order_design}
\clearpage
}}


% --- OLD MATERIAL TO PUT BACK UP
%Introduction
%\include*{chapters/20_introduction}
%\clearpage
% Background in CT
%\include*{chapters/30_basic_category_theory}
%\clearpage
% DP Traced monoidal
%\include*{chapters/50_DP_traced_monoidal}
%\clearpage
% Creating DPs
%\include*{chapters/60_creating_DPs}
%\clearpage
\printbibliography