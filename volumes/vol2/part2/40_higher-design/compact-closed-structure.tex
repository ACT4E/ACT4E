% !TEX root = chapter-standalone.tex


\section{Compact closed structure}
\label{sec:higher-compact-closed}

\begin{ctdefinition}[External and internal hom]
  \label{def:int_ext_hom}
  The set $\Hom_\CatC(A,B)$ of morphisms between $A$ and $B$ is known as the \emph{external hom}, and is canonically defined for every category \CatC. For certain categories, however, there is also an \emph{internal hom} $[A,B] \in \CatC$ which satisfies
  \begin{equation}
    \label{eq:internal_hom}
    \Hom_\CatC(A,B) \simeq \{ f : \One \to [A,B] \},
  \end{equation}
  where $\One$ is the monoidal unit in \CatC; we say that $\Hom_\CatC(A,B)$ is the set of \emph{generalized elements} of $[A,B]$.
\end{ctdefinition}

\begin{example}
  The set of functions between $A$ and $B$ is the internal hom between sets $A,B \in \Set$ (it also happens to be equivalent to the external hom in \Set).
\end{example}

In \DP, the external hom and the internal hom are not equivalent, but the isomorphism $\Hom_\DP(\F{A},\R{B}) \simeq \{ f \colon \One \profto [\F{A},\R{B}] \}$ still holds, allowing us to study the properties of $\Hom_\DP(\F{A},\R{B})$ from ``inside'' \DP.

To say that a category \CatC is \emph{closed} is to say, roughly, that the internal hom exists for all pairs of objects in \CatC. In \DP, this means that there exists a unique poset $[\F{A},\R{B}]$ associated to every pair $\F{A},\R{B}$ and satisfying \cref{eq:internal_hom}.

To say that a category \CatC is \emph{compact closed} is, to say that every object $A \in \CatC$ has a dual object $A^* \in \CatC$, and that, for any $B \in \CatC$, $A$ and $A^*$ satisfy a unique formula for the internal hom: $[A,B] = A^* \times B$. In \DP, the dual of a poset $\F{A} \in \DP$ is $\F{A}\op$, and the formula for the internal hom is $[A,B]=\F{A}\op \times \R{B}$.

\begin{ctdefinition}[\iindex{Compact closed category}]
  \label{def:compact_closed_cat}
  Let~$\tup{\CatC,\otimes,I,\sigma}$ be a symmetric monoidal category. It is called \emph{compact closed} if, for all~$C \in\CatC$ there exists some object~$C^*\in\CatC$ (called the \emph{dual of $C$}), a morphism~$\eta_C\colon I\to C^*\otimes C$ (called the \emph{unit for $C$}), and a morphism~$\epsilon_C\colon C\otimes C^*\to I$ (called the \emph{counit of~$C$}) such that the following diagrams commute for all~$C\in \CatC$:
  \begin{equation*}
    \label{eq:compact_closed_cat_1}
    \includesag{60_compact_1}
  \end{equation*}
  ~
  \begin{equation*}
    \label{eq:compact_closed_cat_2}
    \includesag{60_compact_2}
  \end{equation*}
\end{ctdefinition}

\begin{lemma}[\DP is compact closed]
  \label{lem:DP-compact-closed}
  The symmetric monoidal category~$\tup{\DP,\otimes,\singleton,\sigma}$ is compact closed.
\end{lemma}
\begin{proof}
  Note that we have already shown that~$\tup{\DP,\otimes,\singleton,\sigma}$ is a symmetric monoidal category (\cref{lem:symmetricmonoidaldp}). Define the unit~$\eta_P\colon \F{\singleton} \profto \R{P}\op \times \R{P} $ as
  \begin{equation}
    \label{eq:unit_morphism}
    \begin{aligned}
      \eta_P \colon \F{\singleton} \times (\R{P}\op \times \R{P}) & \toinPos \Bool, \\
      \tup{\F{1},\tup{\R{p}^*,\R{q}}} & \mapsto \R{p} \ordleq_P \R{q},
    \end{aligned}
  \end{equation}
  and define the counit $\epsilon_P\colon \F{P\op} \times \F{P} \profto \R{\{1\}} $ as
  \begin{equation}
    \begin{aligned}
      \epsilon_P \colon (\F{P}\op \times \F{P})\op \times \R{\singleton} & \toinPos \Bool,  \\
      \tup{\tup{\F{p},\F{q}^*}^*, \R{1}} & \mapsto \F{p} \ordleq_P \F{q}.
    \end{aligned}
  \end{equation}

  We now check that the first diagram (\cref{eq:ccc}) holds. To show that the second holds, is similar. We will show that the composite~$\rho^{-1}\then \eta\then \alpha \then \epsilon\then \lambda$, call it~$f\colon \F{\cP} \profto \R{\cP} $, is equal to~$\Unit{\cP }$. First, let's consider each morphism in this composite morphism $f$. For $\rho^{-1}$ one has:
  \begin{equation}
    \rho^{-1}\left(\F{p_1}^*,\tup{\R{p_2},\R{*}}\right)=\F{p_1}\ordleq \R{p_2}.
  \end{equation}
  For $\eta_P$ one has
  \begin{equation}
    \eta\left(\tup{\F{p_2},\F{*}}^*,\tup{\R{p_3},\tup{\R{p_4}^*,\R{p_5}}}\right)=(\F{p_2}\ordleq \R{p_3})\wedge(\R{p_4}\ordleq \R{p_5}).
  \end{equation}
  For $\alpha$ one has
  \begin{equation}
    \alpha\left( \tup{\F{p_3},\tup{\F{p_4}^*,\F{p_5}}}^*,\tup{\tup{\R{p_6},\R{p_7}^*},\R{p_8}}\right)=(\F{p_3}\ordleq \R{p_6})\wedge(\R{p_7}\ordleq \F{p_4})\wedge(\F{p_5}\ordleq \R{p_8}).
  \end{equation}
  For~$\epsilon_P$ one has
  \begin{equation}
    \epsilon\left(\tup{\tup{\F{p_6},\F{p_7}^*},\F{p_8}}^*,\tup{\R{*},\R{p_9}}\right)=(\F{p_6}\ordleq \F{p_7})\wedge (\F{p_8}\ordleq \R{p_9}).
  \end{equation}
  Finally, for~$\lambda$ one has
  \begin{equation}
    \lambda\left(\tup{\F{*},\F{p_9}}^*,\R{p_{10}}\right)=\F{p_9}\ordleq \R{p_{10}}.
  \end{equation}
  The composition formula then says that~$f$ is given by:
  \begin{equation}
    \begin{aligned}
      f(\F{p_1}^*,\R{p_{10}})&=\bigvee_{p_2,\ldots,p_9} (\F{p_1}\ordleq p_2\leq p_3\ordleq p_6\ordleq p_7\ordleq p_4\ordleq p_5\ordleq p_8\ordleq p_9\ordleq \R{p_{10}})\\
      &=\F{p_1}\ordleq \R{p_{10}}\\
      &=\id_P(\F{p_1}^*,\R{p_{10}}).
    \end{aligned}
  \end{equation}
\end{proof}

\begin{example}
  \label{ex:rdproblem}
  In the simplest case, fix some $\text{engine} \in \text{Engines} \definedas \Hom_\DP(\F{\text{thrust}},\R{\text{fuel}})$. Since \DP is compact closed, we can rewrite the R\&D design problem $\text{R\&D}(\F{\text{engine}}^*, \tup{\R{t},\R{m}})$ from \cref{ex:rd} as a design problem of the form~$\One \profto \R{\text{time}} \times\R{\text{money}}$ (\cref{fig:excompactclosed}).
  \begin{figure}[h!]
    \begin{center}
      \includesag{60_research}
    \end{center}
    \caption{Example for compact closure. \label{fig:excompactclosed}}
  \end{figure}
  where $\eta \colon \F{I} \profto \R{\text{thrust}}\op \times \R{\text{thrust}}$ is the \emph{unit} design problem described in \cref{eq:unit_morphism} and the dependent design problem `R\&D $\mid$ engine' is defined by
  \begin{equation}
    \begin{aligned}
    (\text{R\&D } \mid \text{ engine})
      \colon (\F{\mathsf{thrust}}\op \times \F{\mathsf{fuel}})\op \times (\R{\mathsf{time}} \times \R{\mathsf{money}}) &\toinPos \Bool \\
      \tup{\tup{\F{h}^*, \F{f}}^*, \tup{\R{t},\R{m}}} &\mapsto \text{R\&D}(\mathsf{engine}, \tup{\R{t},\R{m}}).
    \end{aligned}
  \end{equation}
  As expected, `R\&D $|$ engine' simply throws away the thrust and fuel information. The composition $(\eta \then \text{engine} \then \text{R\&D } \mid \text{ engine}) (t,m)$ is defined by
  \begin{equation}
    \text{R\&D}(\text{engine}, \tup{t,m}) \wedge \bigvee_{h^*,f} \text{engine}(h^*, f).
  \end{equation}
  In other words, the composition represents a simple threshold of time and money (actually, an antichain in the poset Time $\times$ Money), above which it is feasible to construct `engine', and below which it is not.

  This design problem can then simply be tensored with the original engine design problem to create a compiled design problem of the form $(\text{engine} \otimes \text{R\&D}) \colon \F{\text{Thrust}} \profto \R{\text{Fuel}} \times \R{\text{Time}} \times \R{\text{Money}}$. Since we are simply tensoring, the original engine design problem can still be plugged into a larger design problem in all the usual ways.
\end{example}

\subsubsection{Dependent design problems}
\cref{ex:rdproblem} was relatively simplistic, since we fixed a specific engine $\in \Hom_\DP(\F{\text{thrust}}, \R{\text{fuel}})$. Now, suppose we want to make the engine design problem---``available technologies''---in a rocket design problem \emph{dependent} on the amount of time and money committed in the R\&D design problem, given that time and money could be spent on other things, like building the actual rocket. In other words, we want a design problem involving R\&D, which still retains Fuel and Thrust as ``open'' inputs and outputs. This is a typical kind of design problem one considers when funding science and R\&D projects---a very relevant subject for the authors of this article!

Recall that, for a fixed choice of $f \in \Hom_\DP(\F{A},\R{B})$, any `design problem of design problems' $k \colon \Hom_\DP(\F{A},\F{B}) \profto \R{C}$ can be reframed as a composition of three maps: the unit $\eta_A \colon \F{\One} \profto \R{A}\op \times \R{A}$, $f \colon \F{A} \profto \R{B}$, and the dependent design problem $(k \mid f) \colon \F{A}\op \times \F{B} \profto \R{C}$.

\todo{I'm stuck on this; not quite sure how to define these kinds of problems. We may not be able to do this in \DP as defined... maybe it would be easier in \DPI?}

Before going to the next example, we introduce a special design problem for adding the output of two wires.
% Should we also define multiplication? E.g., perhaps we want the weight of the motor to be some much smaller proportion of the total weight it can carry.

% Recall that, for a fixed choice of engine $\in$ Engines, Jeb's original R\&D design problem can be reframed as a composition of three maps: $\eta : \One \tickar \text{Thrust}\op \times \text{Thrust}$, $\text{engine} : \text{Thrust} \tickar \text{Fuel}$, and a map $\text{R\&D} : \text{Thrust}\op \times \text{Fuel} \tickar \text{Time} \times \text{Money}$.

\begin{example}
  Jeb's R\&D team is made up of three rocket scientists: Howie, who has watched every Fast\&Furious movie, Shirley, an unpaid intern, and Mabie, who was recently prosecuted for insurance fraud. Howie says that given 3 months and \$100,000, he can make the engine much faster at higher fuel inputs. Shirley says that given 1 month and \$10,000, she can make the engine slightly more fuel-efficient at all currently-feasible thrust performances. Mabie promises that with just 1 week and \$10 million, she can make the engine perform spectacularly at a very low fuel input.
\end{example}

One can go to even higher design problems. In the example above, Jeb might want to know whether it's worth spending all that time and money to improve the Jeb-XX so that it is competitive with the Bob-Roc, given the value of the contract from NASA. But this requires us to think about R\&D processes themselves; \ie  it should cost less time and money to achieve Bob-Roc-grade performance, given the Jeb-XX as a starting point, than it does it to achieve Bob-Roc-grade performance, given the Wright engine as a starting point. That is, we would like to treat the set of engine technologies
\[\text{Engines} = \Hom_\DP(\mathsf{thrust}, \mathsf{fuel})\]
as a resource input to a design problem which outputs R\&D processes,
\[\text{R\&D Companies} = \Hom_\DP(\text{Engines}, \text{Time} \times \text{Money}).\]

Notably, the design problem
\begin{center}
  \includesag{60_higher1}
\end{center}
is an element of $\text{R\&D Companies}$ and thus we can consider, simultaneously, ...
\begin{center}
  \includesag{60_higher2}
\end{center}
Each R\&D process itself outputs an engine technology, so one can imagine tracing the diagram above.
