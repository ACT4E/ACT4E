
Consider the monoid of sequences of sets:
%
\begin{equation}
    \label{eq:sets-monoid}
    \Tupcatt \setA  \setB  \setC \setin \listsof\Set.
\end{equation}
%
We consider each element of~$\listsof\Set$ as the name for a set.

The elements of this set are \emph{tuples} of elements.

If~$\ela$ is in~$\setA$ and~$\elb$ is in~$\setB$, the tuple~$\tupcat \ela \elb$ is in~$\Tupcat \setA \setB$:
%
\begin{equation}
    \label{eq:sets-monoid-axiom1}
    \prfperiod{
        \ela \setin \setA
    }{
        \elb \setin \setB
    }{
        \tupcat \ela \elb \setin \Tupcat \setA \setB
    }
\end{equation}
%
Conversely, if~$\ela \setin \setA$ and~$\tupcat \ela \elb \setin \Tupcat \setA \setB$, we can conclude that~$\elb \setin \setB$:
%
\begin{equation}
    \label{eq:sets-monoid-axiom2}
    \prfcomma{
        \ela \setin \setA
    }{
        \tupcat \ela \elb \setin \Tupcat \setA  \setB
    }{
        \phantom{\elb \setin \setB}
    }{
        \rule{39mm}{0pt} \elb \setin  \setB
    }
\end{equation}
%
and, symmetrically,
%
\begin{equation}
    \label{eq:sets-monoid-axiom3}
    \prfcomma{
        \phantom{\ela \setin \setA}
    }{
        \tupcat \ela \elb \setin \Tupcat \setA \setB
    }{
        \elb \setin \setB
    }{
        \ela \setin  \setA \rule{39mm}{0pt}
    }
\end{equation}
%
or, more compactly,
%
\begin{equation}
    \label{eq:sets-monoid-axiom3bis}
    \prfperiod{
        \tupcat \ela \elb \setin \Tupcat \setA \setB
    }{
        \prfdouble{
            \ela \setin \setA
        }{
            \elb \setin \setB
        }
    }
\end{equation}
%
% Because monoid composition~$\setconcat$ is associative, we have
% %
% \begin{equation}
%     \label{eq:sets-monoid-comp}
%     \Tupcat {\Tupcat \setA  \setB}  \setC  =
%     \Tupcatt \setA    \setB \setC  =
%     \Tupcat   \setA  {\Tupcat \setB  \setC}.
% \end{equation}
% %
% Therefore, this works as a cross product that is associative.
}
%We can also define morphisms as lists of functions.
%We construct them as follows:
%
%\begin{equation}
%    \label{eq:sets-monoid-maps2}
%    \prfperiod{
%        \mora \colon \setA \to \setB
%    }{
%        \morb \colon \setC \to \setD
%    }{
%        \tupmorcat \mora \morb \colon \Tupcat \setA   \setC \to \Tupcat \setB  \setD
%    }
%\end{equation}
%
%\begin{definition}[The category \SetL]
%    \label{def:SetL}
%    The category \SetL is defined as follows:
%    \begin{enumerate}
%        \item \emph{Objects:} lists of sets.
%        \item \emph{Morphisms:}
%              A morphism from~$\setA$ to~$\setB$ is a list of~$n$ functions~$\mora_i\colon \setA_i \mto \setB_i$ such that~$\setA = \Tupcatt {\setA_1}  \dots {\setA_n}$ and~$\setB = \Tupcatt {\setB_1} \dots {\setB_m}$.
%        \item \emph{Composition of morphisms:}
%              Composition is given by function composition.
%        \item \emph{Identities:}
%              The identity on an object~$\Tupcatt {\setA_1} \dots {\setA_n}$ is given by~$\tupmorcatt {\catid_{\setA_1}} \dots  {\catid_{\setA_n}} $.
%    \end{enumerate}
%\end{definition}

% \begin{equation}
%     \label{eq:sets-monoid-maps}
%     \prftree{ \mora \colon \setA \to \setC}{ \morb \colon \setB \to \setC}{ \tupcat \mora \morb \colon (\setA\setconcat \setB)\to \setC }.
% \end{equation}
%