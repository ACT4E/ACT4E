% !TEX root = chapter-standalone.tex

\section{Relational Databases}
\label{sec:relational-databases}

A \emph{relational database} like PostgreSQL, MySQL, \etc presents the data to the user as relations.
This does not necessarily mean that the data is stored as tuples, as in the mathematical model, but rather that what the user can do is query and manipulate relations.
This conceptual model is now 50 years old.

Can we use the category \Rel to represent databases~\cite{codd2002relational}?

Suppose we want to buy an electric stepper motor for a robot that we are building, and for this we consult a catalogue of electric stepper motors\footnote{See \href{https://www.pololu.com/category/87/stepper-motors}{pololu.com} for a standard catalogue of electric stepper motors.
}.

The catalogue might be organized as a large table, where on the left-hand side there is a column listing all available motors (identified with a model ID), and the remaining columns correspond to different attributes that each of the models of motor might have, such as the name of the company that manufactures the motor, the size dimensions, the weight, the maximum power, the price, \etc
A simple illustration is provided in \cref{tab:electric_motors_old_one}.

\begin{table*}[h]
    \centering
    \begin{tabular}{c|c|c|c|c|c}
        Motor ID      & Company              & $\unit[\text{Size}]{[mm^3]}$ & \unit[Weight]{[g]} & \unit[Max Power]{[W]}
                      & \unit[Cost]{[USD]} \\
        \hline
        \textsf{1204} & \textsf{SOYO}        & 20 x 20 x 30                 & 60.0               & 2.34                  & 19.95 \\
        \textsf{1206} & \textsf{SOYO}        & 28 x 28 x 45                 & 140.0              & 3.00                  & 19.95 \\
        \textsf{1207} & \textsf{SOYO}        & 35 x 35 x 26                 & 130.0              & 2.07                  & 12.95 \\
        \textsf{2267} & \textsf{SOYO}        & 42 x 42 x 38                 & 285.0              & 4.76                  & 16.95 \\
        \textsf{2279} & \textsf{Sanyo Denki} & 42 x 42 x 31.5               & 165.0              & 5.40                  & 164.95 \\
        \textsf{1478} & \textsf{SOYO}        & 56.4 x 56.4 x 76             & 1,000              & 8.96                  & 49.95 \\
        \textsf{2299} & \textsf{Sanyo Denki} & 50 x 50 x 16                 & 150.0              & 5.90                  & 59.95
    \end{tabular}

    \caption{
        \label{tab:electric_motors_old_one}
        A simplified catalogue of motors.
    }
\end{table*}

% \subsubsection{Can we use \Rel for relational databases?}

Such database table can be seen as representing an $n$-ary relation with $n = 7$, as we are expressing a relation over the sets
\begin{equation}
    \stylesets{M}\cartprod \stylesets{C}\cartprod \stylesets{S}\cartprod \stylesets{W}\cartprod \stylesets{J}\cartprod \stylesets{P},
\end{equation}

where~$\stylesets{M}$ represents the set of motor IDs,~$\stylesets{C}$ the set of companies producing motors,~$\stylesets{S}$ the set of motor sizes,~$\stylesets{W}$ the set of motor weights,~$\stylesets{J}$ the set of possible maximal powers, and~$\stylesets{P}$ the set of possible prices.
An $n$-ary relation is a relation over $n$ sets, just like a binary relation is a relation over $2$ sets.

\begin{definition}[$n$-ary relation]
    \label{def:n-ary-relation}
    An $n$-ary relation on $n$ sets $\tup{\setA_1, \setA_2, \dots, \setA_n}$ is a subset of the product set
    \begin{equation}
        \setA_1 \cartprod \setA_2 \cartprod \cdots \cartprod \setA_n.
    \end{equation}
\end{definition}

\Rel only allows binary relations.
Morphisms in \Rel have 1 source and 1 target.
There is no immediate and natural way to represent $n$-ary relations using \Rel,
or we don't know it yet.

% To represent relational databases categorically, there are at least 3 options.
%
% \paragraph{Option 1: Hack it}
% We will introduce the notion of \emph{products} and \emph{isomorphisms}.
% This will allow us to say that because
% \begin{equation}
%     X_1 \cartprod X_2 \cartprod X_3 \cartprod \cdots \cartprod X_n,
% \end{equation}
% is isomorphic to
% \begin{equation}
%     X_1 \cartprod ( X_2 \cartprod X_3 \cdots \cartprod X_n)
% \end{equation}
% we can talk about $n$-ary relations in terms of binary relations.
% This is not really a natural way to do it.
%
% \paragraph{Option 2: Mutant Morphisms}
% What if morphisms could have more than ``two legs''?
% There are indeed theories that work with more complicated arrows.
% For example: \href{https://ncatlab.org/nlab/show/multicategory}{multicategories}, \href{https://ncatlab.org/nlab/show/polycategory}{polycategories}, \href{https://ncatlab.org/nlab/show/operad}{operads}.
%
% \begin{figure}[h]
%     \centering
%     \includegraphics[width=7cm]{mutants}
%     \caption{Can you imagine how to define composition with mutant morphisms with more than two legs?}
% \end{figure}
%
% \paragraph{Option 3: Categorical databases}
%
% A different perspective is that of \emph{categorical databases}~\cite{spivak2019categorical}.
% In this modeling framework one does not model the data tables as relations directly.
% Rather the data is described as a functor from a category representing the schema to \Set.}
