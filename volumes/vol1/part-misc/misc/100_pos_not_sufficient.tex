\devel{
    \subsection{Why~\Pos is not sufficient for design theory}

    % \todojira{85}{@Andrea: Rewrite this without assuming we know what is a design problem.}
    The category \Pos of \SY{posets} and \SY{monotone maps} that we have described can model many facts that are useful for design theory.
    However, there are also limitations which motivate us to describe a more general category.
    This section describes the usefulness and the limitations of~\Pos.

    \begin{example}[Battery]
        Consider a model of a battery where the capacity is the functionality and the mass of the battery is the resource.
        %(\cref{fig:battery-example}).
        There is certainly a \SY{monotone map} from capacity to mass.
        This map answers the question: ``Given a value of the capacity, what is the minimum mass needed?''.
        Conversely, in the other direction, the map that answers the question: ``Given a certain mass, what is the maximum capacity that can be provided?'' is also a \SY{monotone map}.
    \end{example}

    Therefore, at first sight it might seem that \SY{posets} and \SY{monotone maps} would be sufficient to describe a quantitative theory of design.
    However, there are more general relations to be modeled.
    It is easy enough to describe examples in which having a simple \SY{monotone map} from functionality to resources is not sufficient.

    \begin{example}[Delivery drone]
        Consider the design of a delivery drone, in which the functional requirement is that the drone should be able to make a delivery at a distance~$d$, and we need to reason about how powerful to make the drone.
        In particular, we need to choose at what (average) \emph{velocity}~$v$ the drone should travel and what is the optimal \emph{mission duration}.
        The relation between distance~$d$, velocity~$v$, and mission duration~$T$ is given by~$d=v\cdot T$.
        We can choose to have either a fast drone and short missions, or a slow drone and long missions.
        This is an interesting trade-off.
        Flying fast takes more energy, both for propulsion and for computation (more objects to be observed and processed).
        Flying too slow will also be excessively energy-consuming because of the long mission duration.

        If we consider~$v$ and~$T$ as given, then the map~$\tup{v,T} \mapsto v\cdot T$ is clearly a \SY{monotone function} that gives the distance which the drone is able to cover.
        However, in the other direction, we do not have a simple map, but rather a 1-to-many relation $\mathrm{distance}\to \mathrm{velocity}\times \mathrm{time}$.
        For each fixed value of the distance, there is an entire continuum of values of~$v$ and~$T$ which we can choose, as it can be seen in~\cref{fig:drone-example-antichain}.

        \begin{marginfigure}
            \centering
            \includesag{drone_example_antichain}
            \caption{Antichains in~$\tup{v,T}$ for different values of~$d$.}
            \label{fig:drone-example-antichain}
        \end{marginfigure}

    \end{example}

    In other words, using~\Pos it is not possible to make a theory of \emph{trade-offs}.
    We will introduce a more general category, called the category~\DP of \emph{design problems} (\cref{sec:dpdefinition}), which allows to describe such a theory.

}
