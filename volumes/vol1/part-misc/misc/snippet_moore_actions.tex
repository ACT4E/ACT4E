
Let's now look at how machines like the above act on sequences.

For now we only have defined semi-group, monoid, and group actions, and have not talked yet about (semi)category actions.
Let's consider the set of machines systems with~$\prin = \prout = \prgen$; this is the homset~$\HomSet\Moore\prgen\prgen$.

\

\begin{equation}
    \begin{cases}
        x_{k+1} = \prdyn(u_k, x_k) \\
        y_{k}   = \prreadout(x_{k}).
    \end{cases}
\end{equation}

\

Given a finite input sequence~$u\colon \natnumbers \to \prgen$ of length~$n$, the output is an instantaneous transformation of the state:
%
\begin{equation}
    \label{eq:actions-on-sequences-y}
    \begin{aligned}
        y_0   & = \prreadout(\tupset(x_0)), \\
        y_1   & = \prreadout(\tupset(x_1)), \\
        y_2   & = \prreadout(\tupset(x_2)), \\
        \dots & = \dots \\
        y_{k} & = \prreadout(\tupset(x_{k-1})).
    \end{aligned}
\end{equation}
%
The state is computed recursively as follows:
\begin{equation}
    \label{eq:actions-on-sequences-x}
    \begin{aligned}
        x_0   & = \prdyn(u_0, \prstart), \\
        x_1   & = \prdyn(u_1, x_0), \\
        x_2   & = \prdyn(u_2, x_1), \\
        \dots & = \dots \\
        x_{k} & = \prdyn(u_{k}, x_{k-1}).
    \end{aligned}
\end{equation}
%
Therefore, given a machine~$\mora \colon \prgen \mtoin{\Moore} \prgen$ we have defined a map from $\natnumbers \to \prgen$ to itself.
Let's call it $\act$.
It is defined as a map of the form
%
\begin{equation}
    \label{eq:actions-on-sequences-1}
    \act_{\mora}\colon  (\natnumbers \to \prgen)  \sto  (\natnumbers \to \prgen),
\end{equation}
%
or, more formally,
%
\begin{equation}
    \label{eq:actions-on-sequences-2}
    \act: \HomSet\Moore\prgen\prgen \sto \Endof {\natnumbers \to \prgen}.
\end{equation}
%
Note that both~$\HomSet\Moore\prgen\prgen$ and $\Endof {\natnumbers \to \prgen}$ are semigroups.
Could it be that~$\act$ is a semigroup morphism?
And, consequently, is~$\act$ a covariant semigroup action or a contravariant semigroup action?

Let's check the condition for it being a morphism (\cref{eq:sgrp-mor-comp}):
%
\begin{align}
    \act ( \mora \mthenof{\Moore} \morb) & \mathrel{\stackrel{?
    }{=} } \act ( \mora) \mthenof{\Endof  {\natnumbers \to \prgen}} \act(\morb) \label{eq:actions-on-sequences-left} \\
    \act ( \morb \mthenof{\Moore} \mora) & \mathrel{\stackrel{?}{=}}  \act ( \mora) \mthenof{\Endof  {\natnumbers \to \prgen}} \act(\morb) \label{eq:actions-on-sequences-right}
\end{align}

We can check this graphically.
First, one has
%
\begin{equation*}
    \prfperiod{\prftree{\includesag{moore_right_s_f}}{\includesag{moore_right_s_f_conn}}}{\prftree{\includesag{moore_right_b}}{\includesag{moore_right_b_space}}}{\includesag{moore_right_comp_1}}
\end{equation*}
%
Second, one has
%
\begin{equation*}
    \prfperiod{\prftree{\includesag{moore_right_s}}{\includesag{moore_right_s_space}}}{\prftree{\includesag{moore_right_1}}{\includesag{moore_right_2}}}{\includesag{moore_right_5}}
\end{equation*}
%
Therefore, we have a \emph{covariant} semigroup action.

\begin{publictodo}
    The rest of the section is missing.
\end{publictodo}
