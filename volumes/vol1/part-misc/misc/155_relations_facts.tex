% !TEX root = chapter-standalone.tex

\section{Biproduct, Product, and Coproduct of Design Problems}
\begin{example}
    Just as Beau is about to connect the engine diagram to the larger \SY{design problem} for the X103, his supervisor, Elly May, comes up behind him and catches a glance at his diagram.
    ``Beau, that's almost right: Jeb-XX~$\wedge$ Bob-Roc does indeed work as an approximation of a generic ``engine'' design problem.
    The problem is that the choice of engine is so important and expensive that it isn't up to the engineering team---it's up to the politicians!
    So take~$f \vee g$ out and stick the one reported in~\cref{fig:exbiproduct} instead.
    \begin{figure}[h!]
        \begin{center}
            \includesag{50_rival_pol}
        \end{center}
        \caption{The choice of engine is up to the politicians: Biproduct.}
        \label{fig:exbiproduct}
    \end{figure}

    We'll come back and adjust the parameter to either the Jeb XX or the Bob-Roc after the politicians make their choice.
    ''
\end{example}
\begin{definition}[Biproduct of design problems]
    \label{def:biproduct-dp}
    \SYNDEF{biproduct of design problems}
    Given \SY{design problems}~$f\colon \F{A} \profto \R{B}$ and~$g\colon \F{C} \profto \R{D}$, their \emph{biproduct}~$(f + g)\colon \F{A} + \F{C} \profto \R{B} + \R{D}$ is defined by
    \begin{equation}
        \begin{aligned}
            (f + g)
            \colon (\F{A} + \F{C})\op  \times (\R{B} + \R{D}) & \toinPos \Bool, \\
            \tup{\disunionA{\F{a}}^*, \disunionA{\R{b}}}      & \mapsto f(\F{a}^*, \R{b}), \\
            \tup{\disunionB{\F{c}}^*, \disunionA{\R{b}}}      & \mapsto \false, \\
            \tup{\disunionA{\F{a}}^*, \disunionB{\R{d}}}      & \mapsto \false, \\
            \tup{\disunionB{\F{c}}^*, \disunionB{\R{d}}}      & \mapsto g(\F{c}^*, \R{d}),
        \end{aligned}
    \end{equation}
    and represented as in~\cref{fig:biproductdp}.

    In particular, when~$f, g\colon \F{A} \profto \R{B}$,~$(f + g)\colon \F{A} + \F{A} \profto \R{B} + \R{B}$ is defined by
    \begin{equation}
        \begin{aligned}
            (f + g)
            \colon (\F{A} + \F{A})\op  \times (\R{B} + \R{B}) & \toinPos \Bool, \\
            \tup{\disunionA{\F{a}}^*, \disunionA{\R{b}}}      & \mapsto f(\F{a}^*, \R{b}), \\
            \tup{\disunionB{ \F{a}}^*, \disunionA{\R{b}}}     & \mapsto \false, \\
            \tup{\disunionA{\F{a}}^*, \disunionB{ \R{b}}}     & \mapsto \false, \\
            \tup{\disunionB{ \F{a}}^*, \disunionB{ \R{b}}}    & \mapsto g(\F{a}^*, \R{b}).
        \end{aligned}
    \end{equation}
\end{definition}

\begin{figure}[h!]
    \begin{center}
        \includesag{52_biproduct}
    \end{center}
    \caption{Diagrammatic representation of the \SY{biproduct} of \SY{design problems}. }
    \label{fig:biproductdp}
\end{figure}

Assume $f,g \colon \F{A} \profto \R{B}$.
Intuitively,~$f+g$ can be thought of as: pick either~$f$ or~$g$, then throw away the other one, whereas on any~$\tup{\F{a}^*,\R{b}}$,~$f \vee g$ always picks the better (more feasible) of either~$f(\F{a}^*,\R{b})$ or~$g(\F{a}^*,\R{b})$.
Note that~$f+g$ introduces an extra parameter, since the choice of~$f$ or~$g$ has to be hard-coded into the larger design problem.

In general, the \SY{biproduct} is defined first on objects of the category, so what we are calling a \SY{biproduct} of \SY{design problems} is actually the unique map derived from the biproduct~$A + B$ of \SY{posets} in~\DP, namely the disjoint union.
In the more general setting~$f\colon \F{A} \profto \R{B}$ and~$g \colon \F{C} \profto \R{D}$, the \SY{biproduct} on objects~$\F{A} + \F{C}$ models~$\F{A}$ and $\F{C}$ as \emph{interchangeable entities}:~$f+g$ may output one functionality from either~$\F{A}$ or $\F{C}$, and similarly it requires only one resource from either~$\R{B}$ or~$\R{D}$.
We can emphasize one or the other condition by letting~$\F{A} = \F{C}$ or~$\R{B} =\R{D}$ above, thus deriving the product of \SY{design problems} and coproduct of \SY{design problems}, respectively.

\begin{definition}[Coproduct of design problems]
    \label{def:coproduct}
    Given two \SY{design problems}~$f\colon \F{A} \profto \R{C}$ and~$g \colon \F{B} \profto \R{C}$, their \emph{coproduct}~$(f \sqcup g)\colon \F{A} + \F{B} \profto \R{C}$ is defined by
    \begin{equation}
        \begin{aligned}
            (f \sqcup g)
            \colon (\F{A} + \F{B})\op \times \R{C} & \toinPos \Bool, \\
            \tup{\F{a}^*, \R{c}}                   & \mapsto f(\F{a}^*, \R{c}), \\
            \tup{\F{b}^*, \R{c}}                   & \mapsto g(\F{b}^*, \R{c}),
        \end{aligned}
    \end{equation}
    and represented as in~\cref{fig:coproductdp}.
\end{definition}

\begin{figure}[h!]
    \begin{center}
        \includesag{52_coproduct}
    \end{center}
    \caption{Diagrammatic representation of the coproduct of \SY{design problems}. }
    \label{fig:coproductdp}
\end{figure}

\begin{definition}[Product of design problems]
    \label{def:product-dp}
    Given \SY{design problems}~$f\colon \F{C} \profto \R{A}$,~$g \colon \F{C} \profto \R{B}$, their \emph{product}~$(f \Ctimes g)\colon \F{C} \profto \R{A} + \R{B}$ is defined by
    \begin{equation}
        \begin{aligned}
            (f \Ctimes g)
            \colon \F{C}\op  \times (\R{A} + \R{B}) & \toinPos \Bool, \\
            \tup{\F{c}^*, \R{a}}                    & \mapsto f(\F{c}^*, \R{a}), \\
            \tup{\F{c}^*, \R{b}}                    & \mapsto g(\F{c}^*, \R{b}),
        \end{aligned}
    \end{equation}
    and represented as in~\cref{fig:productdp}.
\end{definition}

\begin{figure}[h!]
    \centering
    \includesag{52_product}
    \caption{Diagrammatic representation of the product of \SY{design problems}. }
    \label{fig:productdp}
\end{figure}

To show that~$A + B$ is in fact a \SY{biproduct} in the categorical sense, we need to show that it satisfies certain properties.
These properties guarantee that the \SY{biproduct} is the most ``efficient'' way of combining two objects in~$~\DP$ in a certain sense, and that the resulting combination is unique when it exists.
Specifically, we need to show that~$A + B$ is both a product~$\tup{+, \pi_A, \pi_B}$ and a coproduct~$\tup{+, \iota_A, \iota_B}$ in~\DP, and satisfies an extra coherence condition:~$\pi_B \after \iota_B = \relid_B$.

\begin{ctdefinition}[Initial and terminal object]
    \label{def:initial-terminal-object}
    Let~\CatC be a category and let~$A \setin \CatC$ be an object.
    We say that~$A$ is an \emph{initial object} if, for all~$B \setin\CatC$, the hom-set~$\Hom_\CatC(A,B)$ has exactly one element.
    We say that~$A$ is a \emph{terminal object} if, for all~$D\setin\CatC$, the hom-set~$\Hom_\CatC(D,A)$ has exactly one element.
\end{ctdefinition}

\begin{ctdefinition}[Finite coproducts and products]
    \label{def:finite-products}
    We say that~\CatC \emph{has finite coproducts} if it has an \SY{initial object} and every pair of objects in~\CatC has a coproduct.
    We say that~\CatC \emph{has finite products} if it has a \SY{terminal object} and every pair of objects in~\CatC has a product.
\end{ctdefinition}

\begin{example}
    The category~\Pos has finite coproducts and finite products.
    The coproduct~$\sqcup_\Pos$ is the disjoint union~$\Cplus$.
    The product~$\times_Pos$ is the \SY{cartesian product}~$\cartprod$.
\end{example}

\begin{lemma}
    The category~\DP has finite coproducts and finite products.
    The coproduct~$\sqcup_\DP$ is the disjoint union~$\setdisunion$ (\cref{sec:disjoint-union}).
    The product~$\times_\DP$ is also the disjoint union~$\setdisunion$.
\end{lemma}

\begin{proof}
    Let~$\inj_\setA$ be the injection of \SY{posets}~$\setA \sto \setA\setdisunion\setB$.
    We define the \SY{design problems} \begin{equation}
        \begin{aligned}
            \comp{\iota_A} \colon \F{A} & \profto \R{A} + \R{B} \\
            \tup{\F{a}^*,\R{c}}         & \mapsto \iota_A(\F{a})\ordleq_{A+B}\R{c},
        \end{aligned}
    \end{equation}
    and
    \begin{equation}
        \begin{aligned}
            \comp{\iota_B} \colon \F{B} & \profto \R{A} + \R{B} \\
            \tup{\F{b}^*,\R{c}}         & \mapsto \iota_B(\F{b})\ordleq_{A+B}\R{c}.
        \end{aligned}
    \end{equation}

    Recall the coproduct of \SY{design problems} in \cref{def:coproduct}.
    To show that~$A+B$ is a coproduct (of posets) in~\DP, we need to show that the coproduct~$f \sqcup g$ of \SY{design problems} is unique and that it satisfies~$f = \comp{\iota_A} \mthen (f \sqcup g)$ and~$g = \comp{\iota_B} \mthen (f \sqcup g)$.
    This can be verified by simply writing out the composition.
    In the following we denote~$\ordleq_{A+B}$ by $\ordleq$.
    We have
    \begin{equation}
        \begin{aligned}
             & \pars{\comp{\iota_A}\mthen (f\sqcup g)}(\F{a}^*,\R{c}) \\&=\bigvee_{x' \setin A+B}\comp{\iota_A}(\F{a}^*,\R{x'})\wedge (f\sqcup g)(\F{x'}^*,\R{c})\\
             & =\bigvee_{\disunionA{x} \setin A+B}
            \pars{\iota_A(\F{a})\ordleq\disunionA{\R{x}}}\wedge  f(\F{x}^*,\R{c})  \vee \bigvee_{\disunionB{x} \setin A+B}
            \pars{\iota_A(\F{a})\ordleq \disunionB{\R{x}}}\wedge  g(\F{x}^*,\R{c}) \\
             & =\bigvee_{\disunionA{x} \setin A+B}
            \pars{ \pars{\F{a}\ordleq \R{x}} \wedge  f(\F{x}^*,\R{c})} \vee \bigvee_{\disunionB{x} \setin A+B}
            \pars{ \false \wedge  g(\F{x}^*,\R{c}) } \\
             & =\bigvee_{\disunionA{x} \setin A+B} \pars{\F{a}\ordleq \R{x}} \wedge  f(\F{x}^*,\R{c}) \\
             & =\bigvee_{x\setin A}\pars{ \F{a}\ordleq \R{x}} \wedge f(\F{x}^*,\R{c}) \\
             & =\pars{ \relid_A \mthen f}\tup{\F{a}^*,\R{c}} \\
             & =f(\F{a}^*,\R{c}).
        \end{aligned}
    \end{equation}
    Similarly:
    \begin{equation}
        \begin{aligned}
             & \pars{\comp{\iota_B}\mthen (f\sqcup g)}(\F{b}^*,\R{c}) \\&=\bigvee_{x' \setin A+B}\comp{\iota_B}(\F{b}^*,\R{x})\wedge (f\sqcup g)(\F{x}^*,\R{c})\\
             & =\bigvee_{\disunionA{x} \setin A+B}
            \pars{\iota_B(\F{b})\ordleq\disunionA{\R{x}}}\wedge  f(\F{x}^*,\R{c})  \vee \bigvee_{\disunionB{x} \setin A+B}
            \pars{\iota_B(\F{b})\ordleq \disunionB{\R{x}}}\wedge  g(\F{x}^*,\R{c}) \\
             & =\bigvee_{\disunionA{x} \setin A+B}
            \pars{\false \wedge  f(\F{x}^*,\R{c})} \vee \bigvee_{\disunionB{x} \setin A+B}
            \pars{\pars{\F{b}\ordleq \R{x}} \wedge  g(\F{x}^*,\R{c}) } \\
             & =\bigvee_{\disunionB{x} \setin A+B} \pars{\F{b}\ordleq \R{x}} \wedge  g(\F{x}^*,\R{c}) \\
             & =g(\F{b}^*,\R{c}).
        \end{aligned}
    \end{equation}

    The proof that the disjoint union~$+$ is also a \SY{categorical product} in~\DP is analogous, now with~$f \colon \F{C} \profto \R{A}$,~$g \colon \F{C} \profto \R{B}$, and replacing the injections~$\iota_A$ with projections:
    \begin{equation}
        \begin{aligned}
            \comp{\pi_A} \colon \F{A}+\F{B} & \profto \R{A} \\
            \tup{\F{c}^*,\R{a}}             & \mapsto \pi_A(\F{c})\ordleq_A \R{a},
        \end{aligned}
    \end{equation}
    and
    \begin{equation}
        \begin{aligned}
            \comp{\pi_B} \colon \F{A}+\F{B} & \profto \R{B} \\
            \tup{\F{c}^*,\R{b}}             & \mapsto \pi_B(\F{c})\ordleq_B \R{b},
        \end{aligned}
    \end{equation}
    and the coproduct~$f \sqcup g$ with the product (on morphisms)~$f \times g \colon \F{C} \profto \R{A} + \R{B}$ from~\cref{def:product-dp}.
    We have
    \begin{equation}
        \begin{aligned}
             & \pars{(f\times g)\mthen \comp{\pi_A}}(\F{c}^*,\R{a}) \\
             & =\bigvee_{x'\setin A+B} (f\times g)(\F{c}^*,\R{x'}) \wedge \comp{\pi_A}(\F{x'}^*,\R{a}) \\
             & =\bigvee_{\disunionA{x}\setin A+B}\pars{ f(\F{c}^*,\R{x}) \wedge \comp{\pi}_A(\disunionA{\F{x}}^*,\R{a})} \vee
            \bigvee_{\disunionB{x}\setin A+B}\pars{ g(\F{c}^*,\R{x}) \wedge \comp{\pi}_A(\disunionA{\F{x}}^*,\R{a})} \\
             & =\bigvee_{\disunionA{x}\setin A+B}\pars{ f(\F{c}^*,\R{x}) \wedge (\F{x}\ordleq \R{a})} \vee
            \bigvee_{\disunionB{x}\setin A+B}\pars{ g(\F{c}^*,\R{x}) \wedge \false } \\
             & =\bigvee_{\disunionA{x}\setin A+B} f(\F{c}^*,\R{x}) \wedge (\F{x}\ordleq \R{a}) \\
             & =f(\F{c}^*,\R{a}).
        \end{aligned}
    \end{equation}
    Similarly:
    \begin{equation}
        \begin{aligned}
             & \pars{(f\times g)\mthen \comp{\pi_B}}(\F{c}^*,\R{b}) \\
             & =\bigvee_{x'\setin A+B} (f\times g)(\F{c}^*,\R{x'}) \wedge \comp{\pi_B}(\F{x'}^*,\R{b}) \\
             & =\bigvee_{\disunionA{x}\setin A+B}\pars{ f(\F{c}^*,\R{x}) \wedge \comp{\pi}_B(\disunionA{\F{x}}^*,\R{b})} \vee
            \bigvee_{\disunionB{x}\setin A+B}\pars{ g(\F{c}^*,\R{x}) \wedge \comp{\pi}_B(\disunionB{\F{x}}^*,\R{b})} \\
             & =\bigvee_{\disunionA{x}\setin A+B}\pars{ f(\F{c}^*,\R{x}) \wedge \false } \vee
            \bigvee_{\disunionB{x}\setin A+B}\pars{ g(\F{c}^*,\R{x}) \wedge (\F{x}\ordleq \R{b}) } \\
             & =\bigvee_{\disunionB{x}\setin A+B} g(\F{c}^*,\R{x}) \wedge (\F{x}\ordleq \R{b}) \\
             & =g(\F{c}^*,\R{b}).
        \end{aligned}
    \end{equation}
\end{proof}

\begin{remark}
    Where it is clear from context, we will not distinguish between the injection of \SY{posets}~$\iota_A$ and its corresponding \SY{design problem}~$\comp{\iota_A}$, and refer to both by~$\iota_A$.
    The same holds for the projections~$\pi_A$ and~$\comp{\pi_A}$.
\end{remark}

\begin{table}[b]
    \begin{small}
        \begin{center}
            \begin{tabular}{llll}
                Category              & \Set                   & \Pos                    & \DP \\
                \hline
                Objects               & sets                   & \SY{posets}             & \SY{posets} \\
                Morphisms             & functions              & \SY{monotone functions} & \SY{design problems} \\
                Product (objects)     & \SY{cartesian product} & \SY{cartesian product}  & \SY{disjoint union} \\
                Product (morphisms)   & (not used)             & (not used)              & $\times_\DP$ \\
                Coproduct (objects)   & \SY{disjoint union}    & \SY{disjoint union}     & \SY{disjoint union} \\
                Coproduct (morphisms) & (not used)             & (not used)              & $\sqcup_\DP$ \\
                Biproduct (morphisms) & none                   & none                    & \SY{disjoint union} \\
                Tensor Product        & $\times$ or~$\sqcup$   & $\times$                & $\times$ \\
                \SY{Initial object}   & $\Emptyset$            & $\Emptyset$             & $\Emptyset$ \\
                \SY{Terminal object}  & $\singleton$           & $\singleton$            & $\singleton$
            \end{tabular}
        \end{center}
    \end{small}
    \caption{A comparison of \Pos, \Set, and \DP.
    }
\end{table}

% \begin{comment}
\begin{table}[b]
    \resizebox{\textwidth}{!}{
        \begin{tabular}{ccccccccccc}
            category               &
            objects                & morphisms                   &
            product (ob)           & product (morph)             & coproduct (ob)      & coproduct (morph)
                                   & \SY{biproduct} (morph)
                                   & tensor product
                                   &
            initial
                                   & terminal \\
            \hline

            \Set                   &
            sets                   & functions                   &
            \SY{cartesian product} & (not used)                  & \SY{disjoint union} & (not used)        &
            none
                                   & $\times$ \emph{or} $\sqcup$
                                   &
            $\Emptyset$
                                   & $\singleton$ \\

            \Pos                   &
            \SY{posets}            & \SY{monotone maps}          &
            \SY{cartesian product} & (not used)                  & \SY{disjoint union} & (not used)
                                   &
            none
                                   & $\times$
                                   &
            $\Emptyset$
                                   & $\singleton$ \\

            \DP                    &
            \SY{posets}            & \SY{design problems}        &
            \SY{disjoint union}    & $\times_\DP$                & \SY{disjoint union} & $\sqcup_\DP$
                                   &
            disjoint union
                                   & $\times$
                                   &
            $\Emptyset$
                                   & $\singleton$ \\

            % category &
            % objects & morphisms &
            % product & coproduct & \SY{biproduct} & tensor product
             % initial  & terminal\\
        \end{tabular}}
    \caption{A comparison of \Pos, \Set, and \DP.}
\end{table}
% \end{comment}

\begin{ctdefinition}[Zero object and morphism]
    \label{def:zero-object}
    We call an object that is both initial and terminal the \emph{zero object} and we indicate it as~$0$.
    For any other pairs of objects~$A, B\setin\CatC$, there is a unique morphism of the form~$A \to 0\to B$; we call it the \emph{zero morphism} and denote it~$0_{A,B}\colon A \to B$.
\end{ctdefinition}

\begin{example}
    \Pos has an \SY{initial object}~$\Emptyset$ and a \SY{terminal object}~$\singleton$, but no zero object.
\end{example}

\begin{lemma}
    In the category~\DP, the empty poset~$\Emptyset$ is a zero object.
    The zero morphism~$0_{\F{\cP},\R{\cQ}}\colon \F{\cP} \profto \R{\cQ}$ is the \SY{design problem} that is always infeasible.
\end{lemma}
\begin{proof}
    First,~$\Emptyset$ is an \SY{initial object} in~$\$DP$.
    Indeed, for any poset~$\cP $, there is a unique \SY{design problem}~$\F{\Emptyset} \profto \R{\cP}$ given by the unique \SY{monotone maps}~$\Emptyset=\F{\Emptyset}\op\times \R{\cP} \to\Bool$.
    Similarly, there is a unique \SY{design problem}~$\F{\cP} \profto \R{\Emptyset}$, so~$\Emptyset$ is also terminal.
    For any \SY{posets}~$\cP,\cQ$, the zero map~$0 \colon \F{\cP} \profto \R{\cQ}$ is the \SY{monotone maps}~$\F{\cP} \op\times \R{\cQ} \toinPos \Bool$ sending everything to~$\false$.
\end{proof}

\begin{ctdefinition}[Biproduct]
    \label{def:biproduct}
    Let~\CatC be a category with a zero-object~$0$, and let~$C_1,C_2\setin\CatC$ be objects.
    Suppose that~$\tup{D,\pi_1,\pi_2,\injA,\injB}$ is a tuple such that~$\tup{D,\pi_1,\pi_2}$ is a product of~$C_1$ and~$C_2$ and~$\tup{D,\injA,\injB}$ is a coproduct of~$C_1$ and~$C_2$.
    We say that it is a \emph{biproduct} if, for each~$1\ordleq i,j\ordleq 2$, we have
    \begin{equation}
        \iota_i\mthen \pi_j=
        \begin{cases}
            \relid_{C_i}, & \text{ if }i=j,     \\
            0,            & \text{ if }i\neq j,
        \end{cases} \label{eq:biproduct-condition}
    \end{equation}
    as maps~$C_i\to C_j$ in~\CatC.
\end{ctdefinition}

\begin{lemma}
    The \SY{disjoint union} is a \SY{biproduct} for~\DP.
\end{lemma}
\begin{proof}
    We have already shown that the \SY{disjoint union} is both a product and a coproduct.
    We just need to verify that~\cref{eq:biproduct-condition} holds for the \SY{design problems}~$\pi_A, \pi_B, \iota_A, \iota_B$.
    This amounts to checking the four conditions:
    \begin{equation}
        \begin{aligned}
            (\iota_A\mthen \pi_A)
                                  & = \relid_{A}, \\
            (\iota_A\mthen \pi_B) & = 0_{A,B}, \\
            (\iota_B\mthen \pi_A) & = 0_{B,A}, \\
            (\iota_B\mthen \pi_B) & = \relid_{B}.
        \end{aligned}
    \end{equation}
    We check only the first two, as the other two are similar.
    To check the second condition, we compute an explicit expression for~$(\iota_A\mthen \pi_B)\colon \F{A} \profto \R{B}$, using the definition of \SY{design problem} series composition:
    \begin{equation}
        \begin{aligned}
            (\iota_A\mthen \pi_B)
            \colon  \F{A}\op\times \R{B} & \toinPos \Bool \\
            \tup{\F{x}^*, \R{z}}         & \mapsto
            \bigvee_{y \setin A + B} \iota_A(\F{x}^*,\R{y}) \wedge \pi_B(\F{y}^*,\R{z}).
        \end{aligned}
    \end{equation}
    We can do a case analysis for~$y\setin A+B$.
    Suppose~$y\setin A$.
    Then~$\pi_B(\F{y}^*,\R{z}) = \false$.
    If~$y \setin B$, then~$\iota_A(\F{x}^*,\R{y}) = \false$.
    Therefore, the sum is always false, and hence~$(\iota_A\mthen \pi_B) = 0_{A,B}$.
    For the first condition, we compute an explicit expression for~$(\iota_A\mthen \pi_A) \colon \F{A} \profto \R{A}$:
    \begin{equation}
        \begin{aligned}
            (\iota_A\mthen \pi_A)
            \colon  \F{A}\op\times \R{A} & \toinPos \Bool \\
            \tup{\F{x}^*, \R{z}}         & \mapsto
            \bigvee_{y \setin A + B} \iota_A(\F{x}^*,\R{y}) \wedge \pi_A(\F{y}^*,\R{z}).
        \end{aligned}
    \end{equation}
    %Note that we are summing over $y \setin A + B$.
    Let us divide the ``sum'' in two parts:
    \begin{equation}
        \bigvee_{y \setin A} \iota_A(\F{x}^*,\R{y}) \wedge \pi_A(\F{y}^*,\R{z}) \quad\vee\quad
        \bigvee_{y \setin B} \iota_A(\F{x}^*,\R{y}) \wedge \pi_A(\F{y}^*,\R{z}).
    \end{equation}
    For~$y \setin B$,~$\iota_A(\F{x}^*,\R{y})$ and~$\pi_A(\F{y}^*,\R{z})$ are both false, and we can therefore ignore the second sum.
    From the definition of~$\iota_A$ we have that for~$y\setin A$,~$ \iota_A(\F{x}^*,\R{y})=\F{x} \ordleq_A \R{y}$, and from the definition of~$\pi_A$ we have that for~$y\setin A$, $\pi_A(\F{y}^*,\R{z})=\F{y} \ordleq_A \R{z}$.
    Thus, we compute:
    \begin{equation}
        \begin{aligned}
            \pars{\iota_A\mthen \pi_A}(\F{x}^*, \R{z}) & = \bigvee_{y \setin A} \iota_A(\F{x}^*,\R{y}) \wedge \pi_A(\F{y}^*,\R{z}) \\
                                                       & = \bigvee_{y \setin A} (\F{x} \ordleq_A \R{y}) \wedge  (\F{y} \ordleq_A \R{z}) \\
                                                       & = \F{x} \ordleq_A \R{z} \\
                                                       & = \relid_A(\F{x}^*, \R{z})
        \end{aligned}
    \end{equation}
\end{proof}

\section{Relationship between intersection and monoidal product}
To define the precise relationship between the monoidal product~$f \otimes g$  and the intersection~$f \wedge g$, we first define two operations,~$\mathsf{split} \colon \F{A} \profto \R{A} \times \R{A}$ and~$\mathsf{fuse} \colon \F{A} \times \F{A} \profto \R{A}$, which correspond to splitting and fusing wires in a diagram:
\begin{equation}
    \begin{aligned}
        \mathsf{split} \colon \F{A}\op \times (\R{A} \times \R{A}) & \to_\Pos \Bool \\
        \tup{\F{x}^*, \tup{\R{y}, \R{z}}}                          & \mapsto (\F{x} \ordleq_A \R{y}) \wedge (\F{x} \ordleq_A \R{z})
    \end{aligned}
\end{equation}
~
\begin{equation}
    \begin{aligned}
        \mathsf{fuse} \colon (\F{A} \times \F{A})\op \times \R{A} & \to_\Pos \Bool \\
        \tup{\tup{\F{x}^*, \F{y}^*}, \R{z}}                       & \mapsto (\F{x} \ordleq_A \R{z}) \wedge (\F{y} \ordleq_A \R{z}).
    \end{aligned}
\end{equation}

\begin{lemma}
    \label{lem:intersection_split}
    For~$f,g \colon \F{A} \profto \R{B}$,
    \begin{equation}
        f \wedge g = \mathsf{split} \mthen (f \otimes g) \mthen \mathsf{fuse},
    \end{equation}
    as shown in~\cref{fig:lemmasplitfuse}.
    \begin{figure}[h!]
        \begin{center}
            \includesag{50_split_1_2}
        \end{center}
        \caption{$f \wedge g = \mathsf{split} \mthen (f \otimes g) \mthen \mathsf{fuse}$. }
        \label{fig:lemmasplitfuse}
    \end{figure}
\end{lemma}

\begin{proof}
    Recall that
    \begin{equation}
        (f\otimes g)(\tup{\F{a_1},\F{a_2}}^*, \tup{\R{b_1},\R{b_2}})
        =f(\F{a_1}^*,\R{b_1})\wedge g(\F{a_2}^*,\R{b_2}).
    \end{equation}
    We have
    \begin{equation}
        \begin{aligned}
             & \pars{\pars{f\otimes g} \mthen \mathsf{fuse}}(\tup{\F{a_1},\F{a_2}}^*,\R{b}) \\
             & =\bigvee_{\tup{b',b''}\setin B\times B}(f(\F{a_1}^*,\R{b'})\wedge g(\F{a_2}^*,\R{b''}))\wedge ((\F{b'}\ordleq \R{b}) \wedge (\F{b''}\ordleq \R{b})) \\
             & =f(\F{a_1}^*,\R{b})\wedge g(\F{a_2}^*,\R{b}).
        \end{aligned}
    \end{equation}
    Thus, we have
    \begin{equation}
        \begin{aligned}
             & (\mathsf{split}\mthen f\otimes g\mthen \mathsf{fuse})(\F{a}^*,\R{b}) \\&=\bigvee_{\tup{a',a''}\setin A\times A} \mathsf{split}(\F{a}^*, \tup{\R{a'},\R{a''}})\wedge (f(\F{a'}^*,\R{b})\wedge g(\F{a''}^*,\R{b}))\\
             & =\bigvee_{\tup{a',a''}\setin A\times A}(\F{a}^*\ordleq \R{a'})\wedge (\F{a}^*\ordleq \R{a''})\wedge f(\F{a'}^*,\R{b})\wedge g(\F{a''}^*,\R{b}) \\
             & =f(\F{a}^*,\R{b})\wedge g(\F{a}^*,\R{b}) \\
             & =(f\wedge g)(\F{a}^*,\R{b}).
        \end{aligned}
    \end{equation}
\end{proof}

Is ``$X \times Y$'' the same as ``$Y \times X$''?
It depends on the context.
Intuitively, for the categories of resources treated in this work, we would not make a distinction between ``having~$X$ and~$Y$'' and ``having~$Y$ and~$X$''.
However, there are contexts in which this is not valid.
For example, if we are using~$X \times Y$ to mean that we will have the resource~$X$ today, and the resource~$Y$ tomorrow, then~$Y \times X$ would not be the same as~$X \times Y$.

In the contexts in which this symmetry holds, we call the category ``symmetric monoidal'': we will talk about this more in detail later in the book.

The word ``symmetric'' is well-suited, because to say that the two objects are equivalent, or ``isomorphic'', we can postulate that we always have a way to get
``$X \times Y$'' starting from ``$Y \times X$'' and vice versa.
Diagrammatically, this is depicted by the two arrows that connect them~(\cref{fig:e17}).

\begin{figure}[h!]
    \centering
    \includesag{30_dpcatfig_e17}
    \caption{Objects equivalence in the symmetric case. }
    \label{fig:e17}
\end{figure}

\todotextjira{389}{\bernina: @JL: \JL{This opening bit seems confusing/misleading to me.
        I can edit/correct later; also I would move it to the section on monoidal products.
    }\AC{yes, move to the part on monoidal products, when talking about symmetry}}
