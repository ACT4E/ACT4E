\section{Definitions}

\begin{definition}[Closed category]
    \label{def:closed-category}
    A category $\CatC$ is \emph{closed} if for any pair of objects $\Obja,\Objb$,
    the collection of morphisms $\HomSet\CatC\Obja\Objb$ can be seen as an object of
    \CatC.

    If the category $\CatC$ is closed, it is usual to indicate the homset as $\inthom\Obja\Objb$.
\end{definition}
\begin{definition}[Internal hom functors]
    \label{def:internal-hom-functors}
    For each object $\Obja$ of a closed category $\CatC$, we define the functor
    \begin{equation}
        \inthomleft{\Obja}: \CatC \fto \CatC\op
    \end{equation}
    defined as
    \begin{equation}
        \inthomleft{\Obja}(\Objb) = \inthom\Obja\Objb
    \end{equation}
    and
    \begin{equation}
        \inthomright{\Obja} \colon \CatC \fto \CatC
    \end{equation}
    defined as
    \begin{equation}
        \inthomright{\Obja}(\Objb) = \inthom\Obja\Objb.
    \end{equation}
\end{definition}

\begin{definition}[Closed monoidal category]
    \label{def:closed-monoidal-category}
    A \emph{closed monoidal category} is a monoidal category that is also closed
    in a ``compatible'' way.
    For any object $\Obja$, consider the functor
    \begin{equation}
        \funa_{\Obja} \colon \CatC \fto \CatC
    \end{equation}
    defined as
    \begin{equation}
        \funa_{\Obja}(\Objb) = \Objb \otimes \Obja,
    \end{equation}
    % and consider the functor defined as
    % \begin{equation}
    %     \inthomright{\Obja} \colon \CatC \fto \CatC
    % \end{equation}
    % defined as
    % \begin{equation}
    %     \inthomright{\Obja}(\Objb) = \inthom\Obja\Objb.
    % \end{equation}
    For the category to be closed monoidal, we require that $\funa$ and $\inthomright{\Obja}$ are a pair of adjoint functors.
\end{definition}

\begin{example}
    The category $\Set$ is monoidal closed.
    The condition
    spells out
    %
    \begin{equation}
        \HomSet\Set{\Obja\cartprod\Objb}{\Objc} \simeq \HomSet\Set{\Obja}{\Objc\cartprod\Objb}.
    \end{equation}

\end{example}

See \cref{def:compact-closed-category} and \cref{def:dualizable-object}.

%
%
% \begin{definition}[Compact closed category]
%     \label{def:compact-closed-category}
%         A \emph{compact closed category}, is a symmetric monoidal category in which every object is dualizable.
% \end{definition}

\begin{definition}[Autonomous category]
    \label{def:autonomous-category}
    A closed monoidal category that is also symmetric.
\end{definition}

\begin{definition}[Global dualizing object]
    \label{def:global-dualizing-object}
    A global dualizing object $\starob$ in a closed category~\CatC is an object such that the internal hom functor
    \begin{equation}
        \inthomleft{\starob}: \CatC \fto \CatC\op
    \end{equation}
    defined as
    \begin{equation}
        \inthomleft{\starob}(\Obja) = \inthom\Obja\starob
    \end{equation}
    % \begin{equation}
    %     (\inthom{-}{\starob}) \colon \CatC \to \CatC\op
    % \end{equation}
    is an equivalence of categories, which implies that for all objects $\Obja$ there is a canonical isomorphism
    \begin{equation}
        d_{\Obja}\colon\Obja\mto \func_{\starob}(\func_{\starob}(\Obja)).
    \end{equation}
\end{definition}

\begin{definition}[\starautonomous category]
    \label{def:star-autonomous-category-1}
    A \starautonomous category is an autonomous (symmetric closed monoidal) category with a global dualizing object: an object $\starob$ such that for each object $\Obja$ the canonical morphism
    obtained by applying twice the dualization
    \begin{equation}
        d_{\Obja}: \Obja \to \inthom{(\inthom\Obja\starob)}{\starob}
    \end{equation}
    is an isomorphism.
\end{definition}

\begin{definition}[\starautonomous category]
    \label{def:star-autonomous-categor-2y}
    A \starautonomous category is a symmetric monoidal category
    with a full and faithful functor
    \begin{equation}
        \funa: \CatC\op \fto \CatC
    \end{equation}
    such that there is a natural isomorphism
    \begin{equation}
        f : \HomSet\CatC{\Obja \otimes \Objb}{\funa(\Objc)} \simeq \HomSet\CatC{\Obja}{\funa(\Objb \otimes \Objc)}
    \end{equation}
\end{definition}

\begin{ctdefinition}[Polycategory]
    \label{def:multicategory}
    A \maindef{polycategory} $\PolyA$ is defined by:

    \constit
    \begin{enumerate}
        \item \emph{Objects:} a collection~$\Obof\PolyA$;
        \item \emph{Morphisms:} for each pair $\polyoba, \polyobb$ of \emph{lists} of elements of $\Obof\PolyA$
              a set
              \begin{equation}
                  \HomSet{\PolyA}{\polyoba}{\polyobb}
              \end{equation}
              elements of which are morphisms~$\polyoba \mto \polyobb$,
              also called ``polymaps''.
        \item \emph{Identity morphisms:} for each object~$\Obja$, a morphism
              \begin{equation}
                  \catidat\Obja \setin \HomSet{\PolyA}{\makelist{\Obja}}{\makelist{\Obja}};
              \end{equation}
        \item \emph{Composition operations:} for any tuple of lists of objects
              $\polyoba, \polyobb_1, \polyobb_2, \polyobc_1, \polyobc_2, \polyobd$ and any object $\Obja$,
              a composition map of the type
              \begin{equation}
                  \begin{aligned}
                      \mthen \colon \HomSet{\PolyA}{\polyoba}{\polyobb_1,\Obja,\polyobb_2}
                      \cartprod
                      \HomSet{\PolyA}{\polyobc_1,\Obja,\polyobc_2}{\polyobd} \\
                      \sto
                      \HomSet{\PolyA}{\polyobc_1,\polyoba,\polyobc_2}{\polyobb_1,\polyobd,\polyobb_2}.
                  \end{aligned}
              \end{equation}
    \end{enumerate}
    \condit
    \begin{enumerate}
        \item \emph{Associativity:} for any triple of morphism $\mora,\morb,\morc$ where
              \begin{equation}
                  \mora \colon \polyoba_1 \mto \polyoba_2,\Obja,\polyoba_3,
                  \quad
                  \morb  \colon \polyobb_1 , \Obja,  \polyobb_2 \mto \polyobb_3 \Objb \polyobb_4,
                  \qquad
                  \morc \colon \polyobc_1 , \Objb ,  \polyobc_2 \mto \polyobc_3.
              \end{equation}
              $(\mora\mthen\morb)\mthen\morc$ and  $\mora\mthen(\morb\mthen\morc)$  coincide
              and so we indicate them as
              \begin{equation}
                  \mora\mthen\morb\mthen\morc \colon \polyobc_1 , \polyobb_1 ,  \polyoba_1 , \polyobb_2  ,\polyobc_2
                  \mto
                  \polyoba_2, \polyobb_3, \polyobc_3 ,\polyobb_4, \polyoba_3.
              \end{equation}
        \item \emph{Neutrality}: for each $\mora: \polyoba, \Obja, \polyobb \mto \polyobc$, it holds that
              \begin{equation}
                  \catidat\Obja  \mthen \mora  =\mora.
              \end{equation}
              For each $\morb: \polyoba \mto \polyobc, \Obja, \polyobd$, it holds that
              \begin{equation}
                  \morb \mthen \catidat\Obja =\morb.
              \end{equation}
        \item \emph{Partial commutativity}:
              For $\mora: \Gamma_1 \mto  \Gamma_2, \Obja, \Gamma_3$, $\morb: \Delta_1 \mto  \Delta_2, \Obja , \Delta_3$, and
              $\morc: \Phi_1, \Obja, \Phi_2, \Objb \Phi_3 \mto \Phi_4$, if at least one of $\Gamma_2$ and $\Delta_2$ is empty,
              and at least one of $\Gamma_3$ and $\Delta_3$ is empty, then $\mora\mthen(\morb\mthen\morc)$ and $\mora\mthen(\morb\mthen\morc)$ coincide.
              Moreover, the symmetric condition (with $\mora: \Gamma_1, \Obja, \Gamma_2 \mto \Gamma_3$) holds as well.
    \end{enumerate}
\end{ctdefinition}
