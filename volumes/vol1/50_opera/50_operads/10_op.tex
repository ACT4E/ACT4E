% !TEX root = chapter-standalone.tex
\section{Definition}
\label{sec:operads-definition}
\todostructurejira{301}{\bernina: @Gioele: Reorganize the videos in bigger chunks}

\linkvideo{spring2021-operads-a:intro-idea-operad} % The basic idea of operad
\linkvideo{spring2021-operads-a:terminology-operad} % Terminology
\linkvideo{spring2021-operads-a:composition-operad} % Composition
\linkvideo{spring2021-operads-a:notation-operad} % Notation
\linkvideo{spring2021-operads-a:operad-sets} % Operad of sets (via product)
\linkvideo{spring2021-operads-a:operad-disks} % Operad of little disks
\linkvideo{spring2021-operads-a:operad-def} % Definition of operad

\begin{ctdefinition}[Multicategory]
    \label{def:multicategory}
    A \maindef{multicategory} $\MulticatM$ is:

    \constit
    \begin{enumerate}
        \item \emph{Objects:} a collection~$\Obof\MulticatM$;
        \item \emph{Morphisms:} for each~$k \setin \natnumbers$, each tuple~$\tup{\Obja_1,\ldots, \Obja_k}$ of objects, and each object~$ \Objb$, one specifies a set~
              \begin{equation}
                  \HomSet{\MulticatM}{\tup{\Obja_1,\ldots,\Obja_k}}{\Objb},
              \end{equation}
              elements of which are morphisms~$\tup{\Obja_1,\ldots,\Obja_k}\mto \Objb$;
        \item \emph{Identity morphisms:} for each object~$\Obja$, a morphism
              \begin{equation}
                  \catidat\Obja \setin \HomSet{\MulticatM}{\tup{\Obja}}{\Obja};
              \end{equation}
        \item \emph{Composition operations:} for each $m \setin \natnumbers$, $k_1, \dots, k_m \setin \natnumbers$, and each collection of objects as below, a function of the type
              \begin{equation}
                  \begin{aligned}
                      \HomSet{\MulticatM}{ \tup{\Obja_{1}^1, \ldots, \Obja_{k_1}^1} }{\Objb_1} \times \ldots \times & \ \HomSet{\MulticatM}{ \tup{\Obja_{1}^m, \ldots, \Obja_{k_m}^m} }{\Objb_m} \\
                      \qquad \times \ \HomSet{\MulticatM}{ \tup{\Objb_{1}, \ldots,\Objb_{m}} }{\Objc}               & \mto \HomSet{\MulticatM}{\tup{\Obja_{1}^1, \ldots, \Obja_{k_m}^m}}{\Objc}
                  \end{aligned}
              \end{equation}
              denoted by
              \begin{equation}
                  \tup{\mora_1,\ldots,\mora_m,\morb} \mapsto \tup{\mora_1,\ldots, \mora_m}\mthen \morb.
              \end{equation}
    \end{enumerate}
    \condit
    \begin{enumerate}
        \item \emph{Associativity:} for each $m \setin \natnumbers$, $k_1, \dots, k_m \setin \natnumbers$, and each collection of morphisms as below, the equation
              \begin{equation}
                  \begin{aligned}
                      \tup{ \tup{\mora_{1}^1,\ldots, \mora_{n_1}^1} \mthen \morb_1, \tup{\mora_{1}^2,\ldots, \mora_{n_2}^2 }\mthen \morb_2,\ldots, \tup{ \mora_{1}^m,\ldots, \mora_{n_m}^m } \mthen \morb_m } \mthen \morc \\ = \ \tup{ \mora_{1}^1,\ldots, \mora_{n_m}^m } \mthen \left ( \tup{ \morb_{1},\ldots, \morb_{m} } \mthen \morc \right ).
                  \end{aligned}
              \end{equation}
        \item \emph{Neutrality}: for each $m \setin \natnumbers$ and each morphism of type $\mora \colon \tup{\Obja_1,\ldots,\Obja_m}\mto \Objb$, the equations
              \begin{equation}
                  \tup{\catidat{\Obja_1},\ldots, \catidat{\Obja_m}}
                  \mthen \mora = \mora = \mora \mthen \catidat\Objb.
              \end{equation}
    \end{enumerate}
\end{ctdefinition}

\begin{ctdefinition}
    \label{def:multicat_from_monoidal}
    Let~$\tup{\CatC, \mtimescat, \idmoncat}$ be a \SY{strict monoidal category}.
    Its associated \SY{mutlicategory} $\MulticatM = \MulticatM(\CatC)$ has:
    \begin{enumerate}
        \item \emph{Objects}: $\Obof{\MulticatM}=\ObC$;
        \item \emph{Morphisms}: For all $n \setin \natnumbers$ and objects as below, we define
              $$\HomSet{\MulticatM}{\tup{\Obja_1,\ldots,\Obja_n}}{\Objb}=\HomSet{\CatC}{\Obja_1 \mtimescat \ldots \mtimescat \Obja_n}{\Objb}$$ when $n \geq 1$ and
              $$\HomSet{\MulticatM}{\tup{ \ }}{\Objb} = \HomSet{\CatC}{\idmoncat}{\Objb} $$
              when $n=0$
        \item \emph{Identity morphism}: $\catidat\Obja \setin \HomSet{\MulticatM}{\tup{\Obja}}{\Obja}=\catidat\Obja \setin \HomSet{\CatC}{\Obja}{\Obja}$;
        \item \emph{Composition of morphisms:} induced from the composition in $\CatC$.
    \end{enumerate}
\end{ctdefinition}

\linkvideo{spring2021-operads-a:operad-multilin-maps} % Operad of multilinear maps
\devel{
    \begin{example}[Multilinear maps]
        \todotextjira{302}{\bernina: Finish trascribing}
    \end{example}
}

\linkvideo{spring2021-operads-a:swiss-cheese-operad}
\devel{
    \begin{example}[Swiss cheese operad]
        \todotextjira{302}{\bernina: Finish trascribing}
    \end{example}
}

\linkvideo{spring2021-operads-b:network-operad} % Network operads
\devel{
    \begin{example}[Network operads]
        \todotextjira{302}{Finish trascribing}
    \end{example}
}

\linkvideo{spring2021-operads-a:operad-monoidal-cat} % Operad from a monoidal category
\begin{remark}
    \label{multicat-from-monoidal}
    For non-strict \SY{monoidal categories} it is also possible to define an associated \SY{multicategory} similarly.
    There are in fact various ways of going about this.
    One way is to choose a convention about bracketing: for instance one might set
    \begin{equation}
        \HomSet{\MulticatM_\CatC}{\tup{\Obja_1,\ldots,\Obja_n}}{\Objb}=\HomSet{\CatC}{ ( \dots (\Obja_1 \mtimescatob \Obja_2) \ldots \mtimescatob \Obja_n)}{\Objb}
    \end{equation}
    by putting all brackets in the left-most manner.
    Another approach is to invoke a well-known theorem (called the ``strictification theorem'' for \SY{monoidal categories}) that says that every \SY{monoidal category} is monoidally equivalent to a \SY{strict monoidal category}.
    Then, we can assume that one is working with the ``strict version'' of a given \SY{monoidal category}.
    We leave out discussing these technical details any further here.
    When using \SY{operads} associated with \SY{monoidal categories} we will sometimes simply act ``as if'' our \SY{monoidal categories} are strict, even when they are not.
\end{remark}

\begin{example}
    Let $\CatC = \Set$ be the category of sets and functions, and view it as equipped with the monoidal structure induced by the \SY{cartesian product}.
    This category is \emph{not} strict monoidal as is, however, in light of \cref{multicat-from-monoidal}, we can still define an associated \SY{operad}.
    As mentioned there, we will sweep the details under the rug for the time-being and act as if the category~\Set were indeed strict monoidal.
    The associated \SY{operad} that we obtain from the \SY{monoidal category} \Set will also be called~\Set.
\end{example}

\linkvideo{spring2021-operads-b:functors-operads} % Functors between operads
\begin{ctdefinition}
    \label{def:functors-multicats}
    Let~$\MulticatM,\MulticatN$ be \SY{multicategories}.
    A \maindef{functor}~$\funa\colon \MulticatM \fto \MulticatN$ between multicategories is:

    \constit
    \begin{enumerate}
        \item A function~$\funaob\colon \Obof\MulticatM\sto \Obof\MulticatN$;
        \item A function
              \begin{equation}
                  \funamor\colon \HomSet{\MulticatM}{\tup{\Obja_1,\ldots,\Obja_n}}{\Objb}\sto \HomSet{\MulticatN}{\tup{\funaob(\Obja_1),\ldots, \funaob(\Obja_n)}}{\funaob(\Objb)}.
              \end{equation}
    \end{enumerate}

    \condit

    Conditions which encode \emph{compatibility with compositions} and \emph{compatibility with identity morphisms}.
    We do not spell these out here; however they are analogous to the ones in the definition of a \SY{functor} between categories.
\end{ctdefinition}

\linkvideo{spring2021-operads-b:operad-algebras} % Algebras for an operad

\begin{ctdefinition}[Algebra of a multicategory]
    \label{def:algebra-of-multicat}
    \SYNDEF{algebra of a multicategory}
    Let~$\MulticatM$ be a \SY{multicategory}.
    An \emph{algebra} for~$\MulticatM$ is a \SY{functor} of \SY{multicategories}
    $$\MulticatM \fto \Set.
    $$
\end{ctdefinition}

\linkvideo{spring2021-operads-b:sets-algebras} % Sets as algebras
\devel{
    \begin{example}[Sets as algebras]
        \todotextjira{302}{\bernina: Finish trascribing}
    \end{example}
}
\linkvideo{spring2021-operads-b:semigroups-algebras} % Semigroups as algebras
\devel{
    \begin{example}[Semigroups as algebras]
        \todotextjira{302}{\bernina: Finish trascribing}

    \end{example}
}

\linkvideo{spring2021-operads-b:single-typed-branches} % Single Typed Branches
\linkvideo{spring2021-operads-b:operad-blueprint} % Operads as blueprints
\linkvideo{spring2021-monads-b:blueprint-monoids} % Blueprinting monoids

\todotext{J: @J: rewrite the two graded exercises below with better/updated notation}

\begin{gradedexercise}[\exname{MonoidsAsAlgebras}]
    \label{ex:MonoidsAsAlgebras}
    Let $\MulticatM$ be the following \SY{operad}.
    It has just a single object, which we call $\star$.
    For each natural number $n \setin \natnumbers$ (including $0$), we let $[n]$ denote the list consisting of $n$ copies of the object $\star$ (so $[0]$ in this case is a synonym for the empty list).
    With this notation, we define $\Hom_\MulticatM([n];1)$ to be the 1-element set $\makeset{ * }$ for each $n \setin \natnumbers$.
    The intuition is that, for each $n$, $\Hom_\MulticatM([n];1)$ represents a single $n$-ary operation.

    In this exercise, the task is to describe what an algebra for this \SY{operad} is.
    Try to prove as many of the statements that you make as you can.
\end{gradedexercise}

\solutionof{MonoidsAsAlgebras}

\linkvideo{spring2021-operads-b:actions-recap} % Recap Actions
\linkvideo{spring2021-operads-b:monoid-act-algebras} % Monoid actions as algebras

\begin{gradedexercise}[\exname{MonoidActionsAsAlgebras}]
    \label{ex:MonoidsActionsAsAlgebras}
    For the duration of this exercise, we fix a \SY{monoid} $\monA$.
    We define an $\MulticatM$ as follows.
    It has a single object, which we call $\star$.
    For each natural number $n \setin \natnumbers$ (including $0$), let $[n]$ denote the list consisting of $n$ copies of the object $\star$.
    We define $\Hom_\MulticatM([n];1)$ to be the set underlying $\monA$ if $n =1$, and we set $\Hom_\MulticatM([n];1) = \Emptyset$ if $n \neq 1$.

    Can you describe what an algebra for this \SY{operad} is?
    Please prove your statements as best you can.
\end{gradedexercise}

\solutionof{MonoidActionsAsAlgebras}

\devel{
    \begin{definition}
        \label{def:cospan}
        Let~\CatC be a category.
        A \maindef{cospan} in~\CatC is a pair of morphisms to a common object:
        \equationsag{cospan}{eq:cospan}
    \end{definition}

    \linkvideo{spring2021-operads-b:co-span-operad} % Cospan Operad
    \linkvideo{spring2021-operads-b:wiring-diags-operads} % Wiring diagram operads

}

\begin{gradedexercise}[\exname{HwkCospanMulticat}]
    \label{ex:HwkCospanMulticat}
    
Consider the multicategory \textbf{Cospan}, where objects are sets and a morphism
\begin{equation}
\tup{\stylesets{X}_1, \dots, \stylesets{X}_n} \mto \stylesets{Y}
\end{equation}
is a cospan of the type
\begin{equation}
\stylesets{X}_1 \setdisunion  \dots \setdisunion  \stylesets{X}_n \overset{\mora_l}{\longrightarrow} \setC \overset{\mora_r}{\longleftarrow}  \stylesets{Y}.
\end{equation}

We can visualize such cospans using diagrams that may be nested inside of each other. A  diagram corresponding to $\stylesets{X}_1 \setdisunion  \dots \setdisunion  \stylesets{X}_n \overset{\mora_l}{\longrightarrow} \setC \overset{\mora_r}{\longleftarrow}  \stylesets{Y}$ is constructed as follows: 
\begin{itemize}
\item draw a larger circle, corresponding to  $\stylesets{Y}$;
\item inside of the larger circle, draw disjoint smaller circles for each of the $\stylesets{X}_k$, for $1 \leq k \leq n$; 
\item for each element $x$ of $\stylesets{X}_1$ through $\stylesets{X}_n$ or of $\stylesets{Y}$, draw a small segment of a wire crossing the border of the circle corresponding to the set of which $x$ is an element;
\item for each element of $\setC$, draw a node in the space inside the larger circle but outside all the smaller circles; 
\item for each wire protruding outward from the smaller circles, connect that wire (which is associated to some element $x$, say) to the node representing the element $\mora_l(x)$ of $\setC$;
\item for each wire protruding inward from the larger circle (which is associated to some element $x$, say), connect that wire to the node representing the element $\mora_r(x)$ of $\setC$.
\end{itemize}

\

Your task: draw a diagram representing each of the following cospans.

Hint: it can also be helpful to draw each of the cospans first. 

\begin{enumerate}
\item The cospan $\stylesets{X}_1 \setdisunion \stylesets{X}_2 \overset{\mora_l}{\longrightarrow} \setC \overset{\mora_r}{\longleftarrow}  \stylesets{Y}$ where 
\begin{itemize}
\item $\stylesets{X}_1 = \makeset{1, 2}$, $\stylesets{X}_2 = \makeset{3, 4}$, $\stylesets{Y} = \makeset{\heartsuit, \star}$, $\stylesets{C} = \makeset{a, b, c}$;
\item $\mora_l$ is given by the table
\begin{center}
\begin{tabular}{c|c}
$x$ & $\mora_l(x)$\\
\hline
$1$ & $a$ \\
$2$ & $b$ \\
$3$ & $b$ \\
$4$ & $c$ 
\end{tabular}
\end{center}
\item $\mora_r$ is given by the table
\begin{center}
\begin{tabular}{c|c}
$x$ & $\mora_r(x)$\\
\hline
$\heartsuit$ & $a$ \\
$\star$ & $c$ 
\end{tabular}
\end{center}
\end{itemize}
\item The cospan $\stylesets{X}_1 \setdisunion \stylesets{X}_2 \setdisunion \stylesets{X}_3 \overset{\mora_l}{\longrightarrow} \setC \overset{\mora_r}{\longleftarrow} \stylesets{Y}$ where 
\begin{itemize}
\item $\stylesets{X}_1 = \makeset{1, 2}$, $\stylesets{X}_2 = \makeset{3, 4}$, $\stylesets{X}_3 = \makeset{5, 6}$, $\stylesets{Y} = \emptyset$, $\stylesets{C} = \makeset{a, b, c}$;
\item $\mora_l$ is given by the table
\begin{center}
\begin{tabular}{c|c}
$x$ & $\mora_l(x)$\\
\hline
$1$ & $b$ \\
$2$ & $a$ \\
$3$ & $a$ \\
$4$ & $c$ \\
$5$ & $c$ \\
$6$ & $b$ 
\end{tabular}
\end{center}
\end{itemize}
\item The cospan $\stylesets{X}_1 \setdisunion \stylesets{X}_2 \overset{\mora_l}{\longrightarrow} \setC \overset{\mora_r}{\longleftarrow} \stylesets{Y}$ where
\begin{itemize}
\item $\stylesets{X}_1 = \makeset{1, 2}$, $\stylesets{X}_2 = \makeset{3, 4, 5, 6}$, $\stylesets{Y} = \makeset{t, u, v, w}$, $\stylesets{C} = \makeset{a, b, c}$;
\item $\mora_l$ is given by the table
\begin{center}
\begin{tabular}{c|c}
$x$ & $\mora_l(x)$\\
\hline
$1$ & $a$ \\
$2$ & $c$ \\
$3$ & $b$ \\
$4$ & $b$ \\
$5$ & $a$ \\
$6$ & $c$ 
\end{tabular}
\end{center}
\item $\mora_r$ is given by the table
\begin{center}
\begin{tabular}{c|c}
$x$ & $\mora_r(x)$\\
\hline
$t$ & $a$ \\
$u$ & $b$ \\
$v$ & $b$ \\
$w$ & $c$ 
\end{tabular}
\end{center}
\end{itemize}
\end{enumerate}

\end{gradedexercise}

\solutionof{HwkCospanMulticat}
