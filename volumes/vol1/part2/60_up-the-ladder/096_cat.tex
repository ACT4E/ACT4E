% !TEX root = standalone.tex


\section{Functor composition}

\JL{I think this should be a definition, not a lemma. }

In the following, we want to show that functors compose.
Given categories~$\CatA,\CatB,\CatC$ and functors~$F\colon \CatA\to \CatB$,~$G\colon \CatB\to \CatC$, we want to show that~$F\then G$ is a functor. To do this, we show that~$F\then G$ preserves identities and compositions.
\begin{itemize}
  \item Given an object~$\obja\in \CatA$, we have:
  \begin{equation*}
    \begin{aligned}
    (F\then G)(\catid_\obja)
      &=G(F(\catid_\obja))\\
      &=G(\catid_{F(\obja)})\\
      &=\catid_{G(F(\obja))},
    \end{aligned}
  \end{equation*}
  where we used that~$F$ and~$G$ are functors (\ie , they preserve identities).
  \item Furthermore, given composable morphisms $f,g\in \CatA$, one has:
  \begin{equation*}
    \begin{aligned}
    (F\then G)(f\then g)
      &=G(F(f)\then F(g))\\
      &=G(F(f))\then G(F(g)),
    \end{aligned}
  \end{equation*}
  where again used that~$F,G$ are functors (\ie , they preserve composition).
\end{itemize}


We can define an identity functor. Given a category~\CatC, we define it as~$\catid_\CatC\colon \CatC \to \CatC$,~$\catid_\CatC(x)=x$ for every object and morphism in~\CatC. To show that this is a valid functor, we need to show that it preserves identities and composition:
\begin{itemize}
  \item Given any~$\obja \in \Ob_\CatC$, we have:
  \begin{equation*}
    \begin{aligned}
      \catid_\CatC(\catid_\obja)&=\catid_\obja\\
      &=\catid_{\catid_\CatC(\obja)}
    \end{aligned}
  \end{equation*}
  Furthermore, given composable morphisms~$f,g$ in~\CatC, we have:
  \begin{equation*}
    \begin{aligned}
      \catid_{\CatC}(f\then g)&=f\then g\\
      &=\catid_{\CatC}(f)\then \catid_{\CatC} (g).
    \end{aligned}
  \end{equation*}
\end{itemize}


\section{A category of categories}

Given the existence of an identity functor and the ability of functors to compose, we can define a category of categories \Category.

\begin{ctdefinition}[Category of small categories]
  \label{def:Category}
  There is a category, called \Category, which is constituted of
  \begin{itemize}
    \item Objects: categories;
    \item Morphisms: functors;
    \item Identity morphisms: identity functors;
    \item Composition: composition of functors.
  \end{itemize}
\end{ctdefinition}
