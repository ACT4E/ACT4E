% !TEX root = standalone.tex


\section{Natural transformations}

We have seen that functors are ``morphisms between categories''. Indeed, categories may be assembled into a category \index{\Category} where the objects are categories and morphisms are functors. It turns out that there is an important third layer to this world of categories: there are also kinds of morphisms \emph{between} functors, and these are known as ``natural transformations''. To represent the three layers of structure involved in the world of categories, we will often draw diagrams like this:
\begin{center}
  \includesag{3-layer-diagrams}
\end{center}

where the points represent categories, the single arrows represent functors, and the double arrows represent natural transformations.


\begin{ctdefinition}[Natural transformation]
  \label{def:natural-transformation}
  Let \CatC and \CatD be categories, and let~$\funa,\funb\colon \CatC\mto \CatD$ be functors. A \emph{\iindex{natural transformation}}~$\ntrafoa \colon \funa\mto \funb$
  \begin{center}
    \includesag{55_natural_1}
  \end{center}
  is defined by the following constituent data, satisfying the following condition.
  \underline{Data:}\\
  \begin{compactenum}
    \item For each object~$\obja\in \CatC$, a morphism $\ntrafoa_\obja \colon \funa(\obja)\mto \funb(\obja)$ in \CatD, called the $\obja$\emph{-component} of $\ntrafoa$.
  \end{compactenum}
  \underline{Condition:}
  \begin{compactenum}
    \item For every morphism~$\mora\colon \obja\mto \objb$ in \CatC, the components of $\ntrafoa$ must satisfy the \emph{naturality condition}
    \begin{equation}
      \funa(\mora)\then \ntrafoa_\objb = \ntrafoa_\obja\then \funb(\morb).
    \end{equation}
    In other words, the following diagram must commute:
    \begin{center}
      \includesag{55_natural_2}
    \end{center}
  \end{compactenum}
  The situation is represented diagrammatically in~\cref{fig:nat_trans_graphically}.
\end{ctdefinition}

\begin{figure}[h!]
  \begin{center}
%\adjustbox{scale=1,center}
  \end{center}
  \caption{\todographics{Shade the categories $\CatC, \CatD$}}
  \label{fig:nat_trans_graphically}
\end{figure}

\begin{ctdefinition}[Natural isomorphism]
  \label{def:nat_iso}
  A natural transformation~$\ntrafoa \colon \funa \to \funb $ is called a \emph{\iindex{natural isomorphism}} if each component~$\ntrafoa_\obja$ is an isomorphism in \CatD.
\end{ctdefinition}




\begin{example}\label{ex:Vect}
  Consider the category~$\Vect_{\reals}$ whose objects are real vector spaces and whose morphisms are linear maps. (For convenience, in the following we sometimes omit reference to the ground field.) Recall that the \emph{dual} of a vector space~$V$ is the vector space
  \begin{equation*}
    V^* \definedas \Hom_{\Vect}(V, \reals),
  \end{equation*}
  \ie , the space of all linear maps from~$V$ to~\reals. Also, recall the if~$f\colon V \to W$ is a linear map, then its dual is a linear map~$f^*\colon W^* \to V^*$.

  Applying the above duality construction twice to a vector space or a linear map gives their double dual. It turns out that this is a functorial operation. That is, there is a functor~$\text{Double dual}\colon \Vect \to \Vect$ that maps every vector space and every linear map to its double dual.

  Furthermore, for any vector space~$V$, there is a ``canonical'' or ``natural'' map~$\ntrafoa_V \colon V \to V^{**}$ defined by
  \begin{equation*}
    \ntrafoa_V(v)(l) = l(v), \quad  v \in V, l \in V^*.
  \end{equation*}
  These form the components of a natural transformation from the identity functor on \Vect to the double dual functor.
  \begin{center}
    \includesag{nat-trafo-ddual}
  \end{center}
\end{example}



