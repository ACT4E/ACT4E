


\section{Example of a ``Free-Forgetful'' adjunction}

Another ``type'' of adjunction that appears frequently can be called ``Free-Forgetful'' adjunction. Such adjunctions are composed of a ``free functor'' and a ``forgetful functor''. These terms are informal, but the idea is this. A free functor~$\CatC \to \CatD$ typically takes an object~$\obja$ of~$\CatC$ and ``freely'' adds some structure to it. ``Free'' means that only those structures and conditions are added that are absolutely necessary to make~$\obja$ an object of~$\CatD$, and otherwise the functor does not impose any constraints or relations. Conversely, a ``forgetful functor'' usually starts from an object~$\objb$ on \CatD which has some structure, and ``forgets'' some of this structure, which results in us being able to view~$\objb$ as an object in~$\CatC$.

For example: any real vector space is built from an underlying set, together with extra structure given by operations (vector addition and scalar multiplication). There is a forgetful functor from the category~$\Vect_{\reals}$ of real vector spaces to~$\Set$ which maps any vector space to its underlying set of vectors. On the other hand, there is a ``free'' construction going the other way: given a set~$X$, we can build the ``free real vector space generated by~$X$''. To do this, we think of the elements of~$X$ as basis vectors, and we build a vector space by taking formal finite~$\reals$-linear combinations of them.

In the following we will consider an example in detail where we ``freely'' generate a category from a directed graph.


Let~$\Graph$ be the category of directed graphs and \Category the category of (small) categories.
There is a functor~$F \colon \Graph \to \Category$ which turns any directed graph~$D = \tup{V,E, s,t}$ into a category whose objects are the vertices~$V$ and whose morphisms are finite directed paths between vertices. This is called the \emph{free category generated by the graph~$D$} (\cref{sec:catsfromgraphs}). There is also a functor~$G \colon \Category \to \Graph$ which turns a category~$\CatC$ into a graph where the set of vertices is~$\Ob_{\CatC}$ and there is a directed edge between vertices for every morphism in \CatC between the corresponding vertices.

Let's first describe this adjunction via \cref{def:adj-iso}. The natural isomorphism
\begin{equation*}
  \tau\colon \Hom_{\Category}(F(- ), - ) \to \Hom_{\Graph}(- , G( - ))
\end{equation*}
is the one whose component at~$\tup{D,\CatC}$ is the isomorphism
\begin{equation*}
  \tau_{D,\CatC} \colon \Hom_{\Category}(F(D), \CatC) \to \Hom_{\Graph}(D, G(\CatC))
\end{equation*}
which assigns to any functor~$F\colon F(D) \to \CatC$ the morphism of graphs~$D \colon G(\CatC)$ given by restricting~$F$ to~$D$ and only keeping track of its action on vertices and edges (i.e., we ignore it's compositional properties and think of it just as a graph morphism).

Now let's consider this adjunction from the perspective of \cref{def:adj-counit}. The component at~$D$ of the counit is the morphism of graphs
\begin{equation*}
  \eta_D \colon D \to  G(F(D))
\end{equation*}
which includes~$D$ into the graph~$G(F(D))$. The latter has an edge from the source to the target of every finite path in~$D$. The paths of length zero are what corresponded to identity morphisms in~$F(D)$, and the paths of length one constitute a copy of~$D$ inside~$G(F(D))$.

What does the unit look like? It's component at~$\CatC$ is a functor
\begin{equation*}
  \epsilon_{\CatC} \colon F(G(\CatC)) \to \CatC.
\end{equation*}

The category~$F(G(\CatC))$ is larger than~$\CatC$: starting with \CatC, the graph~$G(\CatC)$ will contain edges for all the morphisms in~$\CatC$, but it will forget their compositional interlinking. In particular, for example, it will forget which loops denote identity morphisms (i.e., which morphisms act neutrally) and, more generally, it will forget when different compositions of morphism give the same result. In~$F(G(\CatC))$, then, morphism compositions that might have given the same result in~$\CatC$ will now be distinct.
The functor~$\epsilon_{\CatC}$ in a sense ``remembers'' those relations that were true in \CatC and it ``implements'' them by ``projecting''~$F(G(\CatC))$ back to~$\CatC$.

