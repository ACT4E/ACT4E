% !TEX root = standalone.tex


\section{An example}

\JL{An example of a Galois connection between functionalities and resources could be treated here to motivate the adjunction discussion}


\section{Adjunctions: hom-set definition}
In this section we give a definition of adjunction which can be viewed as an analogy with the following situation in linear algebra. Suppose~$V$ and~$W$ are finite-dimensional real vector spaces, equipped with inner products~$(-, -)_V$ and~$(-, -)_W$, respectively. The adjoint of a linear map~$F\colon V \to W$ is a linear map~$F^*\colon W \rightarrow V$ such that
\begin{equation*}
(Fv, w)
  _W = (v, F^*w)_V, \quad \forall v \in V, w \in W.
\end{equation*}

\begin{ctdefinition}[Adjunction, Version 1]
  \label{def:adj-iso}
  \label{def:cat-adjunction-v1}
  Let \CatC and \CatD be categories. An \emph{\iindex{adjunction}} from \CatC to \CatD is given by the following data:
  \begin{compactenum}
    \item A functor~$F\colon \CatC \to \CatD$ (the \emph{left adjoint});
    \item A functor~$G\colon \CatD \to \CatC$ (the \emph{right adjoint});
    \item A natural isomorphism~$\tau : \Hom_{\CatD}(F - , - ) \Longrightarrow \Hom_{\CatC}(- , G - )$
  \end{compactenum}
  We use the notation~$F \dashv G$ to indicate that~$F$ and~$G$ form an adjunction, with $F$ the left adjoint and $G$ the right adjoint.
\end{ctdefinition}

\begin{remark}
  Note that~$\tau$ is a natural isomorphism between functors of the form
  \begin{equation}
    \CatC^{op} \times \CatD \longrightarrow \Cat{Set}
  \end{equation}
\end{remark}




