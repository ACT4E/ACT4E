% !TEX root = standalone.tex


\section{Chains and Antichains}
\label{sec:chains-antichains}
\todo{This part is partially repeated from above. Do chains, antichains, upper/lower sets before "poset as a category". }

\begin{definition}[Chain in a poset]
  \label{def:chain}
  Given a poset~$\posA$, a \emph{chain} is a sequence of elements~${s_i}$ in~$S$ where two successive elements are comparable:
  \begin{equation}
    i \Nleq j \Rightarrow s_i \posAleq s_j.
  \end{equation}
\end{definition}


\begin{definition}[Antichain in a poset]
  \label{def:antichain}
  An \emph{antichain} is a subset~$S$ of a poset where no elements are comparable.
  If~$a,b \in S$, then~$a \posAleq b$ implies~$a=b$.
\end{definition}

We denote the set of antichains of a poset~$\posA$ by~$\antichains\posA$.

\begin{remark}
  Note that given a poset~$\tup{\posA,\posAleq}$, the empty set~$\emptyset$ is both a chain and an antichain.
\end{remark}

In the context of pizza recipes, consider the diagram reported in~\cref{fig:antichain}.
The blue points represent an antichain of recipes~$\{\tup{\unit[1]{\stdcurr},\unit[2]{h}},\tup{\unit[2]{\stdcurr},\unit[1]{h}}\}$.
It is a set of antichains because they do not dominate each other: one is cheaper, but takes longer, and the other is more expensive, but quicker.
The red point represents a recipe which cannot be part of the antichain, since it is dominated by~$\tup{\unit[2]{\stdcurr},\unit[1]{h}}$.

\todographics{do black points, but with labels}
\todographics{Add an antichain that is a continuous line, like $y = 1 - x$}
\begin{figure}[h!]
  \begin{center}
    \includesag{70_antichain}
  \end{center}
  \caption{Example of antichains. }
  \label{fig:antichain}
\end{figure}


\begin{example}
  Let's consider the poset~$\tup{\posA,\posAleq}$ where~$a\posAleq b$ if~$a$ is a divisor of~$b$ and~$\posA=\{1,5,10,11,13,15\}$. A chain of~$\posA$ is~$\{1,5,10,15\}$. An antichain of~$\posA$ is~$\{10,11,13\}$.
\end{example}

\begin{example}
  Consider \cref{ex:hasseinclusion}. Examples of chains are
  \begin{equation}
    \{\varnothing,\{a\},\{a,b\},\{a,b,c\}\}, \quad  \{\varnothing,\{b\},\{b,c\},\{a,b,c\}\}.
  \end{equation}
  Examples of antichains are
  \begin{equation}
    \{\{a\},\{b\},\{c\}\}, \quad \{ \{a,b\},\{a,c\}, \{b,c\}\}.
  \end{equation}
\end{example}

\begin{example}
  \label{ex:battery}
  Suppose you have to choose a battery model based on its cost and its weight, both to be minimized.
  There may be models which dominate others.
  For instance, a model~$\tup{\unit[10]{\stdcurr},\unit[1]{kg}}$ is better than a model~$\tup{\unit[11]{\stdcurr},\unit[1.1]{kg}}$.
  Also, there may be models which are incomparable, which form an antichain.
  For example, you cannot say whether~$\tup{\unit[10]{\stdcurr},\unit[1]{kg}}$ is better than~$\tup{\unit[5]{\stdcurr},\unit[2]{kg}}$. The incomparable models form an antichain.
\end{example}


\section{Upper and lower sets}

\begin{definition}[Upper set]
  \label{def:upperset}
  An \emph{upper set} is a subset~$S$ of a poset~$\posA$ such
  that, if an element is inside, all elements above it are inside as well.
  In formulas:
  \begin{equation}
    \text{$U$ is an upperset of $\posA$} \equiv \forall x\in S, \forall y\in \posA \colon x\posAleq y \Imp y\in S.
  \end{equation}
\end{definition}
  We call~$\uppersets \posA$ the set of upper sets of~$\posA$.

\begin{definition}[Lower set]
  \label{def:lowerset}
  A \emph{lower set} is a subset~$S$ of a poset~$\posA$ if, if a point is inside, all points below it are inside as well.
  In formulas:
  \begin{equation}
    \text{$S$ is a lower set of $\posA$} \equiv \forall x\in S, \forall y\in \posA \colon y\posAleq x \Imp y\in S.
  \end{equation}
\end{definition}
  We call~$\lowersets \posA$ the set of lower sets of~$\posA$.
%
%\begin{remark}
%  Note that if~$A$ is an antichain of a poset~$\posA$, then the set
%  \begin{equation}
%    I(A)=\{x\colon x\posAleq y, y\in A\}
%  \end{equation}
%  is a lower set of~$\posA$.
%\end{remark}

Consider the blue poset of pizza recipes from before. The upper and lower sets of this poset can be represented as in~\cref{fig:upperset}. The upper set can be interpreted as all the potential pizza recipes for which we can find better alternatives in the poset. Similarly, the lower set can be interpreted as all the potential pizza recipes which would be better than the ones in the poset.

\todographics{this picture is not about showing the upper and lower closure. That is later. What we need
to show here is different types of upper/lower sets. For example: a union of closed rectangles, like below and a union of open rectangle. Better yet, something which has a continuous curve for boundary }
\begin{figure}[h!]
  \begin{center}
    \includesag{70_upper_lower_set}
  \end{center}
  \caption{Example of upper and lower sets of a poset of pizza recipes.\todographics{fit in one line}}
  \label{fig:upperset}
\end{figure}

%
%\begin{example}[Upper and lower sets in~\Bool]
%  The booleans~$\{\false, \true \}$ form a poset with~$\false \leq \true\colon(\Bool,\posleq)$ . The subset~$\{\false\} \subseteq \Bool$ is not an upper set, since~$\false \leq \true$ and~$\true \notin \{\false \}$.
%\end{example}



