% !TEX root = standalone.tex


\section{From antichains to uppersets, and viceversa}
\begin{definition}[Upper closure operator]
  \label{def:upperclosure}
  The \emph{upper closure operator} $\upit $ maps a subset to the smallest upper set that includes it:
  \begin{equation}
    \begin{aligned}
      \upit  \colon \powerset\posA&\to \uppersets \posA \\
      S&\mapsto \{y\in \posA \mid \exists x\in S \colon x\posAleq y\}.
    \end{aligned}
  \end{equation}
\end{definition}
\begin{remark}
  Note that, by definition, an upper set is closed to upper closure.
\end{remark}
\begin{remark}
  For any~$S\in \powerset\posA$,~$\upit  S$ is in fact an upper set.
  \begin{proof}
    Suppose~$y\in \upit  S$ and~$z\in \posA$, and suppose $y\posAleq z$. By definition~$\exists x$ s.t.~$x\posAleq y$, meaning that~$x\posAleq z$. Thus,~$z\in \upit  S$, as was to be shown.
  \end{proof}
\end{remark}

\begin{lemma}
  The upper closure operator~$\upit $ is a monotone map.
\end{lemma}
\begin{proof}
  Consider the posets~$\tup{\powerset\posA,\subseteq}$ and~$\tup{\uppersets \posA ,\supseteq}$, and~$S_1,S_2\in \powerset\posA$. It is clear that given~$S_1\subseteq S_2$, one has
  \begin{equation*}
    \{y\in \posA\mid \exists x\in S_1\colon x\posAleq y\} \supseteq \{y\in \posA\mid \exists x\in S_2\colon x\posAleq y\}.
  \end{equation*}
  Therefore,~$\upit  S_1\supseteq \ \upit  S_2$, satisfing the monotonicity property for~$\upit $.
\end{proof}

\begin{lemma}
  \label{up-cl-inj-antichains}
  Let $A$ and $B$ be subsets of $P$ that are antichains. Then
  \begin{equation*}
    \upit  A = \ \upit  B \quad \Rightarrow \quad A = B.
  \end{equation*}
\end{lemma}

\begin{proof}
  First, let's fix an $a \in A$. From $\upit  A = \ \upit  B$ we know that in particular $A \subseteq \ \upit  B$. This means that for our fixed $a \in A$ there exists $b \in B$ such that $b \leq a$. From $\upit  A = \ \upit  B$ it also follows that $B \subseteq \ \upit  A$, so to the  $b \in B$ given above, there exists $a' \in A$ such that $a' \leq b$. In total, we have $a' \leq b \leq a$, and since $A$ is an antichain, we must have $a' = a$. This implies that $a' = b = a$. In particular, we have $a \in B$.

  The above shows that $A \subseteq B$. To show $B \subseteq A$, we can fix any $b \in B$ and repeat the above argumentation, now with the roles of $A$ and $B$ exchanged.
\end{proof}

In the example of the pizza recipes, first, consider the upper set of a single element of the poset, e.g.~$p_1=\tup{\unit[1]{\stdcurr},\unit[2]{h}}$  (\cref{fig:upperclosure_1}).
\begin{figure}[h!]
  \begin{center}
    \includesag{70_upper_closure_1}
  \end{center}
  \caption{The upper closure of a singleton set of pizza recipes. }
  \label{fig:upperclosure_1}
\end{figure}
Then, consider the case of two elements, with~$p_2=\tup{\unit[2]{\stdcurr},\unit[1]{h}}$ (\cref{fig:upperclosure_2}).

\begin{figure}[h!]
  \begin{center}
    \includesag{70_upper_closure_2}
  \end{center}
  \caption{The upper closure of a set of pizza recipes. }
  \label{fig:upperclosure_2}
\end{figure}
Note that the upper set of the subset formed by the two elements is the union of the upper sets of the single elements.

\begin{definition}[Lower closure operator]
  The \emph{lower closure operator} $\downit$ maps a subset to the smallest lower set that includes it, i.e.
  \begin{equation*}
    \begin{aligned}
      \downit \colon \powerset\posA&\to \lowersets \posA\\
      S&\mapsto \{ y\in \posA \mid \exists x\in S \colon y\posAleq x\}.
    \end{aligned}
  \end{equation*}
\end{definition}

\begin{lemma}
  The lower closure operator $\downit$ is a monotone map.
\end{lemma}

\JL{The following proof is a bit redundant... we can say ``analogous to the case of the upper closure operation'' and/or invoke the principle of duality for posets.}
\begin{proof}
  Consider the posets~$\tup{\powerset\posA,\subseteq}$ and~$\tup{\lowersets \posA,\subseteq}$, and let~$S_1,S_2\in \powerset\posA$. It is clear that given~$S_1\subseteq S_2$, one has
  \begin{equation}
    \{y\in \posA\mid \exists x\in S_1\colon y\posAleq x\} \subseteq \{y\in \posA\mid \exists x\in S_2\colon y\posAleq x\}.
  \end{equation}
  Therefore,~$\upit S_1\subseteq \ \upit S_2$, satisfing the monotonicity property for~$\downit$.
\end{proof}



\begin{example}
  Consider the battery example of~\cref{ex:battery}, and the antichain given by the battery models~$a=\tup{\unit[10]{\stdcurr},\unit[1]{kg}}$,~$b=\tup{\unit[20]{\stdcurr},\unit[0.5]{kg}}$, and~$c=\tup{\unit[30]{\stdcurr},\unit[0.25]{kg}}$ (\cref{fig:examplebatt}).
  The lower closure uperator~$\downit\{a,b,c\}$ represents all the battery models which, if existing, would dominate~$\{a,b,c\}$.

\end{example}
\begin{figure}[h!]
  \begin{center}
    \includesag{70_battery_1}
  \end{center}
  \caption{Battery example. From the left: antichain, upper closure, and lower closure.
  \label{fig:examplebatt}}
\end{figure}


\begin{definition}[Min]
  \label{def:Min}
  $\Min \colon \powerset\posA \to \antichains\posA$ is the map that sends a subset~$S$ of a poset to the minimal elements of that subset, i.e., those elements~$a \in S$ such that~$a \posAleq b$ for all~$b \in S$. In formulas:
  \begin{equation*}
    \begin{aligned}
      \Min \colon \powerset\posA &\to \antichains\posA\\
      S&\mapsto \{ x\in S\colon (y\in S)\wedge(y\posAleq x)\Rightarrow (x=y)\}.
    \end{aligned}
  \end{equation*}
  Note that~$\Min(S)$ could be empty.
\end{definition}

\begin{definition}[Max]
  \label{def:Max}
  $\Max \colon \powerset\posA \to \antichains\posA$ is the map that sends a subset~$S$ of a poset to the maximal elements of that subset, i.e., those elements~$a \in S$ such that~$a \ordgeq b$ for all~$b \in S$. In formulas:
  \begin{equation*}
    \begin{aligned}
      \Max \colon \powerset\posA &\to \antichains\posA\\
      S&\mapsto \{ x\in S\colon (y\in S)\wedge(y\ordgeq x)\Rightarrow (x=y)\}.
    \end{aligned}
  \end{equation*}
  Note that~$\Max(S)$ could be empty.
\end{definition}

\todo{This is a remnant of older times. To remove. }

\begin{lemma}
  Given a poset~$\tupp{\posA,\posAleq}$,~$\tupp{\antichains\posA,\ordleq_{\antichains\posA}}$ is a poset with
  \begin{equation}
    \label{eq:orderantichain}
    A\ordleq_{\antichains\posA} B \text{ if and only if } \upit  A \supseteq \ \upit  B.
  \end{equation}
  Furthermore, it is bounded by the top~$\top_{\antichains\posA}=\emptyset$ and the bottom~$\bot_{\antichains\posA}=\{\bot_{\posA}\}$.
\end{lemma}

\begin{proof}
  We need to show the poset properties (\cref{def:poset}).
  We can prove the following:

  \

  \begin{compactitem}
    \item \emph{Reflexivity}: From~$\tupp{\posA,\posAleq}$ being a poset we know that
    \begin{equation}
      \begin{aligned}
        \{y\in \posA \mid \exists x\in A \colon x\posAleq y\} &\supseteq \{y\in \posA \mid \exists x\in A \colon x\posAleq y\},\\
        \upit  A =\ \upit  A
      \end{aligned}
    \end{equation}
    and hence~$A\ordleq_{\antichains\posA}A$.

    \

    \item \emph{Antisymmetry}: One has
    \begin{equation}
      \begin{aligned}
        \left(A\ordleq_{\antichains\posA} B\right) \wedge \left(B\ordleq_{\antichains\posA} A\right)
        &\Leftrightarrow \left(\upit  A \supseteq \ \upit  \ B\right) \wedge \left( \upit   B\supseteq \ \upit  \ A\right)\\
        &\Leftrightarrow \ \upit  A= \ \upit  B \\
        & \Rightarrow A = B.
      \end{aligned}
    \end{equation}
    The last implication is by  \cref{lem:up-cl-inj-antichains}.

    \


    \item \emph{Transitivity}: One has
    \begin{equation}
      \begin{aligned}
        \left(A\ordleq_{\antichains\posA} B\right) \wedge \left(B\ordleq_{\antichains\posA} C\right)&\Leftrightarrow  \left(\upit  A \supseteq \ \upit  \ B\right) \wedge \left( \upit   B\supseteq \ \upit  C\right)\\
        &\Imp \ \upit  A\supseteq \ \upit  C\\
        &\Imp A\ordleq_{\antichains\posA}C.
      \end{aligned}
    \end{equation}
    In order to find the top, we need to find the smallest set~$\top_{\antichains\posA}$ such that~$A\ordleq_{\antichains\posA} \top_{\antichains\posA}$ for all~$A\in \antichains\posA$. In other words, such that~$\upit  A\supseteq \ \upit  \top_{\antichains\posA}$ for all~$A\in \antichains\posA$. This is clearly~$\emptyset$, since~$\upit  \emptyset = \emptyset$. Similarly, in order to find the bottom, we need to find the set~$\bot_{\antichains\posA}$ such that~$\bot_{\antichains\posA} \ordleq_{\antichains\posA} A$ for all~$A\in \antichains\posA$. In other words, such that~$\upit  \bot_{\antichains\posA} \supseteq \ \upit  A$ for all~$A\in \antichains\posA$. We obtain a bottom if we set $\bot_{\antichains\posA} := \top_{\posA}$, since $\top_{\posA} \supseteq A$ for all $A \subseteq P$, and hence, by monotonicity of $\upit $, we have in particular $\upit  \top_{\posA} \supseteq \ \upit  A$ for all antichains $A$.
  \end{compactitem}
\end{proof}



\begin{definition}[Downward closed set]
  \label{def:downward-closed-upperset}
  An upper set~$S$ is \emph{downward-closed} in a poset~$\posA$ if
  \begin{equation}
    S =\, \upit  \Min S.
  \end{equation}
\end{definition}

\begin{remark}
  The set of downward-closed upper sets of~$\posA$ is denoted~$\dcuppersets \posA$.
\end{remark}

\subsection{Measuring posets}
\begin{definition}[Width and height of a poset]
  \label{def:poset-width-height} $\posetwidth(\posA)$ is the maximum
  cardinality of an antichain in~$\posA$ and $\posetheight(\posA)$
  is the maximum cardinality of a chain in~$\posA$.
\end{definition}

\todo{examples}
