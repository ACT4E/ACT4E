% !TEX root = standalone.tex


\section{Opposite of a poset}\label{sec:opposite-of-a-poset}

\begin{definition}
  \label{def:poset-opposite}
  The \emph{opposite} of a poset~$\tup{A, \ordleq} $ is the poset denoted as~$\tup{A\op, \ordleq\op}$ that has the same elements as~$A$ and the reverse ordering (\cref{fig:opposite}).
  For a given~$x \in A$, we use~$x^*$ to represent its corresponding copy in~$A\op$; note that~$x$ and~$x^*$ belong to distinct posets.
  Reversing the order means that, for all $x,y\in A$,
  \begin{equation}
    x \ordleq y \quad \Leftrightarrow \quad y^* \ordleq\op x^*.
  \end{equation}
\end{definition}

\begin{figure}[tbh]
  \centering
  \includesag{40_dpcatfig_opposite}
  \caption{Opposite of a poset.\label{fig:opposite}}
\end{figure}


\begin{example}[Credit and debt]
  Let us define the set
  \begin{equation*}
    \posA=\reals \times \{\stdcurr\}=\{0.00,0.01,0.02,\dots\}
  \end{equation*}
  of all \stdcurr \ monetary quantities approximated to the cent.
  From this set we can define two posets:~$\posA^{+} = \tup{\posA, \ordleq}$ and~$\posA^{-} = \tup{\posA, \ordgeq}$, that are the opposite of each other.
  If the context is that, given two quantities~\unit[1]{\stdcurr} and \unit[2]{\stdcurr}, we prefer \unit[1]{\stdcurr} to \unit[2]{\stdcurr} (for example because it is a cost to pay to acquire a component), then we are working in~$\posA^{+}$, otherwise we are working in~$\posA^{-}$ (for example because it represents the price at which we are selling our product).
  Traditionally, in double-entry ledger systems, the numbers were not written with negative signs, but rather in color: red and black. From this convention we get the idioms ``being in the black'' and ``being in the red''.
\end{example}
