% !TEX root = standalone.tex

\subsection{Disjoint union of posets}

Similarly to what we have done for sets in \cref{sec:coproductset}, we can think of alternatives in the poset case through their disjoint union.

\begin{definition}[Disjoint union of posets]
  Given posets~$\tup{\Obja, \ordleq_\Obja}$ and~$\tup{\Objb, \ordleq_\Objb}$, we can define their \emph{disjoint union}~$\tup{\Obja + \Objb, \ordleq_{\Obja + \Objb}}$, where~$\Obja + \Objb$
  is the disjoint union of the sets~$\Obja$ and~$\Objb$ (\cref{def:disjoint-union}), and the
  order~$\ordleq_{\Obja + \Objb}$ is given by:
  \begin{equation}
    \obja \ordleq_{\Obja + \Obja} \objb \quad\equiv\quad
    \begin{cases}
      \obja \ordleq_A \objb, & \obja,\objb \in \Obja, \\
      \obja \ordleq_B \objb, & \obja,\objb \in \Objb.
      %\false,  & \text{otherwise}.
    \end{cases}
  \end{equation}
  \begin{equation}
    \begin{aligned}
      \ordleq_{\Obja+\Objb}\colon (\Obja+\Objb)\times (\Obja+\Objb)&\to \Bool\\
      \tup{1,\obja_1},\tup{1,\obja_2}&\mapsto (\obja_1\ordleq_{\Obja} \obja_2)\\
      \tup{2,\objb},\tup{1,\obja}&\mapsto \bot\\
      \tup{1,\obja},\tup{2,\objb}&\mapsto \bot\\
      \tup{2,\objb_1},\tup{2,\objb_2}&\mapsto (\objb_1\ordleq_\Objb \objb_2).
    \end{aligned}
  \end{equation}
\end{definition}


\begin{example}
  Consider the posets~$\Obja=\tup{\diamond, \star}$ with~$\diamond \ordleq_\Obja \star$, and~$\Objb=\tup{\dagger,\ast}$, with~$\ast \ordleq_\Objb \dagger$. Their disjoint union can be represented as in \cref{fig:poset-coproduct}.

  \begin{figure}[h!]
    \centering
    \includesag{40_disjoint_union}
    \caption{Disjoint union of posets. \label{fig:poset-coproduct}}
  \end{figure}
\end{example}
