% !TEX root = standalone.tex


\section{Compositionality of monotonicity}
Note that monotonicity is a compositional property.
\begin{lemma}
  Given posets~$A, B, C$ and monotone maps~$f\colon A \to B$ and~$g\colon B \to C$, the composite map~$f\then g\colon  A \to C$ is
  monotone as well.
\end{lemma}
\begin{proof}
  Consider~$a_1,a_2 \in A$,~$b_1,b_2\in B$. We have, by definition,
  \begin{equation}
    \begin{aligned}
      a_1\ordleq_A a_2 &\Imp f(a_1)\ordleq_B f(a_2)\\
      b_1\ordleq_B b_2 &\Imp g(b_1)\ordleq_C g(b_2).
    \end{aligned}
  \end{equation}
  By substituting the above in the map composition formula, one has
  \begin{equation}
    a_1\ordleq_A a_2 \Imp (f\then g)(a_1) \ordleq_C (f\then g)(a_2),
  \end{equation}
  which is the monotonicity condition for~$(f\then g)$.
\end{proof}

