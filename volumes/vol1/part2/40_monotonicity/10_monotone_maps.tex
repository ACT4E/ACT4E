\section{Monotone maps}\label{sec:monotonicity-monotone-maps}
\AC{Add some motivation from design. E.g. generalization of "non-decreasing" and "non-increasing" from real numbers to partial orders}

\begin{definition}[Monotone map]
  \label{def:monotone}
  A \emph{\iindex{monotone map}} between two posets~$\tup{A, \ordleq_A}$ and~$\tup{B, \ordleq_B}$ is a map that preserves the ordering, in the sense that
  \begin{equation*}
    a \ordleq_A b \quad \Imp \quad f(a) \ordleq_B f(b).
  \end{equation*}
\end{definition}
\begin{remark}
  Given a poset~$A$, the map~$\id_A$ is monotone, since for~$a_1,a_2\in A$, one has:
  \begin{equation*}
    \begin{aligned}
      a_1\ordleq_A a_2 \Imp a_1&=a_1\then \id_A\\
      &\ordleq_A a_2\then \id_A \\
      &= a_2.
    \end{aligned}
  \end{equation*}
\end{remark}

\begin{definition}[Antitone map]
  \label{def:antitone}
  An \emph{\iindex{antitone map}} between two posets~$\tup{A, \ordleq_A}$ and~$\tup{B, \ordleq_B}$ is a map that reverses the ordering, in the sense that
  \begin{equation*}
    a \ordleq_A b \quad \Imp \quad f(a) \ordgeq_B f(b).
  \end{equation*}
\end{definition}


\begin{comment}
  A monotone map is an \emph{order isomorphism} if the other direction
  of the implication holds as well:
  \begin{equation}
    a \leq_A b \quad \Leftrightarrow \quad f(a) \leq_B f(b).
  \end{equation}
\end{comment}

\begin{example}[Unit cost, total cost]
  Assume that you want to produce some widgets, and that the manifacturing cost depends on the number of widgets.
  The function describing the total cost~$t\colon \natnumbers\to \nonNegReals$ is a map between the ordered sets~$\natnumbers$ and~$\nonNegReals$, and maps each quantity of widgets to a total manufacturing cost (\cref{fig:total_manufacturing}).
  Clearly, $t$ is a monotone function.
  Conversely, the unit cost function~$u\colon \natnumbers\to \nonNegReals$ is antitone (\cref{fig:unit_manufacturing}).
\end{example}

\begin{figure}[h!]
  \caption{\todo{add picture}}
  \label{fig:unit_manufacturing}
\end{figure}

\begin{figure}[h!]
  \caption{\todo{add picture}}
  \label{fig:total_manufacturing}
\end{figure}





\begin{example}[Rounding functions]
  \label{ex:rounding-functions}
  In this example we look at three rounding functions: \funceil, \funfloor, and \rtntte. Both the maps
  \begin{equation*}
    \begin{aligned}
      \funceil\colon \tup{\reals,\leq}&\to \tup{\natnumbers,\leq}\\
      x&\mapsto i \in \natnumbers \colon i-1<x\leq i,
    \end{aligned}
  \end{equation*}
  and
  \begin{equation*}
    \begin{aligned}
      \funfloor\colon \tup{\reals,\leq}&\to \tup{\natnumbers,\leq}\\
      x&\mapsto i \in \natnumbers \colon i\leq x< i+1.
    \end{aligned}
  \end{equation*}
  are monotone, since~$x\leq y$ implies both~$\funceil(x)\leq \funceil(y)$ and~$\funfloor(x)\leq\funfloor(y)$.
  \todo{Define IEEE754 formally}
  \todo{Add picture with 3}
\end{example}

\begin{example}[Cardinality map]
  In \cref{ex:hasseinclusion} we presented the poset arising from the power set of a set~$A=\{a,b,c\}$ and ordered via subset inclusion.

  The map~$\vert \cdot \vert \colon \powerset A \to \natnumbers$ (cardinality), is a monotone map (\cref{fig:cardinality}).
  \begin{figure}[h!]
    \begin{center}
      \includesag{40_dpcatfig_exmonotone}
    \end{center}
    \caption{The cardinality map is a monotone map. \label{fig:cardinality}}
  \end{figure}
\end{example}

\begin{lemma}
  Consider a discrete poset~$A$ and a poset~$B$. Any map~$f\colon A\to B$ is monotone.
\end{lemma}
\begin{proof}
  Since~$A$ is a discrete poset, one has
  \begin{equation}
    a_1\ordleq_A a_2 \iff a_1=a_2.
  \end{equation}
  Therefore, one has
  \begin{equation}
    \begin{aligned}
      a_1\ordleq_A a_2 &\Imp a_1=a_2\\
      &\Imp f(a_1)=f(a_2)\\
      &\Imp f(a_1)\ordleq_B f(a_2).
    \end{aligned}
  \end{equation}
\end{proof}
Unless indicated otherwise, in this paper all maps between posets are assumed to be monotone or will turn out to be monotone. In a similar way, one can define antitone maps.


\begin{example}
  We now look at an example of \textbf{set-based filtering}, where filtering refers to online inference.
  Suppose that we want to track the value of a quantity~$x\in [0,100]$, without having a priory information about~$x$.
  We are equipped with sensors, which periodically measure the quantity~$x$ with some variable precision, i.e., at time~$t\in \nonNegReals $ they produce an \emph{observation}~$y_t\colon x_t\in [l_t,u_t]$.
  Also, note that the quantity fluctuates randomly, and we bound its ``velocity'' to be~$\dot{x}_t\in [-1,1]$ (except at boundaries).
  At the beginning, our information state~$\bar{i}_0$ could be that~$x\in [0,100]$.
  At time 0, we get an observation~$y_0$, that says~$x\in [21,24]$.
  The new information state can be obtained by ``fusing'' the two inputs we have received about~$x$. This corresponds to the intersection
  \begin{equation*}
    x\in \left( [0,100] \cap [21,24]\right)\equiv x\in [21,24].
  \end{equation*}
  Let's now say we get an observation~$y_1$ which says~$x\in [19,22]$.
  We now need to take into account the evolution/dynamics of the quantity we are tracking. From the interval~$[21,24]$ we know that the variable could have evolved in~$[20,25]$ (dynamics are bounded with a unit increase/decrease). Therefore, the new information state is given by:
  \begin{equation*}
    x\in \left( [20,25] \cap [21,24]\right)\equiv x\in [21,24].
  \end{equation*}
  One of the structures which could sustain this kind of inference, is the of \emph{posets of intervals} (\cref{def:poset_intervals}).
  The Hasse diagram representing a situation related to this diagram could be as reported in \cref{fig:hasse_filtering}.
  \begin{figure}[h!]
    \begin{center}
      \includesag{080_hasse_filtering}
    \end{center}
    \caption{\label{fig:hasse_filtering}}
  \end{figure}
\end{example}
