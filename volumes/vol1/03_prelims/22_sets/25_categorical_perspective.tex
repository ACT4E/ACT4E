% !TEX root = chapter-standalone.tex
\section{The categorical perspective}
\label{sec:categorical perspective}

\todotext{JL: @JL: expand this section more; introduce diagrams}


A central theme in this book is to study how various mathematical structures compose.
For instance, what are general patterns of how functions and sets \emph{relate to one another} via composition?

In time, we will see that many features that sets and functions exhibit can be broadly generalized to other kinds of mathematical entities.

\subsection{Structuralism}

One guiding philosophy for creating and understanding such generalizations is to formulate properties of functions in a way that only uses their ``external compositional aspects'' and does not rely on the fact that we are dealing with sets, which have elements, and so we can ``look inside them''. We call this philosophy ``structuralism" because

\todotext{finish explaining structuralism}

This likely sounds very vague at the moment.
Let us illustrate with some examples.

\begin{example}
    Consider the property that a function may (or may not) have of being \SY{bijective}.
    According to \cref{ex:bijective-functions-are-isomorphisms}, a function~$\mapa\colon \setA \sto \setB$ is \SY{bijective} if and only if it is an isomorphism.
    The latter means, by definition, that there exists~$\mapb \colon \setB \sto \setA$ such that~$\mapa \mthen \mapb = \catidat\setA$ and~$\mapb \mthen \mapa = \catidat\setB$.
    The point is that the equations in the definition of ``isomorphism'' only make use of the operation of function composition, the notion of quality, and the existence of special identity functions.
    There is no mention of elements of sets, as there is in the definition of ``bijective''.
\end{example}

\begin{example}
    The notion of ``subset'' is traditionally defined, as we did above, by saying that~$\setA \setsubseteq \setB$ if and only if~$\forall \ \ela \setin\setA$:
    \begin{equation}
        \prfperiod{
            \ela \setin\setA
        }{
            \ela \setin\setB
        }
    \end{equation}

    There are, however, alternatives that do not refer to ``elements''.
    To see one way, consider the set~$\setTwo = \makeset{ 0, 1}$.
    Any function~$\mapa \colon \setB \to \setTwo$ defines a subset
    \begin{equation}
        \setA_\mapa = \makeset{ \elb \setin\setB \mid \mapa(\elb) = 1 } \setsubseteq \setB.
    \end{equation}
    Conversely, any subset~$\setA \setsubseteq \setB$ defines a function~$\mapa_\setA \colon \setB \sto \setTwo$ by setting
    \begin{equation}
        \mapa_\setA(\elb) = \begin{cases}
            1 & \text{ if }\elb\setin\setA, \\
            0 & \text{ elsewhere}.
        \end{cases}
    \end{equation}
    It can be checked that this defines a 1-to-1 correspondence between functions~$\setB \sto \setTwo$ and subsets of~\setB.
    In other words, there is a \SY{bijection} between the set~$\setTwo^\setB$ and the set~$\powerset \setB$.
    So, instead of using a definition of subset that involves elements of sets, we could work with functions~$\setB \sto \setTwo$.
\end{example}

\begin{gradedexercise}[\exname{SubsetsAsFunctions}]
    \label{ex:SubsetsAsFunctions}

    Let \setB be any set.
    Prove that $\setTwo^\setB \isomorphic \powerset \setB$.
\end{gradedexercise}

\solutionof{SubsetsAsFunctions}

\begin{example}
    Even the notions of ``element of a set'' and ``evaluation of a function at an element" can be described purely in terms of functions and their composition, without needing to ``look inside'' of the sets involved.

    To show how this works, let us first define $\singleton \definedas \makeset{\singletonel}$, a singleton set whose only element is the symbol ``$\singletonel$'' (any singleton set would do; for concreteness we are fixing one and calling it $\singleton$).

    Now we are ready to make an interesting observation: functions~$\singleton \sto \setA$ are in 1-to-1 correspondence with the elements of~\setA.
    A function~$\mapa\colon \singleton \sto \setA$ will have to map ``$\singletonel$'' to some element~$\mapa(\singletonel) \setin\setA$, and since~$\singleton$ has no other elements, that is all that~$\mapa$ does.
    So~$\mapa$ ``picks out'' an element of~\setA.
    We can work with functions~$\singleton \sto \setA$ in place of elements of~\setA.

    Next, let's talk about function evaluation.
    Consider a function~$\mapb \colon \setA \sto \setB$.
    Given an element~$\ela \setin\setA$, this element will be mapped by~$\mapb$ to an element~$\mapb(\ela) \setin\setB$.
    If we use, instead of~$\ela \setin\setA$, the function~$\mapa \colon \singleton \sto \setA$ to which it corresponds, then the element~$\mapb(\ela) \setin\setB$ corresponds to the function~$\mapa \mthen \mapb\colon \singleton \sto \setB$.
    In other words, we can talk about evaluation of a function~$\mapb \colon \setA \sto \setB$ ``at an element of~\setA'' without actually using elements, but rather just using functions and function composition.
\end{example}


\subsection{Diagrams}

Another typical characteristic of ``category theory culture'' is to often use various kinds of diagrams.

\todotext{explain that we willl meet different kinds of diagrams, and explain what kind will be informally introduced here (and say that we'll provide a more formal treatment later..)} 
