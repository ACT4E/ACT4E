\section{Subsets}
Now consider the set~$\makeset{ \sapple, \scarrot, \sgrapes }$, and compare it with~$\makeset{ \sapple, \scarrot, \sgrapes, \stea }$.
Each of the elements of the first set is also an element of the second set;
in such a case we say that the first set is \emph{included} in the second set.
In symbols,
%
\begin{equation}
    \label{eq:subset}
    \makeset{ \sapple, \scarrot, \sgrapes } \setsubseteq \makeset{ \sapple, \scarrot, \sgrapes, \stea }.
\end{equation}
%
Generally, given sets~$\setA$ and~$\setB$, the statement~$\setA \setsubseteq \setB$ is logically equivalent to the statement
%
\begin{equation*}
    \prfperiod{
        \ela \setin \setA
    }{
        \ela \setin \setB
    }
\end{equation*}

Returning to \cref{eq:subset}, the second set is, on the other hand, \emph{not} included in the first set, since~$\stea$ is an element of the second set, but not the first.
If we say a set is ``strictly included'' in another, then we mean ``included in and not equal'';
if the adjective ``strictly'' is not used, then inclusion means for us that equality is also possible.
\todotextjira{499}{@J: seems like a good place to introduce also~$\subset$ - also compare with $<$, $\leq$.}
In other words, for us, it is true that any set~$\setA$ is included in itself:~$\setA \setsubseteq \setA$.
Given sets~$\setA$ and~$\setB$, if it is the case that~$\setA \setsubseteq \setB$, then we say that~$\setA$ is a \emph{subset} of~$\setB$.
Inclusion and equality are related as follows: given sets~$\setA$ and~$\setB$,
%
\begin{equation*}
    \prfdoubleperiod{
        \setA = \setB
    }{
        {\setA \setsubseteq \setB}
        \quad
        {\setB \setsubseteq \setA}
    }
\end{equation*}
%
Equivalently, two sets are equal if they have the same elements:
\begin{equation*}
    \prfdoubleperiod{
        \quad \setA = \setB \quad
    }{
        \prfdouble{
            \ela \setin \setA
        }{
            \ela \setin \setB
        }
    }
\end{equation*}
