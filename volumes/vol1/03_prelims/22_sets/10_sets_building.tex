% !TEX root = chapter-standalone.tex

\section{Building new sets from old}


\todotext{@J: write small intro paragraph or sentence}

\subsection{Union and intersection}

Given sets $\setA$ and $\setB$, their \emph{union} is a new set, denoted $\setA \cup \setB$, characterized by 
\begin{equation*}
\ela \in \setA \cup \setB \quad \Leftrightarrow \quad \ela \in \setA \ \text{or} \ \ela \in \setB. 
\end{equation*}

For example, if $\setA = \{ \sapple, \sbanana \}$ and $\setB = \{ \stea \}$, then 
\begin{equation*}
\setA \cup \setB = \{ \sapple, \sbanana, \stea \}. 
\end{equation*}

Or, if $\setA = \{ \sapple, \sbanana, \scarrot \}$ and $\setB = \{ \sapple, \scarrot, \swater \}$, then
\begin{equation*}
\setA \cup \setB = \{ \sapple, \scarrot \}. 
\end{equation*}

Similarly, the \emph{intersection} of sets $\setA$ and $\setB$, denoted $\setA \cap \setB$, is the set characterized by 
\begin{equation*}
\ela \in \setA \cap \setB \quad \Leftrightarrow \quad \ela \in \setA \ \text{and} \ \ela \in \setB. 
\end{equation*}

So, if $\setA = \{ \sapple, \sbanana, \scarrot, \stea \}$ and $\setB = \{ \stea, \sbanana, \swater \}$, then
\begin{equation*}
\setA \cap \setB = \{ \stea, \sbanana \}. 
\end{equation*}

\subsection{Power set}
\label{sec:power-set}

\begin{ctdefinition}[Power set]
    \label{def:power-set}
    Given a set~$\setA$, we can form a new set whose elements are the subsets of~$\setA$.
    This new set is called the \iindex{\emph{powerset}} of~$\setA$, and we denote it by~$\powerset \setA$.
\end{ctdefinition}

For example, if~$\setA = \{ \sapple, \scarrot, \sbanana \}$, then its powerset is
\begin{equation*}
    \powerset \setA = \{ \emptyset, \{ \sapple \}, \{ \scarrot \}, \{ \sbanana\}, \{ \sapple, \scarrot \}, \{ \scarrot, \sbanana \}, \{ \sapple, \sbanana\}, \{ \sapple, \scarrot, \sbanana \} \}.
\end{equation*}
The powerset is thus a set of sets.

\todotext{@J: do we want to say more here about the power set? is there enough material to justify its own section? or incorporate in sets section above? Or a section on ``new sets from old'' ? Yes, perhaps define operations of complement or difference of sets, and also perhaps comment that intersection and union again land in the powerset.}

\

\todotext{@J: the paragraph below is a bit of an orphan - where to put and how to justify / motivate? I think this was taken from the book ``Naive Set Theory'', by Halmos. Maybe move powerset subsection to be the first in the section... and move arbitrary unions/intersections part to the union and intersection subsection.}

We will use an assumption (or ``axiom'') that will in fact allow us to take ``arbitrary'' unions and intersections.
Namely, given any two sets~$\setA$ and~$\setB$, we assume that there always exists some other set~$\setC$ such that~$\setA \subseteq \setC$ and~$\setB \subseteq \setC$.
In other words, there exists~$\setC$ such that~$\setA \in \powerset \setC$ and~$\setB \in \powerset \setC$.
And more generally, suppose that we are given not just two sets, but a whole collection~$\mathcal{S}$ of sets, in other words, a set~$\mathcal{S}$ of sets (we are using the word ``collection'' interchangeably with the word ``set'').
We will assume there exists a set~$\setC$ such that~$\setA \in \setC$ for all~$\setA \in \mathcal{S}$.



Now we can take the union and intersection of the collection~$\mathcal{S}$ of sets:
%
\begin{equation*}
    \bigcup \mathcal{S} \coloneqq \{ \ela \in \setC \mid \exists \  \setA \in \setC \colon \ela \in \setA \},
\end{equation*}
%
and
%
\begin{equation*}
    \bigcap \mathcal{S} \coloneqq \{ \ela \in \setC \mid \forall \  \setA \in \setC \colon \ela \in \setA \}.
\end{equation*}


\subsection{Tuples}

The notion of a ``tuple'' is is essentially identical to that of a list: a finite sequence or listing of ``things'', with repetitions allowed. 

Usually tuples are referred to in a way that specifies a specific length: tuples of length one are called 1-tuples; tuples of length two are usually called \emph{ordered pairs} or 2-tuples; tuples of length three are called \emph{triples} or $3$-tuples; tuples of length four are calle \emph{quadruples} or 4-tuples; and so on. In general, given $n \in \natnumbers$, a tuple of length $n$ is called an $n$-tuple. 

The notation to indicate a tuple is via angled brackets: for instance, $\tup{\ela, \elb, \elc}$ denotes a 3-tuple, or triple. The items inside the brackets will be called entries or components. 

So what is the difference between lists and tuples? For us, it is essentially context and usage. 

Typically, we will use tuples in situations where we also specify, for each entry of the tuple, a set for which that entry is an element. For example, if $\setA = \{ \sapple, \scarrot, \sbanana \}$ and $\setB = \{ \stea, \swater \}$, then we will sometimes want to specify that $\tup{\sapple, \swater}$ is a 2-tuple where $\sapple \in \setA$ and $\swater \in \setB$, for instance. Or, as another example, we might consider $\tup{3, 7, 8}$ as a triple of natural numbers: here we are specifying that we are thinking of $3$, $7$, and $8$ as elements of the set $\natnumbers$. 

\subsection{Cartesian product}\label{sec:cartesian-product}

Given sets $\setA$ and $\setB$, their \emph{cartesian product} is a new set -- denoted $\setA \times \setB$ -- who's elements are precisely all possible 2-tuples $\tup{\ela, \elb}$ such that the first entry $\ela$ is an element of $\setA$ and the second entry $\elb$ is an element of $\setB$. 

For example, if~$\setA = \{ \sapple, \sbanana, \scarrot\}$ and~$\setB = \{ \stea, \swater \}$, then
\begin{equation*}
    \setA \cartprod \setB = \{ \tup{\sapple, \stea}, \tup{\sapple, \swater}, \tup{\sbanana, \stea}, \tup{\sbanana, \swater},  \tup{\scarrot, \stea}, \tup{\scarrot, \swater}\}.
\end{equation*}
In the special case where~$\setA = \emptyset$ or~$\setB = \emptyset$, then~$\setA \cartprod \setB = \emptyset$.


\begin{remark}
For finite sets $\setA$ and $\setA$, the size of $\setA \times \setB$ is the product (multiplication) of the sizes of $\setA$ and $\setB$. This is one hint of why we think of $\setA \times \setB$ as a kind of multiplication of sets. 
\end{remark}

In set theory, 2-tuples (ordered pairs) are usually defined formally by setting
\begin{equation}
    \label{eq:ordered-pair}
    \tup{\ela, \elb} \coloneqq \{ \{\ela \} , \{ \ela, \elb \} \},
\end{equation}
which allows to give the more formal definition of the cartesian product as
            \begin{equation}
                \setA \times \setB = \{ \elc \in \powerset \powerset (\setA \setunion \setB) \mid \exists \ela \in \setA, \exists \elb \in \setB : \elc = \tup{\ela, \elb} \}.
            \end{equation}

We will, however, simply treat ordered pairs as a primitive given construction (i.e. without reference to formal set theory), and hence we also treat the construction of `cartesian product of sets' as primitive. 

Similarly, we will take the notions of $n$-tuple, for any $n \in \natnumbers$ as primitive, and with it also the notion of $n$-fold cartesian product of sets. That is, if $\setA_1, \setA_2, ... \setA_n$ are sets, then we take 
\begin{equation}
\setA_1 \times \setA_2 \times \dots \times \setA_n
\end{equation}
to be the set of $n$-tuples $\tup{\ela_1, ...., \ela_n}$ with $\ela_i \in \setA_i$ for all $i = 1,2,...,n$. 

\subsection{Disjoint union}

Given sets~$\setA$ and~$\setB$, their disjoint union is the set
\begin{equation*}
    \setA \setdisunion \setB := (\setA \times \{ 1 \}) \cup (\setB \times \{2\}).
\end{equation*}
In other words, an element of  $\setA \setdisunion \setB$ is either a tuple $\tup{\ela, 1}$ for some  $\ela \in \setA$ or a tuple $\tup{\elb, 2}$ for some $\elb \in \setB$. The sets $\{ 1 \}$ and $\{2\}$ here simply provide labels which ``force'' the sets $\setA \times \{ 1 \}$ and $\setB \times \{2\}$ to be disjoint (even if $\setA$ and $\setB$ have elements in common). 

As a simple example, let $\setA = \{ \sapple, \sbanana, \scarrot\}$ and~$\setB = \{ \stea, \swater \}$. Then
\begin{equation*}
    \setA \setdisunion \setB = \{ \tup{\sapple, 1}, \tup{\sbanana, 1}, \tup{\scarrot, 1}, \tup{\stea, 2},  \tup{\swater, 2}\}.
\end{equation*}

A slightly more tricky example is when we consider $\setA \setdisunion \setA$, say. By definition, this is $(\setA \times \{ 1 \}) \cup (\setA \times \{2\})$, so if $\setA = \{ \sapple, \sbanana, \scarrot\}$, then 
\begin{equation*}
\setA \setdisunion \setA = \{ \tup{\sapple, 1}, \tup{\sbanana, 1}, \tup{\scarrot, 1}, \tup{\sapple, 2}, \tup{\sbanana, 2}, \tup{\scarrot, 2} \}.
\end{equation*}


Let us also think about what happens when the empty set is at play. For example, if~$\setA = \emptyset$, then~$\emptyset \setdisunion \setB = \setB \times \{2\}$, and similarly~$\setA \setdisunion \emptyset =  \{ \tup{\ela,1} \mid \ela \in \setA \}$. Also, $\empty \setdisunion \emptyset = \emptyset$.  

\begin{remark}
If $\setA$ and $\setB$ are finite, then the size of $\setA \setdisunion \setB$ is the sum of the sizes of $\setA$ and $\setB$. This is a hint of why we think of $\setA \setdisunion \setB$ as a form of addition of sets. 
\end{remark}

Later we will see that the cartesian product of sets is a special case of a very general construction in category theory, called the categorical product, and that the disjoint union of sets is a special case of a ``dual'' construction, called coproduct.

\devel{
    \todojira{1}{@J: There is a break in content here; it will be filled/bridged}

    We will name here some special sets that we will find convenient to use later.
    These names are not all standard; however they should be intuitively clear.

    In settings where we'd like to have a specific set of a certain finite size to refer to, we will sometimes use this naming convention:
    \begin{align*}
        [0]
            & \definedas \emptyset \\
        [1] & \definedas \{ 1\} \\
        [2] & \definedas \{ 1, 2\} \\
        [3] & \definedas \{ 1, 2, 3\}
    \end{align*}
    \dots and so on.
    Generally,
    %
    \begin{equation*}
        [n]
        \definedas \{ 1, \dots, n\}
    \end{equation*}
    %
    for any $n \in \natnumbers$.

    \todotextjira{499}{@J: AC: but are we actully going to use these?}
}


