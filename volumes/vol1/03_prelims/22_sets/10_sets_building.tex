% !TEX root = chapter-standalone.tex

\section{New sets from old}

In this section we recall some elementary ways of constructing new sets from old.
% 
The idea of \emph{constructing} new things from old things is one of the main themes of the book.

\subsection{Union and intersection}

Given sets~$\setA$ and~$\setB$, their \emph{union} is a new set, denoted~$\setA \setunion \setB$, characterized by
\begin{equation*}
    \ela \in (\setA \setunion \setB) \quad \Leftrightarrow \quad \ela \in \setA \ \text{or} \ \ela \in \setB.
\end{equation*}

For example, if~$\setA = \{ \sapple, \sgrapes \}$ and~$\setB = \{ \stea \}$, then
\begin{equation*}
    \setA \setunion \setB = \{ \sapple, \sgrapes, \stea \}.
\end{equation*}

Or, if~$\setA = \{ \sapple, \sgrapes, \scarrot \}$ and~$\setB = \{ \sapple, \scarrot, \swater \}$, then
\begin{equation*}
    \setA \setintersection \setB = \{ \sapple, \scarrot \}.
\end{equation*}

Similarly, the \emph{intersection} of sets~$\setA$ and~$\setB$, denoted~$\setA \setintersection \setB$, is the set characterized by
\begin{equation*}
    \ela \in (\setA \setintersection \setB) \quad \Leftrightarrow \quad \ela \in \setA \ \text{and} \ \ela \in \setB.
\end{equation*}

So, if~$\setA = \{ \sapple, \sgrapes, \scarrot, \stea \}$ and~$\setB = \{ \stea, \sgrapes, \swater \}$, then
\begin{equation*}
    \setA \setintersection \setB = \{ \stea, \sgrapes \}.
\end{equation*}

\todojira{607}{POST-ALPHUBEL @J: add exercises about the commutativity and associativity of union and intersection}

\subsection{Power set}
\label{sec:power-set}

Given a set~$\setA$, we can form a new set whose elements are precisely all of the subsets of~$\setA$.
This new set is called the \iindex{\emph{powerset}} of~$\setA$; we denote it by~$\powerset \setA$.

For example, if~$\setA = \{ \sapple, \scarrot, \sgrapes \}$, then its powerset is
\begin{equation*}
    \powerset \setA = \{ \emptyset, \{ \sapple \}, \{ \scarrot \}, \{ \sgrapes\}, \{ \sapple, \scarrot \}, \{ \scarrot, \sgrapes \}, \{ \sapple, \sgrapes\}, \{ \sapple, \scarrot, \sgrapes \} \}.
\end{equation*}

\begin{exercise}
    Can you count how many elements the powerset~$\powerset \setA$ has in the following cases?
    \begin{enumerate}
        \item $\setA = \{ \sapple \} $.
        \item $\setA = \{ \sapple, \scarrot \} $.
        \item $\setA = \{ \sapple, \scarrot, \sgrapes \} $.
        \item $\setA = \emptyset $.
    \end{enumerate}
    Can you guess a general formula for the size of the powerset of a finite set?
\end{exercise}

\begin{solution}
    One has:
    \begin{enumerate}
        \item 2.
        \item 4.
        \item 8.
        \item 1.
    \end{enumerate}
    In general, the size of~$\powerset \setA$ is $2$ to the power of the size of~$\setA$.
\end{solution}

Now suppose we fix a set~$\setA$ for a moment.
Given~$\subA \in \powerset \setA$, the \emph{complement} of~$\subA$ with respect to~$\setA$ is
\begin{equation*}
    \setA \backslash \subA = \{ \ela \in \setA \mid \ela \notin \subA \},
\end{equation*}
which is again an element of~$\powerset \setA$.
When the set~$\setA$ is clear, the notation~$\compl{\subA}$ is sometimes used instead of~$\setA \backslash \subA$.

We also note that the operations of union and intersection, when restricted to~$\powerset \setA$, again produce elements of~$\powerset \setA$.
That is, if~$\subA, \subB \in \powerset \setA$, then~$\subA \setunion \subB \in \powerset \setA$, and similarly~$\subA \setintersection \subB \in \powerset \setA$.

The operations~$\setunion$,~$\setintersection$, and~$\compl{( - )}$ on~$\powerset \setA$ obey various rules which are useful to be familiar with.
For example,
\begin{equation*}
    \compl{(\subA \setintersection \subB)} = (\compl{\subA}) \setunion (\compl{\subB}).
\end{equation*}
Can you state more such rules?
A useful visual aid for such calculations are so-called Venn diagrams.

\todojira{608}{POST-ALPHUBEL:Include an example Venn diagram as a figure.}

\subsection{Tuples}

The notion of a ``tuple'' is is essentially identical to that of a list: a finite sequence or listing of ``things'', with repetitions allowed.

Usually tuples are referred to in a way that specifies a specific length: tuples of length one are called 1-tuples; tuples of length two are usually called \emph{ordered pairs} or 2-tuples; tuples of length three are called \emph{triples} or~3-tuples; tuples of length four are calle \emph{quadruples} or 4-tuples; and so on.
In general, given~$n \in \natnumbers$, a tuple of length $n$ is called an $n$-tuple.

The notation to indicate a tuple is via angled brackets: for instance,~$\tup{\ela, \elb, \elc}$ denotes a 3-tuple, or triple.
The items inside the brackets will be called entries or components.

So what is the difference between lists and tuples?
For us, it is essentially context and usage.

Typically, we will use tuples in situations where we also specify, for each entry of the tuple, a set for which that entry is an element.
For example, if~$\setA = \makeset{ \sapple, \scarrot, \sgrapes}$ and~$\setB = \makeset{ \stea, \swater }$, then we will sometimes want to specify that~$\tupp{\sapple, \swater}$ is a 2-tuple where~$\sapple \in \setA$ and~$\swater \in \setB$.
Or, as another example, we might consider~$\tup{3, 7, 8}$ as a triple of natural numbers: here we are specifying that we are thinking of 3, 7, and 8 as elements of the set~$\natnumbers$.

\subsection{Cartesian product}\label{sec:cartesian-product}

Given sets~$\setA$ and~$\setB$, their \emph{cartesian product} is a new set -- denoted~$\setA \cartprod \setB$ -- who's elements are precisely all possible 2-tuples~$\tup{\ela, \elb}$ such that the first entry~$\ela$ is an element of~$\setA$ and the second entry~$\elb$ is an element of~$\setB$.

For example, if~$\setA = \{ \sapple, \sgrapes, \scarrot\}$ and~$\setB = \{ \stea, \swater \}$, then
\begin{equation*}
    \setA \cartprod \setB = \{ \tup{\sapple, \stea}, \tup{\sapple, \swater}, \tup{\sgrapes, \stea}, \tup{\sgrapes, \swater},  \tup{\scarrot, \stea}, \tup{\scarrot, \swater}\}.
\end{equation*}
In the special case where~$\setA = \emptyset$ or~$\setB = \emptyset$, then~$\setA \cartprod \setB = \emptyset$.

\begin{remark}
    For finite sets~$\setA$ and~$\setA$, the size of~$\setA \cartprod \setB$ is the product (multiplication) of the sizes of~$\setA$ and~$\setB$:
    \begin{equation*}
        \vert \setA \cartprod \setB \vert = \vert \setA \vert \cdot \vert \setB \vert.
    \end{equation*}
    This is one reason why we think of~$\setA \cartprod \setB$ as a kind of multiplication of sets.
\end{remark}

In set theory, 2-tuples (ordered pairs) are usually defined formally by setting
\begin{equation}
    \label{eq:ordered-pair}
    \tup{\ela, \elb} \coloneqq \{ \{\ela \} , \{ \ela, \elb \} \},
\end{equation}
which allows to give the more formal definition of the cartesian product as
\begin{equation}
    \setA \cartprod \setB = \{ \elc \in \powerset (\setA \setunion \setB) \mid \exists \ela \in \setA, \exists \elb \in \setB : \elc = \tup{\ela, \elb} \}.
\end{equation}

We will, however, simply treat ordered pairs as a primitive given construction (i.e. without reference to formal set theory), and hence we also treat the construction of `cartesian product of sets' as primitive.

Similarly, we will take the notions of $n$-tuple, for any~$n \in \natnumbers$ as primitive, and with it also the notion of $n$-fold cartesian product of sets.
That is, if~$\setA_1, \setA_2, \ldots, \setA_n$ are sets, then we take
\begin{equation*}
    \setA_1 \cartprod \setA_2 \cartprod \dots \cartprod \setA_n
\end{equation*}
to be the set of $n$-tuples~$\tup{\ela_1, \ldots, \ela_n}$ with~$\ela_i \in \setA_i$ for all~$i = 1,2,\ldots,n$.

\subsection{Disjoint union}
\label{sec:disjoint-union}

Given sets~$\setA$ and~$\setB$, their disjoint union is the set
\begin{equation*}
    \setA \setdisunion \setB \coloneqq (  \{ 1 \} \cartprod \setA) \setunion ( \{2\}  \cartprod \setB).
\end{equation*}
In other words, an element of~$\setA \setdisunion \setB$ is either a tuple~$\disunionA{\ela}$ for some~$\ela \in \setA$ or a tuple $\disunionB{\elb}$ for some~$\elb \in \setB$.
The sets~$\{ 1 \}$ and~$\{2\}$ here simply provide labels which ``force'' the sets~$\{ 1 \}  \cartprod \setA$ and~$ \{2\}  \cartprod \setB$ to be disjoint (even if~$\setA$ and~$\setB$ have elements in common).

%As a simple example, let~$\setA = \{ \sapple, \sgrapes, \scarrot\}$ and~$\setB = \{ \stea, \swater \}$.
%Then
%\begin{equation*}
%    \setA \setdisunion \setB = \{ \disunionA{\sapple}, \disunionA{\sgrapes}, \disunionA{\scarrot}, \disunionB{\stea},  \disunionB{\swater}\}.
%\end{equation*}

%Or consider~$\setA \setdisunion \setA$, say.
%By definition, this is~$(\{ 1 \} \cartprod \setA) \setunion (\{2\} \cartprod \setA)$, so if~$\setA = \{ \sapple, \sgrapes, \scarrot\}$, then
%\begin{equation*}
%    \setA \setdisunion \setA = \{ \disunionA{\sapple}, \disunionA{\sgrapes}, \disunionA{\scarrot}, \disunionB{\sapple}, \disunionB{\sgrapes}, \disunionB{\scarrot} \}.
%\end{equation*}

Consider the sets~$\setA=\{\sbretzel, \stea\}$ and~$\setB=\{\sfondue, \smilk\}$.
Their disjoint union can be represented as:
\equationsag{30_disjoint_union}{eq:fig_disjoint}

We can define the disjoint union of a set with itself; this corresponds to having two distinct copies of the set:
\equationsag{30_disjoint_union_self}{eq:fix_disjointself}

Let us also think about what happens when the empty set is at play.
For example, if~$\setA = \emptyset$, then~$\emptyset \setdisunion \setB =  \{2\} \cartprod \setB$, and similarly~$\setA \setdisunion \emptyset =  \{ \disunionA{\ela} \mid \ela \in \setA \}$.
Also,~$\emptyset \setdisunion \emptyset = \emptyset$.

\begin{remark}
    If~$\setA$ and~$\setB$ are finite, then the size of~$\setA \setdisunion \setB$ is the sum of the sizes of~$\setA$ and~$\setB$:
    \begin{equation*}
        \vert \setA \setdisunion \setB \vert = \vert \setA \vert + \vert \setB \vert.
    \end{equation*}
    This is a reason why we think of~$\setA \setdisunion \setB$ as a form of addition of sets.
\end{remark}

\begin{remark}
    Later we will see that the cartesian product of sets is a special case of a very general construction in category theory, called the categorical product, and that the disjoint union of sets is a special case of a ``dual'' construction, called coproduct.
\end{remark}
