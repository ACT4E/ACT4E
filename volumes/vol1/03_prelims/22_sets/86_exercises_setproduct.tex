\sectionexercises{Set products}

We define a \classname{SetProduct} to be a special \Setoid that remembers its factors
and is able to pack/unpack elements into/out of a a product element.

\classlisting{SetProduct}

The semantics is the following:
\begin{itemize}
    \item The method \funcname{components} returns the ordered list of the setoid components.
    \item The method \funcname{pack} takes a number of arguments and creates an element of the product.
          % \item The method \funcname{projections} returns the list of projection mappings.
          %       The~$i$-th projection mapping takes an element of the product and returns the~$i$-th component.
\end{itemize}

There is also a specialization, which corresponds to a product of finite sets; the only change is that the components are finite sets rather than setoids.

\classlisting{FiniteSetProduct}

\codeboilerplate{FiniteMakeSetProduct}{

}
\classlisting{FiniteMakeSetProduct}{}
\clearpage
\subsection{Representation}
% The interface above means that if you are passed 2 \FiniteSet{}s, you should return a \FiniteSet.
% Otherwise, you should return a \Setoid.

\margindatafilefig{set_product}{$\makeset{a,b} \cartprod \makeset{1,2}$}{fig:set_product}

\margindatafilefig{set_product111}{$\makeset{1} \cartprod \makeset{1} \times \makeset{1}$}{fig:set_product11}%

\margindatafilefig{set_product10}{$\makeset{1} \cartprod\emptyset$}{fig:set_product10}%

\begin{codeexercise}[\exname{TestFiniteSetRepresentationProduct}]
    Extend now the code you wrote for loading sets to allow the format in \cref{fig:set_product,fig:set_product11,fig:set_product10}.
    We add another clause to the parsing algorithm:
    \begin{enumerate}
        \item If it has a field \fieldname{elements}, it is a finite set described with elements directly.
        \item If there is a field \fieldname{product}, it must be a list of sets, and the semantics is the product of those sets.
        \item Otherwise, it is an error --- for now; we will introduce many more ways to describe sets.
    \end{enumerate}
    Test your results using
    %: note this cannot be boilerplate
    \checkexercise{TestFiniteSetRepresentationProduct}
\end{codeexercise}
