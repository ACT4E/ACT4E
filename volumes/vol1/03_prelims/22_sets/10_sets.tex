% !TEX root = chapter-standalone.tex

\section{Sets}

Intuitively speaking, sets describe ``collections of things'' -- whether it be a collection of people, of objects, of abstract symbols, etc.
The ``things'' making up a set are called the \emph{elements} of the set.

\subsection{Naming the elements of a set}

There are several typical ways of indicating a set.
One is ``list-like'': we list the elements of the given set, separated by commas, and surround this list with curled brackets.
For example, the set of persons who are co-authors of this book can be written~$\{ \text{Andrea}, \text{Gioele}, \text{Jonathan} \}$, and the set consisting of the symbols ``$\sbanana$'', ``$\sapple$'', ``$\scarrot$'' and ``$\stea$'' is indicated by $\makeset{ \sbanana, \sapple, \scarrot, \stea}$.
\todotextjira{499}{@J: We call the representation ``list-like'' but actually order and repetition matter for lists.}

In a ``list-like" representation, the order that we list elements in does not matter:
$\makeset{ \sapple, \scarrot, \sbanana, \stea}$ and~$\makeset{ \stea, \sbanana, \sapple, \scarrot}$ and~$\makeset{ \sapple, \scarrot, \stea, \sbanana}$ are, for instance, difference ways of indicating one and the same set.
Also, we do not allow repetitions of elements of sets:
$\makeset{ \stea, \sbanana, \sbanana, \sbanana, \sapple, \scarrot }$ is not a valid way of indicating a set.
Given any two elements of a set, they can be only either equal (identical) or unequal (distinct).
\footnote{There is a notion of ``multiset'', where repetitions are allowed (the ``same'' element can appear multiple times), but we are not considering that notion here.}

\begin{marginfigure}
    \centering
    \includesag{sets_as_clouds}
    \caption{We represent sets as ``clouds'' or ``bags'' of nonrepeating elements.}
    \label{fig:set_as_clouds}
\end{marginfigure}

Because in general the elements of set are not ordered in any way, one often visualizes sets as ``clouds'' or ``bags'' of elements (\cref{fig:set_as_clouds}).



%(One of the ways to formalize set theory rigorously is via the axiomatic route, for which there are various different well-known approaches.)

\subsection{Membership and equality}

A fundamental notion is the relation of ``being an element of'', indicated by the symbol ``$\in$''.
For example, the statement that the symbol ``$\stea$'' is an element of the set~$\makeset{ \sapple, \scarrot, \sbanana, \stea }$ is written ``$\stea \in \makeset{ \sapple, \scarrot, \sbanana, \stea }$''.
Similarly, ``$\text{Gioele} \in \{ \text{Andrea}, \text{Gioele}, \text{Jonathan} \}$'' is a true statement.

Given  two sets, we can talk about whether they are \emph{equal} or not.
Two sets~$\setA$ and~$\setB$ are equal if and only if ``they have the same elements''.
In other words,~$\setA$ and~$\setB$ are equal if and only if the statements
``$\ela \in \setA$'' and ``$\ela \in \setB$'' are logically equivalent:
if these statements are always either both true or both false for any instantiation of the variable~$\ela$.
For example, $\makeset{ \sapple, \scarrot, \sbanana, \stea }$ and~$\makeset{ \stea, \sbanana, \sapple, \scarrot}$ and~$\makeset{ \sapple, \scarrot, \stea, \sbanana }$ are all equal as sets, because their elements are the same.


\subsection{Subsets}

Now consider the set~$\makeset{ \sapple, \scarrot, \sbanana }$, and compare it with~$\makeset{ \sapple, \scarrot, \sbanana, \stea }$.
Each of the elements of the first set is also an element of the second set;
in such a case we say that the first set is \emph{included} in the second set.
In symbols,
%
\begin{equation}
    \label{eq:subset}
    \{ \sapple, \scarrot, \sbanana \} \subseteq \{ \sapple, \scarrot, \sbanana, \stea \}.
\end{equation}
%
Generally, given sets~$\setA$ and~$\setB$, the statement~$\setA \subseteq \setB$ is logically equivalent to the statement
%
\begin{equation*}
    \prfperiod{
        \ela \in \setA
    }{
        \ela \in \setB
    }
\end{equation*}

Returning to \cref{eq:subset}, the second set is, on the other hand, \emph{not} included in the first set, since~$\stea$ is an element of the second set, but not the first.
If we say a set is ``strictly included'' in another, then we mean ``included in and not equal'';
if the adjective ``strictly'' is not used, then inclusion means for us that equality is also possible.
\todotextjira{499}{@J: seems like a good place to introduce also $\subset$ - also compare with $<$, $\leq$.}
In other words, for us it is true that any set~$\setA$ is included in itself:~$\setA \subseteq \setA$.
Given sets~$\setA$ and~$\setB$, if it is the case that~$\setA \subseteq \setB$, then we say that~$\setA$ is a \emph{subset} of~$\setB$.
Inclusion and equality are related as follows: given sets~$\setA$ and~$\setB$,
%
\begin{equation*}
    \prfdoubleperiod{
        \setA = \setB
    }{
        {\setA \subseteq \setB}
        \quad
        {\setB \subseteq \setA}
    }
\end{equation*}
%
Equivalently, two sets are equal if they have the same elements:
\begin{equation*}
    \prfdoubleperiod{
        \quad \setA = \setB \quad
    }{
        \prfdouble{
            \ela \in \setA
        }{
            \ela \in \setB
        }
    }
\end{equation*}


\subsection{New sets from old}

Suppose now for a moment that we have some sets to work with, such as for instance the sets used as examples above.
There are a few basic operations that allow to construct new sets from the ones we have, and to reason about them.

In addition to the ``list-like'' way of specifying a set that we discussed above, many times sets are specified with the help of a logical ``statement'' or ``sentence'' which is used to give a characterization of its elements.
The idea is this: we start out with some given set~$\setB$, and then we consider a statement~$S(\ela)$ which depends on a variable~$\ela$, which we think of as running over the elements of~$\setB$.
We can then ask: for which elements~$\ela$ of~$\setB$ is the statement~$S(\ela)$ true?
These elements form a subset of~$\setB$, often denoted
%
\begin{equation}
    \label{eq:axiom-specialization}
    \{ \ela \in \setB \mid S(\ela) \}.
\end{equation}
%
For example, let~$\setB = \{ \sapple, \sbanana, \scarrot \}$ and consider the statement
%
\begin{equation*}
    S(\ela) = \text{``} \ela \text{ is a fruit''}.
\end{equation*}
%
Then we can form the set
%
\begin{equation*}
    \{ \ela \in \setB \mid \ela \text{ is a fruit} \} = \{ \sapple, \sbanana \}.
\end{equation*}

There is an interesting special case of this way of constructing subsets of a set~$\setB$:
what if, for a given statement~$S(\ela)$, \emph{none} of the elements~$\ela \in \setB$ are such that~$S(\ela)$ is true?
For example, consider the statement
%
\begin{equation*}
    S(\ela) =\text{ ``} \ela \text{ is the name of a planet''}.
\end{equation*}
%
Then, for the set~$\setB = \{ \sapple, \sbanana, \scarrot \}$, the subset
\begin{equation*}
    \{ \ela \in \setB \mid \ela \text{ is the name of a planet} \}
\end{equation*}
has \emph{no} elements: it is ``empty''.
In most formalizations of set theory, mathematicians agree to use a single unique set -- called the \emph{empty set} and denoted~$\emptyset$ -- to play the role of ``the set with no elements''.
In particular, by definition, the empty set is a subset of any set!

And now is probably as good a moment as any to note that the empty set~$\emptyset$ should not be confused with the set~$\{ \emptyset \}$.
While~$\emptyset$ denotes the unique set with~$\emph{no elements}$, the notation~$\{ \emptyset \}$ denotes a set which has precisely one element, and that element happens to be the set~$\emptyset$.
Generally, sets can have other sets as their elements: for example, we might consider the set
%
\begin{equation*}
    \{ \{ \sbanana \}, \{ \sapple, \scarrot, \sbanana, \stea \}, \emptyset \},
\end{equation*}
%
which has three elements: namely~$\{ \sbanana \}$,~$\{ \sapple, \scarrot, \sbanana, \stea \}$, and~$\emptyset$.

This leads us nicely to another fundamental construction.


\subsection{A note on technical details}

The fully rigorous use of set theory and category theory does involve taking care of some technical details, notably in dealing with set-theory paradoxes and issues related to ``size''.
    However, for our purposes in this book, one can (luckily) safely ignore these details. 
    Nevertheless, since terms and technicalities related to this topic inevitable arise when learning (and reading) category theory, we have included a short explanation of some terms in \cref{sec:technical-terms} below.
    For those interested in deeper discussion, we refer to the excellent exposition \url{https://arxiv.org/abs/0810.1279}.

    \todotextjira{467}{@J: This abstract is too long -- move half to the first section}
    \todotextjira{263}{@J: Include set-theory references in the bibliography and link to them}

\todotextjira{499}{@J: AC: splitting things in sections introduced some jumps like this that need to be filled / changed. (note that the next heading is exercises)}


