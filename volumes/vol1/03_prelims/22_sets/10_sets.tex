% !TEX root = chapter-standalone.tex

\section{Sets}

Intuitively speaking, sets describe ``collections of things'' -- whether it be a collection of people, of objects, of abstract symbols, etc.
The ``things'' making up a set are called the \emph{elements} of the set. 

There is also one (unique) special set that has \emph{no} elements! It is called the \emph{empty set}, and it is denoted by the symbol $\emptyset$. 

\subsection{Specifying a set by naming its elements}

One way to specify a set is to write out the elements of the set, separated by commas, and surrounded by curled brackets.
For example, a set of persons who are co-authors of this book is~$\{ \text{Andrea}, \text{Gioele}, \text{Jonathan} \}$, and the set consisting of the symbols ``$\sbanana$'', ``$\sapple$'', ``$\scarrot$'' and ``$\stea$'' is indicated by $\makeset{ \sbanana, \sapple, \scarrot, \stea}$.

When we specify a set in this way, the order in which we write the elements in does not matter:
$\makeset{ \sapple, \scarrot, \sbanana, \stea}$ and~$\makeset{ \stea, \sbanana, \sapple, \scarrot}$ and~$\makeset{ \sapple, \scarrot, \stea, \sbanana}$ are, for instance, difference ways of indicating one and the same set. 

However, we do \emph{not} allow repetitions of elements of sets: we don't consider 
$\makeset{ \stea, \sbanana, \sbanana, \sbanana, \sapple, \scarrot }$ as a valid way of indicating a set. At best, $\makeset{ \stea, \sbanana, \sbanana, \sbanana, \sapple, \scarrot }$ could be interpreted as specifying the set $\makeset{ \stea, \sbanana, \sapple, \scarrot }$. (One might be tempted to read $\makeset{ \stea, \sbanana, \sbanana, \sbanana, \sapple, \scarrot }$ as meaning that ``$\sbanana$" appears ``three times". This corresponds to the notion of ``multiset'' -- which is something different from a set -- but we are not considering multisets here.)

\begin{marginfigure}
    \centering
    \includesag{sets_as_clouds}
    \caption{We represent sets as ``clouds'' or ``bags'' of non-repeating elements.}
    \label{fig:set_as_clouds}
\end{marginfigure}

Because the elements of a set are in general not ordered in any way, we can visualize sets as ``clouds'' or ``bags'' of elements, as in \cref{fig:set_as_clouds}.


\subsection{Lists vs sets}

A different concept from that of a set is the notion of \emph{list}. We will denote lists using square brackets, like so $[\sbanana, \sapple, \scarrot, \stea]$. For lists we will speak of \emph{entries} instead of elements, and for a list, when we write out its entries, the order of the entries \emph{does} matter, in contrast to the situation for sets. For instance, $[ \sapple, \scarrot, \sbanana, \stea]$ and $[\sbanana, \sapple, \scarrot, \stea]$ are not the same list. 

Also, for lists, repetitions \emph{are} allowed: for example, $[\sbanana, \sapple, \scarrot, \stea]$ and $[\sbanana, \sapple, \sapple, \scarrot, \stea]$ are both valid specifations of a list, and these the two lists are distinct from each other. 







%(One of the ways to formalize set theory rigorously is via the axiomatic route, for which there are various different well-known approaches.)

\subsection{Membership}

When some ``thing'' is an element of a set, we also say that that thing is a \emph{member} of that set, or that it \emph{belongs} to that set. The symbol to indicate membership is ``$\in$'': for example, 
\begin{equation}
\stea \in \makeset{ \sapple, \scarrot, \sbanana, \stea }
\end{equation} 
is the statement that $\stea$ is an element of the set~$\makeset{ \sapple, \scarrot, \sbanana, \stea }$. 

You may now be wanting to ask: what counts as a ``thing" here, anyway? In many rigorous treatments of set-theory (there are various different theories), the mathematical world is, roughly speaking, made up of sets. In other words, a ``thing'' is a set.\footnote{To give a glimpse of how this can work, consider the following formal model for the natural numbers, using sets: define the number zero to be the empty set $\emptyset$, define the number one to be the one-element set $\{ \emptyset \}$, define the number two to be the two-element set $\{ \emptyset, \{ \emptyset \} \}$, etc. } We will not delve into the rabbit holes of formal set-theory here, however one basic (and sometimes confusing) point is that sets can be elements of other sets.  

For example, we might consider the set
%
\begin{equation}\label{eq:set-with-sets-as-elements}
    \{ \sapple, \{ \sbanana \}, \{ \sapple, \scarrot, \sbanana, \stea \} \},
\end{equation}
%
which has three elements: namely~$\sapple$, $\{ \sbanana \}$, and~$\{ \sapple, \scarrot, \sbanana, \stea \}$. Each of $\sapple$, $\{ \sbanana \}$, and~$\{ \sapple, \scarrot, \sbanana, \stea \}$ is a ``thing'', and these three things happen to be assembled together in the set \cref{eq:set-with-sets-as-elements}. Based on our conventions (curly brackets specify sets), both $\{ \sbanana \}$ and~$\{ \sapple, \scarrot, \sbanana, \stea \}$ are sets (and for our purposes, we do not need to say rigorously what kind of a thing $\sapple$ might be). 

From the above discussion it is hopefully now clear that, given sets $\setA$, $\setB$, and $\setC$, say, we can build for example a new set 
\begin{equation}
\{ \setA, \setB, \setC \}
\end{equation}
whose elements are $\setA$, $\setB$, and $\setC$. It is hopefully also clear for example that $\emptyset$ and $\{ \emptyset \}$ are different sets (this is sometimes a confusing case!). The former is the emptyset (is has no elements), while the latter is a set which has precisely one element, and that element happens to be the set~$\emptyset$. 

\subsection{Equality}

Given two sets, we say they are equal if and only if ``they have the same elements''. For example, we have seen that $\makeset{ \sapple, \scarrot, \sbanana, \stea }$ and~$\makeset{ \stea, \sbanana, \sapple, \scarrot}$ and~$\makeset{ \sapple, \scarrot, \stea, \sbanana }$ are all equal as sets, because their elements are the same. In particular, this example shows that a given set might have many different names or symbolic representations. 

From one perspective, the criterion for knowing when sets are equal reduces to knowing when elements (or ``things'') are equal. If $\setA$ and $\setB$ are sets, then to check if $\setA = \setB$, we need to check if respective elements of $\setA$ and $\setB$ are equal. 

From another perspective, the question of equality of sets can be expressed in terms of membership: $\setA$ and $\setB$ are equal if and only if the statements $\ela \in \setA$ and $\ela \in \setB$ are logically equivalent ($\ela$ is a variable that can be instantiated with the elements of $\setA$ or $\setB$):  

\begin{equation*}
    \prfdoubleperiod{
        \quad \setA = \setB \quad
    }{
        \prfdouble{
            \ela \in \setA
        }{
            \ela \in \setB
        }
    }
\end{equation*}



%When working with sets, we think of ``equality'' as a fundamental relation (or property) that any two given sets (or ``things'') can have: they can either be equal (identical) or unequal (distinct). Sometimes very important theorems in mathematics consist of proving that some set defined in one way is in fact equal (or inequal) to some other set which is defined in some other way!


\subsection{Subsets}

Consider the set~$\makeset{ \sapple, \scarrot, \sbanana }$, and compare it with~$\makeset{ \sapple, \scarrot, \sbanana, \stea }$.
Each of the elements of the first set is also an element of the second set;
in such a case we say that the first set is a \emph{subset} of or is \emph{included} in the second set. 
In symbols,
%
\begin{equation}
    \label{eq:subset}
    \{ \sapple, \scarrot, \sbanana \} \subseteq \{ \sapple, \scarrot, \sbanana, \stea \}.
\end{equation}
%
Generally, given sets~$\setA$ and~$\setB$, the statement~$\setA \subseteq \setB$ (that $\setA$ is a subset of $\setB$) is logically equivalent to the statement
%
\begin{equation}
    \prf{
        \ela \in \setA
    }{
        \ela \in \setB
    }
\end{equation}
in sequent form and equivalent to the statement 
\begin{equation}\label{eq:inclusion-via-quantifier}
\forall \ela \in \setA : \ela \in \setB
\end{equation}
in terms of the ``for all'' universal quantifier. 

Returning to \cref{eq:subset}, the second set is, on the other hand, \emph{not} included in the first set, since~$\stea$ is an element of the second set, but not the first.
If we say a set $\setA$ is ``strictly included'' in another set $\setB$, then we mean ``included in and not equal''; the notation for this is $\setA \subset \setB$. 

The notation for inclusion and strict inclusion of sets is analogous to the notation in the context of numbers fobr ``less than or equals'', $\ela \leq \elb$,  and "strictly less than'', $\ela < \elb$, respectively. 

In general, if we do not use the adjective ``strictly'', then ``inclusion'' means that equality is also possible. In particular, in our terminology it is true that any set~$\setA$ is included in itself:~$\setA \subseteq \setA$.

Inclusion and equality are related as follows: given sets~$\setA$ and~$\setB$,
%
\begin{equation*}
    \prfdoubleperiod{
        \setA = \setB
    }{
        {\setA \subseteq \setB}
        \quad
        {\setB \subseteq \setA}
    }
\end{equation*}
%
(Many times, in order to prove a statement of the form $\setA = \setB$ it is a useful strategy to prove the two statements $\setA \subseteq \setB$ and $\setB \subseteq \setA$ each.)

With respect to inclusion of sets, the empty set $\emptyset$ has (once again) some slightly tricky behavior: $\emptyset$ is a subset of any other set! To see why this makes sense, consider the formulation \cref{eq:inclusion-via-quantifier}: when $\setA$ is the empty set, this statement is always true, since quantifying ``for all'' over the empty set poses no condition at all. 

\subsection{New sets from old}

Suppose that we have some sets to work with, such as for instance the sets used as examples above.
There are a few basic operations that allow us to construct new sets from the ones we have, and to reason about them.

In addition to the ``naming the elements'' way of specifying a set that we discussed above, many times sets are specified with the help of a logical ``statement'' or ``sentence'' which is used to give a characterization of its elements.
The idea is this: we start out with some given set~$\setB$, and then we consider a statement~$S(\ela)$ which depends on a variable~$\ela$, which we think of as running over the elements of~$\setB$.
We can then ask: for which elements~$\ela$ of~$\setB$ is the statement~$S(\ela)$ true?
These elements form a subset of~$\setB$, often denoted
%
\begin{equation}
    \label{eq:axiom-specialization}
    \{ \ela \in \setB \mid S(\ela) \}.
\end{equation}
%
For example, let~$\setB = \{ \sapple, \sbanana, \scarrot \}$ and consider the statement
%
\begin{equation*}
    S(\ela) =  \ela \text{ is a fruit}.
\end{equation*}
%
Then we can form the set
%
\begin{equation*}
    \{ \ela \in \setB \mid \ela \text{ is a fruit} \} = \{ \sapple, \sbanana \}.
\end{equation*}

There is an interesting special case of this way of constructing subsets of a set~$\setB$:
what if, for a given statement~$S(\ela)$, \emph{none} of the elements~$\ela \in \setB$ are such that~$S(\ela)$ is true? Then the result is the empty set!


As an example, consider the statement
%
\begin{equation*}
    S(\ela) =\text{ ``} \ela \text{ is the name of a planet''}.
\end{equation*}
%
Then, for the set~$\setB = \{ \sapple, \sbanana, \scarrot \}$, 
\begin{equation*}
    \{ \ela \in \setB \mid \ela \text{ is the name of a planet} \} = \emptyset.
\end{equation*}





%\subsection{Sizes of sets}
%
%\
%
\todotext{@J: write a short section on cardinality}



\subsection{A note on technical details}

The fully rigorous use of set theory and category theory does involve taking care of some technical details, notably in dealing with set-theory paradoxes and issues related to ``size''.
    However, for our purposes in this book, one can (luckily) safely ignore these details. 
    Nevertheless, since terms and technicalities related to this topic inevitable arise when learning (and reading) category theory, we have included a short explanation of some terms in \cref{sec:technical-terms} below.
    For those interested in deeper discussion, we refer to the excellent exposition \url{https://arxiv.org/abs/0810.1279}.

    \todotextjira{263}{@J: Include set-theory references in the bibliography and link to them}




