% !TEX root = chapter-standalone.tex

\section{Functions}
\label{sec:functions}
\linkvideo{spring2021-morphisms:functions-nomenclature}

Functions are familiar to all of us -- to some degree -- from our school days.
In this section we present some basic terminology and our way of defining functions formally, and we discuss various ways to \emph{think} about functions.

When speaking about functions, we use the words ``map'' or ``mapping'' as synonyms.

\subsection{Specifying functions}
\label{sec:domain-codomain}

Functions can be specified in a variety of ways.

Sometimes they are indicated with the help of a formula, such as~$\mapa(\ela)=\ela^2$.
Or with the help of a table, as in \cref{fig:square_function_table}.

\begin{marginfigure}
    \centering
    \includesag{square_function_table}
    \caption{A function described via a table.}
    \label{fig:square_function_table}
\end{marginfigure}

Other times a function might be characterized by equations or properties.
For example, the exponential function
\begin{equation}
    \defmap{
        \exp
    }{
        \reals
    }{
        \sto
    }{
        \reals
    }{
        \ela
    }{
        e^\ela
    }
\end{equation}
is known to be characterized by the fact that it satisfies the differential equation~$\mapa'(\ela) = \mapa(\ela)$ and the initial condition~$\mapa(0) =1$.

Still other times one might be able to prove the existence (and perhaps also uniqueness) of some function satisfying some given properties, but one might not have any concrete means to ``evaluate'' or ``calculate'' that function.

Whatever the route may be by which a function is specified, for us an essential non-negotiable part of specifying a function is to say which set is its \emph{source} (or \emph{domain}), and which set is its \emph{target} (or \emph{co-domain}).
That is, a function~$\mapa$ is always something that goes from one set~\setA (the source of~$\mapa$) to another set~\setB (the target of~$\mapa$).
We write this as~$\mapa\colon \setA \sto \setB$.

For example, the formula ``$\mapa(\ela)=\ela^2$'' does not specify a function yet, because we did not yet say what source and target set we are considering.
If we are thinking of~$\ela^2$ as defining a function~$\reals \sto \reals$, then this is one function, and if we are thinking of~$\ela^2$ as defining a function~$\posReals \sto \reals$, then this is another function.
And if we are thinking of~$\ela^2$ as a function~$\natnumbers \sto \natnumbers$, this is yet another.

One way to see why this is important: the function
\begin{equation}
    \defmapcomma{
        \mapa
    }{
        \posReals
    }{
        \sto
    }{
        \reals
    }{
        \ela
    }{
        \ela^2
    }
\end{equation}
is monotone (increasing), while
\begin{equation}
    \defmapcomma{
        \mapb
    }{
        \reals
    }{
        \sto
    }{
        \reals
    }{
        \ela
    }{
        \ela^2
    }
\end{equation}
is not.
They have different properties.

Or, as another example, the function~$\mapb$ can be shown to be a continuous function, while for the function
\begin{equation}
    \defmapcomma{
        \mapc
    }{
        \natnumbers
    }{
        \sto
    }{
        \natnumbers
    }{
        \ela
    }{
        \ela^2
    }
\end{equation}
it is perhaps not immediately clear what the question of continuity even means.

\subsection{Functions as deterministic machines}

One typical way of thinking about functions is in terms of ``input'' and ``output''.
Given a function~$\mapa\colon \setA \sto \setB$ we sometimes speak of ``plugging in'' an element~$\ela \setin\setA$ into the function~$\mapa$, and then it will ``output'' an element~$\mapa(\ela) \setin\setB$.

One reason for this kind of thinking is that sometimes functions describe things that are like a computational process or a machine:
for instance, we might give a software program an input, it might then perform a series of computations, and then output an answer, and all of this might be described by a function.

Another reason for thinking of functions in terms of input and output is because humans often use functions -- \emph{as mathematical entities} -- in a deterministic ``machine'' kind of way.
Starting with some element~$\ela \setin\setA$, we can sometimes use the function~$\mapa$ to \emph{calculate} or otherwise \emph{determine} what the ``output''~$\mapa(\ela) \setin\setB$ is.
For example, if we consider the function
%
\begin{equation}
    \label{eq:square-function-whole-numbers}
    \defmapcomma{
        \mapa
    }{
        \wnumbers
    }{
        \sto
    }{
        \wnumbers
    }{
        \ela
    }{
        \ela^2
    }
\end{equation}
%
then, given any input~$\ela \setin\wnumbers$, we can compute the output~$\mapa(\ela) \setin\wnumbers$ by multiplying the input~$\ela$ with itself.

Mathematically speaking, functions are \emph{deterministic} in the sense that for any input~$\ela$, there is \emph{exactly one} output~$\mapa(\ela)$.
This is in contrast to the fact that a given output~$\mapa(\ela)$ might arise from various possible inputs: for example~$4 \setin\wnumbers$ could be the output of \cref{eq:square-function-whole-numbers} for the input~$2 \setin\wnumbers$ or for the input~$\shortminus 2 \setin\wnumbers$.

\subsection{Functions as relations}

Another point of view is that a function~$\mapa\colon \setA \sto \setB$ defines a certain kind of \emph{relation} between the elements of~\setA and~\setB.
Given an~$\ela \setin\setA$, the function~$\mapa$ tells us that this~$\ela$ is related to a certain~$\elb \setin \setB$, which we happen to call~$\mapa(\ela)$.
This point of view is fully compatible with thinking of functions as ``mathematically deterministic''.
However, it is more general than interpreting functions as describing processes which are ``physically deterministic'' in any sense or where ``the input precedes the output''.

As an illustration, consider a large phone book (of personal mobile numbers), which is just a table of names and phone numbers.
Let~\setA be the set of phone numbers in the book, and~\setB the set of names.
There is a function~$\mapa\colon \setA \sto \setB$ which, given any phone number in~\setA, will output the name of the person to whom that number is registered.
Normally, every number is assigned to a single name, so a name as an ``output'' of the function is completely determined (mathematically speaking) by the number one ``inputs''.
However, there is no ``physical determinism'' here: there is no non-mathematical process by which the name of the person was ``computed'' or ``causally determined'' by the phone number.
Rather, the function we described arises simply from a table of information.

\subsection{A formal definition}

The idea is that we can formally define a function~$\mapa\colon \setA \sto \setB$ by the ordered pairs~$\tup{\ela, \elb} \setin\setA \cartprod \setB$ which are the elements of what might be called the ``graph'' of the function.
In other words, those ordered pairs of the form ``$\tup{\ela, \mapa(\ela)}$''.

\begin{ctdefinition}[Function]
    \label{def:function}
    Let~\setA and~\setB be sets.
    A \maindef{function}~$\mapa\colon \setA \sto \setB$ is a subset
    %
    \begin{equation}
        \mapa \setsubseteq \setA \cartprod \setB,
    \end{equation}
    %
    with the following property:
    %
    \begin{equation}
        \label{eq:function-determinism}
        \forall \ela \setin\setA  \quad  \exists !
        \ \elb \setin\setB \colon \tup{\ela,\elb} \setin\mapa.
    \end{equation}
    Here~\setA is the source (or domain) and~\setB is the target (or co-domain) of~$\mapa$.
\end{ctdefinition}

We emphasize, once again, that the source and target of a function are ``baked in'' as \emph{part of the definition} of the function.

The property \cref{eq:function-determinism} describes the ``mathematical determinism'' that functions are supposed to have: for any~$\ela \setin\setA$ there exists \emph{exactly one} element~$\elb \setin\setB$ that is ``the result'' of the function~$\mapa$ applied to~$\ela$.

Another important aspect of \cref{eq:function-determinism} is that it says that \emph{for every}~$\ela \setin\setA$ there exists a~$\elb \setin\setB$ that is related to~$\ela$ by~$\mapa$.
In other words, we do not allow functions to be ``partially defined''.
For example, the formula ``$\mapa(\ela) = 1/ \ela$'' could be used to define a function~$\reals \backslash \makeset{ 0 } \sto \reals$, but it would \emph{not} be valid for defining a function~$\reals \sto \reals$.

Although we take \cref{def:function} as our \emph{formal} definition of functions, we will continue to use the standard kinds of notation for functions, for example usually writing~$\elb = \mapa(\ela)$ and not~$\tup{\ela, \elb} \setin\mapa$.
The formal definition above is useful to keep in the back of our minds though.
For instance, when thinking about situations involving the empty set.

\subsection{To and from the empty set}

Do there exist functions~$\Emptyset \sto \setB$ for any set~\setB?
What about~$\setA \sto \Emptyset$?

Consulting \cref{def:function}, we can figure out that there is \emph{always} a function $\Emptyset \sto \setB$ (no matter what set \setB is) because the condition ``$\forall \ \ela \setin\setA \dots$'' in \cref{eq:function-determinism} is trivially satisfied, as we are quantifying over~$\setA = \Emptyset$.
In this case,~$\mapa \setsubseteq \setA \cartprod \setB$ corresponds to~$\Emptyset \setsubseteq \Emptyset \cartprod \setB = \Emptyset$.

On the other hand, if $\setA \neq \Emptyset$, there are \emph{no} functions of the type~$\setA \sto \Emptyset$, because the part ``$\exists ! \ \elb \setin\setB$'' of \cref{eq:function-determinism} cannot be satisfied, since here~$\setB = \Emptyset$.

\subsection{Injective, \SY{surjective}, \SY{bijective} functions}
\SYNDEF{injective}
\SYNDEF{surjective}
\SYNDEF{bijective}
\label{sub:injective-and-surjective}

Even if we don't know a lot of the specifics of some functions, there is a lot we can still say about how functions between sets can behave \emph{in general}.
In the following we review a number of basic observations and properties.

Let~$\mapa\colon \setA \sto \setB$ be a function.
\begin{ctdefinition}[Injective function]\label{def:injective-function}
    A~$\mapa$ is said to be \emph{injective} if for all~$\ela_1, \ela_2 \setin\setA$
    \begin{equation}
        \label{eq:injectivity-cond}
        \prfcomma{
            \mapa(\ela_1) = \mapa(\ela_2)
        }{
            \ela_1 = \ela_2
        }
        % \forall \ \ela_1, \ela_2 \setin\setA \colon \mapa(\ela_1) = \mapa(\ela_2)  \Rightarrow \ela_1 = \ela_2
    \end{equation}
\end{ctdefinition}
\begin{ctdefinition}[Surjective function]\label{def:surjective-function}
    and~$\mapa$ is called \emph{surjective} if the condition
    \begin{equation}
        \label{eq:surjectivity-cond}
        \forall \elb \setin\setB \ \exists  \ela \setin\setA \colon \mapa(\ela) = \elb
    \end{equation}
    holds.
\end{ctdefinition}
\begin{ctdefinition}[Injective function]\label{def:bijective-function}
    A function which is both \SY{injective} and \SY{surjective} is called \maindef{bijective}.
\end{ctdefinition}
\todographics{\bernina: @Gioele: add figure of an \SY{injective} function}

\begin{marginfigure}
    \centering
    \includesag{surj_function}
    \caption{A \SY{surjective} function}
    \label{fig:surj-function}
\end{marginfigure}

\subsection{Image, preimage}

The \emph{image} of~$\mapa\colon \setA \sto \setB$ is the set
\begin{equation}
    \mapa(\setA) \definedas \makeset{ \elb \setin\setB \mid \exists \ela \setin\setA \colon \mapa(\ela) = \elb }.
\end{equation}
More generally, given a subset~$\subA \setsubseteq \setA$, its image under~$\mapa$ is
\begin{equation}
    \mapa(\subA) \definedas \makeset{ \elb \setin\setB \mid \exists \ela \setin\subA \colon \mapa(\ela) = \elb }.
\end{equation}
Given a subset~$\subB \setsubseteq \setB$, its \emph{preimage} under~$\mapa$ is
\begin{equation}
    \mapa^{-1}(\subB) \definedas \makeset{ \ela \setin\setA \mid  \mapa(\ela) \setin\subB }.
\end{equation}
An alternative way of phrasing injectivity of~$\mapa$ is to say that for every singleton subset~$\makeset{ \elb } \setsubseteq \setB$, its preimage under~$\mapa$ is either a singleton set or the empty set.
\SY{Surjectivity} of~$\mapa$ is equivalent to saying that~$\mapa(\setA) = \setB$.

\subsection{Function composition}

\begin{marginfigure}
    \centering
    \includesag{function_comp_figure}
    \caption{}
    \label{fig:function_comp_figure}
    \todographics{\bernina: Label the sets $\setA$,$\setB$,$\setC$.}
\end{marginfigure}

\begin{marginfigure}
    \centering
    \includesag{function_comp_figure_b}
    \caption{}
    \label{fig:function_comp_figure_b}
\end{marginfigure}

Importantly, functions can be \emph{composed} when the target set of one functions is the same as the source set of another.
\begin{ctdefinition}[Composition of functions]
    \label{def:composition-of-functions}
    \SYNDEF{composition of functions}
    Given functions~$\mapa\colon \setA \sto \setB$ and~$\mapb \colon \setB \sto \setC$, we denote their \emph{composition} by
    \begin{equation}
        \label{eq:composition-notation-functions}
        \defmapperiod{
            \mapa \mthen \mapb
        }{
            \setA
        }{
            \sto
        }{
            \setC
        }{
            \ela
        }{
            \mapb(\mapa(\ela))
        }
    \end{equation}
\end{ctdefinition}

The notation ``$\mapa \mthen \mapb$'' is different from the more traditional notation ``$\mapb \after \mapa$''.
We speak pronounce it as ``$\mapa$ then~$\mapb$'', which aligns with the fact that, to evaluate the composition~$\mapa \mthen \mapb$ at an element~$\ela \setin\setA$, we first apply~$\mapa$ to compute~$\mapa(\ela)$, and \emph{then} we apply~$\mapb$ to compute the result.

\subsection{Identity functions}

\SYNDEF{identity function}
For every set~\setB there is a special function $\catidat\setB \colon \setB \sto \setB$ which ``does nothing''.
\begin{ctdefinition}[Identity function on a set]\label{def:identity-function}
    The \maindef{identity function} on a set~\setB is given by
    \begin{equation}
        \defmapperiod{
            \mapidat\setB
        }{
            \setB
        }{
            \sto
        }{
            \setB
        }{
            \elb
        }{
            \elb
        }
        % (\elb) = \elb \quad \forall  \elb \setin\setB.
    \end{equation}
\end{ctdefinition}
Because such a function ``does nothing'', it behaves neutrally with respect to the composition of functions: given~$\mapa \colon \setA \sto \setB$ and~$\mapb \colon \setB \sto \setC$, we have~$\mapa \mthen \catidat\setB = \mapa$ and~$\catidat\setB \mthen \mapb = \mapb$.

\begin{marginfigure}
    \centering
    \includesag{identity_function}
    \caption{An identity function}
    \label{fig:identity-function}
\end{marginfigure}

\subsection{Isomorphisms}

Identity functions are used, for example, to say when a function is \emph{invertible} or, synonymously, that it is an \emph{isomorphism}.

\begin{ctdefinition}[Isomorphism]
    \label{def:function-isomorphism}
    \SYNDEF{function isomorphism}
    A function~$\mapa \colon \setA \sto \setB$ is an isomorphism if there exists an \emph{inverse} to~$\mapa$ -- a function~$\mapb \colon \setB \sto \setA$ such that
    \begin{equation}
        \label{eq:isomorphism-equations}
        \mapa \mthen \mapb = \catidat\setA
        \qqand
        \mapb \mthen \mapa = \catidat\setB.
    \end{equation}
\end{ctdefinition}

\begin{exercise}\label{ex:uniqueness-of-inverse-function}
    Show that an inverse to~$\mapa$ is necessarily unique (so we can speak of ``the'' inverse).
\end{exercise}

\begin{solution}
    \emph{Per absurdum}, assume that a map~$\mapa\colon \setA \sto \setB$ possesses two \emph{different} inverses~$\mapb,\mapc\colon \setB \sto \setA$.
    Following the definition of inverse, we have
    \begin{equation}
        \mapa \mthen \mapb=\mapa \mthen \mapc=\catidat\setA,
    \end{equation}
    and
    \begin{equation}
        \mapb \mthen \mapa=\mapc \mthen \mapa=\catidat\setB.
    \end{equation}
    Now, we can write
    \begin{equation}
        \begin{aligned}
            \mapc & =\catidat\setB \mthen \mapc \\
                  & =(\mapb \mthen \mapa)\mthen \mapc \\
                  & =\mapb\mthen (\mapa \mthen \mapc) \\
                  & =\mapb \mthen \catidat\setA \\
                  & =\mapb,
        \end{aligned}
    \end{equation}
    which contradicts the initial assumption.
\end{solution}

\begin{remark}
    Note that if~$\mapb$ is the inverse to~$\mapa$, then also~$\mapa$ is the inverse of~$\mapb$.
\end{remark}

\begin{forslides}
    \begin{equation}
        \label{eq:quiz_inj_1}
        \begin{aligned}
            \mapa\colon \reals & \sto \nonNegReals \\
            \ela               & \mapsto \absvalueof \ela
        \end{aligned}
    \end{equation}
    \begin{equation}
        \label{eq:quiz_inj_2}
        \begin{aligned}
            \mapb\colon \natnumbers & \sto \natnumbers \\
            \ela                    & \mapsto \ela + 33
        \end{aligned}
    \end{equation}
    \begin{equation}
        \label{eq:quiz_isos_1}
        \setA=\makeset{{{\star}}, \circ}
    \end{equation}
    \begin{equation}
        \label{eq:quiz_isos_2}
        \setB=\makeset{1, 2, 3}
    \end{equation}
    \begin{equation}
        \label{eq:quiz_isos_3}
        \setC=\makeset{a, b, c}
    \end{equation}
    \begin{equation}
        \label{eq:quiz_isos_4}
        \setA \sto \setB
    \end{equation}
    \begin{equation}
        \label{eq:quiz_isos_5}
        \setB \sto \setC
    \end{equation}
\end{forslides}

\begin{exercise}
    \label{ex:bijective-functions-are-isomorphisms}
    Show that a function is an isomorphism if and only if it is \SY{bijective}.
\end{exercise}
\begin{solution}
    We show the two directions in turn:
    \begin{equation}
        \prftree[r]{\text{ and }}{\mapa \text{ isomorphism}}{\mapa \text{ \SY{bijective}}} \quad \prftree[r]{.
        }{\mapa \text{ \SY{bijective}}}{\mapa \text{ isomorphism}}
    \end{equation}
    Consider an isomorphism~$\mapa\colon \setA \sto \setB$ and its inverse~$\mapb\colon \setB \sto \setA$.
    Take a~$\elb\setin\setB$ and let~$\ela= \mapb(\elb)$.
    We know that~$\mapa(\ela)=\mapa(\mapb(\elb))=(\mapb\mthen \mapa)(\elb)=\elb$.
    Therefore, $\mapa$ must be \emph{\SY{surjective}}.
    To show injectivity, consider~$\ela,\ela'\setin\setA$ such that~$\mapa(\ela)=\mapa(\ela')$.
    Let~$\elb=\mapa(\ela)$ and~$\ela''=\mapb(\elb)$.
    Then, we have
    \begin{equation}
        \begin{aligned}
            \ela' & =\catidat\setA(\ela') \\
                  & =(\mapa \mthen \mapb)(\ela') \\
                  & =\mapb(\mapa(\ela')) \\
                  & =\mapb(\elb) \\
                  & =\ela''.
        \end{aligned}
    \end{equation}
    However, we also know
    \begin{equation}
        \begin{aligned}
            \ela & =\catidat\setA(\ela) \\
                 & =(\mapa \mthen \mapb)(\ela) \\
                 & =\mapb(\mapa(\ela)) \\
                 & =\mapb(\elb) \\
                 & =\ela''.
        \end{aligned}
    \end{equation}
    Therefore,~$\ela=\ela'$ and~$\mapa$ is \SY{injective} (and therefore \SY{bijective}).

    Now, let a map~$\mapa\colon \setB \sto \setA$ be \SY{bijective}.
    One can define~$\mapb\colon \setB \sto \setA$ in the following way.
    Take a~$\elb\setin\setB$, and since~$\mapa$ is \SY{surjective} (it is \SY{bijective}), there exists a~$\ela\setin\setA$ such that~$\mapa(\ela)=\elb$.
    Let~$\mapb(\elb)=\ela$.
    Since~$\mapa$ is \SY{injective},~$\ela$ must be unique, meaning that~$\mapb$ is well-defined.
    Now we check that indeed~$\mapb$ must be the inverse of~$\mapa$.
    Consider~$\ela \setin\setA$ and~$\elb=\mapa(\ela)$.
    By definition,~$\mapb(\elb)=\ela$, and hence~$(\mapa\mthen \mapb)(\ela)=\mapb(\mapa(\ela))=\mapb(\elb)=\ela$, implying~$\mapa\mthen \mapb=\catidat\setA$.
    Similarly, take~$\elb\setin\setB$ and~$\ela=\mapb(\elb)$.
    Then, by definition we have~$\mapa(\ela)=\elb$, and hence~$(\mapb\mthen \mapa)(\elb)=\mapa(\mapb(\elb))=\mapa(\ela)=\elb$, implying~$\mapb\mthen \mapa=\catidat\setB$.
    Therefore, $\mapa$ is an isomorphism.
\end{solution}

\begin{ctdefinition}[Isomorphisms of sets]\label{def:set-isomorphic}
    \SYNDEF{isomorphism of sets}
    Given sets~\setA and~\setB, we say that they are \emph{isomorphic}, and write~$\setA \isomorphic \setB$,
    if there exists an isomorphism~$\setA \sto \setB$ (or~$\setB \sto \setA$).
\end{ctdefinition}

We can use the notion of isomorphism to make a more formal definition of the size, or cardinality, of a set, as follows.
First we posit that any set \setA has an attribute which we call its cardinality, and denote by $\cardof \setA$ (as before).
Then we say that for any two sets, \setA and \setB, it shall hold that
\begin{equation}
    \prfdoubleperiod{
        \cardof \setA  = \cardof \setB
    }{
        \setA \simeq \setB
    }
\end{equation}

For a finite set \setA, we say it has size $n \setin \natnumbers$ if there exists an isomorphism between~\setA and the set $\makeset{1,2, \dots, n-1, n}$.

\begin{gradedexercise}[\exname{CountingIsos}]
    \label{ex:CountingIsos}

    Let $\setA = \makeset{{{\star}}, {{\circ}} }$, $\setB = \makeset{1, 2, 3}$, and $\setC = \makeset{a, b, c}$.
    \begin{enumerate}
        \item How many isomorphisms are there $\setA \sto \setB$?
        \item How many isomorphisms are there $\setB \sto \setC$?
    \end{enumerate}
\end{gradedexercise}

\solutionof{CountingIsos}

\subsection{Sets of functions}

So far we have mostly been thinking of functions as a way to relate one set to another.
However, in our formal definition, a function~$\mapa \colon \setA \sto \setB$ is a certain kind of element in~$\powerset (\setA \cartprod \setB)$.
Consider all those elements of~$\powerset (\setA \cartprod \setB)$ which are indeed functions: they form a set, the set of all functions from~\setA to~\setB.
A notation we will use for this set is~$\setB^{\setA}$.

Why this notation?

If~\setA and~\setB are finite sets, then~$\setB^{\setA}$ is also a finite set and its size is the size of~\setB to the power of the size of~\setA:
\begin{equation}
    \cardmap(\setB^{\setA})  = {\cardof\setB} ^ {\cardof \setA}.
\end{equation}

\subsection{Arithmetic with sets}

In our notation for \SY{cartesian product}, \SY{disjoint union}, and function sets we have used notation inspired by basic operations in arithmetic, motivated in part by the following formulas for sizes of finite sets:
\begin{align}
    \cardof{\setA \cartprod \setB}     & = \cardof \setA   \cdot \cardof \setB, \\
    \cardof{ \setA \setdisunion \setB} & = \cardof \setA  + \cardof \setB, \\
    \cardmap(\setB^{\setA})            & = {\cardof\setB}^{\cardof\setA}.
\end{align}

The parallels of these operations to operations in arithmetic go further.
For example, the following identities hold already on the level of sets, before computing their size; we simply need to replace ``$=$'' with the symbol ``$\isomorphic$'' for ``isomorphic'':
\begin{align}
    \setA \cartprod (\setB \setdisunion \setC) & \isomorphic (\setA \cartprod \setB) \setdisunion (\setA \cartprod \setC), \\
    \setA^{\setB \setdisunion \setC}           & \isomorphic \setA^\setB \cartprod \setA^\setC, \\
    \pars{\setA^\setB}^\setC                   & \isomorphic \setA^{\setB \cartprod \setC}.
\end{align}

These formulas are actually true for sets of any cardinality, not just finite sets.

\subsection{Product and sum of functions}
\label{sec:prod_dirsum_functions}

In the following, we define the \emph{product} and \emph{sum} of functions.
These concepts go hand-in-hand with the notions of product and sum of sets.

\begin{ctdefinition}[Product of functions]
    \label{def:product-of-functions}
    \SYNDEF{product of functions}
    Given functions~$\mapa\colon \setA \sto \setB$ and~$\mapb\colon \setC \sto \setD$, their \emph{product} is the function:
    \begin{equation}
        \defmapperiod{
            \mapa \funcprod \mapb
        }{
            \setA \cartprod \setC
        }{
            \sto
        }{
            \setB\cartprod \setD
        }{
            \tup{\setAel, \setCel}
        }{
            \tup{\mapa(\setAel),\mapb(\setCel)}
        }
    \end{equation}
\end{ctdefinition}

\begin{example}
    Consider~$\mapa,\mapb \colon \reals \sto \reals$ with~$\mapa(\setAel)=\setAel^2$ and~$\mapb(\setBel)=\setBel+1$.
    We have~$(\mapa \funcprod \mapb)(2,5)=\tup{2^2,5+1}=\tup{4,6}$.
\end{example}

\begin{ctdefinition}[Sum of functions]
    \label{def:sum-of-functions}
    \SYNDEF{sum of functions}
    Given functions~$\mapa\colon \setA \sto \setB$ and~$\mapb\colon \setC \sto \setD$, their \emph{sum} is the function:
    \begin{equation}
        \begin{aligned}
            \mapa \funcsum \mapb \colon \setA \setdisunion \setC & \sto \setB\setdisunion \setD \\
            \disunionA{\setEel}                                  & \mapsto \disunionA{\mapa(\setEel)} \\
            \disunionB{\setFel}                                  & \mapsto \disunionB{\mapb(\setFel)} \\
        \end{aligned}
    \end{equation}
\end{ctdefinition}

\begin{example}
    Consider~$\mapa\colon \wnumbers \sto \natnumbers$ with~$\mapa(\setAel)=\setAel^2$ and~$\mapb\colon \reals \sto \reals$ with~$\mapb(\setBel)=\setBel^3$.
    We have~$(\mapa \funcsum \mapb)(\tup{1,2})=\mapa(2)=2^2=4$, and~$(\mapa \funcsum \mapb)(\tup{2,3})=\mapb(3)=3^3 = 27$.
\end{example}

\subsection{The compositional perspective}

A central theme in this book is to study how various mathematical structures compose.
For instance, what are general patterns of how functions and sets \emph{relate to one another} via composition?

In time, we will see that many features that sets and functions exhibit can be broadly generalized to other kinds of mathematical entities.
A guiding philosophy for making or understanding such generalizations is to formulate properties of functions in a way that only uses their ``external compositional aspects'' and does not rely on the fact that we are dealing with sets, which have elements, and so we can ``look inside them''.

This likely sounds very vague at the moment. Let us illustrate with some examples. 

\begin{example}
Consider the property that a function may (or may not) have of being \SY{bijective}.
According to \cref{ex:bijective-functions-are-isomorphisms}, a function~$\mapa\colon \setA \sto \setB$ is \SY{bijective} if and only if it is an isomorphism.
The latter means, by definition, that there exists~$\mapb \colon \setB \sto \setA$ such that~$\mapa \mthen \mapb = \catidat\setA$ and~$\mapb \mthen \mapa = \catidat\setB$.
The point is that the equations in the definition of ``isomorphism'' only make use of the operation of function composition, the notion of quality, and the existence of special identity functions.
There is no mention of elements of sets, as there is in the definition of ``bijective''.
\end{example}

\begin{example}
The notion of ``subset'' is traditionally defined, as we did above, by saying that~$\setA \setsubseteq \setB$ if and only if~$\forall \ \ela \setin\setA$:
\begin{equation}
    \prfperiod{
        \ela \setin\setA
    }{
        \ela \setin\setB
    }
\end{equation}

There are, however, alternatives that do not refer to ``elements''.
To see one way, consider the set~$\setTwo = \makeset{ 0, 1}$.
Any function~$\mapa \colon \setB \to \setTwo$ defines a subset
\begin{equation}
    \setA_\mapa = \makeset{ \elb \setin\setB \mid \mapa(\elb) = 1 } \setsubseteq \setB.
\end{equation}
Conversely, any subset~$\setA \setsubseteq \setB$ defines a function~$\mapa_\setA \colon \setB \sto \setTwo$ by setting
\begin{equation}
    \mapa_\setA(\elb) = \begin{cases}
        1 & \text{ if }\elb\setin\setA, \\
        0 & \text{ elsewhere}.
    \end{cases}
\end{equation}
It can be checked that this defines a 1-to-1 correspondence between functions~$\setB \sto \setTwo$ and subsets of~\setB.
In other words, there is a \SY{bijection} between the set~$\setTwo^\setB$ and the set~$\powerset \setB$.
So, instead of using a definition of subset that involves elements of sets, we could work with functions~$\setB \sto \setTwo$.
\end{example}

\begin{gradedexercise}[\exname{SubsetsAsFunctions}]
    \label{ex:SubsetsAsFunctions}

    Let \setB be any set.
    Prove that $\setTwo^\setB \isomorphic \powerset \setB$.
\end{gradedexercise}

\solutionof{SubsetsAsFunctions}

\todomistake{\alphubel: Big confusion: We need to adjust the symbols here}

\begin{example}
Even the notions of ``element of a set''  and ``evaluation of a function at an element" can be described purely in terms of functions and their composition, without needing to ``look inside'' of the sets involved. 

To show how this works, let us first define $\singleton \definedas \makeset{\singletonel}$, a singleton set whose only element is the symbol ``$\singletonel$'' (any singleton set would do; for concreteness we are fixing one and calling it $\singleton$).

Now we are ready to make an interesting observation: functions~$\singleton \sto \setA$ are in 1-to-1 correspondence with the elements of~\setA. A function~$\mapa\colon \singleton \sto \setA$ will have to map ``$\singletonel$'' to some element~$\mapa(\singletonel) \setin\setA$, and since~$\singleton$ has no other elements, that is all that~$\mapa$ does.
So~$\mapa$ ``picks out'' an element of~\setA.
We can work with functions~$\singleton \sto \setA$ in place of elements of~\setA.

Next, let's talk about function evaluation. Consider a function~$\mapb \colon \setA \sto \setB$.
Given an element~$\ela \setin\setA$, this element will be mapped by~$\mapb$ to an element~$\mapb(\ela) \setin\setB$.
If we use, instead of~$\ela \setin\setA$, the function~$\mapa \colon \singleton \sto \setA$ to which it corresponds, then the element~$\mapb(\ela) \setin\setB$ corresponds to the function~$\mapa \mthen \mapb\colon \singleton \sto \setB$.
In other words, we can talk about evaluation of a function~$\mapb \colon \setA \sto \setB$ ``at an element of~\setA'' without actually using elements, but rather just using functions and function composition.
\end{example}