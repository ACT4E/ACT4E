% !TEX root = chapter-standalone.tex
\sectionexercises{Setoids union \hardexercise}

The \Setoid interface is reported in \cref{lst:Setoid}.
%
A \Setoid is an object that can tell use whether other objects belong to it.
The \funcname{contains} function is akin to the indicator function.

\codeboilerplate{SetoidOperations}{
    Create the function \funcname{union_setoids}, which, given two \SY{setoids}, returns another \SY{setoid} that is the union of the two given \SY{setoids}.

    Create the function \funcname{intersection_setoids}, which, given two \SY{setoids}, returns another \SY{setoid} that is the intersection of the two given \SY{setoids}.
}
\classlisting{SetoidOperations}

\begin{hint}
    The indicator of the \SY{union of sets} is the OR of the two indicators.
\end{hint}

\subsection{EnumerableSets}

%
An enumerable set is one that can enumerate its elements.
The method \funcname{elements} returns an \emph{iterator} that enumerates the elements.

\codeboilerplate{EnumerableSetsOperations}{
    \
    \begin{itemize}
        \item
              Create an Enumerable set given a function that provides the $i$-th element.
        \item
              Create a function that creates the union of two enumerable sets.
    \end{itemize}
}

\classlisting{EnumerableSetsOperations}

\begin{codeexercise}
    Explain why it is not possible to create a function that, given two enumerable sets, creates the \emph{intersection} of the two sets.
\end{codeexercise}
