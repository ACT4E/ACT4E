% !TEX root = chapter-standalone.tex

\section{Further constructions}
\label{sec:more-constructions}

We review here a few more things that are nice to have in our mathematical toolbox.

\subsection{Arbitrary unions and intersections}

Previously we defined the notion of union and the intersection for any two given sets $\setA$ and $\setB$.

Now consider a set of sets -- let's call it $\mathcal{S}$ -- which might have two elements (say, sets $\setA$ and $\setB$), or it might have 108 elements, or it might have infinitely many elements.
No matter how many sets are elements of $\mathcal{S}$, we can also take the union of all those sets.
This is denoted $\bigcup \mathcal{S}$.

In other words,
%
\begin{equation*}
    \ela \in \bigcup \mathcal{S}  \ \Leftrightarrow \ \exists \setA \in \mathcal{S} : \ela \in \setA.
\end{equation*}
%
Similarly we can define the intersection $\bigcap \mathcal{S}$ of all the sets which are elements of $\mathcal{S}$:
%
\begin{equation*}
    \ela \in \bigcap \mathcal{S} \ \Leftrightarrow \  \forall \setA \in \mathcal{S} : \ela \in \setA.
\end{equation*}

This notation for arbitrary unions and intersections is related to our previous definitions for two sets $\setA$ and $\setB$ via
\begin{equation}
    \setA \cup \setB = \bigcup \{ \setA, \setB \} \quad \text{and} \quad \setA \cap \setB = \bigcap \{ \setA, \setB \}.
\end{equation}

\subsection{Families}

Informally, a family of sets is a collection of sets where each set is labeled with an index.
The index allows us to refer to any given member of the family.

For example, we might want to work with a family of sets labeled with an index from the natural numbers $\natnumbers$:
\begin{equation}\label{eq:sequence-of-sets}
    \setA_1, \setA_2, \setA_3, .
    ..
\end{equation}
The set $\natnumbers$ here is the \emph{index set} -- the set in which the indices $1, 2, 3.
    .. $ etc. live.

Now, what if we need to consider a family of sets where the indices live in the set $\reals$?
Then we cannot write things as in \cref{eq:sequence-of-sets}, because $\reals$ is not a countable set.
Instead we write
\begin{equation}
    \{ \setA_i \}_{i \in \reals}.
\end{equation}
Note that this is not the same thing as set of sets: it is possible here that sets in the family with different indices might be equal as sets.
In other words, it might be that $i \neq j$ but $\setA_i = \setA_j$, for some indices $i$ and $j$.
(It might even be that all of the sets $\setA_i$ are in fact the same set.)

Note that one might also call \cref{eq:sequence-of-sets} a sequence of sets, and recall that, formally, sequences are defined as functions whose source is the set $\natnumbers$.
Our formal definition of a family of sets will have a similar flavor: a family will be defined as a function whose source is the index set of the family.

To spell this out, let $\{ \setA_i \}_{i \in \stylesets{I}}$ be a collection of sets (with index set $\stylesets{I}$).
We can consider the set of sets
\begin{equation}
    \{  \setA_i \mid i \in \stylesets{I} \},\footnote{For those familiar with ZFC set theory: the axiom of replacement ensures that this is a set.
    }
\end{equation}
and in particular also its union $\setA := \bigcup \{  \setA_i \mid i \in \stylesets{I} \}$.
Now, our formal definition of a family $\{ \setA_i \}_{i \in \stylesets{I}}$ of sets is that it is a function
\begin{equation}
    \mapa: \stylesets{I} \to \powerset \setA
\end{equation}
such that $\mapa(i) = \setA_i$ for all $i \in \stylesets{I}$.

\todojira{609}{J: include some sort of a non-arbitrary (and useful??) example to make more concrete and illustrate?}

As a side note: to simplify notation, for unions of families of sets we will usually write $\bigcup_{i \in I} \setA_i $ instead of $\bigcup \{  \setA_i \mid i \in \stylesets{I} \}$, and similarly for intersections.

\subsection{Arbitrary cartesian products and disjoint unions}

We have seen that, given sets $\setA$ and $\setB$, we can form their cartesian product $\setA \cartprod \setB$ and disjoint union $\setA \setdisunion \setB$.
And similarly for any finite number of sets.

What about if we start with a family $\{ \setA_i \}_{i \in \stylesets{I}}$ of sets -- can we still take their cartesian product or disjoint union?
The answer is yes.

For the cartesian product of families we use the symbol $\stylesets{\prod}$ instead of the infix notation $\cartprod$, and for disjoint unions of families we use $\stylesets{\sum}$ instead of the infix notation $\setdisunion$.

We define the cartesian product of a family $\{ \setA_i \}_{i \in \stylesets{I}}$ as
\begin{equation}
    \underset{i \in \stylesets{I}}{\stylesets{\prod}}  \setA_i  := \{ \mapa : \stylesets{I} \to \bigcup_{i \in \stylesets{I} } \setA_i \mid \mapa(i) \in \setA_i \}.
\end{equation}

And for the disjoint union of a family $\{ \setA_i \}_{i \in \stylesets{I}}$ we set
\begin{equation}
    \underset{i \in \stylesets{I}}{\stylesets{\sum}} \setA_i := \bigcup_{i \in \stylesets{I}} \setA_i \cartprod \{ i \}.
\end{equation}

