% !TEX root = chapter-standalone.tex

\section{More constructions}
\label{sec:more-constructions}

In this section we review a few more things that we'd like to have in our mathematical toolbox. 


\subsection{Arbitrary unions and intersections}

Previously we defined the notion of union and the intersection for any two given sets $\setA$ and $\setB$. 

Now consider a set of sets -- let's call it $\mathcal{S}$ -- which might have two elements (say, sets $\setA$ and $\setB$), or it might have 108 elements, or it might have infinitely many elements. No matter how many sets are elements of $\mathcal{S}$, we can also take the union of all those sets. This is denoted $\bigcup \mathcal{S}$. 

In other words, 
%
\begin{equation*}
    \ela \in \bigcup \mathcal{S}  \ \Leftrightarrow \ \exists \setA \in \mathcal{S} : \ela \in \setA.
\end{equation*}
%
Similarly we can define the intersection $\bigcap \mathcal{S}$ of all the sets which are elements of $\mathcal{S}$: 
%
\begin{equation*}
     \ela \in \bigcap \mathcal{S} \ \Leftrightarrow \  \forall \setA \in \mathcal{S} : \ela \in \setA.
\end{equation*}

This notation for arbitrary unions and intersections is related to our previous definitions for two sets $\setA$ and $\setB$ via
\begin{equation}
\setA \cup \setB = \bigcup \{ \setA, \setB \} \quad \text{and} \quad \setA \cap \setB = \bigcap \{ \setA, \setB \}.
\end{equation}



\subsection{Families}



\subsection{Arbitrary products and coproducts}


\subsection{Sets of functions}


\subsection{Arithmetic with sets}