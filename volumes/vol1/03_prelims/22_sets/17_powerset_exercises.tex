\sectionexercises{Power set}
\label{sec:power-set-exercise}

\codeboilerplate{FiniteMakePowerSet}{
    Compute the power set of a finite set.
}

The skeleton is provided in \files{act4e_solutions/sets_power.py} in the class \classname{SolFiniteMakePowerSet}.

\classlisting{FiniteMakePowerSet}

Note that \funcname{powerset} must return a special setoid that is a \classname{FiniteSetOfFiniteSubsets} (\cref{lst:FiniteSetOfFiniteSubsets}), which derives from \classname{SetOfFiniteSubsets}.

\classlisting{SetOfFiniteSubsets}
\classlisting{FiniteSetOfFiniteSubsets}

These classes are declared generic in \TypeVar{C} and \TypeVar{E}.
\TypeVar{C} is the type of the elements inside the subset.
\TypeVar{E} is the type of the subset.

The function \funcname{contents} decomposes a subset in components.
The function \funcnameT{con\-struct} constructs a subset from a sequence of elements.

Note how we are not constraining what is the type of \TypeVar{E}: we do not care how the implementation represents a subset, but just that we can access the elements inside it.

As for \FiniteSet, you will have to implement the class \classname{FiniteSetOfFiniteSubsets} as part of this exercise. Again, you can either do this in \files{act4e_solutions/sets_power.py} or create a seperate file (called \files{act4d_solutions/my_power_sets.py} for instance). 
For the latter, don't forget to import the newly created class in \files{act4e_solutions/sets_power.py} by adding the following line:
\begin{minted}{python}
    from act4e_solutions/my_power_sets.py import MyFiniteSetOfFiniteSubsets
\end{minted}
Moreover, don't forget to define the \funcname{__init__()} for \classname{MyFiniteSetOfFiniteSubsets} and to implement all the methods in the classes \classname{FiniteSetOfFiniteSubsets} inherits from.
That is you need to implement the methods marked by a decorator \pystr|@abstractmethod| in \classname{SetOfFiniteSubsets}, \classname{FiniteSet} and \classname{Setoid}.