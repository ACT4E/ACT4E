% !TEX root = chapter-standalone.tex

\section{Sets}

Intuitively speaking, sets describe ``collections of things'' -- whether it be a collection of people, of objects, of abstract symbols, \etc
The ``things'' making up a set are called the \emph{elements} of the set.

There is also one (unique) special set that has \emph{no} elements.
It is called the \emph{empty set}, and it is denoted by the symbol~$\Emptyset$.

\subsection{Specifying a set by naming its elements}

One way to specify a set is to write out the elements of the set, separated by commas, and surrounded by curled brackets.
For example, the set consisting of the symbols ``$\sgrapes$'', ``$\sapple$'', ``$\scarrot$'' and ``$\stea$'' is indicated by~$\makeset{ \sgrapes, \sapple, \scarrot, \stea}$.

When we specify a set in this way, the order in which we write the elements does not matter:~$\makeset{ \sapple, \scarrot, \sgrapes, \stea}$ and~$\makeset{ \stea, \sgrapes, \sapple, \scarrot}$ and~$\makeset{ \sapple, \scarrot, \stea, \sgrapes}$ are, for instance, different ways of indicating one and the same set.

However, we do \emph{not} allow repetitions of elements of sets: we do not consider~$\makeset{ \stea, \sgrapes, \sgrapes, \sgrapes, \sapple, \scarrot }$ as a valid way of indicating a set.
At best,~$\makeset{ \stea, \sgrapes, \sgrapes, \sgrapes, \sapple, \scarrot }$ could be interpreted as specifying the set~$\makeset{ \stea, \sgrapes, \sapple, \scarrot }$.
(One might be tempted to read $\makeset{ \stea, \sgrapes, \sgrapes, \sgrapes, \sapple, \scarrot }$ as meaning that ``$\sgrapes$'' appears ``three times''.
This corresponds to the notion of ``multiset'' -- which is something different from a set -- but we are not considering multisets here.)

\begin{marginfigure}
    \centering
    \includesag{sets_as_clouds}
    \caption{We represent sets as ``clouds'' or ``bags'' of non-repeating elements.}
    \label{fig:set_as_clouds}
\end{marginfigure}

Because the elements of a set are in general not ordered in any way, we can visualize sets as ``clouds'' or ``bags'' of elements, as in \cref{fig:set_as_clouds}.

\subsection{The size of a set}
\label{sec:cardinality}
The usual name for the size of a set is its \maindef{cardinality}.
If a set has finitely-many elements, then its size or cardinality is just the number of elements it contains.
For example, the set~$\makeset{ \sapple, \sgrapes, \scarrot }$ has cardinality equal to~$3$.

Sets with finite cardinality are called \emph{finite sets}.
If a set is not finite, there are different possible ``infinite sizes''.
We will delve into distinguishing different infinite cardinalities only later when we need to.

Our notation for the size of a set~\setA will be~$\cardof{\setA}$ or $\vert \setA \vert$.
For now, we will agree that~$\cardof{\setA}$ can be either zero (for the empty set), a natural number, or simply ``infinite''.
For sets of infinite size, there are different, distinct possible cardinalities (different infinities); we will discuss a bit later a way of making sense of this.

\subsection{Set membership}
\label{sec:set-membership}
When some ``thing'' is an element of a set, we also say that that thing is a \emph{member} of that set, or that it \emph{belongs} to that set.
\\The symbol to indicate membership is ``$\setin$'': for example,
\begin{equation}
    \stea \setin\makeset{ \sapple, \scarrot, \sgrapes, \stea }
\end{equation}
is the statement that~$\stea$ is an element of the set~$\makeset{ \sapple, \scarrot, \sgrapes, \stea }$.
To indicate that something is \emph{not} an element of a set we use the symbol ``$\notsetin$''.

You may now be wanting to ask: what counts as a ``thing'' here, anyway?
In many rigorous treatments of set theory (there are various different theories), the mathematical world is, roughly speaking, made up of sets.

In other words, a ``thing'' is a set\footnote{
    To give a glimpse of how this can work, consider the following formal model for the natural numbers, using sets: define the number zero to be the empty set~$\Emptyset$, define the number one to be the one-element set~$\makeset{ \Emptyset }$, define the number two to be the two-element set~$\makeset{ \Emptyset, \makeset{ \Emptyset } }$, \etc }.
We will not delve into the rabbit holes of formal set-theory here, however one basic (and sometimes confusing) point is that sets can be elements of other sets.

For example, we might consider the set
%
\begin{equation}
    \label{eq:set-with-sets-as-elements}
    \makeset{ \sapple, \makeset{ \sgrapes }, \makeset{ \sapple, \scarrot, \sgrapes, \stea } },
\end{equation}
%
which has three elements: namely~$\sapple$, $\makeset{ \sgrapes }$, and~$\makeset{ \sapple, \scarrot, \sgrapes, \stea }$.
Each of~$\sapple$, $\makeset{ \sgrapes }$, and~$\makeset{ \sapple, \scarrot, \sgrapes, \stea }$ is a ``thing'', and these three things happen to be assembled together in the set \cref{eq:set-with-sets-as-elements}.
Based on our conventions (curly brackets specify sets), both~$\makeset{ \sgrapes }$ and~$\makeset{ \sapple, \scarrot, \sgrapes, \stea }$ are sets (and for our purposes, we do not need to say rigorously what kind of thing~$\sapple$ might be).

From the above discussion it is hopefully now clear that, given sets~\setA,~\setB, and~\setC, say, we can build for example a new set
\begin{equation}
    \makeset{ \setA, \setB, \setC }
\end{equation}
whose elements are~\setA,~\setB, and~\setC.
It is hopefully also clear for example that~$\Emptyset$ and~$\makeset{ \Emptyset }$ are different sets (this is sometimes a confusing case!).
The former is the empty set (it has no elements), while the latter is a set which has precisely one element, and that element happens to be the set~$\Emptyset$.

\subsection{Equality}

Given two sets, we say they are equal if and only if ``they have the same elements''.
For example, we have seen that~$\makeset{\sapple, \scarrot, \sgrapes, \stea }$ and~$\makeset{ \stea, \sgrapes, \sapple, \scarrot}$ and~$\makeset{ \sapple, \scarrot, \stea, \sgrapes }$ are all equal as sets, because their elements are the same.
In particular, this example shows that a given set might have many names or symbolic representations.

From one perspective, the criterion for knowing when sets are equal reduces to knowing when elements (or ``things'') are equal.
If~\setA and~\setB are sets, then to check if~$\setA = \setB$, we need to check if respective elements of~\setA and~\setB are equal.

From another perspective, the question of equality of sets can be expressed in terms of membership:~\setA and~\setB are equal if and only if the statements~$\ela \setin\setA$ and~$\ela \setin\setB$ are logically equivalent ($\ela$ is a variable that can be instantiated with the elements of~\setA or~\setB):
%
\begin{equation}
    \prfdoubleperiod{
        \quad \setA = \setB \quad
    }{
        \prfdouble{
            \ela \setin\setA
        }{
            \ela \setin\setB
        }
    }
\end{equation}
Note that there are two levels of double lines.
Because the first set of double lines is wider, we read the statement as stating an equivalence between $\setA = \setB$ and $\prfdouble{
        \ela \setin\setA
    }{
        \ela \setin\setB
    }$.

%When working with sets, we think of ``equality'' as a fundamental relation (or property) that any two given sets (or ``things'') can have: they can either be equal (identical) or unequal (distinct). Sometimes very important theorems in mathematics consist of proving that some set defined in one way is in fact equal (or inequal) to some other set which is defined in some other way!

\subsection{Subsets}
\label{sec:set-inclusion}
Consider the set~$\makeset{ \sapple, \scarrot, \sgrapes }$, and compare it with~$\makeset{ \sapple, \scarrot, \sgrapes, \stea }$.
Each of the elements of the first set is also an element of the second set;
in such a case we say that the first set is a \emph{subset} of or is \emph{included} in the second set.
In symbols,
%
\begin{equation}
    \label{eq:subset}
    \makeset{ \sapple, \scarrot, \sgrapes } \setsubseteq \makeset{ \sapple, \scarrot, \sgrapes, \stea }.
\end{equation}
%
Generally, given sets~\setA and~\setB, the statement~$\setA \setsubseteq \setB$ (that~\setA is a subset of~\setB) is logically equivalent to the statement
%
\begin{equation}
    \prf{
        \ela \setin\setA
    }{
        \ela \setin\setB
    }
\end{equation}
in sequent form and equivalent to the statement
\begin{equation}
    \label{eq:inclusion-via-quantifier}
    \forall \ela \setin\setA \colon \ela \setin\setB
\end{equation}
in terms of the ``for all'' universal quantifier.

Returning to \cref{eq:subset}, the second set is, on the other hand, \emph{not} included in the first set, since~$\stea$ is an element of the second set, but not the first.
If we say a set~\setA is ``strictly included'' in another set~\setB, then we mean ``included in and not equal''; the notation for this is~$\setA \setsubset \setB$.

The notation for inclusion and strict inclusion of sets is analogous to the notation in the context of numbers for ``less than or equals'',~$\ela \leq \elb$, and ``strictly less than'',~$\ela < \elb$, respectively.

In general, if we do not use the adjective ``strictly'', then ``inclusion'' means that equality is also possible.
In particular, in our terminology it is true that any set~\setA is included in itself:~$\setA \setsubseteq \setA$.

Inclusion and equality are related as follows: given sets~\setA and~\setB,
%
\begin{equation}
    \prfdoubleperiod
    {\setA \setsubseteq \setB
        \quad
        \setB \setsubseteq \setA}{
        \setA = \setB}
\end{equation}
%
Many times, in order to prove a statement of the form~$\setA = \setB$ it is a useful strategy to prove the two statements~$\setA \setsubseteq \setB$ and~$\setB \setsubseteq \setA$ each.

With respect to inclusion of sets, the empty set~$\Emptyset$ has (once again) some slightly tricky behavior:~$\Emptyset$ is a subset of any other set.
To see why this makes sense, consider the formulation \cref{eq:inclusion-via-quantifier}: when~\setA is the empty set, this statement is always true, since quantifying ``for all'' over the empty set poses no condition at all.

\subsection{Specifying a set via a logical statement}

In addition to the ``naming the elements'' way of specifying a set, many times sets are specified with the help of a logical ``statement'' or ``sentence'' which serves as a condition which characterizes its elements.

The idea is this: we start out with some given set~\setB, and then we consider a statement~$S(\ela)$ which depends on a variable~$\ela$, which we think of as running over the elements of~\setB.
We can then ask: for which elements~$\ela$ of~\setB is the statement~$S(\ela)$ true?
These elements form a subset of~\setB, often denoted
%
\begin{equation}
    \label{eq:axiom-specialization}
    \makeset{ \ela \setin\setB \mid S(\ela) }.
\end{equation}
%
For example, let~$\setB = \makeset{ \sapple, \sgrapes, \scarrot}$ and consider the statement
%
\begin{equation}
    S(\ela) = \text{``$\ela$ is a fruit''}.
\end{equation}
%
Then we can form the set
%
\begin{equation}
    \makeset{ \ela \setin\setB \mid \ela \text{ is a fruit} } = \makeset{ \sapple, \sgrapes }.
\end{equation}

There is an interesting special case of this way of constructing subsets of a set~\setB:
what if, for a given statement~$S(\ela)$, \emph{none} of the elements~$\ela \setin\setB$ are such that~$S(\ela)$ is true?
Then the result is the empty set.

As an example, consider the statement
%
\begin{equation}
    S(\ela) =\text{ ``} \ela \text{ is the name of a planet''}.
\end{equation}
%
Then, for the set~$\setB = \makeset{ \sapple, \sgrapes, \scarrot }$, the set defined as
%
\begin{equation}
    \makeset{ \ela \setin\setB \mid \ela \text{ is the name of a planet} }
\end{equation}
is equal to the empty set.

\subsection{Logical statements quantified over a set}

The above describes how to define a set using a logical statement.
Often times we also conversely use a set to formulate a logical statement.
For example a statement of the kind ``there exists~$\ela \setin \setA$, such that the statement~$P(\ela)$ is true''.
Our notation for this will be
%
\begin{equation}
    \label{eq:exists-over-set}
    \exists \ela \setin \setA \colon P(\ela).
\end{equation}
%
Similarly,
\begin{equation}
    \label{eq:forall-over-set}
    \forall \ela \setin \setA \colon P(\ela).
\end{equation}
denotes the statement ``for all~$\ela \setin \setA$,~$P(\ela)$ is true''.

\begin{remark}[Do you want to be more formal?]
    The statements \cref{eq:exists-over-set} and \cref{eq:forall-over-set} can be formulated in the formats ``$\exists \ela \ Q(\ela)$'' and ``$\forall \ela \ Q(\ela)$'', respectively, that were introduced in \cref{sec:logical-prelim} for the logical symbols~$\exists$ and~$\forall$:
    \begin{equation}
        \exists \ela \  ((\ela \setin \setA) \booland P(\ela)),
    \end{equation}
    and
    \begin{equation}
        \forall \ela\  ((\ela\setin \setA) \Imp P(\ela)).
    \end{equation}
    However, \cref{eq:exists-over-set} and \cref{eq:forall-over-set} are easier to read and are common usage.
\end{remark}

\subsection{Special sets we'll often use}

\phantomsection
\label{sec:natnumbers}
\label{sec:wnumbers}
\label{sec:ratnumbers}
\label{sec:reals}

There are some sets from basic mathematics that we are very familiar with, such as
the set~\natnumbers of natural numbers\footnote{
    Whether zero should be included in the definition of the natural numbers is a question of convention, and there is no clear universal agreement on this.
    We choose to follow the ISO 80000-2 standard~\cite{ISO:2009:IQU} that includes zero as part of the natural numbers.
}
%
\begin{equation}
    \natnumbers = \makeset{0, 1, 2, 3, 4, \ldots },
\end{equation}
%
the set~$\wnumbers$ of integers (or ``whole numbers'')
\begin{equation}
    \wnumbers = \makeset{ 0, +1, -1, +2, -2, +3, -3, \ldots},
\end{equation}
%
or the rational numbers~$\ratnumbers$, real numbers~\reals, and complex numbers $\cnumbers$.
In formal set theory, these sets can actually be quite a nuisance to define rigorously; for our purposes this is unnecessary, and we will just work with these sets as usual.

\paragraph{Singleton sets}
\phantomsection\label{def:singleton}
In many situations in category theory, it often doesn't matter which of the (infinitely-many) one-element sets we are considering; however, it is at the same time often useful to be referring to a specific, explicitly specified one-element set, in oder to be able to ``operate'' with it directly. For this reason it is useful to define a ``standard, default'' \maindef{singleton set} $\singleton$ containing only one element:
\begin{equation}
    \singleton \definedas \makeset{\singletonel}.
\end{equation}
It does not matter which symbol we have chosen as the single element of our default one-element set; we have chosen the symbol ``$\singletonel$'' simply because it is a fairly ``neutral-looking" symbol and one that is not used often to denote things with another mathematical meaning (however we could have instead considered, for example, the singleton sets $\makeset{\star}$, or $\makeset{\heartsuit}$, etc. ). Whenever we use the singleton set $\singleton$, all that we care about it is that it contains exactly one element.

\todotext{JL: write up def of standard two element set?}

\paragraph{The set of booleans}
\phantomsection\label{def:booleans}
The \maindef{set of booleans} is defined as:
\begin{equation}
    \boolset = \makeset{ \false, \true },
\end{equation}
where $\false$ is ``false'' and $\true$ is ``true''.
