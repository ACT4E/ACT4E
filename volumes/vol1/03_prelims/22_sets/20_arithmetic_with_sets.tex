% !TEX root = chapter-standalone.tex
\section{Arithmetic with sets}
\label{sec:arithmetic-with-sets}


In our notation for \SY{cartesian product}, \SY{disjoint union}, and function sets we have used notation inspired by basic operations in arithmetic, motivated in part by the following formulas for sizes of finite sets:
\begin{align}
    \cardof{\setA \cartprod \setB}     & = \cardof \setA \cdot \cardof \setB, \\
    \cardof{ \setA \setdisunion \setB} & = \cardof \setA + \cardof \setB, \\
    \cardmap(\setB^{\setA})            & = {\cardof\setB}^{\cardof\setA}.
\end{align}

The parallels of these operations to operations in arithmetic go further.
For example, consider the following identities which hold for any natural numbers $x, y, z$: 
\begin{align}
    x \cdot y & = y \cdot x,  \\ 
    x + y & = y + x, \\
    x \cdot (y \cdot z) & = (x \cdot y) \cdot z, \\
    x + (y + z) & = (x + y) + z, \\
    (x \cdot y) + (x \cdot z)  & = x \cdot (y + z), \\
    z^{x \cdot y}                    & = \pars{z^y}^x, \\
    (x \cdot y)^z &= x^z \cdot y^z, \\
    x^{(y + z)}           & = x^y \cdot x^z, \\
    (x + y)^z &= \sum_{k = 0}^z {{z}\choose{k}} \ x^k \cdot y^{z - k}. 
\end{align}
These identities also hold on the level of sets, before computing their size; we simply need to replace ``$=$'' with the symbol ``$\isomorphic$'' for ``isomorphic'':
\begin{align}
    \setA \cartprod \setB & \isomorphic \setB \cartprod \setA,  \\
    \setB \setdisunion \setA & \isomorphic \setA \setdisunion \setB,   \\
    (\setA \cartprod \setB) \cartprod \setC & \isomorphic \setA \cartprod (\setB \cartprod \setC), \\
    (\setA \setdisunion \setB) \setdisunion \setC & \isomorphic \setA \setdisunion (\setB \setdisunion \setC), \\
    (\setA \cartprod \setB) \setdisunion (\setA \cartprod \setC)  & \isomorphic \setA \cartprod (\setB \setdisunion \setC), \\
    \setC^{\setA \cartprod \setB}                    & \isomorphic \pars{\setC^\setB}^\setA, \\
    (\setA \cartprod \setB)^{\setC} & \isomorphic  \setA^\setC \cartprod \setB^\setC, \\
    \setA^{\setB \setdisunion \setC}           & \isomorphic \setA^\setB \cartprod \setA^\setC, \\
    (\setA \setdisunion \setB)^{\setC} & \isomorphic \stylesets{\sum_{\subA \setin \powerset \setC}} \setA^\subA \cartprod \setB^{\setC \backslash \subA} 
\end{align}

\todotext{JL: also talk about empty set and one-element sets acting like zero and one... }

We can say even more: not only do we have the above statements about certain sets being isomorphic, but in fact in each case there exists a particular, special isomorphism which mathematicians might call "natural" or "canonical". These terms are often used in an informal way to describe situations where some mathematical structure, such as a certain function, exists without the need for any particular ``extra'' or ``ad hoc'' choices to be made. Often this situation involves a family of cases, and a ``canonical choice'' is one which is possible to construct for all cases at once, without reference to the details of any particular case. We'll illustrate this idea by describing a canonical isomorphism for each of the ``equations'' above. We also do this we will use many of these ismorphisms over and over again, and hence it makes sense to get to know them and give them names. 


\subsection{The empty set is like the number zero}

\todotext{@JL: write up}

\subsection{Singletons are like the number one}

\todotext{@JL: write up}

\subsection{Commutativity}
   
The equations $x \cdot y = y \cdot x$ and $x + y = y + x$ describe the \emph{commutativity property} of multiplication and of addition, respectively. 

Let us look at the corresponding relationship on the level of sets, for the case of multiplication:
\begin{equation}
\setA \cartprod \setB \simeq \setB \cartprod \setA. 
\end{equation}
The canonical isomorphism in this case, which gives proof for this relationship, is
 \begin{equation}
 \defmapperiod{
           \braiding
        }{
            \setA \cartprod \setB
        }{
            \to
        }{
            \setB \cartprod \setA
        }{
            \tup{\ela, \elb}
        }{
            \tup{\elb, \ela}
        }
\end{equation}
This function (which we've called $\braiding$ which is short for "braiding"), is canonical because the "shape" or "recipe" of the function does not depend on the particular sets $\setA$ and $\setB$; this "shape" or "recipe" works for $\emph{any}$ two sets $\setA$ and $\setB$. It is in this sense that it works for "all cases at once" and does not depend on any extra choices that depend on particular sets.

In contrast, for example, if we were to look at the particular sets $\setA = \makeset{0, 1, 2}$ and $\setB = \makeset{\star, \dagger}$, then, besides the canonical function $\braiding \colon \setA \cartprod \setB \to \setB \cartprod \setA$, there are also other isomorphisms possible, such as for instance the function $\setA \cartprod \setB \to \setB \cartprod \setA$ which maps like this: 
\begin{align}
\tup{0, \star} \mapsto \tup{\star, 1} \\
\tup{1, \star} \mapsto \tup{\star, 2} \\
\tup{2, \star} \mapsto \tup{\star, 3} \\
\tup{0, \dagger} \mapsto \tup{\dagger, 1} \\
\tup{1, \dagger} \mapsto \tup{\dagger, 2} \\
\tup{2, \dagger} \mapsto \tup{\dagger, 3}.
\end{align}
It may be written more compactly by the recipe ``$\tup{x, y} \mapsto \tup{y, x +1 \text{ mod } 3}$". However, this recipe still depends on specific features of the  set $\setA$ and would not work for any two arbitrary sets (namely, it uses the fact that the elements of $\setA$ in this example are integers for which the operation ``mod 3'' makes sense). Thus this function is not ``canonical" (while $\braiding$ is). 

\todotext{J: @J: maybe explain the term ''canonical'' now more above, in the part on empty set and singletons...}

The above discussion regards multiplication. A similar situation holds for addition. The statement $\setA \setdisunion \setB \isomorphic \setB \setdisunion \setA$ is the analogue, on the level of sets, of the commutativity property for the operation of addition for natural numbers, and for this relationship there is also proof via a canonical isomorphism. We call this canonical isomorphism also $\braiding$. 

\begin{exercise}\label{ex:braiding-disjoint-union}
Can you guess the canonical isomorphism $\braiding \colon \setB \setdisunion \setA \to \setA \setdisunion \setB$? Prove that your guess is indeed an isomorphism. 
\end{exercise}

\begin{solution}
\todotext{JL: write up}
\end{solution}

\subsection{Associativity}

The equations $x \cdot (y \cdot z)  = (x \cdot y) \cdot z$ and $x + (y + z) = (x + y) + z$ describe, respectively, the \emph{associativity} property of multiplication and addition of natural numbers. Correspondingly, there are canonical isomorphisms
\begin{equation}
\associator \colon (\setA \cartprod \setB) \cartprod \setC \to \setA \cartprod (\setB \cartprod \setC)
\end{equation}
and 
\begin{equation}
\associator \colon (\setA \setdisunion \setB) \setdisunion \setC \to \setA \setdisunion (\setB \setdisunion \setC)
\end{equation}
(which we both call by the same name). For the case of the cartesian product, the isomorphism $\associator$ is
\begin{equation}
 \defmapperiod{
           \associator
        }{
            (\setA \cartprod \setB) \cartprod \setC 
        }{
            \to
        }{
           \setA \cartprod (\setB \cartprod \setC)
        }{
            \tup{\tup{\ela, \elb}, \elc}
        }{
            \tup{\ela, \tup{\elb, \elc}}
        }
\end{equation}


\begin{exercise}\label{ex:associator-cart-prod-and-disjoint-union}
Guess how $\associator$ is defined for the sum (disjoint union) of sets, and check that your guess is indeed an isomorphism.
\end{exercise}

\begin{solution}
\todotext{JL: write up}
\end{solution}

\subsection{Distributivity}

Consider the statement above that 
\begin{equation}
(\setA \cartprod \setB) \setdisunion (\setA \cartprod \setC)  \isomorphic \setA \cartprod (\setB \setdisunion \setC).
\end{equation}
This is analogous to the \emph{distributivity} property of multiplication and addition of numbers:
\begin{equation}
(x \cdot y) + (x \cdot z)  = x \cdot (y + z).
\end{equation}

\begin{exercise}\label{ex:distributivity-isomorphism}
Guess the canonical isomorphism $\distributor \colon (\setA \cartprod \setB) \setdisunion (\setA \cartprod \setC) \to \setA \cartprod (\setB \setdisunion \setC)$ and verify that it is indeed an isomorphism. 
\end{exercise}

\begin{solution}
\todotext{JL: write up}
\end{solution}

\subsection{Functions out of products}

Next we look at the equation $z^{x \cdot y} = (z^y)^x$, and the corresponding relationship $ \setC^{\setA \cartprod \setB}  \isomorphic \pars{\setC^\setB}^\setA$. What might be a canonical isomorphism proving this latter relationship?

The elements of the set $ \setC^{\setA \cartprod \setB}$ are functions of two variables, of the type
\begin{equation}\label{eq:currying-function-of-two-variables}
\mora \colon \setA \cartprod \setB \to \setC, 
\end{equation}
while elements of the set $\pars{\setC^\setB}^\setA$ are functions of the type
\begin{equation} \label{eq:currying-function-to-functions}
\morb \colon \setA \to \setC^\setB;
\end{equation}
in other words, functions of a single variable but which evaluate, for each input, to a function of the type $\setB \to \setC$. A canonical way to turn a function of the type \cref{eq:currying-function-of-two-variables} into a function of the type \cref{eq:currying-function-to-functions} is by ``partial evaluation'': if we think of $\mora$ as having two input slots, we can partially evaluate $\mora$ by inserting an element of $\setA$ into the first slot of $\mora$, while leaving the second slot ``open'' or ``variable''. In other words, for any element $\ela \setin \setA$, we can create, from $\mora$, the function 
\begin{equation}
 \defmapperiod{
           \mora(\ela, -)
        }{
            \setB 
        }{
            \to
        }{
           \setC 
        }{
            \elb
        }{
            \mora (\ela, \elb)
        }
\end{equation}
But since we can do this for any $\ela \in \setA$, we have also just described a function
\begin{equation}
\begin{aligned}
          & \setA & \to & \ \ \setC^\setB, \\
         & \ela & \mapsto & \ \  \mora ( \ela, -),
\end{aligned}
\end{equation}
or, in other words, an element of $\pars{\setC^\setB}^\setA$. Since this recipe works for any function $\mora \colon \setA \cartprod \setB \to \setC$, we also now have the following function
\begin{equation}
\defmapcomma{
         \curry
        }{
           \setC^{\setA \cartprod \setB}
        }{
           \to
        }{
           \pars{\setC^\setB}^\setA 
        }{
           \mora
        }{
           (\ela \mapsto \mora ( \ela, -))
           }
\end{equation}
which we call $\curry$. This function has the desired type to be an isomorphism proving that $\setC^{\setA \cartprod \setB}  \isomorphic \pars{\setC^\setB}^\setA$.

\begin{exercise}\label{ex:currying-is-an-iso}
Prove that  $\curry \colon \setC^{\setA \cartprod \setB} \to \pars{\setC^\setB}^\setA$ is an isomorphism. 
\end{exercise}

\begin{solution}
\todotext{JL: write up}
\end{solution}

The name``$\curry$'' stands for ``Curry'' (and the operation of applying it is called ``currying"). Curry is the last name of the mathematician and logician Haskell Curry (\cref{fig:haskell-curry}) who did not discover this operation, but likely made it well-known. It is after his first name that the functional programming language Haskell is named. 

\begin{marginfigure}
    \includegraphics[width=3cm]{haskell}
    \caption{Haskell Curry (1900-1982)}
    \label{fig:haskell-curry}
\end{marginfigure}

\begin{remark}
In our discussion of currying above, we started with a function $\mora \colon \setB \cartprod \setA \to \setC$ and partially evaluated it in the first variable, and left the second variable still ``open''. Alternatively, though, we could have partially evaluated $\mora$ in the \emph{second} variable, and left the first variable open, thus producing a function $\mora(-, \elb) \colon \setA \to \setC$ and leading, in a way analogous to the above, to an alternative currying operation 
\begin{equation}
\setC^{\setA \cartprod \setB} \to \pars{\setC^\setA}^\setB.
\end{equation}
In order to distinguish this operation from the currying operation previously defined, we might call this ``co-currying'', or make a distinction between ``left'' and ``right'' currying. Often, mathematicians and computer scientists follow the convention that currying is always done by partially evaluating in the first variable. This is perhaps justified by the fact that the ``co-currying'' isomorphism
\begin{equation}
\setC^{\setA \cartprod \setB} \to \pars{\setC^\setA}^\setB
\end{equation}
arising from partial evaluation in the second variable can be expressed as a composite
\begin{equation}
\setC^{\setA \cartprod \setB} \overset{{\setC^{\braiding}}}{\longrightarrow} \setC^{\setB \cartprod \setA} \overset{\curry}{\longrightarrow} \pars{\setC^\setA}^\setB
\end{equation}
of the exponentiation of $\braiding \colon \setB \cartprod \setA \to \setA \cartprod \setB$ composed with $\curry \colon \setC^{\setB \cartprod \setA} \to \pars{\setC^\setA}^\setB$.  
\end{remark}

\subsection{Functions into products}

Now let's consider the analogue of the equation
\begin{equation}
(x \cdot y)^z = x^z \cdot y^z
\end{equation}
for natural numbers, namely
\begin{equation}
 (\setA \cartprod \setB)^{\setC}  \isomorphic  \setA^\setC \cartprod \setB^\setC.
\end{equation}
In oder to describe a canonical isomorphism which is a proof of this statement, we make an important observation about the cartesian product of sets. Namely, for any sets $\setA$ and $\setB$, we always have two ``projection functions'' from the cartesian product $\setA \cartprod \setB$ to $\setA$ and $\setB$, respectively. These are defined, respectively, by 
\begin{equation}
\begin{aligned}
    \proj_1 \colon   & \setA \cartprod \setB & \to & \ \ \setA, \\
         & \tup{\ela, \elb} & \mapsto & \ \  \ela,
\end{aligned}
\end{equation}
and
\begin{equation}
\begin{aligned}
    \proj_2 \colon   & \setA \cartprod \setB & \to & \ \ \setB, \\
         & \tup{\ela, \elb} & \mapsto & \ \  \elb,
\end{aligned}
\end{equation}
and are called the \emph{canonical projections} associated to the cartesian product. 

With the help of these projections, we define a canonical isomorphism 
\begin{equation}
\begin{aligned}
    \mora \colon   & (\setA \cartprod \setB)^{\setC}  & \to & \ \  \setA^\setC \cartprod \setB^\setC, \\
         & \phi & \mapsto & \ \  \tup{\phi \mthen \proj_1, \phi \mthen \proj_2}.
\end{aligned}
\end{equation}
Illustrated diagrammatically, what $\mora$ does is, given a function 
\begin{equation}
\begin{tikzcd}
& \setC \arrow[d, "\phi"'] & \\
& \setA \cartprod \setB &
\end{tikzcd}
\end{equation}
it maps it to the pair of functions $\phi \mthen \proj_1$ and $\phi \mthen \proj_2$
\begin{equation}
\begin{tikzcd}
& \setC \arrow[dl, "\phi \mthen \proj_1"']   \arrow[d, "\phi"'] \arrow[dr, "\phi \mthen \proj_2"] & \\
\setA & \setA \cartprod \setB  \arrow[l, "\proj_1"] \arrow[r, "\proj_2"'] & \setB
\end{tikzcd}
\end{equation}
obtained by post-composing $\phi$ with the projections $\proj_1$ and $\proj_2$, respectively. 

The inverse to $\mora$ is the function $\morb \colon  \setA^\setC \cartprod \setB^\setC  \to (\setA \cartprod \setB)^{\setC}$ which takes a pair of functions $\tup{\psi_1, \psi _2} \setin \setA^\setC \cartprod \setB^\setC$ and maps it to the element of $(\setA \cartprod \setB)^{\setC}$ which is the function 
\begin{equation}
\begin{aligned}
   & \setC   & \to & \ \ \setA \cartprod \setB, \\
         & \elc & \mapsto & \ \ \tup{\psi_1(\elc), \psi_2(\elc)}.
\end{aligned}
\end{equation}

\begin{gradedexercise}[\exname{CanIsoFunctionsIntoProducts}]
Check that f and g are indeed mutually inverse. 
\end{gradedexercise}

\solutionof{CanIsoFunctionsIntoProducts}

\subsection{Functions out of sums}

Here we consider the equation
\begin{equation}
x^{(y + z)}           = x^y \cdot x^z
\end{equation}
and it's analogue on the level of sets, 
\begin{equation}
\setA^{\setB \setdisunion \setC}           \isomorphic \setA^\setB \cartprod \setA^\setC.
\end{equation}

Similar to the observation that cartesian products come along with canonical projection functions, we make the  observation here that sums of sets come equipped with canonical \emph{inclusion functions}
\begin{equation}
\begin{aligned}
    \inj_1 \colon   & \setB  & \to & \ \ \setB \setdisunion \setC \\
         & \elb & \mapsto & \ \  \tup{1, \elb},
\end{aligned}
\end{equation}
and
\begin{equation}
\begin{aligned}
    \inj_2 \colon   & \setC & \to & \ \ \setB \setdisunion \setC, \\
         & \elc & \mapsto & \ \  \tup{2, \elc}.
\end{aligned}
\end{equation}

Using these, we construct the following canonical isomorphism:
\begin{equation}
\begin{aligned}
    \mora \colon   & \setA^{\setB \setdisunion \setC} & \to & \ \ \setA^\setB \cartprod \setA^\setC, \\
         & \phi & \mapsto & \ \  \tup{\inj_1 \mthen \phi, \inj_2 \mthen \phi}.
\end{aligned}
\end{equation}
Illustrated diagrammatically, $\mora$ takes any function 
\begin{equation}
\begin{tikzcd}
&  \setA  & \\
& \setB \setdisunion \setC \arrow[u, "\phi"] &
\end{tikzcd}
\end{equation}
it maps it to the pair of functions $\inj_1 \mthen \phi$ and $\inj_2 \mthen \phi$
\begin{equation}
\begin{tikzcd}
& \setC    & \\
\setB \arrow[ur, "\inj_1 \mthen \phi"] \arrow[r, "\inj_1"'] & \setB \setdisunion \setC \arrow[u, "\phi"]     & \setC \arrow[l, "\inj_2"] \arrow[ul, "\inj_2 \mthen \phi"']
\end{tikzcd}
\end{equation}
obtained by pre-composing $\phi$ with the inclusions $\inj_1$ and $\inj_2$, respectively. 

The inverse to $\mora$ is the function $\mapb \colon \setA^\setB \cartprod \setA^\setC \to \setA^{\setB \setdisunion \setC}$ which maps any pair of functions $\tup{\psi_1, \psi _2} \setin \setA^\setB \cartprod \setA^\setC$ to the element of $\setA^{\setB \setdisunion \setC}$ given by the function
\begin{equation}
\begin{aligned}
   & \setB \setdisunion \setC & \to & \ \ \setA , \\
         & \eld & \mapsto & \begin{cases} \psi_1(\elb) \text{ if } \eld = \tup{1,\elb} \text{ for some } \elb \in \setB, \\
          \psi_2(\elc)\text{ if } \eld = \tup{2, \elc}  \text{ for some } \elc \in \setC. \end{cases}
\end{aligned}
\end{equation}

\begin{gradedexercise}[\exname{CanIsoFunctionsOutOfSums}]
Check that f and g are indeed mutually inverse. 
\end{gradedexercise}

\solutionof{CanIsoFunctionsOutOfSums}


\subsection{Functions into sums}

The equation 
\begin{equation}
(x + y)^z = \sum_{k = 0}^z {{z}\choose{k}} \ x^k \cdot y^{z - k}
\end{equation}
is known as the binomial theorem, and the numbers
\begin{equation}
{{z}\choose{k}} = \frac{z!}{(z-k)! \ k!}
\end{equation}
are called the binomial coefficients. On the level of sets and functions, the binomial theorem is analogous to the statement
\begin{equation}
(\setA \setdisunion \setB)^{\setC}  \isomorphic \stylesets{\sum_{\subA \setin \powerset \setC}} \setA^\subA \cartprod \setB^{\setC \backslash \subA}.
\end{equation}
The binomial coefficients are related to combinatorics; the number
\begin{equation}
{{z}\choose{k}}
\end{equation}
describes the number of different ways that one may choose a $k$-element subset out of a fixed $z$-element set. Using this connection as a guide, we can re-write the right-hand side of our statement about sets and functions into a form that is closer to the right-hand side of the binomial theorem: 
\begin{equation}
\stylesets{\sum_{\subA \setin \powerset \setC}} \setA^\subA \cartprod \setB^{\setC \backslash \subA} \simeq \sum_{k = 0}^{\vert \setC \vert} \ \underset{\substack{\subA \setin \powerset \setC \\ \vert \subA \vert = k}}{\stylesets{\sum}} \setA^\subA \cartprod \setB^{\setC \backslash \subA}.
\end{equation}
Now, on the right-hand side, for each fixed $k$, the cardinality of the set 
\begin{equation}
\underset{\substack{\subA \setin \powerset \setC \\ \vert \subA \vert = k}}{\stylesets{\sum}} \setA^\subA \cartprod \setB^{\setC \backslash \subA}
\end{equation}
is precisely 
\begin{equation}
{{z}\choose{k}} \ x^k \cdot y^{z - k},
\end{equation}
with $z = \vert \setC \vert$. 

Next, let's show that there is a canonical isomorphism
\begin{equation}
\mora \colon (\setA \setdisunion \setB)^{\setC}  \to \stylesets{\sum_{\subA \setin \powerset \setC}} \setA^\subA \cartprod \setB^{\setC \backslash \subA}.
\end{equation}
Unpacking the definition of unordered sums of sets, the target set of this function is
\begin{equation}
\bigsetunion_{\subA \setin  \powerset \setC} \makeset{\subA} \cartprod (\setA^{\subA} \cartprod \setB^{\setC \backslash \subA} ) 
\end{equation}
To define $\mora$, we begin by choosing an arbitrary $\phi \setin (\setA \setdisunion \setB)^{\setC}$; that is, a function 
\begin{equation}
\phi \colon \setC \to \setA \setdisunion \setB.
\end{equation}
Because $\setA \setdisunion \setB$ is the \emph{disjoint} union of $\makeset{1} \cartprod \setA$ and $\makeset{2} \cartprod \setB$, their respective pre-images under $\phi$ have empty intersection, and their union is all of $\setC$. In other words, if we set 
\begin{equation}
\subB \definedas \phi^{-1}(\makeset{1} \cartprod \setA),
\end{equation}
then
\begin{equation}
\phi^{-1}(\makeset{2} \cartprod \setB)  = \setC \backslash \subB. 
\end{equation}
The image $\mora(\phi)$ should be an element of $\makeset{\subA} \cartprod (\setA^{\subA} \cartprod \setB^{\setC \backslash \subA} )$ for some $\subA \setin \powerset \setC$. We will use $\subA = \subB$. What remains for us to do, then, is to define an element of $\setA^{\subB}$ and an element of $\setB^{\setC \backslash \subB}$, using $\phi$. To obtain an element of $\setA^{\subB}$, we use the composite
\begin{equation}
\subB \overset{\phi \vert_\subB}{\to} \makeset{1} \cartprod \setA \to \setA, 
\end{equation}
and to obtain an element of $\setB^{\setC \backslash \subB}$, we use
\begin{equation}
\setC \backslash \subB \overset{\phi \vert_{\setC \backslash \subB}}{\to} \makeset{2} \cartprod \setB \to \setB. 
\end{equation}

The intuition behind this definition of $\mora$ is that any function $\phi \colon \setC \to \setA \setdisunion \setB
$ essentially amounts to two functions ``glued together'': one, $\phi \vert_\subB$, which accounts for all the elements of $\setC$ that are mapped by $\phi$ to $\makeset{1} \cartprod \setA$, and another, $\phi \vert_{\setC \backslash \subB}$, which accounts for all the elements of $\setC$ that are mapped by $\phi$ to $\makeset{2} \cartprod \setB$. 

This intuition can also help us define an inverse
\begin{equation}
\morb \colon \stylesets{\sum_{\subA \setin \powerset \setC}} \setA^\subA \cartprod \setB^{\setC \backslash \subA} \to (\setA \setdisunion \setB)^{\setC}   
\end{equation}
to $\mora$. For this, fix an $\subA \setin \powerset \setC$. Given an element 
\begin{equation}
\tup{\subA, \psi_1, \psi_2} \setin \makeset{\subA} \cartprod (\setA^{\subA} \cartprod \setB^{\setC \backslash \subA} ), 
\end{equation}
we define $\morb(\tup{\subA, \psi_1, \psi_2})$ to be the function
\begin{equation}
\begin{aligned}
 & \setC & \to & \ \setA \setdisunion \setB \\
 & \elc & \mapsto & \begin{cases} \  \tup{1, \psi_1(\elc)} \text{ if } \elc \in \subA, \\
          \ \tup{2, \psi_2(\elc)} \text{ if } \elc \in \setC \backslash \subA. \end{cases}
\end{aligned}
\end{equation}
Roughly speaking, $\morb(\tup{\subA, \psi_1, \psi_2}) \colon \setC \to \setA \setdisunion \setB$ is just the ``gluing together" of $\psi_1$ and $\psi_2$. 

\begin{gradedexercise}[\exname{CanIsoFunctionsIntoSums}]
Check that f and g are indeed mutually inverse. 
\end{gradedexercise}

\solutionof{CanIsoFunctionsIntoSums}

