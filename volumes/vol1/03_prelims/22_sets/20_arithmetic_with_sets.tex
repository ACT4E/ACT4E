% !TEX root = chapter-standalone.tex
\section{Arithmetic with sets}
\label{sec:arithmetic-with-sets}


In our notation for \SY{cartesian product}, \SY{disjoint union}, and function sets we have used notation inspired by basic operations in arithmetic, motivated in part by the following formulas for sizes of finite sets:
\begin{align}
    \cardof{\setA \cartprod \setB}     & = \cardof \setA \cdot \cardof \setB, \\
    \cardof{ \setA \setdisunion \setB} & = \cardof \setA + \cardof \setB, \\
    \cardmap(\setB^{\setA})            & = {\cardof\setB}^{\cardof\setA}.
\end{align}

The parallels of these operations to operations in arithmetic go further.
For example, consider the following identities which hold for any natural numbers $x, y, z$: 
\begin{align}
    x \cdot y & = y \cdot x,  \\ 
    x + y & = y + x, \\
    x \cdot (y \cdot z) & = (x \cdot y) \cdot z, \\
    x + (y + z) & = (x + y) + z, \\
    (x \cdot y) + (x \cdot z)  & = x \cdot (y + z), \\
    x^{y \cdot z}                    & = \pars{x^y}^z, \\
    (x \cdot y)^z &= x^z \cdot y^z, \\
    x^{(y + z)}           & = x^y \cdot x^z, \\
    (x + y)^z &= \sum_{i = 0}^z {{z}\choose{i}} \ x^i \cdot y^{z - i}. 
\end{align}
These identities also hold on the level of sets, before computing their size; we simply need to replace ``$=$'' with the symbol ``$\isomorphic$'' for ``isomorphic'':
\begin{align}
    \setA \cartprod \setB & \isomorphic \setB \cartprod \setA, \label{eq:commutativity-prod-sets} \\
    \setB \setdisunion \setA & \isomorphic \setA \setdisunion \setB, \label{eq:commutativity-sum-sets}  \\
    (\setA \cartprod \setB) \cartprod \setC & \isomorphic \setA \cartprod (\setB \cartprod \setC), \\
    (\setA \setdisunion \setB) \setdisunion \setC & \isomorphic \setA \setdisunion (\setB \setdisunion \setC), \\
    (\setA \cartprod \setB) \setdisunion (\setA \cartprod \setC)  & \isomorphic \setA \cartprod (\setB \setdisunion \setC), \\
    \setA^{\setB \cartprod \setC}                    & \isomorphic \pars{\setA^\setB}^\setC, \\
    (\setA \cartprod \setB)^{\setC} & \isomorphic  \setA^\setC \cartprod \setB^\setC, \\
    \setA^{\setB \setdisunion \setC}           & \isomorphic \setA^\setB \cartprod \setA^\setC, \\
    (\setA \setdisunion \setB)^{\setC} & \isomorphic \stylesets{\sum_{\subA \setin \powerset \setC}} \setA^\subA \cartprod \setB^{\compl{\subA}} \label{eq:binomial-theorem-sets}
\end{align}

\todotext{JL: also talk about empty set and one-element sets acting like zero and one... }

We can say even more: not only do we have the above statements about certain sets being isomorphic, but in fact in each case there exists a particular, special isomorphism which mathematicians might call "natural" or "canonical". These terms are often used in an informal way to describe situations where some mathematical structure, such as a certain function, exists without the need for any particular ``extra'' or ``ad hoc'' choices to be made. Often this situation involves a family of cases, and a ``canonical choice'' is one which is possible to construct for all cases at once, without reference to the details of any particular case. We'll illustrate this idea by describing a canonical isomorphism for each of the ``equations'' \cref{eq:commutativity-prod-sets} through \cref{eq:binomial-theorem-sets}. We also do this we will use many of these ismorphisms over and over again, and hence it makes sense to get to know them and give them names. 

\subsection{Commutativity}
   
The equations $x \cdot y = y \cdot x$ and $x + y = y + x$ describe the \emph{commutativity property} of multiplication and of addition, respectively. 

Let us look at the corresponding relationship on the level of sets, for the case of multiplication:
\begin{equation}
\setA \cartprod \setB \simeq \setB \cartprod \setA. 
\end{equation}
The canonical isomorphism in this case, which gives proof for this relationship, is
 \begin{equation}
 \defmapperiod{
           \braiding
        }{
            \setA \cartprod \setB
        }{
            \to
        }{
            \setB \cartprod \setA
        }{
            \tup{\ela, \elb}
        }{
            \tup{\elb, \ela}
        }
\end{equation}
This function (which we've called $\braiding$ which is short for "braiding"), is canonical because the "shape" or "recipe" of the function does not depend on the particular sets $\setA$ and $\setB$; this "shape" or "recipe" works for $\emph{any}$ two sets $\setA$ and $\setB$. It is in this sense that it works for "all cases at once" and does not depend on any extra choices that depend on particular sets.

In contrast, for example, if we were to look at the particular sets $\setA = \makeset{0, 1, 2}$ and $\setB = \makeset{\star, \dagger}$, then, besides the canonical function $\braiding \colon \setA \cartprod \setB \to \setB \cartprod \setA$, there are also other isomorphisms possible, such as for instance the function $\setA \cartprod \setB \to \setB \cartprod \setA$ which maps like this: 
\begin{align}
\tup{0, \star} \mapsto \tup{\star, 1} \\
\tup{1, \star} \mapsto \tup{\star, 2} \\
\tup{2, \star} \mapsto \tup{\star, 3} \\
\tup{0, \dagger} \mapsto \tup{\dagger, 1} \\
\tup{1, \dagger} \mapsto \tup{\dagger, 2} \\
\tup{2, \dagger} \mapsto \tup{\dagger, 3}.
\end{align}
It may be written more compactly by the recipe ``$\tup{x, y} \mapsto \tup{y, x +1 \text{ mod } 3}$". However, this recipe still depends on specific features of the  set $\setA$ and would not work for any two arbitrary sets (namely, it uses the fact that the elements of $\setA$ in this example are integers for which the operation ``mod 3'' makes sense). Thus this function is not ``canonical" (while $\braiding$ is). 

The above discussion regards multiplication. A similar situation holds for addition. The statement $\setA \setdisunion \setB \isomorphic \setB \setdisunion \setA$ is the analogue, on the level of sets, of the commutativity property for the operation of addition for natural numbers, and for this relationship there is also proof via a canonical isomorphism. We call this canonical isomorphism also $\braiding$. 

\begin{exercise}\label{ex:braiding-disjoint-union}
Can you guess the canonical isomorphism $\braiding \colon \setB \setdisunion \setA \to \setA \setdisunion \setB$? Prove that your guess is indeed an isomorphism. 
\end{exercise}

\begin{solution}
\todotext{JL: write up}
\end{solution}

\subsection{Associativity}

The equations $x \cdot (y \cdot z)  = (x \cdot y) \cdot z$ and $x + (y + z) = (x + y) + z$ describe, respectively, the \emph{associativity} property of multiplication and addition of natural numbers. Correspondingly, there are canonical isomorphisms
\begin{equation}
\associator \colon (\setA \cartprod \setB) \cartprod \setC \to \setA \cartprod (\setB \cartprod \setC)
\end{equation}
and 
\begin{equation}
\associator \colon (\setA \setdisunion \setB) \setdisunion \setC \to \setA \setdisunion (\setB \setdisunion \setC)
\end{equation}
(which we both call by the same name). For the case of the cartesian product, the isomorphism $\associator$ is
\begin{equation}
 \defmapperiod{
           \associator
        }{
            (\setA \cartprod \setB) \cartprod \setC 
        }{
            \to
        }{
           \setA \cartprod (\setB \cartprod \setC)
        }{
            \tup{\tup{\ela, \elb}, \elc}
        }{
            \tup{\ela, \tup{\elb, \elc}}
        }
\end{equation}


\begin{exercise}\label{ex:associator-cart-prod-and-disjoint-union}
Guess how $\associator$ is defined for the sum (disjoint union) of sets, and check that your guess is indeed an isomorphism.
\end{exercise}

\begin{solution}
\todotext{JL: write up}
\end{solution}

\subsection{Distributivity}

Consider the statement above that 
\begin{equation}
(\setA \cartprod \setB) \setdisunion (\setA \cartprod \setC)  \isomorphic \setA \cartprod (\setB \setdisunion \setC).
\end{equation}
This is analogous to the \emph{distributivity} property of multiplication and addition of numbers:
\begin{equation}
(x \cdot y) + (x \cdot z)  = x \cdot (y + z).
\end{equation}

\begin{exercise}\label{ex:distributivity-isomorphism}
Guess the canonical isomorphism $\distributor \colon (\setA \cartprod \setB) \setdisunion (\setA \cartprod \setC) \to \setA \cartprod (\setB \setdisunion \setC)$ and verify that it is indeed an isomorphism. 
\end{exercise}

\begin{solution}
\todotext{JL: write up}
\end{solution}

\subsection{Functions out of products}

Next we look at the equation $x^{y \cdot z} = (x^y)^z$, and the corresponding relationship $ \setA^{\setB \cartprod \setC}  \isomorphic \pars{\setA^\setB}^\setC$. What might be a canonical isomorphism proving this latter relationship?

The elements of the set $ \setA^{\setB \cartprod \setC}$ are functions of two variables, of the type
\begin{equation}\label{eq:currying-function-of-two-variables}
\mora \colon \setB \cartprod \setC \to \setA, 
\end{equation}
while elements of the set $\pars{\setA^\setB}^\setC$ are functions of the type
\begin{equation} \label{eq:currying-function-to-functions}
\morb \colon \setC \to \setA^\setB;
\end{equation}
in other words, functions of a single variable but which evaluate, for each input, to a function of the type $\setB \to \setA$. A canonical way to turn a function of the type \cref{eq:currying-function-of-two-variables} into a function of the type \cref{eq:currying-function-to-functoins} is by ``partial evaluation'': if we think of $\mora$ as having two input slots, we can partially evaluate $\mora$ by inserting an element of $\setC$ into the second slot of $\mora$, while leaving the first slot ``open'' or ``variable''. In other words, for any element $\elc \setin \setC$, we can create, from $\mora$, the function 
\begin{equation}
 \defmapperiod{
           \mora( -, \elc)
        }{
            \setB 
        }{
            \to
        }{
           \setA 
        }{
            \elb
        }{
            \mora (\elb, \elc)
        }
\end{equation}
But since we can do this for any $\elc \in \setC$, we have also just described a function
\begin{equation}
\begin{aligned}
          & \setC & \to & \ \ \setA^\setB, \\
         & \elc & \mapsto & \ \  \mora ( -, \elc),
\end{aligned}
\end{equation}
or, in other words, an element of $\pars{\setA^\setB}^\setC$. Since this recipe works for any function $\mora \colon \setB \cartprod \setC \to \setA$, we also now have the following function
\begin{equation}
\defmapcomma{
         \curry
        }{
           \setA^{\setB \cartprod \setC}
        }{
           \to
        }{
           \pars{\setA^\setB}^\setC 
        }{
           \mora
        }{
           (\elc \mapsto \mora ( -, \elc))
           }
\end{equation}
which we call $\curry$. This function has the desired type to be an isomorphism proving that $\setA^{\setB \cartprod \setC}  \isomorphic \pars{\setA^\setB}^\setC$.

\begin{exercise}\label{ex:currying-is-an-iso}
Prove that  $\curry \colon \setA^{\setB \cartprod \setC} \to \pars{\setA^\setB}^\setC$ is an isomorphism. 
\end{exercise}

The name ``$\curry$'' stands for ``Curry'', which is the last name of Haskell Curry. 

\begin{solution}
\todotext{JL: write up
\end{solution}


\subsection{Functions into products}


\subsection{Functions out of sums}


\subsection{Functions into sums}



\subsection{Arbitrary cartesian products and disjoint unions}

We have seen that, given sets~\setA and~\setB, we can form their \SY{cartesian product}~$\setA \cartprod \setB$ and disjoint union~$\setA \setdisunion \setB$.
And similarly for any finite number of sets.

What about if we start with a family~$\makeset{ \setA_\setIel }_{\setIel \setin\setI}$ of sets -- can we still take their \SY{cartesian product} or disjoint union?
The answer is yes.

For the \SY{cartesian product} of families we use the symbol~$\stylesets{\prod}$ instead of the infix notation~$\cartprod$, and for disjoint unions of families we use~$\stylesets{\sum}$ instead of the infix notation~$\setdisunion$.
(The upper case greek letter sigma $\sum$ stands for ``sum'', the upper case greek letter pi $\prod$ stands for ``product''.)

We define the \SY{cartesian product} of a family~$\makeset{ \setA_{\setIel} }_{\setIel \setin\setI}$ as
\begin{equation}
    \underset{\setIel \setin\setI}{\stylesets{\prod}} \setA_{\setIel} \definedas \makeset{ \mapa \colon \setI \to \bigsetunion_{\setIel \setin\setI } \setA_{\setIel} \mid \mapa(\setIel) \setin\setA_{\setIel}}.
\end{equation}

And for the \SY{disjoint union} of a family~$\makeset{ \setA_{\setIel} }_{\setIel \setin\setI}$ we set
\begin{equation}
    \underset{\setIel \setin\setI}{\stylesets{\sum}} \setA_{\setIel} \definedas \bigsetunion_{\setIel \setin\setI} \makeset{ \setIel } \cartprod \setA_{\setIel}.
\end{equation}

\todojira{659}{\alphubel: Make sure we really need the above.}
