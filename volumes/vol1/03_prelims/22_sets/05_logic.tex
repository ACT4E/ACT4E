% !TEX root = chapter-standalone.tex


\section{Logical preliminaries}

We assume the reader to have some familiarity with basic logical concepts and notation, in the way that they are typically used for reasoning and writing proofs in undergraduate mathematics. We recall the most basic notions, also to fix our notation. 

The building blocks for reasoning mathematically are logical statements (also called propositions, assertions, logical formulae, etc) which, in principle, may be evaluated to be either true or false, depending on the situation and the assumptions made.  In particular, a statement might depend on variables, and the truth or falsity of the statement might vary depending on how these variables are evaluated. 

\subsection{Basic operations}

Deductive reasoning is based on logical implications: statements of the form ``if statement $P$ is true, then statement $Q$ is true" or, synonymously, ``$P$ implies $Q$". The statement ``$P$ implies $Q$" is abbreviated with the notation $P \Imp Q$. Here "$P \Imp Q$" is itself a statement that may be true or false. So we can think of ``$\Imp$'' as a function (or operation) that takes two logical statements as its arguments, and returns back a new logical statement. Such logical operations are also called \emph{connectives}; we can use them to connect given logical statements to form new ones. 

Besides implication, two very familiar logical operations are ``and'' and ``or'', usually denoted in infix notation by the symbols $\wedge$ and $\vee$, respectively. If $P$ and $Q$ are logical statements, then $P \wedge Q$ is the logical statement which is true precisely when both $P$ and $Q$ are true, and otherwise is false. And $P \vee Q$ is the logical statement that is true precisely when either $P$ or $Q$, or both, are true. 

A logical operation that only takes on argument is \emph{negation}: if $P$ is a statement, its negation $\lnot P$ is the statement that is true if and only if $P$ is false. 

It is useful to also include in our logical language the symbols $\true$ and $\false$ for ``true'' and ``false'', respectively. We may think of these as statements that have a constant truth-value, or as logical operations that take zero arguments. 


\subsection{Logical calculation}

Various rules, for example expressed using equations, relate the different logical operations to each other and dictate how to calculate with them. For instance, rules such as $\lnot (\lnot P) = P$ or $P \Imp Q = \lnot P \vee Q$. It is a matter of convention which connectives and rules are taken as primitive, and which are then derived from the primitive ones. 

An example of a compound statement is the statement ``$(P \Imp Q) \wedge (Q \Imp P)$'', which is abbreviated with the notation ``$P \Eqv Q$''. When the statement $P \Eqv Q$ is true, we say that $P$ and $Q$ are \emph{equivalent} or that $P$ is true \emph{if and only if} $Q$ is true. 

Note that we use parentheses above to make clear in which order to evaluate a compound logical formula. Conventions about how strongly different logical operations ``bind'' allow us to use less parentheses. For example, logical negation is taken to bind at the strongest level, so we can write $\lnot P \Imp Q$ instead of $(\lnot P) \Imp Q$, without any ambiguity. 





\subsection{Universal quantifiers}

The following symbols denote so-called `universal quantifiers'; they abbreviate the following English-language phrases:
\begin{align}
\forall x &\quad \quad \text{``for all $x$''} \\
\exists x &\quad \quad \text{``there exists $x$''}
\end{align}

\subsection{Sequent notation}


...