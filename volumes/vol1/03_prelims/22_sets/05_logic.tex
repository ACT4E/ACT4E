% !TEX root = chapter-standalone.tex


\section{Logical preliminaries}

We assume the reader to have some familiarity with basic logical concepts and notation, in the way that they are typically used for reasoning and writing proofs in undergraduate mathematics. We recall the most basic notions, also to fix our notation. 

The building blocks for reasoning mathematically are logical statements (also called assertions, propositions, logical formulae, etc) which, in principle, may be evaluated to be either or true or false, depending on the situation and the assumptions made.  In particular, a statement might depend on variables, and the truth or falsity of the statement might vary depending on how these variables are evaluated. 

Implication and equivalence... 

Logical connectives are operations which allow us to manipulate and join together logical statements, building new statements from old. The most familiar connectives are ``and'' and ``or'', denoted in infix notation by the symbols $\wedge$ and $\vee$, respectively. If $P$ and $Q$ are logical statements, then $P \wedge Q$ is the logical statement which is true if and only if both $P$ and $Q$ are true. And $P \vee Q$ is the logical statement that is true precisely when either $P$ or $Q$, or both, are true. 

Given a logical statement $P$, its negation $\lnot P$ is the statement that is true if and only if $P$ is false. 

The following symbols denote so-called `universal quantifiers'; they abbreviate the following meanings:
\begin{align}
\forall x &\quad \quad \text{``for all $x$''} \\
\exists x &\quad \quad \text{``there exists $x$''}
\end{align}

Logical implication