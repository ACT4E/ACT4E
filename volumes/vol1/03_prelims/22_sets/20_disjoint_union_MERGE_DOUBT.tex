% !TEX root = chapter-standalone.tex

\section{Disjoint union}

Given sets~$\setA$ and~$\setB$, their disjoint union is the set
\begin{equation*}
    \setA \setdisunion \setB := (\setA \times \{ 1 \}) \cup (\setB \times \{2\}).
\end{equation*}
In other words, an element of  $\setA \setdisunion \setB$ is either a tuple $\tup{\ela, 1}$ for some  $\ela \in \setA$ or a tuple $\tup{\elb, 2}$ for some $\elb \in \setB$.
The sets $\{ 1 \}$ and $\{2\}$ here simply provide labels which ``force'' the sets $\setA \times \{ 1 \}$ and $\setB \times \{2\}$ to be disjoint (even if $\setA$ and $\setB$ have elements in common).

As a simple example, let $\setA = \{ \sapple, \sbanana, \scarrot\}$ and~$\setB = \{ \stea, \swater \}$.
Then
\begin{equation*}
    \setA \setdisunion \setB = \{ \tup{\sapple, 1}, \tup{\sbanana, 1}, \tup{\scarrot, 1}, \tup{\stea, 2},  \tup{\swater, 2}\}.
\end{equation*}

A slightly more tricky example is when we consider $\setA \setdisunion \setA$, say.
By definition, this is $(\setA \times \{ 1 \}) \cup (\setA \times \{2\})$, so if $\setA = \{ \sapple, \sbanana, \scarrot\}$, then
\begin{equation*}
    \setA \setdisunion \setA = \{ \tup{\sapple, 1}, \tup{\sbanana, 1}, \tup{\scarrot, 1}, \tup{\sapple, 2}, \tup{\sbanana, 2}, \tup{\scarrot, 2} \}.
\end{equation*}

Let us also think about what happens when the empty set is at play.
For example, if~$\setA = \emptyset$, then~$\emptyset \setdisunion \setB = \setB \times \{2\}$, and similarly~$\setA \setdisunion \emptyset =  \{ \tup{\ela,1} \mid \ela \in \setA \}$.
Also, $\empty \setdisunion \emptyset = \emptyset$.

\begin{remark}
    If $\setA$ and $\setB$ are finite, then the size of $\setA \setdisunion \setB$ is the sum of the sizes of $\setA$ and $\setB$.
    This is a hint of why we think of $\setA \setdisunion \setB$ as a form of addition of sets.
\end{remark}

Later we will see that the cartesian product of sets is a special case of a very general construction in category theory, called the categorical product, and that the disjoint union of sets is a special case of a ``dual'' construction, called coproduct.

\devel{
    \todojira{1}{@J: There is a break in content here; it will be filled/bridged}

    We will name here some special sets that we will find convenient to use later.
    These names are not all standard; however they should be intuitively clear.

    In settings where we'd like to have a specific set of a certain finite size to refer to, we will sometimes use this naming convention:
    \begin{align*}
        [0]
            & \definedas \emptyset \\
        [1] & \definedas \{ 1\} \\
        [2] & \definedas \{ 1, 2\} \\
        [3] & \definedas \{ 1, 2, 3\}
    \end{align*}
    \dots and so on.
    Generally,
    %
    \begin{equation*}
        [n]
        \definedas \{ 1, \dots, n\}
    \end{equation*}
    %
    for any $n \in \natnumbers$.

    \todotextjira{499}{@J: AC: but are we actully going to use these?}
}
