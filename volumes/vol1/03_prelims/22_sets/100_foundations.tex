% !TEX root = chapter-standalone.tex

\section{A word on foundations}\label{sec:foundations}

\todojira{602}{POST-ALPHUBEL? maybe we can give the reader a very very short overview and orientation of how set theory and type theory fit into foundations for mathematics and also how they relate to computation and computer science??}

The topic ``foundations of mathematics'' refers to the ongoing quest for systems of thought and calculation that allow mathematical knowledge to be as epistemologically secure (``rigorous'') and as systematic as possible.
In our times, this typically involves using \emph{formal languages} to formulate logical systems which make our mathematical reasoning ``computational'' or ``formal'', and using these to set up basic mathematical theories in terms of which, ideally, all of mathematics can in principle be encoded.

For the vast majority of mathematical activity, it is not necessary to worry about foundational questions.
Most mathematicians, scientists, and engineers do not need to study the inner workings of mathematics in order to use it or to build new theories.
A fitting analogy might be with software: in most settings, a software engineer need only write code in higher-level programming languages; it is not necessary that she master the details of compilers and machine code.
However, it is useful to know \emph{that there is such a thing as machine code and compilers}, and sometimes one may indeed want to engage with that realm.

In contemporary mainstream mathematics, foundations are usually treated using the logical system known as first-order predicate calculus and the basic theory called ZFC set theory (ZFC stands for Zermelo-Fraenkel with Choice).
There is, though, a plurality of rigorous logical systems and foundational theories, and research in these areas is active and ongoing, both in terms of studying the foundations of mathematics and in terms of studying logical systems and formal languages in their own right.

These areas also share a large overlap with research in theoretical computer science, where formal languages, verification, computation etc. are all key topics.
Category theory in particular has brought new impulses to mathematical foundations and has become a strong bridge between computer science, logic, and higher-level mathematics (as well as other disciplines).

Without delving into technical details, let's now take a very very cursory tour of some of the topics in the ``machine room'' -- to give a hint of what's there, and to introduce some basic terminology.

\subsection{Formal languages}

Just like natural languages (such as the English you are reading here, for example), formal languages are built using `words' (also called expressions, strings, or formulas) which themselves are built from `letters' (or symbols) which belong to a given alphabet. In a formal language, however, one is very precise about the rules according to which expressions can be formed and manipulated -- the `syntax rules' and `grammar', so to speak. 

\todotext{@J: include a simple example to illustrate (the one at Wikipedia is quite good, for example)}



\subsection{Logic}

What is logic about?

What is a logical system or ``a logic''?

A \emph{logical system} (or \emph{logical language}) is typical made up of a formal language, a deductive system (axioms and rules of inference), and a semantics.


SEP: The deductive system is to capture, codify, or simply record arguments that are valid for the given language, and the semantics is to capture, codify, or record the meanings, or truth-conditions for at least part of the language

Wikipedia: A logical system or language (not be confused with the kind of "formal language" discussed above which is described by a formal grammar), is a deductive system (see section above; most commonly first order predicate logic) together with additional (non-logical) axioms and a semantics.
According to model-theoretic interpretation, the semantics of a logical system describe whether a well-formed formula is satisfied by a given structure.
A structure that satisfies all the axioms of the formal system is known as a model of the logical system.
A logical system is sound if each well-formed formula that can be inferred from the axioms is satisfied by every model of the logical system.
Conversely, a logic system is complete if each well-formed formula that is satisfied by every model of the logical system can be inferred from the axioms.

An example of a logical system is Peano arithmetic.

\

Explain the terms formal system, formal language.
..

\

Explain / mention the role of propositional logic and first-order propositional logic

\

Mention constructive mathematics?

\subsection{Set theory}

Mention ZFC.
..

\subsection{Type theory}

Explain the basic ideas ?
or just say we'll get into it in the exercises part of the book?

\

Explain here the rough role of category theory in the foundations story?

\subsection{Computer science and computation}

Mention the links to CS and computation?

\

Mention computational trinitarianism?

\subsection{Set theory and category theory}

The fully rigorous use of set theory and category theory does involve taking care of some technical details, notably in dealing with set-theory paradoxes and issues related to ``size''.
However, for our purposes in this book, one can (luckily) safely ignore these details.
Nevertheless, since terms and technicalities related to this topic inevitable arise when learning (and reading) category theory, we have chosen to include in our exposition a short explanation of some relevant terms.
For those interested in deeper discussion, we refer to the excellent exposition \url{https://arxiv.org/abs/0810.1279}.

\todotextjira{263}{@J: Include set-theory references in the bibliography and link to them}



