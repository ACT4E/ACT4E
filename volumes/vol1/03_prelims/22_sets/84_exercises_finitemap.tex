\sectionexercises{Mappings}

\Cref{lst:Mapping} shows our interface for \Mapping.

\classlisting{Mapping}

A \Mapping has a source and a target, which, in general, are \SY{setoids}.

A \Mapping is able to take an element from the source \SY{setoid} and return an element in the second \SY{setoid}.

Here we use Python's syntax sugar.
If an object defines a \funcname{__call__} method, then these are \emph{syntactically} equivalent:

\begin{minted}{python}
assert myobject(x) == myobject.__call__(x)
\end{minted}

Therefore, once you implement a \Mapping you can pass it around like if it was a Python function.

We also define a subclass \FiniteMap which is constrained to have both source and target be a \FiniteSet rather than a \Setoid (\cref{lst:FiniteMap}).

\begin{marginfigure}
    \begin{minipage}{4cm}
        \begin{minted}{yaml}
source:
  elements: [a, b, c]
target:
  elements: [1, 2]
values:
  - [a, 1]
  - [b, 2]
  - [c, 1]
\end{minted}
    \end{minipage}
    \caption{Format for representing maps.}
    \label{fig:mapexample}
\end{marginfigure}

\classlisting{FiniteMap}

We define a YAML representation for finite maps as in~\cref{fig:mapexample}.

The fields \fieldname{source} and \fieldname{target} are two sets.
These are expressions that you can pass to the set constructors you developed previously.

The field \fieldname{values} is an array of pairs: a pair~$\tup{x, y}$ means that~$x$ maps to~$y$.

For the data to be well-formed it is necessary that for each element in the domain, there is exactly one row for that element.
If that is not the case, throw an exception \classname{InvalidFormat}.

\todo{use \str{codeboilerplate} to be consistent?}

\begin{codeexercise}[\exname{TestFiniteMapRepresentation}]
    Create a function to load the data by implementing the interface in \cref{lst:FiniteMapRepresentation}.
\end{codeexercise}

\classlisting{FiniteMapRepresentation}

As part of the exercise, you have to implement \classname{FiniteMap} with all its necessary methods.
