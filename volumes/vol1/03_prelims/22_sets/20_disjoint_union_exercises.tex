\sectionexercises{Disjoint union}

A \SetDisjointUnion is a particular type of set, which has two extra methods: \funcname{pack} and \funcname{unpack}:
\begin{itemize}
    \item The method \funcname{pack} allows constructing an element of the disjoint union in a given component.
    \item The method \funcname{unpack} allows deducing which component the element comes from and get its original value.
\end{itemize}

\begin{widepar}
    \aligninner{%
        \begin{minipage}{16cm}

            \classlisting{SetDisjointUnion}

            \classlisting{FiniteSetDisjointUnion}
        \end{minipage}
    }
\end{widepar}

\codeboilerplate{FiniteMakeSetDisjointUnion}{
    Write code to create the disjoint union of a list of sets.
}

\begin{widepar}
    \aligninner{%
        \begin{minipage}{16cm}
            \classlisting{FiniteMakeSetDisjointUnion}
        \end{minipage}
    }
\end{widepar}
