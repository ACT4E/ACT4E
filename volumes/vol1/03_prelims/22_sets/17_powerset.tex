% !TEX root = chapter-standalone.tex

\section{Power set}
\label{sec:power-set}

\begin{ctdefinition}[Power set]
    \label{def:power-set}
    Given a set~$\setA$, we can form a new set whose elements are the subsets of~$\setA$.
    This new set is called the \iindex{\emph{powerset}} of~$\setA$, and we denote it by~$\powerset \setA$.
\end{ctdefinition}

For example, if~$\setA = \{ \sapple, \scarrot, \sbanana \}$, then its powerset is
\begin{equation*}
    \powerset \setA = \{ \emptyset, \{ \sapple \}, \{ \scarrot \}, \{ \sbanana\}, \{ \sapple, \scarrot \}, \{ \scarrot, \sbanana \}, \{ \sapple, \sbanana\}, \{ \sapple, \scarrot, \sbanana \} \}.
\end{equation*}
The powerset is thus a set of sets.

\todotext{@J: do we want to say more here about the power set? is there enough material to justify its own section? or incorporate in sets section above? Or a section on ``new sets from old'' ? }

\

\todotext{@J: the paragraph below is a bit of an orphan - where to put and how to justify / motivate? I think this was taken from the book ``Naive Set Theory'', by Halmos.}

We will use an assumption (or ``axiom'') that will in fact allow us to take ``arbitrary'' unions and intersections.
Namely, given any two sets~$\setA$ and~$\setB$, we assume that there always exists some other set~$\setC$ such that~$\setA \subseteq \setC$ and~$\setB \subseteq \setC$.
In other words, there exists~$\setC$ such that~$\setA \in \powerset \setC$ and~$\setB \in \powerset \setC$.
And more generally, suppose that we are given not just two sets, but a whole collection~$\mathcal{S}$ of sets, in other words, a set~$\mathcal{S}$ of sets (we are using the word ``collection'' interchangeably with the word ``set'').
We will assume there exists a set~$\setC$ such that~$\setA \in \setC$ for all~$\setA \in \mathcal{S}$.



Now we can take the union and intersection of the collection~$\mathcal{S}$ of sets:
%
\begin{equation*}
    \bigcup \mathcal{S} \coloneqq \{ \ela \in \setC \mid \exists \  \setA \in \setC \colon \ela \in \setA \},
\end{equation*}
%
and
%
\begin{equation*}
    \bigcap \mathcal{S} \coloneqq \{ \ela \in \setC \mid \forall \  \setA \in \setC \colon \ela \in \setA \}.
\end{equation*}

