% !TEX root = chapter-standalone.tex

\section{Cartesian product of sets}

\subsection{Tuples}

The notion of a ``tuple'' is is essentially identical to that of a list: a finite sequence or listing of ``things'', with repetitions allowed. 

Usually tuples are referred to in a way that specifies a specific length: tuples of length one are called 1-tuples; tuples of length two are usually called \emph{ordered pairs} or 2-tuples; tuples of length three are called \emph{triples} or $3$-tuples; tuples of length four are calle \emph{quadruples} or 4-tuples; and so on. In general, given $n \in \natnumbers$, a tuple of length $n$ is called an $n$-tuple. 

The notation to indicate a tuple is via angled brackets: for instance, $\tup{\ela, \elb, \elc}$ denotes a 3-tuple, or triple. The items inside the brackets will be called entries or components. 

So what is the difference between lists and tuples? For us, it is essentially context and usage. 

Typically, we will use tuples in situations where we also specify, for each entry of the tuple, a set for which that entry is an element. For example, if $\setA = \{ \sapple, \scarrot, \sbanana \}$ and $\setB = \{ \stea, \swater \}$, then we will sometimes want to specify that $\tup{\sapple, \swater}$ is a 2-tuple where $\sapple \in \setA$ and $\swater \in \setB$, for instance. Or, as another example, we might consider $\tup{3, 7, 8}$ as a triple of natural numbers: here we are specifying that we are thinking of $3$, $7$, and $8$ as elements of the set $\natnumbers$. 

\subsection{Cartesian product} 

Given sets $\setA$ and $\setB$, their \emph{cartesian product} is a new set -- denoted $\setA \times \setB$ -- who's elements are precisely all possible 2-tuples $\tup{\ela, \elb}$ such that the first entry $\ela$ is an element of $\setA$ and the second entry $\elb$ is an element of $\setB$. 

For example, if~$\setA = \{ \sapple, \sbanana, \scarrot\}$ and~$\setB = \{ \stea, \swater \}$, then
\begin{equation*}
    \setA \cartprod \setB = \{ \tup{\sapple, \stea}, \tup{\sapple, \swater}, \tup{\sbanana, \stea}, \tup{\sbanana, \swater},  \tup{\scarrot, \stea}, \tup{\scarrot, \swater}\}.
\end{equation*}
In the special case where~$\setA = \emptyset$ or~$\setB = \emptyset$, then~$\setA \cartprod \setB = \emptyset$.

\begin{remark}
For finite sets $\setA$ and $\setA$, the size of $\setA \times \setB$ is the product (multiplication) of the sizes of $\setA$ and $\setB$. This is one hint of why we think of $\setA \times \setB$ as a kind of multiplication of sets. 
\end{remark}

In set theory, 2-tuples (ordered pairs) are usually defined formally by setting
\begin{equation}
    \label{eq:ordered-pair}
    \tup{\ela, \elb} \coloneqq \{ \{\ela \} , \{ \ela, \elb \} \},
\end{equation}
which allows to give the more formal definition of the cartesian product as
            \begin{equation}
                \setA \times \setB = \{ \elc \in \powerset \powerset (\setA \setunion \setB) \mid \exists \ela \in \setA, \exists \elb \in \setB : \elc = \tup{\ela, \elb} \}.
            \end{equation}

We will, however, simply treat ordered pairs as a primitive given construction (i.e. without reference to formal set theory), and hence we also treat the construction of `cartesian product of sets' as primitive. 

Similarly, we will take the notions of $n$-tuple, for any $n \in \natnumbers$ as primitive, and with it also the notion of $n$-fold cartesian product of sets. That is, if $\setA_1, \setA_2, ... \setA_n$ are sets, then we take 
\begin{equation}
\setA_1 \times \setA_2 \times \dots \times \setA_n
\end{equation}
to be the set of $n$-tuples $\tup{\ela_1, ...., \ela_n}$ with $\ela_i \in \setA_i$ for all $i = 1,2,...,n$. 


