% !TEX root = chapter-standalone.tex

\section{Cartesian product of sets}

\subsection{Tuples}

The notion of a ``tuple'' is is essentially identical to that of a list: a finite sequence or listing of ``things'', with repetitions allowed. 

Usually tuples are referred to in a way that specifies a specific length: tuples of length one are called 1-tuples; tuples of length two are usually called \emph{ordered pairs} or 2-tuples; tuples of length three are called \emph{triples} or $3$-tuples; tuples of length four are calle \emph{quadruples} or 4-tuples; and so on. In general, given $n \in \natnumbers$, a tuple of length $n$ is called an $n$-tuple. 

The notation to indicate a tuple is via angled brackets: for instance, $\tup{\ela, \elb, \elc}$ denotes a 3-tuple, or triple. The items inside the brackets will be called entries or components. 

So what is the difference between lists and tuples? For us, it is essentially context and usage. 

Typically, we will use tuples in situations where we also specify, for each entry of the tuple, a set for which that entry is an element. For example, if $\setA = \{ \sapple, \scarrot, \sbanana \}$ and $\setB = \{ \sapple, \swater \}$, then we will sometimes want to specify that $\tup{\sapple, \swater}$ is a 2-tuple where $\sapple \in \setA$ and $\swater \in \setB$, for instance. Or, as another example, we might consider $\tup{3, 7, 8}$ as a triple \emph{of natural numbers}: here we are specifying that we are thinking of $3$, $7$, and $8$ as elements of the set $\natnumbers$. 

\subsection{Cartesian product} 

Given sets $\setA$ and $\setB$, their \emph{cartesian product} is a new set -- denoted $\setA \times \setB$ -- who's elements are precisely all possible 2-tuples $\tup{\ela, \elb}$ such that the first entry $\ela$ is an element of $\setA$ and the second entry $\elb$ is an element of $\setB$. 

\

\

Let's start with just two things, call them~$\ela$ and~$\elb$.

 \emph{ordered pair}~$\tup{\ela,\elb}$ is a list where the order \emph{does} matter.
In other words, the ordered pair~$\tup{\elb,\ela}$ is \emph{not} equal to the ordered pair~$\tup{\ela,\elb}$.
Ordered pairs can be defined rigorously using sets, by setting
\begin{equation}
    \label{eq:ordered-pair}
    \tup{\ela, \elb} \coloneqq \{ \{\ela \} , \{ \ela, \elb \} \},
\end{equation}
using the the nesting of the elements of this set to encode the order.
We will not usually need this technical definition however;
the important point is to clarify that this notion describes ordered lists, specifically denoted with certain symbols (the use of commas and brackets).
Generalizing to lists of more than two things, given any finite number~$n \in \natnumbers$ of things~$\ela_1$, $\ela_1$, \dots,~$\ela_n$, we similarly define the ordered list~$\tup{\ela_1, \ela_2, \ldots , \ela_n}$, which we call a \emph{tuple}.

\label{def:cartesian-product}
% \todotextjira{288}{Below there was cited ``formal definition of ordered pairs'' but it doesn't exist. Add?}
\begin{definition}[Cartesian product of sets]
    Given sets~$\setA$ and~$\setB$, their \emph{cartesian product}, denoted~$\setA \cartprod \setB$, is the set
    \begin{equation*}
        \{ \tup{\ela, \elb} \mid \ela \in \setA \text{ and } \elb \in \setB \}.
        \footnote{In order to accurately follow the schema given in \cref{eq:axiom-specialization} and using the formal definition of ordered pairs it would in fact be correct to define the cartesian product as
            \begin{equation}
                \setA \times \setB = \{ \elc \in \powerset \powerset (\setA \setunion \setB) \mid \exists \ela \in \setA, \exists \elb \in \setB : \elc = \tup{\ela, \elb} \}.
            \end{equation}
        }
    \end{equation*}
\end{definition}
For example, if~$\setA = \{ \sapple, \sbanana, \scarrot\}$ and~$\setB = \{ \stea, \swater \}$, then
\begin{equation*}
    \setA \cartprod \setB = \{ \tup{\sapple, \stea}, \tup{\sapple, \swater}, \tup{\sbanana, \stea}, \tup{\sbanana, \swater},  \tup{\scarrot, \stea}, \tup{\scarrot, \swater}\}.
\end{equation*}
We remark that in the special case where~$\setA = \emptyset$ or~$\setB = \emptyset$, then~$\setA \cartprod \setB = \emptyset$.
