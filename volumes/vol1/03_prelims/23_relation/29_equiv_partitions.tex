% !TEX root = chapter-standalone.tex

\section{Equivalence relations}
\linkvideo{spring2021-relations:relations:equivalence-rel} % Equivalence relations

\begin{ctdefinition}[Equivalence relation]
    \label{def:equivalence-relation}
    An endorelation~$\relA\colon \setA\mto \setA$ is an \emph{\iindex{equivalence relation}} if it is symmetric, reflexive, and transitive.
    We write~$\ela\sim \elb$ if~$\inrel{\ela}{\relA}{\elb}$.
\end{ctdefinition}

\begin{example}
    The relation ``equals'' on~\natnumbers  is an equivalence relation.
    The relation ``less than or equal'' on~\natnumbers  is not.
\end{example}

\begin{example}
    The relation on~\natnumbers  ``differing by a multiple of 3''
    \begin{equation}
        \prfdoubleperiod{
            \inrel{\ela}{\relA}{\elb}
        }{
            (\ela-\elb) \mod 3 = 0
        }
    \end{equation}
    is an equivalence relation.
    Indeed, the relation is reflexive, and symmetric.
    Furthermore, if~$\ela$ differs by a multiple of 3 from~$\elb$ and~$\elb$ differs by a multiple of 3 from~$\elc$, then~$\ela$ differs by a multiple of 3 from~$\elc$ (transitivity).
\end{example}

\begin{example}
    The relation ``has the same birthday as'' on the set of all people is an equivalence relation.
    It is symmetric, because if Anna has the same birthday as Bob, then Bob has the same birthday as Anna.
    It is reflexive because every person has the same birthday as themselves.
    % It is transitive because if Anna has the same birthday as Bob, and Bob has the same birthday as Clara, then Anna has the same birthday as Clara.
\end{example}

\begin{example}
    Let~$\mapa\colon \setA \sto \setB$ be a function between sets.
    The following defines an equivalence relation:
    \begin{equation}
        \prfdoubleperiod{
            \ela\sim \elb
        }{
            \mapa(\ela)=\mapa(\elb)
        }
    \end{equation}
\end{example}

\begin{ctdefinition}[Partition]
    \label{def:partition}
    A \emph{\iindex{partition}} of a set~\setA is a collection~$\makeset{\setA_i}_{i\setin I}$ of subsets~$\setA_i\setsubseteq \setA$ such that
    \begin{enumerate}
        \item $\setA_i\setintersection \setA_j=\Emptyset \quad \forall i\neq j$;
        \item $\bigsetunion_{i\setin I}\setA_i=\setA$.
    \end{enumerate}
\end{ctdefinition}

\begin{remark}
    Equivalence relations are a way to group together elements of a set which we think of as ``the same'' in some respect.
    There is a one-to-one correspondence between equivalence relations on a set~\setA and partitions on~\setA.
\end{remark}

\begin{marginfigure}
    \centering
    \includesag{030_information_networks}
    \caption{
        \label{fig:info_network}
    }
\end{marginfigure}

\begin{example}
    An example of partitions can be shown through information networks.
    An exemplary network is depicted in \cref{fig:info_network}.
    Here, nodes represent data centers, and the arrows represent information flows.
    We say that data centers~$x$ and~$y$ are equivalent ($x\sim y$) if and only if there is a path from~$x$ to~$y$ and a path from~$y$ to~$x$.
    In \cref{fig:info_network}, we have that~$a\sim b$,~$e\sim d$, and also every center is equivalent with itself.
\end{example}

\vfill
\begin{gradedexercise}[\exname{CountingEquivalenceRelations}]
    \label{ex:CountingEquivalenceRelations}
    Let~$\setA = \makeset{ 1, 2, 3, 4 }$.
    How many different equivalence relations are there on \setA?
    Explain how you found your answer.
\end{gradedexercise}

\solutionof{CountingEquivalenceRelations}
