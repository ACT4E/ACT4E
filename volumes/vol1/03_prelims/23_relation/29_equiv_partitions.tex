% !TEX root = chapter-standalone.tex

\section{Equivalence relations}

\linkvideo{spring2021-relations:relations:equivalence-rel} % Equivalence relations

 \SY{Equivalence relations} are a way to group together elements of a set which wish to think of as ``the same'' in some respect. They appear all over mathematics. 

\begin{ctdefinition}[Equivalence relation]
    \label{def:equivalence-relation}
    An endorelation~$\relA\colon \setA\mto \setA$ is an \maindef{equivalence relation} if it is \SY{symmetric}, \SY{reflexive}, and \SY{transitive}.
    
    If $\relA$ is an equivalence relation, we often write~$\ela\equival_\relA \elb$, or simply $\ela\equival \elb$, instead of~$\inrel{\ela}{\relA}{\elb}$.
\end{ctdefinition}

\begin{example}
    The relation ``equals'' on~\natnumbers is an \SY{equivalence relation}.
    The relation ``less than or equal'' on~\natnumbers is not.
\end{example}

\begin{example}
    The relation on~\natnumbers ``differing by a multiple of 3''
    \begin{equation}
        \prfdoubleperiod{
            \inrel{\ela}{\relA}{\elb}
        }{
            (\ela-\elb) \mod 3 = 0
        }
    \end{equation}
    is an \SY{equivalence relation}.
    Indeed, the relation is \SY{reflexive}, and \SY{symmetric}.
    Furthermore, if~$\ela$ differs by a multiple of 3 from~$\elb$ and~$\elb$ differs by a multiple of 3 from~$\elc$, then~$\ela$ differs by a multiple of 3 from~$\elc$ (transitivity).
\end{example}

\begin{example}
    The relation ``has the same birthday as'' on the set of all people is an \SY{equivalence relation}.
    It is \SY{symmetric}, because if Anna has the same birthday as Bob, then Bob has the same birthday as Anna.
    It is \SY{reflexive} because every person has the same birthday as themselves.
    % It is \SY{transitive} because if Anna has the same birthday as Bob, and Bob has the same birthday as Clara, then Anna has the same birthday as Clara.
\end{example}

\begin{example}
    Let~$\mapa\colon \setA \sto \setB$ be a function between sets.
    The following defines an \SY{equivalence relation} $\equivalparam{\mapa}$ on the set $\setA$:
    \begin{equation}
        \prfdoubleperiod{
            \ela\equivalparam{\mapa} \elb
        }{
            \mapa(\ela)=\mapa(\elb)
        }
    \end{equation}
\end{example}

\begin{ctdefinition}[Partition]
    \label{def:partition}
    A \maindef{partition} of a set~\setA is a collection~$\makeset{\setA_\setIel}_{\setIel\setin \setI}$ of subsets~$\setA_\setIel\setsubseteq \setA$ such that
    \begin{enumerate}
        \item $\setA_\setIel\setintersection \setA_\setJel=\Emptyset \quad \forall \setIel\neq \setJel$;
        \item $\bigsetunion_{\setIel\setin \setI}\setA_\setIel=\setA$.
    \end{enumerate}
\end{ctdefinition}

\begin{remark}
    There is a one-to-one correspondence between \SY{equivalence relations} on a set~\setA and partitions on~\setA.
\end{remark}

\begin{quiz}[\exname{AnOddRelation}]
Consider the (endo)relation $R$ on the natural numbers $\mathbb{N}$ defined by $nRm$ if and only if $2$ is odd. Then $R$ is ... (select the properties that apply)
\begin{enumerate}[label=(\alph*)]
\item injective.
\item single valued.
\item surjective.
\item everywhere defined.
\item reflexive.
\item symmetric.
\item transitive.
\item none of the above.
\end{enumerate}
\end{quiz}

\begin{marginfigure}
    \centering
    \includesag{030_information_networks}
    \caption{
        \label{fig:info_network}
    }
\end{marginfigure}

\begin{example}
    An example of partitions can be shown through information networks.
    An exemplary network is depicted in \cref{fig:info_network}.
    Here, nodes represent data centers, and the arrows represent information flows.
    We say that data centers~$\setAel$ and~$\setBel$ are equivalent ($\ela\equival \elb$) if and only if there is a path from~$\ela$ to~$\elb$ and a path from~$\elb$ to~$\ela$.
    In \cref{fig:info_network}, we have that~$a\equival b$,~$e\equival d$, and also every center is equivalent with itself.
\end{example}

\vfill
\begin{gradedexercise}[\exname{CountingEquivalenceRelations}]
    \label{ex:CountingEquivalenceRelations}
    Let~$\setA = \makeset{ 1, 2, 3, 4 }$.
    How many different \SY{equivalence relations} are there on \setA?
    Explain how you found your answer.
\end{gradedexercise}

\solutionof{CountingEquivalenceRelations}

\devel{

    \todotext{@J: Discuss how isomorphism is an equivalence relation on the 'set' of all sets, and how this gives a notion of cardinality...}

    We can use the notion of isomorphism to make a more formal definition of the size, or cardinality, of a set, as follows.
    First we posit that any set \setA has an attribute which we call its cardinality, and denote by $\cardof \setA$ (as before).
    Then we say that for any two sets, \setA and \setB, it shall hold that
    \begin{equation}
        \prfdoubleperiod{
            \cardof \setA = \cardof \setB
        }{
            \setA \simeq \setB
        }
    \end{equation}
}

