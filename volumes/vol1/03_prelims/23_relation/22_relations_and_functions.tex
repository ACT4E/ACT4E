% !TEX root = chapter-standalone.tex

\section{Relations and functions}
\linkvideo{spring2021-relations:relations:fun-rel} % Functions and relations

Every function between sets can be thought as a relation: this was the basis of our formal definition of function, \cref{def:function}.
Let us restate that definition once again here, using notation that we have developed since.

\begin{ctdefinition}[Function as a special type of relation]
    \label{def:functions_as_relations}
    Let~\setA and~\setB be sets.
    A \SY{relation}~$\relA \setsubseteq \setA \cartprod \setB$ is a \emph{function} if it satisfies the following two conditions:
    \begin{enumerate}
        \item for all~$\ela \setin \setA$ there exists and element $\elb \setin \setB$ such that $\inrel{\ela}{\relA}{\elb}$;
        \item for all~$\ela, \elnb{1}, \elnb{2}$, this holds:
              \begin{equation}
                  \prfperiod{
                      \inrel{\ela}{\relA}{\elnb{1}}
                  }{
                      \inrel{\ela}{\relA}{\elnb{2}}
                  }{
                      \elnb{1} = \elnb{2}
                  }
              \end{equation}
    \end{enumerate}
\end{ctdefinition}

Although we will mostly continue to think about functions in the ``usual'' way (as opposed to the perspective of \cref{def:functions_as_relations}), it is illuminating -- both for understanding relations and functions -- to study the relationships between the two points of view.

\subsection{From functions to relations}

Recall how to go from viewing a function in the ``usual'' way to viewing it as a relation, as in \cref{def:functions_as_relations}.

As an illustration, consider the sets~$\setA = \makeset{ \sbretzel, \sfondue, \schoco }$ and~$\setB = \makeset{ \stea, \smilk, \swater, \swine }$, and the function~$\mapa \colon \setA \sto \setB$ defined (in the ``usual'' way) by
\begin{equation}
    \mapa (\sbretzel) = \stea, \quad \mapa (\sfondue) = \swater, \quad \mapa(\schoco) = \swine.
\end{equation}
\begin{marginfigure}
    \centering
    \includesag{30_rel_4}
    \caption{Visualization of the function \cref{eq:fun-rel}.}
    \label{fig:example_fun_rel}
\end{marginfigure}
This way of specifying the function $\mora$ may be depicted graphically as in \cref{fig:example_fun_rel}.

The relation that this function defines, in the sense of \cref{def:functions_as_relations}, is
%
\begin{equation}
    \label{eq:fun-rel}
    \makeset{ \tupp{\sbretzel, \stea}, \tupp{\sfondue, \swater}, \tupp{\schoco, \swine} } \setsubseteq \setA \cartprod \setB.
\end{equation}
\begin{marginfigure}
    \centering
    \includesag{30_rel_fun_graph}
    \caption{The graph of the function \cref{eq:fun-rel}.}
    \label{fig:example_fun_graph}
\end{marginfigure}

This relation \cref{eq:fun-rel} is what is often called the \emph{graph} of $\mora$.
That is, it is the set of tuples in~$\setA \cartprod \setB$ which are a pairing of an element of the source set~\setA with the element which is its image under~$\mapa$.
In \cref{fig:example_fun_graph}, the graph of \cref{eq:fun-rel} is visualized by highlighting the elements of the graph among all the elements of~$\setA \cartprod \setB$.

In general, any function~$\mapa \colon \setA \sto \setB$ corresponds to the relation
\begin{equation}
    \makeset{ \tup{\ela,\elb} \setin \setA \cartprod \setB \mid \elb= \mapa(\ela) }.
\end{equation}

\subsection{From relations to functions}

Let's start now with a relation~$\relA \setsubseteq \setA \cartprod \setB$ satisfying the conditions of \cref{def:functions_as_relations} and see how this corresponds to a function~$\mapa_{\relA} \colon \setA \sto \setB$ in the ``usual'' sense.

Choose an arbitrary~$\ela \setin \setA$.
According to point~1 in \cref{def:functions_as_relations}, there exists a~$\elb \setin \setB$ such that~$\inrel{\ela}{\relA}{\elb}$.
Choose such a~$\elb$, and call it~$\mapa_{\relA}(\ela)$.
This gives us recipe to get from any~$\ela$ to a~$\elb$.
But given a specific~$\ela \setin \setA$, what if we choose~$\elb$ differently each time we apply the recipe?
Point~2 guarantees that this can't happen: it says that the element~$\mapa_{\relA}(\ela)$ that we associate to a given~$\ela \setin \setA$ is in fact uniquely determined by that~$\ela$.
Put another way, the condition~$2$ says: if~$\mapa_{\relA}(\elna{1}) \neq \mapa_{\relA}(\elna{2})$, then~$\elna{1} \neq\elna{2}$.

\begin{remark}
    Of course, not every relation corresponds to a function -- namely precisely those that do not satisfy the conditions in \cref{def:functions_as_relations}.
    For example, the relation in \cref{fig:example_rel} is not a function.
    In fact, it does not satisfy \emph{either} of the conditions in \cref{def:functions_as_relations}.
\end{remark}

\subsection{Identity relations}

We have already discussed how, for any set $\setA$, there is always an identity function
\begin{equation}
        \defmapperiod{
            \mapidat\setA
        }{
            \setA
        }{
            \sto
        }{
            \setA
        }{
            \elb
        }{
            \elb
        }
        % (\elb) = \elb \quad \forall  \elb \setin\setB.
    \end{equation}
which ``does nothing''. If we turn such functions into relations, we call the result \emph{identity relations}. 

\begin{definition}
Let $\setA$ be any set. The \emph{identity relation} on $\setA$ is 
\begin{equation}
\catid_\setA = \makeset { \tup{\ela, \elb} \setin \setA \cartprod \setA \mid \ela = \elb}.
\end{equation}

\end{definition}

\subsection{Composing functions}

If we define functions as special kinds of relations, how is relation composition related to the ``usual'' way of composing of functions?
The answer is that these two apparently different ways of composing functions actually give the same result.

\begin{lemma}
    \label{lem:comprelfun}
    Let~$\relA \setsubseteq \setA \cartprod \setB$ and~$\relB \setsubseteq \setB \cartprod \setC$ be relations which are functions.
    Then their composition~$\relA \mthen \relB \setsubseteq \setA \cartprod \setC$ is again a function, and it corresponds to the ``usual'' composition of the functions corresponding to \relA and $\relB$.
\end{lemma}

\begin{proof}
    First let us check that when \relA and \relB are composed as relations, the result is again a function.
    For this we check that~$\relA \mthen \relB$ satisfies the two conditions stated in \cref{def:functions_as_relations}.

    \begin{enumerate}
        \item Choose an arbitrary~$\ela \setin \setA$.
              We need to show that there exists~$\elc \setin \setC$ such that~$\inrel{\ela}{\relA \mthen \relB}{\elc}$.
              Since~\relA is a function, there exists~$\elb \setin \setB$ such that~$\inrel{\ela}{\relA}{\elb}$.
              Choose such a~$\elb \setin \setB$.
              Then, because~\relB is a function, there exists~$\elc \setin \setC$ such that~$\inrel{\elb}{\relB}{\elc}$.
              By the definition of composition of relations, we see that~$\elc$ is such that~$\inrel{\ela}{\relA \mthen \relB}{\elc}$.
        \item Let~$\inrel{\elna{1}}{\relA \mthen \relB}{\elnc{1}}$,~$\inrel{\elna{2}}{\relA \mthen \relB}{\elnc{2}}$.
              We need to show that if~$\ela_1 = \ela_2$, then~$\elc_1 = \elc_2$.
              So suppose~$\elna{1} = \elna{2}$.
              Since~$\inrel{\elna{1}}{\relA \mthen \relB}{\elnc{1}}$,~$\inrel{\elna{2}}{\relA \mthen \relB}{\elnc{2}}$, there exist~$\elnb{1}, \elnb{2} \setin \setB$ such that, respectively,
              \begin{equation}
                  \inrel{\elna{1}}{\relA}{\elnb{1}} \booland \inrel{\elnb{1}}{\relB}{\elnc{1}},
              \end{equation}
              \begin{equation}
                  \inrel{\elna{2}}{\relA}{\elnb{2}} \booland \inrel{\elnb{2}}{\relB}{\elnc{2}}.
              \end{equation}
              Since~$\elna{1} = \elna{2}$ and~\relA is a function, we conclude that~$\elnb{1} = \elnb{2}$ must hold.
              Now, since~\relB is also a function, this implies that~$\elnc{1} = \elnc{2}$, which is what was to be shown.
    \end{enumerate}

    Second let us check that relation composition of functions gives the same result as the ``usual'' composition of functions.
    Let~$\mapa_{\relA}$ and~$\mapb_{\relB}$ denote the relations~\relA and~\relB when we are thinking of them in the ``usual'' way of thinking about functions.
    Our goal is to show that~$\mapa_{\relA} \mthen \mapb_{\relB}$ corresponds to~$\relA \mthen \relB$; in other words, that the latter is the graph of former.

    Suppose first that~$\tup{\ela, \elc}$ is in the graph of~$\mapa_{\relA} \mthen \mapb_{\relB}$, so~$\elc = (\mapa_{\relA} \mthen \mapb_{\relB})(\ela)$.
    In particular~$\elc =  \mapb_{\relB}(\mapa_{\relA}(\ela))$, which means there exists~$\elb = \mapa_{\relA}(\ela) \setin \setB$ such that~$\tup{\ela, \elb} \setin \relA$ and~$\tup{\elb, \elc} \setin \relB$.
    This implies that~$\tup{\ela, \elc} \setin \relA \mthen \relB$.

    Conversely, suppose~$\tup{\ela, \elc} \setin \relA \mthen \relB$.
    By the definition of relation composition there must exist~$\elb \setin \setB$ such that~$\tup{\ela, \elb} \setin \relA$ and~$\tup{\elb, \elc} \setin \relB$, which means~$\elb = \mapa_{\relA}(\ela)$ and~$\elc = \mapb_{\relB}(\elb)$.
    Thus,~$\elc = \mapb_{\relB}(\mapa_{\relA}(\ela))$.
\end{proof}

\subsection{Relations via functions}

\label{rem:rel-three-fun-descriptions}
Though not every relation \emph{is} a function, we can however think about relations \emph{in terms of} functions.
Here are three ways:
\begin{enumerate}
    \item We can think of a relation~$\relA \colon \setA \mto\setB$ as a function~$\setA \cartprod \setB \sto \boolset$.

          Given~\relA we can define a function~$\phi_{\relA} \colon \setA \cartprod \setB \sto \makeset{ \false, \true }$ from it by setting
          \begin{equation}
              \phi_{\relA}(\tup{\ela, \elb}) =
              \begin{cases}
                  \true  & \text{if } \inrel{\ela}{\relA}{\elb}, \\
                  \false & \text{otherwise}.
              \end{cases}
          \end{equation}
          Conversely, given a function~$\phi\colon \setA \cartprod \setB \sto \makeset{ \false, \true }$ we can define a relation~$\relA_{\phi} \setsubseteq \setA \cartprod \setB$ from it by setting
          \begin{equation}
              \relA_{\phi} = \makeset{ \tup{\ela, \elb} \setin \setA \cartprod \setB \mid \phi_{\relA}(\tup{\ela, \elb}) = \true }.
          \end{equation}
          These two constructions are inverse to one-another.

    \item We can think of a relation~$\relA \colon \setA \mto \setB$ as a function~$\setA  \sto \powerset (\setB)$.

          Given~\relA we can define a function~$\hat \phi_{\relA} \colon \setA \sto \powerset (\setB)$ via
          \begin{equation}
              \hat \phi_{\relA} (\ela) = \makeset{ \elb \setin \setB \mid \inrel{\ela}{\relA}{\elb}}.
          \end{equation}
          Conversely, given a function~$\hat \phi \colon \setA \sto \powerset (\setB)$, we can define
          \begin{equation}
              \relA_{\hat \phi} = \makeset{ \tup{\ela, \elb} \setin \setA \cartprod \setB \mid \elb \setin \hat \phi_{\relA}(\ela)   }.
          \end{equation}
          These two constructions are inverse to one another, too.

    \item We can think of a relation~$\relA \setsubseteq \setA \cartprod \setB$ as a function~$\setB  \sto \powerset (\setA)$.

          Given~\relA we can define a function~$\check \phi_{\relA} \colon \setB \sto \powerset (\setA)$ via
          \begin{equation}
              \check \phi_{\relA} (\elb) = \makeset{ \ela \setin \setA \mid \inrel{\ela}{\relA}{\elb}}.
          \end{equation}
          Conversely, given a function~$\check \phi \colon \setB \sto \powerset (\setA)$, we can define
          \begin{equation}
              \relA_{\check \phi} = \makeset{ \tup{\ela, \elb} \setin \setA \cartprod \setB \mid \ela \setin \check \phi_{\relA}(\elb)}.
          \end{equation}
          These two constructions are \emph{also} inverse to one another.
\end{enumerate}

\begin{marginfigure}
    \centering
    \includesag{30_rel_1_bis}
    \caption{}
    \label{fig:example_rel_again}
\end{marginfigure}
\vfill
\begin{gradedexercise}[\exname{Rel3Functions}]
    \label{ex:Rel3Functions}
    For the relation~\relA illustrated in~\cref{fig:example_rel_again}, write out the three functions that describe it, respectively, in the three ways outlined in \cref{rem:rel-three-fun-descriptions}.
\end{gradedexercise}

\solutionof{Rel3Functions}
