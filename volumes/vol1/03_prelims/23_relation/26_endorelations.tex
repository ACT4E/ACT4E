% !TEX root = chapter-standalone.tex

\section{Endorelations}
\label{sec:endorelations}

\linkvideo{spring2021-relations:relations:endorel} % Endorelations

\begin{table*}[bp]
    \caption{Summary of \SY{endorelation} properties.}
    \setlength{\tabcolsep}{15pt}
    \begin{tabular}{cccc}
        Reflexive                                                                    & Total      & Symmetric     & Transitive \\[+10pt]
        \prftree{\true}{\inrel{\ela}{\relA}{\ela}}                                   &
        \prftree{\true}{\inrel{\ela}{\relA}{\elb} \boolor \inrel{\elb}{\relA}{\ela}} &
        \prfdouble{
            \inrel \ela \relA \elb
        }{
            \inrel \elb \relA \ela
        }                                                                            &
        \prftree{\inrel{\ela}{\relA}{\elb}}{\inrel{\elb}{\relA}{\elc}}{\inrel{\ela}{\relA}{\elc}} \\[+10pt]
        Irreflexive                                                                  & Asymmetric & Antisymmetric & \\[+10pt]
        \prftree{
            \inrel \ela \relA \ela
        }{
            \false
        }                                                                            &
        \prftree{
            \inrel \ela \relA \elb
        }{
            \inrel \elb \relA \ela
        }{
            \false
        }                                                                            &
        \prftree{
            \inrel \ela \relA \elb
        }{
            \inrel \elb \relA \ela
        }{
            \ela = \elb
        }                                                                            & \\
    \end{tabular}
    \label{tab:endo_properties}
\end{table*}

\begin{ctdefinition}[Endorelation]
    \label{def:endorelation}
    An \maindef{endorelation} on a set~\setA is a relation~$\relA\colon \setA \mto \setA$.
\end{ctdefinition}

\begin{example}
    ``Equality'' on a set~\setA is an endorelation~$\relstyle{=}_\setA$ of the form
    \begin{equation}
        \relstyle{=}_\setA \definedas \makeset{\tup{\ela,\elb}\setin \setA\cartprod \setA \mid \ela=\elb}.
    \end{equation}
\end{example}

\begin{example}
    Take~$\setA=\natnumbers$.
    The relation ``less than or equal'' is an \SY{endorelation} of the form
    \begin{equation}
        \relstyle{\leq} \definedas \makeset{\tup{\ela,\elb}\setin \natnumbers\cartprod \natnumbers\mid \ela \leq \elb}.
    \end{equation}
\end{example}

\begin{example}
    The relation depicted in \cref{fig:power_internodal} is an \SY{endorelation} between the set of high voltage nodes.
\end{example}

\begin{ctdefinition}[Symmetry, asymmetry, and antisymmetry]
    \label{def:endo_sym_asym_antisym}
    \label{def:antisymmetry}
    \label{def:symmetric-relation}
    \label{def:asymmetric-relation}
    \label{def:antisymmetric-relation}
    \SYNDEF{symmetric relation}
    \SYNDEF{asymmetric relation}
    \SYNDEF{antisymmetric relation}
    An endorelation~$\relA\colon \setA\mto \setA$ is \emph{symmetric} if
    \begin{equation}\label{eq:symmetric}
        \prfdoublecomma{
            \inrel \ela \relA \elb
        }{
            \inrel \elb \relA \ela
        }
    \end{equation}
    is \emph{asymmetric} if
    \begin{equation}\label{eq:asymmetric}
        \prfcomma{
            \inrel \ela \relA \elb
        }{
            \inrel \elb \relA \ela
        }{
            \false
        }
    \end{equation}
    and is \emph{antisymmetric} if
    \begin{equation}\label{eq:antisymmetric}
        \prfperiod{
            \inrel \ela \relA \elb
        }{
            \inrel \elb \relA \ela
        }{
            \ela = \elb
        }
    \end{equation}
\end{ctdefinition}

\begin{ctdefinition}[Reflexivity and irreflexivity of endorelations]
    \label{def:endo_reflexive_irreflexive}
    \SYNDEF{reflexive relation}
    \SYNDEF{irreflexive relation}
    \label{def:reflexive-relation}
    \label{def:irreflexive-relation}
    An endorelation~$\relA\colon \setA\mto \setA$ is \emph{reflexive} if
    \begin{equation}\label{eq:reflexive}
        \prfcomma{
            \true
        }{
            \inrel \ela \relA \ela
        }
    \end{equation}
    and is \emph{irreflexive} if
    \begin{equation}\label{eq:irreflexive}
        \prfperiod{
            \inrel \ela \relA \ela
        }{
            \false
        }
    \end{equation}
\end{ctdefinition}

\begin{ctdefinition}[Totality of endorelations]
    \label{def:total-relation}
    \SYNDEF{total relation}
    An endorelation~$\relA\colon \setA\mto \setA$ is \emph{total} if
    \begin{equation}\label{eq:total}
        \prfperiod{
            \true
        }{
            (\inrel\ela\relA\elb )\,\boolor\, (\inrel\elb\relA\ela)
        }
    \end{equation}
\end{ctdefinition}

\begin{example}
    The relation ``less than or equal'' on~\natnumbers is not \SY{symmetric}.
    It is \SY{reflexive} since~$n\leq n \ \forall n\setin \natnumbers$, and it is \SY{transitive} since~$l\leq m$ and~$m\leq n$ implies~$l\leq m$.
\end{example}

\begin{example}
    The relation depicted in \cref{fig:power_internodal} is \SY{reflexive} (each node is connected to itself).
\end{example}
\begin{marginfigure}
    \centering
    \includesag{030_ex_sym_rel}
    \caption{Example of \SY{symmetric} endorelation.}
    \label{fig:ex_sym_rel}
\end{marginfigure}
\begin{example}
    The \SY{endorelation} depicted in \cref{fig:ex_sym_rel} is a \SY{symmetric} relation on~$\setA=\makeset{\sbretzel,\sfondue}$.
\end{example}

% \devel{\includepdf[scale=0.8,pages={8},nup=1x3,frame,pagecommand={}]{ACT4E-06-posets.pdf} }% equivalence}
