% !TEX root = chapter-standalone.tex

\section{Transitivity}

\begin{ctdefinition}[Transitivity of endorelations]
    \label{def:transitive-relation}
    \SYNDEF{transitive relation}
    An endorelation~$\relA\colon \setA\mto \setA$ is \emph{transitive} if
    \begin{equation}\label{eq:transitive}
        \prfperiod{
            \inrel \ela \relA \elb
        }{
            \inrel \elb \relA \elc
        }{
            \inrel \ela \relA \elc
        }
    \end{equation}
\end{ctdefinition}

\begin{example}
    The relation ``has the same birthday as'' is \SY{transitive} because if Anna has the same birthday as Bob, and Bob has the same birthday as Clara, then Anna has the same birthday as Clara.
\end{example}

\begin{ctdefinition}[Transitive closure]
    \label{def:transitive-closure}
    The \maindef{transitive closure} of an endorelation~\relA on a set~\setA is the unique relation $\traclos{\relA}$ on~\setA satisfying the following conditions:
    \begin{enumerate}
        \item $\setA \subseteq \traclos{\relA}$;
        \item $\traclos{\relA}$ is transitive;
        \item if $\relB$ is a relation on $\setA$ that satisfies the previous two points, then $\traclos{\relA} \subseteq \relB$.
    \end{enumerate}
\end{ctdefinition}

%\begin{example}
%    Consider a relation~$\relA$ on~$\setA=\{\sbretzel,\swine,\sfondue\}$ such that $\inrel \sbretzel \relA \swine$ and $\inrel \swine \relA \sfondue$.
%    The \SY{transitive} closure of~$\relA$ is~$\relA^+$, where~$\inrel \sbretzel \relA^+ \swine$, $\inrel \swine \relA^+ \sfondue$, and~$\inrel \sbretzel \relA^+ \sfondue$.
%\end{example}

\begin{example}
    \label{exa:party}
    Consider a relation~\relA on a set of people
    \begin{equation}
        \setA=\makesett{\text{Gioele}, \text{Andrea}, \text{Jonny}, \text{Emilio}, \text{Raff}},
    \end{equation}
    which describes who invites which friend to a party:
    \begin{equation}
        \label{eq:transitive_clos_a}
        \middlesag{transitive_clos_a}
    \end{equation}
    \todographics{\bernina: @Gioele: For this example, it is better to use the graph representation of a relation.
    }
    In other words, Gioele invites Andrea and Emilio, Andrea invites Jonny, and Emilio invites Raff.
    The \SY{transitive closure}~$\traclos{\relA}$ of~\relA describes all invitations resulting from \SY{transitivity}.
    \begin{equation}
        \label{eq:transitive_clos_b}
        \middlesag{transitive_clos_b}
    \end{equation}
    In particular, Gioele invites Jonny and Raff as well, due to the fact that Andrea invites Jonny, and Emilio invites Raff.
\end{example}

