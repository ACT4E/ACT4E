% !TEX root = chapter-standalone.tex

\chaptersecond{Relations}{chapter-polyball}{chap:relation}{%
    This chapter describes \emph{relations} from a categorical perspective.
    Binary relations have much in common with functions: they are arrows in a category called \Rel and they can be composed when target/source are compatible.
}

\label{chap:relation}
\subimport{}{10_distribution_networks}
\subimport{}{20_relations}
\codeexercises{
    \subimport{}{20_relations_exercises}
}

\subimport{}{21_relations_compose}
\devel{
    \subimport{}{21_relations_compose_exercises}
}

\subimport{}{22_relations_and_functions}
\subimport{}{25_relations_properties}
\codeexercises{
    \subimport{}{25_relations_properties_exercises}
}
\subimport{}{26_transpose}
\codeexercises{
    \subimport{}{26_transpose_exercises}
}
\subimport{}{26_endorelations}
\codeexercises{
    \subimport{}{26_endorelations_exercises}
}
\subimport{}{28_transitive}
\codeexercises{
    \subimport{}{28_transitive_exercises}
}
\subimport{}{29_equiv_partitions}
\subimport{}{30_relational}

\codeexercises{
    \subimport{}{50_code_exercises}
}

