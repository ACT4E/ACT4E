% !TEX root = chapter-standalone.tex

\section{Distribution  networks}\label{sec:connection-distribution-networks}
%\linkvideo{spring2021-relations:relations} % Relations
\linkvideo{spring2021-relations:relations:distr-networks} % Distribution networks
Consider the type of networks that arise for example in the context of electrical power grids.
In a simplified model for a certain region or country, we may have the following kinds of components: power plants (places where electrical power is produced), high voltage transmission lines and nodes, transformers stations, low voltage transmission lines and nodes, and consumers, such as homes and businesses.
The situation is depicted in \cref{fig:power_nodes}.

\begin{figure*}[p]
    \centering
    \includesag{30_net_1}
    \caption{Components of electrical power grids.}
    \label{fig:power_nodes}
\end{figure*}

To model the connectivity between the components of the power grid, we now draw arrows between components that are connected.
We set the direction of the arrows to flow from energy production, via transmission components, to energy consumption, as depicted in \cref{fig:power_nodes_connected}.
\begin{figure*}[p]
    \centering
    \includesag{30_net_2}
    \caption{Connectivity between components in electric power grids.}
    \label{fig:power_nodes_connected}
\end{figure*}

A possible question one asks about such a power distribution network is: which consumers are serviced by which power sources?
For example, power sources such as a solar power plant may fluctuate due to weather conditions, while other power sources, such as a nuclear power plant, may shut down every once in a while due to maintenance work.
To see which consumers are connected to which power plants, we can follow ``connectivity paths'' traced by sequences of arrows, as in~\cref{fig:power_paths}.
There, two possible connectivity paths are depicted (in red and orange, respectively).

\begin{figure*}[p]
    \centering
    \includesag{30_net_3}
    \caption{Connection between consumers and power plants.}
    \label{fig:power_paths}
\end{figure*}

We also will want to know the overall connectivity structure of transmission lines.
For example, some lines may go down during a storm, and we want to ensure enough redundancy in our system.
In addition to the connections modeled in \cref{fig:power_nodes_connected}, we can also include, for example, information about the connectivity of high voltage nodes among themselves, as in \cref{fig:power_internodal}.

\begin{figure*}[p]
    \centering
    \includesag{30_net_4}
    \caption{Connectivity between high voltage nodes.}
    \label{fig:power_internodal}
\end{figure*}

\begin{marginfigure}
    \centering
    \includegraphics[width=\linewidth]{dist_net_6}
    \caption{Alternative visualization for connectivity. \todographicsjira{47}{\bernina:Do own drawing}}
    \label{fig:power_graph}
\end{marginfigure}

\begin{marginfigure}
    \centering
    \includegraphics[width=\linewidth]{power_dist_network}
    \caption{A schematic view of a power grid.}
    \label{fig:power_schema}
\end{marginfigure}

The information encoded in \cref{fig:power_internodal} and \cref{fig:power_nodes_connected} can also be displayed as a single graph, see \cref{fig:power_graph,fig:power_internodal}.

If we ignore the directionality of the arrows, this is analogous to a depiction of type shown in \cref{fig:power_schema}, which is a schema of a power grid~\cite{Cuffe17}\footnote{See \url{https://en.wikipedia.org/wiki/Electrical_grid}}.
