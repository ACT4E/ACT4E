\sectionexercises{Transitivity}
\codeboilerplate{FiniteEndorelationOperations}{

    Compute the \SY{transitive closure} of an endorelation.

}
\classlisting{FiniteEndorelationOperations}{}

\begin{hint}
    Finding the \SY{transitive closure} requires an iterative algorithm.

    Suppose that you have an \SY{endorelation} \relA.
    For each element $\ela$, you need to ``follow the arrows'' along the relation.
    For example, if $\inrel \ela \relA \elb$ and $\inrel \ela \relA \elc$, in the \SY{transitive closure} of $\ela$ there are $\elb, \elc$ as well as all the elements in the \SY{transitive closure} of $\elb$ and $\elc$.

    This property suggests a recursive solution; however, note that there might be loops, and if you do not take appropriate measures, the code will be stuck in an infinite loop.
    If you choose the recursive approach, you must remember the elements that you have already seen in the recursion.

    The cleanest approach is not recursive.
    For each element, you want to find the elements reachable from it.
    You can do this by iterating over ``hops''.
    First, you add the elements that are reachable in 1 hop, and you mark those for ``expansion'' later.
    Then, you expand those elements to find the elements that are reachable in 2 hops.
    But do not mark for expansion those that have already been seen.

    If you choose the second approach, you have created something that could later be expanded into a ``search algorithm'' (Dijkstra's algorithm), which finds not only which elements are reachable from a start element, but also the minimum number of hops to get there.
\end{hint}
