% !TEX root = chapter-standalone.tex

\section{Properties of relations}
\linkvideo{spring2021-relations:relations:prop-rel} % Properties of relations

% \linkvideo{spring2021-tradeoffs:relations-properties} % Recap of relations
We have seen that relations generalize functions -- every function defines a relation, via its graph, but not every relation comes from a function in this way.
Many notions that we are familiar with for functions also generalize to relations.
Here are a few.

\begin{ctdefinition}[Properties of a relation]
	\label{def:rel_properties}
	We say that a relation  $\relA \colon \setA \mto \setB$ is:
	\begin{enumerate}
		\item \emph{Injective} if
		      \begin{equation}
			      \prfsemi{
				      \inrel \ela \relA \elb
			      }{
				      \inrel \elc \relA \elb
			      }{
				      \ela=\elc
			      }
		      \end{equation}
		\item \emph{Single-valued} if
		      \begin{equation}
			      \prfsemi{
				      \inrel \ela \relA \elb
			      }{
				      \inrel \ela \relA \eld
			      }{
				      \elb = \eld
			      }
		      \end{equation}
		\item \emph{Surjective} if for all~$\elb\in \setB$ there exists an~$\ela\in \setA$:~$\inrel{\ela}{\relA}{\elb}$;
		\item \emph{Everywhere-defined} if for all~$\ela\in \setA$ there exists an ~$\elb \in \setB\colon \inrel{\ela}{\relA}{\elb}$.
	\end{enumerate}
\end{ctdefinition}

\begin{example}
	The relation depicted in \cref{fig:example_rel} is injective but not surjective.
	It is not single-valued, nor everywhere-defined.
\end{example}

