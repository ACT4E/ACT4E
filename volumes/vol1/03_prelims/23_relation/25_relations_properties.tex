% !TEX root = chapter-standalone.tex

\section{Properties of relations}
\linkvideo{spring2021-relations:relations:prop-rel} % Properties of relations

% \linkvideo{spring2021-tradeoffs:relations-properties} % Recap of relations
We have seen that relations generalize functions -- every function defines a relation, via its graph, but not every relation comes from a function in this way.
Many notions that we are familiar with for functions also generalize to relations.
Here are a few.

\begin{ctdefinition}[Properties of a relation]
    \label{def:rel_properties}
    \SYNDEF{injective relation}
    \SYNDEF{single-valued}
    \SYNDEF{everywhere-defined}
    \SYNDEF{surjective relation}
    We say that a relation~$\relA \colon \setA \mto \setB$ is:
    \begin{enumerate}
        \item \label{def:injective-relation} \emph{Injective} if
              \begin{equation}
                  \prfsemi{
                      \inrel \ela \relA \elb
                  }{
                      \inrel \elc \relA \elb
                  }{
                      \ela=\elc
                  }
              \end{equation}
        \item \label{def:single-valued-relation} \emph{Single-valued} if
              \begin{equation}
                  \prfsemi{
                      \inrel \ela \relA \elb
                  }{
                      \inrel \ela \relA \eld
                  }{
                      \elb = \eld
                  }
              \end{equation}
        \item \label{def:surjective-relation} \emph{Surjective} if for all~$\elb\setin \setB$ there exists an element~$\ela\setin \setA$ such that ~$\inrel{\ela}{\relA}{\elb}$;
        \item \label{def:everywhere-defined-relation} \emph{Everywhere-defined} if for all~$\ela\setin \setA$ there exists an element~$\elb \setin \setB$ such that $\inrel{\ela}{\relA}{\elb}$.
    \end{enumerate}
\end{ctdefinition}

\begin{example}
    The relation depicted in \cref{fig:example_rel} is \SYN{injective relation}{injective} but not \SYN{surjective relation}{surjective}.
    It is not \SY{single-valued}, nor \SY{everywhere-defined}.
\end{example}

\begin{quiz}[\exname{InjectiveOrNot}]
Which of the following endorelations is injective?
\begin{enumerate}[label=(\alph*)]
\item $R=\{(x,y)\in\mathbb{R}\times\mathbb{R}\vert x=y^{2}\}$ on $\mathbb{R}$.
\item $R=\{(x,y)\in\mathbb{R}\times\mathbb{R}\vert x=y\}$ on $\mathbb{R}$.
\item $R=\{(x,y)\in\mathbb{R}\times\mathbb{R}\vert y=x^{2}\}$ on $\mathbb{R}$.
\item $R=\{(x,y)\in\mathbb{R}\times\mathbb{R}\vert x^2+y^{2}=1\}$ on $\mathbb{R}$.
\end{enumerate}
\end{quiz}


\begin{quiz}[\exname{SurjectiveOrNot}]
Which of the following endorelations is surjective?
\begin{enumerate}[label=(\alph*)]
\item $R=\{(x,y)\in\mathbb{R}\times\mathbb{R}\vert x=y^{2}\}$ on $\mathbb{R}$.
\item $R=\{(x,y)\in\mathbb{R}\times\mathbb{R}\vert x=y\}$ on $\mathbb{R}$.
\item $R=\{(x,y)\in\mathbb{R}\times\mathbb{R}\vert y=x^{2}\}$ on $\mathbb{R}$.
\item $R=\{(x,y)\in\mathbb{R}\times\mathbb{R}\vert x^2+y^{2}=1\}$ on $\mathbb{R}$.
\end{enumerate}
\end{quiz}