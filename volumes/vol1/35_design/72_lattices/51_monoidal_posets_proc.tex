\section{Monoidal-time procedures}
\label{sec:ProcMod}

\todojira{581}{\alphubel: @Andrea: Review Flow and nomenclature for monoidal-time procedures}

% \todojira{671}{\alphubel: @Andrea: Move this somewhere else? }
% Now that we know about posets,  \SY{monotone functions} and monoidal posets, we can extend the example of \ProcSizeTime in \vref{def:ProcSizeTime}.

Before, we thought of sized sets (\cref{def:sized-set}) as a datatype that can be measured with integer sizes.
However, this does not capture some important cases.
For example, if we are dealing with \emph{trees}, from the point of view of computation it could be important to think about multidimensional sizes: for example, we might want to account for number of nodes, number of edges, maximum branching factor, and so on.
Still, we want to know when an instance is bigger than another: this is a perfect job for \SY{posets}.

\begin{definition}[Poset-sized sets]
    \label{def:poset-sized-set}
    A \emph{poset-sized set} is a tuple $\tupp{\setA, \Sigma_{\setA}, \sizefun}$, where \setA is a set, $\Sigma_\setA$ is a poset, and $\sizefun \colon \setA \sto \Sigma_\setA$ is the size function.
\end{definition}

In \ProcSizeTime, we assumed that time was measured using real numbers.
We can generalize this to an arbitrary poset, for example allowing counting ``number of operations''.
We need an additional structure: in \cref{eq:timefun-composition} we needed a ``+'' to sum the time of successive procedures.
Therefore, we assume that we have a time \SY{monoidal poset} \TimeMonoidal.

\begin{definition}
    [Semicategory $\ProcMod$]
    \label{def:ProcMod}
    For a given \SY{monoidal poset} $\tupp{\TimeMonoidal, \monpostimes_{\TimeMonoidal}}$, the \SY{semicategory}~$\ProcMod$ consists of the following constituents:
    \begin{enumerate}
        \item \emph{Objects}: The objects are poset-sized sets.
        \item \emph{Morphisms}: A morphism
              \begin{equation}
                  \mora \colon \Obja \mtoin\ProcSize \Obja
              \end{equation}
              between the two objects
              \begin{equation}
                  \Obja = \tupp{\setA, \Sigma_{\setA}, \sizefun_{\setA}}
                  \qqand
                  \Objb = \tupp{\setB, \Sigma_{\setA}, \sizefun_{\setB}}
              \end{equation}
              is a tuple
              \begin{equation}
                  \tupp{\mora_e, \sizetran, \timefun },
              \end{equation}
              where:
              \begin{enumerate}
                  \item $\mora_e : \setA \sto \setB$ is the function computed;
                  \item $\sizetran: \Sigma_{\setA} \toinPos \Sigma_{\setB}$ is a \SY{monotone function} that keeps track of how the size changes.
                  \item $\timefun: \Sigma_{\setA} \toinPos \TimeMonoidal$ is a \SY{monotone function} that gives computation time as a function of instance size;
              \end{enumerate}

        \item \emph{Composition}: The composition of
              \begin{equation}
                  \tupp{\mora_1, \sizetran_1,  \timefun_1  }
                  \qqand
                  \tupp{\morb_2, \sizetran_2,  \timefun_2  }
              \end{equation}
              is given by
              \begin{equation}
                  \tupp{
                      \mora_{1;2},
                      \sizetran_{1;2},
                      \timefun_{1,2}
                  },
              \end{equation}
              where
              \begin{equation}
                  \mora_{1;2} = \mora_1 \mthen \morb_2,
              \end{equation}
              \begin{equation}
                  \sizetran_{1;2} = \sizetran_1 \mthen \sizetran_2,
              \end{equation}
              and $\timefun_{1,2}$ is defined as
              \begin{equation}
                  \defmapperiodset{
                      \timefun_{1,2}
                  }{
                      \Sigma_{\setA}
                  }{
                      \TimeMonoidal
                  }{
                      \sigma_{\setA}
                  }{
                      \timefun_1(\sigma_{\setA}) \monpostimes_{\TimeMonoidal} \timefun_2(\sizetran_1(\sigma_{\setA}))
                  }
              \end{equation}
    \end{enumerate}
\end{definition}

\iflabelexists{subsec:monoidal-space-time}{
    \begin{remark}[Spoilers]
        This example will be extended in \cref{subsec:monoidal-space-time}, where we will add a notion of memory and parallel computation.
    \end{remark}
}
