% !TEX root = chapter-standalone.tex

\section{Union of Design Problems}

\linkvideo{spring2021-functorial-comp-b:solving-queries:or-and-and} % Join and Meet in DP

Let~$\adpa\colon \F{\posgenA}\profto \R{\posgenB}$ and~$\adpb\colon \F{\posgenA}\profto \R{\posgenB}$ be \SY{design problems}.
We define the \emph{union}~$\adpa \dpjoin \adpb$ to be the \SY{design problem} which is feasible whenever \emph{either}~$\adpa$ or~$\adpb$ is feasible.
This models~$\adpa$ and~$\adpb$ as interchangeable technologies: either we can replace the other.

\begin{definition}[Union of design problems]
    \label{def:dp-join}
    Given two \SY{design problems}~$\adpa \colon \F{\posgenA} \profto \R{\posgenB}$ and~$\adpb\colon \F{\posgenA} \profto \R{\posgenB}$, their \emph{union}~$\adpa \dpjoin \adpb\colon \F{\posgenA} \profto \R{\posgenB}$ is defined by
    \begin{equation}
        \label{eq:union-dp-eq}
        \defmapperiod{
            (\adpa \dpjoin \adpb)
        }{
            \F{\posgenA}\posop \Ptimes \R{\posgenB}
        }{
            \toinPos
        }{
            \Bool
        }{
            \tup{\FposgenAelop, \RposgenBel}
        }{
            \adpa(\FposgenAelop, \RposgenBel) \boolor \adpb(\FposgenAelop, \RposgenBel)
        }
    \end{equation}
\end{definition}

The union of \SY{design problems} is represented as in~\cref{fig:uniondp}.

\begin{figure}[h!]
    \centering
    \includesag{52_union}
    \caption{Diagrammatic representation of the union of \SY{design problems}. }
    \label{fig:uniondp}
\end{figure}
\toexclude{
    \begin{example}
        Jeb's Spaceship Parts is locked in a deadly rivalry with Starshow Bob to supply engines for the new X103 space orbiter.
        Neither knows the exact operational scenario that the X103 will encounter, but have provided a range of performance benchmarks for their engines (\cref{fig:exunion_1}).
        \begin{figure}[h!]
            \centering
            \includesag{50_rival_jeb_bob}
            \caption{Example of two engine producers. }
            \label{fig:exunion_1}
        \end{figure}
        Back at NASA headquarters, Beau has uploaded Jeb and Bob's data in order to construct the \SY{design problem} reported in~\cref{fig:exunion_2}.
        \begin{figure}[h!]
            \centering
            \includesag{50_rival_beau}
            \caption{Example of the union of the engine \SY{design problems}. }
            \label{fig:exunion_2}
        \end{figure}
    \end{example}
}

\section{Intersection of Design Problems}

Given two \SY{design problems}~$\adpa, \adpb \colon \F{\posgenA} \profto \R{\posgenB}$, we can define a \SY{design problem}~$\adpa \dpmeet \adpb$ that is feasible if and only if~$\adpa$ and~$\adpb$ are both feasible.
We call~$\adpa \dpmeet \adpb$ the \emph{intersection} of~$\adpa$ and~$\adpb$.
One interpretation of~$\adpa \dpmeet \adpb$ is that~$\adpa$ and~$\adpb$ are two slightly different models of the same process, and we want to make sure that the design is conservatively feasible for both models.

\begin{definition}[Intersection of design problems ]
    \label{def:intersection_dp}
    \label{def:dp-meet}
    Given \SY{design problems}~$\adpa\colon \F{\posgenA} \profto \R{\posgenB}$ and~$\adpb\colon \F{\posgenA} \profto \R{\posgenB}$,
    their \emph{intersection} is denoted~$(\adpa \dpmeet \adpb)\colon \F{\posgenA} \profto \R{\posgenB}$, defined by:
    \begin{equation}
        \defmapperiod{
            (\adpa \dpmeet \adpb)
        }{
            \F{\posgenA}\posop \Ptimes \R{\posgenB}
        }{
            \toinPos
        }{
            \Bool
        }{
            \tup{\FposgenAelop, \RposgenBel}
        }{
            \adpa(\FposgenAelop, \RposgenBel) \booland \adpb(\FposgenAelop, \RposgenBel)
        }
    \end{equation}
\end{definition}
The intersection of \SY{design problems} is represented as in~\cref{fig:intersectiondp}.

\begin{figure}[h!]
    \centering
    \includesag{52_intersection}
    \caption{Diagrammatic representation of the intersection of \SY{design problems}. }
    \label{fig:intersectiondp}
\end{figure}

We can directly generalize the intersection~$\adpa \dpmeet \adpb$ by allowing~$\adpa$ and~$\adpb$ to have different domain and codomains,~$\adpa \colon \F{\posgenA} \profto \R{\posgenB}$ and~$\adpb \colon \F{\posgenC} \profto \R{\posgenD}$.
We call this putting two \SY{design problems} ``in parallel''.
