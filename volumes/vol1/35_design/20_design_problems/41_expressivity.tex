\subsection{Expressivity of  design problems}

The results are significant because CDPIs induce a rich family of optimization problems.

We are not assuming, let alone strong properties like convexity, even weaker properties like differentiability or continuity of the constraints.
In fact, we are not even assuming that functionality and resources are continuous spaces; they could be arbitrary discrete \SY{posets}.
(Indeed, in that case, completeness and \SY{Scott continuity} are trivially satisfied.)

Moreover, even assuming topological continuity of all spaces and maps considered, CDPIs are strongly not convex.
What makes them nonconvex is the possibility of introducing feedback interconnections.
To show this, we will give an example of a 1-dimensional problem with a continuous~\ftor for which the feasible set is disconnected (and in particular non-convex).
\todomistake{\bernina: not use $\ftor$}
\begin{marginfigure}
    \centering
    %\subfloat[\label{fig:Simple-DP}]{\centering{}\includegraphics[scale=0.43]{gmcdptro_nonconvex1b}}
    \subfloat[\label{fig:Simple-DP}]{\centering{}\includesag{simple-dp-nonconvex}} \\
    \subfloat[\label{fig:nonconvex3}]{\centering{}\includegraphics[scale=0.43]{gmcdptro_nonconvex3}}
    \caption{One feedback connection and a topologically continuous~\ftor are sufficient to induce a disconnected feasible set.}
    \label{fig:ceil-1}
\end{marginfigure}

\begin{example}[Non-convexity]
    \label{exa:one}
    Consider the CDPI in \cref{fig:Simple-DP}.
    The \uline{m}inimal resources~$M$ are the objectives of this optimization problem:
    \begin{equation}
        M\definedas
        \begin{cases}
            \with          & \fun,\res\setin\funsp=\ressp, \\
            \Min_{\posleq} & \res,                         \\
                           & \res\setin\ftor(\fun),        \\
                           & \res\posleq\fun.
        \end{cases}
    \end{equation}
    The feasible set~$\feasibleset{}\setsubseteq\funsp\Ptimes \ressp$ is the set of functionality and resources that satisfy the constraints~$\res\setin\ftor(\fun)$ and~$\res\posleq\fun$:
    %
    \begin{equation}
        \feasibleset{}=\makesett{ \tup{\fun,\res}\setin\funsp\Ptimes \ressp\colon (\res\setin\ftor(\fun))\booland(\res\posleq\fun)} .
        \label{eq:feasible}
    \end{equation}
    %
    The \uline{p}rojection~$\feasiblesetprojfun$ of~$\feasibleset{}$ to the functionality space is:
    %
    \begin{equation}
        \feasiblesetprojfun=\makesett{ \fun\mid\tup{\fun,\res} \setin\feasibleset{}}.
    \end{equation}
    In the scalar case ($\funsp=\ressp=\tupp{\nonNegRealsComp,{{\Rleq}}}$), the map~$\ftor\colon\funsp\toinPos\uppersets \ressp$ is simply a map~$\ftor\colon\F{\nonNegRealsComp}\to \uppersets\R{\nonNegRealsComp}$.
    The set~$\feasiblesetprojfun$ of feasible functionality is described by
    \begin{equation}
        \feasiblesetprojfun=\makeset{\fun\setin\nonNegRealsComp\colon \ftor(\fun)\leq\fun}.
        \label{eq:Pfeasible}
    \end{equation}
    \cref{fig:nonconvex3} shows an example of a continuous map~\ftor that gives a disconnected feasible set~$\feasiblesetprojfun$.
    Moreover,~$\feasiblesetprojfun$ is disconnected under any order-preserving nonlinear re-parametrization.

\end{example}
\todotext{@AC: redo example with implementation space.}
