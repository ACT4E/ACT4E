% !TEX root = chapter-standalone.tex

\section{What is ``co-design''?}
\label{sec:design-what-is-co-design}

The word ``co-design'' is not a new one.
In this book, we will use a meaning that incorporates and extends the existing meaning.

We take the ``Co'' in ``co-design'' to have four meanings:
\begin{enumerate}
    \item ``co'' for ``compositional'';
    \item ``co'' for ``collaborative'';
    \item ``co'' for ``computational'';
    \item ``co'' for ``continuous''.
\end{enumerate}
These meanings together describe the aspects of modern engineering design.

\subsection{``Co'' for ``compositional''}

The first meaning has to do with composition:

\begin{quote}
    co-design = design everything together
\end{quote}

We use the word ``co-design'' to refer to any decision procedure that has to do with making simultaneous choices about the components of a system to achieve system-level goals.
This includes the choice of components, the interconnection of components, and the configuration of components.
We will see that in most cases, choices that are made at the level of components without looking at the entire system are doomed to be suboptimal.
\iflabelexists{subsec:introduction-systems-and-components}{
    Slightly modifying Haiken's quote in~\cref{subsec:introduction-systems-and-components}, we choose this as our slogan:
}
\iflabeldoesnotexist{subsec:introduction-systems-and-components}{
    Slightly modifying this quote:
    \begin{quote}
        \emph{A system is composed of components; \\
            a component is something you understand.}\\ \\
        Howard Aiken
    \end{quote}
    We choose this as our quote:
}

\begin{quote}
    A system is composed of components;\\
    a component is something you understand
    \textbf{how to design}.
\end{quote}

\subsection{``Co'' for ``collaborative''}

In a second broad meaning, ``co-'' stands for ``collaborative'':

\begin{quote}
    co-design = design everything, together
\end{quote}

There are two types of collaborations.
First, there is the collaboration between human and machine, in the definition and solution of \SY{design problems}.
Second, and most importantly, is the collaboration among different experts or teams of experts in the design process.

The typical situation is that the system design is suboptimal because every expert only knows one component and there are rigid interfaces/contracts designed early on.
The problem here is sharing of knowledge across teams, specifically, knowledge about the design of systems.

In this case, this is the slogan:

\begin{quote}
    \enquote{A system is composed of components;\\
        a component is something that \textbf{somebody} understands
        \textbf{how to design}.
    }
\end{quote}

There is a tight link between the ``composition'' and ``collaboration'' aspects.

As Conway\footnote{John Horton Conway (1937--2020) was a mathematician.
    Probably the most popular idea of his was the invention of the Game of Life, which inspired countless works on cellular automata.
    We remember him for the discovery of the \emph{surreal numbers}, which should be just called \emph{numbers}, as they contain all other ordered fields.
} first observed for software systems:

\begin{quote}
    \enquote{Organizations which design systems [\dots] are constrained to produce designs which are copies of the communication structures of these organizations.}
\end{quote}

This ``mirroring'' hypothesis between system and organization was explored formally and found to hold~\cite{maccormack12exploring}.
The ultimate reason is that ``the organization's governance structures, problem-solving routines and communication patterns constrain the space in which it searches for new solutions''.
This appears to be true for generic systems in addition to software.

In the end, civilization is about dividing up the work, and so we must choose where one's work ends and the other's work begins.
But we need to keep talking if we want that everything works together.

\subsection{``Co'' for ``computational''}

The third meaning of ``co-'' in ``co-design'' will be \textbf{computational}.
It is the age of machines, and we need machines to understand what we are doing.

Therefore, we strive to create not only a qualitative modeling for co-design, but also a formal and quantitative description that will be suitable for setting up an optimization problem that can be solved to obtain an optimal design.

Our slogan becomes:

\begin{quote}
    \enquote{A system is composed of components;\\
        a component is something that \textbf{somebody} understands
        \textbf{how to design} \textbf{well enough to teach a computer}.
    }
\end{quote}

\subsection{``Co'' for ``continuous''}

The fourth meaning of ``co-'' is \textbf{continuous}.
We look at designs not as something that exists as a single decision in time, but rather as something that continuous to exist and evolve, independently on the designer.

% <!-- Also, in addition to considering design as a creative process,
% typical
% some of the theory to be developed is used in post-hoc analysis.
% In the biological sciences the system is already designed (evolved)
% but the idea of biological organisms as systems is pretty clear. -->

% <!--
% Complexity also depends on what exactly is the task at hand with which we are confronted. Given
% a system, is the problem to simulate it? to decompose it? to predict future states? to assess
% the causes of a failure? -->
