% !TEX root = chapter-standalone.tex

\section{The category of  design problems \DP}

Finally, we can declare that the \SY{design problems} so defined are morphisms in a category that we call~\DP.

% We will show that the class of all \SY{design problems} forms a category, which we call~\iindex{\DP}.

\begin{definition}[Category of design problems \DP]
    \SYNDEF{category of design problems}
    \label{def:DP}
    \SYNDEF{design problem}
    The category of design problems \DP consists of the following constituents:
    \begin{enumerate}
        \item \emph{Objects}: The objects of~\DP are \SY{posets}.
        \item \emph{Morphisms}: The morphisms of~\DP are \SY{design problems} (\cref{def:design-problem}).
        \item \emph{Identity morphism}: The \SY{identity morphism}~$\dpid_\posgenA \colon \funposA \profto \resposA$ is given by \cref{def:dp-identity}.
        \item \emph{Composition operation}: Given morphisms~$\adpa \colon  \funposA \profto \resposB$ and~$\adpb \colon \funposB \profto \resposC$, their composition~$\adpa \dpthen \adpb\colon \funposA\profto \resposC$ is given by \cref{def:dp-series}.
    \end{enumerate}
\end{definition}

We have already shown that the composition operator ``$\dpthen$'' is \SY{associative} and \SY{unital}, and that the composition of two \SY{design problems} is a \SY{design problem} (closure).
Therefore, \DP is a category.

\DP is called \feas or~$\Prof_{\Bool}$ in~\cite{fong2019}.

% \iflabelexists{sec:DPI-semicat}{
%     \begin{remark}
%         We will see in \cref{sec:DPI-semicat} that the DPIs (design problems with implementations) also have a categorical interpretation as a semicategory \DPI and that there is a functor from~\DPI to~\DP.
%     \end{remark}
% }

\subsection{Relation between \DPI and \DP}

We have already seen in \cref{rem:DP-from-DPI} that we can obtain a DP from a DPI.
We can make this more formal and say that there exists a \SY{forgetful semifunctor} from \DPI to \DP.

\begin{definition}
    \label{def:dpitodpsemi}
    The \SY{forgetful semifunctor}~$\dpitodp\colon \DPI\fto \DP$ is given by:
    \begin{enumerate}
        \item Identity on the objects:~$\dpitodpob(\posA)=\posA$.
        \item Given~$\adp=\tup{\funsp,\ressp,\impsp,\prov,\req}$, the action on morphisms is given by
              \begin{equation}
                  \defmapperiod{\dpitodpmor(\adp)}
                  {\funspop \Ptimes \ressp}
                  {\toinPos}
                  {\Bool}
                  {\tup{\fun\opel, \res}}
                  {\exists \imp \setin \setI\colon (\fun \posleqof{\funsp}\prov(\imp))\booland (\req(\imp)\posleqof{\ressp} \res)}
              \end{equation}
    \end{enumerate}
\end{definition}

\begin{lemma}
    \cref{def:dpitodpsemi} indeed defines a \SY{semifunctor}.
\end{lemma}
\begin{proof}
    Consider
    \begin{equation}
        \begin{aligned}
            \adpa & =\tup{\funposA,\resposB,\impspn{1},\provn{1},\reqn{1}}, \\
            \adpb & =\tup{\funposB,\resposC,\impspn{2},\provn{2},\reqn{2}},
        \end{aligned}
    \end{equation}
    We need to show
    \begin{equation}
        \dpitodpmor(\adpa \dpithenDPI \adpb)=\dpitodpmor(\adpa) \dpthenDP \dpitodpmor(\adpb).
    \end{equation}
    Let's start with the left term.
    One has
    \begin{equation}
        \label{eq:dpitodp_a}
        \begin{aligned}
            \dpitodpmor(\adpa \dpithenDPI \adpb)(\funposAel\F{\opel},\resposCel) & =
            \exists \imp \setin \impsp\colon (\funposAel \posleqof{\posA}\provn{1}(\impn{1}))\booland (\reqn{2}(\impn{2})\posleqof{\posC} \resposCel),
        \end{aligned}
    \end{equation}
    where~$\impsp=\makeset{\impn{1}\tupconcat \impn{2}\setin \makecprod{\impspn{1}, \impspn{2}} \mid \reqn{1}(\impn{1})\posleqof{\posB}\provn{2}(\impn{2})}$.

    On the other hand,
    \begin{equation}
        \label{eq:dpitodp_b}
        \begin{aligned}
             & (\dpitodpmor(\adpa) \dpthenDP \dpitodpmor(\adpb))(\funposAel\F{\opel},\resposCel) \\
             & =\bigvee_{\posBel \setin \posBset}\dpitodpmor(\adpa)(\funposAel\F{\opel},\resposBel)\booland \dpitodpmor(\adpb)(\funposBel\F{\opel},\resposCel) \\
             & =\bigvee_{\posBel \setin \posBset} \pars{{\exists \impn{1} \setin \impspn{1}\colon (\funposAel \posleqof{\posA}\provn{1}(\impn{1}))\booland (\reqn{1}(\impn{1})\posleqof{\posB} \resposBel)}} \\
             & \booland \pars{{\exists \impn{2} \setin \impspn{2}\colon (\funposBel \posleqof{\posB}\provn{2}(\impn{2}))\booland (\reqn{2}(\impn{2})\posleqof{\posC} \resposCel)}}
        \end{aligned}
    \end{equation}
    Consider the following cases:
    \begin{itemize}
        \item If~$\dpitodpmor(\adpa \dpithenDPI \adpb)(\funposAel\F{\opel},\resposCel)=\true$, there exist~$\impn{1}\setin \impspn{1},\impn{2}\setin \impspn{2}$ for which
              \begin{equation}
                  \begin{aligned}
                      \funposAel         & \posleqof{\posA}\provn{1}(\impn{1}), \\
                      \reqn{2}(\impn{2}) & \posleqof{\posC} \resposCel, \\
                      \reqn{1}(\impn{1}) & \posleqof{\posB}\provn{2}(\impn{2}).
                  \end{aligned}
              \end{equation}
              The first two terms are clear, and the last term implies that there exists a~$\posBel\setin \posB$ such that
              \begin{equation}
                  (\funposBel\posleqof{\posB}\provn{2}(\impn{2}))
                  \booland (\reqn{1}(\impn{1})\posleqof{\posB}\resposBel),
              \end{equation}
              implying~$(\dpitodpmor(\adpa) \dpthenDP \dpitodpmor(\adpb))(\funposAel\F{\opel},\resposCel)=\true$.
        \item The case
              \begin{equation}
                  \prftree{\dpitodpmor(\adpa \dpithenDPI \adpb)(\funposAel\F{\opel},\resposCel)=\false}{(\dpitodpmor(\adpa) \dpthenDP \dpitodpmor(\adpb))(\funposAel\F{\opel},\resposCel)=\false}
              \end{equation}
              follows analogously.
        \item The other direction is easier to show, since clearly
              \begin{equation}
                  \prftree{(\dpitodpmor(\adpa) \dpthenDP \dpitodpmor(\adpb))(\funposAel\F{\opel},\resposCel)=\true}{\dpitodpmor(\adpa \dpithenDPI \adpb)(\funposAel\F{\opel},\resposCel)=\true}
              \end{equation}
              and
              \begin{equation}
                  \prftree{(\dpitodpmor(\adpa) \dpthenDP \dpitodpmor(\adpb))(\funposAel\F{\opel},\resposCel)=\false}{\dpitodpmor(\adpa \dpithenDPI \adpb)(\funposAel\F{\opel},\resposCel)=\false}
              \end{equation}
              by inspecting \cref{eq:dpitodp_a} and \cref{eq:dpitodp_b}.
    \end{itemize}
\end{proof}

In the other direction, we can take a DP and find a corresponding DPI.

Given a DP~$\adp\colon \funspop \Ptimes \ressp \toinPos \Bool$, we can define a DPI $\tup{\funsp,\ressp, \impsp,\prov,\req}$
by setting
\begin{equation}
    \begin{aligned}
        \impsp & =\makeset{\tup{\fun,\res}\setin \funsp \cartprod \ressp \colon \adp(\fun\F{\opel},\res)}, \\
        \prov  & \colon \tup{\fun,\res}\mapsto \fun, \\
        \req   & \colon \tup{\fun,\res}\mapsto \res,
    \end{aligned}
\end{equation}
However, this operation is not a semifunctor, since it does not preserve composition.

\devel {

    We obtain another \SY{semifunctor}.

    \begin{definition}
        \label{def:dptodpisemi}
        The \SY{semifunctor}~$\dptodpi\colon \DP\fto \DPI$ is given by:
        \begin{enumerate}
            \item Identity on the objects:~$\dpitodpob(\posA)=\posA$.
            \item Given~$\adp\colon \funspop \Ptimes \ressp \toinPos \Bool$, the action on morphisms is given by
                  \begin{equation}
                      \dptodpimor(\adp)=\tup{\funsp,\ressp, \impsp,\prov,\req},
                  \end{equation}
                  where
                  \begin{equation}
                      \begin{aligned}
                          \impsp & =\makeset{\tup{\fun,\res}\setin \funsp \cartprod \ressp \colon \adp(\fun\F{\opel},\res)}, \\
                          \prov  & \colon \tup{\fun,\res}\mapsto \fun, \\
                          \req   & \colon \tup{\fun,\res}\mapsto \res,
                      \end{aligned}
                  \end{equation}
        \end{enumerate}
    \end{definition}

    \begin{lemma}
        \cref{def:dptodpisemi} indeed defines a \SY{semifunctor}.
    \end{lemma}
    \todotext{
        @Gioele: the last passage in the proof is wrong. (the two sets have different dimensions!)
    }
    \begin{proof}
        Consider~$\adpa\colon \funposA\Ptimes \resposB \toinPos \Bool$ and~$\adpb\colon \funposB\Ptimes \resposC\toinPos \Bool$.
        We need to show
        \begin{equation}
            \dptodpimor(\adpa \dpthenDP \adpb)=\dptodpimor(\adpa)\dpithenDPI \dptodpimor(\adpb).
        \end{equation}
        Let's start with the left term.
        One has
        \begin{equation}
            \dptodpimor(\adpa \dpthenDP \adpb)=\tup{\funposA,\resposC,\impsp,\prov,\req},
        \end{equation}
        where
        \begin{equation}
            \begin{aligned}
                \impsp & =\makeset{\tup{\funposAel,\resposCel}\setin \funposA \cartprod \resposC \colon (\adpa \dpthenDP \adpb)(\funposAel,\resposCel)}, \\
                       & =\makeset{\tup{\funposAel,\resposCel}\setin \funposA \cartprod \resposC \colon \bigvee_{\posBel\setin \posB}\adpa(\funposAopel,\resposBel)\booland \adpb(\funposBopel,\resposCel)} \\
                \prov  & =\projstart, \\
                \req   & =\projend,
            \end{aligned}
        \end{equation}
        where~$\projstart$ is the projection mapping returning the first element of a tuple, and~$\projend$ is the projection mapping returning the last element of a tuple.
        For the right-hand side, instead, one has
        \begin{equation}
            \dptodpimor(\adpa) \dpithenDPI \dptodpimor(\adpb)=\tup{\funposA,\resposC,\impsp\colI{'},\prov\F{'},\req\R{'}},
        \end{equation}
        where
        \begin{equation}
            \begin{aligned}
                \impsp\colI{'} & =\makeset{\impn{1}\tupconcat \impn{2}\setin \makecprod{\impspn{1}, \impspn{2}}\mid \reqn{1}(\impn{1})\posleqof{\posB}\provn{2}(\impn{2})} \\
                \impspn{1}     & =\makeset{\tup{\funposAel,\resposBel}\setin \funposA \cartprod \resposB \colon \adpa(\funposAel,\resposBel)}, \\
                \impspn{2}     & =\makeset{\tup{\funposBel,\resposCel}\setin \funposB \cartprod \resposC \colon \adpb(\funposBel,\resposCel)}, \\
                \provn{1}      & =\projstart \\
                \provn{2}      & =\projstart \\
                \reqn{1}       & =\projend \\
                \reqn{2}       & =\projend \\
                \prov\F{'}     & =\projstart \\
                \req\R{'}      & =\projend.
            \end{aligned}
        \end{equation}
        One can already notice that~$\prov=\prov\F{'}$ and~$\req=\req\R{'}$.
        To finish the proof, we can massage~$\impsp\colI{'}$:
        \begin{equation}
            \begin{aligned}
                \impsp\colI{'} & =\makeset{\impn{1}\tupconcat \impn{2}\setin \makecprod{\impspn{1}, \impspn{2}}\mid \reqn{1}(\impn{1})\posleqof{\posB}\provn{2}(\impn{2})} \\
                               & =\makeset{\tup{\funposAel,\resposBel,\funposBel\F{'},\resposCel}\setin \funposA \cartprod \resposB\cartprod \funposB\cartprod \resposC \colon \adpa(\funposAel\F{\opel},\resposBel) \booland \adpb(\funposBel\F{\opel},\resposCel)
                \booland \resposBel \posleqof{\posB}\funposBel\F{'}} \\
                               & =\makeset{\tup{\funposAel,\resposCel}\setin \funposA \cartprod \resposC \colon \bigvee_{\posBel\setin \posB}\adpa(\funposAopel,\resposBel)\booland \adpb(\funposBopel,\resposCel)} \\
                               & =\impsp.
            \end{aligned}
        \end{equation}

    \end{proof}
}