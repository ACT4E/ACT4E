% !TEX root = chapter-standalone.tex

\section[Companion and conjoint use]{Union and intersection with companions and conjoints}

We can also re-define the sum~$\join$ and intersection~$\meet$ using \SY{companions} and conjoints, which allows us to introduce some useful constructions.

\begin{definition}[Diagonal function]
    \label{def:diagonal-function}
    Define the \maindef{diagonal function}~$\diag_\posA\colon \posA \to {\posA \Ptimes \posA}$:
    \begin{equation}
        \begin{aligned}
            \diag_\posA \colon \posA & \to {\posA \Ptimes \posA}, \\
            \posAel                  & \mapsto \tup{\posAel, \posAel}.
        \end{aligned}
    \end{equation}
\end{definition}

\begin{definition}[Codiagonal function]
    \label{def:codiagonal-function}
    Define the \maindef{codiagonal function}~$\codiag_\posA\colon \posA \Pplus \posA \to \posA $:
    \begin{equation}
        \begin{aligned}
            \codiag_\posA \colon \posA \Pplus \posA & \to \posA, \\
            \disunionA{\posAel}                     & \mapsto \posAel, \\
            \disunionB{\posAel}                     & \mapsto \posAel.
        \end{aligned}
    \end{equation}
\end{definition}

Using the diagonal function, \cref{eq:union-dp-eq} can be rewritten as the following lemma.

\begin{lemma}
    Given~$\adpa, \adpb\colon \funposA \profto \resposB$, we have:
    \begin{equation}
        \adpa \dpjoin \adpb = \conj{\diag_\posA} \dpthen (\adpa + \adpb)\dpthen \comp{\diag_\posB}.
    \end{equation}
\end{lemma}
\begin{proof}
    First, note that
    \begin{equation}
        \begin{aligned}
            \conj{\diag_\posA}\colon \funposA                       & \profto \resposA+\resposA \\
            \tup{\funposAel_\F{1}\Fop,\disunionA{\resposAel_\R{2}}} & \mapsto \funposAel_\F{1}\posleq \resposAel_\R{2} \\
            \tup{\funposAel_\F{1}\Fop,\disunionA{\resposAel_\R{3}}} & \mapsto \funposAel_\F{1}\posleq \resposAel_\R{3}
        \end{aligned}
    \end{equation}
    and
    \begin{equation}
        \begin{aligned}
            \comp{\diag_\posB}\colon \funposB+\funposB              & \profto \resposB \\
            \tup{\disunionA{\funposBel_\F{1}\Fop,\resposBel_\R{3}}} & \mapsto \funposBel_\F{1}\posleq \resposBel_\R{3} \\
            \tup{\disunionB{\funposBel_\F{2}\Fop,\resposBel_\R{3}}} & \mapsto \funposBel_\F{2} \posleq \resposBel_\R{3}.
        \end{aligned}
    \end{equation}
    We start by looking at~$\underbrace{\conj{\diag_\posA}\dpthen (\adpa+\adpb)}_{\star}\colon \funposA \profto \resposB+\resposB$.
    \begin{equation}
        \begin{aligned}
             &
            \star (\tup{\funposAel\Fop,\resposBel}) \\& =\bigvee_{\posAel\elprime\setin \posA\Pplus\posA} \conj{\diag_\posA}(\tupp{\funposAel\Fop,\resposAel\F{'}})\booland (\adpa+\adpb)(\tupp{\funposAel\F{'^*},\resposBel}) \\
             & =\pars{ \bigvee_{\disunionA{\posAel\elprime}\setin \posA\Pplus\posA} (\funposAel\posleq \resposAel\R{'})\booland \adpa(\funposAel\F{'^*},\resposBel) }\boolor \pars{ \bigvee_{\disunionB{\posAel\elprime}\setin \posA\Pplus\posA} (\funposAel\posleq \resposAel\R{'})\booland \adpb(\funposAel\F{'^*},\resposBel) } \\
             & =\adpa(\funposAel\Fop,\resposBel) \boolor \adpb(\funposAel\Fop,\resposBel).
        \end{aligned}
    \end{equation}
    %
    We now look at~$\star \dpthen \comp{\diag_\posB}\colon \funposA \profto \resposB$:
    \begin{equation}
        \begin{aligned}
             & (\star \dpthen \comp{\diag_\posB})(\funposAel\Fop,\resposBel') \\
             & =\bigvee_{\posBel\setin \posB+\posB} \star(\funposAel\Fop,\resposBel)\booland \comp{\diag_\posB}(\funposBel\Fop,\resposBel\R{'}) \\
             & =\pars{\bigvee_{\disunionA{\posBel}\setin \posB+\posB} \adpa(\funposAel\Fop,\resposBel) \booland (\funposBel\posleq \resposBel\R{'})} \boolor
            \pars{\bigvee_{\disunionB{\posBel}\setin \posB+\posB} \adpb(\funposAel\Fop,\resposBel) \booland (\funposBel\posleq \resposBel\R{'})} \\
             & =\adpa(\funposAel\Fop,\resposBel\R{'})\boolor \adpb(\funposAel\Fop,\resposBel\R{'}).
        \end{aligned}
    \end{equation}
\end{proof}

\devel{
    Similarly, using the codiagonal function, we can prove the following.
    \begin{lemma}
        Given~$\adpa, \adpb\colon \funposA \profto \resposB$, we have:
        \begin{equation}
            \adpa \dpmeet \adpb = \comp{\codiag_\posgenA} \dpthen(\adpa + \adpb) \dpthen \conj{\codiag_\posgenB}.
        \end{equation}
    \end{lemma}
    \missingproof
    \devel{
        \begin{proof}
            First, note that
            \begin{equation}
                \begin{aligned}
                    \comp{\codiag_\posgenA} \colon \funposA        & \profto \resposA\cartprod \resposA \\
                    \tup{\funposAel_1^*,\resposAel_2,\resposAel_3} & \mapsto \codiag_\posgenA(\funposAel_1)\posleq \tup{\resposAel_2,\resposAel_3} \\
                                                                   & = \tup{\funposAel_1,\funposAel_1}\posleq \tup{\resposAel_2,\resposAel_3} \\
                                                                   & = (\funposAel_1\posleq \resposAel_2) \booland (\funposAel_1\posleq \resposAel_3)
                \end{aligned}
            \end{equation}
            and
            \begin{equation}
                \begin{aligned}
                    \conj{\codiag_\posgenB} \colon \funposB\cartprod \funposB & \profto \resposB \\
                    \tup{\tup{\funposBel_1,\funposBel_2}^*,\resposBel_3}      & \mapsto \tup{\funposBel_1,\funposBel_2}\posleq \codiag_\posgenB(\resposBel_3) \\
                                                                              & = (\funposBel_1\posleq \resposBel_3) \booland (\funposBel_2\posleq \resposBel_3).
                \end{aligned}
            \end{equation}
            We start by looking at~$\comp{\codiag_\posgenA} \dpthen (\adpa+\adpb) \colon \funposA\profto \resposB+\resposB$:
            \begin{equation}
                \begin{aligned}
                    \pars{\comp{\codiag_\posgenA}\dpthen (\adpa+\adpb)}\pars{\tup{\funposAel^*,\resposBel}} & =\bigvee_{} \XXX
                \end{aligned}
            \end{equation}
            \todotextjira{418}{\bernina: @Gioele: Adjust signatures, have to find a good way to write it down}
        \end{proof}
        Unlike~$\conj{\diag} = \mathsf{split}$ and $\comp{\diag} = \mathsf{fuse}$,~$\comp{\codiag}$ and $\conj{\codiag}$ do not have an intuitive diagrammatic representation.
    }}
