% !TEX root = chapter-standalone.tex

\section{Bool-enriched categories}
\label{sec:bool-enrichment}

\begin{ctdefinition}[Bool-category]
    \label{def:bool-category}
    A $\Bool$-category $\CatE$ is:

    \constit
    \begin{enumerate}
        \item a set of objects $\Ob_\CatE$;
        \item for every pair of objects $\tup{\Obja, \Objb}$ an element $\CatE(\Obja, \Objb) \setin\makeset{\false, \true}$;
        \item for every triple of objects $\tup{\Obja, \Objb, \Objc}$, a morphism in $\Bool$ of the type
              \begin{equation}
                  \CatE(\Obja, \Objb) \booland \CatE(\Objb, \Objc) \posleq \CatE(\Obja, \Objc);
              \end{equation}
        \item for every object $\Obja$, a morphism in $\Bool$ of the type
              \begin{equation}
                  \true \posleq \CatE(\Obja, \Obja).
              \end{equation}
    \end{enumerate}
\end{ctdefinition}

In the above definition one might argue that the last two items could be labeled as \emph{conditions} rather than \emph{constituents}.
We treat them constituents here in order to emphasize that we are thinking of the given inequalities not as logical statements but rather as pieces of data -- morphisms -- that come from $\Bool$ (if they exist, these morphisms are, however, unique).
When we generalize $\Bool$-categories to $\CatV$-categories, with $\CatV$ a symmetric monoidal category, the above inequalities will be replaced by morphisms in $\CatV$.

Note in particular that the fourth item above,
\begin{equation}
    \true \posleq \CatE(\Obja, \Obja)
\end{equation}
is equivalent with the logical statement
\begin{equation}
    \CatE(\Obja, \Obja) = \true.
\end{equation}

However, we write it in this form -- in terms of a morphism, and not an isomorphism, in $\Bool$ -- because this reflects the general pattern that will emerge when we generalize to $\CatV$-categories.

\begin{comment}
\begin{ctdefinition}
    \label{def:bool-functor}
    A $\Bool$-functor $\funa \colon \CatE \fto \CatF$ between $\Bool$-categories is:

    \constit
    \begin{enumerate}
        \item a function $\funa \colon \Ob_\CatE \to \Ob_\CatF$;
        \item for every pair of objects $\tup{\Obja, \Objb}$ of $\CatE$, a morphism in $\Bool$ of the type
              \begin{equation}
                  \CatE(\Obja, \Objb) \leq \CatF(\funa \Obja, \funa \Objb).
              \end{equation}
    \end{enumerate}
\end{ctdefinition}

\begin{ctdefinition}
    \label{def:bool-cat-opposite}
    Let $\CatE$ be a $\Bool$-category.
    Its opposite $\CatE\op$ is the $\Bool$-category with
    \begin{enumerate}
        \item $\Ob_{\CatE\op} = \Ob_\CatE$;
        \item $\CatE\op(\Obja, \Objb) = \CatE(\Objb, \Obja)$.
    \end{enumerate}

    The morphisms
    \begin{equation}
        \CatE\op(\Obja, \Objb) \wedge \CatE\op(\Objb, \Objc) \leq \CatE\op(\Obja, \Objc)
    \end{equation}
    and
    \begin{equation}
        \true \leq \CatE\op(\Obja, \Obja)
    \end{equation}
    are determined from $\CatE$.
\end{ctdefinition}

\begin{ctdefinition}
    \label{def:bool-cat-product}
    Given $\Bool$-categories $\CatE$ and $\CatF$, their product $\CatE \Ctimes \CatF$ is the $\Bool$-category with
    \begin{enumerate}
        \item $\Ob_{\CatE \Ctimes \CatF} = \Ob_\CatE \cartprod \Ob_\CatF$;
        \item $(\CatE \Ctimes \CatF)(\Obja, \Objb) = \CatE (\Obja, \Objb) \wedge \CatF (\Obja, \Objb)$ for every pair of objects $\Obja, \Objb$.
    \end{enumerate}
    Morphisms
    \begin{equation}
        (\CatE(\Obja, \Objb) \wedge \CatF(\Obja, \Objb)) \wedge (\CatE(\Objb, \Objc) \wedge \CatF(\Obja, \Objb)) \leq \CatE(\Obja, \Objc) \wedge \CatF(\Obja, \Objc)
    \end{equation}
    and
    \begin{equation}
        \true \leq \CatE(\Obja, \Obja) \wedge \CatF(\Obja, \Objb)
    \end{equation}
    are determined from $\CatE$ and $\CatF$.
\end{ctdefinition}

\end{comment}

