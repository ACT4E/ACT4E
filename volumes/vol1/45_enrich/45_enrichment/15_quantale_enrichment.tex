% !TEX root = chapter-standalone.tex

\section{Enrichment via monoidal posets}
\label{sec:quantale-enrichment}

\todotext{J: @J: maybe redo this section with pre-orders instead of posets?}

\begin{ctdefinition}
    \label{def:monoidal-poset-again}
    A monoidal poset is:

    \constit
    \begin{enumerate}
        \item A poset $\posgenA = \tup{\posAset, \posleq}$;
        \item A monotone map $\mtimescat \colon \posgenA \Ptimes \posgenA \mto \posgenA$, the \emph{monoidal product};
        \item An element $\idmoncat \setin \posAset$, the \emph{monoidal unit}.
    \end{enumerate}

    \condit
    \begin{enumerate}
        \item The monoidal product is associative:
              \begin{equation}
                  (\ela \mtimescat \elb) \mtimescat \elc = \ela \mtimescat (\elb \mtimescat \elc) \qquad \forall \ela, \elb, \elc \setin \posAset.
              \end{equation}
        \item The monoidal unit is neutral:
              \begin{equation}
                  \idmoncat \mtimescat \ela = \ela = \ela \mtimescat \idmoncat \qquad \forall \ela \setin \posAset.
              \end{equation}

    \end{enumerate}

\end{ctdefinition}

\begin{ctdefinition}[enriched category - monoidal poset version]
    \label{def:enriched-cat-monoidal-poset}
    Let $\CatV = \tup{\posAset, \posleq, \mtimescat, \idmoncat}$ be a monoidal poset.
    A $\CatV$-category $\CatE$ is:

    \constit
    \begin{enumerate}
        \item A set of objects $\Ob_\CatE$;
        \item For every pair of objects $\tup{\Obja, \Objb}$ an element $\CatE(\Obja, \Objb) \setin \Ob_\CatV$;
        \item For every triple of objects $\tup{\Obja, \Objb, \Objc}$, a morphism in $\CatV$ of the type
              \begin{equation}
                  \CatE(\Obja, \Objb) \mtimescat \CatE(\Objb, \Objc) \leq \CatE(\Obja, \Objc);
              \end{equation}
        \item For every object $\Obja$, a morphism in $\CatV$ of the type
              \begin{equation}
                  \idmoncat \leq \CatE(\Obja, \Obja).
              \end{equation}
    \end{enumerate}
\end{ctdefinition}

\begin{ctdefinition}
    \label{def:functor-monoidal-poset-enriched}
    Let $\CatV$ be a monoidal poset.
    A $\CatV$-functor $\funa \colon \CatE \fto \CatF$ between $\CatV$-categories is:

    \constit
    \begin{enumerate}
        \item A function $\funa \colon \Ob_\CatE \to \Ob_\CatF$;
        \item For every pair of objects $\tup{\Obja, \Objb}$ of $\CatE$, a morphism in $\CatV$ of the type
              \begin{equation}
                  \CatE(\Obja, \Objb) \leq \CatF(\funa \Obja, \funa \Objb).
              \end{equation}
    \end{enumerate}
\end{ctdefinition}

\begin{ctdefinition}
    \label{def:monoidal-poset-cat-op}
    Let $\CatV = \tup{\posAset, \posleq, \mtimescat, \idmoncat}$ be a symmetric monoidal poset and let $\CatE$ be a $\CatV$-category.
    Its opposite $\CatE\op$ is the $\CatV$-category with
    \begin{enumerate}
        \item $\Ob_{\CatE\op} = \Ob_\CatE$;
        \item $\CatE\op(\Obja, \Objb) = \CatE(\Objb, \Obja)$.
    \end{enumerate}

    The morphisms
    \begin{equation}
        \CatE\op(\Obja, \Objb) \mtimescat \CatE\op(\Objb, \Objc) \leq \CatE\op(\Obja, \Objc)
    \end{equation}
    and
    \begin{equation}
        \idmoncat \leq \CatE\op(\Obja, \Obja)
    \end{equation}
    are determined from $\CatE$.
\end{ctdefinition}

\begin{ctdefinition}
    \label{def:monoidal-poset-cat-product}
    Let $\CatV = \tup{\posAset, \posleq, \mtimescat, \idmoncat}$ be a symmetric monoidal poset.
    Given $\CatV$-categories $\CatE$ and $\CatF$, their product $\CatE \Ctimes \CatF$ is the $\CatV$-category with
    \begin{enumerate}
        \item $\Ob_{\CatE \Ctimes \CatF} = \Ob_\CatE \cartprod \Ob_\CatF$;
        \item $(\CatE \Ctimes \CatF)(\Obja, \Objb) = \CatE (\Obja, \Objb) \mtimescat \CatF (\Obja, \Objb)$ for every pair of objects $\Obja, \Objb$.
    \end{enumerate}
    The morphisms
    \begin{equation}
        (\CatE(\Obja, \Objb) \mtimescat \CatF(\Obja, \Objb)) \mtimescat (\CatE(\Objb, \Objc) \mtimescat \CatF(\Objb, \Objc)) \leq \CatE(\Obja, \Objc) \mtimescat \CatF(\Obja, \Objc)
    \end{equation}
    and
    \begin{equation}
        \idmoncat \leq \CatE(\Obja, \Obja) \mtimescat \CatF(\Obja, \Obja)
    \end{equation}
    are determined from $\CatE$ and $\CatF$.
\end{ctdefinition}

\subsection{Internal-homs}

\begin{ctdefinition}
    \label{def:left-internal-hom-monoidal-poset}
    Let $\posgenA = \tup{\posAset, \posleq, \mtimescat, \idmoncat}$ be a monoidal poset.
    A left internal-hom structure on $\posgenA$ is:

    \constit
    \begin{enumerate}
        \item A monotone map $\leftinthom{\ }{\ } \colon \posgenA\op \Ctimes \posgenA \mto \posgenA$;
    \end{enumerate}

    \condit
    \begin{enumerate}
        \item For every element $\ela \setin \posAset$, we have an adjunction
              \begin{equation}
                  \ela \mtimescat ( - ) \quad \adjunction \quad \leftinthom{x}{( - )}.
              \end{equation}
    \end{enumerate}
    A monoidal poset equipped with a left internal-hom structure is called \emph{left closed}.
\end{ctdefinition}

\begin{remark}
    \label{rem:left-internal-hom-unpacked-monoidal-poset}
    For fixed $\ela \setin \posAset$, the adjunction
    \begin{equation}
        \ela \mtimescat ( - ) \quad \adjunction \quad \leftinthom{x}{( - )}
    \end{equation}
    means that for all $\elb, \elc \setin \posAset$,
    \begin{equation}
        \ela \mtimescat \elb \posleq \elc \quad \text{if and only if} \quad \elb \posleq \leftinthom{\ela}{\elc},
    \end{equation}
    or, alternatively and equivalently, that for all $\elb \setin \posAset$,
    \begin{equation}
        \elb \posleq \leftinthom{\ela}{(\ela \mtimescat \elb)} \quad \text{and} \quad \ela \mtimescat (\leftinthom{\ela}{\elb}) \posleq \elb.
    \end{equation}
\end{remark}

\begin{ctdefinition}
    \label{def:right-internal-hom-monoidal-poset}
    Let $\posgenA = \tup{\posAset, \posleq, \mtimescat, \idmoncat}$ be a monoidal poset.
    A right internal-hom structure on $\posgenA$ is:

    \constit
    \begin{enumerate}
        \item A monotone map $\rightinthom{}{} \colon \posgenA\op \Ctimes \posgenA \mto \posgenA$;
    \end{enumerate}

    \condit
    \begin{enumerate}
        \item For every element $\ela \setin \posAset$, we have an adjunction
              \begin{equation}
                  ( - ) \mtimescat \ela \quad \adjunction \quad \rightinthom{(-)}{\ela} .
              \end{equation}
    \end{enumerate}
    A monoidal poset equipped with a right internal-hom structure is called \emph{right closed}.
\end{ctdefinition}

\begin{lemma}
    \label{lem:symmetric-left-right-closed-monoidal-poset-cat}
    Let $\posgenA = \tup{\posAset, \posleq, \mtimescat, \idmoncat}$ be a monoidal poset.
    If $\mtimescat$ is symmetric, then $\posgenA$ is left-closed if and only if it is right-closed.
\end{lemma}

\begin{ctdefinition}
    \label{def:moindal-poset-cat-closed}
    We call a symmetric monoidal poset simply a \emph{closed monoidal poset} if it is either left or right closed.
\end{ctdefinition}

\begin{lemma}
    \label{lem:monoidal-closed-poset-self-enriched}
    Let $\CatV$ be a left-closed monoidal poset.
    Then $\CatV$ itself may be viewed as a $\CatV$-enriched category $\CatE$ with:
    \begin{enumerate}
        \item $\Ob_\CatE = \Ob_\CatV$;
        \item For every pair of objects $\tup{\Obja \Objb}$ of $\CatE$, the hom-object $\CatE(\Obja, \Objb)$ defined as
              \begin{equation}
                  \CatE(\Obja, \Objb) = \leftinthom{\Obja}{\Objb}
              \end{equation}
    \end{enumerate}
\end{lemma}

\subsection{$\CatV$-profunctors}

\begin{ctdefinition}
    \label{def:profunctor-monoidal-poset-enriched}
    Let $\CatV = \tup{\stylesets{V}, \posleq, \mtimescat, \idmoncat, \leftinthom{\ }{\ }}$ be a closed monoidal poset, and let $\CatE$, $\CatF$ be $\CatV$-categories.
    A $\CatV$-profunctor $\profa$ from $\CatE$ to $\CatF$, denoted $\profa \colon \CatE \profto \CatF$, is a $\CatV$-functor
    \begin{equation}
        \profa \colon \CatE\op \Ctimes \CatF \fto \CatV.
    \end{equation}
    In more detail, a $\CatV$-profunctor $\profa \colon \CatE \mto \CatF$ is:

    \constit

    \begin{enumerate}
        \item A function $\profa \colon \Ob_{\CatE\op} \cartprod \Ob_{\CatF} \mto \Ob_\CatV$;
    \end{enumerate}

    \condit

    \begin{enumerate}
        \item For every pair of objects $\tup{\Obja, \Obja'}, \tup{\Objb, \Objb'}$ in $\CatE\op \Ctimes \CatF$, a morphism in $\CatV$ of the type
              \begin{equation}
                  \CatE\op(\Obja, \Objb) \mtimescat \CatF(\Obja', \Objb') \posleq \leftinthom{\profa(\Obja, \Obja')}{\profa(\Objb, \Objb')}.
              \end{equation}
    \end{enumerate}
\end{ctdefinition}

\begin{ctdefinition}
    \label{def:unital-commutative-quantale}
    Let $\CatV = \tup{\stylesets{V}, \posleq, \mtimescat, \idmoncat, \leftinthom{\ }{\ }}$ be a closed monoidal poset.
    It is called a \emph{unital commutative quantale} if it has all joins.

    This means that for any $\setA \subseteq \stylesets{V}$, the least upper bound $\bigvee\setA$ exists.
    In particular, $\bigvee \emptyset$ is a least element of $\CatV$.
\end{ctdefinition}

\begin{ctdefinition}
    \label{def:enriched-profunctor-composition-monoidal-poset}
    Let $\CatV = \tup{\stylesets{V}, \posleq, \mtimescat, \idmoncat, \leftinthom{\ }{\ }}$ be a unital commutative quantale, and let $\profa \colon \CatE \profto \CatF$ and $\profb \colon \CatF \profto \CatG$ be $\CatV$-profunctors between $\CatV$-categories.
    Their composition $\profa \fthen \profb \colon \CatE \profto \CatG$ is the profunctor defined on objects by
    \begin{equation}
        (\profa \fthen \profb)(\ela, \elc) = \bigvee_{\elb \setin \Ob_\CatF} (\profa (\ela, \elb) \mtimescat \profb (\elb, \elc)).
    \end{equation}
\end{ctdefinition}

