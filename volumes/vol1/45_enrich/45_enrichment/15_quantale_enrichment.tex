% !TEX root = chapter-standalone.tex

\section{Enrichment via monoidal posets}
\label{sec:quantale-enrichment}

\begin{ctdefinition}
A monoidal poset is: 

\constit
\begin{enumerate}
\item A poset $\posgenA = \tup{\posAset, \posleq}$;
\item A monotone map $\mtimescat \colon \posgenA \Ptimes \posgenA \mto \posgenA$, the \emph{monoidal product}; 
\item An element $\idmoncat \setin \posAset$, the \emph{monoidal unit}. 
\end{enumerate}

\condit
\begin{enumerate}
\item The monoidal product is associative: 
\begin{equation}
(\ela \mtimescat \elb) \mtimescat \elc = \ela \mtimescat (\elb \mtimescat \elc) \qquad \forall \ela, \elb, \elc \setin \posAset.
\end{equation}
\item The monoidal unit is neutral:
\begin{equation}
\idmoncat \mtimescat \ela = \ela = \ela \mtimescat \idmoncat \qquad \forall \ela \setin \posAset.
\end{equation}
 
\end{enumerate}


\end{ctdefinition}


\begin{ctdefinition}[enriched category - monoidal poset version]
Let $\CatV$ be a monoidal poset. A $\CatV$-category $\CatE$ is:

\constit
\begin{enumerate}
\item a set of objects $\Ob_\CatE$; 
\item for every pair of objects $\tup{\Obja, \Objb}$ an element $\CatE(\Obja, \Objb) \setin \Ob_\CatV$; 
\item for every triple of objects $\tup{\Obja, \Objb, \Objc}$, a morphism in $\CatV$ of the type
\begin{equation}
\CatE(\Obja, \Objb) \wedge \CatE(\Objb, \Objc) \leq \CatE(\Obja, \Objc);
\end{equation}
\item for every object $\Obja$, a morphism in $\CatV$ of the type
\begin{equation}
\true \leq \CatE(\Obja, \Obja).
\end{equation}
\end{enumerate}
\end{ctdefinition}

\begin{ctdefinition}
Let $\CatV$ be a monoidal poset. A $\CatV$-functor $\funa \colon \CatE \fto \CatF$ between $\CatV$-categories is:

\constit
\begin{enumerate}
\item a function $\funa \colon \Ob_\CatE \to \Ob_\CatF$;
\item for every pair of objects $\tup{\Obja, \Objb}$ of $\CatE$, a morphism in $\CatV$ of the type
\begin{equation}
\CatE(\Obja, \Objb) \leq \CatF(\funa \Obja, \funa \Objb). 
\end{equation}
\end{enumerate}
\end{ctdefinition}


\begin{ctdefinition}
Let $\CatV$ be a monoidal poset and let $\CatE$ be a $\CatV$-category. Its opposite $\CatE\op$ is the $\CatV$-category with 
\begin{enumerate}
\item $\Ob_{\CatE\op} = \Ob_\CatE$;
\item $\CatE\op(\Obja, \Objb) = \CatE(\Objb, \Obja)$.
\end{enumerate}

The morphisms 
\begin{equation}
\CatE\op(\Obja, \Objb) \wedge \CatE\op(\Objb, \Objc) \leq \CatE\op(\Obja, \Objc)
\end{equation}
and
\begin{equation}
\true \leq \CatE\op(\Obja, \Obja)
\end{equation}
are determined from $\CatE$.
\end{ctdefinition}

\begin{ctdefinition}
Let $\CatV$ be a commutative monoidal poset. Given $\CatV$-categories $\CatE$ and $\CatF$, their product $\CatE \Ctimes \CatF$ is the $\CatV$-category with 
\begin{enumerate}
\item $\Ob_{\CatE \Ctimes \CatF} = \Ob_\CatE \cartprod \Ob_\CatF$;
\item $(\CatE \Ctimes \CatF)(\Obja, \Objb) = \CatE (\Obja, \Objb) \wedge \CatF (\Obja, \Objb)$ for every pair of objects $\Obja, \Objb$. 
\end{enumerate}
The morphisms 
\begin{equation}
(\CatE(\Obja, \Objb) \wedge \CatF(\Obja, \Objb)) \wedge (\CatE(\Objb, \Objc) \wedge \CatF(\Obja, \Objb)) \leq \CatE(\Obja, \Objc) \wedge  \CatF(\Obja, \Objc)
\end{equation}
and
\begin{equation}
\true \leq \CatE(\Obja, \Obja) \wedge \CatF(\Obja, \Objb)
\end{equation}
are determined from $\CatE$ and $\CatF$.
\end{ctdefinition}


\begin{ctdefinition}
Let $\posgenA = \tup{\posAset, \mtimescat, \idmoncat}$ be a monoidal poset. A left internal-hom structure on $\posgenA$ is:

\constit
\begin{enumerate}
\item A monotone map $\leftinthom \colon \posgenA\op \Ctimes \posgenA \mto \posgenA$; 
\end{enumerate}

\condit
\begin{enumerate}
\item For every element $\ela \setin \posAset$, we have an adjunction 
\begin{equation}
\ela \mtimescat ( - ) \quad  \adjunction \quad \ela \leftinthom ( - ).
\end{equation}
\end{enumerate}
A monoidal poset equipped with a left internal-hom structure is called \emph{left closed}.
\end{ctdefinition}

\begin{ctdefinition}
Let $\posgenA = \tup{\posAset, \mtimescat, \idmoncat}$ be a monoidal poset. A right internal-hom structure on $\posgenA$ is:

\constit
\begin{enumerate}
\item A monotone map $\rightinthom \colon \posgenA\op \Ctimes \posgenA \mto \posgenA$; 
\end{enumerate}

\condit
\begin{enumerate}
\item For every element $\ela \setin \posAset$, we have an adjunction 
\begin{equation}
( - ) \mtimescat \ela \quad  \adjunction \quad ( - ) \rightinthom \ela.
\end{equation}
\end{enumerate}
A monoidal poset equipped with a right internal-hom structure is called \emph{right closed}.
\end{ctdefinition}

\begin{lemma}
Let $\posgenA = \tup{\posAset, \mtimescat, \idmoncat}$ be a monoidal poset. If $\mtimescat$ is commutative, then $\posgenA$ is left-closed if and only if it is right-closed.
\end{lemma}


\begin{lemma}
Let $\CatV$ be a left-closed monoidal poset. Then $\CatV$ itself may be viewed as a $\CatV$-enriched category $\CatE$ with:
\begin{enumerate}
\item $\Ob_\CatE = \Ob_\CatV$;
\item For every pair of objects $\tup{\Obja \Objb}$ of $\CatE$, the hom-object
\begin{equation}
\CatE(\Obja, \Objb) = \Obja \leftinthom \Objb.
\end{equation}
\end{enumerate}
\end{lemma}

