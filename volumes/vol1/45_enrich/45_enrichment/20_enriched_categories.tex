% !TEX root = chapter-standalone.tex


\section{Enriched categories}
\linkvideo{spring2021-enrichment:intro-exa-enrich} % Introductory example about routing
\linkvideo{spring2021-enrichment:recap-monoidal} % Recap on monoidal categories
\linkvideo{spring2021-enrichment:enrich-cat-def} % Definition of enriched category
\label{sec:enrichment-enriched-categories}

\begin{ctdefinition}[Enriched category]
    \label{def:enriched_cat}
    Let~$\tup{\Cat{V},\mtimescat, \idmoncat, \associator, \leftunitor,\rightunitor}$ be a monoidal category.
    A category~$\CatC$ \emph{enriched} in~$\Cat{V}$ is composed of:
    \begin{compactenum}
        \item The set of objects~$\Ob_\CatC$;
        \item For all~$\Obja,\Objb\in \Ob_\CatC$, an object~$\HomSet{\CatC}{\Obja}{\Objb}$, called the \emph{hom-object} from~$\Obja$ to~$\Objb$.
        \item For all~$\Obja, \Objb, \Objc\in \Ob_\CatC$, there exists a morphism~$\mthen_{\Obja,\Objb,\Objc}$ in~$\Cat{V}$:
        \begin{equation}
            \mthen_{\Obja,\Objb,\Objc}\colon \HomSet{\CatC}{\Obja}{\Objb}\mtimescat \HomSet{\CatC}{\Objb}{\Objc}\mto \HomSet{\CatC}{\Obja}{\Objc}.
        \end{equation}
        This is called \emph{composition morphism}.
        \item For each~$\Obja\in \Ob_\CatC$, a morphism~$\catid_\Obja\colon \idmoncat \mto \HomSet{\CatC}{\Obja}{\Obja}$, called \emph{identity element}.
    \end{compactenum}
    Furthermore, for any~$\Obja,\Objb,\Objc,\Objd\in \Ob_\CatC$, the following diagrams must commute.
\end{ctdefinition}

\begin{widepar}
    \begin{center}
        \includesag{enrichment_assoc}
    \end{center}

    \begin{center}
        \includesag{enrichment_unital}
    \end{center}
\end{widepar}

\begin{definition}
    \label{def:prod_enrich_cat}
    Let \CatA and \CatB be categories enriched in a symmetric monoidal category~$\Cat{V}$. Their \emph{product} is a~$\Cat{V}$-enriched category~$\CatA\times \CatB$ with:
    \begin{compactenum}
        \item $\Ob_{\CatA\times \CatB}\coloneqq \Ob_\CatA \times \Ob_\CatB$;
        \item $\HomSet{\CatA\times \CatB}{\tup{\Obja,\Objb}}{\tup{\Obja',\Objb'}}\coloneqq \HomSet{\CatA}{\Obja}{\Obja'}\mtimescat \HomSet{\CatB}{\Objb}{\Objb'}$, for two objects~$\tup{\Obja,\Objb}$ and~$\tup{\Obja',\Objb'}$ in~$\Ob_{\CatA\times \CatB}$.
    \end{compactenum}
\end{definition}


\todotextjira{247}{put contents of lecture}

\showslides{
    \begin{forslides}


        \begin{equation*}
            \label{eq:upos_lat_1}
            h_{d_1}(\fun)\cup h_{d_2}(\fun)
        \end{equation*}
        \begin{equation*}
            \label{eq:upos_lat_2}
            h_{d_1}(\fun)\cap h_{d_2}(\fun)
        \end{equation*}
        \begin{equation*}
            \label{eq:upos_lat_3}
            d_1\join d_2
        \end{equation*}
        \begin{equation*}
            \label{eq:upos_lat_4}
            d_1\meet d_2
        \end{equation*}
        \begin{equation*}
            \label{eq:upos_lat_5}
            \posbot_{\UPos(\F{A},\R{B})}\colon \F{a}\mapsto \emptyset
        \end{equation*}
        \begin{equation*}
            \label{eq:upos_lat_6}
            \top_{\UPos(\F{A},\R{B})}\colon \F{a}\mapsto \R{B}
        \end{equation*}
        \begin{definition}[Lattice homomorphism]
            \label{def:lattice_homo}
            Given two bounded lattices~$\posA, \posB$, a \emph{lattice homomorphism} is a map~$\mora\colon \posA\to \posB$ which preserves meets, joins, top, and bottom:
            \begin{equation}
                \begin{aligned}
                    \mora(\posAel \meet_\posA \posBel) &= \mora(\posAel) \meet_\posB \mora(\posBel)\\
                    \mora(\posAel \join_\posA \posBel) &= \mora(\posAel) \join_\posB \mora(\posBel)\\
                    \mora(\posbot_\posA)&=\posbot_\posB\\
                    \mora(\postop_\posA)&=\postop_\posB
                \end{aligned}
            \end{equation}
        \end{definition}
        \begin{equation*}
            \label{eq:poset_bool_a}
            \tup{\posA,\posleq}
        \end{equation*}
        \begin{equation*}
            \label{eq:poset_bool_b}
            \postop,\posbot
        \end{equation*}
        \begin{equation*}
            \label{eq:poset_bool_c}
            \posbot\mto \posbot, \quad \posbot\mto \postop, \quad \postop\mto \postop
        \end{equation*}
        \begin{equation*}
            \label{eq:poset_bool_d}
            \mtimescat_\Bool \coloneqq \booland
        \end{equation*}
        \begin{equation*}
            \label{eq:poset_bool_e}
            \posAel\posleq \posBel
        \end{equation*}
        \begin{equation*}
            \label{eq:poset_bool_f}
            \posAel,\posBel \in \posA
        \end{equation*}
        \begin{equation*}
            \label{eq:poset_bool_g}
            \HomSet{\CatC}{\Obja}{\Objb}\in \Ob_\Bool
        \end{equation*}
        \begin{equation*}
            \label{eq:poset_bool_h}
            \stylemorph{m}_{\Obja,\Objb,\Objc}\colon \HomSet{\CatC}{\Obja}{\Objb}\mtimescat \HomSet{\CatC}{\Objb}{\Objc}\mto \HomSet{\CatC}{\Obja}{\Objc}.
        \end{equation*}
        \begin{equation*}
            \label{eq:poset_bool_i}
            \idmoncat_\Bool \coloneqq \postop
        \end{equation*}
        \begin{equation*}
            \label{eq:poset_bool_k}
            \stylemorph{j}_\Obja\colon \idmoncat \mto \HomSet{\CatC}{\Obja}{\Obja}
        \end{equation*}
        \begin{equation*}
            \label{eq:poset_bool_j}
            \tup{\Set, \times, \singleton}
        \end{equation*}
        \begin{equation*}
            \label{eq:poset_bool_l}
            \HomSet{\CatC}{\Obja}{\Objb}\in \Ob_\Set
        \end{equation*}
        \begin{equation*}
            \label{eq:poset_bool_m}
            \prftree{\mora \in \HomSet{\CatC}{\Obja}{\Objb}}{\morb \in \HomSet{\CatC}{\Objb}{\Objc}}{{\mora\mthen \morb \in \HomSet{\CatC}{\Obja}{\Objc}}}
        \end{equation*}
        \begin{equation*}
            \label{eq:poset_pos_a}
            \prftree[double]{\mora \posleq_\Pos \morb}{\mora(\posAel)\posleq_\posB \morb(\posAel), \quad \forall \posAel \in \posA}
        \end{equation*}
        \begin{equation*}
            \label{eq:poset_pos_b}
            \mora,\morb\colon \posA\mto_\Pos \posB
        \end{equation*}
        \begin{definition}[Metric space]
            \label{def:metric_space}
            A \emph{metric space}~$\tup{\setA,\stylemaps{d}}$ consists of:
            \begin{compactenum}
                \item A set~$\setA$, elements of which are called \emph{points};
                \item A map~$\stylemaps{d}\colon \setA\times \setA \to \mathbb{R}_{\geq 0}$, called \emph{distance}.
            \end{compactenum}
            The constituents must satisfy:
            \begin{compactitem}
                \item $\stylemaps{d}(\setAel,\setAel)=0$, for all~$\setAel\in \setA$;
                \item If~$\stylemaps{d}(\setAel, \setBel)=0$, then~$\setAel=\setBel$, for all~$\setAel,\setBel\in \setA$;
                \item $\stylemaps{d}(\setAel, \setBel)=\stylemaps{d}(\setBel, \setAel)$, for all~$\setAel,\setBel\in \setA$;
                \item $\stylemaps{d}(\setAel, \setBel)+\stylemaps{d}(\setBel, \setCel)\geq \stylemaps{d}(\setAel, \setCel)$, for all~$\setAel,\setBel,\setCel\in \setA$.
            \end{compactitem}
        \end{definition}
        \begin{equation*}
            \label{eq:r_with_inft}
            \mathbb{R}_{\geq 0}\cup \{\infty\}
        \end{equation*}
        \begin{equation*}
            \label{eq:cost_cat}
            \Cat{Cost}=\tup{[0,\infty], \geq, 0, +}
        \end{equation*}
        \begin{equation*}
            \label{eq:enriched_fun_a}
            \funa_{\Obja, \Objb} \colon \HomSet\CatC{\Obja}{\Objb} \mto \HomSet\CatD{\funa(\Obja)}{\funa(\Objb)}
        \end{equation*}
        \begin{definition}
            \label{def:cost_enrich_a}
            Given two metric spaces~$\tup{X,d_X}$,~$\tup{Y,d_Y}$, a function~$f\colon X\to Y$ is \emph{Lipschitz} if there exists a~$K>0$ such that
            \begin{equation}
                d_Y(f(x),f(y))\leq Kd_X(x,y),\quad \forall x\in X,y\in Y.
            \end{equation}
        \end{definition}
    \end{forslides}
}

\devel{
% A locally-posetal 2-category is also known as a category enriched over posets.
%\begin{definition}\label{def:loc_pos_cat}
%A \emph{locally-posetal 2-category} is a category \CatC, with an additional pre-order structure $\leq$ on the hom-set $\CatC(p,q)$, for every two objects $p,q\in\CatC$, which additionally satisfies the condition that if $f\leq g$ in $\Hom(p,q)$ and $j\leq k$ in $\Hom(q,r)$, then $f;j\leq g;k$ in $\Hom(p,r)$.
%\end{definition}
%
    The following is obvious, but we record it anyway.
    \begin{lemma}
        \label{lem:loc_pos_op}
        If \CatC is locally posetal, then so is $\CatC\op$.
    \end{lemma}

    \begin{example}
        \label{ex:enrichbool}Every poset (as a category) is enriched in \Bool, since between any two elements $a,b$ of the poset, either the morphism $a \leq b$ exists ($\Hom_A(a,b) = \true$) or it does not ($\Hom_A(a,b) = \false$).
    \end{example}

    \begin{example}
        The poset \Bool is enriched in \Bool.
    \end{example}

    \begin{example}
        The category \Pos of posets is enriched in \Pos, where the partial order on monotone maps is given by $f \Imp g$ (\ie  for $f,g : A \to B$, $f(a) \leq g(a) \forall a \in A$).
    \end{example}

%\begin{proposition}\label{prop:Pos_loc_pos}
% Suppose that $f,g\colon\cP \to\cQ $ and $j,k\colon\cQ \to\cR $ are monotone maps with $f\Imp g$ and $j\Imp k$. Then $f;j\Imp g;k$ in \Pos.
% \[
% \begin{tikzcd}
% 	\cP \ar[r, bend left=35pt, "f"]\ar[r, bend right=35pt, "g"']\ar[r,phantom, "\Downarrow"]
% 	&\cQ \ar[r, bend left=35pt, "j"]\ar[r, bend right=35pt, "k"']\ar[r,phantom, "\Downarrow"]
% 	&\cR
% \end{tikzcd}
% \]
% \end{proposition}
% \begin{proof}
% Assume $f\Imp g$ and $j\Imp k$, and choose $p\in\cP $; we want to show that $j(f(p))\leq_R k(g(p))$. We have $f(p)\leqQ  g(p)$ by assumption and since $j$ is monotone, we have $j(f(p))\leq_R j(g(p))$. But since $j\Imp k$, we also have $j(g(p))\leq_R k(g(p))$, and the result follows by transitivity.
% \end{proof}
%
%With the above, we have shown that the category \Pos actually has the structure of \emph{locally-posetal 2-category}.

    \begin{example}
        \Category is enriched in \Category.
    \end{example}

    \begin{example}
        Every category is enriched in \Set.
        \linkvideo{spring2021-enrichment:set-enriched} % Enrichment in Set
    \end{example}


    \begin{example}
        \linkvideo{spring2021-enrichment:bool-enriched} % A category enriched in Bool is a preorder
        A poset~$\tup{P, \leq}$ can be consider a category enriched in the
        category~\Bool. First, recall the construction that makes each poset into a
        category~(\cref{sec:posetsarecats}). The poset~$P$ as a category is a category with the objects being the
        elements of~$P$, and with a morphism~$f\colon x \to y$ existing if and only if~$x\leq y$.

        \Bool as a category contains two elements,~$\true$ and~$\false$, with
        the three morphism~$\false \to \true$, $\true \to \true$, and~$\false \to \false$. This is equivalent to say that there is a morphism between~$a, b \in \Bool$ if and only if~$a \Rightarrow b$. So we can set~$\Rightarrow\ \equiv\ \to_{\Bool}$.

        \Bool can be also considered a monoidal category, by letting~$\otimes$ be
        the \emph{and} operation, so that
        \begin{equation}
            a \otimes b\ =\ a \booland b.
        \end{equation}
        Looking at~$P$ again, we can show how it can be considered a category enriched in~\Bool. For any two points~$a, b$ of~$P$, either~$a \leq b$, or not: There are two choices. The hom-set~$\Hom(a; b)$ is either non-empty
        (if~$a \leq b$) or empty (if~$a \not\leq b$). We can make the correspondence that an empty hom-set corresponds to~$\false$ and a non-empty hom-set corresponds to~$\true$.

        Now we can verify that the condition~\cref{eq:enriched-condition} holds. We
        know that, in a poset,~$x \leq y$ and~$y \leq z$ implies~$x \leq z$.
%
        Rewritten in the language of categories, this is:
        \begin{equation*}
        (\Hom(x;y)\ \text{non-empty})
            \booland
            (\Hom(y; z)\ \text{non-empty})
            \quad
            \Rightarrow
            \quad
            \Hom(x; z)\ \text{non-empty}.
        \end{equation*}
        By making the identification~$\Rightarrow\ \equiv\ \to_{\Bool}$ and~$\booland\ \equiv\ \mtimes_{\Bool}$, we can rewrite the above as
        \begin{widepar}
            \begin{equation*}
            (\Hom(x;y)\ \text{non-empty})
                \mtimes_{\Bool}
                (\Hom(y; z)\ \text{non-empty})
                \quad
                \to_{\Bool}
                \quad
                \Hom(x; z)\ \text{non-empty},
            \end{equation*}
        \end{widepar}
        This is the specialization of~\cref{eq:enriched-condition}
        for~\CatC a poset and~\CatD equal to~\Bool.
    \end{example}
    \todotext{@Gioele: figure out right link for eq:enriched-condition}
    \begin{example}
        $\Pos$ is enriched in $\Pos$.
        \linkvideo{spring2021-enrichment:pos-en-pos} % Pos is enriched in Pos
    \end{example}

    \begin{example}[Enrichment in cost]
        \linkvideo{spring2021-enrichment:cost-en} % Enrichment in cost
    \end{example}

    \begin{example}
        A category ``enriched in \Set'' is just a regular category, as we have defined it.

        Recall that~\Set is a monoidal category with $\mtimes_{\Set}$ equal to the Cartesian product $\cartprod$.

        Take an arbitrary category~\CatC. For all~$x, y$ in~\CatC, we know by definition
        that~$\HomSet{\CatC}{x}{y}$ is a set.

        Consider three objects~$x,y,z$ in~\CatC. We know from the definition of a
        category that if there exists a morphism~$f: x \to y$ and another~$g: y \to z$,
        then there also exists~$(f\then g)\colon x \to z$.

        Consider now the set of all morphisms between $x, y$, given by~$\Hom(x;y)$, and
        all morphisms between~$y$ and~$z$, given by~$\Hom(y; z)$. What is the relation
        between those hom-sets and~$\Hom(x; z)$?

        It is not quite true that~$\Hom(x; z)$ is the Cartesian product~$\Hom(x, y)
        \cartprod \Hom(y; z)$. If there are~$m$ morphisms between~$x$ and~$y$, and~$n$
        morphisms between~$y$ and~$z$, there are not necessarily~$m \cdot n$ morphisms
        from~$x$ to~$z$, because we are not guaranteed that all the compositions~$(f\then g)$
        will be distinct morphisms.

        What we are guaranteed is that all of the compositions will be mapped to something in~$\Hom(x; z)$; or, in other words, we are guaranteed that there
        is a map $\phi$ of the type
        \begin{equation*}
            \phi \colon \Hom(x; y) \times \Hom(y; z) \rightarrow \Hom(x; z).
        \end{equation*}
        This $\phi$ is a morphism in \Set, and it is the witness required by~\cref{eq:enriched-condition}.

    \end{example}
}

\linkvideo{spring2021-enrichment:bounded-lat-en} % Enrichment in BoundedLat
\begin{gradedexercise}[\exname{DPIsEnrichedInPos}]
    \label{ex:DPIsEnrichedInPos}
    Show that the category of design problems (objects are posets and morphisms are $\Bool$-profunctors) can be viewed as a category enriched in the monoidal category of posets.

%We recall that a lattice is a poset $\tup{\posA,\ordleq}$ such that for any two elements $\ela, \elb \in P$, there exists a greatest lower bound, denoted $\ela \meet \elb$, and a least upper bound, denoted $\ela \join \ela$. Equivalently, we can think of the poset as equipped with two additional binary operations -- called meet and join -- which compute the greatest lower bound and least upper bound of any two given elements. From this point of view, a lattice is a quadruple $\tup{\posA,\ordleq,\join,\meet}$.

%Given lattices $$\styleobj{L_1} = \tup{\posA_1,\ordleq_1,\join_1,\meet_1}$$ and $$\styleobj{L_2} = \tup{\posA_2,\ordleq_2,\join_2,\meet_2}$$
%a morphism $\mora : \styleobj{L_1} \mto \styleobj{L_2}$ is a monotone map which, additionally, satisfies the equations
%$$\mora (\ela \meet \elb) = \mora (\ela) \meet \mora ( \elb) \quad \text{and} \quad \mora (\ela \join \elb) = \mora (\ela) \join \mora ( \elb).$$

    The monoidal product $\styleobj{P_1} \times \styleobj{P_2}$ of posets is defined straightforwardly as the product poset.
    The monoidal unit in this category is any choice of 1-element poset.
\end{gradedexercise}

\solutionof{DPIsEnrichedInPos}

\begin{gradedexercise}[\exname{ProductOfEnrichedCats}]
    \label{ex:ProductOfEnrichedCats}
    Let $\CatV$ be a symmetric monoidal category, and let $\CatC$ and $\CatD$ be $\CatV$-enriched categories. Write down a definition of the cartesian product $\CatC \times \CatD$ of $\CatC$ and $\CatD$ such that $\CatC \times \CatD$ is again a $\CatV$-enriched category, and provide a proof that this is indeed true for your definition!

    Is it needed that $\CatV$ be \emph{symmetric} monoidal? If yes, were is symmetry needed?
\end{gradedexercise}
\solutionof{ProductOfEnrichedCats}

