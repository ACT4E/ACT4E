\showslides{

\section{Lecture slides materials}

\begin{forslides}
    \subsection{Section: Profunctors}

    \begin{ctdefinition}[\DP]
        \label{def:dp-again}
        Given two \SY{posets}~$\funsp$ and~$\ressp$, a \emph{design problem} with functionality $\funsp$ and resources~$\ressp$ is a \SY{monotone function} of the form
        \begin{equation}
            \label{eq:dp-again-1}
            \adp \colon \funspop \Ptimes \ressp \ftoin{\Pos} \Bool.
        \end{equation}
        This is also written as
        \begin{equation}
            \label{eq:dp-again-2}
            \adp\colon \funsp \profto \ressp.
        \end{equation}
    \end{ctdefinition}

    \begin{equation}
        \label{eq:dp-again-3}
        \tupp{\fun\op, \res} \quad \mapsto \quad \exists \imp \setin \impsp :  (\fun \funleq \prov(\imp)) \booland (\req(\imp) \resleq \res)
    \end{equation}
    %
    \begin{equation}
        \label{eq:dp-again-set}
        \tupp{\fun\op, \res} \quad \mapsto \quad \makeset{
            \imp \setin \impsp :  (\fun \funleq \prov(\imp)) \booland (\req(\imp) \resleq \res)
        }
    \end{equation}
    %
    \begin{equation}
        \label{eq:dp-again-prof}
        \adp \colon \funspop \Ptimes \ressp \ftoin{\Pos} \powerset \impsp
    \end{equation}
    %
    \begin{equation}
        \label{eq:profunctors-rela}
        \relA \colon \setA \cartprod  \setB \sto \boolset.
    \end{equation}

    \begin{definition}[\CatV-profunctor]
        \label{def:enriched-profunctor}
        Let \CatA and \CatB be categories enriched in~\CatV.
        A \SY{profunctor} enriched in~\CatV from \CatA to \CatB, denoted~$\funa\colon \CatA \profto \CatB$, is a \SY{functor} enriched in~\CatV:
        \begin{equation}
            \funa\colon \CatA\catop \Ctimes \CatB \fto \CatV.
        \end{equation}
    \end{definition}

    \subsection{Section: Hom}

    \begin{equation}
        \label{eq:hom1}
        \mora = \tup{\mora_1\op,\mora_2}
    \end{equation}
    %
    \begin{equation}
        \label{eq:hom1b}
        \morb = \tup{\morb_1\op,\morb_2}
    \end{equation}
    %
    \begin{equation}
        \label{eq:hom2b}
        \CatCop \Ctimes\CatC
    \end{equation}
    %
    \begin{equation}
        \label{eq:CatC}
        \CatC
    \end{equation}
    %
    \begin{equation}
        \label{eq:hom-X}
        \Obja
    \end{equation}
    %
    \begin{equation}
        \label{eq:hom-X-op}
        \Obja^\ast
    \end{equation}
    %
    \begin{equation}
        \label{eq:hom-Y}
        \Objb
    \end{equation}
    %
    \begin{equation}
        \label{eq:hom-Y-op}
        \Objb^\ast
    \end{equation}
    %
    \begin{equation}
        \label{eq:hom-Z}
        \Objc
    \end{equation}
    %
    \begin{equation}
        \label{eq:hom-Z-op}
        \Objc^\ast
    \end{equation}
    %
    \begin{equation}
        \label{eq:hom-W}
        \Objd
    \end{equation}
    %
    \begin{equation}
        \label{eq:hom-W-op}
        \Objd^\ast
    \end{equation}
    %
    \begin{equation}
        \label{eq:f1}
        \mora_1
    \end{equation}
    %
    \begin{equation}
        \label{eq:f2}
        \mora_2
    \end{equation}
    %
    \begin{equation}
        \label{eq:g1}
        \morb_1
    \end{equation}
    %
    \begin{equation}
        \label{eq:g2}
        \morb_2
    \end{equation}
    %
    \begin{equation}
        \label{eq:g1f1}
        \morb_1\mthen \mora_1
    \end{equation}
    %
    \begin{equation}
        \label{eq:f2g2}
        \mora_2\mthen \morb_2
    \end{equation}
    
\end{forslides}

}