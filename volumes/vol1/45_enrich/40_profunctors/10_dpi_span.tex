% !TEX root = chapter-standalone.tex


\section{DPIs are spans}
\label{sec:spans}
\todostructurejira{277}{@J: Not sure this is the place to talk about spans.
Maybe define spans before DP, then reference from DPs?}

\begin{ctdefinition}[Span]
    \label{def:span}
    Given a category~\CatC, a \emph{\iindex{span}} from an object~$\Obja$ to an object~$\Objb$ is a diagram of the form
    \begin{center}
        \includesag{51_span}
    \end{center}
    where~$\Objc$ is some other object of~\CatC.
\end{ctdefinition}


\begin{example}
    \todotextjira{278}{Bad example.
    Just arbitrary points of a category, nothing special.
    Instead, do the case of \Berg where you can only go down.
    Choose a famous peak and use as the other two points two cities reachable on either side of the peak.}
    
    Consider the category~\Berg, introduced in \cref{sec:trekking}.
    An example of span in this category is reported in~\cref{fig:exmountains}.
    \begin{figure}[h!]
        \centering
        \includesag{130_mountains}
        \caption{Swiss peaks can be thought of as a span in~\Berg.}
        \label{fig:exmountains}
    \end{figure}
    Recall that \textsf{Matterhorn Peak}, \textsf{Jungfrau Peak}, and \textsf{Pilatus Peak} are objects of~\Berg, and the arrows are morphisms in~\Berg (paths from one location to the other).
\end{example}

We can interpret DPIs as spans.
Precisely, given the two maps~$\prov \colon \impsp \to \funsp$ (an implementation \textbf{prov}ides a functionality) and~$\req\colon \impsp \to \ressp$ (an implementation \textbf{req}uires resources) one has the span in~$\Set$:
\begin{center}
    \includesag{130_catalogue}
\end{center}


%\begin{definition}[Design problem induced by a catalogue]
%  Every catalogue~$\tup{\impsp,\prov,\req}$ \emph{induces} a design problem of the form~$d\colon \funsp \profto \ressp$, with
%  \begin{equation*}
%    \begin{aligned}
%      d\colon \funsp \op \times \ressp &\toinPos \Bool\\
%      \tup{\fun^*,\res}&\mapsto \bigvee_{\imp\in \impsp}\left(\prov(\imp)\funleq \fun \right)\wedge \left( \req(\imp)\resleq \res \right)
%    \end{aligned}
%  \end{equation*}
%\end{definition}
