% !TEX root = chapter-standalone.tex

\section{$\Hom$ Profunctor}
\linkvideo{spring2021-profunctors:hom-prof} % The Hom profunctor

In this section, we will see that given any category \CatC, its set of morphisms~$\Hom$ can be seen as a \SY{profunctor} of the form:
\begin{equation}
    \label{eq:hom-signature}
    \Hom_{\CatC}\colon \CatC \op \Ctimes \CatC \toinCat \Set
\end{equation}
First, we need to specify how this \SY{profunctor} acts on objects and on morphisms.
It maps any two objects~$\Obja,\Objb\setin \CatC$ to their~$\Hom$-sets:
\begin{equation}
    \label{eq:hom-action}
    \Hom_{\CatC}(\tup{\Obja^*,\Objb})=\HomSet{\CatC}{\Obja}{\Objb}
\end{equation}
The action on the morphisms is more involved.
First, a morphism in~$\CatCop \Ctimes \CatC$ has signature
%
\begin{equation}
    \label{eq:hom0}
    \mora \colon \tup{\Obja^*,\Objb}\mto \tup{\Objc^*, \Objd}.
\end{equation}
%
In particular,~$\mora$ can be written as~$\tup{\mora_1\op,\mora_2}$ with~$\mora_1\op\colon \Obja^*\mto \Objc^*$ and~$\mora_2\colon \Objb\mto \Objd$.
Now, the \SY{profunctor} maps each morphism~$\mora \colon \tup{\Obja^*,\Objb}\mto \tup{\Objc^*, \Objd}$ to a morphism
%
\begin{equation}
    \label{eq:hom2}
    \Hom_{\CatC}(\mora)\colon \HomSet{\CatC}{\Obja}{\Objb}\mto \HomSet{\CatC}{\Objc}{\Objd}.
\end{equation}
%
More specifically, we have:
%
\begin{equation}
    \label{eq:hom3}
    \prftree{
        \mora_1\colon \Objc \mto \Obja
    }{
        \mora_2\colon \Objb\mto \Objd
    }{
        \morb\colon \Obja\mto \Objb
    }{
        (\mora_1\mthen \morb \mthen \mora_2)\colon \Objc\mto \Objd
    }.
\end{equation}
%
Graphically, we have
\equationsag{hom_prof_new}{eq:hom_prof_new}
%
Therefore, the \SY{functor} maps each morphism~$\mora=\tup{\mora_1\op,\mora_2}\setin \HomSet{\CatC}{\Obja}{\Objb}$ to the morphism in \Set given by:
%
\begin{equation}
    \label{eq:hom4}
    \begin{aligned}
        \Hom_{\CatC}(\mora)\colon \HomSet{\CatC}{\Obja}{\Objb} & \mto \HomSet{\CatC}{\Objc}{\Objd} \\
        \morb                                                  & \mapsto (\mora_1\mthen \morb \mthen \mora_2).
    \end{aligned}
\end{equation}
%
Graphically, starting from morphisms in \CatC
%
\equationsag{hom_prof_0}{eq:hom_prof_0}
%
the \SY{profunctor} acts as a \emph{template}
%
\equationsag{hom_prof_0_0}{eq:hom_prof_0_0}

\linkvideo{spring2021-profunctors:hom-prof-check} % Checking that Hom is a profunctor

We now need to check that this indeed forms a \SY{profunctor}.
In other words, we want to check that this is a \SY{functor} of the specific kind described.
First, we want to check that
%
\begin{equation}
    \label{eq:hom5}
    \Hom_{\CatC}(\morab)=\Hom_{\CatC}(\mora)\mthen \Hom_{\CatC}(\morb).
\end{equation}
Consider any~$\mora=\tup{\mora_1\op, \mora_2}$ and~$\morb=\tup{\morb_1\op,\morb_2}$ in~$\CatCop \Ctimes \CatC$.

We know that
\begin{equation}
    \label{eq:hom6}
    \begin{aligned}
        \morab & =\tup{\mora_1\op \mthen \morb_1\op, \mora_2\mthen \morb_2} \\
               & =\tup{(\morb_1\mthen \mora_1)\op, \mora_2\mthen \morb_2}.
    \end{aligned}
\end{equation}
Therefore, we can write
\begin{equation}
    \label{eq:hom7}
    \begin{aligned}
        \Hom_{\CatC}(\morab)(z) & =(\morb_1\mthen \mora_1)\mthen z \mthen (\mora_2\mthen \morb_2) \\
                                & =\morb_1\mthen (\mora_1\mthen z \mthen \mora_2) \mthen \morb_2 \\
                                & =(\mora_1\mthen z \mthen \mora_2)\mthen \Hom_{\CatC}(\morb) \\
                                & =(\Hom_{\CatC}(\mora)\mthen \Hom_{\CatC}(\morb))(z),
    \end{aligned}
\end{equation}
proving the composition property.
Graphically, starting from \cref{fig:hom_prof_1} and \cref{fig:hom_prof_2}, we can write \cref{fig:hom_prof_3} and \cref{fig:hom_prof_4}.

\begin{marginfigure}
    \centering
    \includesag{hom_prof_1}
    \caption{\label{fig:hom_prof_1}}
\end{marginfigure}
\begin{marginfigure}
    \centering
    \includesag{hom_prof_2}
    \caption{\label{fig:hom_prof_2}}
\end{marginfigure}
\begin{marginfigure}
    \centering
    \includesag{hom_prof_3}
    \caption{\label{fig:hom_prof_3}}
\end{marginfigure}
\begin{figure}[h!]
    \centering
    \includesag{hom_prof_4}
    \caption{\label{fig:hom_prof_4}}
\end{figure}

We now want to check that
\begin{equation}
    \label{eq:hom_identity}
    \Hom_{\CatC}(\catidat{\tup{\Obja^*,\Objb}})=\catidat{\Hom_{\CatC}(\tup{\Obja^*,\Objb})}.
\end{equation}
%
We have
%
\begin{equation}
    \begin{aligned}
        \Hom_{\CatC}(\catidat{\tup{\Obja^*,\Objb}})\colon \HomSet{\CatC}{\Obja}{\Objb} & \mto \HomSet{\CatC}{\Obja}{\Objb} \\
        \morb                                                                          & \mapsto (\catidat\Obja \mthen \morb \mthen \catidat\Objb)=\morb.
    \end{aligned}
\end{equation}
%
Analogously:
%
\begin{equation}
    \begin{aligned}
        \catidat{\Hom_{\CatC}(\tup{\Obja^*,\Objb})}=\catidat{\HomSet{\CatC}{\Obja^*}{\Objb}} \colon \HomSet{\CatC}{\Obja}{\Objb} & \mto \HomSet{\CatC}{\Obja}{\Objb} \\
        \morb                                                                                                                    & \mapsto \morb.
    \end{aligned}
\end{equation}

\devel{\includepdf[scale=0.8,pages={22-26},nup=1x3,frame,pagecommand={}]{ACT4E-09-design.pdf}}
