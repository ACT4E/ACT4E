% !TEX root = chapter-standalone.tex

\section{Arrow construction}

\todotextjira{259}{@J: Intro to write}

\begin{ctdefinition}[Arrow category]
    \label{def:arrow_category}
    Given any category~$\CatC$, its \emph{\iindex{arrow category}}~$\Arrow\CatC$ is the category in which:
    \begin{enumerate}
        \item \emph{Objects:} An object~$\Obja \in \Arrow\CatC$ is a morphism~$\Obja\colon \Obja_0\mto \Obja_1$ of~$\CatC$;
        \item \emph{Morphisms:} A morphism~$\mora\colon \Obja \mto \Objb$ in~$\Arrow\CatC$ is a commutative square
              \begin{center}
                  \includesag{arrow_cat}
              \end{center}
              in~$\CatC$;
        \item \emph{Composition:} Composition in~$\Arrow \CatC$ is given by playing commutative squares side by side.
              Consider~$\mora\colon \Obja \mto \Objb$ and~$\morb\colon \Objb \mto \Objc$ in~$\Arrow\CatC$.
              Then:
              \begin{center}
                  \includesag{180_arrow_comp}
              \end{center}
    \end{enumerate}
\end{ctdefinition}
\todotextjira{539}{@J: Give the formula for composition explicitly?
}

\todotextjira{406}{@J: Explain why composition is well defined.
    We didn't talk that much
    about the use of commutative diagrams so this is a good example.
}
\todotextjira{406}{Add remark about functor category isomorphism}
