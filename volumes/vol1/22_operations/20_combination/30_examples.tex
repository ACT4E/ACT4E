\section{Other examples}

\subsection{\statusdraft{Product and coproduct for power set}}
\cref{ex:subset_prod} and \cref{ex:subset_coprod} are specific instances of the power set lattice.

\todotext{meet/join are inverted}
\begin{definition}[Power set as lattice]
  \label{def:power-set-as-lattice}
Given a set~$S$, its power set~$\powerset S$ (the set of all subsets) is a lattice where, given~$A,B\in \powerset S$:
\begin{compactitem}
  \item Order is given by inclusion:
  \begin{equation*}
    A\posleq B \definedas A\subseteq B;
  \end{equation*}
  \item The meet is given by the union of sets:
  \begin{equation*}
    A\meet B \definedas A\cup B;
  \end{equation*}
  \item The join is given by the intersection of sets:
  \begin{equation*}
    A\join b \definedas A\cap B;
  \end{equation*}
  \item The top element is the set~$S$ itself:
  \begin{equation*}
    \postop = S;
  \end{equation*}
  \item The bottom element is the empty set:
  \begin{equation*}
    \posbot = \emptyset.
  \end{equation*}
\end{compactitem}
\end{definition}

The Hasse diagram reported in \cref{fig:prod_coprod_power} illustrates the structure of the power set lattice for three sets~$A,B,C\in \powerset S$.

\begin{figure}[h]
  \begin{center}
    \includesag{060_powerset_coprod}
  \end{center}
  \caption{}
  \label{fig:prod_coprod_power}
\end{figure}
As previously discovered, the lattice can be seen as a category.
In this category, the meet~$\meet$ is the product, and the join~$\join$ is the coproduct.
Specifically, for~$A,B\subseteq S$ the product corresponds to~$A\cap B$, and the projection maps~$\pi_A\colon A\cap B\to A$ and~$\pi_B\colon A\cap B\to B$ simply state the inclusions of~$A\cap B$ in~$A$ and~$B$.
Similarly, the coproduct corresponds to~$A\cup B$, and the injection maps~$\injA\colon A\to A\cup B$ and~$\injB\colon B\to A\cup B$ simply state the inclusion of~$A,B$ in~$A\cup B$.

\subsection{Product and coproduct for logical sequents}

\todostructure{Add here content of Andrea's slides}
