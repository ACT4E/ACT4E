% !TEX root = chapter-standalone.tex

\section{Other examples}

\subsection{Product and coproduct for power set}
\Cref{exa:intersection-as-prod} and \cref{ex:subset_coprod} are specific instances of the power set lattice.

\begin{ctdefinition}[Power set as lattice]
    \label{def:power-set-as-lattice}
    Given a set~$\stylesets{S}$, its power set~$\powerset \stylesets{S}$ (the set of all subsets) is a lattice where, given~$\setA,\setB\setin \powerset \stylesets{S}$:
    \begin{itemize}
        \item Order is given by inclusion:
              \begin{equation*}
                  \setA\posleq \setB \definedas \setA\subseteq \setB;
              \end{equation*}
        \item The join is given by the union of sets:
              \begin{equation*}
                  \setA\join \setB \definedas \setA\setunion \setB;
              \end{equation*}
        \item The meet is given by the intersection of sets:
              \begin{equation*}
                  \setA\meet \setB \definedas \setA\setintersection \setB;
              \end{equation*}
        \item The top element is the set~$\stylesets{S}$ itself:
              \begin{equation*}
                  \postop = \stylesets{S};
              \end{equation*}
        \item The bottom element is the empty set:
              \begin{equation*}
                  \posbot = \emptyset.
              \end{equation*}
    \end{itemize}
\end{ctdefinition}

\begin{marginfigure}
    \centering
    \includesag{060_powerset_coprod}
    \caption{}
    \label{fig:prod_coprod_power}
\end{marginfigure}

The Hasse diagram reported in \cref{fig:prod_coprod_power} illustrates the structure of the power set lattice for three sets~$\setA,\setB,\setC\setin \powerset \stylesets{S}$.

As previously discovered, the lattice can be seen as a category.
In this category, the meet~$\meet$ is the product, and the join~$\join$ is the coproduct.
Specifically, for~$\setA,\setB\subseteq\stylesets{S}$ the product corresponds to~$\setA\setintersection \setB$, and the projection maps~$\projA\colon \setA\setintersection \setB\to \setA$ and~$\projB\colon \setA\setintersection \setB\to \setB$ simply state the inclusions of~$\setA\setintersection \setB$ in~$\setA$ and~$\setB$.
Similarly, the coproduct corresponds to~$\setA\setunion \setB$, and the injection maps~$\injA\colon \setA\to \setA\setunion \setB$ and~$\injB\colon \setB\to \setA\setunion \setB$ simply state the inclusion of~$\setA,\setB$ in~$\setA\setunion \setB$.

\subsection{Product and coproduct for logical sequents}

\begin{ctdefinition}[Propositions as lattice]
    \label{def:prop-as-lattice}
    Given a set of propositions~$\posAset$ (equivalence classes of propositions), we define the following.
    \begin{itemize}
        \item Order is given by implication:
              \begin{equation*}
                  \prfdoubleperiod{\posAel\posAleq \posBel}{\posAel \implies \posBel}
              \end{equation*}
              Clearly, since~$\posAel \posAleq \posAel$,
              \begin{equation*}
                  \prfdoublecomma{\posAel \Leftrightarrow \posBel}{\posAel =\posBel}
              \end{equation*}
              and
              \begin{equation*}
                  \prfdoublecomma{\posAel \implies \posBel}{\posBel \implies \posCel}{\posAel \implies \posCel}
              \end{equation*}
              this defines a poset over~$\posAset$.
        \item The join is given by the ``or'' connective:
              \begin{equation*}
                  \posAel \join \posBel \definedas \posAel \boolor \posBel;
              \end{equation*}
        \item The join is given by the ``and'' connective:
              \begin{equation*}
                  \posAel \meet \posBel \definedas \posAel \booland \posBel;
              \end{equation*}
        \item The top element is~$\true$:
              \begin{equation*}
                  \postop = \true,
              \end{equation*}
              because
              \begin{equation*}
                  \prfcomma{\true}{\posAel \implies \posAel}
              \end{equation*}
              for all~$\posAel\setin \posA$.
        \item The bottom element is~$\false$:
              \begin{equation*}
                  \posbot = \false.
              \end{equation*}
    \end{itemize}
\end{ctdefinition}
\begin{marginfigure}
    \centering
    \includesag{060_propositions_coprod}
    \caption{}
    \label{fig:prod_coprod_prop}
\end{marginfigure}

The diagram reported in \cref{fig:prod_coprod_prop} illustrates the structure of the propositions lattice for three elements~$\posAel,\posBel,\posCel\setin \posA$.

As previously discovered, the lattice can be seen as a category.
In this category, the meet~$\meet$ is the product, and the join~$\join$ is the coproduct.

Specifically, for~$\posAel,\posBel\setin \posA$ the product corresponds to~$\posAel\booland \posBel$, and the projection maps~$\projA\colon \posAel\booland \posBel\to \posAel$ and~$\projB\colon \posAel\booland \posBel\to \posBel$ simply state the implications~$\posAel \booland \posBel \implies \posAel$ and~$\posAel \booland \posBel \implies \posBel$
Similarly, the coproduct corresponds to~$\posAel\boolor \posBel$, and the injection maps~$\injA\colon \posAel\to \posAel\boolor \posBel$ and~$\injB\colon \posBel\to \posAel\boolor \posBel$ simply state the implications~$\posAel \implies \posAel \boolor \posBel$ and~$\posBel \implies \posAel \boolor \posBel$.
