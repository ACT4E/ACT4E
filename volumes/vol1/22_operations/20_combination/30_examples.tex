% !TEX root = chapter-standalone.tex


\section{Other examples}

\subsection{Product and coproduct for power set}
\Cref{exa:intersection-as-prod} and \cref{ex:subset_coprod} are specific instances of the power set lattice.

\begin{definition}[Power set as lattice]
    \label{def:power-set-as-lattice}
    Given a set~$\stylesets{S}$, its power set~$\powerset \stylesets{S}$ (the set of all subsets) is a lattice where, given~$\setA,\setB\in \powerset \stylesets{S}$:
    \begin{compactitem}
        \item Order is given by inclusion:
        \begin{equation*}
            \setA\posleq \setB \definedas \setA\subseteq \setB;
        \end{equation*}
        \item The join is given by the union of sets:
        \begin{equation*}
            \setA\join \setB \definedas \setA\cup \setB;
        \end{equation*}
        \item The meet is given by the intersection of sets:
        \begin{equation*}
            \setA\meet \setB \definedas \setA\cap \setB;
        \end{equation*}
        \item The top element is the set~$\stylesets{S}$ itself:
        \begin{equation*}
            \postop = \stylesets{S};
        \end{equation*}
        \item The bottom element is the empty set:
        \begin{equation*}
            \posbot = \emptyset.
        \end{equation*}
    \end{compactitem}
\end{definition}

The Hasse diagram reported in \cref{fig:prod_coprod_power} illustrates the structure of the power set lattice for three sets~$\setA,\setB,\setC\in \powerset \stylesets{S}$.

\begin{figure}[h]
    \centering
    \includesag{060_powerset_coprod}
    \caption{}
    \label{fig:prod_coprod_power}
\end{figure}

As previously discovered, the lattice can be seen as a category.
In this category, the meet~$\meet$ is the product, and the join~$\join$ is the coproduct.
Specifically, for~$\setA,\setB\subseteq \stylesets{S}$ the product corresponds to~$\setA\cap \setB$, and the projection maps~$\pi_\setA\colon \setA\cap \setB\to \setA$ and~$\pi_\setB\colon \setA\cap \setB\to \setB$ simply state the inclusions of~$\setA\cap \setB$ in~$\setA$ and~$\setB$.
Similarly, the coproduct corresponds to~$\setA\cup \setB$, and the injection maps~$\injA\colon \setA\to \setA\cup \setB$ and~$\injB\colon \setB\to \setA\cup \setB$ simply state the inclusion of~$\setA,\setB$ in~$\setA\cup \setB$.

\subsection{Product and coproduct for logical sequents}
\publictodomessage

\todojira{92}{Add here content of Andrea's slides}
