% !TEX root = chapter-standalone.tex

\section{Coproducts}
\label{sec:coproductset}

%\linkvideo{spring2021-coproducts:coproducts} % Coproducts

There exists a ``dual'' notion to ``product'' that is called ``coproduct''.
Just like the notion of categorical product generalized the definition of the cartesian product of two sets, the categorical coproduct generalizes the definition of the \emph{disjoint union} of two sets.
%We'll start by illustrating the idea behind coproducts using an example. Suppose that we are considering two vending machines: from the first one can buy food and from the second one a drink. We can think of this situation in terms of resources, by saying that having \transmuted{money} is enough to get \transmuted{food} via the first machine, or to get a \transmuted{drink} via the second machine (\cref{fig:vending_1}).
%
%\begin{figure}[h!]
%    \centering
%    \includesag{60_vending_1}
%    \caption{Alternative ways to use \transmuted{money}. \label{fig:vending_1}}
%\end{figure}
%
%From this we would like to conclude that we can use \transmuted{money} to buy \textbf{either} \transmuted{food} \textbf{or} a \transmuted{drink} (\cref{fig:vending_2}).
%
%\begin{figure}[h!]
%    \centering
%    \includesag{60_vending_2}
%    \caption{We can use \transmuted{money} to buy either \transmuted{food} or a \transmuted{drink}.\label{fig:vending_2}}
%\end{figure}
\showslides{
    \begin{forslides}

        \begin{definition}[Union of sets]
            \label{def:union_sets}
            The \emph{union} of sets~$\setA$ and~$\setB$ is
            \begin{equation}
                \setA \cup \setB=\{\ela \colon \ela \in \setA \text{ or }\ela \in \setB\}.
            \end{equation}
        \end{definition}
        \begin{equation*}
            \label{eq:divides_1}
            \setA,\setB\in \natnumbers
        \end{equation*}
        \begin{equation*}
            \label{eq:divides_2}
            \setA
        \end{equation*}
        \begin{equation*}
            \label{eq:divides_3}
            \setB
        \end{equation*}
        \begin{equation*}
            \label{eq:divides_4}
            6\to 12
        \end{equation*}
        \begin{equation*}
            \label{eq:coprod_1}
            \coprodMap{f}{g}\colon A+B\to P
        \end{equation*}
        \begin{equation}
            \label{eq:coprod_2}
            \begin{aligned}
                \coprodMap{f}{g} \colon  \setA + \setB & \to \stylesets{P} \\
                \stylesets{x}                          & \mapsto
                \begin{cases}
                    \mapa(\stylesets{x}), & \text{if } \stylesets{x}=\iota_\setA(\setAel),\quad \setAel \in \setA, \\
                    \mapb(\stylesets{x}), & \text{if } \stylesets{x}=\iota_\setB(\setBel),\quad \setBel \in \setB.
                \end{cases}
            \end{aligned}
        \end{equation}
        \begin{equation*}
            \label{eq:coprod_3}
            \begin{aligned}
                \iota_\setA \colon \setA & \to \setA +\setB \\
                \iota_\setB \colon \setB & \to \setA +\setB
            \end{aligned}
        \end{equation*}
    \end{forslides}
}
\begin{comment}
Suppose that we are considering a hybrid car that contains two engines: an electric engine and an internal combustion engine.
Both can produce \transmuted{motion}, but each from a different source of energy.
The electric engine uses \transmuted{electric energy}; the internal combustion engine uses \transmuted{gasoline}.
The situation is as in \cref{fig:e14}.

\begin{figure}[h!]
    \centering
    \includesag{30_dpcatfig_e14}
    \caption{Alternative ways to generate $\mathsf{motion}$.}
    \label{fig:e14}
\end{figure}

From this we would like to conclude that we can obtain \textsf{motion} from \textbf{either} \textsf{gasoline} \textbf{or} \textsf{electric energy} (\cref{fig:e16b}).

\begin{figure}[h!]
    \centering
    \includesag{30_dpcatfig_e15}
    \caption{We can generate $\mathsf{motion}$ from either $\mathsf{gasoline}$ or $\mathsf{electric} \ \mathsf{energy}$.}
    \label{fig:e15}
\end{figure}

To define the idea of ``\textbf{either} \transmuted{food} \textbf{or} \transmuted{drink}'' we can refer to the idea of disjoint union of sets (\cref{def:disjoint-union}).
\end{comment}

Given sets~$\setA$ and~$\setB$, their disjoint union~$\setA \setdisunion \setB$ is a set that contains a distinct copy of~$\setA$ and~$\setB$ each.
If an element is contained in both~$\setA$ and~$\setB$, then there will be two distinct copies of it in the disjoint union~$\setA \setdisunion \setB$.

\begin{definition}[Disjoint union of sets]
    \label{def:disjoint-union}
    The \emph{disjoint union} (or sum) of sets~$\setA$ and~$\setB$ is
    \begin{equation}
        \setA \setdisunion \setB=\{\disunionA\ela \mid \ela\in \setA\}\cup\{\disunionB\elb \mid \elb\in \setB\}.
    \end{equation}
\end{definition}

\linkvideo{spring2021-coproducts:coproducts:coprod-intro-ex} % Introductory examples of coproduct
\begin{example}
    Consider the sets~$\setA=\{\star, \diamond\}$ and~$\setB=\{\ast, \dagger\}$.
    Their disjoint union can be represented as in \cref{fig:disjoint}.
\end{example}
\begin{figure}[h!]
    \centering
    \includesag{30_disjoint_union}
    \caption{Example of a disjoint union of sets.}
    \label{fig:disjoint}
\end{figure}

We can define the disjoint union of a set with itself; this corresponds to having two distinct copies of the set~(\cref{fig:disjointself}).

\begin{figure}[h!]
    \centering
    \includesag{30_disjoint_union_self}
    \caption{Disjoint union of a set with itself.}
    \label{fig:disjointself}
\end{figure}

\devel{
    \todotextjira{401}{@J: Fix!}
    In the case of the cartesian product of two sets we had projection maps, as in \XXX.
    For the disjoint union of sets, we have instead \emph{inclusion maps}.
    Thus we have a diagram of this form:

    \begin{figure}[h!
        ]
        \centering
        \includesag{050_coprod_disunion_diagram}
        \caption{}
        \label{fig:coprod_disunion_diagram}
    \end{figure}
}
\begin{marginfigure}
    \centering
    \includesag{050_example_coprod_max_cont}
    \caption{Taking the minimum}
    \label{fig:exa_coprod_max_cont}
\end{marginfigure}
\begin{example}
    \label{exa:min-as-prod-cont-to-rename}
    This example is ``dual'' to \cref{exa:min-as-prod-cont}.
    The category in question in the same one: objects are elements of~$\reals$ and morphisms are inequalities.
    The coproduct is ``taking the maximum''; its universal property is illustrated in \cref{fig:exa_prod_min_cont}.
    It says that if~$t \in \reals$ is such that~$t \geq x_1$ and $t \geq x_2$, then~$t \geq \min \{ x_1, x_2 \}$.

\end{example}

\begin{marginfigure}
    \centering
    \includesag{050_example_coprod_lcm_cont}
    \caption{Taking the least common multiple}
    \label{fig:exa_coprod_lcm_cont}
\end{marginfigure}
\begin{example}
    \label{exa:lcm-as-coprod-cont}
    This example is ``dual'' to \cref{exa:gcd-as-prod-cont}.
    The category we're working in has natural numbers as its objects, and morphisms are given by the relation "divides".
    The coproduct is ``taking the least common multiple''; its universal property is visualized in \cref{fig:exa_prod_gcd}.
    % For a concrete example, let $m = 12$ and $n = 18$, so $\gcd \{12, 18 \} = 6$. If we take $t = 3$, which divides both $12$ and $18$, we see that, indeed, $3$ also divides $6 = \gcd \{12, 18 \}$. And if we take $t = 2$, which \emph{also} divides both $12$ and $18$, we see that it is \emph{also} true that $2$ also divides $6 = \gcd \{12, 18 \}$.
\end{example}

\begin{marginfigure}
    \centering
    \includesag{050_example_coprod_union_cont}
    \caption{Taking the union}
    \label{fig:exa_coprod_union_cont}
\end{marginfigure}

\begin{example}
    \label{exa:union-as-coprod-cont}
    This example is ``dual'' to \cref{exa:intersection-as-prod-cont}.
    Given a set~$\setA$ and arbitrary subsets~$\subA_1, \subA_2 \subseteq \setA$, we drew an arrow~$\subA_1 \to \subA_2$ iff~$\subA_1 \subseteq \subA_2$.
    The category in question here has, as its objects, the subsets of $\setA$, and its morphisms are inclusions between them.
    The coproduct is ``taking the union''; its universal property is visualized in \cref{fig:exa_prod_intersection}.
    As a concrete example, consider again~$\setA = \{ 1, 2, 3, 4 \}$,~$\subA_1 = \{ 1, 2, 3 \}$, and~$\subA_2 = \{2, 3, 4 \}$.
    So~$\subA_1 \cup \subA_2 = \{1,2, 3,4\}$.
    If we choose~$\setC = \{ 2 \}$, we see that~$\setC \subseteq \subA_1$ and~$\setC \subseteq \subA_2$, and that also~$\setC \subseteq \subA_1 \cup \subA_2$ (as it must, according to the universal property).
    The situation is similar if we choose~$\setC = \{ 1\}$ or~$T = \emptyset$.
\end{example}

\begin{marginfigure}
    \centering
    \includesag{050_example_coprod_disjunction_cont}
    \caption{Taking the disjunction}
    \label{fig:exa_coprod_disjunction_cont}
\end{marginfigure}
\begin{example}
    \label{exa:disjunction-as-coprod-cont}
    This example is ``dual'' to \cref{exa:intersection-as-prod-cont}.
    Again consider the set~$\setA = \{ \true, \false \}$ of logical propositions and for any~$\ela, \elb  \in \setA$, we drew an arrow~$\ela \to \elb$ iff
    \begin{equation*}
        \prfperiod{\ela}{\elb}
    \end{equation*}
    The category we are working with has~$\setA$ as its set of objects, and its morphisms are logical implications.
    The coproduct is ``taking the disjunction'' (the logical operation ``or''); the universal property is shown in \cref{fig:exa_prod_conjunction}.
\end{example}

\devel{
    \begin{marginfigure}
        \centering
        \includesag{050_example_coprod_join_cont}
        \caption{Taking the join}
        \label{fig:exa_coprod_join_cont}
    \end{marginfigure}
    \begin{example}
        \label{exa:join-as-coprod-cont}
        \todotextjira{401}{@J: Finish}
        This example is ``dual'' to \cref{exa:meet-as-prod-cont}.
        We considered \XXX and we drew an arrow~$\ela \to \elb$ iff~$\ela \leq \elb$.
        The category at play here is the one corresponding to the poset underlying \XXX .
        The categorical product of two elements is their join (least upper bound); the universal property is illustrated in \cref{fig:exa_prod_meet}.
        \todotextjira{401}{@J: finish \cref{exa:join-as-coprod-cont}}
    \end{example}
}

As you can see from the above list of examples, the notion of coproduct involves diagrams of the type in \cref{fig:coprod_generic}.

\begin{figure}[h!]
    \centering
    \includesag{050_coprod_generic}
    \caption{}
    \label{fig:coprod_generic}
\end{figure}

As mentioned above, the disjoint union is a particular instance -- in the category \Set  -- of the notion of ``coproduct''.
We will now give the definition of a coproduct in an arbitrary category.
Note that it is very similar to the definition that we gave, in the previous section, for the product -- but with a few twists.
Analogous remarks to those we gave following the definition of the product apply here!

\linkvideo{spring2021-coproducts:coproducts:cat-prod} % Categorical coproduct
\begin{ctdefinition}[Categorical Coproduct]
    \label{def:catcoproduct}
    Let~\CatC be a category and let~$\Obja, \Objb \in \ObC$ be objects.
    The \emph{\iindex{coproduct}} of~$\Obja$ and~$\Objb$ is:  \\
    \constit
    \begin{enumerate}
        \item an object~$\styleobj{Z} \in \ObC$ (``the coproduct of $\Obja$ and $\Objb$'')
        \item \emph{injection morphisms}~$\injA \colon \Obja \mto \styleobj{Z} $ and~$\injB \colon \Objb \mto \styleobj{Z} $
    \end{enumerate}
    \condit
    \begin{enumerate}
        \item For any~$\styleobj{T} \in \ObC$ and any morphisms~$\mora \colon  \Obja \mto \styleobj{T}, \morb \colon \Objb \mto \styleobj{T}$, there exists a \emph{unique} morphism~$\psi_{\mora, \morb} \colon \styleobj{Z} \mto \styleobj{T}$ such that~$\mora = \injA\mthen \psi_{\mora, \morb}$ and~$\morb = \injB \mthen \psi_{\mora,\morb}$.
    \end{enumerate}
\end{ctdefinition}

\begin{remark}
    Diagrammatically, the condition above states that diagrams as in \cref{fig:coprod_general_1} commute.
    \begin{figure}[h!]
        \centering
        \includesag{60_defcoproduct}
        \caption{}
        \label{fig:coprod_general_1}
        \label{fig:def-coproduct-diagram}
    \end{figure}
    Similarly as was the case with the categorical product, ``the coproduct'' of~$\Obja$ and~$\Objb$ is unique only ``up to isomorphism''.
    Nevertheless, we will usually simply write~$\coprodMap{\Obja}{\Objb}$ for ``the'' coproduct (in place of~$\Objc$ above), and we will usually write~$\coprodMap{\mora}{\morb}$ in place of~$\psi_{\mora, \morb}$.
    The diagram in \cref{fig:coprod_general_1} then looks as in \cref{fig:def-coproduct-diagram-generic}
\end{remark}

\begin{figure}
    \centering
    \includesag{60_defcoproduct_generic}
    \caption{}
    \label{fig:def-coproduct-diagram-generic}
\end{figure}
%
%Note that~$\coprodMap{X}{Y}$ is different from~$\coprodMap{Y}{X}$, but the two are isomorphic~(\cref{fig:e16}).
%
%\begin{figure}[h!]
%    \centering
%    \includesag{30_dpcatfig_e16}
%    \caption{$\coprodMap{X}{Y}$ and~$\coprodMap{Y}{X}$ are isomorphic. \label{fig:e16}}
%\end{figure}

%For the case of vending machines, the inclusion maps are as in \cref{fig:inclusionvending}.
%
%\begin{figure}[h!]
%    \centering
%    \includesag{60_inclusion_vending}
%    \caption{Inclusion maps for the vending machine example. \label{fig:inclusionvending}}
%\end{figure}

\linkvideo{spring2021-coproducts:coproducts:batt-coproduct} % Extended coproduct example

% \begin{figure}[h!]
%   \begin{center}
%   %\includesag{60_coprod_batt_1}
%   \includesag{60_coprod_batt_bis}
%   \end{center}
%   \caption{Battery technologies, companies, prices, and a catalogue.}
%   \label{fig:coprod_batteries_1}
% \end{figure}

\begin{example}
    Let's consider two battery producers, each producing specific battery technologies.
    The first company produces a set~
    \begin{equation}
        \label{eq:exa-batt-setA}
        \setA=\{\technology{LiPo}, \technology{LCO},\technology{NiH2}\}
    \end{equation}
    of technologies, and the second one a set~
    \begin{equation}
        \label{eq:exa-batt-setB}
        \setB=\{\technology{LFP},\technology{LMO},\technology{LiPo}\}.
    \end{equation}
    Each technology has, for a specific desired battery mass, a specific price, belonging to a set of prices~
    \begin{equation}
        \label{eq:exa-batt-prices}
        \stylesets{P}=\{\unit[50]{},\unit[60]{},\unit[70]{},\unit[80]{}\}\cartprod \{CHF\}.
    \end{equation}
    We specify the price mappings for different technologies via the functions~$\mapa \colon \setA\to \stylesets{P}$ and~$\mapb\colon \setB\to \stylesets{P}$.
    A battery vendor wants to sell batteries from both producers and wants to create a battery catalogue, which needs to take into account which technology comes from which producer, to be able to distribute the earnings from the sales fairly.
    To this end, the disjoint union of the sets of technology is considered:
    \begin{equation}
        \label{eq:exa-batt-disunion}
        \setA+\setB=\{\disunionA{\technology{LiPo}},\disunionA{\technology{LCO}},\disunionA{\technology{NiH2}},\disunionB{\technology{LFP}},\disunionB{\technology{LMO}},\disunionB{\technology{LiPo}}\}.
    \end{equation}
    It is possible to map each technology in~$\setA,\setB$ to its own representative in~$\setA+\setB$ via the so-called injection maps:
    \begin{equation}
        \label{eq:exa-batt-inclusion-1}
        \begin{aligned}
            \iota_\setA\colon \setA & \to \setA+\setB             \\
            \setAel                 & \mapsto \disunionA{\setAel}
        \end{aligned}
    \end{equation}
    \begin{equation}
        \label{eq:exa-batt-inclusion-2}
        \begin{aligned}
            \iota_\setB\colon \setB & \to \setA+\setB              \\
            \setBel                 & \mapsto \disunionB{\setBel}.
        \end{aligned}
    \end{equation}
    This situation is graphically represented in \cref{fig:coprod_batteries_1}, and mimics the coproduct diagram presented in \cref{def:catcoproduct}.

    \begin{figure*}[tbh]
        \centering
        %\includesag{60_coprod_batt_1}
        \includesag{60_coprod_batt_bis}
        \caption{Battery technologies, companies, prices, and a catalogue.}
        \label{fig:coprod_batteries_1}
    \end{figure*}

    \begin{figure*}[tbh]
        \centering
        %\includesag{60_coprod_batt_2}
        \includesag{60_coprod_batt_2_bis}
        \caption{Example: why the union is not the coproduct in \Set.}
        \label{fig:coprod_batteries_2}
    \end{figure*}

    Here, the universal property says that there is a \textbf{unique} function~$\coprodMap{\mapa}{\mapb}\colon \setA+\setB\to \stylesets{P}$ such that
    \begin{equation}
        \label{eq:exa-batt-factoring-maps}
        \iota_\setA\mthen (\mapa+ \mapb)=\mapa \text{ and }\iota_\setB\mthen (\mapa+ \mapb)=\mapb.
    \end{equation}
    If we take a~$\styleelements{x}\in \setA+\setB$ is either ``from~$\setA$ or from~$\setB$'':
    \begin{equation}
        \label{eq:exa-batt-explain-sum-map}
        \text{either } \exists \setAel\in \setA\colon \styleelements{x}=\iota_\setA(\setAel) \text{ or }\exists \setBel\in \setB\colon \styleelements{x}=\iota_\setB(\setBel).
    \end{equation}
    From this, we can deduce that the desired map~$\coprodMap{\mapa}{\mapb}$ is:
    \begin{equation}
        \label{eq:exa-batt-def-sum-map}
        \begin{aligned}
            \coprodMap{\mapa}{\mapb} \colon  \setA + \setB & \to \stylesets{P} \\
            \styleelements{x}                              & \mapsto
            \begin{cases}
                \mapa(\styleelements{x}), & \text{if } \styleelements{x}=\iota_\setA(\setAel),\quad \setAel \in \setA, \\
                \mapb(\styleelements{x}), & \text{if } \styleelements{x}=\iota_\setB(\setBel),\quad \setBel \in \setB.
            \end{cases}
        \end{aligned}
    \end{equation}
    This is a specific example of \Set/\FinSet, in which the coproduct is a generalization of the concept of disjoint union.
    Now, we could spontaneously ask ourselves: why does the union not ``suffice'' for the coproduct definition in \Set?
    To see this, let's consider the same situation as before, but now having the catalogue of technologies given by~$\setA\cup \setB$ (\cref{fig:coprod_batteries_2}).
    The interpretation of maps~$\mapa,\mapb$ does not change, and injections work as depicted.
    Note, however, that when asked for a map from the technology~$\technology{LiPo}\in \setA\cup \setB$, we have no notion of the company which produces it, and we are therefore unsure whether to assign it to~$\mapa(\technology{LiPo})=\unit[50]{CHF}$ or~$\mapb(\technology{LiPo})=\unit[60]{CHF}$.
    Indeed, the unique map~$\coprodMap{\mapa}{\mapb}$ required by the universal property of the coproduct cannot exist, since in case~$\setA\cap \setB\neq \emptyset$, any element~$\styleelements{x}\in \setA\cap \setB$ should be simultaneously sent to~$\mapa(\styleelements{x})$ and~$\mapb(\styleelements{x})$.

\end{example}

% \begin{figure}[h!]
%   \begin{center}
%   %\includesag{60_coprod_batt_2}
%   \includesag{60_coprod_batt_2_bis}
%   \end{center}
%   \caption{Example of why the union is not the coproduct in \Set.}
%   \label{fig:coprod_batteries_2}
% \end{figure}

\begin{example}
    Given~$\setA,\setB\in \Obof\Rel$ (so~$\setA$ and~$\setB$ are sets) their coproduct is the disjoint union~$\setA+\setB$.
    The disjoint union of sets comes equipped with inclusion functions~$\iota_\setA\colon \setA\mto \setA+\setB$ and~$\iota_\setB\colon \setB\mto \setA+\setB$.
    If we turn these functions into relations
    \begin{equation*}
        \begin{aligned}
            R_{\iota_\setA} & \subseteq \setA\cartprod (\setA+\setB)  \\
            R_{\iota_\setB} & \subseteq \setB\cartprod (\setA+\setB).
        \end{aligned}
    \end{equation*}
    then these are the injection morphisms for the coproduct in \Rel.
    As an aside, we note that in \Rel products and coproducts are \emph{both} given by the disjoint union of sets.
    We will see later why this is might be expected.
\end{example}

\begin{example}
    Let~$m,n\in \natnumbers$, and draw an arrow~$m\to n$ if~$m$ divides~$n$.
    For instance, 6 divides 12 and hence there is an arrow~$6\to 12$.
    The coproduct between any two~$m,n\in \natnumbers$ in this category is given by the least common multiple.
\end{example}

\begin{example}
    Let's consider the ordered set~$\tup{\reals,\leq}$, where given~$x_1,x_2\in \reals$ we can draw an arrow~$x_1\to x_2$ if~$x_1\leq x_2$.
    By following the coproduct's commutative diagram, we know that the coproduct of~$x_1$ and~$x_2$ is a~$z\in \reals$ such that
    \begin{itemize}
        \item $x_1\leq z$;
        \item $x_2\leq z$;
        \item For all~$x\in \reals$ with~$x_1\leq x$ and~$x_2\leq x$, we have~$z\leq x$.
    \end{itemize}
    In other words, the coproduct of~$x_1,x_2\in \reals$ is given by~$\max\{x_1,x_2\}$, and is also called \emph{join}.
\end{example}

\begin{example}
    \label{ex:subset_coprod}
    Let~$\stylesets{S}$ be a set, and~$\setA,\setB\subseteq \stylesets{S}$ subsets.
    We can draw an arrow~$\setA\to \setB$ if~$\setA\subseteq \setB$.
    By following the coproduct's commutative diagram, it is easy to see that the coproduct of~$\setA$ and~$\setB$ is given by~$\setA\cup\setB$: the ``smallest'' set containing both~$\setA$ and~$\setB$.
\end{example}

\begin{example}[Adapted from~\cite{spivak2014category}]
    \label{def:ex_graph}
    \index{\Graph}
    We can define a category of graphs~\Graph.
    Objects of this category are graphs~$\graph=\tupp{\vertices,\arcs,\source,\target}$.
    Morphisms are \emph{graph homomorphisms}~(\cref{def:graph_homom})

    Given two graphs~$\graph=\tupp{\vertices,\arcs,\source,\target}$ and~$\graph'=\tupp{\vertices',\arcs',\source',\target'}$, their coproduct is a graph
    \begin{equation*}
        \graph+\graph'=\tupp{\vertices+\vertices',\arcs+\arcs',\source+\source',\target+\target'}.
    \end{equation*}
    In~$\graph+\graph'$, an arrow connects~$\vertexa_1$ to $\vertexa_2$ if:
    \begin{itemize}
        \item $\vertexa_1,\vertexa_2\in \vertices$ or~$\vertexa_1,\vertexa_2\in \vertices'$ (if both vertices belong to the same graph), and
        \item an arrow between from~$\vertexa_1$ to~$\vertexa_2$ exists in~$\graph$ or~$\graph'$.
    \end{itemize}
    Given~$\source \colon \arcs \to \vertices$ and~$\source'\colon \arcs'\to \vertices'$, we have:
    \begin{equation*}
        \begin{aligned}
            \source+ \source'\colon \arcs+ \arcs' & \to \vertices+ \vertices' \\
            x                                     & \mapsto
            \begin{cases}
                \source(x)  & \text{if } x\in \arcs   \\
                \source'(x) & \text{if } x\in \arcs'.
            \end{cases}
        \end{aligned}
    \end{equation*}
    and
    \begin{equation*}
        \begin{aligned}
            \target+\target '\colon \arcs+ \arcs ' & \to \vertices+ \vertices' \\
            x                                      & \mapsto
            \begin{cases}
                \target(x)  & \text{if } x\in \arcs   \\
                \target'(x) & \text{if } x\in \arcs'.
            \end{cases}
        \end{aligned}
    \end{equation*}
    This is nicely graphically representable.
    Consider two graphs as in~\cref{fig:ex_graphs_1}.

    \begin{figure}[h!]
        \centering
        \begin{tabular}{cc}
            \includesag{60_graph_1_1} & \includesag{60_graph_1_2}
        \end{tabular}
        \caption{Example of graphs for which we want to consider the coproduct. }
        \label{fig:ex_graphs_1}
    \end{figure}

    Their coproduct is a graph including the ``disjoint union'' of both original graphs, without connecting them (\cref{fig:graphs_2}).

    \begin{figure}[h!]
        \centering
        \includesag{60_graph_2_1}
        \caption{Example of coproduct of graphs. }
        \label{fig:graphs_2}
    \end{figure}

\end{example}
