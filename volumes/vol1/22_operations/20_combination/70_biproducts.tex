% !TEX root = chapter-standalone.tex
\section{Biproducts}

\devel{\includepdf[scale=0.8,pages={4,5},nup=1x3,frame,pagecommand={}]{ACT4E-06-posets.pdf}}


\todotext{Write intro text. Fill in with example of direct sum of vector spaces, and other examples... }


\begin{example}
  We can define a category \iindex{\Vect}, composed of:
  \begin{compactitem}
    \item \emph{Objects}: vector spaces;
    \item \emph{Morphisms}: linear maps;
    \item \emph{Identity morphisms}: identity maps;
    \item \emph{Composition}: composition of linear maps.
  \end{compactitem}
  It is a good exercise to prove that \Vect really forms a category. In the following, we want to look at the coproduct in \Vect. It is given by the \emph{direct sum} of vector spaces. Recall that given vector spaces~$V$ and~$W$, their direct sum is the set
  \begin{equation*}
    V\oplus W\definedas \{\tup{v,w}\mid v\in V,w\in W\},
  \end{equation*}
  equipped with a notion of addition and scalar multiplication derived component-wise from~$V$ and~$W$. For addition, this means that given~$\tup{v_1,w_1},\tup{v_2,w_2}\in V\oplus W$, their sum in~$V \oplus W$ is
  \begin{equation*}
    \tup{v_1,w_1}+\tup{v_2,w_2}\definedas \tup{v_1+v_2,w_1+w_2}.
  \end{equation*}
  The injection morphisms for the coproduct are given by:
  \begin{equation*}
    \begin{aligned}
      \iota_V \colon V&\to V\oplus W\\
      v&\mapsto \tup{v,0_W},\\
      \iota_W \colon W&\to V\oplus W\\
      w&\mapsto \tup{0_V,w},
    \end{aligned}
  \end{equation*}
  where~$0_V$ and~$0_W$ represent the zero vectors in~$V$ and~$W$. Let's now look at the universal property in this case, by considering any vector space $U\in \Ob_\Vect$, and linear maps~$S\colon V\to U$,~$T\colon W\to U$. The universal property says that we need a unique linear map~$S+T\colon V\oplus W \to U$ such that~$S=\iota_V\then h$ and~$T=\iota_W\then g$. By taking any~$\tup{v,w}\in V\oplus W$, we can write:
  \begin{equation*}
    \begin{aligned}
      h(\tup{v,w})&=h(\tup{v,0_W}+\tup{0_V,w})\\
      &=h(\iota_V(v)+\iota_W(w)) \\
      &=h(\iota_V(v))+h(\iota_W(v)) \qquad \qquad (h \text{ is linear})\\
      &=(\iota_V\then h)(v)+(\iota_W\then h)(w)\\
      &\overset{!}{=}Sv+Tw.
    \end{aligned}
  \end{equation*}
  We can hence write the map $S+T$ as
  \begin{equation*}
    \begin{aligned}
      S+T\colon V\oplus W&\to U\\
      \tup{v,w}&\mapsto Sv+Tw.
    \end{aligned}
  \end{equation*}
\end{example}