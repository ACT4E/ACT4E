% !TEX root = chapter-standalone.tex

\section{Yoneda}
\publictodomessage

\todotext{J: add yoneda lemma}

\todotext{J: add theorem about indiscernibles}

\todojira{104}{\bernina: @JL: Fill the section}


\section{Yoneda Embedding Theorem \& Lemma}
Here we will cover one of the most important results in Category Theory.
It gives us insight on the role that objects play
in a category, and how we can actually 'recognize' what their role is.
We will first define certain types of functors for any (locally small) category $\CatC$, and see how they actually reveal more hidden stucture on $\CatC$.

\begin{marginfigure}
    \centering
    \includegraphics[width=5cm]{representable-functor}
    %\includesag{diagram_two_functors}
    \label{fig:representable-functor}
\end{marginfigure}

\subsection{Representable Functors}
\begin{ctdefinition}[Hom-Functor]
    Let $\CatC$ be a category, and $\Obja \setin \Ob_\CatC$.
    We define the \SY{hom-functor} of $\Obja$ as $\Hom_\CatC(\Obja, -)\colon\CatC\fto\Set$.
    It maps $\Objd \setin \Ob_\CatC$ to the hom-set $\Hom_\CatC(\Obja, \Objd)$.
\end{ctdefinition}

\begin{ctdefinition}[Representable Functor]
    Let $\CatC$ be a category, and $\Obja, \Objb \setin \Ob_\CatC$ be objects.
    A functor $\funa\colon\CatC \fto \Set$ is said \SY{representable}, of representing object $\Objc\setin\Ob_\CatC$, if there exists a natural isomoprhism between $\funa$ and the functor $\Hom_\CatC[\Objc, -]$.
    Moreover, any morphism $\mora\colon\Obja\fto\Objb$ gets mapped to $\funa\fthen\mora$.

    That is, for each object $\Obja \setin \Ob_\CatC$, $\funa(\Obja)$ maps $\Obja$ to a set, which is naturally isomorphic to all the ways to map $\Objc$ to $\Obja$ in $\CatC$.
\end{ctdefinition}

Recall that we have defined a category of functors in the previous section.
We can now consider the category of representable fucntors in $\CatC$, which we will denote $[\CatC, \Set]$.
This new category will provide some more insight on 'unicity' of objects within $\CatC$.
More specifically, we will see that objects are characterized by their ``incoming`` and ``outgoing`` moprhisms.

\begin{theorem}[Yoneda Embedding]
    Let $\CatC$ be a category, and $\Obja, \Objb \setin \Ob_\CatC$.
    Then there is a natural bijection of sets between the hom-set $\Hom(\Obja,\Objb)$ and the hom-set $\Hom_{[\CatC, \Set]}(\Hom_\CatC(\Objb, -), \Hom_\CatC(\Obja, -))$ in the category of represenatble functors.
\end{theorem}
Let's break this down a bit.
We have two objects $\Obja, \Objb$ representing two functors $\Hom_\CatC(\Obja,-), \Hom_\CatC(\Objb,-)$.
These functors can be also considered as objects in $[\CatC, \Set]$.
We are essentially saying that the ways to map $\Obja$ to $\Objb$ in $\CatC$ is isomorphic to the ways to map $\Hom_\CatC(\Obja,-)$ to $ \Hom_\CatC(\Objb,-)$ in $[\CatC, \Set]$.

\begin{example}
    \begin{marginfigure}
        \centering
        \includesag{rep-functor-1}
        %\includegraphics[width=5cm]{rep-functor-1}
        \caption{\label{fig:rep-functor-1}}
    \end{marginfigure}

    \begin{marginfigure}
        \centering
        \includesag{rep-functor-2}
        %\includegraphics[width=5cm]{rep-functor-2}
        \caption{\label{fig:rep-functor-2}}
    \end{marginfigure}
    Let us consider~$\Obja, \Objb, \Objd \setin \Ob_\CatC$, with the following hom-sets: $\Hom_\CatC(\Obja, \Objd) = \makeset{\moran{1},\moran{2}}, \Hom_\CatC(\Objb, \Objd) = \makeset{\morbn{1},\morbn{2}}$.
    We can thus infer that $\Hom_\CatC(\Obja, \Objb) = \makeset{\moran{1}\mthen\morbn{1}, \moran{1}\mthen\morbn{2} \moran{2}\mthen\morbn{1}, \moran{2}\mthen\morbn{2}}$.

    Analogously, the ways to map $\Hom_\CatC(\Objb, \Objd)$ to $\Hom_\CatC(\Obja, \Objd)$ is given by the set $\makeset{\morbn{1} \mapsto \moran{1}, \morbn{1} \mapsto \moran{2}, \morbn{2} \mapsto \moran{1}, \morbn{2} \mapsto \moran{2}}$.
    Both sets are bijective.
    Moreover, we see that in this case, both objects $\Obja, \Objb$ have the same number of ``ins`` and ``outs`` with respect to $\Objd$.
    A general result in Category Theory is that if the hom-sets of $\Obja$ and $\Objb$ are isomorphic for all $\Objd \setin \Ob_\CatC$, then $\Obja$ and $\Objb$ are isomoprhic, and essentially play the same role within $\CatC$.
\end{example}




