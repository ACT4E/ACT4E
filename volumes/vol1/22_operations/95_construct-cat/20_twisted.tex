% !TEX root = chapter-standalone.tex
\section{Intervals as categories}

\subsection{Twisted arrow category}


\todotext{Finish writing this definition}

\begin{ctdefinition}[Twisted arrow category]\label{def:twisted-arrow-category}
  \label{def:twisted-arrow}
  Given a category \CatC, we denote its \emph{twisted arrow category} by $\twisted{\CatC}$. This is a category which is composed of:
  \begin{compactenum}
    \item \emph{Objects:} Arrows (morphisms) in \CatC.
    \item \emph{Morphisms:} A morphism between two arrows $\mora\colon \Obja \mto \Objb $, $\morb\colon \Objc \mto \Objd$ is given by a pair of arrows $\tup{\morc,\mord}$ such that the following diagram commutes:
    \begin{center}
      \includesag{180_twistedarrow}
    \end{center}
    \item \emph{Identity morphisms:}
    \item \emph{Composition:}
  \end{compactenum}
\end{ctdefinition}

\begin{gradedexercise}
Prove that \cref{def:twisted-arrow-category} does define a category.
\end{gradedexercise}

\todotext{Write solution}

\begin{example}[Intervals]\label{exa:twisted-arrow-poset}
  Consider a poset~$\posA$. The twisted arrow category~$\twisted{\posA}$ is isomorphic to the poset (viewed as a category) of nonempty \emph{intervals} in $\posA$. Recall that, given elements $a, b \in \posA$, the interval $[a,b]$ is
  $$[a,b]:=\{p \in \posA \mid a\posAleq p \posAleq b\}.$$
\end{example}

\begin{exercise}
Prove the statement in \cref{exa:twisted-arrow-poset}.
\end{exercise}
\begin{solution}

\todotext{Write the solution.}
\end{solution}
\begin{remark}
  Recall \cref{sec:posetsarecats} and note that the map which sends a poset (a category) to its twisted arrow category is a functor, which sends objects of the poset \todotext{Finish this remark}
\end{remark}

\subsection{Arrow category}

We are working on this section, and more content will appear.


\todotext{to write}
