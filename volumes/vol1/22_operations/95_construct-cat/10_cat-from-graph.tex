% !TEX root = chapter-standalone.tex


\section{Generating categories from graphs}
\label{sec:catsfromgraphs}
\todojira{108}{Put this section after all concrete examples}

The following definition provides a way of turning any graph into a category.

\begin{ctdefinition}[Free category on a graph]
    \label{def:free-category}
    Let~$G=(V,A,s,t)$ be a graph.
    The \emph{free category on~$G$}, denoted~$\Free(G)$, has as objects the vertices~$V$ of~$G$, and given vertices~$x\in V$ and~$y\in V$, the morphisms~$\Free(G)(x,y)$ are the paths from~$x$ to~$y$.
    %A path is a sequence of ``consecutive'' edges, \text{\ie } the source of a subsequent edge is equal to the target of its predecessor. We also formally allow for ``empty paths'', \text{\ie } a sequence of "zero"-many edges which starts and ends at the same vertex.
    The composition of morphisms is given by concatenation of paths, and for any object~$x \in V$, the associated identity morphism~$\id_x$ is the trivial path which starts and ends at~$x$.
\end{ctdefinition}

\todographicsjira{107}{Show a picture of a graph and its induced category.}


We leave it to the reader to check that the above definition does indeed define a category.
%\text{\ie } to check that the composition of paths is again a path, and that the associative law and the law for identity morphisms hold.


\todojira{109}{@Gioele: Let's do it ourselves}


\begin{gradedexercise}[\exname{HowManyMorphisms}]
    \label{ex:HowManyMorphisms}
    Consider the following five graphs.
    For each graph $G$, how many morphisms in total are there in the associated category~$\Free(G)$?
    \todographicsjira{110}{Adjust style of graphs}
    %\begin{figure}[h!]
    \begin{center}
        \includesag{20_dpcatfig_example_graphs}
    \end{center}
    %\end{figure}
\end{gradedexercise}
\solutionof{HowManyMorphisms}



