% !TEX root = chapter-standalone.tex


\section{Generating categories from graphs}
\label{sec:catsfromgraphs}
\todostructure{Put this section after all concrete examples}
To begin, we recall some formal definitions related to (directed) graphs.


\todographics{Need nice pictures of graphs and various quantities.}
\begin{definition}[Graph]
  \label{def:graph}
  A (directed) \emph{\iindex{graph}}~$G=\tup{V,A,s,t}$ consists of a set of vertices~$V$, a set of arrows~$A$, and two functions~$s,t\colon A\to V$, called the \emph{source} and \emph{target} functions, respectively. Given~$a\in A$ with~$s(a)=v$ and~$t(a)=w$, we say that~$a$ is an \emph{arrow} from~$v$ to~$w$.
\end{definition}

\begin{remark}
  Both directed graphs and undirected graphs play a prominent role in many kinds of mathematics. In this text, we work primarily with directed graphs and so, from now on, we will drop the ``directed'': unless indicated otherwise, the word ``graph'' will mean ``directed graph''.
\end{remark}

\begin{definition}[Path]
  \label{def:path}
  Let~$G$ be a graph. A \emph{path} in~$G$ is a sequence of arrows such that the target of one arrow is the source of the next. The \emph{length} of a path is the number of arrows in the sequence. We also formally allow for sequences made up of ``zero-many'' arrows (such paths therefore have length zero). We call such paths \emph{trivial} or \emph{empty}. If paths describe a journey, then trivial paths correspond to ``not going anywhere''. The notions of source and target for arrows extend, in an obvious manner, to paths. For trivial paths, the source and target always coincide.
\end{definition}

The following definition provides a way of turning any graph into a category.

\begin{ctdefinition}[Free category on a graph]
  \label{def:free-category}
  Let~$G=(V,A,s,t)$ be a graph. The \emph{free category on~$G$}, denoted~$\Free(G)$, has as objects the vertices~$V$ of~$G$, and given vertices~$x\in V$ and~$y\in V$, the morphisms~$\Free(G)(x,y)$ are the paths from~$x$ to~$y$.
%A path is a sequence of ``consecutive'' edges, \text{\ie } the source of a subsequent edge is equal to the target of its predecessor. We also formally allow for ``empty paths'', \text{\ie } a sequence of "zero"-many edges which starts and ends at the same vertex.
  The composition of morphisms is given by concatenation of paths, and for any object~$x \in V$, the associated identity morphism~$\id_x$ is the trivial path which starts and ends at~$x$.
\end{ctdefinition}

\todographics{Show a picture of a graph and its induced category.}


We leave it to the reader to check that the above definition does indeed define a category.
%\text{\ie } to check that the composition of paths is again a path, and that the associative law and the law for identity morphisms hold.


\todo[inline]{Let's do it ourselves}


\begin{exercise}
  Consider the following five graphs. For each graph $G$, how many morphisms in total are there in the associated category~$\Free(G)$?

%\begin{figure}[h!]
  \begin{center}
    \includesag{20_dpcatfig_example_graphs}
  \end{center}
%\end{figure}
\end{exercise}
\begin{solution}
  \todotext{to write}
\end{solution}
