
\section{Opposite Category}

\begin{ctdefinition}[Opposite category]
  \label{def:oppositecat}
  Given a category~\CatC, the \emph{\iindex{opposite category}}~$\CatC\op$ has the same objects as~\CatC, but a morphism~$\mora\colon \obja\to \objb$ in~$\CatC\op$ is the same as a morphism~$\mora\colon \objb\to \obja$ in~\CatC. Furthermore, a composite of morphisms~$\mora \mthen \morb$ in~$\CatC\op$ is the composite~$\morb\then \mora$ in~\CatC.
\end{ctdefinition}

\begin{example}[Opposite of a poset]
  We have defined the opposite of a poset in \cref{sec:opposite-of-a-poset}.
  Since a poset is a category (\cref{sec:posetsarecats})
  this is an example of the above definition. Consider~\CatC representing any poset. We can think of~$(\cdot)\op$ as a functor of the form
  \begin{equation}
  (\cdot)
    \op\colon \CatC \to \CatC\op,
  \end{equation}
  which preserves objects (\ie , every object~$A \in \CatC$ is mapped to~$A^*\in \CatC\op$), and maps each morphism~$f\colon A\to B$ in~\CatC to a morphism~$f'\colon B\to A$ in~$\CatC\op$. This is a functor:
  \begin{itemize}
    \item For any object~$A$, the identity morphism~$\id_A\colon A\to A$ in~\CatC is itself, \ie , it is mapped to~$\id_A'\colon A\to A$, meaning that~$(\id_A)\op=\id_{(A)\op}=\id_A$.
    \item Given two morphisms~$a\ordleq b$,~$b\ordleq c$ in~\CatC, one has
    \begin{equation}
      \begin{aligned}
      (a\ordleq_{\CatC} c)
        \op &=c\ordleq_{\CatC\op} a\\
        &=(b\ordleq_{\CatC\op} a)\wedge (c\ordleq_{\CatC\op}b)\\
        &=(a\ordleq_{\CatC} b)\op \wedge (b\ordleq_{\CatC} c)\op.
      \end{aligned}
    \end{equation}
  \end{itemize}
\end{example}
