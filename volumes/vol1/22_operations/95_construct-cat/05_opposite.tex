% !TEX root = chapter-standalone.tex

\section{Opposite Category}

\begin{ctdefinition}[Opposite category]
    \label{def:opposite-category}
    \label{def:oppositecat}
    Given a category~\CatC, its \emph{\iindex{opposite category}}~$\CatC\op$ is specified by:
    \begin{enumerate}
        \item \emph{Objects}: $\Obof{\CatC\op} = \Obof\CatC$.
              Given~$\Obja \in  \Obof\CatC$, we will sometimes (though not always) write $\Obja\op$ to signify when we are thinking of~$\Obja$ as an object of~$\Obof\CatC\op$.

        \item \emph{Morphisms}: Given objects~$\Obja\op,  \Objb\op \in \Obof{\CatC\op} = \Obof\CatC$,
              \begin{equation}
                  \label{eq:homset-op-cat}
                  \HomSet{\CatC\op}{\Obja\op}{\Objb\op} \coloneqq \HomSet{\CatC}{\Objb}{\Obja}.
              \end{equation}
              %
              Given~$\mora \in \HomSet{\CatC}{\Objb}{\Obja}$, when we are thinking of it as an element of~$\HomSet{\CatC\op}{\Obja\op}{\Objb\op}$, we will sometimes write~$\mora\op$.
              %
        \item \emph{Identity morphisms}: Given~$\Obja\op \in \Obof{\CatC\op}$, its identity morphism is
              \begin{equation}
                  \catid_{\Obja\op} \coloneqq \catid_{\Obja}\op.
              \end{equation}
              %
        \item \emph{Composition}: Let morphisms~$\mora\op \in \HomSet{\CatC\op}{\Obja\op}{\Objb\op}$ and~$\morb\op \in \HomSet{\CatC\op}{\Objb\op}{\Objc\op}$, then
              \begin{equation}
                  \mora\op \then_{\CatC\op} \morb\op \coloneqq (\morb \then_\CatC \mora)\op.
              \end{equation}
    \end{enumerate}
\end{ctdefinition}

\vfill
\begin{gradedexercise}[\exname{OppositeCat}]
    \label{ex:OppositeCat}
    Verify that \cref{def:oppositecat} defines a category.
    In other words, check that its constituents satisfy the conditions of associative and unitality.
\end{gradedexercise}

\solutionof{OppositeCat}

\todographicsjira{105}{Graphics for morphisms in definition}

\devel{

    \todojira{106}{The example below needs some attention I think.
        It seems mixed-up / confusing to me.
    }
    \begin{example}[Opposite of a poset]
        We have defined the opposite of a poset in \cref{sec:opposite-of-a-poset}.
        Since a poset is a category (\cref{sec:posetsarecats}) this is an example of the above definition.
        Consider a category~\CatC representing any poset.
        We can think of~$(\cdot)\op$ as a functor of the form
        \begin{equation}
            (\cdot)
            \op\colon \CatC \fto \CatC\op,
        \end{equation}
        which preserves objects (every object~$\Obja \in \ObC$ is mapped to~$A^*\in \Ob_{\CatC\op}$), and maps each morphism~$\mora\colon \Obja\to \Objb$ in~\CatC to a morphism~$\morb\colon \Objb\to \Obja$ in~$\CatC\op$.
        This is a functor:
        \begin{itemize}
            \item For any object~$\Obja$, the identity morphism~$\catid_\Obja\colon \Obja\to \Obja$ in~\CatC is itself: it is mapped to~$\catid_\Obja'\colon \Obja\to \Obja$, meaning that~$(\catid_\Obja)\op=\catid_{(\Obja)\op}=\catid_\Obja$.
            \item Given two morphisms~$\mora \posleq \morb$,~$\morb\posleq \morc$ in~\CatC, one has
                  \begin{equation}
                      \begin{aligned}
                          (\mora \ordleq_{\CatC} \morc)
                          \op & =\morc\posleq_{\CatC\op} \mora                                           \\
                              & =(\morb \posleq_{\CatC\op} \mora)\wedge (\morc\posleq_{\CatC\op}\mora)   \\
                              & =(\mora\posleq_{\CatC} \morb)\op \wedge (\morb\ordleq_{\CatC} \morc)\op.
                      \end{aligned}
                  \end{equation}
        \end{itemize}
    \end{example}
}
