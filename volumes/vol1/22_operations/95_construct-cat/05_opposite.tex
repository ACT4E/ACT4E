% !TEX root = chapter-standalone.tex
\section{Opposite Category}

\todotext{Rewrite this definition more clearly}

\begin{ctdefinition}[Opposite category]
  \label{def:oppositecat}
  Given a category~\CatC, the \emph{\iindex{opposite category}}~$\CatC\op$ has the same objects as~\CatC, but a morphism~$\mora\colon \Obja\to \Objb$ in~$\CatC\op$ is the same as a morphism~$\mora\colon \Objb\to \Obja$ in~\CatC. Furthermore, a composite of morphisms~$\mora \mthen \morb$ in~$\CatC\op$ is the composite~$\morb\then \mora$ in~\CatC.
\end{ctdefinition}
\todographics{Morphisms in def???}

\devel{

\JL{The example below needs some attention I think... several things seem mixed up / confusing to me.}
\begin{example}[Opposite of a poset]
  We have defined the opposite of a poset in \cref{sec:opposite-of-a-poset}.
  Since a poset is a category (\cref{sec:posetsarecats})
  this is an example of the above definition. Consider a category~\CatC representing any poset. We can think of~$(\cdot)\op$ as a functor of the form
  \begin{equation}
  (\cdot)
    \op\colon \CatC \fto \CatC\op,
  \end{equation}
  which preserves objects (every object~$\Obja \in \ObC$ is mapped to~$A^*\in \Ob_{\CatC\op}$), and maps each morphism~$\mora\colon \Obja\to \Objb$ in~\CatC to a morphism~$\morb\colon \Objb\to \Obja$ in~$\CatC\op$. This is a functor:
  \begin{itemize}
    \item For any object~$\Obja$, the identity morphism~$\catid_\Obja\colon \Obja\to \Obja$ in~\CatC is itself: it is mapped to~$\catid_\Obja'\colon \Obja\to \Obja$, meaning that~$(\catid_\Obja)\op=\catid_{(\Obja)\op}=\catid_\Obja$.
    \item Given two morphisms~$\mora \posleq \morb$,~$\morb\posleq \morc$ in~\CatC, one has
    \begin{equation}
      \begin{aligned}
      (\mora \ordleq_{\CatC} \morc)
        \op &=\morc\posleq_{\CatC\op} \mora\\
        &=(\morb \posleq_{\CatC\op} \mora)\wedge (\morc\posleq_{\CatC\op}\mora)\\
        &=(\mora\posleq_{\CatC} \morb)\op \wedge (\morb\ordleq_{\CatC} \morc)\op.
      \end{aligned}
    \end{equation}
  \end{itemize}
\end{example}
}