% !TEX root = chapter-standalone.tex
\showslides{

    \section{Lecture slides materials}

    \begin{forslides}

        \begin{equation}
            \label{eq:combination-000}
            \setA
        \end{equation}

        \begin{equation}
            \label{eq:combination-001}
            \setB
        \end{equation}

        \begin{equation}
            \label{eq:combination-002}
            \setA = \makeset{\sbanana, \sapple, \sgrapes, \scarrot}
        \end{equation}

        \begin{equation}
            \label{eq:combination-003}
            \setB = \makeset{ \sbretzel, \sfondue }
        \end{equation}

        \begin{equation}
            \label{eq:combination-004}
            \projA
        \end{equation}

        \begin{equation}
            \label{eq:combination-005}
            \projB
        \end{equation}

        \begin{equation}
            \label{eq:combination-006}
            \setA \cartprod \setB
        \end{equation}

        \begin{equation}
            \label{eq:combination-007}
            \projA(\tup{\ela,\elb}) =  \ela
        \end{equation}

        \begin{equation}
            \label{eq:combination-008}
            \projB(\tup{\ela,\elb}) = \elb
        \end{equation}

        \begin{equation}
            \label{eq:combination-009}
            \elna{1},\elna{2}\setin \reals
        \end{equation}

        \begin{equation}
            \label{eq:combination-010}
            \elna{1} \mto \elna{2}
        \end{equation}

        \begin{equation}
            \label{eq:combination-011}
            \elna{1}\Rleq \elna{2}
        \end{equation}

        \begin{equation}
            \label{eq:combination-012}
            \elna{1} = 3
        \end{equation}

        \begin{equation}
            \label{eq:combination-013}
            \elna{2} = 7
        \end{equation}

        \begin{equation}
            \label{eq:combination-014}
            3 \geq \min \makeset{3, 7} \leq 7
        \end{equation}

        \begin{equation}
            \label{eq:combination-015}
            m, n \setin \natnumbers
        \end{equation}

        \begin{equation}
            \label{eq:combination-016}
            m \mto n
        \end{equation}

        \begin{equation}
            \label{eq:combination-017}
            m | n
        \end{equation}

        \begin{equation}
            \label{eq:combination-018}
            m = 6
        \end{equation}

        \begin{equation}
            \label{eq:combination-019}
            n = 9
        \end{equation}

        \begin{equation}
            \label{eq:combination-020}
            \gcd \makeset{6, 9 } = 3
        \end{equation}

        \begin{equation}
            \label{eq:combination-021}
            3 | 6
        \end{equation}

        \begin{equation}
            \label{eq:combination-022}
            3 | 9
        \end{equation}

        \begin{equation}
            \label{eq:combination-023}
            \subAn{1}, \subAn{2} \setsubseteq \setA
        \end{equation}

        \begin{equation}
            \label{eq:combination-024}
            \subAn{1} \mto \subAn{2}
        \end{equation}

        \begin{equation}
            \label{eq:combination-025}
            \subAn{1} \setsubseteq \subAn{2}
        \end{equation}

        \begin{equation}
            \label{eq:combination-026}
            \setA = \makeset{ 1, 2, 3, 4 }
        \end{equation}

        \begin{equation}
            \label{eq:combination-027}
            \subAn{1} = \makeset{ 1, 2, 3 }
        \end{equation}

        \begin{equation}
            \label{eq:combination-028}
            \subAn{2} = \makeset{2, 3, 4 }
        \end{equation}

        \begin{equation}
            \label{eq:combination-029}
            \subA_1 \setintersection \subA_2 = \makeset{2, 3 }
        \end{equation}

        \begin{equation}
            \label{eq:combination-030}
            \makeset{ 1, 2, 3 } \setsupseteq \makeset{2, 3 } \setsubseteq \makeset{2, 3, 4 }
        \end{equation}

        \begin{equation}
            \label{eq:combination-031}
            \setA = \makeset{ \true, \false }
        \end{equation}

        \begin{equation}
            \label{eq:combination-032}
            \ela, \elb  \setin \setA
        \end{equation}

        \begin{equation}
            \label{eq:combination-033}
            \ela \mto \elb
        \end{equation}

        \begin{equation}
            \label{eq:combination-034}
            \ela \Imp \elb
        \end{equation}

        \begin{equation}
            \label{eq:combination-035}
            \ela = \true
        \end{equation}

        \begin{equation}
            \label{eq:combination-036}
            \elb= \false
        \end{equation}

        \begin{equation}
            \label{eq:combination-037}
            \true \booland \false = \false
        \end{equation}

        \begin{equation}
            \label{eq:combination-038}
            \true \Leftarrow \false \Imp \false
        \end{equation}

        \begin{equation}
            \label{eq:combination-039}
            \Snacks=\makeset{\sapple,\sbanana,\scarrot}
        \end{equation}

        \begin{equation}
            \label{eq:combination-040}
            \Drinks=\makeset{\swater,\stea}
        \end{equation}

        \begin{equation}
            \label{eq:combination-041}
            \Participants
        \end{equation}

        \begin{equation}
            \label{eq:combination-042}
            \eats \colon \Participants \sto \Snacks
        \end{equation}

        \begin{equation}
            \label{eq:combination-043}
            \drinks \colon  \Participants \sto \Drinks
        \end{equation}

        \begin{equation}
            \label{eq:combination-044}
            \Snacks\cartprod \Drinks
        \end{equation}

        \begin{equation}
            \label{eq:combination-045}
            \meal \colon \Participants\sto \Snacks\cartprod \Drinks
        \end{equation}

        \begin{equation}
            \label{eq:combination-046}
            \eats
        \end{equation}

        \begin{equation}
            \label{eq:combination-047}
            \drinks
        \end{equation}

        \begin{equation}
            \label{eq:combination-048}
            \catprodphi_{\eats,\drinks}
        \end{equation}

        \begin{equation}
            \label{eq:combination-049}
            \eats =\catprodphi_{\eats,\drinks}\mthen \proj_1
        \end{equation}

        \begin{equation}
            \label{eq:combination-050}
            \drinks=\catprodphi_{\eats,\drinks}\mthen \proj_2
        \end{equation}

        \begin{equation}
            \label{eq:combination-051}
            \Participants
        \end{equation}

        \begin{equation}
            \label{eq:combination-052}
            \eats\colon \Participants \mto \Snacks
        \end{equation}

        \begin{equation}
            \label{eq:combination-053}
            \drinks\colon \Participants \mto \Drinks
        \end{equation}

        \begin{equation}
            \label{eq:combination-054}
            \catprodphi_{\eats,\drinks} \colon \Participants \to \Snacks \cartprod \Drinks
        \end{equation}

        \begin{equation}
            \label{eq:combination-055}
            \Snacks \mfrom \Snacks \cartprod \Drinks \mto \Drinks
        \end{equation}

        \begin{equation}
            \label{eq:combination-056}
            \Snacks \mfrom  \Participants \mto \Drinks
        \end{equation}

        \begin{equation}
            \label{eq:combination-057}
            \catprodphi_{\eats,\drinks}
        \end{equation}

        \begin{equation}
            \label{eq:combination-058}
            \Obja \mfrom \text{``product of } \Obja \text{ and } \Objb \text{''}  \mto \Objb
        \end{equation}

        \begin{equation}
            \label{eq:combination-059}
            \Obja \mfrom \styleobj{T} \mto \Objb
        \end{equation}

        \begin{equation}
            \label{eq:combination-060}
            \setA
        \end{equation}

        \begin{equation}
            \label{eq:combination-061}
            \setB
        \end{equation}

        \begin{equation}
            \label{eq:combination-062}
            \setA \setdisunion \setB
        \end{equation}

        \begin{equation}
            \label{eq:combination-063}
            \reals
        \end{equation}

        \begin{equation}
            \label{eq:combination-064}
            t \setin \reals
        \end{equation}

        \begin{equation}
            \label{eq:combination-065}
            t \geq \elna{1}
        \end{equation}

        \begin{equation}
            \label{eq:combination-066}
            t \geq \elna{2}
        \end{equation}

        \begin{equation}
            \label{eq:combination-067}
            t \geq \max \makeset{ \elna{1}, \elna{2} }
        \end{equation}

        \begin{equation}
            \label{eq:combination-068}
            \subAn{1}, \subAn{2} \setsubseteq \setA
        \end{equation}

        \begin{equation}
            \label{eq:combination-069}
            \subAn{1} \to \subAn{2}
        \end{equation}

        \begin{equation}
            \label{eq:combination-070}
            \subAn{1} \setsubseteq \subAn{2}
        \end{equation}

        \begin{equation}
            \label{eq:combination-071}
            \setA = \makeset{ \sbretzel, \sfondue, \schoco, \sburger }
        \end{equation}

        \begin{equation}
            \label{eq:combination-071b}
            \subAn{1} = \makeset{ \sbretzel, \sfondue, \schoco }
        \end{equation}

        \begin{equation}
            \label{eq:combination-072}
            \subAn{2} = \makeset{\sfondue, \schoco, \sburger }
        \end{equation}

        \begin{equation}
            \label{eq:combination-073}
            \subAn{1} \setunion \subAn{2} = \setA
        \end{equation}

        \begin{equation}
            \label{eq:combination-074}
            \setC = \makeset{ \sfondue }
        \end{equation}

        \begin{equation}
            \label{eq:combination-075}
            \setC \setsubseteq \subAn{1}
        \end{equation}

        \begin{equation}
            \label{eq:combination-076}
            \setC \setsubseteq \subAn{2}
        \end{equation}

        \begin{equation}
            \label{eq:combination-077}
            \setC \setsubseteq \subAn{1} \setunion \subAn{2}
        \end{equation}

        \begin{equation}
            \label{eq:combination-078}
            \setC = \makeset{ \sbretzel }
        \end{equation}

        \begin{equation}
            \label{eq:combination-079}
            T = \Emptyset
        \end{equation}

        \begin{equation}
            \label{eq:combination-080}
            \setA = \makeset{ \true, \false }
        \end{equation}

        \begin{equation}
            \label{eq:combination-081}
            \ela, \elb \setin \setA
        \end{equation}

        \begin{equation}
            \label{eq:combination-082}
            \ela \to \elb
        \end{equation}

        \begin{equation}
            \label{eq:combination-083}
            \prfperiod{\ela}{\elb}
        \end{equation}

        \begin{equation}
            \label{eq:combination-084}
            \ela \Rightarrow \elb
        \end{equation}

        \begin{equation}
            \label{eq:combination-085}
            \coprodMapob{\Obja}{\Objb}
        \end{equation}

        \begin{equation}
            \label{eq:combination-086}
            \Objc
        \end{equation}

        \begin{equation}
            \label{eq:combination-087}
            \coprodMapmor{\mora}{\morb}
        \end{equation}

        \begin{equation}
            \label{eq:combination-088}
            \catcoprodpsi_{\mora, \morb}
        \end{equation}

        \begin{equation}
            \label{eq:combination-089}
            \setA=\makeset{\technology{LiPo}, \technology{LCO},\technology{NiH2}}
        \end{equation}

        \begin{equation}
            \label{eq:combination-090}
            \setB=\makeset{\technology{LFP},\technology{LMO},\technology{LiPo}}
        \end{equation}

        \begin{equation}
            \label{eq:combination-091}
            \setC=\makeset{\unit[50]{},\unit[60]{},\unit[70]{},\unit[80]{}}\cartprod \makeset{\text{\standardcurrency}}
        \end{equation}

        \begin{equation}
            \label{eq:combination-092}
            \mapa \colon \setA\to \setC
        \end{equation}

        \begin{equation}
            \label{eq:combination-093}
            \mapb\colon \setB \sto \setC
        \end{equation}

        \begin{equation}
            \label{eq:combination-094}
            \setA \setdisunion \setB=
            \makeset{
                \disunionA{\technology{LiPo}},
                \disunionA{\technology{LCO}},
                \disunionA{\technology{NiH2}},
                \disunionB{\technology{LFP}},
                \disunionB{\technology{LMO}},
                \disunionB{\technology{LiPo}}
            }
        \end{equation}

        \begin{equation}
            \label{eq:combination-095}
            \setA\setdisunion\setB
        \end{equation}

        \begin{equation}
            \label{eq:combination-096}
            \setA
        \end{equation}

        \begin{equation}
            \label{eq:combination-097}
            \setB
        \end{equation}

        \begin{equation}
            \label{eq:combination-098}
            \begin{aligned}
                \inj_\setA\colon \setA & \sto \setA\setdisunion\setB \\
                \setAel                & \mapsto \disunionA{\setAel}
            \end{aligned} \end{equation}

        \begin{equation}
            \label{eq:combination-099}
            \begin{aligned}
                \inj_\setB\colon \setB & \sto \setA\setdisunion\setB \\
                \setBel                & \mapsto \disunionB{\setBel}.
            \end{aligned}    \end{equation}

        \begin{equation}
            \label{eq:combination-100}
            \coprodMapmor{\mapa}{\mapb}\colon \setA \setdisunion \setB \sto \setC
        \end{equation}

        \begin{equation}
            \label{eq:combination-101}
            \inj_\setA\mthen (\mapa \funcsum \mapb)=\mapa
        \end{equation}

        \begin{equation}
            \label{eq:combination-102}
            \inj_\setB\mthen (\mapa \funcsum \mapb)=\mapb
        \end{equation}

        \begin{equation}
            \label{eq:combination-103}
            \styleelements{x}\setin \setA \setdisunion \setB
        \end{equation}

        \begin{equation}
            \label{eq:combination-104}
            \exists \setAel\setin \setA\colon \styleelements{x}=\inj_\setA(\setAel)
        \end{equation}

        \begin{equation}
            \label{eq:combination-105}
            \exists \setBel\setin \setB\colon \styleelements{x}=\inj_\setB(\setBel)
        \end{equation}

        \begin{equation}
            \label{eq:combination-106}
            \coprodMapmor{\mapa}{\mapb}
        \end{equation}

        \begin{equation}
            \label{eq:combination-107}
            \begin{aligned}
                \coprodMapmor{\mapa}{\mapb} \colon  \setA \setdisunion \setB & \to \setC \\
                \styleelements{x}                                            & \mapsto
                \begin{cases}
                    \mapa(\styleelements{x}), & \text{if } \styleelements{x}=\inj_\setA(\setAel),\quad \setAel \setin \setA, \\
                    \mapb(\styleelements{x}), & \text{if } \styleelements{x}=\inj_\setB(\setBel),\quad \setBel \setin \setB.
                \end{cases}
            \end{aligned}
        \end{equation}

        \begin{equation}
            \label{eq:combination-108}
            \setA\setunion \setB
        \end{equation}

        \begin{equation}
            \label{eq:combination-109}
            \mapa
        \end{equation}

        \begin{equation}
            \label{eq:combination-110}
            \mapb
        \end{equation}

        \begin{equation}
            \label{eq:combination-111}
            \technology{LiPo}\setin \setA\setunion \setB
        \end{equation}

        \begin{equation}
            \label{eq:combination-112}
            \mapa(\technology{LiPo})=\unit[50]{CHF}
        \end{equation}

        \begin{equation}
            \label{eq:combination-113}
            \mapb(\technology{LiPo})=\unit[60]{CHF}
        \end{equation}

        \begin{equation}
            \label{eq:combination-114}
            \coprodMapmor{\mapa}{\mapb}
        \end{equation}

        \begin{equation}
            \label{eq:combination-115}
            \setA\setintersection \setB\neq \Emptyset
        \end{equation}

        \begin{equation}
            \label{eq:combination-116}
            \styleelements{x}\setin \setA\setintersection \setB
        \end{equation}

        \begin{equation}
            \label{eq:combination-117}
            \mapa(\styleelements{x})
        \end{equation}

        \begin{equation}
            \label{eq:combination-118}
            \mapb(\styleelements{x})
        \end{equation}

        \begin{equation}
            \label{eq:combination-119}
            \setA,\setB\setin \Obof\Rel
        \end{equation}

        \begin{equation}
            \label{eq:combination-120}
            \relA
        \end{equation}

        \begin{equation}
            \label{eq:combination-121}
            \setA \setdisunion \setB
        \end{equation}

        \begin{equation}
            \label{eq:combination-122}
            \inj_\setA\colon \setA\mto \setA \setdisunion\setB
        \end{equation}

        \begin{equation}
            \label{eq:combination-123}
            \inj_\setB\colon \setB\mto \setA \setdisunion\setB
        \end{equation}

        \begin{equation}
            \label{eq:combination-124}
        \end{equation}

        \begin{equation}
            \label{eq:combination-125}
            m,n\setin \natnumbers
        \end{equation}

        \begin{equation}
            \label{eq:combination-126}
            m\to n
        \end{equation}

        \begin{equation}
            \label{eq:combination-127}
            m | n
        \end{equation}

        \begin{equation}
            \label{eq:combination-128}
            6\to 12
        \end{equation}

        \begin{equation}
            \label{eq:combination-129}
            6 | 12
        \end{equation}

        \begin{equation}
            \label{eq:combination-130}
            \realswithleq
        \end{equation}

        \begin{equation}
            \label{eq:combination-131}
            \elna{1},\elna{2}\setin \reals
        \end{equation}

        \begin{equation}
            \label{eq:combination-132}
            \elna{1}\to \elna{2}
        \end{equation}

        \begin{equation}
            \label{eq:combination-133}
            \elna{1}\leq \elna{2}
        \end{equation}

        \begin{equation}
            \label{eq:combination-134}
            \elna{1}
        \end{equation}

        \begin{equation}
            \label{eq:combination-135}
            \elna{2}
        \end{equation}

        \begin{equation}
            \label{eq:combination-136}
            z\setin \reals
        \end{equation}

        \begin{equation}
            \label{eq:combination-137}
            \elna{1}\leq \elb
        \end{equation}

        \begin{equation}
            \label{eq:combination-138}
            \elna{2}\leq \elb
        \end{equation}

        \begin{equation}
            \label{eq:combination-139}
            \ela\setin \reals
        \end{equation}

        \begin{equation}
            \label{eq:combination-140}
            \elna{1}\leq \ela
        \end{equation}

        \begin{equation}
            \label{eq:combination-141}
            \elna{2}\leq \ela
        \end{equation}

        \begin{equation}
            \label{eq:combination-142}
            \elb\leq \ela
        \end{equation}

        \begin{equation}
            \label{eq:combination-143}
            \max\makeset{\elna{1},\elna{2}}
        \end{equation}

        \begin{equation}
            \label{eq:combination-144}
            \stylesets{S}
        \end{equation}

        \begin{equation}
            \label{eq:combination-145}
            \setA,\setB\setsubseteq \stylesets{S}
        \end{equation}

        \begin{equation}
            \label{eq:combination-146}
            \setA \sto \setB
        \end{equation}

        \begin{equation}
            \label{eq:combination-147}
            \setA\setsubseteq \setB
        \end{equation}

        \begin{equation}
            \label{eq:combination-148}
            \setA\setunion\setB
        \end{equation}

        \begin{equation}
            \label{eq:combination-149}
            \Obja \mto \text{``coproduct of } \Obja \text{ and } \Objb \text{''}  \mfrom \Objb
        \end{equation}

        \begin{equation}
            \label{eq:combination-150}
            \Obja \mto \styleobj{T} \mfrom \Objb
        \end{equation}

        \begin{equation}
            \label{eq:combination-151}
            \setC
        \end{equation}

        \begin{equation}
            \label{eq:combination-152}
            \proj_1\colon \setC \sto \setA
        \end{equation}

        \begin{equation}
            \label{eq:combination-153}
            \proj_2\colon \setC \sto \setB
        \end{equation}

        \begin{equation}
            \label{eq:combination-154}
            \proj_1
        \end{equation}

        \begin{equation}
            \label{eq:combination-155}
            \proj_2
        \end{equation}

        \begin{equation}
            \label{eq:combination-156}
            \makeset{ \sbretzel }
        \end{equation}

        \begin{equation}
            \label{eq:combination-157}
            \makeset{ \schoco }
        \end{equation}

        \begin{equation}
            \label{eq:combination-158}
            \makeset{ \sbretzel, \schoco }
        \end{equation}

        \begin{equation}
            \label{eq:combination-159}
            \makeset{ \sbretzel, \schoco, \sburger }
        \end{equation}

        \begin{equation}
            \label{eq:combination-160}
            \setA
        \end{equation}

        \begin{equation}
            \label{eq:combination-161}
            \setA
        \end{equation}

        \begin{equation}
            \label{eq:combination-162}
            \setA
        \end{equation}

        \begin{equation}
            \label{eq:combination-163}
            \setA
        \end{equation}

        \begin{equation}
            \label{eq:combination-164}
            \setA
        \end{equation}

        \begin{equation}
            \label{eq:combination-165}
            \setA
        \end{equation}

        \begin{equation}
            \label{eq:combination-166}
            \setA
        \end{equation}

        \begin{equation}
            \label{eq:combination-167}
            \setA
        \end{equation}

        \begin{equation}
            \label{eq:combination-168}
            \setA
        \end{equation}

        \begin{equation}
            \label{eq:combination-169}
            \setA
        \end{equation}

        \begin{equation}
            \label{eq:combination-170}
            \setA
        \end{equation}

        \begin{equation}
            \label{eq:combination-171}
            \setA
        \end{equation}

        \begin{equation}
            \label{eq:combination-172}
            \setA
        \end{equation}

        \begin{equation}
            \label{eq:combination-173}
            \setA
        \end{equation}

        \begin{equation}
            \label{eq:combination-174}
            \setA
        \end{equation}

        \begin{equation}
            \label{eq:combination-175}
            \setA
        \end{equation}

        \begin{equation}
            \label{eq:combination-176}
            \setA
        \end{equation}

        \begin{equation}
            \label{eq:combination-177}
            \setA
        \end{equation}

        \begin{equation}
            \label{eq:combination-178}
            \setA
        \end{equation}

        \begin{equation}
            \label{eq:combination-179}
            \setA
        \end{equation}

        \begin{equation}
            \label{eq:combination-180}
            \setA
        \end{equation}

        \begin{equation}
            \label{eq:combination-181}
            \setA
        \end{equation}

        \begin{equation}
            \label{eq:combination-182}
            \setA
        \end{equation}

        \begin{equation}
            \label{eq:combination-183}
            \setA
        \end{equation}

        \begin{equation}
            \label{eq:combination-184}
            \setA
        \end{equation}

        \begin{equation}
            \label{eq:combination-185}
            \setA
        \end{equation}

        \begin{equation}
            \label{eq:combination-186}
            \setA
        \end{equation}

        \begin{equation}
            \label{eq:combination-187}
            \setA
        \end{equation}

        \begin{equation}
            \label{eq:combination-188}
            \setA
        \end{equation}

        \begin{equation}
            \label{eq:combination-189}
            \setA
        \end{equation}

        \subsection{Section: Products}

        \begin{equation}
            \label{eq:prod-slides-1}
            \setA = \makeset{1, 2, 3, 4}
        \end{equation}
        %
        \begin{equation}
            \label{eq:prod-slides-2}
            \setB = \makeset{ \ast, \dagger }
        \end{equation}
        %
        \begin{equation}
            \label{eq:prod-slides-3}
            \Snacks=\makeset{\sapple,\sbanana,\scarrot}
        \end{equation}
        %
        \begin{equation}
            \label{eq:prod-slides-4}
            \Drinks=\makeset{\swater,\stea}
        \end{equation}
        %
        \begin{equation}
            \label{eq:prod-slides-5}
            \Participants
        \end{equation}
        %
        \begin{equation}
            \label{eq:prod-slides-6}
            \eats \colon \Participants \sto \Snacks
        \end{equation}
        %
        \begin{equation}
            \label{eq:prod-slides-7}
            \drinks \colon  \Participants \sto \Drinks
        \end{equation}
        %
        \begin{equation}
            \label{eq:prod-slides-8}
            \meal \colon \Participants\sto \Snacks \cartprod \Drinks
        \end{equation}
        %
        \begin{equation}
            \label{eq:prod-slides-9}
            \Snacks \cartprod \Drinks
        \end{equation}
        %
        \begin{equation}
            \label{eq:prod-slides-10}
            \eats
        \end{equation}
        %
        \begin{equation}
            \label{eq:prod-slides-11}
            \drinks
        \end{equation}
        %
        \begin{equation}
            \label{eq:prod-slides-12}
            \meal
        \end{equation}
        %
        \begin{equation}
            \label{eq:prod-slides-13}
            \setA
        \end{equation}
        %
        \begin{equation}
            \label{eq:prod-slides-14}
            \subA_1, \subA_2 \setsubseteq \setA
        \end{equation}
        %
        \begin{equation}
            \label{eq:prod-slides-15}
            \subA_1 \to \subA_2 \text{ iff } \subA_1 \setsubseteq \subA_2
        \end{equation}
        %
        \begin{equation}
            \label{eq:prod-slides-16}
            \setA = \makeset{ 1, 2, 3, 4 }
        \end{equation}
        %
        \begin{equation}
            \label{eq:prod-slides-17}
            \subA_1 = \makeset{ 1, 2, 3 }
        \end{equation}
        %
        \begin{equation}
            \label{eq:prod-slides-18}
            \subA_2 = \makeset{2, 3, 4 }
        \end{equation}
        %
        \begin{equation}
            \label{eq:prod-slides-19}
            \subA_1 \setintersection \subA_2 = \makeset{2, 3 }
        \end{equation}
        %
        \begin{equation}
            \label{eq:prod-slides-20}
            \makeset{ 1, 2, 3 } \setsupseteq \makeset{2, 3 } \setsubseteq \makeset{2, 3, 4 }
        \end{equation}
        %
        \begin{equation}
            \label{eq:coprod-slides-1}
            \setA=\makeset{\star, \diamond}
        \end{equation}
        %
        \begin{equation}
            \label{eq:coprod-slides-2}
            \setB=\makeset{\ast, \dagger}
        \end{equation}
        %

    \end{forslides}

}