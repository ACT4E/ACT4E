% !TEX root = chapter-standalone.tex

\section{More examples}

\subsection{Product and coproduct for power set}
\Cref{exa:intersection-as-prod} and \cref{ex:subset_coprod} are specific instances of the power set \SY{lattice}.

\begin{ctdefinition}[Power set as lattice]
    \label{def:power-set-as-lattice}
    Given a set~$\setC$, its power set~$\powerset \setC$ (the set of all subsets) is a \SY{lattice} where, given~$\setA,\setB\setin \powerset \setC$:
    \begin{itemize}
        \item Order is given by inclusion:
              \begin{equation}
                  \setA\posleq \setB \definedas \setA\setsubseteq \setB;
              \end{equation}
        \item The \SY{join} is given by the \SY{union of sets}:
              \begin{equation}
                  \setA\join \setB \definedas \setA\setunion \setB;
              \end{equation}
        \item The \SY{meet} is given by the \SY{intersection of sets}:
              \begin{equation}
                  \setA\meet \setB \definedas \setA\setintersection \setB;
              \end{equation}
        \item The \SY{top element} is the set~$\setC$ itself:
              \begin{equation}
                  \postop = \setC;
              \end{equation}
        \item The \SY{bottom element} is the empty set:
              \begin{equation}
                  \posbot = \Emptyset.
              \end{equation}
    \end{itemize}
\end{ctdefinition}

\begin{marginfigure}
    \centering
    \includesag{060_powerset_coprod}
    \caption{}
    \label{fig:prod_coprod_power}
\end{marginfigure}

The Hasse diagram reported in \cref{fig:prod_coprod_power} illustrates the structure of the power set \SY{lattice} for three sets~$\setA,\setB,\setC\setin \powerset \stylesets{S}$.

As previously discovered, the \SY{lattice} can be seen as a category.
In this category, the \SY{meet}~$\meet$ is the product, and the \SY{join}~$\join$ is the coproduct.

Specifically, for~$\setA,\setB\setsubseteq\stylesets{S}$ the product corresponds to~$\setA\setintersection \setB$, and the projection maps~$\projA\colon \setA\setintersection \setB \sto \setA$ and~$\projB\colon \setA\setintersection \setB \sto \setB$ simply state the inclusions of~$\setA\setintersection \setB$ in~\setA and~\setB.

Similarly, the coproduct corresponds to~$\setA\setunion \setB$, and the injection maps~$\injA\colon \setA \sto \setA\setunion \setB$ and~$\injB\colon \setB \sto \setA\setunion \setB$ simply state the inclusion of~$\setA,\setB$ in~$\setA\setunion \setB$.
\vfill\clearpage
\subsection{Product and coproduct for logical sequents}

\begin{ctdefinition}[Propositions as lattice]
    \label{def:prop-as-lattice}
    Given a set of propositions~$\posAset$ (equivalence classes of propositions), we define the following.
    \begin{itemize}
        \item Order is given by implication:
              \begin{equation}
                  \prfdoubleperiod{
                      \posAel\posAleq \posBel
                  }{
                      \posAel \Imp \posBel
                  }
              \end{equation}
              Clearly, since~$\posAel \posAleq \posAel$,
              \begin{equation}
                  \prfdoublecomma{
                      \posAel \Leftrightarrow \posBel
                  }{
                      \posAel =\posBel
                  }
              \end{equation}
              and
              \begin{equation}
                  \prfdoublecomma{
                      \posAel \Imp \posBel
                  }{
                      \posBel \Imp \posCel
                  }{
                      \posAel \Imp \posCel
                  }
              \end{equation}
              this defines a \SY{poset} over~$\posAset$.
        \item The \SY{join} is given by the ``or'' connective:
              \begin{equation}
                  \posAel \join \posBel \definedas \posAel \boolor \posBel;
              \end{equation}
        \item The \SY{meet} is given by the ``and'' connective:
              \begin{equation}
                  \posAel \meet \posBel \definedas \posAel \booland \posBel;
              \end{equation}
        \item The \SY{top element} is~$\true$,
              because, for all~$\posAel\setin \posAset$, $\posAel \Imp \true$.
              %   \begin{equation}
              %       \prfperiod{
              %           \true
              %       }{
              %           \posAel \Imp \posAel
              %       }
              %   \end{equation}
        \item The \SY{bottom element} is~$\false$, because for all~$\posAel\setin \posAset$, $\false \Imp \posAel$.
              %   \begin{equation}
              %       \posbot = \false.
              %   \end{equation}
    \end{itemize}
\end{ctdefinition}
\begin{marginfigure}
    \centering
    \includesag{060_propositions_coprod}
    \caption{}
    \label{fig:prod_coprod_prop}
\end{marginfigure}

The diagram reported in \cref{fig:prod_coprod_prop} illustrates the structure of the proposition \SY{lattice} for three elements~$\posAel,\posBel,\posCel\setin \posAset$.

As previously discovered, the \SY{lattice} can be seen as a category.
In this category, the \SY{meet}~$\meet$ is the product, and the \SY{join}~$\join$ is the coproduct.

Specifically, for~$\posAel,\posBel\setin \posAset$ the product corresponds to~$\posAel\booland \posBel$, and the projection maps~$\projA\colon \posAel\booland \posBel\to \posAel$ and~$\projB\colon \posAel\booland \posBel\to \posBel$ simply state the implications~$\posAel \booland \posBel \Imp \posAel$ and~$\posAel \booland \posBel \Imp \posBel$.

Similarly, the coproduct corresponds to~$\posAel\boolor \posBel$, and the injection maps~$\injA\colon \posAel\to \posAel\boolor \posBel$ and~$\injB\colon \posBel\to \posAel\boolor \posBel$ simply state the implications~$\posAel \Imp \posAel \boolor \posBel$ and~$\posBel \Imp \posAel \boolor \posBel$.
