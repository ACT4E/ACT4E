% !TEX root = chapter-standalone.tex


\section{Limits and colimits}


\todotext{add a couple of intro words here}

\subsection{Products as terminal objects}

In order to work up to the definition of limits and colimits, we'll first explain how ``the'' categorical product $\prodMapob \Obja \Objb$ of two objects in a category $\CatC$ can be viewed as a terminal object in some other, slightly more complicated category. 

Namely, consider the following new category, which we'll temporarily call $\CatC_{\Obja, \Objb}$, whose objects are diagrams in $\CatC$ of the form
\equationsag{categorical_product_diagram_generic}{fig:prod_diagram_generic}

where $\Obja$ and $\Objb$ are fixed, but $\Objc$ ranges over all objects in $\CatC$, and $\projA$ and $\projB$ range over all morphisms in $\CatC$ of the appropriate respective type. To define morphisms, suppose we are given two objects in $\CatC_{\Obja, \Objb}$, 
\equationsag{categorical_product_diagram_generic}{fig:prod_diagram_generic_2}
and
\equationsag{categorical_product_diagram_generic_bis}{fig:prod_diagram_generic_bis}
We declare that a morphism in $\CatC_{\Obja, \Objb}$ from the one diagram to the other is a morphism $\stylemorph{\phi} \colon \Objc \mto \Objcprime$ in $\CatC$ such that the following diagram is commutative: 
\equationsag{categorical_product_diagram_morphism}{fig:prod_diagram_morphism}
Composition in $\CatC_{\Obja, \Objb}$ is induced from composition in $\CatC$ in the obvious way, and similarly identity morphisms in $\CatC_{\Obja, \Objb}$ are induced from identity morphisms in $\CatC$. It is easy then to check that $\CatC_{\Obja, \Objb}$ does indeed define a category. 

And now we claim that specifying a categorical product of $\Obja$ and $\Objb$ in $\CatC$ is equivalent to specifying a terminal object in $\CatC_{\Obja, \Objb}$. To help see this, we'll borrow notation from \cref{def:categorical-product}. A terminal object in $\CatC_{\Obja, \Objb}$ is a diagram of the form
\equationsag{categorical_product_diagram_generic}{fig:prod_diagram_generic}
such that for any other diagram of this form, 
\equationsag{categorical_product_diagram_test}{fig:prod_diagram_test}
there exists a unique morphism in $\CatC_{\Obja, \Objb}$ from the latter to the former. This though is equivalent to saying that there exists a unique morphism $\stylemorph{\phi} \colon \styleobj{O} \mto \Objc$ in $\CatC$ such the following diagram commutes:
\equationsag{categorical_product_diagram_universal_prop}{fig:categorical_product_diagram_universal_prop}


And this was precisely our definition of categorical product. Previously we used the notation $\stylemorph{\phi}_{\mora, \morb}$ for $\stylemorph{\phi}$ in order to emphasize that the required existence and uniqueness of this morphism depends not only on it's source (as a variable), but also on the choice of $\mora$ and $\morb$. In the present discussion, this is accounted for by considering the whole diagram \cref{fig:prod_diagram_test} as the source of $\stylemorph{\phi}$. 

\begin{remark}
An analogous story (and construction) may be given for coproducts, in a way such that a coproduct of $\Obja$ and $\Objc$ in $\CatC$ may be viewed as an \emph{initial} object in an appropriately defined auxiliary category. 
\end{remark}

\subsection{Cones and cocones}

We will now reframe and generalize the discussion in the previous subsection, leading to the definition of limits (and colimits). 



\subsection{(Co)cones as natural transformations} 

    Let~$\CatC, \CatD$ be categories, and~$\funa, \funb\colon\CatC\fto\CatD$ be functors. 
    
    Suppose~$\funa$ is constant, which means that~$\funa$ maps all objects~$\Obja_i$ of~$\CatC$ to a single object~$\Objd \setin \CatD$.
    A natural transformation from~$\funa(\Obja)$ to~$\funb(\Objb)$ forms what we call the cone over~$\funb$, illustrated in figure \cref{fig:cone-under-F}.
    
    Analogously, if we consider~$\funb$ to be a constant functor, we talk about the cone under~$\funa$ (or cocone), as seen in figure \cref{fig:cone-over-G}.



\begin{marginfigure}
    \centering
    \includesag{cone-over-G}
    \caption{Cone over~$\funb$.
    }
    \label{fig:cone-over-G}
\end{marginfigure}

\begin{marginfigure}
    \centering
    \includesag{cone-over-F}
    \label{fig:cone-under-F}
    \caption{Cone under~$\funa$.}
\end{marginfigure}


