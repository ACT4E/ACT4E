% !TEX root = chapter-standalone.tex


\section{Limits and colimits}


    Let~$\CatC, \CatD$ be categories, and~$\funa, \funb\colon\CatC\fto\CatD$ be functors. 
    
    Suppose~$\funa$ is constant, which means that~$\funa$ maps all objects~$\Obja_i$ of~$\CatC$ to a single object~$\Objd \setin \CatD$.
    A natural transformation from~$\funa(\Obja)$ to~$\funb(\Objb)$ forms what we call the cone over~$\funb$, illustrated in figure \cref{fig:cone-under-F}.
    
    Analogously, if we consider~$\funb$ to be a constant functor, we talk about the cone under~$\funa$ (or cocone), as seen in figure \cref{fig:cone-over-G}.



\begin{marginfigure}
    \centering
    \includesag{cone-over-G}
    \caption{Cone over~$\funb$.
    }
    \label{fig:cone-over-G}
\end{marginfigure}

\begin{marginfigure}
    \centering
    \includesag{cone-over-F}
    \label{fig:cone-under-F}
    \caption{Cone under~$\funa$.}
\end{marginfigure}


