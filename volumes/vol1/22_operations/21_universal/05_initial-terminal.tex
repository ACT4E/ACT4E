% !TEX root = chapter-standalone.tex

\section{Initial and terminal objects}
\label{sec:initional-terminal}


\todotext{insert an applied, simple, motivating example here?? what is a useful application of terminal objects at all??}


\begin{ctdefinition}
Let $\CatC$ be a category. A \emph{terminal object} in $\CatC$ is: 

\constit
\begin{enumerate}
\item an object $\termobja \setin \ObC$; 
\end{enumerate}
\condit
\begin{enumerate}
\item for any object $\Objb \setin \ObC$, there exists a unique morphism of the type $\Objb \to \termobja$. 
\end{enumerate}
\end{ctdefinition}


\begin{example}\label{exa:terminal-object-in-Set}
In the category $\CatC = \Set$, the set $\termobja = \makeset{ \singletonel}$ containing the single element ``$\singletonel$'' is an example of a terminal object. 

Indeed, given any other set $\Objb = \setB$, there always exists a function $\setB \to \makeset{ \singletonel}$, namely the one defined by sending every element of $\setB$ to the single element ``$\singletonel$'' (or, if $\setB$ is empty, we also know, from the formal definition of functions, that there is a function $\emptyset \to \makeset{ \singletonel}$). Furthermore, for every possible choice of $\setB$, the functions $\setB \to \makeset{ \singletonel}$ just described are the only possible choice. In the special case $\setB = \emptyset$, this follows from the definitions, and when $\setB$ is non-empty, this is evidently true because in order to define a function $\setB \to \makeset{ \singletonel}$ we have no room for choices: every element of $\setB$ has to be sent somewhere, and the single only possible choice is always ``$\singletonel$''. Thus, in summary, for any set $\setB$, a function $\setB \to \makeset{ \singletonel}$ always \emph{exists}, and it is also the \emph{unique} function of this type (having this source and target). 
\end{example}

\begin{remark}
The category $\CatC = \Rel$ has the same objects as $\Set$, but different (more) morphisms. In $\Rel$, the set $\makeset{ \singletonel}$ is not a terminal object! Consider for example the set $\setB = \makeset{1,2,3}$. While there is only one unique function $\setB \to \makeset{ \singletonel}$, there are eight relations $\setB \to \makeset{ \singletonel}$ (can you see what they are?). Thus, the uniqueness part of the condition in the definition of terminal object is not fulfilled here. This comparison between $\Set$ and $\Rel$ illustrates the point that ``being a terminal object'' is about a relationship between a specific object and the entire category we are viewing it as a part of. If we change the ambient category we are taking as our ``context", then the status of an object as being terminal or not may also change. 
\end{remark}

\begin{example}\label{exa:term-obj-Rel}
In the category $\Rel$, the empty set $\emptyset$ is a terminal object. 
\end{example}

\begin{example}\label{exa:term-obj-Mon}
The following monoid $\monoidA$ is a terminal object in the category $\Mon$ of monoids. As it's underlying set, we take $\monoidAset = \makeset{ \singletonel}$. It's composition operation is the unique function $\makeset{ \singletonel} \cartprod \makeset{ \singletonel} \to \makeset{ \singletonel}$, and its identity element is ``$\singletonel$''. 
\end{example}

\begin{example}\label{exa:term-obj-Pos}
Consider the poset $\posA$ whose underlying set is $\posAset = \makeset{ \singletonel}$ and where the order relation is (necessarily) ``$\singletonel \leq \singletonel$''. This is a terminal object in the category $\Pos$ of posets and monotone maps. 
\end{example}

\begin{example}\label{exa:term-obj-Cat}
In the category $\Category$ of locally small categories and functors, consider the category $\mathbf{1}$ which has only one object, which we call ``$\singletonel$", and the only morphism is the identity morphism $\catid_\singletonel$. The category $\mathbf{1}$ is a terminal object in $\Category$. 
\end{example}

\begin{example}\label{exa:term-obj-DP}
In the category $\DP$, the empty poset is a terminal object.
\end{example}

\begin{remark}\label{rem:terminal-object-renaming}
In the category $\CatC = \Set$, the set $\termobjb = \makeset{ \star}$ is also a terminal object. In fact, by the same reasoning that was giving in \cref{exa:terminal-object-in-Set}, we can see that \emph{any} one-element set is a terminal object in $\Set$. Furthermore, one can also reason that, on the other hand, any terminal object in $\Set$ must necessarily be a one-element set: the empty set does not come into question, because there are no functions into the empty set (existence breaks down), and any set with two or more elements does also not work, because when defining a function into it, we then have choices (uniqueness breaks down). Another way to see that terminal objects in $\Set$ have to be one-element sets is via the following proposition. 
\end{remark}

\begin{proposition}\label{prop:terminal-objs-isom}
Let $\termobja$ and $\termobjb$ be two terminal objects in a category $\CatC$.  Then $\termobja$ and $\termobjb$ are isomorphic.  
\end{proposition}

\begin{proof}
Because $\termobja$ is terminal, there exists a morphism $\mora : \termobja \to \termobjb$. Similarly, because $\termobja$ is terminal, there exists a morphism $\morb : \termobjb \to \termobja$. To see that $\mora \mthen \morb = \catid_\termobja$ must be the case, we invoke the uniqueness of morphisms $\mora : \termobja \to \termobja$, which is given by the fact that $\termobja$ is terminal. Indeed, $\mora \mthen \morb$ and $\catid_\termobja$ are two morphisms from $\termobja$ to $\termobja$, and because only one morphism of this type may exist, these two morphisms must be equal. Similarly, because $\morb \mthen \mora$ and $\catid_\termobjb$ are both morphisms $\termobjb \to \termobjb$, by virtue of $\termobjb$ being terminal, these two morphisms must be equal. Hence $f$ and $g$ are mutually inverse; they are isomorphisms. 
\end{proof}

\begin{remark}
The terminal objects described in \cref{exa:term-obj-Mon}, \cref{exa:term-obj-Pos}, and \cref{exa:term-obj-Cat} all have a similar flavor to the one-element set as a terminal object in $\Set$. This is related to the fact that all of the ambient categories in these examples have objects and morphisms that are based on underlying sets and functions. A remark analogous to \cref{rem:terminal-object-renaming} applies to all of them: we would obtain a different (but isomorphic) terminal object by relabeling ``$\singletonel$'' with ``$\star$'', for example. 
\end{remark}

\todotext{insert more examples}

\todotext{make remark somewhere about explanation point notation}

\begin{ctdefinition}
Let $\CatC$ be a category. An \emph{initial object} in $\CatC$ is: 

\constit
\begin{enumerate}
\item an object $\initobja \setin \ObC$; 
\end{enumerate}
\condit
\begin{enumerate}
\item for any object $\Objb \setin \ObC$, there exists a unique morphism of the type $\initobja \to \Objb$. 
\end{enumerate}
\end{ctdefinition}

\begin{example}
In the category $\CatC = \Set$, the set $\emptyset$ is an initial object. Indeed, via the formal definition of functions as special kinds of relations, there is always a unique function from the empty set to any other set. 
\end{example}

\begin{example}\label{exa:term-obj-Rel}
The empty set $\emptyset$ is also an initial object in the category $\Rel$. This is an example where a given object is both terminal and initial in the same category. 
\end{example}

\todotext{insert more examples}


\begin{gradedexercise}
Let $\initobja$ and $\initobjb$ be two initial objects in a category $\CatC$.  Prove that $\initobja$ and $\initobjb$ are isomorphic.  
\end{gradedexercise}


