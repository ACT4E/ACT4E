% !TEX root = chapter-standalone.tex

\section{Lattice interaction with composition}

\todotextjira{627}{\alphubel: @Andrea: description and intro}
\todojira{628}{\alphubel: reference to previous parts}
\todojira{629}{\alphubel: better sectioning}

\subsection{Monoidal composition}

\begin{lemma}
    \label{lem:times_vee}
    Consider~$\adpa,\adpb\setin \HomSet{\DP}{\F{\posgenA}}{\R{\posgenB}}$ and~$\adpc\setin \HomSet{\DP}{\F{\posgenC}}{\R{\posgenD}}$.
    We have
    \begin{equation}
        (\adpa \dpjoin \adpb)
        \mtimescat \adpc=(\adpa \mtimescat \adpc) \dpjoin (\adpb\mtimescat \adpc).
    \end{equation}
\end{lemma}

\begin{proof}
    We have:
    \begin{equation}
        \begin{aligned}
            ~ & ((\adpa \dpjoin \adpb)\mtimescat \adpc)(\tup{\FposgenAel,\F{\posgenCel}}\Fop, \tup{\RposgenBel,\R{\posgenDel}}) \\
              & = (\adpa \dpjoin \adpb)(\FposgenAel\Fop,\RposgenBel)\booland \adpc(\F{\posgenCel^*},\R{\posgenDel}) \\
              & =(\adpa(\FposgenAelop,\RposgenBel) \boolor \adpb(\FposgenAelop,\RposgenBel))\booland \adpc(\F{\posgenCel^*},\R{\posgenDel}) \\
              & =(\adpa(\FposgenAelop,\RposgenBel) \booland \adpc(\F{\posgenCel^*},\R{\posgenDel})) \boolor (\adpb(\FposgenAelop,\RposgenBel)\booland \adpc(\F{\posgenCel^*},\R{\posgenDel})) \\
              & =((\adpa \mtimescat \adpc) \dpjoin (\adpb\mtimescat \adpc))(\tup{\FposgenAel,\F{\posgenCel}}\Fop, \tup{\RposgenBel,\R{\posgenDel}}).
        \end{aligned}
    \end{equation}
\end{proof}

\begin{lemma}
    \label{lem:times_wedge}
    Consider~$\adpa,\adpb\setin \HomSet{\DP}{\F{\posgenA}}{\R{\posgenB}}$ and~$\adpc\setin \HomSet{\DP}{\F{\posgenC}}{\R{\posgenD}}$.
    We have
    \begin{equation}
        (\adpa \dpmeet \adpb)
        \mtimescat \adpc=(\adpa \mtimescat \adpc) \dpmeet (\adpb\mtimescat \adpc).
    \end{equation}
\end{lemma}
\begin{proof}
    We have:
    \begin{equation}
        \begin{aligned}
            ~ & ((\adpa \dpmeet \adpb)\mtimescat \adpc)(\tup{\FposgenAel,\F{\posgenCel}}\Fop, \tup{\RposgenBel,\R{\posgenDel}}) \\
            = & (\adpa \dpmeet \adpb)(\FposgenAelop,\RposgenBel)\booland \adpc(\F{\posgenCel^*},\R{\posgenDel}) \\
            = & (\adpa(\FposgenAelop,\RposgenBel) \booland \adpb(\FposgenAelop,\RposgenBel))\booland \adpc(\F{\posgenCel^*},\R{\posgenDel}) \\
            = & (\adpa(\FposgenAelop,\RposgenBel) \booland \adpc(\F{\posgenCel^*},\R{\posgenDel})) \booland (\adpb(\FposgenAelop,\RposgenBel)\booland \adpc(\F{\posgenCel^*},\R{\posgenDel})) \\
            = & ((\adpa \mtimescat \adpc) \dpmeet (\adpb\mtimescat \adpc))(\tup{\FposgenAel,\F{\posgenCel}}\Fop, \tup{\RposgenBel,\R{\posgenDel}}).
        \end{aligned}
    \end{equation}
\end{proof}

% \begin{lemma}
%     \label{lem:vee_vee}
%     Consider~$\adpa,\adpb,\adpc\setin \HomSet{\DP}{\F{\posgenA}}{\R{\posgenB}}$.
%     We have
%     \begin{equation}
%         (\adpa \dpjoin \adpb)
%         \dpjoin \adpc=(\adpa \dpjoin \adpc) \dpjoin (\adpb\dpjoin \adpc).
%     \end{equation}
% \end{lemma}
% \begin{proof}
%     We have:
%     \begin{equation}
%         \begin{aligned}
%             ~ & ((\adpa \dpjoin \adpb)\dpjoin \adpc)(\FposgenAelop,\RposgenBel) \\
%               & =  (\adpa \dpjoin \adpb)(\FposgenAelop,\RposgenBel)\boolor \adpc(\FposgenAelop,\RposgenBel) \\
%               & =(\adpa(\FposgenAelop,\RposgenBel) \boolor \adpb(\FposgenAelop,\RposgenBel))\boolor \adpc(\FposgenAelop,\RposgenBel) \\
%               & =(\adpa(\FposgenAelop,\RposgenBel) \boolor  \adpc(\FposgenAelop,\RposgenBel)) \boolor (\adpb(\FposgenAelop,\RposgenBel)\boolor \adpc(\FposgenAelop,\RposgenBel)) \\
%               & =((\adpa \dpjoin \adpc) \dpjoin (\adpb\dpjoin \adpc))(\FposgenAelop,\RposgenBel).
%         \end{aligned}
%     \end{equation}
% \end{proof}

\subsection{Trace}

\begin{lemma}
    \label{lem:trace_vee}
    Consider~$\adpa,\adpb\setin \HomSet{\DP}{\F{\posgenC}\Ptimes \F{\posgenA}}{\R{\posgenC} \Ptimes\R{\posgenB}}$.
    We have
    \begin{equation}
        \Tr_{\F{\posgenA},\F{\posgenB}}^{\posC}(\adpa \dpjoin \adpb)=\Tr_{\F{\posgenA},\F{\posgenB}}^{\posC}(\adpa)\dpjoin \Tr_{\F{\posgenA},\F{\posgenB}}^{\posC}(\adpb)
    \end{equation}
\end{lemma}
\begin{proof}
    We have:
    \begin{equation}
        \begin{aligned}
            ~ & \Tr_{\F{\posgenA},\F{\posgenB}}^{\posC}(\adpa \dpjoin \adpb)(\FposgenAelop,\RposgenBel) \\
              & = \bigvee_{\posCel\setin \posC} (\adpa \dpjoin \adpb)(\tup{\F{\posgenCel},\FposgenAel}\Fop, \tup{\R{\posgenCel},\RposgenBel}) \\
              & =\bigvee_{\posCel\setin \posC} \adpa(\tup{\F{\posgenCel},\FposgenAel}\Fop, \tup{\R{\posgenCel},\RposgenBel})\boolor \adpb(\tup{\F{\posgenCel},\FposgenAel}\Fop, \tup{\R{\posgenCel},\RposgenBel}) \\
              & =\bigvee_{\posCel\setin \posC} \adpa(\tup{\F{\posgenCel},\FposgenAel}\Fop, \tup{\R{\posgenCel},\RposgenBel})\boolor \bigvee_{\posCel\setin \posC} \adpb(\tup{\F{\posgenCel},\FposgenAel}\Fop, \tup{\R{\posgenCel},\RposgenBel}) \\
              & =\Tr_{\F{\posgenA},\F{\posgenB}}^{\posC}(\adpa)(\FposgenAelop,\RposgenBel)\boolor \Tr_{\F{\posgenA},\F{\posgenB}}^{\posC}(\adpb)(\FposgenAelop,\RposgenBel).
        \end{aligned}
    \end{equation}
\end{proof}

\begin{remark}
    Consider~$\adpa,\adpb\setin \HomSet{\DP}{\F{\posgenC}\Ptimes \F{\posgenA}}{\R{\posgenC} \Ptimes\R{\posgenB}}$.
    In general, we have
    \begin{equation}
        \Tr_{\F{\posgenA},\F{\posgenB}}^{\posC}(\adpa \dpmeet \adpb)\neq \Tr_{\F{\posgenA},\F{\posgenB}}^{\posC}(\adpa)\dpmeet \Tr_{\F{\posgenA},\F{\posgenB}}^{\posC}(\adpb)
    \end{equation}
    Specifically, on one hand we have
    \begin{equation}
        \begin{aligned}
            \Tr_{\F{\posgenA},\F{\posgenB}}^{\posC}(\adpa \dpmeet \adpb)(\FposgenAelop,\RposgenBel) & = \bigvee_{\posCel\setin \posC} (\adpa \dpmeet \adpb)(\tup{\F{\posgenCel},\FposgenAel}\Fop, \tup{\R{\posgenCel},\RposgenBel}) \\
                                                                                                    & =\bigvee_{\posCel\setin \posC} \adpa(\tup{\F{\posgenCel},\FposgenAel}\Fop, \tup{\R{\posgenCel},\RposgenBel})\booland \adpb(\tup{\F{\posgenCel},\FposgenAel}\Fop, \tup{\R{\posgenCel},\RposgenBel}),
        \end{aligned}
    \end{equation}
    and on the other hand we have
    \begin{equation}
        \begin{aligned}
             & \Tr_{\F{\posgenA},\F{\posgenB}}^{\posC}(\adpa)(\FposgenAelop,\RposgenBel)\dpmeet \Tr_{\F{\posgenA},\F{\posgenB}}^{\posC}(\adpb)(\FposgenAelop,\RposgenBel) \\
             & =\bigvee_{\posCel\setin \posC} \adpa(\tup{\F{\posgenCel},\FposgenAel}\Fop, \tup{\R{\posgenCel},\RposgenBel})\booland \bigvee_{\posCel\setin \posC} \adpb(\tup{\F{\posgenCel},\FposgenAel}\Fop, \tup{\R{\posgenCel},\RposgenBel}).
        \end{aligned}
    \end{equation}
    Clearly, these two equations are different, as soon as we have incomparable~$\posgenCel,\posgenCel'\setin \posgenC$, and~$\posgenAel\setin \posgenA$,~$\posgenBel\setin \posgenB$ such that
    \begin{equation}
        \begin{aligned}
            \adpa(\tupp{\F{\posgenCel},\FposgenAel}\Fop, \tupp{\R{\posgenCel},\RposgenBel})  & =\true, \\
            \adpb(\tupp{\F{\posgenCel},\FposgenAel}\Fop, \tupp{\R{\posgenCel},\RposgenBel})  & =\true, \\
            \adpb(\tupp{\F{\posgenCel'},\FposgenAel}\Fop, \tupp{\R{\posgenCel},\RposgenBel}) & =\false.
        \end{aligned}
    \end{equation}
\end{remark}

\todo{If needed, let's add a specific case}
