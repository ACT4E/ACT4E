% !TEX root = chapter-standalone.tex

\section{Interaction with composition}
\todotext{\alphubel: Better referencing to previous parts.}
In the previous part, we introduced the concept of order in~\DP, and proved that \SY{hom-sets} of~\DP form a \SY{bounded lattice}.
In this section, we show that monoidal product and trace (\cref{def:dp-trace}) of \SY{design problems} are order-preserving operations.

\todographicsjira{432}{@Gioele: (polish) These relations are very well explained with diagrams.}

\begin{lemma}
    \label{lem:tens_pres_order}
    Given~$\adpa,\adpb \setin \HomSet{\DP}{\F{\posgenA}}{\R{\posgenB}}$ and~$\adpc,\adpd \setin \HomSet{\DP}{\F{\posgenC}}{\R{\posgenD}}$, we have:

    \begin{equation}
        \prfperiod{
            \adpa\posDPleq \adpb
        }{
            \adpc\posDPleq \adpd
        }{
            (\adpa\mtimescat \adpc )\posDPleq (\adpb\mtimescat \adpd)
        }
    \end{equation}
    In other words, the monoidal product preserves order on~\DP.
\end{lemma}

\begin{proof}
    We have
    \begin{equation}
        \label{eq:tens_pres_ord}
        \begin{aligned}
            ~ & \pars{ \adpa\mtimescat \adpc} (\tup{\FposgenAel,\F{\posgenCel}}\Fop, \tup{\RposgenBel,\R{\posgenDel}}) \\
              & = \adpa(\FposgenAelop,\RposgenBel)\booland \adpc(\F{\posgenCel^*},\R{\posgenDel}) \\
              & \posDPleq \adpb(\FposgenAelop,\RposgenBel)\booland \adpd(\F{\posgenCel^*},\R{\posgenDel}) \\
              & =\pars{ \adpb\mtimescat \adpc} (\tup{\FposgenAel,\F{\posgenCel}}\Fop, \tup{\RposgenBel,\R{\posgenDel}}).
        \end{aligned}
    \end{equation}
    Therefore,~$\mtimescat$ is order-preserving on \DP.
\end{proof}

\begin{lemma}
    \label{lem:trace_pres_order}
    Given~$\adpa,\adpb\setin \HomSet{\DP}{\F{\posgenC}\Ptimes \F{\posgenA}}{\R{\posgenC}\Ptimes \R{\posgenB}}$, we have:

    \begin{equation}
        \prfperiod{\adpa\posDPleq \adpb}{\Tr_{\F{\posgenA},\R{\posgenB}}^\posC(\adpa)\posDPleq \Tr_{\F{\posgenA},\R{\posgenB}}^\posC(\adpb)}
    \end{equation}
    In other words, trace preserves order on~\DP.
\end{lemma}

\begin{proof}
    We have
    \begin{equation}
        \label{eq:trace_pres_ord}
        \begin{aligned}
            ~ & \Tr_{\F{\posgenA},\R{\posgenB}}^\posC(\adpa)(\FposgenAelop,\RposgenBel) \\
              & = \bigvee_{\posCel\setin \posC}\adpa(\tup{\F{\posgenCel},\FposgenAel}\Fop, \tup{\R{\posgenCel},\RposgenBel}) \\
              & \posDPleq \bigvee_{\posCel\setin \posC}\adpb(\tup{\F{\posgenCel},\FposgenAel}\Fop, \tup{\R{\posgenCel},\RposgenBel}) \\
              & =\Tr_{\F{\posgenA},\R{\posgenB}}^\posC(\adpb)(\FposgenAelop,\RposgenBel).
        \end{aligned}
    \end{equation}
    Therefore,~$\Tr$ is order-preserving on~\DP.
\end{proof}

\begin{lemma}
    \label{lem:coprod_mon}
    Given~$\adpa,\adpb,\adpc,\adpd \setin \HomSet{\DP}{\F{\posgenA}}{\R{\posgenB}}$ we have:
    \begin{equation}
        \prfperiod{\adpa\posleqof\DP \adpb\ }{\ \adpc\posleqof\DP\adpd}{(\adpa \dpjoin\adpc)\posleqof\DP (\adpb\dpjoin \adpd)}
    \end{equation}
\end{lemma}
\begin{proof}
    For any~$\FposgenAel\setin \F{\posgenA}$,~$\RposgenBel\setin \R{\posgenB}$:
    \begin{equation}
        \begin{aligned}
            ~ & (\adpa \dpjoin \adpc)(\FposgenAelop,\RposgenBel) \\
              & =\adpa(\FposgenAelop,\RposgenBel)\boolor \adpc(\FposgenAelop,\RposgenBel) \\
              & \posleqof\DP \adpb(\FposgenAelop,\RposgenBel)\boolor \adpc(\FposgenAelop,\RposgenBel) \\
              & \posleqof\DP \adpb(\FposgenAelop,\RposgenBel)\boolor \adpd(\FposgenAelop,\RposgenBel).
        \end{aligned}
    \end{equation}
\end{proof}

\begin{lemma}
    \label{lem:intersection_mon}
    Given~$\adpa,\adpb,\adpc,\adpd \setin \HomSet{\DP}{\F{\posgenA}}{\R{\posgenB}}$ we have:
    \begin{equation}
        \prfperiod{\adpa\posleqof\DP \adpb\ }{\ \adpc\posleqof\DP \adpd}{(\adpa \dpmeet \adpc)\posleqof\DP (\adpb\dpmeet \adpd)}
    \end{equation}
\end{lemma}
\begin{proof}
    For any~$\FposgenAel\setin \F{\posgenA}$,~$\RposgenBel\setin \R{\posgenB}$:
    \begin{equation}
        \begin{aligned}
            ~ & (\adpa \dpmeet \adpc)(\FposgenAel^*,\RposgenBel) \\
              & =\adpa(\FposgenAelop,\RposgenBel)\booland \adpc(\FposgenAelop,\RposgenBel) \\
              & \posleqof\DP \adpb(\FposgenAelop,\RposgenBel)\booland \adpc(\FposgenAelop,\RposgenBel) \\
              & \posleqof\DP \adpb(\FposgenAelop,\RposgenBel)\booland \adpd(\FposgenAelop,\RposgenBel).
        \end{aligned}
    \end{equation}
\end{proof}
