% !TEX root = chapter-standalone.tex

\section{Morphism-representable wire crossings}
\label{sec:symmetric-stacking-categories}

\todotext{Explain why we start with categories}

\todotext{J: @J: fill this subsection}

\begin{ctdefinition}
    \label{def:sym-stacking-cat}
    A \emph{symmetric stacking category} is an \SY{associative stacking category} $\tup{\CatC, \mtimescatob, \mtimescatmor}$ with

    \constit

    \begin{enumerate}
        \item For any two objects $\Obja, \Objb \setin \Ob_\CatC$ there exists an isomorphism
              \begin{equation}
                  \permmap_{\Obja, \Objb} \colon \Obja \mtimescatob \Objb \mto \Objb \mtimescatob \Obja,
              \end{equation}
              called the \emph{braiding}.
    \end{enumerate}

    \condit

    \begin{enumerate}
        \item \emph{Naturality}: For any morphisms $\mora \colon \Obja \mto \Objc$, $\morb \colon \Objb \mto \Objd$, the diagram
              \begin{equation}\label{eq:naturality-square-symm-stacking-cat}
                  \middlesag{naturality-square-symm-stacking-cat}
              \end{equation}
              commutes.
        \item \emph{Compatibility with nesting}:
              \begin{equation}\label{eq:hexagon-identity-symm-stacking-cat}
                  \middlesag{hexagon-identity-symm-stacking-cat}
              \end{equation}

        \item \emph{Symmetry}: For all $\Obja, \Objb \setin \Ob_\CatC$,
              \begin{equation}
                  \permmap_{\Obja, \Objb} \mthen \permmap_{\Objb, \Obja} = \catid_{\Obja \mtimescatob \Objb}.
              \end{equation}
    \end{enumerate}
\end{ctdefinition}

We note that the braidings $\permmap_{\Obja, \Objb}$ of a symmetric stacking category induce source and target permutation operations as in the \cref{def:symmetric-stacking-category} of symmtric stacking semicatory.
To see how this works, recall that every permutation can be written in a (non-unique) way as a composition of transpositions.

\begin{lemma}
    Every symmetric stacking category $\tup{\CatC, \mtimescatob, \mtimescatmor, \permmap}$, when equipped with the source and target permutation operations induced by the braidings $\permmap_{\Obja, \Objb}$, is a symmetric stacking semicategory.
\end{lemma}

\todotext{J: @JL: is this lemma true like this, or do we need any more assumptions/constituents/conditions for the definition of symmetric stacking category? check, because we are not assuming functorial stacking!!}

\todotext{J: @JL: make two example of symmetric strict monoidal \SY{semicategory} using natural numbers, once with addition and once with multiplication}

\devel{
    \begin{ctdefinition}[Commutative stacking category]
        \label{def:commutative-stacking-cat}
        A \maindef{commutative stacking category} is a \SY{symmetric stacking category} $\tup{\CatC, \mtimescatob, \mtimescatmor}$ with

        \condit

        \begin{enumerate}
            \item \emph{Commutativity}:
                  \begin{equation}
                      \Obja \mtimescatob \Objb = \Objb \mtimescatob \Obja
                  \end{equation}
                  for all object $\Obja, \Objb \setin \Ob_\CatC$.
        \end{enumerate}
    \end{ctdefinition}

    \todotext{J: @JL: do we need any more assumptions/constituents/conditions? check, because we are not assuming functorial stacking!!}

    \todotext{J: @J: given an example}
}
\subsection{Symmetric strict monoidal categories}

\todotext{define this concept and discuss examples}

\begin{ctdefinition}[Symmetric strict monoidal category]
    \label{def:symmetric-strict-monoidal-cat}
    A \maindef{symmetric strict monoidal category} is a \SY{strict monoidal category} $\tup{\CatC, \mtimescat, \idmoncat}$ which is a symmetric stacking category.
    This means that $\CatC$ is equipped with braidings
    \begin{equation}
        \permmap_{\Obja, \Objb} \colon \Obja \mtimescat \Objb \mto \Objb \mtimescat \Obja \end{equation}
    which satisfy the conditions of naturality, compatibility with nesting, and symmetry.
\end{ctdefinition}

\subsection{Props}

\begin{ctdefinition}\label{def:prop}
    A \emph{prop} is a symmetric strict monoidal category $\CatC$ where the collection of objects is the natural numbers $\natnumbers$, the monoidal product on objects is addition of natural numbers, and the monoidal unit is $0 \setin \natnumbers$.
\end{ctdefinition}

\begin{example}\label{exa:prop-finset}
    There is a prop $\stylecat{FinSet}$ where
    \begin{itemize}
        \item the set of morphisms from $m$ to $n$ (for $m, n \setin \natnumbers$) is defined to be the set of functions from $\makeset{1, \dots, m}$ to $\makeset{1, \dots, n}$;
        \item composition is the usual composition of functions;
        \item identity morphisms are identity functions;
        \item the monoidal product of functions $\mora \colon \makeset{1, \dots, m} \mto \makeset{1, \dots, n}$ and $\mora\stylemorph{'} \colon \makeset{1, \dots, m'} \mto \makeset{1, \dots, n'}$ is the ``disjoint union''
              \begin{equation}
                  \mora + \mora\stylemorph{'} \colon \makeset{1, \dots, m + m'} \mto \makeset{1, \dots, n + n'}.
              \end{equation}
    \end{itemize}
\end{example}

\begin{example}\label{exa:prop-mat}
    We define a prop $\stylecat{Mat}_\reals$ where:
    \begin{itemize}
        \item morphisms from $m$ to $n$ are $n \times m$ matrices with entries in $\reals$ (we also allow zero-dimensional matrices);
        \item composition is matrix multiplication ;
        \item identity morphisms are identity matrices;
        \item the monoidal product of matrices $\mat{A} \colon \styleobj{m} \mto \styleobj{n}$ and $\mat{B} \colon \styleobj{m'} \mto \styleobj{n'}$ is
              \begin{equation}
                  \begin{bmatrix}
                      \mat{A} & \mat{0} \\
                      \mat{0} & \mat{B}
                  \end{bmatrix}
                  \colon \styleobj{m} \catsumob \styleobj{m'} \mto \styleobj{n} \catsumob \styleobj{n'}.
              \end{equation}
    \end{itemize}
\end{example}

\begin{example}\label{exa:prop-linrel}
    We define a prop $\stylecat{LinRel}_\reals$ where:
    \begin{itemize}
        \item morphisms from $m$ to $n$ are linear relations $\reals^m \mto \reals^n$ (in other words, linear subspaces of $\reals^m \oplus \reals^n$);
        \item composition is composition of relations;
        \item identity morphisms are identity relations;
        \item the monoidal product of linear relations $\relA \colon \styleobj{m} \mto \styleobj{n}$ and $\relB \colon \styleobj{m'} \mto \styleobj{n'}$ is $\relA \oplus \relB \colon \reals^m \oplus \reals^{m'} \mto \reals^n \oplus \reals^{n'}$, where
              \begin{equation}
                  \relA \oplus \relB = \makeset{\tup{\tup{v,v'},\tup{w,w'}} \mid \tup{v,w} \setin\relA \text{ and } \tup{v', w'} \setin\relB }.
              \end{equation}

    \end{itemize}
\end{example}

\devel{
    \todotext{Maybe move this where we start doing some algebraic perspective}
    \begin{definition}\label{def:prop-expressions}
        Let $\stylesets{G}$ be a set and $s, t \colon \stylesets{G} \mto \natnumbers$ functions which specify the source and target of elements of $\stylesets{G}$.
        The set $E(\stylesets{G})$ of $\stylesets{G}$\emph{-generated prop expressions} (or just \emph{expressions}) is defined inductively as follows.
        \begin{itemize}
            \item The empty morphism $\catid_\styleobj{0} \colon \styleobj{0} \mto \styleobj{0}$, the identity morphism $\catid_\styleobj{1} \colon \styleobj{1} \mto \styleobj{1}$, and the symmetry $\stylenat{\sigma} \colon \styleobj{2} \mto \styleobj{2}$ are expressions.
            \item The generators $g \setin\stylesets{G}$ are expressions $g \colon s(g) \mto t(g)$.
            \item If $\mora \colon \styleobj{a} \mto \styleobj{b}$ and $\morb \colon \styleobj{x} \mto \styleobj{y}$ are expressions, then $\mora + \morb \colon \styleobj{a} + \styleobj{x} \mto \styleobj{b} + \styleobj{y}$ is an expression.
            \item If $\mora \colon \styleobj{a} \mto \styleobj{b}$ and $\morb \colon \styleobj{y} \mto \styleobj{z}$ are expressions, then $\mora \mthen \morb \colon \styleobj{a} \mto \styleobj{c}$ is an expression.
        \end{itemize}
        The functions $s$ and $t$ are extended from $\stylesets{G}$ to $E(\stylesets{G})$ in the obvious way.
        If $\mora \colon m \mto n$ is an expression, the pair $\tup{\styleobj{m}, \styleobj{n}}$ is called its \emph{arity}.
    \end{definition}

    \begin{definition}\label{def:presentation-of-a-prop}
        A \emph{presentation} $\tup{\stylesets{G}, s, t, \stylesets{R}}$ of a prop is a set $\stylesets{G}$, source and target functions $s, t \colon \stylesets{G} \mto \natnumbers$, and a relation $\stylesets{R} \setsubseteq E(\stylesets{G}) \cartprod E(\stylesets{G})$ between expressions such that for any $\tup{e_1, e_2} \setin\stylesets{R}$, the expressions $e_1$ and $e_2$ have the same arity.

        The prop $\CatC$ \emph{presented} by $\tup{\stylesets{G}, s, t, \stylesets{R}}$ is the prop whose morphisms are elements of $E(\stylesets{G})$ quotiented by the equations (or "relations") $e_1 = e_2$ for $\tup{e_1, e_2} \setin\stylesets{R}$, and by the axioms for symmetric strict monoidal categories.
    \end{definition}
}