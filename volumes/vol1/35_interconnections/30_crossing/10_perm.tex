
\section{A strict monoidal category of permutations}
\label{sec:parallelism-and-permutations}

\todotext{J: @J: write some brief intro words here}

% \subsection{A strict monoidal category of permutations}

Recall that a \emph{permutation} of order $n \setin\natnumbers$ is a bijective function from the set $\makeset{1,2, \dots , n}$ to itself.
For fixed $n \setin\natnumbers$, the set of permutations of order $n$ has a group structure, where the composition operation is composition of functions.
We call this group $\Perms(n)$.
For the special case of $n = 0$, we interpret $\Perms(0)$ as containing the unique function from the empty set $\emptyset$ to itself; this function is bijective.

There are several typical notations for indicating permutations.
We will use the following one, and illustrate it by example.
If, for instance, $\permmap \setin \Perms(3)$ is the permutation
\begin{align*}
    \makeset{1,2,3} & \to \makeset{1,2,3} \\
    1               & \mapsto 2 \\
    2               & \mapsto 3 \\
    3               & \mapsto 1
\end{align*}
then we indicate this by writing $\permmap = (2, 3, 1)$.
The way we think of this notation is that we start with $(1, 2, 3)$ and then apply $\permmap$ to each entry.
In other words, in general, our notation for some $\permmap \setin \Perms(n)$ is $(\permmap(1), \dots, \permmap(n))$.

To help us talk about permuting inputs and output of morphisms in stacking semicategories, we make the following definitions.

\begin{definition}[Permutation category]\label{def:Permcat}
    The \emph{permutation category} $\Permcat$ is
    \begin{enumerate}
        \item \emph{Objects}: the natural numbers $\natnumbers$;
        \item \emph{Morphisms}: permutations in $\Perms(n)$, for all $n \setin \natnumbers$.
        \item \emph{Composition}: composition of permutations;
        \item \emph{Identities}: identity functions (which are permutations).
              We will denote the identity permutation in $\Perms(n)$ by $1_n$.
    \end{enumerate}
\end{definition}

\begin{definition}\label{def:Permcat-1}
    For any two objects $n, m \setin\natnumbers$ of $\Permcat$, let
    \begin{equation}
        n \mtimescatob m \definedas n + m,
    \end{equation}
    and for any two morphisms $\permmap = (\permmap(1), \dots, \permmap(n))$, $\pi = (\pi(1), \dots, \pi(m))$ of $\Permcat$, let
    \begin{equation}
        \permmap \mtimescatmor \pi \definedas (\permmap(1), \dots, \permmap(n), \pi(1) + n, \dots, \pi(m) + n).
    \end{equation}
\end{definition}

\begin{lemma}
    The operations $\mtimescatob$ and $\mtimescatmor$ make $\Permcat$ into a functorial stacking category $\tup{\Permcat, \mtimescat}$.
\end{lemma}

\todotext{\alphubel: prove the lemma as the solution to a graded exercise}

\begin{lemma}
    $\tup{\Permcat, \mtimescat}$ is a strict monoidal category, with unit $\idmoncat$ given by $0 \setin \natnumbers$.
\end{lemma}

\todotext{\alphubel: prove the lemma}

\subsection{Block permutations}

We will also need to describe permutations that permute ``blocks'' of elements of $\makeset{1, \dots, n}$.
Suppose we are given natural numbers $k_1, \dots, k_r$ such that $k_1 + \dots + k_r = n$.
We call this a partition of $n$ and we think of it as partitioning the set $\makeset{1, \dots, n}$ into blocks of size $k_j$, $1 \leq j \leq r$.

Given a permutation $\sigma \setin \Perms(r)$, there is a way to turn it into a \emph{block permutation} $\sigma_*(k_1, \dots, k_r) \setin\Perms(r)$.

We'll start with an example, then give a general formula.
Suppose $n = 5$, $r=2$, and $k_1 = 3$ and $k_2 = 2$.
Let $\sigma = (2, 1) \setin \Perms(2)$.
Then
$$
    \sigma_*(k_1, k_2) = \sigma_*(3,2) = (4,5,1,2,3) \setin \Perms(5).
$$
What is happening is that we start with $(1,2,3,4,5)$, we split it into the blocks $((1,2,3),(4,5))$ of length $3$ and $2$ respectively, we apply $\sigma$ to the blocks to get $((4,5),(1,2,3))$, and then we forget the subdivision into blocks, obtaining $(4,5,1,2,3)$.

The general formula to compute $\sigma_*(k_1,\dots, k_r)$ is that, for each $j = 1, \dots, r$, in the sequence $(\sigma(1), \dots, \sigma(r))$ we replace $\sigma(j)$ with the sequence
$$
    k_1 + \dots + k_{j-1} + 1, \quad k_1 + \dots + k_{j-1} + 2, \quad \dots, \quad k_1 + \dots + k_{j-1} + k_j.
$$

