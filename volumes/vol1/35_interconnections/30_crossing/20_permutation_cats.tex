% !TEX root = chapter-standalone.tex

\section{Symmetric stacking semicategories}
\label{sec:symmetric-stacking-semicategories}
\todotext{Consider changing ``symmetric'' to a different word}

\todotext{Add motivating example}

Let~\CatC be an \SY{associative stacking} semicategory, and consider the morphisms
\begin{equation}
    \mora \mtimescatmor \morb \colon \Obja \mtimescatob \Objb \mto \Objc \mtimescatob \Objd
\end{equation}
and
\begin{equation}
    \morc \mtimescatmor \mord \colon \Objd \mtimescatob \Objc \mto \Objf \mtimescatob \Obje.
\end{equation}
If~$\Objc \mtimescatob \Objd \neq \Objd \mtimescatob \Objc$, then the morphisms~$\mora \mtimescatmor \morb$ and~$\morc \mtimescatmor \mord$ are, in our model, technically not composable.
However, in some examples, we will in fact want to be able to compose such morphisms, because in those cases~$\Objc \mtimescatob \Objd$ and~$\Objd \mtimescatob \Objc$ are essentially equal, even if they aren't equal on the nose.

For example,~$\mora, \morb, \morc, \mord$ might be systems, represented graphically by boxes with wires for input and output.
The stacked system~$\mora \mtimescatmor \morb$ will have output wires for~$\Objc$ and $\Objd$, and the stacked system~$\morc \mtimescatmor \mord$ will have input wires for~$\Objd$ and~$\Objc$.
In some applications we are often indeed able to connect the one output wire labeled with~$\Objc$ to the input wire also labeled with~$\Objc$, and similarly for~$\Objd$.

To model this for stacking \SY{semicategories}, we introduce operations that correspond to crossing (or permuting) wires, both on the input and output sides of a morphism.
For instance, if $\mora$ is a morphism that can be written as having the type
\begin{equation}\label{eq:wire-permutation-exa-1}
    \mora \colon \Objan{1} \mtimescatob \Objan{2} \mtimescatob \Objan{3} \mto \Objbn{1} \mtimescatob \Objbn{2}
\end{equation}
then we want to be able to use any permutation $\sigma \setin \Perms(3)$ as a recipe to permute the input wires of $\mora$.
Specifically, given for example $\sigma = (3, 1, 2) \setin \Perms(3)$, we want a corresponding function
\begin{equation}
    \sourceperm{\sigma} \colon \HomSet{\CatC}{ \Objan{1} \mtimescatob \Objan{2} \mtimescatob \Objan{3}}{\Objbn{1} \mtimescatob \Objbn{2}} \to \HomSet{\CatC}{\Objan{3} \mtimescatob \Objan{1} \mtimescatob \Objan{2}}{\Objbn{1} \mtimescatob \Objbn{2}}
\end{equation}
which maps any morphism $\mora$ of the type \cref{eq:wire-permutation-exa-1} to a morphism that corresponds to $\mora$ with the its input wires permuted according to $\sigma$.

One subtlety here is that, for instance, the representation of the source object of the morphism \cref{eq:wire-permutation-exa-1} as a three-fold stacking is not unique.
For example, it could be the case that there exist objects $\Objcn{1}$ and $\Objcn{2}$ such that
\begin{equation}\label{eq:wire-permutation-exa-two-different-reps-1}
    \Objan{1} \mtimescatob \Objan{2} \mtimescatob \Objan{3} = \Objcn{1} \mtimescatob \Objcn{2}.
\end{equation}
In this case, $\mora$ can also be viewed as having the type
\begin{equation}
    \mora \colon \Objcn{1} \mtimescatob \Objcn{2} \mto \Objbn{1} \mtimescatob \Objbn{2},
\end{equation}
and in this representation, it is no longer appropriate to permute the inputs of $\mora$ using a permutation in $\Perms(3)$ but rather one in $\Perms(2)$.

Thus the representation of a morphism as a box with some number of input wires and some number of output wires is just one of potentially many different representations of the very same morphism and the application of a permutation to a morphism is always relative to a specific representation.

One special case of situations like the one indicated in \cref{eq:wire-permutation-exa-two-different-reps-1} is when we use parentheses to group together an $n$-fold stacking of objects into ``blocks'' of size $k_1, \dots, k_r$.
For example, given a stacking of objects
\begin{equation}\label{eq:wire-permutation-exa-two-different-reps-2}
    \Objan{1} \mtimescatob \Objan{2} \mtimescatob \Objan{3} \mtimescatob \Objan{4} \mtimescatob \Objan{5}
\end{equation}
we can use parentheses, as below, to think of \cref{eq:wire-permutation-exa-two-different-reps-1} as a stacking of just \emph{two} objects:
\begin{equation}\label{eq:wire-permutation-exa-two-different-reps-3}
    (\Objan{1} \mtimescatob \Objan{2} \mtimescatob \Objan{3}) \mtimescatob (\Objan{4} \mtimescatob \Objan{5}).
\end{equation}
In other words, there are objects $\Objcn{1} = \Objan{1} \mtimescatob \Objan{2} \mtimescatob \Objan{3}$ and $\Objcn{2} = \Objan{4} \mtimescatob \Objan{5}$ such that
\begin{equation}
    \Objan{1} \mtimescatob \Objan{2} \mtimescatob \Objan{3} \mtimescatob \Objan{4} \mtimescatob \Objan{5} = \Objcn{1} \mtimescatob \Objcn{2}.
\end{equation}

\begin{marginfigure}
    \centering
    \includesag{parallelism_symmetric-stacking-source-perm}
    %\includegraphics[scale=0.1]{parallelism_symmetric-stacking-source-perm}
    \caption{A diagram for a source permutation map.}
    \label{fig:symmetric-stacking-left-perm}
\end{marginfigure}

\begin{marginfigure}
    \centering
    \includesag{parallelism_symmetric-stacking-target-perm}
    %\includegraphics[scale=0.1]{parallelism_symmetric-stacking-target-perm}
    \caption{A diagram for a target permutation map.}
\end{marginfigure}

\begin{marginfigure}
    \centering
    \includesag{parallelism_symmetric-stacking-source-perm-evaluated}
    %\includegraphics[scale=0.1]{parallelism_symmetric-stacking-source-perm-evaluated}
    \caption{A source permutation applied to a morphism $\mora \colon \Obja \mto \Objb$ whose source has a factorization $\Obja = \Objan{1} \mtimescatob \Objan{2}  \mtimescatob \Objan{3}$ and whose target has a factorization $\Objb = \Objbn{1} \mtimescatob \Objbn{2}  \mtimescatob \Objbn{3}$.}
    \label{fig:symmetric-stacking-left-perm-evaluated}
\end{marginfigure}

\begin{marginfigure}
    \centering
    \includesag{parallelism_symmetric-stacking-target-perm-evaluated}
    %\includegraphics[scale=0.1]{parallelism_symmetric-stacking-target-perm-evaluated}
    \caption{The target permutation using the same permutation as in \cref{fig:symmetric-stacking-left-perm} and applied to the same morphism.}
    \label{fig:symmetric-stacking-target-perm}
\end{marginfigure}
\todotext{some more words here}
\begin{ctdefinition}[Symmetric stacking semicategory]
    \label{def:symmetric-stacking-category}
    A \maindef{symmetric stacking semicategory} is an \SY{associative stacking} \SY{semicategory}~$\tup{\CatC, \mtimescatob, \mtimescatmor}$ with:

    \constit

    \todotext{The $\sourceperm{\permmap}$ depends also on the X and Y.
    }
    \begin{itemize}
        \item Functions
              \begin{equation}
                  \sourceperm{\permmap} \colon \HomSet{\CatC}{\mtimescatob_{i=1}^n \Objan{i}}{\Objb} \to \HomSet{\CatC}{\mtimescatob_{i=1}^n \Obja_{\permmap^{-1}(i)}}{\Objb}
              \end{equation}
              and
              \begin{equation}
                  \targetperm{\permmap} \colon \HomSet{\CatC}{\Obja}{\mtimescatob_{i=1}^n \Objbn{i}} \to \HomSet{\CatC}{\Obja}{\mtimescatob_{i=1}^n \Objb_{\permmap(i)}}
              \end{equation}
              for every~$\permmap \setin \Perms(n)$,~$n \geq 1$.
              We call these functions \emph{source permutations} and \emph{target permutations}, respectively.
    \end{itemize}

    \condit

    \begin{itemize}

        \item \emph{Compatibility with permutation composition:}

              The equations
              \begin{equation}
                  \label{eq:perm-compatibility-with-perm-comp}
                  \sourceperm{\permmap} \mthen \sourceperm{\pi} = \sourceperm{(\pi\mthen\permmap)}
                  \quad \text{ and } \quad
                  \targetperm{\permmap} \mthen \targetperm{\pi} = \targetperm{(\permmap \mthen \pi)}
              \end{equation}
              hold for all~$\permmap, \pi \setin \Perms(n)$,~$n \geq 2$.

        \item \emph{Compatibility with permutation identities:}

              For any $n \setin\natnumbers$, if $\permmap = 1_n$ is the identity permutation in $\Perms(n)$, then
              \begin{equation}
                  \label{eq:perm-compatibility-with-perm-identity-1}
                  \sourceperm{(1_n)} \colon \HomSet{\CatC}{\mtimescatob_{i=1}^n \Objan{i}}{\Objb} \to \HomSet{\CatC}{\mtimescatob_{i=1}^n \Obja_{i}}{\Objb}
              \end{equation}
              and
              \begin{equation}
                  \label{eq:perm-compatibility-with-perm-identity-2}
                  \targetperm{(1_n)} \colon \HomSet{\CatC}{\Obja}{\mtimescatob_{i=1}^n \Objbn{i}} \to \HomSet{\CatC}{\Obja}{\mtimescatob_{i=1}^n \Objb_i}
              \end{equation}
              are the identity functions.

        \item \emph{Compatibility with block permutations:}

              For any $n \setin\natnumbers$ and any partition $n = k_1 + \dots + k_r$, if $\Objcn{1}, \dots, \Objcn{r}$ are equal to the $r$ factors in a ``partitioned'' stacking of the form
              \begin{align*}
                  \Obja = (\Objan{1} \mtimescatob \dots \mtimescatob \Objan{k_1}) & \mtimescatob (\Objan{k_1 + 1} \mtimescatob \dots \mtimescatob \Objan{k_1 + k_2}) \ \mtimescatob \\ & \dots \mtimescatob (\Objan{k_1 + \dots + k_{r-1} +1} \mtimescatob \dots \mtimescatob \Objan{n})
              \end{align*}
              then for any $\sigma \setin\Perms(r)$, the source permutation map $\sourceperm{\sigma}$ with respect to this $r$-fold stacking is equal to the source permutation map $\sourceperm{(\sigma_*(k_1, \dots, k_r))}$ with respect to the $n$-fold stacking
              \begin{equation*}
                  \Obja = \Objan{1} \mtimescatob \dots \mtimescatob \Objan{n}.
              \end{equation*}
              (Here, $\sigma_*(k_1, \dots, k_r)$ is the block permutation induced by $\sigma$ and the partition $n = k_1 + \dots + k_r$.)

              The analogous statement for target permutation maps is also required to hold.

        \item \emph{Compatibility with semicategory composition:}

              For all $\mora \setin \HomSet{\CatC}{\Obja}{\mtimescatob_{i=1}^n \Objbn{i}}$ and $\morb \setin \HomSet{\CatC}{\mtimescatob_{i=1}^n \Objb_{\permmap(i)}}{\Objc}$
              \begin{equation}
                  \label{eq:perm-compatibility-with-composition}
                  \targetperm{\permmap} (\mora) \mthen \morb = \mora \mthen \sourceperm{\permmap}(\morb).
              \end{equation}

        \item \emph{Compatibility with stacking:}

              For all $\mora \setin\HomSet{\CatC}{\mtimescatob_{i=1}^n \Objan{i}}{\Objbn{1}}$, $\morb \setin\HomSet{\CatC}{\mtimescatob_{i=n+1}^{n+m} \Objan{i}}{\Objbn{2}}$,
              \begin{equation}
                  \sourceperm{\permmap}(\mora) \mtimescatmor \sourceperm{\pi}(\morb) = \sourceperm{(\permmap \mtimescat \pi)}(\mora \mtimescatmor \morb)
              \end{equation}
              holds for all $\permmap \setin\Perms(n), \pi \setin\Perms(m)$, where $\mtimescat$ is the stacking operation in $\Permcat$.

              Similarly, for all $\mora \setin\HomSet{\CatC}{\Objan{1}}{\mtimescatob_{i=1}^n \Objbn{i}} $ and $\morb \setin\HomSet{\CatC}{\Objan{2}}{\mtimescatob_{i=n+1}^{n+m} \Objbn{i}}$,
              \begin{equation}
                  \targetperm{\permmap}(\mora) \mtimescatmor \targetperm{\pi}(\morb) = \sourceperm{(\permmap \mtimescat \pi)}(\mora \mtimescatmor \morb)
              \end{equation}
              holds for all $\permmap \setin\Perms(n), \pi \setin\Perms(m)$.

        \item \emph{Sliding:}

              Given~$\moran{i} \colon \Objan{i} \mto \Objbn{i}$,~$1 \leq i \leq n$, it holds that
              \begin{equation}
                  \label{eq:perm-compatibility-with-stacking-1}
                  \targetperm{\permmap} ( \mtimescatmor_{i=1}^n \moran{i} ) = \sourceperm{\permmap}( \mtimescatmor_{i=1}^n \mora_{\permmap(i)} )
              \end{equation}
              and
              \begin{equation}
                  \label{eq:perm-compatibility-with-stacking-2}
                  \sourceperm{\permmap} ( \mtimescatmor_{i=1}^n \moran{i} ) = \targetperm{\permmap} ( \mtimescatmor_{i=1}^n \mora_{\permmap^{-1}(i)} ).
              \end{equation}

    \end{itemize}
\end{ctdefinition}

\todotext{@J: make figures to illustrate compatibility with stacking...}

\todotext{J: @J: also I think we need/should say something about how block permutations related to their non-block forms... how is this dealt with for classical symmetric monoidal categories? coherence? perhaps call this property something like ``compatibility with refinement''...}

\todotext{@J: check if this definition is compatible with the definition of a symmetric \SY{strict monoidal category}...}

In terms of diagrams, the condition of compatibility with composition \cref{eq:perm-compatibility-with-composition} is illustrated in terms of diagrams in \cref{fig:symmetric-stacking-comp-compat}.

\begin{figure*}[h]
    \centering
    \subfloat[\label{fig:symmetric-stacking-comp-compat-1}$\targetperm{\permmap} (\mora) \mthen \morb$]{
        \includesag{parallelism_symmetric-stacking-comp-compat-1}
        %\includegraphics[scale=0.1]{parallelism_symmetric-stacking-comp-compat-1}
    } \qquad \qquad
    \subfloat[\label{fig:symmetric-stacking-comp-compat-2}$\mora \mthen \sourceperm{\permmap}(\morb)$]{
        \includesag{parallelism_symmetric-stacking-comp-compat-2}
        %\includegraphics[scale=0.1]{parallelism_symmetric-stacking-comp-compat-2}
    }
    \caption{Compatibility with composition. }
    \label{fig:symmetric-stacking-comp-compat}
\end{figure*}

In \cref{fig:compatibility-with-stacking} the first equation \cref{eq:perm-compatibility-with-stacking-1} for compatibility with stacking is illustrated diagrammatically.

\begin{figure*}[h]
    \centering
    \subfloat[\label{fig:compatibility-with-stacking-1}$\targetperm{\permmap} ( \mtimescatmor_{i=1}^n \moran{i} )$]{
        \includesag{parallelism_compatibility-with-stacking-1}
        %\includegraphics[scale=0.1]{parallelism_compatibility-with-stacking-1}
    } \qquad \qquad
    \subfloat[\label{fig:compatibility-with-stacking-2}$\sourceperm{\permmap}(  \mtimescatmor_{i=1}^n \mora_{\permmap(i)}) $]{
        \includesag{parallelism_compatibility-with-stacking-2}
        %\includegraphics[scale=0.1]{parallelism_compatibility-with-stacking-2}
    }
    \caption{Sliding. }
    \label{fig:compatibility-with-stacking}
\end{figure*}

\cref{fig:compatibility-perm-comp} illustrates the first equation \cref{eq:perm-compatibility-with-perm-comp} for compatibility with permutation composition.

\begin{figure}[h]
    \centering
    \subfloat[\label{fig:compatibility-perm-comp-1}$\sourceperm{\permmap} \mthen \sourceperm{\pi}$]{
        \includesag{parallelism_compatibility-perm-comp-1}
        %\includegraphics[scale=0.1]{parallelism_compatibility-perm-comp-1}
    } \qquad \qquad
    \subfloat[\label{fig:compatibility-perm-comp-2}$\sourceperm{(\pi\mthen\permmap)} $]{
        \includesag{parallelism_compatibility-perm-comp-2}
        %\includegraphics[scale=0.1]{parallelism_compatibility-perm-comp-2}
    }
    \caption{Compatibility with composition. }
    \label{fig:compatibility-perm-comp}
\end{figure}

%\begin{figure*}[b]
%    %\includegraphics[width=8cm]{symmetric}
%    \centering
%    \subfloat[]{
%        \includesag{symmetric_stacking}
%    }
%    \subfloat[]{
%        \includesag{symmetric_stacking_bis}
%    }
%    \caption{
%        Illustration of \cref{eq:symmetric-condition}.
%    }
%    \label{fig:stacking-symmetric}
%\end{figure*}

\todotext{\alphubel: spell out Moore as an example}

\todotext{\alphubel: spell out proper LTI as an example}

\begin{example}
    \SetL is a symmetric stacking semicategory in a straightforward manner.
\end{example}

\todotext{\alphubel: explain this example}

\begin{example}
    \LTI with $D=0$ is a symmetric stacking semicategory.
\end{example}

\begin{lemma}
    \label{lem:effects-not-symmetric}
    \Effects is not a symmetric stacking semicategory.
\end{lemma}
\begin{proof}
    Since \Effects is not functorial stacking (see Lemma \cref{lem:effects-not-functorial}), we may safely conclude that \Effects is not symmetric.
    Consider two morphisms $\funa_1\colon\Obja_1\fto\Objb_1, \funa_2\colon\Obja_2\fto\Objb_2$ in \Effects and $\sigma$ a permutation.
    Then, in general permuting after the stakcing of $\funa_1 \mtimescat \funa_2$ is not equal to applying the permutation before the stacking of $\funa_2 \mtimescat \funa_1$.
    This is due to the difference in order of effects on $\EfW$.
\end{proof}
\begin{marginfigure}
    \centering
    \includegraphics{effects-not-symmetric}
    %\includegraphics[scale=0.1]{parallelism_symmetric-stacking-source-perm}
    \caption{Non-symmetry of effects}
    \label{fig:effects-not-symmetric}
\end{marginfigure}

\todotext{\alphubel: prove this lemma}

\todographicsjira{431}{\alphubel: @Andrea: Add figure for this lemma, in the same style as \cref{fig:effects-non-functorial} }

\todojira{699}{\alphubel: Example of \LTI with $d=0$ a symmetric semicat}

\devel{
    \subsection{Commutative stacking semicategories}
    \begin{ctdefinition}[Commutative stacking semicategory]
        \label{def:commutative-stacking-semicat}
        A \maindef{commutative stacking semicategory} is an \SY{associative stacking semicategory} $\tup{\CatC, \mtimescatob, \mtimescatmor}$ with

        \condit

        \begin{enumerate}
            \item \emph{Commutativity}:
                  \begin{equation}
                      \Obja \mtimescatob \Objb = \Objb \mtimescatob \Obja
                  \end{equation}
                  for all object $\Obja, \Objb \setin \Ob_\CatC$.
        \end{enumerate}
    \end{ctdefinition}

    \todotext{This is the same as having a multiset of objsects (order doesn't count).
        Say that if commutativity is satisified then it is symmetric in a trivial way (all the functions are identities.)
    }

    \todotext{J: @JL: do we need any more assumptions/constituents/conditions? check, because we are not assuming functorial stacking!!}

    \todotext{J: @J: given an example}
}

% \devel{

%     \section{Strict monoidal semicategories}

%     \begin{ctdefinition}\label{def:strict-monoidal-semicat}
%         A \emph{strict monoidal semicategory} is a functorial stacking semicategory $\tup{\CatC, \mtimescat}$ with

%         \constit

%         \begin{itemize}
%             \item an object $\idmoncat \setin \Obof{\CatC}$, called the \emph{monoidal unit}
%         \end{itemize}

%         \condit

%         \begin{itemize}
%             \item For any object $\Obja$ of \CatC,
%                   \begin{equation}
%                       \Obja \mtimescatob \idmoncat = \Obja \qquad \text{and} \qquad  \idmoncat \mtimescatob \Obja = \Obja.
%                   \end{equation}
%             \item The monoidal unit $\idmoncat$ has an identity morphism $\catid_\idmoncat$, and for any morphism $\mora \colon \Obja \mto \Objb$,
%                   \begin{equation}
%                       \mora \mtimescatmor \catid_\idmoncat = \mora \qquad \text{and} \qquad \catid_\idmoncat \mtimescat  \mora = \mora.
%                   \end{equation}
%         \end{itemize}

%     \end{ctdefinition}

%     \todotext{@JL: write def of symmetric strict monoidal semicategory}

% }
