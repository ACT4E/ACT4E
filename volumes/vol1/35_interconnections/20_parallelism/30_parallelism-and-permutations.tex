% !TEX root = chapter-standalone.tex

\section{Parallelism and permutations}
\label{sec:parallelism-and-permutations}

\todotext{J: @J: write some brief intro words here}

\subsection{Symmetric stacking semicategories}

Let~\CatC be an \SY{associative stacking} semicategory, and consider morphisms
\begin{equation}
    \mora \mtimescatmor \morb \colon \Obja \mtimescatob \Objb \mto \Objc \mtimescatob \Objd
\end{equation}
and
\begin{equation}
    \morc \mtimescatmor \mord \colon \Objd \mtimescatob \Objc \mto \Objf \mtimescatob \Obje.
\end{equation}
If~$\Objc \mtimescatob \Objd \neq \Objd \mtimescatob \Objc$, then the morphisms~$\mora \mtimescatmor \morb$ and~$\morc \mtimescatmor \mord$ are, in our model, technically not composable.
However, in some examples, we will in fact want to be able to compose such morphisms, because in those cases~$\Objc \mtimescatob \Objd$ and~$\Objd \mtimescatob \Objc$ are essentially equal, even if they aren't equal on the nose.

For example,~$\mora, \morb, \morc, \mord$ might be systems, represented graphically by boxes with wires for input and output.
The stacked system~$\mora \mtimescatmor \morb$ will have output wires for~$\Objc$ and $\Objd$, and the stacked system~$\morc \mtimescatmor \mord$ will have input wires for~$\Objd$ and~$\Objc$.
In some applications we are often indeed able to connect the one output wire labeled with~$\Objc$ to the input wire also labeled with~$\Objc$, and similarly for~$\Objd$.

To model this for stacking \SY{semicategories}, we introduce operations that correspond to crossing (or permuting) wires, both on the input and output sides of morphism.
Recall that, for each~$n \setin \natnumbers$, the symmetric group $S_n$ is the group of \SY{bijections} of the set~$\makeset{1, 2, \ldots, n }$ with itself.

%Now consider a a stacking semicategory $\CatC$. For each natural number $n \geq 2$ and each permutation $\sigma \setin S_n$, we define an operation $P_\sigma: \Ob_\CatC \to \Ob_\CatC$ by
%\begin{equation}
%P_\sigma: \Objan{1} \mtimescatob \dots \mtimescatob  \Obja_n \mapsto \Obja_{\sigma(1)} \mtimescatob \dots \mtimescatob  \Obja_{\sigma(n)}
%\end{equation}

\begin{marginfigure}
    \centering
    \includegraphics[scale=0.1]{parallelism_symmetric-stacking-source-perm}
    \caption{A diagram for a source permutation map.}
    \label{fig:symmetric-stacking-left-perm}
\end{marginfigure}

\begin{marginfigure}
    \centering
    \includegraphics[scale=0.1]{parallelism_symmetric-stacking-target-perm}
    \caption{A diagram for a target permutation map.}
\end{marginfigure}

\begin{marginfigure}
    \centering
    \includegraphics[scale=0.1]{parallelism_symmetric-stacking-source-perm-evaluated}
    \caption{A source permutation applied to a morphism $\mora \colon \Obja \mto \Objb$ whose source has a factorization $\Obja = \Objan{1} \mtimescatob \Objan{2}  \mtimescatob \Objan{3}$ and whose target has a factorization $\Objb = \Objbn{1} \mtimescatob \Objbn{2}  \mtimescatob \Objbn{3}$.}
    \label{fig:symmetric-stacking-left-perm-evaluated}
\end{marginfigure}

\begin{marginfigure}
    \centering
    \includegraphics[scale=0.1]{parallelism_symmetric-stacking-target-perm-evaluated}
    \caption{The target permutation using the same permutation as in \cref{fig:symmetric-stacking-left-perm} and applied to the same morphism.}
    \label{fig:symmetric-stacking-target-perm}
\end{marginfigure}

\begin{ctdefinition}[Symmetric stacking semicategory]
    \label{def:symmetric-stacking-category}
    A \maindef{symmetric stacking semicategory} is an \SY{associative stacking} \SY{semicategory}~\CatC with:

    \constit

    \begin{itemize}
        \item Functions
              \begin{equation}
                  \sourceperm{\sigma} \colon \HomSet{\CatC}{\mtimescatob_{i=1}^n \Objan{i}}{\mtimescatob_{i=1}^n \Objbn{i}} \to \HomSet{\CatC}{\mtimescatob_{i=1}^n \Obja_{\sigma^{-1}(i)}}{\mtimescatob_{i=1}^n \Objbn{i}}
              \end{equation}
              and
              \begin{equation}
                  \targetperm{\sigma} \colon \HomSet{\CatC}{\mtimescatob_{i=1}^n \Objan{i}}{\mtimescatob_{i=1}^n \Objbn{i}} \to \HomSet{\CatC}{\mtimescatob_{i=1}^n \Objan{i}}{\mtimescatob_{i=1}^n \Objb_{\sigma(i)}}
              \end{equation}
              for every~$\sigma \setin \Perms(n)$,~$n \geq 1$.
              We call these functions \emph{source permutations} and \emph{target permutations}, respectively.
    \end{itemize}

    \condit

    \begin{itemize}

        \item \emph{Compatibility with composition:}

              For all $\mora \setin \HomSet{\CatC}{\Obja}{\mtimescatob_{i=1}^n \Objbn{i}}$ and $\morb \setin \HomSet{\CatC}{\mtimescatob_{i=1}^n \Objb_{\sigma(i)}}{\Objc}$
              \begin{equation}
                  \label{eq:perm-compatibility-with-composition}
                  \targetperm{\sigma} (\mora) \mthen \morb = \mora \mthen \sourceperm{\sigma}(\morb).
              \end{equation}

        \item \emph{Compatibility with stacking:}

              Given~$\moran{i} \colon \Objan{i} \mto \Objbn{i}$,~$1 \leq i \leq n$, it holds that
              \begin{equation}
                  \label{eq:perm-compatibility-with-stacking-1}
                  \targetperm{\sigma} ( \mtimescatmor_{i=1}^n \moran{i} ) = \sourceperm{\sigma}( \mtimescatmor_{i=1}^n \mora_{\sigma(i)} )
              \end{equation}
              and
              \begin{equation}
                  \label{eq:perm-compatibility-with-stacking-2}
                  \sourceperm{\sigma} ( \mtimescatmor_{i=1}^n \moran{i} ) = \targetperm{\sigma} ( \mtimescatmor_{i=1}^n \mora_{\sigma^{-1}(i)} ).
              \end{equation}

        \item \emph{Compatibility with permutation composition:}

              The equations
              \begin{equation}
                  \label{eq:perm-compatibility-with-perm-comp}
                  \sourceperm{\sigma} \mthen \sourceperm{\pi} = \sourceperm{(\pi\mthen\sigma)}
                  \quad \text{ and } \quad
                  \targetperm{\sigma} \mthen \targetperm{\pi} = \targetperm{(\sigma \mthen \pi)}
              \end{equation}
              hold for all~$\sigma, \pi \setin S_n$,~$n \geq 2$.

    \end{itemize}
\end{ctdefinition}

\todotext{@J: check if this definition is compatible with the definition of a symmetric \SY{strict monoidal category}, in particular coherence conditions}

In terms of diagrams, the condition of compatibility with composition \cref{eq:perm-compatibility-with-composition} is illustrated in terms of diagrams in \cref{fig:symmetric-stacking-comp-compat}.

\begin{figure}[h]
    \centering
    \subfloat[\label{fig:symmetric-stacking-comp-compat-1}$\targetperm{\sigma} (\mora) \mthen \morb$]{
        \includegraphics[scale=0.1]{parallelism_symmetric-stacking-comp-compat-1}
    } \qquad \qquad
    \subfloat[\label{fig:symmetric-stacking-comp-compat-2}$\mora \mthen \sourceperm{\sigma}(\morb)$]{
        \includegraphics[scale=0.1]{parallelism_symmetric-stacking-comp-compat-2}
    }
    \caption{Compatibility with composition. }
    \label{fig:symmetric-stacking-comp-compat}
\end{figure}

In \cref{fig:compatibility-with-stacking} the first equation \cref{eq:perm-compatibility-with-stacking-1} for compatibility with stacking is illustrated diagrammatically.

\begin{figure}[h]
    \centering
    \subfloat[\label{fig:compatibility-with-stacking-1}$\targetperm{\sigma} ( \mtimescatmor_{i=1}^n \moran{i} )$]{
        \includegraphics[scale=0.1]{parallelism_compatibility-with-stacking-1}
    } \qquad \qquad
    \subfloat[\label{fig:compatibility-with-stacking-2}$\sourceperm{\sigma}(  \mtimescatmor_{i=1}^n \mora_{\sigma(i)}) $]{
        \includegraphics[scale=0.1]{parallelism_compatibility-with-stacking-2}
    }
    \caption{Compatibility with composition. }
    \label{fig:compatibility-with-stacking}
\end{figure}

\cref{fig:compatibility-perm-comp} illustrates the first equation \cref{eq:perm-compatibility-with-perm-comp} for compatibility with permutation composition.

\begin{figure}[h]
    \centering
    \subfloat[\label{fig:compatibility-perm-comp-1}$\sourceperm{\sigma} \mthen \sourceperm{\pi}$]{
        \includegraphics[scale=0.1]{parallelism_compatibility-perm-comp-1}
    } \qquad \qquad
    \subfloat[\label{fig:compatibility-perm-comp-2}$\sourceperm{(\pi\mthen\sigma)} $]{
        \includegraphics[scale=0.1]{parallelism_compatibility-perm-comp-2}
    }
    \caption{Compatibility with composition. }
    \label{fig:compatibility-perm-comp}
\end{figure}

%\begin{figure*}[b]
%    %\includegraphics[width=8cm]{symmetric}
%    \centering
%    \subfloat[]{
%        \includesag{symmetric_stacking}
%    }
%    \subfloat[]{
%        \includesag{symmetric_stacking_bis}
%    }
%    \caption{
%        Illustration of \cref{eq:symmetric-condition}.
%    }
%    \label{fig:stacking-symmetric}
%\end{figure*}

\begin{example}
    \SetL is symmetric in a straightforward manner.
\end{example}

\begin{lemma}
    \label{lem:effects-not-symmetric}
    \Effects is not symmetric.
\end{lemma}

\todographicsjira{431}{\alphubel: @Andrea: Add figure for this lemma, in the same style as \cref{fig:effects-non-functorial} }

\todojira{699}{\alphubel: Example of \LTI with $d=0$ a symmetric semicat}

\devel{ 
\begin{ctdefinition}[Commutative stacking semicategory]
    \label{def:commutative-stacking-semicat}
    A \maindef{commutative stacking semicategory} is a \SY{stacking semicategory} $\tup{\CatC, \mtimescatob, \mtimescatmor}$ with

    \condit

    \begin{enumerate}
        \item \emph{Symmetry}:
              \begin{equation}
                  \Obja \mtimescatob \Objb = \Objb \mtimescatob \Obja
              \end{equation}
              for all object $\Obja, \Objb \setin \Ob_\CatC$.
    \end{enumerate}
\end{ctdefinition}

\todotext{J: @JL: do we need any more assumptions/constituents/conditions? check, because we are not assuming functorial stacking!!}

\todotext{J: @J: given an example}

}
% \devel{

%     \section{Strict monoidal semicategories}

%     \begin{ctdefinition}\label{def:strict-monoidal-semicat}
%         A \emph{strict monoidal semicategory} is a functorial stacking semicategory $\tup{\CatC, \mtimescat}$ with

%         \constit

%         \begin{itemize}
%             \item an object $\idmoncat \setin \Obof{\CatC}$, called the \emph{monoidal unit}
%         \end{itemize}

%         \condit

%         \begin{itemize}
%             \item For any object $\Obja$ of \CatC,
%                   \begin{equation}
%                       \Obja \mtimescatob \idmoncat = \Obja \qquad \text{and} \qquad  \idmoncat \mtimescatob \Obja = \Obja.
%                   \end{equation}
%             \item The monoidal unit $\idmoncat$ has an identity morphism $\catid_\idmoncat$, and for any morphism $\mora \colon \Obja \mto \Objb$,
%                   \begin{equation}
%                       \mora \mtimescatmor \catid_\idmoncat = \mora \qquad \text{and} \qquad \catid_\idmoncat \mtimescat  \mora = \mora.
%                   \end{equation}
%         \end{itemize}

%     \end{ctdefinition}

%     \todotext{@JL: write def of symmetric strict monoidal semicategory}

% }

\devel{
\subsection{Symmetric stacking categories}

\todotext{J: @J: fill this subsection}


\begin{ctdefinition}
    \label{def:sym-stacking-cat}
    A \emph{symmetric stacking category} is a \SY{stacking category} $\tup{\CatC, \mtimescatob, \mtimescatmor}$  with

    \constit

    \begin{enumerate}
        \item For any two objects $\Obja, \Objb \setin \Ob_\CatC$ there exists an isomorphism
              \begin{equation}
                  \sigma_{\Obja, \Objb} \colon \Obja \mtimescatob \Objb \mto \Objb \mtimescatob \Obja,
              \end{equation}
              called the \emph{braiding}.
    \end{enumerate}

    \condit

    \begin{enumerate}
        \item \emph{Naturality}: For any morphisms $\mora \colon \Obja \mto \Objc$, $\morb \colon \Objb \mto \Objd$, the diagram
                  [ INSERT DIAGRAM ]
              commutes.
        \item \emph{Commuativity}: For all $\Obja, \Objb \setin \Ob_\CatC$,
              \begin{equation}
                  \sigma_{\Obja, \Objb} \mthen \sigma_{\Objb, \Obja} = \catid_{\Obja \mtimescatob \Objb}.
              \end{equation}
    \end{enumerate}

\end{ctdefinition}

\todotext{J: @JL: do we need any more assumptions/constituents/conditions? check, because we are not assuming functorial stacking!!}

\todotext{J: @JL: make two example of symmetric strict monoidal \SY{semicategory} using natural numbers, once with addition and once with multiplication}

\begin{ctdefinition}[Commutative stacking category]
    \label{def:commutative-stacking-cat}
    A \maindef{commutative stacking category} is a \SY{symmetric stacking category} $\tup{\CatC, \mtimescatob, \mtimescatmor}$  with

    \condit

    \begin{enumerate}
        \item \emph{Symmetry}:
              \begin{equation}
                  \Obja \mtimescatob \Objb = \Objb \mtimescatob \Obja
              \end{equation}
              for all object $\Obja, \Objb \setin \Ob_\CatC$.
    \end{enumerate}
\end{ctdefinition}

\todotext{J: @JL: do we need any more assumptions/constituents/conditions? check, because we are not assuming functorial stacking!!}

\todotext{J: @J: given an example}

}
