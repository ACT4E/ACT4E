% !TEX root = chapter-standalone.tex

\section[\DP is (strictly) symmetric monoidal]{\DP is a (strictly) symmetric monoidal category}
\label{sec:parallelism-DP-monoidal}

\subsection{The Category~$\cCat{\Pos}$}

We define a category analogous to~$\cCat{\Pos}$, but its objects are ``posets of tuples''.

Given posets~$\posA_1,\ldots, \posA_n$ we define
\begin{equation*}
    \cObj{\posA_1,\ldots, \posA_n}\coloneqq \tup{\cObj{\posAsetn{1},\ldots, \posAsetn{n}}, \posleqof{\cObj{\posA_1,\ldots, \posA_n}}},
\end{equation*}
where~$\cObj{\posAsetn{1},\ldots, \posAsetn{n}}$ is a set of tuples and~$\tup{\posAnel{1},\ldots, \posAnel{n}}\posleqof{\cObj{\posA_1,\ldots, \posA_n}}\tup{\posAnel{1}',\ldots, \posAnel{n}'}$ if and only if~$\posAnel{i}\posleqof{\posA_i} \posAnel{i}'$ for all~$i\in \{0,\ldots, n\}$.
%
%\begin{equation*}
%    \prfdouble{\tup{\posAnel{1},\ldots, \posAnel{n}}\posleqof{\cObj{\posA_1,\ldots, \posA_n}}\tup{\posAnel{1}',\ldots, \posAnel{n}'}}
%    {\posAnel{1} \posleqof{\posA_1}\posAnel{1}' \booland \ldots \booland \posAnel{n} \posleqof{\posA_n} \posAnel{n}'}
%\end{equation*}

Analogously to what we did for \cCat{\Set}, one can define a multiplication operation in \cCat{\Pos}.
Given~$\cObj{\posA_1,\ldots, \posA_m}$ and~$\cObj{\posB_1,\ldots, \posB_n}$, one has
\begin{widepar}
    \begin{equation*}
        \begin{aligned}
            \cObj{\posA_1,\ldots, \posA_m}\cprod \cObj{\posB_1,\ldots, \posB_n} & \definedas \tup{\cObj{\posAsetn{1},\ldots, \posAsetn{m}} \cprod \cObj{\posBsetn{1},\ldots, \posBsetn{n}}, \posleqof{\cObj{\posA_1,\ldots, \posA_m}\cprod \cObj{\posB_1,\ldots, \posB_n}}} \\
                                                                                & =\tup{\cObj{\posAsetn{1},\ldots, \posAsetn{m},\posBsetn{1},\ldots, \posBsetn{n}}, \posleqof{\cObj{\posA_1,\ldots, \posA_m}\cprod \cObj{\posB_1,\ldots, \posB_n}}},
        \end{aligned}
    \end{equation*}
\end{widepar}
where~$\tup{\posAnel{1},\ldots, \posAnel{m},\posBnel{1},\ldots, \posBnel{n}} \posleqof{\cObj{\posA_1,\ldots, \posA_m}\cprod \cObj{\posB_1,\ldots, \posB_n}} \tup{\posAnel{1}',\ldots, \posAnel{m}',\posBnel{1}',\ldots, \posBnel{n}'}$ if and only if~$\posAnel{i}\posleqof{\posA_i}\posAnel{i}'$ and~$\posBnel{j}\posleqof{\posB_j}\posBnel{j}'$ for all~$i\in \{0,\ldots, m\}$,~$j\in \{0,\ldots, n\}$.
%\begin{equation*}
%    \prfdoubleperiod{\tup{\posAnel{1},\ldots, \posAnel{m},\posBnel{1},\ldots, \posBnel{n}} \posleqof{\cObj{\posA_1,\ldots, \posA_m}\cprod \cObj{\posB_1,\ldots, \posB_n}} \tup{\posAnel{1}',\ldots, \posAnel{m}',\posBnel{1}',\ldots, \posBnel{n}'}}
%    {\posAnel{1} \posleqof{\posA_1}\posAnel{1}' \booland \ldots \booland \posAnel{m} \posleqof{\posA_m} \posAnel{m}' \booland \ldots \booland \posBnel{1} \posleqof{\posB_1} \posBnel{1}' \booland \ldots \booland \posBnel{n} \posleqof{\posB_n} \posBnel{n}'}
%\end{equation*}

The resulting poset of tuples is written as~$\cObj{\posA_1,\ldots, \posA_m,\posB_1,\ldots, \posB_n}$.

\begin{definition}[The category~$\cCat{\Pos}$]
    \label{def:PosL}
    The category~$\cCat{\Pos}$ is:
    \begin{enumerate}
        \item \emph{Objects:} posets of tuples.
        \item \emph{Morphisms:}
              Monotone functions between posets of tuples.
        \item \emph{Composition:}
              Composition is the usual composition of functions.
        \item \emph{Identities:}
              Identity functions.
    \end{enumerate}
\end{definition}

\begin{lemma}[\cCat{\Pos} is associative stacking]
    \cCat{\Pos} is associative stacking.
\end{lemma}

\begin{proof}

    We show that~$\cCat{\Pos}$ is an associative stacking semicategory.
    We first show that it is a stacking category.
    The stacking operation on objects corresponds to the multiplication operation~$\cprod$ defined at the beginning of this section.

    The stacking operation on morphisms is defined as
    \begin{equation*}
        \defmapcomma{\mtimescatmor}{\Mor_{\cCat{\Pos}} \cartprod \Mor_{\cCat{\Pos}}}{\mto}{\Mor_{\cCat{\Pos}}}{\tup{\mora,\morb}}{\mora \cprod \morb},
    \end{equation*}
    where~$(\mora \cprod \morb)(\tup{\posAnel{1},\ldots, \posAnel{m}} \tupconcat \tup{\posBnel{1},\ldots, \posBnel{n}})\definedas \mora(\tup{\posAnel{1},\ldots, \posAnel{m}})\tupconcat \morb(\tup{\posBnel{1},\ldots, \posBnel{n}})$.
    To show that the two operations are compatible, we start from
    %\begin{widepar}
    \begin{equation*}
        \prfperiod{\mora\colon \cObj{\posA_1,\ldots,\posA_m}\mto \cObj{\posC_1,\ldots,\posC_o}}
        {\morb\colon \cObj{\posB_1,\ldots,\posB_n}\mto \cObj{\posD_1,\ldots,\posD_t}}
        {\mora \cprod \morb \colon \cObj{\posA_1, \ldots, \posA_m, \posB_1, \ldots, \posB_n} \mto \cObj{\posC_1, \ldots, \posC_o, \posD_1, \ldots, \posD_t}}
    \end{equation*}
    %\end{widepar}

    The crucial part to show is that~$\mora \cprod \morb$ is a monotone map.
    This is easy to show, by leveraging the monotonicity of~$\mora$ and~$\morb$.
    For brevity, in the following we write~$\posAnel{}=\tup{\posAnel{1},\ldots, \posAnel{m}}$,~$\posAnel{}'=\tup{\posAnel{1}',\ldots, \posAnel{m}'}$,~$\posBnel{}=\tup{\posBnel{1},\ldots, \posBnel{n}}$, and~$\posBnel{}'=\tup{\posBnel{1}',\ldots, \posBnel{n}'}$.
    One has
    \begin{equation*}
        \prfcomma{\posAnel{}\tupconcat \posBnel{} \posleqof{\cObj{\posA_1,\ldots, \posA_m,\posB_1,\ldots, \posB_n}} \posAnel{}'\tupconcat \posBnel{}'}
        {\prftree{
                \left(\posAnel{} \posleqof{\cObj{\posA_1,\ldots, \posA_m}} \posAnel{}'\right) \booland \left(\posBnel{} \posleqof{\cObj{\posB_1,\ldots, \posB_n}} \posBnel{}'\right)}
            {
                \prftree{
                    \left(\mora(\posAnel{}) \posleqof{\cObj{\posC_1,\ldots, \posC_o}} \mora(\posAnel{}')\right) \booland \left(\morb(\posBnel{})\posleqof{\cObj{\posD_1,\ldots, \posD_t}} \morb(\posBnel{}')\right)}
                {
                    \prftree{
                        {\mora(\posAnel{}) \tupconcat \morb(\posBnel{})  \posleqof{\cObj{\posC_1,\ldots, \posC_o, \posD_1,\ldots, \posD_t}} \mora(\posAnel{}') \tupconcat \morb(\posBnel{}')}}
                    {(\mora \cprod \morb)(\posAnel{}\tupconcat \posBnel{}) \posleqof{\cObj{\posC_1,\ldots, \posC_o,\posD_1,\ldots, \posD_t}} (\mora \cprod \morb)(\posAnel{}'\tupconcat \posBnel{}')
                    }}}
        }
    \end{equation*}
    which shows monotonicity.

    The proof of associativity is parallel to the proof for \cCat{\Set}.
\end{proof}

\begin{lemma}
    \cCat{\Pos} is functorial stacking.
\end{lemma}
\begin{proof}
    The proof is analogous to the one for \cCat{\Set} in \cref{ex:setfunstack}.
\end{proof}

\begin{definition}[The category \cCat{\DP}]
    The category~$\cCat{\DP}$ is:
    \begin{enumerate}
        \item \emph{Objects:}
              posets of tuples.
        \item \emph{Morphisms:}
              A morphism from~$\posA =\cObj{\posA_1,\ldots, \posA_m}$ to~$\posB = \cObj{\posB_1,\ldots, \posB_m}$ is design problem:
              \begin{equation*}
                  \adp\colon \cObj{\posA_1,\ldots, \posA_m}\op \acprod \cObj{\posB_1,\ldots, \posB_m} \toinPos \Bool.
              \end{equation*}
        \item \emph{Composition:}
              Composition is the usual composition of design problems.
        \item \emph{Identities:}
              The identity morphism on an object~$\cObj{\posA_1,\ldots,\posA_n}$ is given by the identity design problem~$\catid_{\cObj{\posA_1,\ldots, \posA_n}}$.
    \end{enumerate}
\end{definition}

Analogously to what we did for~$\cCat{\Set}$ and~$\cCat{\Pos}$, one can define a multiplication operation for~$\cCat{\DP}$.
Given two design problems
\begin{equation*}
    \begin{aligned}
        \adpa\colon \cObj{\posA_1,\ldots,\posA_m}\op \acprod \cObj{\posC_1,\ldots,\posC_o} & \toinPos \Bool \\
        \adpb\colon \cObj{\posB_1,\ldots,\posB_n}\op \acprod \cObj{\posD_1,\ldots,\posD_t} & \toinPos \Bool,
    \end{aligned}
\end{equation*}
one can define
\begin{equation}
    \label{eq:dpmulti}
    \adpa \cprod \adpb \colon \cObj{\posA_1, \ldots, \posA_m, \posB_1, \ldots, \posB_n}\op \acprod \cObj{\posC_1, \ldots, \posC_o, \posD_1, \ldots, \posD_t}\toinPos \Bool
\end{equation}
where
\begin{equation*}
    \begin{aligned}
        (\adpa \cprod \adpb) & (\tup{\posAnel{1},\ldots, \posAnel{m},\posBnel{1},\ldots, \posBnel{n}^*}, \tup{\posCnel{1},\ldots, \posCnel{k},\posDnel{1},\ldots, \posDnel{o}}) \\
                             & =\adpa(\tup{\posAnel{1},\ldots, \posAnel{m}}^*,\tup{\posCnel{1},\ldots, \posCnel{k}})\booland
        \adpb(\tup{\posBnel{1},\ldots, \posBnel{n}}^*,\tup{\posDnel{1},\ldots, \posDnel{o}})
    \end{aligned}
\end{equation*}

\begin{lemma}[\cCat{\DP} is associative stacking]
    \cCat{\DP} is associative stacking.
\end{lemma}

\begin{proof}
    We show that~$\cCat{\DP}$ is an associative stacking semicategory.
    We first show that it is a stacking category.
    The stacking operation on objects corresponds to the multiplication operation~$\cprod$ defined at the beginning of this section.

    The stacking operation on morphisms is defined as
    \begin{equation*}
        \defmapcomma{\mtimescatmor}{\Mor_{\cCat{\DP}} \cartprod \Mor_{\cCat{\DP}}}{\mto}{\Mor_{\cCat{\DP}}}{\tup{\adpa,\adpb}}{\adpa \cprod \adpb},
    \end{equation*}

    To show that the two operations are compatible, we start from
    \begin{widepar}
        \begin{equation*}
            \prfperiod{\adpa\colon \cObj{\posA_1,\ldots,\posA_m}\op \acprod \cObj{\posC_1,\ldots,\posC_o}\toinPos \Bool}
            {\adpb\colon \cObj{\posB_1,\ldots,\posB_n}\op \acprod \cObj{\posD_1,\ldots,\posD_t}\toinPos \Bool}
            {\adpa \cprod \adpb \colon \cObj{\posA_1, \ldots, \posA_m, \posB_1, \ldots, \posB_n}\op \acprod \cObj{\posC_1, \ldots, \posC_o, \posD_1, \ldots, \posD_t}\toinPos \Bool}
        \end{equation*}
    \end{widepar}

    The crucial part to show is that~$\adpa \cprod \adpb$ is a design problem.
    This is easy to show, by leveraging the fact that~$\adpa$ and~$\adpb$ are design problems.
    We start with the first implication.
    For brevity, we write~$\posAnel{}=\tup{\posAnel{1},\ldots, \posAnel{m}}$,~$\posBnel{}=\tup{\posBnel{1},\ldots, \posBnel{n}}$,~$\posCnel{}=\tup{\posCnel{1},\ldots, \posCnel{o}}$, and~$\posDnel{}=\tup{\posDnel{1},\ldots, \posDnel{t}}$.
    One has:
    \begin{equation*}
        \prfperiod{
        \posAnel{}\tupconcat \posBnel{}\posleqof{\cObj{\posA_1, \ldots, \posA_m, \posB_1, \ldots, \posB_n}} \posAnel{}'\tupconcat \posBnel{}'
        }
        {\prftree{(\posAnel{} \posleqof{\cObj{\posA_1,\ldots, \posA_m}} \posAnel{}') \booland (\posBnel{} \posleqof{\cObj{\posB_1,\ldots, \posB_n}} \posBnel{}')}
        {\prftree{(\adpa(\posAnel{}^*,\posCnel{})\posgeqof{\Bool} \adpa(\posAnel{}'^*,\posCnel{}))
                \booland
                (\adpb(\posBnel{}^*,\posDnel{})\posgeqof{\Bool} \adpb(\posBnel{}'^*,\posDnel{}))}
            {
                \prftree{
                    (\adpa(\posAnel{}^*,\posCnel{})\booland \adpb(\posBnel{}^*,\posDnel{}))
                    \posgeqof{\Bool}
                    (\adpa(\posAnel{}'^*,\posCnel{})\booland \adpb(\posBnel{}'^*,\posDnel{}))
                }
                {(\adpa \cprod \adpb)((\posAnel{}\tupconcat \posBnel{})^*,\posCnel{}\tupconcat \posDnel{})\posgeqof{\Bool}
                    (\adpa \cprod \adpb)((\posAnel{}'\tupconcat \posBnel{}')^*,\posCnel{}\tupconcat \posDnel{})} }
            }
        }
    \end{equation*}
    For the second iteration, one has:
    \begin{equation*}
        \prfperiod{
        \posCnel{}\tupconcat \posDnel{}\posleqof{\cObj{\posC_1, \ldots, \posC_o, \posD_1, \ldots, \posD_t}} \posCnel{}'\tupconcat \posDnel{}'
        }
        {\prftree{(\posCnel{} \posleqof{\cObj{\posC_1,\ldots,\posC_o}} \posCnel{}') \booland (\posDnel{}\posleqof{\cObj{\posD_1,\ldots, \posD_t}} \posDnel{}')}
        {\prftree{(\adpa(\posAnel{}^*,\posCnel{})\posleqof{\Bool}
                \adpa(\posAnel{}^*,\posCnel{}'))
                \booland
                (\adpb(\posBnel{}^*,\posDnel{})\posleqof{\Bool}
                \adpb(\posBnel{}^*,\posDnel{}'))}
            {
                \prftree{
                    (\adpa(\posAnel{}^*,\posCnel{})\booland \adpb(\posBnel{}^*,\posDnel{}))
                    \posleqof{\Bool}
                    (\adpa(\posAnel{}^*,\posCnel{}')\booland \adpb(\posBnel{}^*,\posDnel{}'))
                }
                {(\adpa \cprod \adpb)((\posAnel{}\tupconcat \posBnel{})^*,\posCnel{}\tupconcat \posDnel{})
                    \posleqof{\Bool}
                    (\adpa \cprod \adpb)((\posAnel{}\tupconcat \posBnel{})^*,\posCnel{}'\tupconcat \posDnel{}')} }
            }
        }
    \end{equation*}
    Therefore,~$\adpa \cprod \adpb$ is really a design problem.

    Associativity can be shown following what we did for \Set.
\end{proof}

\begin{lemma}[\cCat{\DP} is functorial stacking]
    \cCat{\DP} is functorial stacking.
\end{lemma}

\begin{proof}
    Consider
    \begin{equation*}
        \begin{aligned}
            \adpa\colon \cObj{\posA_1,\ldots, \posA_i}\op \acprod \cObj{\posC_1,\ldots, \posC_l} & \toinPos \Bool \\
            \adpb\colon \cObj{\posB_1,\ldots, \posB_m}\op \acprod \cObj{\posD_1,\ldots, \posD_n} & \toinPos \Bool \\
            \adpc\colon \cObj{\posC_1,\ldots, \posC_l}\op \acprod \cObj{\posE_1,\ldots, \posE_o} & \toinPos \Bool \\
            \adpd\colon \cObj{\posD_1,\ldots, \posD_n}\op \acprod \cObj{\posF_1,\ldots, \posF_v} & \toinPos \Bool
        \end{aligned}
    \end{equation*}
    We want to prove
    \begin{equation*}
        (\adpa \mthen \adpc) \mtimescatmor (\adpb \mthen \adpd)=(\adpa \mtimescatmor \adpb) \mthen (\adpc \mtimescatmor \adpd).
    \end{equation*}
    For brevity, in the following we write
    \begin{align*}
        \posAnel{} & =\tup{\posAnel{1},\ldots, \posAnel{i}}, &
        \posBnel{} & =\tup{\posBnel{1},\ldots, \posBnel{l}} \\
        \posCnel{} & =\tup{\posCnel{1},\ldots, \posCnel{m}}, &
        \posDnel{} & =\tup{\posDnel{1},\ldots, \posDnel{n}} \\
        \posEnel{} & =\tup{\posEnel{1},\ldots, \posEnel{o}}, &
        \posFnel{} & =\tup{\posFnel{1},\ldots, \posFnel{v}}
    \end{align*}
    Let's start from the left-hand side.
    One has
    \begin{equation*}
        (\adpa \mthen \adpc)(\posAnel{}^*,\posEnel{})=\bigvee_{\posCnel{}\in \cObj{\posC_1,\ldots, \posC_l}}
        \adpa(\posAnel{}^*, \posCnel{}) \booland \adpc(\posCnel{}^*,\posEnel{})
    \end{equation*}
    and
    \begin{equation*}
        \begin{aligned}
            (\adpb \mthen \adpd)(\posBnel{}^*,\posFnel{})=\bigvee_{\posDnel{}\in \cObj{\posD_1,\ldots, \posD_n}}
            \adpb(\posBnel{}^*, \posDnel{}) \booland \adpd(\posDnel{}^*\posFnel{})
        \end{aligned}
    \end{equation*}
    Therefore, we know
    \begin{equation*}
        \begin{aligned}
             & ((\adpa \mthen \adpc)\mtimescatmor (\adpb \mthen \adpd))
            ((\posAnel{}\tupconcat \posBnel{})^*,\posEnel{}\tupconcat \posFnel{}) \\
             & =\bigvee_{\posCnel{}\in \cObj{\posC_1,\ldots, \posC_l}}
            \adpa(\posAnel{}^*, \posCnel{}) \booland \adpc(\posCnel{}^*,\posEnel{}) \booland
            \bigvee_{\posDnel{}\in \cObj{\posD_1,\ldots, \posD_n}} \adpb(\posBnel{}^*, \posDnel{}) \booland \adpd(\posDnel{}^*,\posFnel{}).
        \end{aligned}
    \end{equation*}

    On the other hand, one has
    \begin{equation*}
        (\adpa \mtimescatmor \adpb)((\posAnel{}\tupconcat \posBnel{})^*, \posCnel{}\tupconcat \posDnel{})
        =\adpa(\posAnel{}^*,\posCnel{}) \booland \adpb(\posBnel{}^*,\posDnel{})
    \end{equation*}
    and
    \begin{equation*}
        (\adpc \mtimescatmor \adpd)((\posCnel{}\tupconcat \posDnel{})^*, \posEnel{}\tupconcat \posFnel{})
        =\adpc(\posCnel{}^*,\posEnel{}) \booland \adpd(\posDnel{}^*,\posFnel{})
    \end{equation*}
    Therefore, we know
    \begin{equation*}
        \begin{aligned}
             & ((\adpa \mtimescatmor \adpb)\mthen (\adpc \mtimescatmor \adpd))((\posAnel{}\tupconcat \posBnel{})^*,\posEnel{}\tupconcat \posFnel{}) \\
             & =\bigvee_{\posCnel{}\tupconcat \posDnel{}\in \cObj{\posC_1,\ldots, \posC_m,\posD_1,\ldots, \posD_n}}
            (\adpa \mtimescatmor \adpb)((\posAnel{}\tupconcat \posBnel{})^*, \posCnel{}\tupconcat \posDnel{})
            \booland (\adpc \mtimescatmor \adpd)((\posCnel{}\tupconcat \posDnel{})^*, \posEnel{}\tupconcat \posFnel{}) \\
             & =\bigvee_{\posCnel{}\tupconcat \posDnel{}\in \cObj{\posC_1,\ldots, \posC_m,\posD_1,\ldots, \posD_n}}
            \adpa(\posAnel{}^*,\posCnel{}) \booland \adpb(\posBnel{}^*,\posDnel{})
            \booland
            \adpc(\posCnel{}^*,\posEnel{}) \booland \adpd(\posDnel{}^*,\posFnel{}) \\
             & =\bigvee_{\posCnel{}\in \cObj{\posC_1,\ldots, \posC_l}}
            \adpa(\posAnel{}^*, \posCnel{}) \booland \adpc(\posCnel{}^*,\posEnel{}) \booland
            \bigvee_{\posDnel{}\in \cObj{\posD_1,\ldots, \posD_n}} \adpb(\posBnel{}^*, \posDnel{}) \booland \adpd(\posDnel{}^*,\posFnel{}),
        \end{aligned}
    \end{equation*}
    proving the statement.

\end{proof}

\devel{

    \begin{definition}[The category~$\RelL$]
        \label{def:RelL}
        The category~$\RelL$ is:
        \begin{enumerate}
            \item \emph{Objects:} lists~$\Tupcatt {\setA_1}  \dots {\setA_n}$,~$n \setin \natnumbers$, of sets.
                  This includes the empty list~$\Tupca {}$.
            \item \emph{Morphisms:}
                  Morphisms from~$\setA =\Tupcatt {\setA_1}  \dots {\setA_n}$ to~$\setB = \Tupcatt {\setB_1} \dots {\setB_m}$ are all relations:
                  \begin{equation*}
                      \relA\subseteq ({\setA_1} \cartprod (\setA_2 \cartprod (\ldots \cartprod {\setA_n}))) \cartprod ({\setB_1} \cartprod (\setB_2 \cartprod (\ldots \cartprod {\setB_m}))).
                  \end{equation*}
            \item \emph{Composition:}
                  Composition is the usual composition of relations.
            \item \emph{Identities:}
                  The identity morphism on an object~$\Tupcatt {\setA_1} \dots {\setA_n}$ is given by the identity relation~$\catid_{{\setA_1} \cartprod (\setA_2 \cartprod (\ldots \cartprod {\setA_n}))}$.
        \end{enumerate}
    \end{definition}

}

\subsection{Parallel Composition}
\linkvideo{spring2021-functorial-comp-b:solving-queries:solving-parallel} % Parallel composition
%
%
\begin{example}
    After the X101 spontaneously combusted in low Earth orbit, the astronauts at Jeb's Spaceship Parts go on strike.
    They demand that the engineers take into account safety and living conditions on the future X102.
    As long as the propulsion and life support systems of the X102 do not interact, we can simply tensor the two design problems representing these systems into one, big co-design problem (\cref{fig:examplemonoidal}).
    \begin{figure}[h!]
        \centering
        \includesag{50_engine_tensor_1_2}
        \caption{Example of tensor of design problems. }
        \label{fig:examplemonoidal}
    \end{figure}
\end{example}

In~\DP, putting two design problems in parallel corresponds to their \emph{monoidal product}.

\begin{definition}[Monoidal product in~\DP]
    \label{def:monoidalproduct}
    Given two design problems~$\adpa \colon \funposA \profto \resposB$ and~$\adpb \colon \funposC \profto \resposD$, their \emph{monoidal product}~$\adpa \mtimescat \adpb \colon \funposA \cartprod \funposC \profto \resposB\cartprod \resposD$ is their conjunction:
    \begin{equation}
        \label{eq:monoidalprod_dp}
        \begin{aligned}
            \adpa \mtimescat \adpb \colon (\funposA \cartprod \funposC)\op \cartprod (\resposB \cartprod \resposD) & \toinPos \Bool, \\
            \tup{\tup{\funposAel,\funposCel}\Fop,\tup{\resposBel,\resposDel}}                                      & \mapsto \adpa(\funposAopel,\resposBel) \booland \adpb(\funposCopel,\resposDel).
        \end{aligned}
    \end{equation}
\end{definition}
The diagrammatic representation of the monoidal product is reported in~\cref{fig:dpmonoidal}.

\begin{figure}[h!]
    \centering
    \includesag{50_monoidal}
    \caption{Monoidal product of design problems.}
    \label{fig:dpmonoidal}
\end{figure}

\begin{remark}
    For~$\adpa \colon \funposA\profto \resposB$ and~$\adpb \colon \funposC \profto \resposD$, the monoidal product
    \begin{equation}
        \left(\adpa \mtimescat \adpb\right)(\tup{\funposAopel, \funposCopel}, \tup{\resposBel,\resposDel})
    \end{equation}
    is true if \emph{both}~$\adpa(\funposAopel,\resposBel)$ and~$\adpb(\funposCopel,\resposDel)$ are true, and false otherwise.
\end{remark}

\begin{lemma}
    \label{lem:monoidal_functorial}
    The monoidal product~$\mtimescat$ is functorial (\cref{def:functor}) in~\DP.
\end{lemma}
\begin{proof}
    First, consider posets~$\posA,\posB\setin \Ob_\DP$.
    We show that~$\catid_\posA\mtimescat \catid_\posB = \catid_{\posA\cartprod \posB}$
    It holds that
    \begin{equation}
        \begin{aligned}
            ~ & \left( \catid_{\posA}\mtimescat \catid_{\posB}\right)
            \left( \tup{\F{\posgenAel_1},\F{\posgenBel_1}}\Fop,\tup{\R{\posgenAel_2},\R{\posgenBel_2}}\right) \\
              & =
            \catid_\posgenA(\F{\posgenAel_1^*},\R{\posgenAel_2})\booland \catid_\posgenB(\F{\posgenBel_1^*},\R{\posgenBel_2}) \\
              & =\left( \F{\posgenAel_1}\posleqof\posA \R{\posgenAel_2}\right)\booland \left( \F{\posgenBel_1}\posleqof\posB \R{\posgenBel_2}\right) \\
              & =\tup{\F{\posgenAel_1},\F{\posgenBel_1}}\posleqof{\posA\cartprod \posB}\tup{\R{\posgenAel_2},\R{\posgenBel_2}} \\
              & =\catid_{\posA\cartprod \posB}\left( \tup{\F{\posgenAel_1},\F{\posgenBel_1}}^*,\tup{\R{\posgenAel_2},\R{\posgenBel_2}}\right).
        \end{aligned}
    \end{equation}
    Furthermore, consider the design problems
    \begin{equation*}
        \adpa\colon \funposA_1 \profto \resposB_1,\quad \adpb\colon \funposA_2\profto \resposB_2, \quad \adpc\colon \funposB_1\profto \resposC_1,\quad \adpd\colon \funposB_2\profto \resposC_2.
    \end{equation*}
    We need to show that
    \begin{equation}
        \underbrace{\left( (\adpa\fthen \adpc) \mtimescat (\adpb\fthen \adpd)\right)}_{\star}=\left( (\adpa\mtimescat \adpb)\fthen (\adpc\mtimescat \adpd)\right).
    \end{equation}
    It holds that
    \begin{equation}
        \begin{aligned}
            ~ & \star \left( \tup{\F{\posgenAel_1},\F{\posgenAel_2}}\Fop,\tup{\R{\posgenCel_1},\R{\posgenCel_2}}\right) \\
              & = (\adpa\fthen \adpc)(\F{\posgenAel_1}\Fop,\R{\posgenCel_1})\booland (\adpa\fthen \adpd)(\F{\posgenAel_2}\Fop,\R{\posgenCel_2}) \\
              & =\left(\bigvee_{\posBel_\styleelements{1}\setin \posB_\stylepos{1}}\left( \adpa(\F{\posgenAel_1^*},\R{\posgenBel_1})\booland \adpc(\F{\posgenBel_1^*},\R{\posgenCel_1})\right)\right) \booland\left(\bigvee_{\posBel_\styleelements{2}\setin \posB_\stylepos{2}}\left( \adpb(\F{\posgenAel_2^*},\R{\posgenBel_2})\booland \adpd(\F{\posgenBel_2^*},\R{\posgenCel_2})\right)\right) \\
              & =\bigvee_{\posBel_\styleelements{1}\setin \posB_\stylepos{1}}\bigvee_{\posBel_\styleelements{2}\setin \posB_\stylepos{2}} \left(\adpa(\F{\posgenAel_1^*},\R{\posgenBel_1})\booland \adpc(\F{\posgenBel_1^*},\R{\posgenCel_1})\booland \adpa(\F{\posgenAel_2^*},\R{\posgenBel_2})\booland \adpd(\F{\posgenBel_2^*},\R{\posgenCel_2}) \right) \\
              & =\bigvee_{\tup{\posBel_\styleelements{1},\posBel_\stylepos{2}}\setin \posB_\stylepos{1}\cartprod \posB_\stylepos{2}} \left(\adpa(\F{\posgenAel_1^*},\R{\posgenBel_1})\booland \adpb(\F{\posgenAel_2^*},\R{\posgenBel_2})\booland \adpc(\F{\posgenBel_1^*},\R{\posgenCel_1})\booland \adpd(\F{\posgenBel_2^*},\R{\posgenCel_2}) \right) \\
              & =\left( (\adpa\mtimescat \adpb)\fthen (\adpc\mtimescat \adpd)\right)\left(\tup{\F{\posgenAel_1},\F{\posgenAel_2}}\Fop,\tup{\R{\posgenCel_1},\R{\posgenCel_2}} \right).
        \end{aligned}
    \end{equation}
    Therefore,~$\mtimescat$ is functorial in~\DP.
\end{proof}

\begin{lemma}
    $\tup{\DP,\mtimescat, \singleton}$ is a monoidal category.
\end{lemma}
\begin{proof}
    To show that~\DP is a monoidal category, we have to first identify the constituents presented in \cref{def:monoidal-cat}.
    First, recall~$\singleton$ to be singleton: this is the monoidal unit.
    In \cref{lem:monoidal_functorial} we have shown that~$\mtimescat$ is a functor.
    Furthermore, we identify
    \begin{itemize}
        \item $\leftunitor_\posA \colon \F{\singleton} \cartprod \funposA \profto \resposA$, for all~$\posA\setin \Ob_\DP$, is the left unitor.
              This is given by
              \begin{equation}
                  \leftunitor_\posA\left( \disunionA{\F{\posgenAel_1}}\Fop,\R{\posgenAel_2}\right)\definedas \F{\posgenAel_1}\posleqof\posA \R{\posgenAel_2}.
              \end{equation}
              To prove that this is an isomorphism, we define its inverse~$\leftunitor_\posA^{-1}\colon \funposA\profto \R{\singleton} \cartprod \resposA$ and show that~$\leftunitor_\posA\fthen \leftunitor_\posA^{-1}=\id_{\singleton\cartprod \posA}$ and~$\leftunitor_\posA^{-1}\then \leftunitor_\posA=\catid_{\posA}$.
              One has
              \begin{equation}
                  \begin{aligned}
                      ~ & \left( \leftunitor_\posA^{-1}\fthen \leftunitor_\posA\right)(\tup{\F{\posgenAel_1^*},\R{\posgenAel_2}}) \\
                        & =
                      \bigvee_{\tup{\singletonel,\posAel}\setin  \singleton\cartprod \posA} \leftunitor_\posA^{-1}(\F{\posgenAel_1^*},\tup{\R{\singletonel},\R{\posgenAel}})\booland \leftunitor_\posA(\tup{\F{\singletonel},\FposgenAel}\Fop,\R{\posgenAel_2}) \\
                        & = \bigvee_{\tup{\singletonel,\posAel}\setin  \singleton\cartprod \posA}(\F{\posgenAel_1}\posleq \R{\posgenAel}) \booland \FposgenAel\posleq \R{\posgenAel_2} \\
                        & =\F{\posgenAel_1}\posleq \R{\posgenAel_2} \\
                        & =\id_\posA(\F{\posgenAel_1^*},\R{\posgenAel_2}).
                  \end{aligned}
              \end{equation}
              Similarly, one can show that~$\leftunitor_\posA\fthen \leftunitor_\posA^{-1}=\id_{\singleton \cartprod \posA}$.
        \item $\rightunitor_\posA\colon \funposA \cartprod \R{\singleton} \profto \resposA$, for all~$\posA\setin \Ob_\DP$, is the right unitor.
              This is given by
              \begin{equation}
                  \rightunitor\left( \tup{\F{\posgenAel_1},\F{\singletonel}}\Fop,\R{\posgenAel_2}\right)\definedas \F{\posgenAel_1}\posleqof\posgenA \R{\posgenAel_2}.
              \end{equation}
              The proof that~$\rightunitor_\posgenA$ is an isomorphism is analogous to the one for~$\leftunitor_\posgenA$.
        \item $\associator_{\posgenA,\posgenB,\posgenC}\colon (\funposA\cartprod \funposB)\cartprod \funposC \profto \resposA\cartprod (\resposB\cartprod \resposC)$ for all~$\posgenA,\posgenB,\posgenC \setin \Ob_\DP$, is the associator.
              It is given by
              \begin{equation}
                  \begin{aligned}
                      \associator_{\posA,\posB,\posC} & (\tup{\tup{\F{\posgenAel_1},\F{\posgenBel_1}},\F{\posgenCel_1}}\Fop,\tup{\R{\posgenAel_2},\tup{\R{\posgenBel_2},\R{\posgenCel_2}}}) \\
                                                      & \definedas (\F{\posgenAel_1}\posleqof\posA \R{\posgenAel_2}) \booland (\F{\posgenBel_1} \posleqof\posB \R{\posgenBel_2})\booland (\F{\posgenCel_1}\posleqof\posC \R{\posgenCel_2}).
                  \end{aligned}
              \end{equation}
              To prove that~$\associator_{\posA,\posB,\posC}$ is an isomorphism, we first define its inverse
              \begin{equation}
                  \associator_{\posA,\posB,\posC}^{-1}\colon \funposA\cartprod (\funposB\cartprod \funposC) \profto (\resposA\cartprod \resposB)\cartprod \resposC
              \end{equation}
              and show~$\associator_{\posA,\posB,\posC}^{-1}\fthen \associator_{\posA,\posB,\posC}=\associator_{\posA,\posB,\posC}\fthen \associator_{\posA,\posB,\posC}^{-1}= \catid_{\posA\cartprod \posB\cartprod \posC}$.
              One has
              \begin{equation}
                  \begin{aligned}
                       & \left( \associator_{\posA,\posB,\posC}^{-1}\fthen \associator_{\posA,\posB,\posC} \right)(\tup{\F{\posgenAel_1},\tup{\F{\posgenBel_1},\F{\posgenCel_1}}}\Fop,\tup{\R{\posgenAel_2},\tup{\R{\posgenBel_2},\R{\posgenCel_2}}}) \\
                       & =\bigvee_{\tup{\tup{\posAel,\posBel},\posCel}\setin (\posA\cartprod \posB)\cartprod \posC}
                      \associator_{\posA,\posB,\posC}^{-1}(\tup{\F{\posgenAel_1},\tup{\F{\posgenBel_1},\F{\posgenCel_1}}}\Fop,\tup{\tup{\R{\posgenAel},\RposgenBel},\R{\posgenCel}})\booland \\
                       & \associator_{\posA,\posB,\posC}(\tup{\tup{\FposgenAel,\FposgenBel},\F{\posgenCel}}\Fop,\tup{\R{\posgenAel_2},\tup{\R{\posgenBel_2},\R{\posgenCel_2}}}) \\
                       & =\bigvee_{\tup{\tup{\posAel,\posBel},\posCel}\setin (\posA\cartprod \posB)\cartprod \posC}\left( (\F{\posgenAel_1}\posleq \R{\posgenAel}) \booland (\F{\posgenBel_1}\posleq \RposgenBel) \booland (\F{\posgenCel_1}\posleq \R{\posgenCel})\right)\booland \\
                       & \left((\FposgenAel\posleq \R{\posgenAel_2})\booland (\FposgenBel\posleq \R{\posgenBel_2}) \booland (\F{\posgenCel}\posleq \R{\posgenCel_2}\right) \\
                       & =(\F{\posgenAel_1}\posleq \R{\posgenAel_2}) \booland (\F{\posgenBel_1}\posleq \R{\posgenBel_2}) \booland (\F{\posgenCel_1}\posleq \R{\posgenCel_2}) \\
                       & =\catid_{\posA\cartprod \posB\cartprod \posC}(\tup{\F{\posgenAel_1},\F{\posgenBel_1},\F{\posgenCel_1}}\Fop,\tup{\R{\posgenAel_2},\R{\posgenBel_2},\R{b_3}}).
                  \end{aligned}
              \end{equation}
              The proof for~$\associator_{\posA,\posB,\posC}\fthen \associator_{\posA,\posB,\posC}^{-1}$ is analogous.
    \end{itemize}
    Therefore,~$\tup{\DP,\mtimescat, \singleton}$ is a monoidal category.
\end{proof}
