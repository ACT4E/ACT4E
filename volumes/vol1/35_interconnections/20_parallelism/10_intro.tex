% !TEX root = chapter-standalone.tex

\section{Modeling parallelism}
\label{sec:modeling-parallelism}

We have talked a lot about composition, and considered many examples.
However, the types of compositions we studied were, so far, mostly of the ``in series composition'' kind.
For instance, we considered the series composition of travel routes (\cref{exa:car-category}) and trekking routes (\cref{sec:trekking}), functions (\cref{sec:functions}) and relations (\cref{sec:connection-relations}), engineering component dependencies (\cref{sec:dependencies-design}) and \SY{Moore machines} (\cref{sec:moore-machines}), \etc

In this chapter, we will consider composition both in series and \emph{in parallel}.
For example, given functions~$\mora\colon \setA \mto \setB$ and~$\morb\colon \setB \mto \setC$, because the target set of the function~$\mora$ matches the source set of~$\morb$ they may be composed in series to obtain a function~$\morab\colon \setA \mto \setC$.
On the other hand, any two functions~$\moran{1} \colon \setA \mto \setB$ and~$\moran{2} \colon \setC \mto \setD$ may be composed ``in parallel'' by taking their \SY{cartesian product}: we obtain the function~$\moran{1} \funcprod \moran{2}\colon \setA \cartprod \setC \mto \setB \cartprod \setD$.
This parallel composition of~$\moran{1}$ and~$\moran{2}$ does not rely on any match-up of target and source sets, but it does rely on the ``additional structure'' provided by the \SY{cartesian product}.

Such ``additional structure'' will be formalized in this chapter using the notion of a \emph{monoidal structure}.

Composing components in parallel is of course a very familiar notion in engineering, and the mathematical concepts we develop here will, in particular, model parallel composition in this engineering sense.
In the context of co-design of complex systems, for example, we have seen that series composition corresponds to relating the functionalities of one component to the required resources of a next component.

\todographicsjira{185}{\alphubel: @Gioele: figure for parallelism 1/2}

Parallel composition, on the other hand, will correspond to taking two components and thinking of them as a single component whose functionality and resource space are given by the cartesian products of the respective constituent functionality and resource spaces of the original two components.

\todographicsjira{186}{\alphubel: @Gioele: figure for parallelism 2/2}

In general, a monoidal structure will be a notion of ``product'' and ``neutral element'' that a category may be equipped with, in which case such is called a \emph{monoidal category}.
One thing that could potentially be confusing at this point is the following.
At the beginning of this book, we studied \SY{monoids} as a basic kind of algebraic gadget whose composition operation (also called multiplication) was generalized to the series composition encoded in the definition of a category.
In this chapter, we will also use the basic pattern of a \SY{monoid} as inspiration, but now for parallel composition!
Thus, parallel composition is also ``monoid-like'', and hence the name \emph{monoidal structure}.

\subsection{Types of stacking operations}

There are various properties that we can consider for semicategories equipped with a stacking operation.
This leads to a number of definitions, each of which have a variant for the case when the semicategory in question is in fact a category.
Here is a short overview of the most important definitions.
%
\begin{enumerate}
    \item \emph{Stacking semicategory} (\cref{def:simple-stacking-semi-cat}): a \SY{semicategory} in which it is possible to stack two morphisms.
    \item \emph{Associative stacking semicategory} (\cref{def:assoc-stacking-semi-cat}): a \SY{semicategory} in which the stacking operation is \SY{associative} and makes objects and morphisms into a \SY{semigroup}.
    \item \emph{Monoidal stacking semicategory}: an associative stacking \SY{semicategory} in which there is a special object which is neutral for the stacking operation.
    \item \emph{Functorial stacking semicategory} (\cref{def:functorial-stacking-semi-cat}): a stacking \SY{semicategory} in which the stacking operation is also a \SY{functor}.
    \item \emph{Strict monoidal stacking semicategory} (\cref{def:strict-monoidal-semicat}): an associative stacking \SY{semicategory} which is both monoidal and functorial.
    \item \emph{Symmetric stacking semicategory}: a stacking \SY{semicategory} in which there is a series of symmetry operations that can permute the objects in a stack.
\end{enumerate}
%
In the next part we will talk about \emph{monoidal categories} (\cref{def:monoidal-cat}), which can be seen as a generalization of \SY{associative stacking} \SY{semicategories} where the stacking operation is not required to be \SY{associative} ``on the nose'', but rather only up to isomorphism.

\todotext{This list / overview needs to be updated}
\todotext{The figure is not PDF!}

\begin{figure*}[h!]
    \includegraphics[width=1.35\textwidth]{kinds-of-stacking.png}
\end{figure*}

\begin{remark}
    [Choices in definitions~$\star$]
    Of all these definitions, the only ``classical'' one is that of \emph{monoidal category}, and its specializations, such as \SY{strict monoidal category}, \SY{braided monoidal category}, and \SY{symmetric monoidal category}.
    We will look at all of those in~\cref{sec:parallelism-mon-cat}.

    We define the three non-classical notions mentioned above for the following reasons:
    \begin{itemize}
        \item Using the strictification construction of \SetL all the \SY{monoidal categories} that we need through the book are strict.
              Therefore, we can study parallel and feedback composition without knowing \SY{natural transformations}.

        \item There are important cases of \SY{semicategories} without identities with a stacking operation that do not fit the notion of \SY{monoidal category}.
              An example is discrete-time linear systems.
              For these semicategories we also want to consider appropriate traces.
        \item There are important examples in which there is a way to stack morphisms, but the stacking is not functorial.
              For example, we will describe a category of effectful computation with side effects.
    \end{itemize}
\end{remark}