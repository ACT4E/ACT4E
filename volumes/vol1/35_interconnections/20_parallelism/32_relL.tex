% !TEX root = chapter-standalone.tex

\section{The Category~$\cCat{\Rel}$}

\begin{definition}[The category~$\cCat{\Rel}$]
    \label{def:RelL}
    The category~$\cCat{\Rel}$ is:
    \begin{enumerate}
        \item \emph{Objects:} sets of tuples.
        \item \emph{Morphisms:}
              relations between sets of tuples.
        \item \emph{Composition:}
              Composition is the usual composition of relations.
        \item \emph{Identities:}
              The identity morphism on an object~$\cObj{\setAn{1},\ldots, \setAn{m}}$ is given by the identity relation~$\catid_{\cObj{\setAn{1},\ldots, \setAn{m}}}$.
    \end{enumerate}
\end{definition}

\begin{exercise}
    \label{ex:relassocstack}
    Consider the stacking operation on objects we have defined for \cCat{\Set}, and define a stacking operation on morphisms, such that \cCat{\Rel} is associative stacking.
\end{exercise}

\begin{solution}
    \todo{to write}
\end{solution}
\begin{exercise}
    \label{ex:relfuncstack}
    Prove that the structure defined in \cref{ex:relassocstack} makes \cCat{\Rel} functorial stacking.
\end{exercise}

\begin{solution}
    \todo{to write}
\end{solution}

