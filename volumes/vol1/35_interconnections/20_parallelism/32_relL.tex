% !TEX root = chapter-standalone.tex

\section{The category~\RelL}
We define a category~\RelL where the objects are sets of tuples (as in~\SetL).

\begin{definition}[The category~\RelL]
    \label{def:RelL}
    \SYNDEF{category of tuple-sets and relations}
    The category~\RelL is the \SY{subcategory} of~\Rel where the objects are sets of tuples.
    % \begin{enumerate}
    %     \item \emph{Objects:} sets of tuples (as in~\SetL)
    %     \item \emph{Morphisms:}
    %           relations between sets of tuples.
    %     \item \emph{Composition:}
    %           Composition is the usual composition of relations.
    %     \item \emph{Identities:}
    %           The identity morphism on an object~$\cObj{\setAn{1},\ldots, \setAn{m}}$ is given by the identity relation~$\catid_{\cObj{\setAn{1},\ldots, \setAn{m}}}$.
    % \end{enumerate}
\end{definition}

\begin{lemma}\label{lem:RelL-associative-stacking}
    \RelL is  \SY{associative stacking} with the structure induced by tuple concatenation.
\end{lemma}
\begin{proof}
    The multiplication $\mtimescatob$ is $\cprod$, the same as the one defined for $\SetL$.

    The multiplication $\mtimescatmor$ is defined as follows:
    \begin{equation}
        \prf{
            \relA\setsubseteq \setA \cartprod \setB
        }{
            \quad
        }{
            \relB\setsubseteq \setC\cartprod \setD
        }{
            \pars{\relA\mtimescatmor\relB} \setsubseteq
            \pars{\makecprod{\setA,\setC}} \cartprod
            \pars{\makecprod{\setB,\setD}}
        }
    \end{equation}
    where
    \begin{equation}
        \pars{
            \relA
            \mtimescatmor
            \relB
        }
        =
        \makesett{
            \tup{\setAeln{}\tupconcat \setCeln{},\setBeln{}\tupconcat \setDeln{}}
            \mid
            % \setAeln{} \setin \setA,
            % \setBeln{} \setin \setB,
            % \setCeln{} \setin \setC,
            % \setDeln{} \setin \setD:
            (\inrel{\setAeln{}}\relA{\setBeln{}})
            \booland
            (\inrel{\setCeln{}}\relB{\setDeln{}})
        }.
    \end{equation}
    The rest of the proof is left as an exercise.
\end{proof}

% What is a relation in this context?
% Given two sets of tuples~$\cObj{\setAn{1},\ldots, \setAn{m}}$ and~$\cObj{\setBn{1},\ldots, \setBn{n}}$, a relation between them is given by
% \begin{equation}
%     \relA \setsubseteq \cObj{\setAn{1},\ldots, \setAn{m}} \cartprod \cObj{\setBn{1},\ldots, \setBn{n}}.
% % \end{equation}
% We can define a multiplication operation for relations.
% Consider
% \begin{equation}
%     \begin{aligned}
%         \relA\colon \cObj{\setAn{1},\ldots, \setAn{m}} & \mtoin{\RelL}\cObj{\setBn{1},\ldots, \setBn{n}}, \\
%         \relB\colon \cObj{\setCn{1},\ldots, \setCn{o}} & \mtoin{\RelL}\cObj{\setDn{1},\ldots, \setDn{p}}.
%     \end{aligned}
% \end{equation}
% We have
% \begin{equation}
%     \relA\mtimescatmor \relB \colon \cObj{\setAn{1},\ldots, \setAn{m}}\cprod \cObj{\setCn{1},\ldots, \setCn{o}}\mtoin{\RelL}
%     \cObj{\setBn{1},\ldots, \setBn{n}}\cprod \cObj{\setDn{1},\ldots, \setDn{p}}.
% \end{equation}
% Specifically, the multiplication is given by:
% \begin{equation}
%     \label{eq:multiofrels}
%     \defmapperiod{
%         \mtimescatmor
%     }{
%         \Mor_{\RelL}\cartprod \Mor_{\RelL}
%     }{
%         \mto
%     }{
%         \Mor_{\RelL}
%     }{
%         \tup{\relA,\relB}
%     }{
%         \makeset{
%             \tup{\setAeln{}\tupconcat \setCeln{},\setBeln{}\tupconcat \setDeln{}}
%             \mid
%             (\inrel{\setAeln{}}\relA{\setBeln{}}) \booland (\inrel{\setCeln{}}\relB{\setDeln{}})
%         }
%     }
% \end{equation}

\begin{exercise}
    \label{ex:relassocstack}
    Consider the stacking operations on objects and on morphisms introduced in this section.
    Prove that~\RelL is \SY{associative stacking}.
\end{exercise}

\begin{solution}
    To prove the statement we first check that the stacking operations satisfy \cref{def:simple-stacking-semi-cat}, we then show that they are compatible, and finally show associativity.
    The stacking operation on objects was already checked for~\SetL.
    The stacking operation on morphisms clearly returns a valid relation.
    Furthermore, compatibility is satisfied:
    \begin{equation}
        \prfperiod{
            \relA\colon \cObj{\setAn{1},\ldots, \setAn{m}} \mtoin{\RelL}\cObj{\setBn{1},\ldots, \setBn{n}}
        }
        {
            \relB\colon \cObj{\setCn{1},\ldots, \setCn{o}} \mtoin{\RelL}\cObj{\setDn{1},\ldots, \setDn{p}}
        }
        {
            \relA\mtimescatmor \relB \colon \cObj{\setAn{1},\ldots, \setAn{m}}\cprod \cObj{\setCn{1},\ldots, \setCn{o}}\mtoin{\RelL}
            \cObj{\setBn{1},\ldots, \setBn{n}}\cprod \cObj{\setDn{1},\ldots, \setDn{p}}
        }
    \end{equation}
    Finally, associativity for the operation on objects was already shown for~\SetL.
    % The stacking operation on morphisms follows directly from \cref{eq:multiofrels}.
\end{solution}
\begin{gradedexercise}[\exname{RelFunStack}]
\label{ex:RelFunStack}
Prove that the structure defined in \cref{ex:relassocstack} makes~\RelL \SY{functorial stacking}.
\end{gradedexercise}

\solutionof{RelFunStack}

