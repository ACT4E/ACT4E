% !TEX root = chapter-standalone.tex

\section{The Category~$\cCat{\Rel}$}
We define a category~$\cCat{\Rel}$ where the objects are sets of tuples (as in~$\cCat{\Set}$).

What is a relation in this context?
Given two sets of tuples~$\cObj{\setAn{1},\ldots, \setAn{m}}$ and~$\cObj{\setBn{1},\ldots, \setBn{n}}$, a relation between them is given by
\begin{equation*}
    \relA \subseteq \cObj{\setAn{1},\ldots, \setAn{m}} \cartprod \cObj{\setBn{1},\ldots, \setBn{n}}.
\end{equation*}
We can define a multiplication operation for relations.
Consider
\begin{equation*}
    \begin{aligned}
        \relA\colon \cObj{\setAn{1},\ldots, \setAn{m}} & \mto_{\cCat{\Rel}}\cObj{\setBn{1},\ldots, \setBn{n}}, \\
        \relB\colon \cObj{\setCn{1},\ldots, \setCn{o}} & \mto_{\cCat{\Rel}}\cObj{\setDn{1},\ldots, \setDn{p}}.
    \end{aligned}
\end{equation*}
One has
\begin{equation*}
    \relA\mtimescatmor \relB \colon \cObj{\setAn{1},\ldots, \setAn{m}}\cprod \cObj{\setCn{1},\ldots, \setCn{o}}\mto_{\cCat{\Rel}}
    \cObj{\setBn{1},\ldots, \setBn{n}}\cprod \cObj{\setDn{1},\ldots, \setDn{p}}.
\end{equation*}
Specifically, the multiplication is given by:
\begin{equation*}
    \defmapperiod{\mtimescatmor}
    {\Mor_{\cCat{\Rel}}\cartprod \Mor_{\cCat{\Rel}}}
    {\mto}
    {\Mor_{\cCat{\Rel}}}
    {\tup{\relA,\relB}}
    {\{\tup{\setAeln{}\tupconcat \setCeln{},\setBeln{}\tupconcat \setDeln{}} \mid \tup{\setAeln{},\setBeln{}}\setin \relA \booland \tup{\setCeln{},\setDeln{}}\setin \relB\}}
\end{equation*}

\begin{definition}[The category~$\cCat{\Rel}$]
    \label{def:RelL}
    The category~$\cCat{\Rel}$ is:
    \begin{enumerate}
        \item \emph{Objects:} sets of tuples (as in~$\cCat{\Set}$)
        \item \emph{Morphisms:}
              relations between sets of tuples.
        \item \emph{Composition:}
              Composition is the usual composition of relations.
        \item \emph{Identities:}
              The identity morphism on an object~$\cObj{\setAn{1},\ldots, \setAn{m}}$ is given by the identity relation~$\catid_{\cObj{\setAn{1},\ldots, \setAn{m}}}$.
    \end{enumerate}
\end{definition}

\begin{exercise}
    \label{ex:relassocstack}
    Consider the stacking operation on objects we have defined for \cCat{\Set}, and define a stacking operation on morphisms, such that \cCat{\Rel} is associative stacking.
\end{exercise}

\begin{solution}
    \todo{to write}
\end{solution}
\begin{exercise}
    \label{ex:relfuncstack}
    Prove that the structure defined in \cref{ex:relassocstack} makes \cCat{\Rel} functorial stacking.
\end{exercise}

\begin{solution}
    \todo{to write}
\end{solution}

