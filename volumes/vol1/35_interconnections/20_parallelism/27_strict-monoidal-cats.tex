% !TEX root = chapter-standalone.tex

\section{Strict monoidal categories}
\label{sec:strict-monoidal-categories}

The following definition is analogous to the definition of a functorial stacking semicategory; the one difference is that we start with a functorical stacking category.

\begin{ctdefinition}
    \label{def:strict-monoidal-category}
    A \maindef{strict monoidal category} is a \SY{strict monoidal stacking category} which is a \SY{functorial stacking category} $\tup{\CatC, \mtimescat}$.

    Unpacked somewhat, this means there is a stacking functor
    \begin{equation}
        \mtimescat \colon \CatC \cartprod \CatC \fto \CatC
    \end{equation}
    and an object $\idmoncat \setin \Obof{\CatC}$, the \emph{monoidal unit}, satisfying
    \begin{equation}
        \Obja \mtimescat \idmoncat = \Obja \qquad \text{and} \qquad \idmoncat \mtimescat \Obja = \Obja
    \end{equation}
    and
    \begin{equation}
        \mora \mtimescat \catid_\idmoncat = \mora \qquad \text{and} \qquad \catid_\idmoncat \mtimescat \mora = \mora.
    \end{equation}
    for any object $\Obja$ and any morphism $\mora$ in $\CatC$.
\end{ctdefinition}

\todotext{give example that <Set> and its cousings are strict monoidal categories}

\todotext{@JL: make/include more examples of strict monoidal categories... there are quite a few nice ones i think... scour Baez materials...}

