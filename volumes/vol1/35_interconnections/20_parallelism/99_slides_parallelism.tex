% !TEX root = chapter-standalone.tex

\section[For slides]{For slides}

\begin{forslides}

stacking sets and functions with product
    \begin{equation}
        \label{eq:parallelism-000}
        \setA \mtimescatob \setB = \setA \cartprod \setB
    \end{equation}

    \begin{equation}
        \label{eq:parallelism-001}
        \mora \mtimescatmor \morb = \mora \catprodmor \morb
    \end{equation}

    \begin{equation}
        \label{eq:parallelism-002}
        \setA 
    \end{equation}

    \begin{equation}
        \label{eq:parallelism-003}
        \setB 
    \end{equation}

    \begin{equation}
        \label{eq:parallelism-004}
       \setA \cartprod \setB
    \end{equation}

    \begin{equation}
        \label{eq:parallelism-005}
        \mora \colon \setA \sto \setB 
    \end{equation}

    \begin{equation}
        \label{eq:parallelism-006}
        \morb \colon \setC \sto \setD 
    \end{equation}

    \begin{equation}
        \label{eq:parallelism-007}
        \mora \catprodmor \morb \colon \setA \cartprod \setC \sto \setB \cartprod \setD 
    \end{equation}

    \begin{equation}
        \label{eq:parallelism-008}
       \setC
    \end{equation}

    \begin{equation}
        \label{eq:parallelism-009}
         \setD
    \end{equation}

    \begin{equation}
        \label{eq:parallelism-010}
        \setC \cartprod \setD
    \end{equation}

    \begin{equation}
        \label{eq:parallelism-011}
       \mora
    \end{equation}

    \begin{equation}
        \label{eq:parallelism-012}
        \morb
    \end{equation}

\begin{equation}
        \label{eq:parallelism-012b}
        \mora \catprodmor \morb
    \end{equation}


stacking sets and functions with sum
    \begin{equation}
        \label{eq:parallelism-013}
       \setA \mtimescatob \setB = \setA \setdisunion \setB
    \end{equation}

    \begin{equation}
        \label{eq:parallelism-014}
        \mora \mtimescatmor \morb = \mora \funcsum \morb
    \end{equation}
    
    \begin{equation}
        \label{eq:parallelism-015}
         \setA \setdisunion \setB
    \end{equation}

    \begin{equation}
        \label{eq:parallelism-016}
       \setC \setdisunion \setC
    \end{equation}

    \begin{equation}
        \label{eq:parallelism-017}
         \mora \funcsum \morb
    \end{equation}

    \begin{equation}
        \label{eq:parallelism-018}
       \mora \funcsum \morb \colon \setA \setdisunion \setB \sto \setC \setdisunion \setD
    \end{equation}

stacking posets with product
    \begin{equation}
        \label{eq:parallelism-019}
        \posA \mtimescatob \posB = \posA \Ptimes \posB
    \end{equation}

    \begin{equation}
        \label{eq:parallelism-020}
        \mora \mtimescatmor \morb = \mora \catprodmor \morb
    \end{equation}

    \begin{equation}
        \label{eq:parallelism-021}
        \mora \colon \posA \mto \posB
    \end{equation}

    \begin{equation}
        \label{eq:parallelism-022}
        \morb \colon \posC \mto \posD
    \end{equation}

    \begin{equation}
        \label{eq:parallelism-023}
        \mora \funcprod \morb \colon \posA \Ptimes \posB \mto \posC \Ptimes \posD
    \end{equation}

stacking matrices direct sum
    \begin{equation}
        \label{eq:parallelism-024}
        \mat{A}
    \end{equation}

    \begin{equation}
        \label{eq:parallelism-025}
        \mat{B}
    \end{equation}

    \begin{equation}
        \label{eq:parallelism-026}
        \mat{A} \colon \styleobj{k} \mto \styleobj{l}
    \end{equation}

    \begin{equation}
        \label{eq:parallelism-027}
        \mat{B} \colon \styleobj{m} \mto \styleobj{n}
    \end{equation}

    \begin{equation}
        \label{eq:parallelism-028}
        \begin{bmatrix}
        \mat{A} & \mat{0} \\
        \mat{0} & \mat{B} 
        \end{bmatrix}
        \colon \styleobj{k} \catsumob \styleobj{m} \mto \styleobj{l} \catsumob \styleobj{n}
    \end{equation}
    
    \begin{equation}
        \label{eq:parallelism-029}
        \begin{bmatrix}
        \mat{A} & \mat{0} \\
        \mat{0} & \mat{B} 
        \end{bmatrix}
    \end{equation}

    \begin{equation}
        \label{eq:parallelism-030}
        \styleobj{k} \catsumob \styleobj{m}
    \end{equation}

    \begin{equation}
        \label{eq:parallelism-031}
        \styleobj{l} \catsumob \styleobj{n}
    \end{equation}

    \begin{equation}
        \label{eq:parallelism-032}
        \styleobj{k}
    \end{equation}

    \begin{equation}
        \label{eq:parallelism-033}
        \styleobj{m}
    \end{equation}

    \begin{equation}
        \label{eq:parallelism-034}
        \styleobj{l}
    \end{equation}

    \begin{equation}
        \label{eq:parallelism-035}
       \styleobj{n}
    \end{equation}

    \begin{equation}
        \label{eq:parallelism-036}
         \mat{A} \mtimescatmor \mat{B} = \mat{A} \oplus \mat{B}
    \end{equation}

    \begin{equation}
        \label{eq:parallelism-037}
       \styleobj{k} \mtimescatob \styleobj{m} = \styleobj{k} \catsumob \styleobj{m}
    \end{equation}

stacking matrices tensor
    \begin{equation}
        \label{eq:parallelism-038}
        \mat{A} \mtimescatmor \mat{B} = \mat{A} \otimes \mat{B}
    \end{equation}

    \begin{equation}
        \label{eq:parallelism-039}
       \styleobj{k} \mtimescatob \styleobj{m} = \styleobj{k} \catprodob \styleobj{m}
    \end{equation}

    \begin{equation}
        \label{eq:parallelism-040}
       \mat{A} \otimes \mat{B} \colon \styleobj{k} \catprodob \styleobj{m} \mto \styleobj{l} \catprodob \styleobj{n}
    \end{equation}

    \begin{equation}
        \label{eq:parallelism-041}
        \mat{A} \otimes \mat{B} = 
        \begin{bmatrix}
        a_{ij} \mat{B}
        \end{bmatrix}_{i,j}
    \end{equation}

    \begin{equation}
        \label{eq:parallelism-042}
         \styleobj{k} \catprodob \styleobj{m}
    \end{equation}

    \begin{equation}
        \label{eq:parallelism-043}
        \styleobj{l} \catprodob \styleobj{n}
    \end{equation}

set with cartesian product is not associative stacking
    \begin{equation}
        \label{eq:parallelism-044}
        (\setA \cartprod \setB) \cartprod \setC \neq \setA \cartprod (\setB \cartprod \setC)
    \end{equation}

    \begin{equation}
        \label{eq:parallelism-045}
        \tup{\tup{\ela, \elb}, \elc} 
    \end{equation}

    \begin{equation}
        \label{eq:parallelism-046}
        \tup{\ela, \tup{\elb, \elc}} 
    \end{equation}

    \begin{equation}
        \label{eq:parallelism-047}
        (\setA \cartprod \setB) \cartprod \setC
    \end{equation}

    \begin{equation}
        \label{eq:parallelism-048}
        \setA \cartprod (\setB \cartprod \setC)
    \end{equation}

    \begin{equation}
        \label{eq:parallelism-049}
        \SetL
    \end{equation}

    \begin{equation}
        \label{eq:parallelism-050}
        \PosL
    \end{equation}

Mat is associative stacking
    \begin{equation}
        \label{eq:parallelism-051}
        ( \mat{A} \mtimescatmor \mat{B} ) \mtimescatmor \mat{C}
    \end{equation}

    \begin{equation}
        \label{eq:parallelism-052}
        \mat{A} \mtimescatmor ( \mat{B} \mtimescatmor \mat{C} )
    \end{equation}

    \begin{equation}
        \label{eq:parallelism-053}
        \begin{bmatrix}
        \mat{A} & \mat{0} \\
        \mat{0} & \mat{B}
        \end{bmatrix}
        \mtimescatmor \mat{C}
    \end{equation}

    \begin{equation}
        \label{eq:parallelism-054}
        \mat{A} \mtimescatmor 
        \begin{bmatrix}
        \mat{B} & \mat{0} \\
        \mat{0} & \mat{C}
        \end{bmatrix}
    \end{equation}

    \begin{equation}
        \label{eq:parallelism-055}
        \begin{bmatrix}
        \mat{A} & \mat{0} & \mat{0}  \\
        \mat{0} & \mat{B} & \mat{0} \\
        \mat{0} & \mat{0} & \mat{C}
        \end{bmatrix}
    \end{equation}

    \begin{equation}
        \label{eq:parallelism-056}
        \Snacks \mfrom  \Participants \mto \Drinks
    \end{equation}

    \begin{equation}
        \label{eq:parallelism-057}
        \catprodphi_{\eats,\drinks}
    \end{equation}

    \begin{equation}
        \label{eq:parallelism-058}
        \Obja \mfrom \text{``product of } \Obja \text{ and } \Objb \text{''}  \mto \Objb
    \end{equation}

    \begin{equation}
        \label{eq:parallelism-059}
        \Obja \mfrom \styleobj{T} \mto \Objb
    \end{equation}

    \begin{equation}
        \label{eq:parallelism-060}
        \setA
    \end{equation}

    \begin{equation}
        \label{eq:parallelism-061}
        \setB
    \end{equation}

    \begin{equation}
        \label{eq:parallelism-062}
        \setA \setdisunion \setB
    \end{equation}

    \begin{equation}
        \label{eq:parallelism-063}
        \reals
    \end{equation}

    \begin{equation}
        \label{eq:parallelism-064}
        t \setin \reals
    \end{equation}

    \begin{equation}
        \label{eq:parallelism-065}
        t \geq \elna{1}
    \end{equation}

    \begin{equation}
        \label{eq:parallelism-066}
        t \geq \elna{2}
    \end{equation}

    \begin{equation}
        \label{eq:parallelism-067}
        t \geq \max \makeset{ \elna{1}, \elna{2} }
    \end{equation}

    \begin{equation}
        \label{eq:parallelism-068}
        \subAn{1}, \subAn{2} \setsubseteq \setA
    \end{equation}

    \begin{equation}
        \label{eq:parallelism-069}
        \subAn{1} \to \subAn{2}
    \end{equation}

    \begin{equation}
        \label{eq:parallelism-070}
        \subAn{1} \setsubseteq \subAn{2}
    \end{equation}

    \begin{equation}
        \label{eq:parallelism-071}
        \setA = \makeset{ \sbretzel, \sfondue, \schoco, \sburger }
    \end{equation}

    \begin{equation}
        \label{eq:parallelism-071b}
        \subAn{1} = \makeset{ \sbretzel, \sfondue, \schoco }
    \end{equation}

    \begin{equation}
        \label{eq:parallelism-072}
        \subAn{2} = \makeset{\sfondue, \schoco, \sburger }
    \end{equation}

    \begin{equation}
        \label{eq:parallelism-073}
        \subAn{1} \setunion \subAn{2} = \setA
    \end{equation}

    \begin{equation}
        \label{eq:parallelism-074}
        \setC = \makeset{ \sfondue }
    \end{equation}

    \begin{equation}
        \label{eq:parallelism-075}
        \setC \setsubseteq \subAn{1}
    \end{equation}

    \begin{equation}
        \label{eq:parallelism-076}
        \setC \setsubseteq \subAn{2}
    \end{equation}
    
\begin{comment}
    \begin{equation}
        \label{eq:parallelism-077}
        \setC \setsubseteq \subAn{1} \setunion \subAn{2}
    \end{equation}

    \begin{equation}
        \label{eq:parallelism-078}
        \setC = \makeset{ \sbretzel }
    \end{equation}

    \begin{equation}
        \label{eq:parallelism-079}
        T = \Emptyset
    \end{equation}

    \begin{equation}
        \label{eq:parallelism-080}
        \setA = \makeset{ \true, \false }
    \end{equation}

    \begin{equation}
        \label{eq:parallelism-081}
        \ela, \elb \setin \setA
    \end{equation}

    \begin{equation}
        \label{eq:parallelism-082}
        \ela \to \elb
    \end{equation}

    \begin{equation}
        \label{eq:parallelism-083}
        \prfperiod{\ela}{\elb}
    \end{equation}

    \begin{equation}
        \label{eq:parallelism-084}
        \ela \Rightarrow \elb
    \end{equation}

    \begin{equation}
        \label{eq:parallelism-085}
        \coprodMapob{\Obja}{\Objb}
    \end{equation}

    \begin{equation}
        \label{eq:parallelism-086}
        \Objc
    \end{equation}

    \begin{equation}
        \label{eq:parallelism-087}
        \coprodMapmor{\mora}{\morb}
    \end{equation}

    \begin{equation}
        \label{eq:parallelism-088}
        \catcoprodpsi_{\mora, \morb}
    \end{equation}

    \begin{equation}
        \label{eq:parallelism-089}
        \setA=\makeset{\technology{LiPo}, \technology{LCO},\technology{NiH2}}
    \end{equation}

    \begin{equation}
        \label{eq:parallelism-090}
        \setB=\makeset{\technology{LFP},\technology{LMO},\technology{LiPo}}
    \end{equation}

    \begin{equation}
        \label{eq:parallelism-091}
        \setC=\makeset{\unit[50]{},\unit[60]{},\unit[70]{},\unit[80]{}}\cartprod \makeset{\text{\standardcurrency}}
    \end{equation}

    \begin{equation}
        \label{eq:parallelism-092}
        \mapa \colon \setA\to \setC
    \end{equation}

    \begin{equation}
        \label{eq:parallelism-093}
        \mapb\colon \setB \sto \setC
    \end{equation}

    \begin{equation}
        \label{eq:parallelism-094}
        \setA \setdisunion \setB=
        \makeset{
            \disunionA{\technology{LiPo}},
            \disunionA{\technology{LCO}},
            \disunionA{\technology{NiH2}},
            \disunionB{\technology{LFP}},
            \disunionB{\technology{LMO}},
            \disunionB{\technology{LiPo}}
        }
    \end{equation}

    \begin{equation}
        \label{eq:parallelism-095}
        \setA\setdisunion\setB
    \end{equation}

    \begin{equation}
        \label{eq:parallelism-096}
        \setA
    \end{equation}

    \begin{equation}
        \label{eq:parallelism-097}
        \setB
    \end{equation}

    \begin{equation}
        \label{eq:parallelism-098}
        \begin{aligned}
            \inj_\setA\colon \setA & \sto \setA\setdisunion\setB \\
            \setAel                & \mapsto \disunionA{\setAel}
        \end{aligned} \end{equation}

    \begin{equation}
        \label{eq:parallelism-099}
        \begin{aligned}
            \inj_\setB\colon \setB & \sto \setA\setdisunion\setB \\
            \setBel                & \mapsto \disunionB{\setBel}.
        \end{aligned}    \end{equation}

    \begin{equation}
        \label{eq:parallelism-100}
        \coprodMapmor{\mapa}{\mapb}\colon \setA \setdisunion \setB \sto \setC
    \end{equation}

    \begin{equation}
        \label{eq:parallelism-101}
        \inj_\setA\mthen (\mapa \funcsum \mapb)=\mapa
    \end{equation}

    \begin{equation}
        \label{eq:parallelism-102}
        \inj_\setB\mthen (\mapa \funcsum \mapb)=\mapb
    \end{equation}

    \begin{equation}
        \label{eq:parallelism-103}
        \styleelements{x}\setin \setA \setdisunion \setB
    \end{equation}

    \begin{equation}
        \label{eq:parallelism-104}
        \exists \setAel\setin \setA\colon \styleelements{x}=\inj_\setA(\setAel)
    \end{equation}

    \begin{equation}
        \label{eq:parallelism-105}
        \exists \setBel\setin \setB\colon \styleelements{x}=\inj_\setB(\setBel)
    \end{equation}

    \begin{equation}
        \label{eq:parallelism-106}
        \coprodMapmor{\mapa}{\mapb}
    \end{equation}

    \begin{equation}
        \label{eq:parallelism-107}
        \begin{aligned}
            \coprodMapmor{\mapa}{\mapb} \colon  \setA \setdisunion \setB & \to \setC \\
            \styleelements{x}                                            & \mapsto
            \begin{cases}
                \mapa(\styleelements{x}), & \text{if } \styleelements{x}=\inj_\setA(\setAel),\quad \setAel \setin \setA, \\
                \mapb(\styleelements{x}), & \text{if } \styleelements{x}=\inj_\setB(\setBel),\quad \setBel \setin \setB.
            \end{cases}
        \end{aligned}
    \end{equation}

    \begin{equation}
        \label{eq:parallelism-108}
        \setA\setunion \setB
    \end{equation}

    \begin{equation}
        \label{eq:parallelism-109}
        \mapa
    \end{equation}

    \begin{equation}
        \label{eq:parallelism-110}
        \mapb
    \end{equation}

    \begin{equation}
        \label{eq:parallelism-111}
        \technology{LiPo}\setin \setA\setunion \setB
    \end{equation}

    \begin{equation}
        \label{eq:parallelism-112}
        \mapa(\technology{LiPo})=\unit[50]{CHF}
    \end{equation}

    \begin{equation}
        \label{eq:parallelism-113}
        \mapb(\technology{LiPo})=\unit[60]{CHF}
    \end{equation}

    \begin{equation}
        \label{eq:parallelism-114}
        \coprodMapmor{\mapa}{\mapb}
    \end{equation}

    \begin{equation}
        \label{eq:parallelism-115}
        \setA\setintersection \setB\neq \Emptyset
    \end{equation}

    \begin{equation}
        \label{eq:parallelism-116}
        \styleelements{x}\setin \setA\setintersection \setB
    \end{equation}

    \begin{equation}
        \label{eq:parallelism-117}
        \mapa(\styleelements{x})
    \end{equation}

    \begin{equation}
        \label{eq:parallelism-118}
        \mapb(\styleelements{x})
    \end{equation}

    \begin{equation}
        \label{eq:parallelism-119}
        \setA,\setB\setin \Obof\Rel
    \end{equation}

    \begin{equation}
        \label{eq:parallelism-120}
        \relA
    \end{equation}

    \begin{equation}
        \label{eq:parallelism-121}
        \setA \setdisunion \setB
    \end{equation}

    \begin{equation}
        \label{eq:parallelism-122}
        \inj_\setA\colon \setA\mto \setA \setdisunion\setB
    \end{equation}

    \begin{equation}
        \label{eq:parallelism-123}
        \inj_\setB\colon \setB\mto \setA \setdisunion\setB
    \end{equation}

    \begin{equation}
        \label{eq:parallelism-124}
    \end{equation}

    \begin{equation}
        \label{eq:parallelism-125}
        m,n\setin \natnumbers
    \end{equation}

    \begin{equation}
        \label{eq:parallelism-126}
        m\to n
    \end{equation}

    \begin{equation}
        \label{eq:parallelism-127}
        m | n
    \end{equation}

    \begin{equation}
        \label{eq:parallelism-128}
        6\to 12
    \end{equation}

    \begin{equation}
        \label{eq:parallelism-129}
        6 | 12
    \end{equation}

    \begin{equation}
        \label{eq:parallelism-130}
        \realswithleq
    \end{equation}

    \begin{equation}
        \label{eq:parallelism-131}
        \elna{1},\elna{2}\setin \reals
    \end{equation}

    \begin{equation}
        \label{eq:parallelism-132}
        \elna{1}\to \elna{2}
    \end{equation}

    \begin{equation}
        \label{eq:parallelism-133}
        \elna{1}\leq \elna{2}
    \end{equation}

    \begin{equation}
        \label{eq:parallelism-134}
        \elna{1}
    \end{equation}

    \begin{equation}
        \label{eq:parallelism-135}
        \elna{2}
    \end{equation}

    \begin{equation}
        \label{eq:parallelism-136}
        z\setin \reals
    \end{equation}

    \begin{equation}
        \label{eq:parallelism-137}
        \elna{1}\leq \elb
    \end{equation}

    \begin{equation}
        \label{eq:parallelism-138}
        \elna{2}\leq \elb
    \end{equation}

    \begin{equation}
        \label{eq:parallelism-139}
        \ela\setin \reals
    \end{equation}

    \begin{equation}
        \label{eq:parallelism-140}
        \elna{1}\leq \ela
    \end{equation}

    \begin{equation}
        \label{eq:parallelism-141}
        \elna{2}\leq \ela
    \end{equation}

    \begin{equation}
        \label{eq:parallelism-142}
        \elb\leq \ela
    \end{equation}

    \begin{equation}
        \label{eq:parallelism-143}
        \max\makeset{\elna{1},\elna{2}}
    \end{equation}

    \begin{equation}
        \label{eq:parallelism-144}
        \stylesets{S}
    \end{equation}

    \begin{equation}
        \label{eq:parallelism-145}
        \setA,\setB\setsubseteq \stylesets{S}
    \end{equation}

    \begin{equation}
        \label{eq:parallelism-146}
        \setA \sto \setB
    \end{equation}

    \begin{equation}
        \label{eq:parallelism-147}
        \setA\setsubseteq \setB
    \end{equation}

    \begin{equation}
        \label{eq:parallelism-148}
        \setA\setunion\setB
    \end{equation}

    \begin{equation}
        \label{eq:parallelism-149}
        \Obja \mto \text{``coproduct of } \Obja \text{ and } \Objb \text{''}  \mfrom \Objb
    \end{equation}

    \begin{equation}
        \label{eq:parallelism-150}
        \Obja \mto \styleobj{T} \mfrom \Objb
    \end{equation}

    \begin{equation}
        \label{eq:parallelism-151}
        \setC
    \end{equation}

    \begin{equation}
        \label{eq:parallelism-152}
        \proj_1\colon \setC \sto \setA
    \end{equation}

    \begin{equation}
        \label{eq:parallelism-153}
        \proj_2\colon \setC \sto \setB
    \end{equation}

    \begin{equation}
        \label{eq:parallelism-154}
        \proj_1
    \end{equation}

    \begin{equation}
        \label{eq:parallelism-155}
        \proj_2
    \end{equation}

    \begin{equation}
        \label{eq:parallelism-156}
        \makeset{ \sbretzel }
    \end{equation}

    \begin{equation}
        \label{eq:parallelism-157}
        \makeset{ \schoco }
    \end{equation}

    \begin{equation}
        \label{eq:parallelism-158}
        \makeset{ \sbretzel, \schoco }
    \end{equation}

    \begin{equation}
        \label{eq:parallelism-159}
        \makeset{ \sbretzel, \schoco, \sburger }
    \end{equation}

    \begin{equation}
        \label{eq:parallelism-160}
        \setA
    \end{equation}

    \begin{equation}
        \label{eq:parallelism-161}
        \setA
    \end{equation}

    \begin{equation}
        \label{eq:parallelism-162}
        \setA
    \end{equation}

    \begin{equation}
        \label{eq:parallelism-163}
        \setA
    \end{equation}

    \begin{equation}
        \label{eq:parallelism-164}
        \setA
    \end{equation}

    \begin{equation}
        \label{eq:parallelism-165}
        \setA
    \end{equation}

    \begin{equation}
        \label{eq:parallelism-166}
        \setA
    \end{equation}

    \begin{equation}
        \label{eq:parallelism-167}
        \setA
    \end{equation}

    \begin{equation}
        \label{eq:parallelism-168}
        \setA
    \end{equation}

    \begin{equation}
        \label{eq:parallelism-169}
        \setA
    \end{equation}

    \begin{equation}
        \label{eq:parallelism-170}
        \setA
    \end{equation}

    \begin{equation}
        \label{eq:parallelism-171}
        \setA
    \end{equation}

    \begin{equation}
        \label{eq:parallelism-172}
        \setA
    \end{equation}

    \begin{equation}
        \label{eq:parallelism-173}
        \setA
    \end{equation}

    \begin{equation}
        \label{eq:parallelism-174}
        \setA
    \end{equation}

    \begin{equation}
        \label{eq:parallelism-175}
        \setA
    \end{equation}

    \begin{equation}
        \label{eq:parallelism-176}
        \setA
    \end{equation}

    \begin{equation}
        \label{eq:parallelism-177}
        \setA
    \end{equation}

    \begin{equation}
        \label{eq:parallelism-178}
        \setA
    \end{equation}

    \begin{equation}
        \label{eq:parallelism-179}
        \setA
    \end{equation}

    \begin{equation}
        \label{eq:parallelism-180}
        \setA
    \end{equation}

    \begin{equation}
        \label{eq:parallelism-181}
        \setA
    \end{equation}

    \begin{equation}
        \label{eq:parallelism-182}
        \setA
    \end{equation}

    \begin{equation}
        \label{eq:parallelism-183}
        \setA
    \end{equation}

    \begin{equation}
        \label{eq:parallelism-184}
        \setA
    \end{equation}

    \begin{equation}
        \label{eq:parallelism-185}
        \setA
    \end{equation}

    \begin{equation}
        \label{eq:parallelism-186}
        \setA
    \end{equation}

    \begin{equation}
        \label{eq:parallelism-187}
        \setA
    \end{equation}

    \begin{equation}
        \label{eq:parallelism-188}
        \setA
    \end{equation}

    \begin{equation}
        \label{eq:parallelism-189}
        \setA
    \end{equation}
\end{comment}
\end{forslides}