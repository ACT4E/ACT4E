% !TEX root = chapter-standalone.tex

\section[For slides]{For slides}

\begin{forslides}

    stacking sets and functions with product
    \begin{equation}
        \label{eq:parallelism-000}
        \setA \mtimescatob \setB = \setA \cartprod \setB
    \end{equation}

    \begin{equation}
        \label{eq:parallelism-001}
        \mora \mtimescatmor \morb = \mora \catprodmor \morb
    \end{equation}

    \begin{equation}
        \label{eq:parallelism-002}
        \setA
    \end{equation}

    \begin{equation}
        \label{eq:parallelism-003}
        \setB
    \end{equation}

    \begin{equation}
        \label{eq:parallelism-004}
        \setA \cartprod \setB
    \end{equation}

    \begin{equation}
        \label{eq:parallelism-005}
        \mora \colon \setA \sto \setB
    \end{equation}

    \begin{equation}
        \label{eq:parallelism-006}
        \morb \colon \setC \sto \setD
    \end{equation}

    \begin{equation}
        \label{eq:parallelism-007}
        \mora \catprodmor \morb \colon \setA \cartprod \setC \sto \setB \cartprod \setD
    \end{equation}

    \begin{equation}
        \label{eq:parallelism-008}
        \setC
    \end{equation}

    \begin{equation}
        \label{eq:parallelism-009}
        \setD
    \end{equation}

    \begin{equation}
        \label{eq:parallelism-010}
        \setC \cartprod \setD
    \end{equation}

    \begin{equation}
        \label{eq:parallelism-011}
        \mora
    \end{equation}

    \begin{equation}
        \label{eq:parallelism-012}
        \morb
    \end{equation}

    \begin{equation}
        \label{eq:parallelism-012b}
        \mora \catprodmor \morb
    \end{equation}

    stacking sets and functions with sum
    \begin{equation}
        \label{eq:parallelism-013}
        \setA \mtimescatob \setB = \setA \setdisunion \setB
    \end{equation}

    \begin{equation}
        \label{eq:parallelism-014}
        \mora \mtimescatmor \morb = \mora \funcsum \morb
    \end{equation}

    \begin{equation}
        \label{eq:parallelism-015}
        \setA \setdisunion \setB
    \end{equation}

    \begin{equation}
        \label{eq:parallelism-016}
        \setC \setdisunion \setC
    \end{equation}

    \begin{equation}
        \label{eq:parallelism-017}
        \mora \funcsum \morb
    \end{equation}

    \begin{equation}
        \label{eq:parallelism-018}
        \mora \funcsum \morb \colon \setA \setdisunion \setB \sto \setC \setdisunion \setD
    \end{equation}

    stacking posets with product
    \begin{equation}
        \label{eq:parallelism-019}
        \posA \mtimescatob \posB = \posA \Ptimes \posB
    \end{equation}

    \begin{equation}
        \label{eq:parallelism-020}
        \mora \mtimescatmor \morb = \mora \catprodmor \morb
    \end{equation}

    \begin{equation}
        \label{eq:parallelism-021}
        \mora \colon \posA \mto \posB
    \end{equation}

    \begin{equation}
        \label{eq:parallelism-022}
        \morb \colon \posC \mto \posD
    \end{equation}

    \begin{equation}
        \label{eq:parallelism-023}
        \mora \funcprod \morb \colon \posA \Ptimes \posB \mto \posC \Ptimes \posD
    \end{equation}

    stacking matrices direct sum
    \begin{equation}
        \label{eq:parallelism-024}
        \mat{A}
    \end{equation}

    \begin{equation}
        \label{eq:parallelism-025}
        \mat{B}
    \end{equation}

    \begin{equation}
        \label{eq:parallelism-026}
        \mat{A} \colon \styleobj{k} \mto \styleobj{l}
    \end{equation}

    \begin{equation}
        \label{eq:parallelism-027}
        \mat{B} \colon \styleobj{m} \mto \styleobj{n}
    \end{equation}

    \begin{equation}
        \label{eq:parallelism-028}
        \begin{bmatrix}
            \mat{A} & \mat{0} \\
            \mat{0} & \mat{B}
        \end{bmatrix}
        \colon \styleobj{k} \catsumob \styleobj{m} \mto \styleobj{l} \catsumob \styleobj{n}
    \end{equation}

    \begin{equation}
        \label{eq:parallelism-029}
        \begin{bmatrix}
            \mat{A} & \mat{0} \\
            \mat{0} & \mat{B}
        \end{bmatrix}
    \end{equation}

    \begin{equation}
        \label{eq:parallelism-030}
        \styleobj{k} \catsumob \styleobj{m}
    \end{equation}

    \begin{equation}
        \label{eq:parallelism-031}
        \styleobj{l} \catsumob \styleobj{n}
    \end{equation}

    \begin{equation}
        \label{eq:parallelism-032}
        \styleobj{k}
    \end{equation}

    \begin{equation}
        \label{eq:parallelism-033}
        \styleobj{m}
    \end{equation}

    \begin{equation}
        \label{eq:parallelism-034}
        \styleobj{l}
    \end{equation}

    \begin{equation}
        \label{eq:parallelism-035}
        \styleobj{n}
    \end{equation}

    \begin{equation}
        \label{eq:parallelism-036}
        \mat{A} \mtimescatmor \mat{B} = \mat{A} \oplus \mat{B}
    \end{equation}

    \begin{equation}
        \label{eq:parallelism-037}
        \styleobj{k} \mtimescatob \styleobj{m} = \styleobj{k} \catsumob \styleobj{m}
    \end{equation}

    stacking matrices tensor
    \begin{equation}
        \label{eq:parallelism-038}
        \mat{A} \mtimescatmor \mat{B} = \mat{A} \otimes \mat{B}
    \end{equation}

    \begin{equation}
        \label{eq:parallelism-039}
        \styleobj{k} \mtimescatob \styleobj{m} = \styleobj{k} \catprodob \styleobj{m}
    \end{equation}

    \begin{equation}
        \label{eq:parallelism-040}
        \mat{A} \otimes \mat{B} \colon \styleobj{k} \catprodob \styleobj{m} \mto \styleobj{l} \catprodob \styleobj{n}
    \end{equation}

    \begin{equation}
        \label{eq:parallelism-041}
        \mat{A} \otimes \mat{B} =
        \begin{bmatrix}
            a_{ij} \mat{B}
        \end{bmatrix}_{i,j}
    \end{equation}

    \begin{equation}
        \label{eq:parallelism-042}
        \styleobj{k} \catprodob \styleobj{m}
    \end{equation}

    \begin{equation}
        \label{eq:parallelism-043}
        \styleobj{l} \catprodob \styleobj{n}
    \end{equation}

    set with cartesian product is not associative stacking
    \begin{equation}
        \label{eq:parallelism-044}
        (\setA \cartprod \setB) \cartprod \setC \neq \setA \cartprod (\setB \cartprod \setC)
    \end{equation}

    \begin{equation}
        \label{eq:parallelism-045}
        \tup{\tup{\ela, \elb}, \elc}
    \end{equation}

    \begin{equation}
        \label{eq:parallelism-046}
        \tup{\ela, \tup{\elb, \elc}}
    \end{equation}

    \begin{equation}
        \label{eq:parallelism-047}
        (\setA \cartprod \setB) \cartprod \setC
    \end{equation}

    \begin{equation}
        \label{eq:parallelism-048}
        \setA \cartprod (\setB \cartprod \setC)
    \end{equation}

    \begin{equation}
        \label{eq:parallelism-049}
        \SetL
    \end{equation}

    \begin{equation}
        \label{eq:parallelism-050}
        \PosL
    \end{equation}

    Mat with sum is associative stacking
    \begin{equation}
        \label{eq:parallelism-051}
        ( \mat{A} \mtimescatmor \mat{B} ) \mtimescatmor \mat{C}
    \end{equation}

    \begin{equation}
        \label{eq:parallelism-052}
        \mat{A} \mtimescatmor ( \mat{B} \mtimescatmor \mat{C} )
    \end{equation}

    \begin{equation}
        \label{eq:parallelism-053}
        \begin{bmatrix}
            \mat{A} & \mat{0} \\
            \mat{0} & \mat{B}
        \end{bmatrix}
        \mtimescatmor \mat{C}
    \end{equation}

    \begin{equation}
        \label{eq:parallelism-054}
        \mat{A} \mtimescatmor
        \begin{bmatrix}
            \mat{B} & \mat{0} \\
            \mat{0} & \mat{C}
        \end{bmatrix}
    \end{equation}

    \begin{equation}
        \label{eq:parallelism-055}
        \begin{bmatrix}
            \mat{A} & \mat{0} & \mat{0} \\
            \mat{0} & \mat{B} & \mat{0} \\
            \mat{0} & \mat{0} & \mat{C}
        \end{bmatrix}
    \end{equation}

    Mat with sum is functorial stacking

    \begin{align}\label{eq:parallelism-056}
        (\mat{A} \mthen \mat{B}) \mtimescatmor (\mat{C} \mthen \mat{D}) & = (\mat{B}\mat{A}) \mtimescatmor (\mat{D}\mat{C}) \\
                                                                        & =
        \begin{bmatrix}
            \mat{BA} & \mat{0}  \\
            \mat{0}  & \mat{DC}
        \end{bmatrix}
    \end{align}

    \begin{align}\label{eq:parallelism-057}
        (\mat{A} \mtimescatmor \mat{C}) \mthen (\mat{B} \mtimescatmor \mat{D}) & =
        \begin{bmatrix}
            \mat{A} & \mat{0} \\
            \mat{0} & \mat{C}
        \end{bmatrix}
        \mthen
        \begin{bmatrix}
            \mat{B} & \mat{0} \\
            \mat{0} & \mat{D}
        \end{bmatrix} \\
                                                                               & =
        \begin{bmatrix}
            \mat{B} & \mat{0} \\
            \mat{0} & \mat{D}
        \end{bmatrix}
        \begin{bmatrix}
            \mat{A} & \mat{0} \\
            \mat{0} & \mat{C}
        \end{bmatrix} \\
                                                                               & =
        \begin{bmatrix}
            \mat{BA} & \mat{0}  \\
            \mat{0}  & \mat{DC}
        \end{bmatrix}
    \end{align}

    \begin{equation}
        \label{eq:parallelism-058}
        \Obja \mfrom \text{``product of } \Obja \text{ and } \Objb \text{''} \mto \Objb
    \end{equation}

    \begin{equation}
        \label{eq:parallelism-059}
        \Obja \mfrom \styleobj{T} \mto \Objb
    \end{equation}

    Mat with sum is monoidal stacking

    \begin{equation}
        \label{eq:parallelism-060}
        \styleobj{n} \catsumob \styleobj{0} = \styleobj{n} = \styleobj{0} \catsumob \styleobj{n}
    \end{equation}

    \begin{equation}
        \label{eq:parallelism-061}
        \styleobj{n} \catprodob \styleobj{1} = \styleobj{n} = \styleobj{1} \catprodob \styleobj{n}
    \end{equation}

    \begin{equation}
        \label{eq:parallelism-062}
        \idmoncat = \styleobj{0}
    \end{equation}

    \begin{equation}
        \label{eq:parallelism-063}
        \idmoncat = \styleobj{1}
    \end{equation}

    \begin{equation}
        \label{eq:parallelism-064}
        \begin{bmatrix}\\ \end{bmatrix} \colon \styleobj{0} \mto \styleobj{0}
    \end{equation}

    \begin{equation}
        \label{eq:parallelism-065}
        \mat{ 1 } \colon \styleobj{1} \mto \styleobj{1}
    \end{equation}

    $\SetL$ is monoidal stacking

    \begin{equation}
        \label{eq:parallelism-066}
        \catid_\idmoncat \cprod \mora \colon \emptycobj \cprod \cObj{\setAn{1}, \ldots, \setAn{n}} \mto \emptycobj \cprod \cObj{\setBn{1}, \ldots, \setBn{m}}
    \end{equation}

    \begin{equation}
        \label{eq:parallelism-067}
        \mora \colon \cObj{\setAn{1}, \ldots, \setAn{n}} \mto \cObj{\setBn{1}, \ldots, \setBn{m}}
    \end{equation}

    \begin{equation}
        \label{eq:parallelism-068}
        \mora \cprod \catid_\idmoncat \colon \cObj{\setAn{1}, \ldots, \setAn{n}} \cprod \emptycobj \mto \cObj{\setBn{1}, \ldots, \setBn{m}} \cprod \emptycobj
    \end{equation}

    \begin{align*}
        \label{eq:parallelism-069}
        (\catid_\idmoncat \cprod \mora) (\tup{} \tupconcat \tup{\elna{1}, \elna{2}, \ldots, \elna{n}}) & = \catid_\idmoncat (\tup{}) \tupconcat \mora(\tup{\elna{1}, \elna{2}, \ldots, \elna{n}}) \\ &= \tup{} \tupconcat \mora(\tup{\elna{1}, \elna{2}, \ldots, \elna{n}}) \\
                                                                                                       & = \mora(\tup{\elna{1}, \elna{2}, \ldots, \elna{n}})
    \end{align*}

    \begin{equation}
        \label{eq:parallelism-070}
        \subAn{1} \setsubseteq \subAn{2}
    \end{equation}

    \begin{equation}
        \label{eq:parallelism-071}
        \setA = \makeset{ \sbretzel, \sfondue, \schoco, \sburger }
    \end{equation}

    \begin{equation}
        \label{eq:parallelism-071b}
        \subAn{1} = \makeset{ \sbretzel, \sfondue, \schoco }
    \end{equation}

    \begin{equation}
        \label{eq:parallelism-072}
        \subAn{2} = \makeset{\sfondue, \schoco, \sburger }
    \end{equation}

    \begin{equation}
        \label{eq:parallelism-073}
        \subAn{1} \setunion \subAn{2} = \setA
    \end{equation}

    \begin{equation}
        \label{eq:parallelism-074}
        \setC = \makeset{ \sfondue }
    \end{equation}

    \begin{equation}
        \label{eq:parallelism-075}
        \setC \setsubseteq \subAn{1}
    \end{equation}

    \begin{equation}
        \label{eq:parallelism-076}
        \setC \setsubseteq \subAn{2}
    \end{equation}

    \begin{equation}
        \label{eq:parallelism-077}
        \mora \colon \Obja \mto \Objb
    \end{equation}

    \begin{equation}
        \label{eq:parallelism-078}
        \mora \mthen \morb \colon \styleobj{A} \mto \styleobj{C}
    \end{equation}

    \begin{equation}
        \label{eq:parallelism-079}
        \mora \colon \styleobj{A} \mtimescatob \styleobj{B} \mto \styleobj{C} \mtimescatob \styleobj{D} \mtimescatob \styleobj{E}
    \end{equation}

    \begin{equation}
        \label{eq:parallelism-080}
        \morb \colon \Obja \mto \Objb
    \end{equation}

    \begin{equation}
        \label{eq:parallelism-081}
        \morc \colon \styleobj{U} \mtimescatob \styleobj{V} \mto \styleobj{T}
    \end{equation}

    \begin{equation}
        \label{eq:parallelism-082}
        \mtimescatmor
    \end{equation}

    \begin{equation}
        \label{eq:parallelism-083}
        \mora \mtimescatmor \morb \mtimescatmor \morc
    \end{equation}

    \begin{equation}
        \label{eq:parallelism-084}
        \mora \mtimescatmor \morc \colon \styleobj{A} \mtimescatob \styleobj{C} \mto \styleobj{B} \mtimescatob \styleobj{D}
    \end{equation}

    \begin{equation}
        \label{eq:parallelism-085}
        \catid_\styleobj{A} \colon \styleobj{A} \mto \styleobj{A}
    \end{equation}

    \begin{equation}
        \label{eq:parallelism-086}
        \mora \mthen \catid_\styleobj{B} = \mora
    \end{equation}

    \begin{equation}
        \label{eq:parallelism-087}
        \mora \colon \styleobj{A} \mto \styleobj{B}
    \end{equation}

    \begin{equation}
        \label{eq:parallelism-088}
        \catid_\styleobj{A} \mthen \mora  = \mora
    \end{equation}

    \begin{equation}
        \label{eq:parallelism-089}
        \mora \colon \idmoncat \mto \styleobj{A}
    \end{equation}

    \begin{equation}
        \label{eq:parallelism-090}
        \morb \colon \idmoncat \mtimescatob \styleobj{B} \mto \Obja \mtimescatob \Objb
    \end{equation}

    \begin{equation}
        \label{eq:parallelism-091}
        \morb \colon \styleobj{B} \mto \Obja \mtimescatob \Objb
    \end{equation}

    \begin{equation}
        \label{eq:parallelism-092}
        \styleobj{C} \mtimescatob \styleobj{D} \mtimescatob \styleobj{E} = \Obja \mtimescatob \Objb
    \end{equation}

    \begin{equation}
        \label{eq:parallelism-093}
        \catid_{\Obja} \mtimescatmor \catid_{\Objb} = \catid_{\Obja \mtimescatob \Objb}
    \end{equation}

    \begin{equation}
        \label{eq:parallelism-094}
        \Obja \mtimescatob \Objb
    \end{equation}

    \begin{equation}
        \label{eq:parallelism-095}
        \Obja
    \end{equation}

    \begin{equation}
        \label{eq:parallelism-096}
        \Objb
    \end{equation}

    \begin{equation}
        \label{eq:parallelism-097}
        (\mora \mtimescatmor \stylemorph{h} \mtimescatmor \stylemorph{i}) \mthen (\stylemorph{g} \mtimescatmor \catid_\styleobj{S} \mtimescatmor \catid_\styleobj{T}) \mthen (\stylemorph{k} \mtimescatmor \catid_\styleobj{T}) \mthen (\catid_\styleobj{U} \mtimescatmor \stylemorph{j})
    \end{equation}

    \begin{equation}
        \label{eq:parallelism-098}
        \catid_{\idmoncat} \mtimescatmor \mora = \mora
    \end{equation}

    \begin{equation}
        \label{eq:parallelism-099}
        \stylenat{\sigma}_{\Obja, \Objb} \colon \Obja \mtimescatob \Objb \mto \Objb \mtimescatob \Obja
    \end{equation}

    \begin{equation}
        \label{eq:parallelism-100}
        (\mora \mtimescatmor \morb) \mthen \stylenat{\sigma}_{\Objc, \Objd}
    \end{equation}

    \begin{equation}
        \label{eq:parallelism-101}
        \stylenat{\sigma}_{\Obja, \Objb} \mthen (\morb \mtimescatmor \mora)
    \end{equation}

    \begin{equation}
        \label{eq:parallelism-102}
        \stylenat{\sigma}_{\Obja, \Objb} \mthen \stylenat{\sigma}_{\Objb, \Obja} = \catid_{\Obja \mtimescatob \Objb}
    \end{equation}

    \begin{equation}
        \label{eq:parallelism-103}
        \stylenat{\sigma}_{\Obja, \Objb \mtimescat \Objc}
    \end{equation}

    \begin{equation}
        \label{eq:parallelism-104}
        (\stylenat{\sigma}_{\Obja, \Objb} \mtimescatmor \catid_\Objc) \mthen (\catid_\Objb \mtimescatmor \stylenat{\sigma}_{\Obja, \Objb})
    \end{equation}

    \begin{equation}
        \label{eq:parallelism-105}
        \cong
    \end{equation}

    \begin{equation}
        \label{eq:parallelism-106}
        \ntrafoa_{\tup{\setA, \setB, \setC}} \colon (\setA \cartprod \setB) \cartprod \setC  \mto  \setA \cartprod (\setB \cartprod \setC)
    \end{equation}

    \begin{equation}
        \label{eq:parallelism-107}
        \makeset{ * } \cartprod \setA  \neq \setA
    \end{equation}

    \begin{equation}
        \label{eq:parallelism-108}
        \stylenat{\lambda}_\setA \colon \makeset{ * } \cartprod \setA \mto \setA
    \end{equation}

    \begin{equation}
        \label{eq:parallelism-109}
        \stylenat{\rho}_\setA \colon \setA  \cartprod \makeset{ * } \mto \setA
    \end{equation}

    \begin{equation}
        \label{eq:parallelism-110}
        \begin{bmatrix}
            a
        \end{bmatrix}
        \colon \styleobj{1} \mto \styleobj{1}
    \end{equation}

    \begin{equation}
        \label{eq:parallelism-111}
        \begin{bmatrix}
            1 \\
            1
        \end{bmatrix}
        \colon \styleobj{1} \mto \styleobj{2}
    \end{equation}

    \begin{equation}
        \label{eq:parallelism-112}
        \begin{bmatrix}
            \quad
        \end{bmatrix}
        \colon \styleobj{0} \mto \styleobj{1}
    \end{equation}

    \begin{equation}
        \label{eq:parallelism-113}
        \begin{bmatrix}
            1 & 1
        \end{bmatrix}
        \colon \styleobj{2} \mto \styleobj{1}
    \end{equation}

    \begin{equation}
        \label{eq:parallelism-114}
        \begin{bmatrix}
            \quad
        \end{bmatrix}
        \colon \styleobj{1} \mto \styleobj{0}
    \end{equation}

    \begin{equation}
        \label{eq:parallelism-115}
        a \setin \reals
    \end{equation}


    \begin{equation}
        \label{eq:parallelism-116}
        \styleelements{x}\setin \setA\setintersection \setB
    \end{equation}

    \begin{equation}
        \label{eq:parallelism-117}
        \mapa(\styleelements{x})
    \end{equation}

    \begin{equation}
        \label{eq:parallelism-118}
        \mapb(\styleelements{x})
    \end{equation}

    \begin{equation}
        \label{eq:parallelism-119}
        \setA,\setB\setin \Obof\Rel
    \end{equation}

    \begin{equation}
        \label{eq:parallelism-120}
        \relA
    \end{equation}

    \begin{equation}
        \label{eq:parallelism-121}
        \setA \setdisunion \setB
    \end{equation}

    \begin{equation}
        \label{eq:parallelism-122}
        \inj_\setA\colon \setA\mto \setA \setdisunion\setB
    \end{equation}

    \begin{equation}
        \label{eq:parallelism-123}
        \inj_\setB\colon \setB\mto \setA \setdisunion\setB
    \end{equation}

    \begin{equation}
        \label{eq:parallelism-124}
    \end{equation}

    \begin{equation}
        \label{eq:parallelism-125}
        m,n\setin \natnumbers
    \end{equation}

    \begin{equation}
        \label{eq:parallelism-126}
        m\to n
    \end{equation}

    \begin{equation}
        \label{eq:parallelism-127}
        m | n
    \end{equation}

    \begin{equation}
        \label{eq:parallelism-128}
        6\to 12
    \end{equation}

    \begin{equation}
        \label{eq:parallelism-129}
        6 | 12
    \end{equation}

    \begin{equation}
        \label{eq:parallelism-130}
        \realswithleq
    \end{equation}

    \begin{equation}
        \label{eq:parallelism-131}
        \elna{1},\elna{2}\setin \reals
    \end{equation}

    \begin{equation}
        \label{eq:parallelism-132}
        \elna{1}\to \elna{2}
    \end{equation}

    \begin{equation}
        \label{eq:parallelism-133}
        \elna{1}\leq \elna{2}
    \end{equation}

    \begin{equation}
        \label{eq:parallelism-134}
        \elna{1}
    \end{equation}

    \begin{equation}
        \label{eq:parallelism-135}
        \elna{2}
    \end{equation}

    \begin{equation}
        \label{eq:parallelism-136}
        z\setin \reals
    \end{equation}

    \begin{equation}
        \label{eq:parallelism-137}
        \elna{1}\leq \elb
    \end{equation}

    \begin{equation}
        \label{eq:parallelism-138}
        \elna{2}\leq \elb
    \end{equation}

    \begin{equation}
        \label{eq:parallelism-139}
        \ela\setin \reals
    \end{equation}

    \begin{equation}
        \label{eq:parallelism-140}
        \elna{1}\leq \ela
    \end{equation}

    \begin{equation}
        \label{eq:parallelism-141}
        \elna{2}\leq \ela
    \end{equation}

    \begin{equation}
        \label{eq:parallelism-142}
        \elb\leq \ela
    \end{equation}

    \begin{equation}
        \label{eq:parallelism-143}
        \max\makeset{\elna{1},\elna{2}}
    \end{equation}

    \begin{equation}
        \label{eq:parallelism-144}
        \stylesets{S}
    \end{equation}

    \begin{equation}
        \label{eq:parallelism-145}
        \setA,\setB\setsubseteq \stylesets{S}
    \end{equation}

    \begin{equation}
        \label{eq:parallelism-146}
        \setA \sto \setB
    \end{equation}

    \begin{equation}
        \label{eq:parallelism-147}
        \setA\setsubseteq \setB
    \end{equation}

    \begin{equation}
        \label{eq:parallelism-148}
        \setA\setunion\setB
    \end{equation}

    \begin{equation}
        \label{eq:parallelism-149}
        \Obja \mto \text{``coproduct of } \Obja \text{ and } \Objb \text{''}  \mfrom \Objb
    \end{equation}

    \begin{equation}
        \label{eq:parallelism-150}
        \Obja \mto \styleobj{T} \mfrom \Objb
    \end{equation}

    \begin{equation}
        \label{eq:parallelism-151}
        \setC
    \end{equation}

    \begin{equation}
        \label{eq:parallelism-152}
        \proj_1\colon \setC \sto \setA
    \end{equation}

    \begin{equation}
        \label{eq:parallelism-153}
        \proj_2\colon \setC \sto \setB
    \end{equation}

    \begin{equation}
        \label{eq:parallelism-154}
        \proj_1
    \end{equation}

    \begin{equation}
        \label{eq:parallelism-155}
        \proj_2
    \end{equation}

    \begin{equation}
        \label{eq:parallelism-156}
        \makeset{ \sbretzel }
    \end{equation}

    \begin{equation}
        \label{eq:parallelism-157}
        \makeset{ \schoco }
    \end{equation}

    \begin{equation}
        \label{eq:parallelism-158}
        \makeset{ \sbretzel, \schoco }
    \end{equation}

    \begin{equation}
        \label{eq:parallelism-159}
        \makeset{ \sbretzel, \schoco, \sburger }
    \end{equation}

    \begin{equation}
        \label{eq:parallelism-160}
        \DPL
    \end{equation}

    \begin{equation}
    \label{eq:parallelism-161}
    \makecprod{
        \left ( \makecprod{\cObj{\setAn{1}, \ldots, \setAn{l}}, \cObj{\setBn{1}, \ldots, \setBn{m}}} \right )
        ,
        \cObj{\setCn{1}, \ldots, \setCn{n}}
    }
\end{equation}

    \begin{equation}
        \label{eq:parallelism-162}
            \makecprod{
        \cObj{\setAn{1}, \ldots, \setAn{l}},
        \left (
        \makecprod{
            \cObj{\setBn{1}, \ldots, \setBn{m}},
            \cObj{\setCn{1}, \ldots, \setCn{n}}
        }
        \right )
    }
    \end{equation}

    \begin{equation}
        \label{eq:parallelism-163}
            \cObj{\setAn{1}, \ldots, \setAn{l}, \setBn{1}, \ldots, \setBn{m}, \setCn{1}, \ldots, \setCn{n}}
    \end{equation}

    \begin{equation}
        \label{eq:parallelism-164}
        \setA
    \end{equation}

    \begin{equation}
        \label{eq:parallelism-165}
        \setA
    \end{equation}

    \begin{equation}
        \label{eq:parallelism-166}
        \setA
    \end{equation}

    \begin{equation}
        \label{eq:parallelism-167}
        \setA
    \end{equation}

    \begin{equation}
        \label{eq:parallelism-168}
        \setA
    \end{equation}

    \begin{equation}
        \label{eq:parallelism-169}
        \setA
    \end{equation}

    \begin{equation}
        \label{eq:parallelism-170}
        \setA
    \end{equation}

    \begin{equation}
        \label{eq:parallelism-171}
        \setA
    \end{equation}

    \begin{equation}
        \label{eq:parallelism-172}
        \setA
    \end{equation}

    \begin{equation}
        \label{eq:parallelism-173}
        \setA
    \end{equation}

    \begin{equation}
        \label{eq:parallelism-174}
        \setA
    \end{equation}

    \begin{equation}
        \label{eq:parallelism-175}
        \setA
    \end{equation}

    \begin{equation}
        \label{eq:parallelism-176}
        \setA
    \end{equation}

    \begin{equation}
        \label{eq:parallelism-177}
        \setA
    \end{equation}

    \begin{equation}
        \label{eq:parallelism-178}
        \setA
    \end{equation}

    \begin{equation}
        \label{eq:parallelism-179}
        \setA
    \end{equation}

    \begin{equation}
        \label{eq:parallelism-180}
        \setA
    \end{equation}

    \begin{equation}
        \label{eq:parallelism-181}
        \setA
    \end{equation}

    \begin{equation}
        \label{eq:parallelism-182}
        \setA
    \end{equation}

    \begin{equation}
        \label{eq:parallelism-183}
        \setA
    \end{equation}

    \begin{equation}
        \label{eq:parallelism-184}
        \setA
    \end{equation}

    \begin{equation}
        \label{eq:parallelism-185}
        \setA
    \end{equation}

    \begin{equation}
        \label{eq:parallelism-186}
        \setA
    \end{equation}

    \begin{equation}
        \label{eq:parallelism-187}
        \setA
    \end{equation}

    \begin{equation}
        \label{eq:parallelism-188}
        \setA
    \end{equation}

    \begin{equation}
        \label{eq:parallelism-189}
        \setA
    \end{equation}
    

\end{forslides}