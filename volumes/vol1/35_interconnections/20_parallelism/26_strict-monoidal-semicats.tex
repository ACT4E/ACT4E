% !TEX root = chapter-standalone.tex

\section{Strict monoidal semicategories}
\label{sec:strict-monoidal-semicategories}


\begin{ctdefinition}[Strict monoidal semicategory]
    \label{def:strict-monoidal-semicat}
    A \maindef{strict monoidal semicategory} is a \SY{functorial stacking semicategory} $\tup{\CatC, \mtimescat}$ with

    \constit

    \begin{enumerate}
        \item an object $\idmoncat \setin \Obof{\CatC}$, called the \emph{monoidal unit}
    \end{enumerate}

    \condit

    \begin{enumerate}
        \item For any object $\Obja$ of \CatC,
              \begin{equation}
                  \Obja \mtimescatob \idmoncat = \Obja \qquad \text{and} \qquad \idmoncat \mtimescatob \Obja = \Obja.
              \end{equation}
        \item The monoidal unit $\idmoncat$ has an \SY{identity morphism} $\catid_\idmoncat$, and for any morphism $\mora \colon \Obja \mto \Objb$,
              \begin{equation}
                  \mora \mtimescatmor \catid_\idmoncat = \mora \qquad \text{and} \qquad \catid_\idmoncat \mtimescatmor  \mora = \mora.
              \end{equation}
    \end{enumerate}
\end{ctdefinition}


\todotextjira{752}{\alphubel: @GZ write up Moore as an example of a strict monoidal semicat, where the monoidal unit is built using the empty tuples sets, and show that the monoidal unit indeed does have an identity}

\todotext{J: @J: what is a good definition of symmetric strict monoidal \SY{semicategory} ? Take inspiration from the \SY{adjunctions} for \SY{semicategories} paper that AC posted recently in order to encode ``isomorphism'' without saying it with identities?}



