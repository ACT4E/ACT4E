% !TEX root = chapter-standalone.tex

\section{Strict monoidal (semi)categories}

\subsection{Strict monoidal semicategories}

\begin{ctdefinition}[Strict monoidal semicategory]
    \label{def:strict-monoidal-semicat}
    A \maindef{strict monoidal semicategory} is a \SY{functorial stacking semicategory} $\tup{\CatC, \mtimescat}$ with

    \constit

    \begin{itemize}
        \item an object $\idmoncat \setin \Obof{\CatC}$, called the \emph{monoidal unit}
    \end{itemize}

    \condit

    \begin{itemize}
        \item For any object $\Obja$ of \CatC,
              \begin{equation}
                  \Obja \mtimescatob \idmoncat = \Obja \qquad \text{and} \qquad \idmoncat \mtimescatob \Obja = \Obja.
              \end{equation}
        \item The monoidal unit $\idmoncat$ has an \SY{identity morphism} $\catid_\idmoncat$, and for any morphism $\mora \colon \Obja \mto \Objb$,
              \begin{equation}
                  \mora \mtimescatmor \catid_\idmoncat = \mora \qquad \text{and} \qquad \catid_\idmoncat \mtimescatmor  \mora = \mora.
              \end{equation}
    \end{itemize}

\end{ctdefinition}

\begin{example}
    We can show that \Moore is a strict monoidal semicategory.
    The monoidal unit is given by $\cObj{}$.
    Its identity morphism is the Moore machine
    \begin{equation*}
        \catid_\idmoncat=\tup{\cObj{},\cObj{},\cObj{},\prdyn_\idmoncat, \prreadout_\idmoncat, \tup{}},
    \end{equation*}
    where
    \begin{equation*}
        \defmapcomma{\prdyn_{\idmoncat}}
        {\cObj{} \cprod \cObj{}}
        {\sto}
        {\cObj{}}
        {\tup{}\tupconcat \tup{}}
        {\tup{}}
    \end{equation*}
    and
    \begin{equation*}
        \defmapperiod{\prreadout_{\idmoncat}}
        {\cObj{}}
        {\sto}
        {\cObj{}}
        {\tup{}}
        {\tup{}}
    \end{equation*}
    Clearly, $\prObja \cprod \cObj{}=\cObj{}\cprod \prObja=\prObja$ for every $\prObja\setin \Ob_\Moore$.
    Furthermore, consider a Moore machine $\mora\colon \prin \mto \prout$ with
    \begin{equation*}
        \mora=\tup{\prin,\prst, \prout, \prdyn,\prreadout,\prstart}.
    \end{equation*}
    One has:
    \begin{equation*}
        \begin{aligned}
            \mora \mtimescatmor \catid_\idmoncat & =\tup{\prin \cprod \cObj{},\prst \cprod \cObj{},\prout \cprod \cObj{},\prdyn_{\mora \mtimescatmor\catid_\idmoncat}, \prreadout_{\mora \mtimescatmor\catid_\idmoncat}, \prstart\tupconcat \tup{}} \\
                                                 & =\tup{\prin,\prst ,\prout,\prdyn, \prreadout, \prstart}=\mora,
        \end{aligned}
    \end{equation*}
    where we used
    \begin{equation*}
        \begin{aligned}
            \prdyn_{\mora \mtimescatmor\catid_\idmoncat}\colon \prin \cprod \cObj{}\cprod \prst \cprod \cObj{} & \sto \prst \cprod \cObj{} \\
            \prinel \tupconcat \tup{}\tupconcat \prstel \tupconcat \tup{}                                      & \mapsto \prdyn(\prinel,\prstel)\tupconcat \prdyn_\idmoncat(\tup{},\tup{})=\prdyn(\prinel,\prstel)
        \end{aligned}
    \end{equation*}
    and
    \begin{equation*}
        \begin{aligned}
            \prreadout_{\mora \mtimescatmor\catid_\idmoncat}\colon \prst \cprod \cObj{} & \sto \prout \cprod \cObj{} \\
            \prstel \tupconcat \tup{}                                                   & \mapsto \prreadout(\prstel)\tupconcat \prreadout_\idmoncat(\tup{})=\prreadout(\prstel)
        \end{aligned}
    \end{equation*}
    to show the equivalences $\prdyn=\prdyn_{\mora \mtimescatmor \catid_\idmoncat}$ and $\prreadout=\prreadout_{\mora \mtimescatmor \catid_\idmoncat}$.
    The argument for $\catid_\idmoncat \mtimescatmor \mora$ follows analogously.
\end{example}

\todotext{J: @JL: is Moore an example?}

\todotext{J: @J: what is a good definition of symmetric strict monoidal \SY{semicategory} ? Take inspiration from the \SY{adjunctions} for \SY{semicategories} paper that AC posted recently in order to encode ``isomorphism'' without saying it with identities?}

\begin{ctdefinition}[Commutative strict monoidal semicategory]
    \label{def:commutative-strict-monoidal}
    A \maindef{commutative strict monoidal semicategory} is a \SY{strict monoidal semicategory} $\tup{\CatC, \mtimescat, \idmoncat}$ with

    \condit

    \begin{enumerate}
        \item \emph{Symmetry}:
              \begin{equation}
                  \Obja \mtimescatob \Objb = \Objb \mtimescatob \Obja
              \end{equation}
              for all object $\Obja, \Objb \setin \Ob_\CatC$.
    \end{enumerate}
\end{ctdefinition}

\todotext{J: @J: given an example}

\subsection{Strict monoidal categories}

\begin{ctdefinition}
    \label{def:strict-monoidal-category}
    A \maindef{strict monoidal category} is a strict monoidal \SY{semicategory} $\tup{\CatC, \mtimescat, \idmoncat}$ where $\CatC$ is in fact a category.
\end{ctdefinition}

\begin{ctdefinition}
    \label{def:sym-strict-monoidal-semicat}
    A \emph{symmetric strict monoidal category} is a \SY{strict monoidal category} $\tup{\CatC, \mtimescat, \idmoncat}$ with

    \constit

    \begin{enumerate}
        \item For any two objects $\Obja, \Objb \setin \Ob_\CatC$ there exists an isomorphism
              \begin{equation}
                  \sigma_{\Obja, \Objb} \colon \Obja \mtimescatob \Objb \mto \Objb \mtimescatob \Obja,
              \end{equation}
              called the \emph{braiding}.
    \end{enumerate}

    \condit

    \begin{enumerate}
        \item \emph{Naturality}: For any morphisms $\mora \colon \Obja \mto \Objc$, $\morb \colon \Objb \mto \Objd$, the diagram
                  [ INSERT DIAGRAM ]
              commutes.
        \item \emph{Commuativity}: For all $\Obja, \Objb \setin \Ob_\CatC$,
              \begin{equation}
                  \sigma_{\Obja, \Objb} \mthen \sigma_{\Objb, \Obja} = \catid_{\Obja \mtimescatob \Objb}.
              \end{equation}
    \end{enumerate}

\end{ctdefinition}

\todotext{@JL: make two example of symmetric strict monoidal \SY{semicategory} using natural numbers, once with addition and once with multiplication}

\todotext{@JL: make more examples of strict monoidal categories}

