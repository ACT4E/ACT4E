% !TEX root = chapter-standalone.tex

\section{Strict monoidal semicategories}
\label{sec:strict-monoidal-semicategories}


\begin{ctdefinition}[Strict monoidal semicategory]
    \label{def:strict-monoidal-semicat}
    A \maindef{strict monoidal semicategory} is a strict monoidal stacking semicategory that is functorial stacking. 
    \end{ctdefinition}

\begin{example}
    We can look at \Moore and ask whether it is a strict monoidal semicategory.
    The monoidal unit is given by $\cObj{}$.
    Its identity morphism is the Moore machine
    \begin{equation*}
        \catid_\idmoncat=\tup{\cObj{},\cObj{},\cObj{},\prdyn_\idmoncat, \prreadout_\idmoncat, \tup{}},
    \end{equation*}
    where
    \begin{equation*}
        \defmapcomma{\prdyn_{\idmoncat}}
        {\cObj{} \cprod \cObj{}}
        {\sto}
        {\cObj{}}
        {\tup{}\tupconcat \tup{}}
        {\tup{}}
    \end{equation*}
    and
    \begin{equation*}
        \defmapperiod{\prreadout_{\idmoncat}}
        {\cObj{}}
        {\sto}
        {\cObj{}}
        {\tup{}}
        {\tup{}}
    \end{equation*}
    Clearly, $\prObja \cprod \cObj{}=\cObj{}\cprod \prObja=\prObja$ for every $\prObja\setin \Ob_\Moore$.
    Furthermore, consider a Moore machine $\mora\colon \prin \mto \prout$ with
    \begin{equation*}
        \mora=\tup{\prin,\prst, \prout, \prdyn,\prreadout,\prstart}.
    \end{equation*}
    One has:
    \begin{equation*}
        \begin{aligned}
            \mora \mtimescatmor \catid_\idmoncat & =\tup{\prin \cprod \cObj{},\prst \cprod \cObj{},\prout \cprod \cObj{},\prdyn_{\mora \mtimescatmor\catid_\idmoncat}, \prreadout_{\mora \mtimescatmor\catid_\idmoncat}, \prstart\tupconcat \tup{}} \\
                                                 & =\tup{\prin,\prst ,\prout,\prdyn, \prreadout, \prstart}=\mora,
        \end{aligned}
    \end{equation*}
    where we used
    \begin{equation*}
        \begin{aligned}
            \prdyn_{\mora \mtimescatmor\catid_\idmoncat}\colon \prin \cprod \cObj{}\cprod \prst \cprod \cObj{} & \sto \prst \cprod \cObj{} \\
            \prinel \tupconcat \tup{}\tupconcat \prstel \tupconcat \tup{}                                      & \mapsto \prdyn(\prinel,\prstel)\tupconcat \prdyn_\idmoncat(\tup{},\tup{})=\prdyn(\prinel,\prstel)
        \end{aligned}
    \end{equation*}
    and
    \begin{equation*}
        \begin{aligned}
            \prreadout_{\mora \mtimescatmor\catid_\idmoncat}\colon \prst \cprod \cObj{} & \sto \prout \cprod \cObj{} \\
            \prstel \tupconcat \tup{}                                                   & \mapsto \prreadout(\prstel)\tupconcat \prreadout_\idmoncat(\tup{})=\prreadout(\prstel)
        \end{aligned}
    \end{equation*}
    to show the equivalences $\prdyn=\prdyn_{\mora \mtimescatmor \catid_\idmoncat}$ and $\prreadout=\prreadout_{\mora \mtimescatmor \catid_\idmoncat}$.
    The argument for $\catid_\idmoncat \mtimescatmor \mora$ follows analogously.
    \todo{Important: add comment regarding the fact that \Moore is not functorial stacking on the nose, therefore cannot be considered here?}
\end{example}

\todotext{J: @JL: is Moore an example?}

\todotext{J: @J: what is a good definition of symmetric strict monoidal \SY{semicategory} ? Take inspiration from the \SY{adjunctions} for \SY{semicategories} paper that AC posted recently in order to encode ``isomorphism'' without saying it with identities?}



