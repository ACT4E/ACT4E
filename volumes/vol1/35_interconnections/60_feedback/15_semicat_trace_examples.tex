% !TEX root = chapter-standalone.tex
\section{Examples of Semicategorical Trace}
\begin{example}
    Consider the semicategory \Moore of Moore machines.
    We show that when equipped with stacking and trace operations as previously defined, it fulfills the trace axioms presented in \cref{def:traced-fun-stack-scat}.

    \textbf{Vanishing I:}
    Consider a Moore machine $\mora\colon \prin \mto \prout$ with
    \begin{equation*}
        \mora=\tup{\prin,\prst, \prout, \prdyn,\prreadout,\prstart}.
    \end{equation*}
    We need to show
    \begin{equation*}
        \Tr_{\prin,\prout}^{\idmoncat}(\mora)=\mora.
    \end{equation*}
    Clearly, we have:
    \begin{equation*}
        \begin{aligned}
            \Tr_{\prin,\prout}^{\idmoncat}(\mora) & =\tup{\prin, \prst, \prout, \prdyn_{\Tr \mora}, \prreadout_{\Tr \mora},\prstart}, \\
                                                  & =\tup{\prin, \prst, \prout, \prdyn, \prreadout,\prstart} \\
                                                  & =\mora,
        \end{aligned}
    \end{equation*}
    where we used
    \begin{equation*}
        \defmapcomma{\prdyn_{\Tr \mora}}
        {\prin \cprod \prst}
        {\sto}
        {\prst}
        {\prinel\tupconcat \prstel}
        {\prdyn(\prinel \tupconcat \proj_\idmoncat (\prreadout(\prstel))\tupconcat \prstel)=\prdyn(\prinel \tupconcat \tup{} \tupconcat \prstel)=\prdyn(\prinel \tupconcat \prstel)},
    \end{equation*}
    and
    \begin{equation*}
        \defmapcomma{\prreadout_{\Tr \mora}}
        {\prst}
        {\sto}
        {\prout}
        {\prstel}
        {\proj_\prout(\prreadout(\prstel))=\prreadout(\prstel)},
    \end{equation*}
    which show the equivalences $\prdyn_{\Tr \mora}=\prdyn$ and $\prreadout_{\Tr \mora}=\prreadout$.

    \textbf{Vanishing II:}
    Consider a Moore machine $\mora\colon \prObja \cprod \prObjc \cprod \prObjd \mto \prObjb \cprod \prObjc \cprod \prObjd$ with
    \begin{equation*}
        \mora=\tup{\prObja \cprod \prObjc \cprod \prObjd,\prst, \prObjb \cprod \prObjc \cprod \prObjd, \prdyn,\prreadout,\prstart}.
    \end{equation*}
    We need to show
    \begin{equation*}
        \Tr_{\prObja, \prObjb}^{\prObjc \cprod \prObjd}(\mora)=\Tr_{\prObja, \prObjb}^{\prObjc }\left(\Tr_{\prObja\cprod \prObjc,\prObjb\cprod \prObjc }^{\prObjd}(\mora)\right)
    \end{equation*}
    Let's start from the left-hand side.
    We have:
    \begin{equation*}
        \Tr_{\prObja, \prObjb}^{\prObjc \cprod \prObjd}(\mora)=\tup{\prObja, \prst, \prObjb, \prdyna, \prreadouta, \prstart},
    \end{equation*}
    with
    \begin{equation*}
        \defmapcomma{\prdyna}
        {\prObja \cprod \prst}
        {\sto}
        {\prst}
        {\prObjael\tupconcat \prstel}
        {\prdyn(\prObjael\tupconcat \proj_{\prObjc \cprod \prObjd}(\prreadout(\prstel))\tupconcat \prstel)}
    \end{equation*}
    and
    \begin{equation*}
        \defmapperiod{\prreadouta}
        {\prst}
        {\sto}
        {\prObjb}
        {\prstel}
        {\proj_\prObjb(\prreadout(\prstel))}
    \end{equation*}

    We can look at the right-hand side of the axiom in two steps.
    First, we have:
    \begin{equation*}
        \Tr_{\prObja\cprod \prObjc,\prObjb\cprod \prObjc }^{\prObjd}(\mora)=
        \tup{\prObja\cprod \prObjc, \prst, \prObjb\cprod \prObjc, \prdynb, \prreadoutb, \prstart}
    \end{equation*}
    with
    \begin{equation*}
        \defmapcomma{\prdynb}
        {\prObja \cprod \prObjc\cprod \prst}
        {\sto}
        {\prst}
        {\prObjael\tupconcat \prObjcel \tupconcat \prstel}
        {\prdyn(\prObjael\tupconcat \prObjcel \tupconcat \proj_\prObjd(\prreadout(\prstel))\tupconcat \prstel)}
    \end{equation*}
    and
    \begin{equation*}
        \defmapperiod{\prreadoutb}
        {\prst}
        {\sto}
        {\prObjb\cprod \prObjc}
        {\prstel}
        {\proj_{\prObjb\cprod \prObjc} (\prreadout(\prstel))}
    \end{equation*}
    Furthermore:
    \begin{equation*}
        \begin{aligned}
            \Tr_{\prObja, \prObjb}^{\prObjc }\left(\Tr_{\prObja\cprod \prObjc,\prObjb\cprod \prObjc }^{\prObjd}(\mora)\right)
             & =\tup{\prObja, \prst, \prObjb, \prdync,\prreadoutc,\prstart} \\
             & =\tup{\prObja, \prst, \prObjb, \prdyna, \prreadouta, \prstart} \\
             & =\Tr_{\prObja, \prObjb}^{\prObjc \cprod \prObjd}(\mora),
        \end{aligned}
    \end{equation*}
    where we used
    \begin{equation*}
        \begin{aligned}
            \prdync\colon \prObja\cprod \prst & \sto \prst \\
            \prObjael\tupconcat \prstel       & \mapsto
            \prdynb(\prObjael \tupconcat \proj_\prObjc(\prreadoutb(\prstel))\tupconcat \prstel) \\
                                              & =\prdyn(\prObjael \tupconcat \proj_\prObjc(\proj_{\prObjb\cprod \prObjc}(\prreadout(\prstel))) \tupconcat \proj_\prObjd(\prreadout(\prstel))\tupconcat \prstel) \\
                                              & =\prdyn(\prObjael \tupconcat \proj_\prObjc(\prreadout(\prstel)) \tupconcat \proj_\prObjd(\prreadout(\prstel))\tupconcat \prstel) \\
                                              & =\prdyn(\prObjael \tupconcat \proj_{\prObjc\cprod \prObjd}(\prreadout(\prstel)) \tupconcat \prstel)
        \end{aligned}
    \end{equation*}
    %
    and
    %
    \begin{equation*}
        \begin{aligned}
            \prreadoutc\colon \prst & \sto \prObjb \\
            \prstel                 & \mapsto
            \proj_\prObjb(\prreadoutb(\prstel)) \\
                                    & =\proj_\prObjb(\proj_{\prObjb\cprod \prObjc} (\prreadout(\prstel))) \\
                                    & =\proj_\prObjb(\prreadout(\prstel)),
        \end{aligned}
    \end{equation*}
    to show the equivalences $\prdyna=\prdync$ and $\prreadouta=\prreadoutc$.

    \textbf{Superposing:}
    Consider Moore machines $\mora\colon \prObja \cprod \prObjc \mto \prObjb \cprod \prObjc$ and $\morb\colon \prObje\mto \prObjf$ with
    \begin{equation*}
        \begin{aligned}
            \mora & =\tup{\prObja \cprod \prObjc,\prst_\mora, \prObjb \cprod \prObjc, \prdyn_\mora,\prreadout_\mora,\prstart_\mora} \\
            \morb & =\tup{\prObje,\prst_\morb, \prObjf, \prdyn_\morb,\prreadout_\morb,\prstart_\morb}.
        \end{aligned}
    \end{equation*}
    We need to show:
    \begin{equation*}
        \Tr_{\prObje\cprod \prObja, \prObjf\cprod \prObjb}^{\prObjc}(\morb \mtimescatmor \mora)=\morb \mtimescatmor \Tr_{\prObja,\prObjb}^{\prObjc}(\mora).
    \end{equation*}
    Let's start from the left-hand side.
    We have:
    \begin{equation*}
        \morb\mtimescatmor \mora=\tup{\prObje\cprod \prObja \cprod \prObjc, \prst_\morb \cprod \prst_\mora, \prObjf\cprod \prObjb \cprod \prObjc, \prdyn_{\morb\mtimescatmor\mora},\prreadout_{\morb\mtimescatmor \mora},\prstart_\morb\tupconcat \prstart_\mora}
    \end{equation*}
    Furthermore:
    \begin{equation*}
        \Tr_{\prObje\cprod \prObja, \prObjf\cprod \prObjb}^{\prObjc}(\morb \mtimescatmor \mora)=
        \tup{\prObje\cprod \prObja, \prst_\morb \cprod \prst_\mora, \prObjf\cprod \prObjb, \prdyna,\prreadouta,\prstart_\morb\tupconcat \prstart_\mora}
    \end{equation*}
    with
    \begin{equation*}
        \begin{aligned}
            \prdyna\colon \prObje\cprod \prObja\cprod \prst_\morb \cprod \prst_\mora        & \sto \prst_\morb \cprod \prst_\mora \\
            \prObjeel\tupconcat \prObjael \tupconcat \prstel_\morb \tupconcat \prstel_\mora & \mapsto
            \prdyn_{\morb\mtimescatmor \mora}(\prObjeel\tupconcat \prObjael\tupconcat \proj_\prObjc(\prreadout_{\morb\mtimescatmor \mora}(\prstel_\morb \tupconcat \prstel_\mora))\tupconcat \prstel_\morb \tupconcat \prstel_\mora) \\
                                                                                            & =\prdyn_\morb(\prObjeel\tupconcat \prstel_\morb)\tupconcat \prdyn_\mora(\prObjael\tupconcat \proj_\prObjc(\prreadout_{\morb}(\prstel_\morb)\tupconcat \prreadout_\mora(\prstel_\mora)) \tupconcat \prstel_\mora) \\
                                                                                            & =\prdyn_\morb(\prObjeel\tupconcat \prstel_\morb)\tupconcat \prdyn_\mora(\prObjael\tupconcat \proj_\prObjc(\prreadout_\mora(\prstel_\mora)) \tupconcat \prstel_\mora)
        \end{aligned}
    \end{equation*}
    and
    \begin{equation*}
        \begin{aligned}
            \prreadouta\colon \prst_\morb \cprod \prst_\mora & \sto \prObjf\cprod \prObjb \\
            \prstel_\morb \tupconcat \prstel_\mora           & \mapsto \proj_{\prObjf\cprod \prObjb}(\prreadout_{\morb \mtimescatmor \mora}(\prstel_\morb \tupconcat \prstel_\mora)) \\
                                                             & =\proj_{\prObjf\cprod \prObjb}(\prreadout_{\morb}(\prstel_\morb)\tupconcat \prreadout_{\mora}(\prstel_\mora)) \\
                                                             & =\prreadout_\morb(\prstel_\morb)\tupconcat \proj_\prObjb(\prreadout_\mora(\prstel_\mora)).
        \end{aligned}
    \end{equation*}
    For the right-hand side, we have:
    \begin{equation*}
        \Tr_{\prObja,\prObjb}^{\prObjc}(\mora)=\tup{\prObja,\prst_\mora, \prObjb,\prdyn_{\Tr \mora},\prreadout_{\Tr \mora},\prstart_\mora},
    \end{equation*}
    and
    \begin{equation*}
        \begin{aligned}
            \morb \mtimescatmor \Tr_{\prObja,\prObjb}^{\prObjc}(\mora) & =\tup{\prObje\cprod \prObja,\prst_\morb \cprod \prst_\mora, \prObje\cprod \prObjb,\prdynb,\prreadoutb,\prstart_\morb \tupconcat \prstart_\mora}, \\
                                                                       & =\tup{\prObje\cprod \prObja,\prst_\morb \cprod \prst_\mora, \prObje\cprod \prObjb,\prdyna,\prreadouta,\prstart_\morb \tupconcat \prstart_\mora}
        \end{aligned}
    \end{equation*}
    where we used
    \begin{equation*}
        \begin{aligned}
            \prdynb\colon \prObje\cprod \prObja\cprod \prst_\morb \cprod \prst_\mora       & \sto \prst_\morb \cprod \prst_\mora \\
            \prObjeel\tupconcat \prObjael\tupconcat \prstel_\morb \tupconcat \prstel_\mora & \mapsto
            \prdyn_\morb(\prObjeel \tupconcat \prstel_\morb)\tupconcat \prdyn_{\Tr \mora}(\prObjael\tupconcat \prstel_\mora) \\
                                                                                           & =\prdyn_\morb(\prObjeel \tupconcat \prstel_\morb)\tupconcat \prdyn_\mora(\prObjael\tupconcat \proj_\prObjc(\prreadout(\prstel_\mora))\tupconcat \prstel_\mora)
        \end{aligned}
    \end{equation*}
    and
    \begin{equation*}
        \begin{aligned}
            \prreadoutb\colon \prst_\morb \cprod \prst_\mora & \sto \prObje\cprod \prObjb \\
            \prstel_\morb \tupconcat \prstel_\mora           & \mapsto
            \prreadout_\morb(\prstel_\morb)\tupconcat \prreadout_{\Tr \mora}(\prstel_\mora) \\
                                                             & =\prreadout_\morb(\prstel_\morb)\tupconcat \proj_\prObjb(\prreadout_\mora(\prstel_\mora))
        \end{aligned}
    \end{equation*}
    to show the equivalences $\prdyna=\prdynb$ and $\prreadouta=\prreadoutb$.
\end{example}
%
\begin{example}
    Consider the subsemicategory of \LTI only allowing proper LTI systems as morphisms with the trace defined in \cref{def:trace-lti-prop}.
    The trace axioms are satified.
    We prove the statements not involving identities one by one.

    \textbf{Vanishing I:}
    Consider a proper LTI system $\mora\colon i\mto j$, given by $\genericplti{\mora}$ with state dimension $s\setin \natnumbers$.
    We need to check
    \begin{equation*}
        \Tr_{i,j}^0 (\mora)=\mora.
    \end{equation*}
    Using the trace definition, we can write:
    \begin{equation*}
        \Tr_{i,j}^0 (\mora)=\tup{\prstart{\mora}, \mat{A}_\mora+\rowscols{\mat{B}_{\mora}}{:}{i+1:i+k}\rowscols{\mat{C}_{\mora}}{j+1:j+k}{:}, \rowscols{\mat{B}_{\mora}}{:}{1:i}, \rowscols{\mat{C}_{\mora}}{1:j}{:}}.
    \end{equation*}
    However, we have:
    \begin{equation*}
        \mat{B}_\mora=\begin{bmatrix}
            \rowscols{\mat{B}_{\mora}}{:}{1:i} & \mat{0}^{s\times 0}
        \end{bmatrix},\quad
        \mat{C}_\mora=\begin{bmatrix}
            \rowscols{\mat{C}_{\mora}}{1:j}{:} \\ \mat{0}^{0\times s}
        \end{bmatrix},
    \end{equation*}
    and therefore
    \begin{equation*}
        \Tr_{i,j}^0 (\mora)=\tup{\prstart{\mora}, \mat{A}_\mora, \mat{B}_\mora, \mat{C}_\mora}=\mora.
    \end{equation*}

    \textbf{Vanishing II:}
    Consider a proper LTI system $\mora\colon i+k+o\mto j+k+o$, given by $\genericplti{\mora}$ with state dimension $s\setin \natnumbers$.
    We need to check
    \begin{equation*}
        \Tr_{i,j}^{k+o} (\mora)=\Tr_{i,j}^{k}\left(\Tr_{i+k,j+k}^o(\mora)\right).
    \end{equation*}
    Let's start with the left-hand side of the statement.
    We have:
    \begin{equation}
        \label{eq:vanishing2-left}
        \Tr_{i,j}^{k+o} (\mora)=\tup{\prstart, \mat{A}_\mora+\rowscols{\mat{B}_\mora}{:}{i+1:i+k+o}\rowscols{\mat{C}_\mora}{j+1:j+k+o}{:}, \rowscols{\mat{B}_\mora}{:}{1:i},\rowscols{\mat{C}_\mora}{1:j}{:}}.
    \end{equation}
    Let's look at the right-hand side.
    We have:
    \begin{widepar}
        \begin{equation*}
            \Tr_{i+k,j+k}^{o} (\mora)=\tup{\prstart, \mat{A}+\rowscols{\mat{B}_\mora}{:}{i+k+1:i+k+o} \rowscols{\mat{C}_\mora}{j+k+1:j+k+o}{:}, \underbrace{\rowscols{\mat{B}_\mora}{:}{1:i+k}}_{\mat{B}_{\mora,\star}},\underbrace{\rowscols{\mat{C}_\mora}{1:j+k}{:}}_{\mat{C}_{\mora,\star}}},
        \end{equation*}
    \end{widepar}
    which we can leverage to express $\Tr_{i,j}^{k}\left(\Tr_{i+k,j+k}^o(\mora)\right)=\ast$.
    Clearly, $\projA(\ast)=\prstart$.
    Furthermore:
    \begin{equation*}
        \begin{aligned}
            \projB(\ast) & =\mat{A}_\mora + \rowscols{\mat{B}_\mora}{:}{i+k+1:i+k+o}\rowscols{\mat{C}_\mora}{j+k+1:j+k+o}{:}+\rowscols{\mat{B}_{\mora,\star}}{:}{i+1:i+k}\rowscols{\mat{C}_{\mora,\star}}{:}{j+1:j+k} \\
                         & =\mat{A}_\mora + \rowscols{\mat{B}_\mora}{:}{i+k+1:i+k+o}\rowscols{\mat{C}_\mora}{j+k+1:j+k+o}{:}+\rowscols{\mat{B}_{\mora}}{:}{i+1:i+k}\rowscols{\mat{C}_{\mora}}{:}{j+1:j+k} \\
                         & =\mat{A}_\mora + \begin{bmatrix}
                                                \rowscols{\mat{B}_{\mora}}{:}{i+1:i+k} & \rowscols{\mat{B}_\mora}{:}{i+k+1:i+k+o}
                                            \end{bmatrix} \begin{bmatrix}
                                                              \rowscols{\mat{C}_{\mora}}{:}{j+1:j+k} \\
                                                              \rowscols{\mat{C}_\mora}{j+k+1:j+k+o}{:}
                                                          \end{bmatrix} \\
                         & =\mat{A}+\rowscols{\mat{B}_\mora}{:}{i+1:i+k+o}\rowscols{\mat{C}_\mora}{j+1:j+k+o}{:}.
        \end{aligned}
    \end{equation*}
    Furthermore, $\projC(\ast) = \rowscols{\mat{B}_\mora}{:}{1:i}$ and $\projD(\ast) = \rowscols{\mat{C}_\mora}{1:j}{:}$.
    Clearly, the found results correspond to \cref{eq:vanishing2-left}.

    \textbf{Superposing:}
    Consider the proper LTI systems $\mora\colon i+k\mto j+k$, given by $\genericplti{\mora}$, and $\morb\colon l\mto m$, given by $\genericplti{\morb}$.
    We need to check:
    \begin{equation*}
        \Tr_{l+i,m+j}^k(\morb \mtimescatmor \mora)=\morb \mtimescatmor \Tr_{i,j}^k(\mora).
    \end{equation*}
    Let's start with the left-hand side.
    We have:
    \begin{equation*}
        \morb \mtimescatmor \mora=\tup{\begin{bmatrix}\prstart_\morb\\ \prstart_\mora \end{bmatrix}, \begin{bmatrix} \mat{A}_\morb&\mat{0}\\ \mat{0}&\mat{A}_\mora\end{bmatrix},
            \begin{bmatrix} \mat{B}_\morb&\mat{0}\\ \mat{0}&\mat{B}_\mora\end{bmatrix},\begin{bmatrix} \mat{C}_\morb&\mat{0}\\ \mat{0}&\mat{C}_\mora\end{bmatrix}}
    \end{equation*}
    Now observe that $\mat{B}_\morb$ has $l$ columns, $\mat{B}_\mora$ has $i+k$ columns, $\mat{C}_\morb$ has $m$ rows, and $\mat{C}_\mora$ has $j+k$ rows.
    Let's denote $\Tr_{l+i,m+j}^k(\morb \mtimescatmor \mora)$ by $\ast$.
    Clearly:
    \begin{equation*}
        \projA(\ast)=\begin{bmatrix}\prstart_\morb &\prstart_\mora\end{bmatrix}.
    \end{equation*}
    We have:
    \begin{equation*}
        \begin{aligned}
            \projB(\ast) & =\begin{bmatrix} \mat{A}_\morb&\mat{0}\\ \mat{0}&\mat{A}_\mora\end{bmatrix}+\rowscols{\begin{bmatrix} \mat{B}_\morb&\mat{0}\\ \mat{0}&\mat{B}_\mora\end{bmatrix}}{:}{l+i+1:l+i+k}\rowscols{\begin{bmatrix} \mat{C}_\morb&\mat{0}\\ \mat{0}&\mat{C}_\mora\end{bmatrix}}{m+j+1:m+j+k}{:} \\
                         & =\begin{bmatrix} \mat{A}_\morb&\mat{0}\\ \mat{0}&\mat{A}_\mora\end{bmatrix}+\begin{bmatrix} \mat{0}\\ \rowscols{\mat{B}_\mora}{:}{i+1:i+k}\end{bmatrix}
            \begin{bmatrix}
                \mat{0} & \rowscols{\mat{C}_\mora}{j+1:j+k}{:}
            \end{bmatrix} \\
                         & =\begin{bmatrix} \mat{A}_\morb&\mat{0}\\ \mat{0}&\mat{A}_\mora\end{bmatrix}
            \begin{bmatrix}
                \mat{0} & \mat{0}                                                                  \\
                \mat{0} & \rowscols{\mat{B}_\mora}{:}{i+1:i+k}\rowscols{\mat{C}_\mora}{j+1:j+k}{:}
            \end{bmatrix}
        \end{aligned}
    \end{equation*}
    Furthermore, we have
    \begin{equation*}
        \projC(\ast)=\begin{bmatrix}
            \mat{B}_\morb & \mat{0}                          \\
            \mat{0}       & \rowscols{\mat{B}_\mora}{:}{1:i}
        \end{bmatrix}
    \end{equation*}
    and
    \begin{equation*}
        \projD(\ast)=\begin{bmatrix}
            \mat{C}_\morb & \mat{0}                          \\
            \mat{0}       & \rowscols{\mat{C}_\mora}{1:j}{:}
        \end{bmatrix}
    \end{equation*}
    We can now look at the right-hand side.
    First:
    \begin{equation*}
        \Tr_{i,j}^{k} (\mora)=\tup{\prstart, \mat{A}_\mora+\rowscols{\mat{B}_\mora}{:}{i+1:i+k} \rowscols{\mat{C}_\mora}{j+1:j+k}{:}, \rowscols{\mat{B}_\mora}{:}{1:i},\rowscols{\mat{C}_\mora}{1:j}{:}},
    \end{equation*}
    Therefore, we have:
    \begin{equation*}
        \projA(\morb \mtimescatmor \Tr_{i,j}^k(\mora))=\begin{bmatrix}\prstart_\morb &\prstart_\mora\end{bmatrix},
    \end{equation*}
    \begin{equation*}
        \projB(\morb \mtimescatmor \Tr_{i,j}^k(\mora))=\begin{bmatrix}
            \mat{A}_\morb & \mat{0}                                                                                    \\
            \mat{0}       & \mat{A}_\mora + \rowscols{\mat{B}_\mora}{:}{i+1:i+k} \rowscols{\mat{C}_\mora}{j+1:j+k}{:},
        \end{bmatrix}
    \end{equation*}
    %
    \begin{equation*}
        \projC(\morb \mtimescatmor \Tr_{i,j}^k(\mora))=\begin{bmatrix}
            \mat{B}_\morb & \mat{0}                          \\
            \mat{0}       & \rowscols{\mat{B}_\mora}{:}{1:i}
        \end{bmatrix},
    \end{equation*}
    and
    \begin{equation*}
        \projD(\morb \mtimescatmor \Tr_{i,j}^k(\mora))=
        \begin{bmatrix}
            \mat{C}_\morb & \mat{0}                          \\
            \mat{0}       & \rowscols{\mat{C}_\mora}{1:j}{:}
        \end{bmatrix},
    \end{equation*}
    which clearly correspond to the left-hand side.
\end{example}
%

\devel{
    \begin{example}
        \LTI with the trace as previously defined, satifies the trace axioms.

        We prove the statements one by one.

        \textbf{Naturality in $i$:}
        Consider the LTI systems $\mora\colon i+k\mto j+k$ and $\morb\colon m\mto i$.
        We need to check:
        \begin{equation*}
            \underbrace{\Tr_{m,j}^k((\morb \mtimescatmor \catid_k)\mthen \mora)}_{(1)}=\morb \mthen \Tr_{i,j}^k(\mora).
        \end{equation*}

        We start developing the left-hand side of the equation, referred to as (1).
        To do so, we first write $\morb \mtimescatmor \catid_k\colon m+k\mto i+k$.
        \begin{equation*}
            \morb \mtimescatmor \catid_k=\tupp{\prstart_\morb, \mat{A}_\morb, \begin{bmatrix} \mat{B}_\morb & \mat{0}\end{bmatrix},\begin{bmatrix} \mat{C}_\morb \\ \mat{0}\end{bmatrix}, \begin{bmatrix} \mat{D}_\morb & \mat{0}\\ \mat{0}&\idmat^{k\times k}\end{bmatrix}}.
        \end{equation*}

        Furthermore, we have:
        \begin{widepar}
            \begin{equation*}
                (\morb \mtimescatmor \catid_k)\mthen \mora =\tupp{\begin{bmatrix}\prstart_\morb\\ \prstart_\mora \end{bmatrix},
                    \begin{bmatrix} \mat{A}_\morb&\mat{0}\\ \mat{B}_\mora \begin{bmatrix} \mat{C}_\morb \\ \mat{0}\end{bmatrix}&\mat{A}_\mora \end{bmatrix},
                    \begin{bmatrix}\mat{B}_\morb & \mat{0}\\\rowscols{\mat{B}_\mora}{:}{1:i}\mat{D}_\morb &\rowscols{\mat{B}_\mora}{:}{i+1:i+k} \end{bmatrix},
                    \begin{bmatrix}\rowscols{\mat{D}_\mora}{:}{1:i}\mat{C}_\morb&\mat{C}_\mora\end{bmatrix},
                    \begin{bmatrix} \rowscols{\mat{D}_\mora}{:}{1:i}\mat{D}_\morb & \rowscols{\mat{D}_\mora}{:}{i+1:i+k}\end{bmatrix}}.
            \end{equation*}
        \end{widepar}

        With these intermediate calculations, we can start looking at (1) using the trace formula, component by component.
        Clearly, we have
        \begin{equation*}
            \projA ((1))=\begin{bmatrix}\prstart_\morb\\ \prstart_\mora \end{bmatrix}.
        \end{equation*}
        Furthermore, we have
        \begin{widepar}
            \begin{equation}
                \label{eq:lti-nat-1}
                \begin{aligned}
                    \projB ((1)) & =\begin{bmatrix} \mat{A}_\morb&\mat{0}\\ \mat{B}_\mora \begin{bmatrix} \mat{C}_\morb \\ \mat{0}\end{bmatrix}&\mat{A}_\mora \end{bmatrix}
                    +\rowscols{\begin{bmatrix}\mat{B}_\morb & \mat{0}\\\rowscols{\mat{B}_\mora}{:}{1:i}\mat{D}_\morb &\rowscols{\mat{B}_\mora}{:}{i+1:i+k} \end{bmatrix}}{:}{m+1:m+k}
                    \mat{F}
                    \rowscols{\begin{bmatrix}\rowscols{\mat{D}_\mora}{:}{1:i}\mat{C}_\morb&\mat{C}_\mora\end{bmatrix}}{j+1:j+k}{:},
                \end{aligned}
            \end{equation}
        \end{widepar}
        where we parametrize
        \begin{equation*}
            \mat{F} = \left(\idmat-\rowscols{\begin{bmatrix} \rowscols{\mat{D}_\mora}{:}{1:i}\mat{D}_\morb & \rowscols{\mat{D}_\mora}{:}{i+1:i+k}\end{bmatrix}}{j+1:j+k}{m+1:m+k}\right)^{-1},
        \end{equation*}
        for brevity.
        We can simplify \cref{eq:lti-nat-1} by observing that $\mat{B}_\morb$ has $m$ columns, $\mat{B}_\mora$ has $i+k$ columns, $\mat{D}_\mora$ has $j+k$ roes and $i+k$ columns, and $\mat{C}_\morb$ has $i$ rows.
        We get:
        \begin{equation}
            \label{eq:lti-nat-2}
            \begin{aligned}
                 & \projB ((1)) \\
                 & =\begin{bmatrix} \mat{A}_\morb&\mat{0}\\ \mat{B}_\mora \begin{bmatrix} \mat{C}_\morb \\ \mat{0}\end{bmatrix}&\mat{A}_\mora \end{bmatrix}
                +\begin{bmatrix}\mat{0}\\ \rowscols{\mat{B}_\mora}{:}{i+1:i+k} \end{bmatrix}
                \mat{F}
                \rowscols{\begin{bmatrix}\rowscols{\mat{D}_\mora}{:}{1:i}\mat{C}_\morb&\mat{C}_\mora\end{bmatrix}}{j+1:j+k}{:} \\
                 & =\begin{bmatrix} \mat{A}_\morb&\mat{0}\\ \rowscols{\mat{B}_\mora}{:}{1:i}\mat{C}_\morb &\mat{A}_\mora \end{bmatrix}
                +\begin{bmatrix}\mat{0}\\ \rowscols{\mat{B}_\mora}{:}{i+1:i+k} \end{bmatrix}
                \mat{F}
                \rowscols{\begin{bmatrix}\rowscols{\mat{D}_\mora}{:}{1:i}\mat{C}_\morb&\mat{C}_\mora\end{bmatrix}}{j+1:j+k}{:} \\
                 & =\begin{bmatrix} \mat{A}_\morb                                 & \mat{0}       \\
                \rowscols{\mat{B}_\mora}{:}{1:i}\mat{C}_\morb & \mat{A}_\mora\end{bmatrix}
                +\begin{bmatrix}\mat{0}\\ \rowscols{\mat{B}_\mora}{:}{i+1:i+k} \end{bmatrix}
                \mat{F'}
                \begin{bmatrix}\rowscols{\mat{D}_\mora}{j+1:j+k}{1:i}\mat{C}_\morb&\rowscols{\mat{C}_\mora}{j+1:j+k}{:}\end{bmatrix} \\
                 & =\begin{bmatrix}
                        \mat{A}_\morb & \mat{0}                                                                                        \\
                        \rowscols{\mat{B}_\mora}{:}{1:i}\mat{C}_\morb+
                        \rowscols{\mat{B}_\mora}{:}{i+1:i+k}\mat{F'} \rowscols{\mat{D}_\mora}{j+1:j+k}{1:i}\mat{C}_\morb
                                      & \mat{A}_\mora+\rowscols{\mat{B}_\mora}{:}{i+1:i+k}\mat{F'}\rowscols{\mat{C}_\mora}{j+1:j+k}{:}
                    \end{bmatrix},
            \end{aligned}
        \end{equation}
        with
        \begin{equation*}
            \mat{F'} = \left(\idmat-\rowscols{\mat{D}_\mora}{j+1:j+k}{i+1:i+k}\right)^{-1}.
        \end{equation*}
        Furthermore, we have
        \begin{equation*}
            \begin{aligned}
                \projC((1)) & =
                \begin{bmatrix} \mat{B}_\morb \\ \rowscols{\mat{B}_\mora}{:}{1:i} \mat{D}_\morb \end{bmatrix}
                + \begin{bmatrix} \mat{0}\\ \rowscols{\mat{B}_\mora}{:}{i+1:i+k}\end{bmatrix}
                \mat{F'}\rowscols{\mat{D}_\mora}{j+1:j+k}{1:i}\mat{D}_\morb,
            \end{aligned}
        \end{equation*}
        %
        \begin{equation*}
            \begin{aligned}
                 & \projD((1)) \\
                 & =
                \begin{bmatrix} \rowscols{\mat{D}_\mora}{1:j}{1:i}\mat{C}_\morb&\rowscols{\mat{C}_\mora}{1:j}{:} \end{bmatrix}
                +\rowscols{\mat{D}_\mora}{1:j}{i+1:i+k}\mat{F'}
                \begin{bmatrix}
                    \rowscols{\mat{D}_\mora}{j+1:j+k}{1:i}\mat{C}_\morb & \rowscols{\mat{C}_\mora}{j+1:j+k}{:}
                \end{bmatrix},
            \end{aligned}
        \end{equation*}
        and
        \begin{equation*}
            \begin{aligned}
                \projE((1)) & =\rowscols{\mat{D}_\mora}{:}{1:i}\mat{D}_\morb + \rowscols{\mat{D}_\mora}{:}{i+1:i+k}\mat{F'}\rowscols{\mat{D}_\mora}{j+1:j+k}{i+1:i+k}.
            \end{aligned}
        \end{equation*}

        We are now ready to look at the right-hand side of the axiom to prove, which we refer to as (2).
        First, applying the definition of trace, we have:
        \begin{equation*}
            \begin{aligned}
                \Tr_{i,j}^{k}(\mora)=\langle\prstart_\mora, & \mat{A}_\mora +\rowscols{\mat{B}_\mora}{:}{i+1:i+k}\mat{F'}\rowscols{\mat{C}_\mora}{j+1:j+k}{:}, \\
                                                            & \rowscols{\mat{B}_\mora}{:}{1:i}+\rowscols{\mat{B}_\mora}{:}{i+1:i+k}\mat{F'}\rowscols{\mat{D}_\mora}{j+1:j+k}{1:i}, \\
                                                            & \rowscols{\mat{C}_\mora}{1:j}{:}+ \rowscols{\mat{D}_\mora}{1:j}{i+1:i+k} \mat{F'} \rowscols{\mat{C}_\mora}{j+1:j+k}{:}, \\
                                                            & \rowscols{\mat{D}_\mora}{1:j}{1:i}+ \rowscols{\mat{D}_\mora}{1:j}{i+1:i+k}\mat{F'}\rowscols{\mat{D}_\mora}{j+1:j+k}{i+1:i+k}\rangle
            \end{aligned}
        \end{equation*}
        Now, we can compute (2) component by component.
        We have:
        \begin{equation*}
            \projA ((2))=\begin{bmatrix}\prstart_\morb\\ \prstart_\mora \end{bmatrix}= \projA((1)).
        \end{equation*}
        Furthermore, we have
        \begin{equation*}
            \begin{aligned}
                 & \projB((2)) \\
                 & =\begin{bmatrix}
                        \mat{A}_\morb & \mat{0}                                                                                        \\
                        \rowscols{\mat{B}_\mora}{:}{1:i}\mat{C}_\morb+
                        \rowscols{\mat{B}_\mora}{:}{i+1:i+k}\mat{F'} \rowscols{\mat{D}_\mora}{j+1:j+k}{1:i}\mat{C}_\morb
                                      & \mat{A}_\mora+\rowscols{\mat{B}_\mora}{:}{i+1:i+k}\mat{F'}\rowscols{\mat{C}_\mora}{j+1:j+k}{:}
                    \end{bmatrix} \\
                 & =\projB((1)).
            \end{aligned}
        \end{equation*}

        \begin{equation*}
            \begin{aligned}
                \projC((2)) & =
                \begin{bmatrix} \mat{B}_\morb \\ \rowscols{\mat{B}_\mora}{:}{1:i} \mat{D}_\morb \end{bmatrix}
                + \begin{bmatrix} \mat{0}\\ \rowscols{\mat{B}_\mora}{:}{i+1:i+k}\end{bmatrix}
                \mat{F'}\rowscols{\mat{D}_\mora}{j+1:j+k}{1:i}\mat{D}_\morb \\
                            & =\projC((1)),
            \end{aligned}
        \end{equation*}
        %
        \begin{equation*}
            \begin{aligned}
                 & \projD((2)) \\
                 & =
                \begin{bmatrix} \rowscols{\mat{D}_\mora}{1:j}{1:i}\mat{C}_\morb&\rowscols{\mat{C}_\mora}{1:j}{:} \end{bmatrix}
                +\rowscols{\mat{D}_\mora}{1:j}{i+1:i+k}\mat{F'}
                \begin{bmatrix}
                    \rowscols{\mat{D}_\mora}{j+1:j+k}{1:i}\mat{C}_\morb & \rowscols{\mat{C}_\mora}{j+1:j+k}{:}
                \end{bmatrix} \\
                 & =\projD((2)),
            \end{aligned}
        \end{equation*}
        and
        \begin{equation*}
            \begin{aligned}
                \projE((2)) & =\rowscols{\mat{D}_\mora}{:}{1:i}\mat{D}_\morb + \rowscols{\mat{D}_\mora}{:}{i+1:i+k}\mat{F'}\rowscols{\mat{D}_\mora}{j+1:j+k}{i+1:i+k} \\
                            & =\projE((1)),
            \end{aligned}
        \end{equation*}
        proving naturality in $i$.

        \textbf{Naturality in $j$:}
        Consider the LTI systems $\mora\colon i+k\mto j+k$ and $\morb\colon j\mto m$.
        We need to check:
        \begin{equation*}
            \underbrace{\Tr_{i,m}^k(\mora \mthen (\morb \mtimescatmor \catid_k))}_{(1)}=\Tr_{i,j}^k(\mora)\mthen \morb.
        \end{equation*}
        We start developing the left-hand side of the equation, referred to as (1).
        To do so, we first write $\morb \mtimescatmor \catid_k\colon j+k\mto m+k$:
        \begin{equation*}
            \morb \mtimescatmor \catid_k=\tupp{\prstart_\morb, \mat{A}_\morb, \begin{bmatrix} \mat{B}_\morb & \mat{0}\end{bmatrix},\begin{bmatrix} \mat{C}_\morb \\ \mat{0}\end{bmatrix}, \begin{bmatrix} \mat{D}_\morb & \mat{0}\\ \mat{0}&\idmat^{k\times k}\end{bmatrix}}.
        \end{equation*}
        Furthermore, we have:
        \begin{equation*}
            \begin{aligned}
                \mora \mthen (\morb \mtimescatmor \catid_k) =\langle\begin{bmatrix}\prstart_\mora\\ \prstart_\morb \end{bmatrix},
                 & \begin{bmatrix} \mat{A}_\mora&\mat{0}\\ \mat{B}_\morb \rowscols{\mat{C}_\mora}{1:j}{:} &\mat{A}_\morb \end{bmatrix},
                \begin{bmatrix}\mat{B}_\mora \\ \mat{B}_\morb \rowscols{\mat{D}_\mora}{1:j}{:} \end{bmatrix}, \\
                 & \begin{bmatrix}\begin{bmatrix}\mat{D}_\morb \rowscols{\mat{C}_\mora}{1:j}{:}\\ \rowscols{\mat{C}_\mora}{j+1:j+k}{:}\end{bmatrix}& \begin{bmatrix} \mat{C}_\morb \\ \mat{0}\end{bmatrix} \end{bmatrix},
                \begin{bmatrix}\mat{D}_\morb \rowscols{\mat{D}_\mora}{1:j}{:}\\ \rowscols{\mat{D}_\mora}{j+1:j+k}{:}\end{bmatrix}
                \rangle
            \end{aligned}
        \end{equation*}
        With these intermediate calculations, we can start looking at (1) using the trace formula, component by component.
        Clearly, we have
        \begin{equation*}
            \projA ((1))=\begin{bmatrix}\prstart_\mora\\ \prstart_\morb \end{bmatrix}.
        \end{equation*}
        Furthermore, we have
        \begin{equation}
            \label{eq:lti-natbis-1}
            \begin{aligned}
                \projB ((1)) & =\begin{bmatrix} \mat{A}_\mora&\mat{0}\\ \mat{B}_\morb \rowscols{\mat{C}_\mora}{1:j}{:} &\mat{A}_\morb \end{bmatrix}
                + \begin{bmatrix}\rowscols{\mat{B}_\mora}{:}{i+1:i+k} \\ \rowscols{\begin{bmatrix}\mat{B}_\morb \rowscols{\mat{D}_\mora}{1:j}{:}\end{bmatrix}}{:}{i+1:i+k} \end{bmatrix} \mat{F}\begin{bmatrix} \rowscols{\mat{C}_\mora}{j+1:j+k}{:}&\mat{0}\end{bmatrix} \\
                             & =\begin{bmatrix}
                                    \mat{A}_\mora +   \rowscols{\mat{B}_\mora}{:}{i+1:i+k}\mat{F}\rowscols{\mat{C}_\mora}{j+1:j+k}{:}                                                                                                            & \mat{0}        \\
                                    \mat{B}_\morb \rowscols{\mat{C}_\mora}{1:j}{:}+ \rowscols{\begin{bmatrix}\mat{B}_\morb \rowscols{\mat{D}_\mora}{1:j}{:}\end{bmatrix}}{:}{i+1:i+k}\mat{F}\rowscols{\mat{C}_\mora}{j+1:j+k}{:} & \mat{A}_\morb,
                                \end{bmatrix}
            \end{aligned}
        \end{equation}
        where we parametrize
        \begin{equation*}
            \mat{F} = \left(\idmat-\rowscols{\mat{D}_\mora}{j+1:j+k}{i+1:i+k}\right)^{-1},
        \end{equation*}
        for brevity.

        Furthermore, we have
        \begin{equation*}
            \begin{aligned}
                \projC((1)) & = \begin{bmatrix}\rowscols{\mat{B}_\mora}{:}{1:i} \\ \rowscols{\begin{bmatrix}\mat{B}_\morb \rowscols{\mat{D}_\mora}{1:j}{:}\end{bmatrix}}{:}{1:i} \end{bmatrix}+ \begin{bmatrix}\rowscols{\mat{B}_\mora}{:}{i+1:i+k} \\ \rowscols{\begin{bmatrix}\mat{B}_\morb \rowscols{\mat{D}_\mora}{1:j}{:}\end{bmatrix}}{:}{i+1:i+k} \end{bmatrix}\mat{F}\rowscols{\begin{bmatrix}\mat{D}_\morb \rowscols{\mat{D}_\mora}{1:j}{:}\\ \rowscols{\mat{D}_\mora}{j+1:j+k}{:}\end{bmatrix}}{m+1:m+k}{1:i} \\
                            & =\begin{bmatrix}\rowscols{\mat{B}_\mora}{:}{1:i} \\ \rowscols{\begin{bmatrix}\mat{B}_\morb \rowscols{\mat{D}_\mora}{1:j}{:}\end{bmatrix}}{:}{1:i} \end{bmatrix}+ \begin{bmatrix}\rowscols{\mat{B}_\mora}{:}{i+1:i+k} \\ \rowscols{\begin{bmatrix}\mat{B}_\morb \rowscols{\mat{D}_\mora}{1:j}{:}\end{bmatrix}}{:}{i+1:i+k} \end{bmatrix}\mat{F}\rowscols{\mat{D}_\mora}{j+1:j+k}{1:i}
            \end{aligned}
        \end{equation*}
        %
        \begin{equation*}
            \begin{aligned}
                \projD((1)) & = \begin{bmatrix} \mat{D}_\morb\rowscols{\mat{C}_\mora}{1:j}{:}&\mat{C}_\morb\end{bmatrix}+\rowscols{\begin{bmatrix}\mat{D}_\morb \rowscols{\mat{D}_\mora}{1:j}{:}\end{bmatrix}}{:}{i+1:i+k} \mat{F}\begin{bmatrix} \rowscols{\mat{C}_\mora}{j+1:j+k}{:}&\mat{0}\end{bmatrix},
            \end{aligned}
        \end{equation*}
        and
        \begin{equation*}
            \begin{aligned}
                \projE((1)) & = \rowscols{\begin{bmatrix}\mat{D}_\morb \rowscols{\mat{D}_\mora}{1:j}{:}\end{bmatrix}}{:}{1:i}+\rowscols{\begin{bmatrix}\mat{D}_\morb \rowscols{\mat{D}_\mora}{1:j}{:}\end{bmatrix}}{:}{i+1:i+k}\mat{F}\rowscols{\mat{D}_\mora}{j+1:j+k}{1:i}.
            \end{aligned}
        \end{equation*}
        We are now ready to look at the right-hand side of the axiom to prove, which we refer to as (2).
        First, applying the definition of trace, we have:
        \begin{equation*}
            \begin{aligned}
                \Tr_{i,j}^{k}(\mora)=\langle\prstart_\mora, & \mat{A}_\mora + \rowscols{\mat{B}_\mora}{:}{i+1:i+k}\mat{F}\rowscols{\mat{C}_\mora}{j+1:j+k}{:}, \\
                                                            & \rowscols{\mat{B}_\mora}{:}{1:i}+\rowscols{\mat{B}_\mora}{:}{i+1:i+k}\mat{F}\rowscols{\mat{D}_\mora}{j+1:j+k}{1:i}, \\
                                                            & \rowscols{\mat{C}_\mora}{1:j}{:}+ \rowscols{\mat{D}_\mora}{1:j}{i+1:i+k} \mat{F} \rowscols{\mat{C}_\mora}{j+1:j+k}{:}, \\
                                                            & \rowscols{\mat{D}_\mora}{1:j}{1:i}+ \rowscols{\mat{D}_\mora}{1:j}{i+1:i+k}\mat{F}\rowscols{\mat{D}_\mora}{j+1:j+k}{1:i}\rangle.
            \end{aligned}
        \end{equation*}
        Now, we can compute (2) component by component.
        Clearly we have:
        \begin{equation*}
            \projA((2))=\begin{bmatrix} \prstart_\mora \\ \prstart_\morb\end{bmatrix}=\projA((1)).
        \end{equation*}
        Furthermore, we have
        \begin{equation*}
            \begin{aligned}
                \projB((2)) & =\begin{bmatrix}
                                   \mat{A}_\mora + \rowscols{\mat{B}_\mora}{:}{i+1:i+k}\mat{F}\rowscols{\mat{C}_\mora}{j+1:j+k}{:}                                                   & \mat{0}       \\
                                   \mat{B}_\morb \left( \rowscols{\mat{C}_\mora}{1:j}{:}+ \rowscols{\mat{D}_\mora}{1:j}{i+1:i+k} \mat{F} \rowscols{\mat{C}_\mora}{j+1:j+k}{:}\right) & \mat{A}_\morb
                               \end{bmatrix} \\
                            & =\projB((1)),
            \end{aligned}
        \end{equation*}
        %
        \begin{equation*}
            \begin{aligned}
                \projC((2)) & =\begin{bmatrix} \rowscols{\mat{B}_\mora}{:}{1:i}+\rowscols{\mat{B}_\mora}{:}{i+1:i+k}\mat{F}\rowscols{\mat{D}_\mora}{j+1:j+k}{1:i} \\ \mat{B}_\morb \left( \rowscols{\mat{D}_\mora}{1:j}{1:i}+ \rowscols{\mat{D}_\mora}{1:j}{i+1:i+k}\mat{F}\rowscols{\mat{D}_\mora}{j+1:j+k}{1:i}\right)\end{bmatrix} \\
                            & =\projC((1)),
            \end{aligned}
        \end{equation*}
        %
        \begin{equation*}
            \begin{aligned}
                \projD((2)) & =\begin{bmatrix}\mat{D}_\morb \left(\rowscols{\mat{C}_\mora}{1:j}{:}+ \rowscols{\mat{D}_\mora}{1:j}{i+1:i+k} \mat{F} \rowscols{\mat{C}_\mora}{j+1:j+k}{:}\right)&\mat{C}_\morb \end{bmatrix} \\
                            & =\projD((1)),
            \end{aligned}
        \end{equation*}
        and
        \begin{equation*}
            \begin{aligned}
                \projE((2)) & =\mat{D}_\morb\left(\rowscols{\mat{D}_\mora}{1:j}{1:i}+ \rowscols{\mat{D}_\mora}{1:j}{i+1:i+k}\mat{F}\rowscols{\mat{D}_\mora}{j+1:j+k}{1:i} \right) \\
                            & =\projE((1)),
            \end{aligned}
        \end{equation*}
        proving naturality in $j$.

        \textbf{Dinaturality in $k$:}
        Consider LTI systems $\mora\colon i+k\mto j+m$ and $\morb\colon m\mto k$.
        We need to show:
        \begin{equation*}
            \underbrace{\Tr_{i, j}^{k} (\mora \mthen (\catidat j \mtimescatmor \morb))}_{(1)} = \underbrace{\Tr_{i, j}^{m}((\catidat i \mtimescatmor \morb) \mthen \mora)}_{(2)}.
        \end{equation*}
        We start developing the left-hand side of the equation, referred to as (1).
        To do so, we first write $\catid_j \mtimescatmor \morb \colon j+m\mto j+k$.
        \begin{equation*}
            \catid_j \mtimescatmor \morb =\tupp{\prstart_\morb, \mat{A}_\morb, \begin{bmatrix} \mat{0}&\mat{B}_\morb\end{bmatrix},\begin{bmatrix}  \mat{0}\\ \mat{C}_\morb\end{bmatrix}, \begin{bmatrix}\idmat^{j\times j}  & \mat{0}\\ \mat{0}&\mat{D}_\morb \end{bmatrix}}.
        \end{equation*}
        Furthermore, we have:
        \begin{widepar}
            \begin{equation*}
                \mora \mthen (\catid_j \mtimescatmor \morb )=\tup{\begin{bmatrix} \prstart_\mora \\ \prstart_\morb \end{bmatrix},
                    \begin{bmatrix} \mat{A}_\mora                                     & \mat{0}       \\
                \mat{B}_\morb\rowscols{\mat{C}_\mora}{j+1:j+m}{:} & \mat{A}_\morb
                    \end{bmatrix},
                    \begin{bmatrix}
                        \mat{B}_\mora \\
                        \mat{B}_\morb \rowscols{\mat{D}_\mora}{j+1:j+m}{:}
                    \end{bmatrix},
                    \begin{bmatrix}
                        \rowscols{\mat{C}_\mora}{1:j}{:}                   & \mat{0}       \\
                        \mat{D}_\morb \rowscols{\mat{C}_\mora}{j+1:j+m}{:} & \mat{C}_\morb
                    \end{bmatrix},
                    \begin{bmatrix}
                        \rowscols{\mat{D}_\mora}{1:j}{:} \\
                        \mat{D}_\morb \rowscols{\mat{D}_\mora}{j+1:j+m}{:}
                    \end{bmatrix}}
            \end{equation*}
        \end{widepar}
        With these intermediate calculations, we can start looking at (1) using the trace formula, component by component.
        Clearly, we have
        \begin{equation*}
            \projA((1))=\begin{bmatrix} \prstart_\mora \\ \prstart_\morb \end{bmatrix}.
        \end{equation*}
        Furthermore, we have
        \begin{widepar}
            \begin{equation*}
                \begin{aligned}
                    \projB((1)) & =\begin{bmatrix} \mat{A}_\mora                                     & \mat{0}       \\
                \mat{B}_\morb\rowscols{\mat{C}_\mora}{j+1:j+m}{:} & \mat{A}_\morb
                                   \end{bmatrix} + \begin{bmatrix}
                                                       \rowscols{\mat{B}_\mora}{:}{i+1:i+k} \\
                                                       \rowscols{\begin{bmatrix}\mat{B}_\morb \rowscols{\mat{D}_\mora}{j+1:j+m}{:}\end{bmatrix}}{:}{i+1:i+k}
                                                   \end{bmatrix}\mat{F}
                    \begin{bmatrix}
                        \mat{D}_\morb \rowscols{\mat{C}_\mora}{j+1:j+m}{:} & \mat{C}_\morb
                    \end{bmatrix} \\
                                & =\begin{bmatrix}
                                       \mat{A}_\mora +\rowscols{\mat{B}_\mora}{:}{i+1:i+k}\mat{F}\mat{D}_\morb \rowscols{\mat{C}_\mora}{j+1:j+m}{:} & \rowscols{\mat{B}_\mora}{:}{i+1:i+k}\mat{F}\mat{C}_\morb                                                                                                \\
                                       \mat{B}_\morb\rowscols{\mat{C}_\mora}{j+1:j+m}{:}+\rowscols{\begin{bmatrix}\mat{B}_\morb \rowscols{\mat{D}_\mora}{j+1:j+m}{:}\end{bmatrix}}{:}{i+1:i+k}\mat{F}\mat{D}_\morb \rowscols{\mat{C}_\mora}{j+1:j+m}{:}
                                                                                                                                                    & \mat{A}_\morb+\rowscols{\begin{bmatrix}\mat{B}_\morb \rowscols{\mat{D}_\mora}{j+1:j+m}{:}\end{bmatrix}}{:}{i+1:i+k}\mat{F}\mat{C}_\morb
                                   \end{bmatrix}
                \end{aligned}
            \end{equation*}
        \end{widepar}
        where
        \begin{equation*}
            \mat{F}=\left(\idmat - \rowscols{\begin{bmatrix}
                    \mat{D}_\morb \rowscols{\mat{D}_\mora}{j+1:j+m}{:}
                \end{bmatrix}}{:}{i+1:i+k}\right)^{-1}.
        \end{equation*}
        %
        Furthermore, we have:
        \begin{widepar}
            \begin{equation*}
                \begin{aligned}
                    \projC((1)) & =\begin{bmatrix}
                                       \rowscols{\mat{B}_\mora}{:}{1:i} \\
                                       \rowscols{\begin{bmatrix}\mat{B}_\morb \rowscols{\mat{D}_\mora}{j+1:j+m}{:}\end{bmatrix}}{:}{1:i}
                                   \end{bmatrix}+ \begin{bmatrix}
                                                      \rowscols{\mat{B}_\mora}{:}{i+1:i+k} \\
                                                      \rowscols{\begin{bmatrix}\mat{B}_\morb \rowscols{\mat{D}_\mora}{j+1:j+m}{:}\end{bmatrix}}{:}{i+1:i+k}
                                                  \end{bmatrix}\mat{F}
                    \rowscols{\begin{bmatrix}
                                      \mat{D}_\morb \rowscols{\mat{D}_\mora}{j+1:j+m}{:}
                                  \end{bmatrix}}{:}{1:i} \\
                                & =\begin{bmatrix}
                                       \rowscols{\mat{B}_\mora}{:}{1:i}+\rowscols{\mat{B}_\mora}{:}{i+1:i+k}\mat{F}\rowscols{\begin{bmatrix}\mat{D}_\morb \rowscols{\mat{D}_\mora}{j+1:j+m}{:}\end{bmatrix}}{:}{1:i}
                                       \\
                                       \rowscols{\begin{bmatrix}\mat{B}_\morb \rowscols{\mat{D}_\mora}{j+1:j+m}{:}\end{bmatrix}}{:}{1:i}+\rowscols{\begin{bmatrix}\mat{B}_\morb \rowscols{\mat{D}_\mora}{j+1:j+m}{:}\end{bmatrix}}{:}{i+1:i+k}\mat{F}\rowscols{\begin{bmatrix}\mat{D}_\morb \rowscols{\mat{D}_\mora}{j+1:j+m}{:}\end{bmatrix}}{:}{1:i}

                                   \end{bmatrix},
                \end{aligned}
            \end{equation*}
        \end{widepar}
        %
        \begin{equation*}
            \begin{aligned}
                \projD((1)) & =\begin{bmatrix}
                                   \rowscols{\mat{C}_\mora}{1:j}{:} & \mat{0}
                               \end{bmatrix}+
                \rowscols{\mat{D}_\mora}{1:j}{i+1:i+k}\mat{F}
                \begin{bmatrix}
                    \mat{D}_\morb \rowscols{\mat{C}_\mora}{j+1:j+m}{:} & \mat{C}_\morb
                \end{bmatrix} \\
                            & =\begin{bmatrix}
                                   \rowscols{\mat{C}_\mora}{1:j}{:}+\rowscols{\mat{D}_\mora}{1:j}{i+1:i+k}\mat{F}\mat{D}_\morb \rowscols{\mat{C}_\mora}{j+1:j+m}{:}
                                    & \rowscols{\mat{D}_\mora}{1:j}{i+1:i+k}\mat{F}\mat{C}_\morb
                               \end{bmatrix},
            \end{aligned}
        \end{equation*}
        and
        \begin{equation*}
            \begin{aligned}
                \projE((1)) & =\rowscols{\mat{D}_\mora}{1:j}{1:i}+\rowscols{\mat{D}_\mora}{1:j}{i+1:i+k}\mat{F}\rowscols{\begin{bmatrix}
                                                                                                                                 \mat{D}_\morb \rowscols{\mat{D}_\mora}{j+1:j+m}{:}
                                                                                                                             \end{bmatrix}}{:}{1:i}
            \end{aligned}
        \end{equation*}

        We are now ready to look at the right-hand side of the axiom to prove, which we refer to as (2).
        First, we have:
        \begin{equation*}
            \catid_i \mtimescatmor \morb =\tupp{\prstart_\morb, \mat{A}_\morb, \begin{bmatrix} \mat{0}&\mat{B}_\morb\end{bmatrix},\begin{bmatrix}  \mat{0}\\ \mat{C}_\morb\end{bmatrix}, \begin{bmatrix}\idmat^{i\times i}  & \mat{0}\\ \mat{0}&\mat{D}_\morb \end{bmatrix}}.
        \end{equation*}
        Furthermore, we have:
        \begin{equation*}
            \begin{aligned}
                (\catid_i \mtimescatmor \morb)\mthen \mora =
                \langle\begin{bmatrix} \prstart_\morb \\ \prstart_\mora \end{bmatrix},
                 & \begin{bmatrix}\mat{A}_{\morb}                                   & \mat{0}         \\
               \rowscols{\mat{B}_\mora}{:}{i+1:i+k}\mat{C}_\morb & \mat{A}_{\mora}\end{bmatrix}, \\
                 & \begin{bmatrix}
                       \mat{0}                          & \mat{B}_\morb                                     \\
                       \rowscols{\mat{B}_\mora}{:}{1:i} & \rowscols{\mat{B}_\mora}{:}{i+1:i+k}\mat{D}_\morb
                   \end{bmatrix}, \\
                 & \begin{bmatrix}
                       \rowscols{\mat{D}_\mora}{:}{i+1:i+k}\mat{C}_\morb & \mat{C}_\mora
                   \end{bmatrix}, \\
                 & \begin{bmatrix}
                       \rowscols{\mat{D}_\mora}{:}{1:i} & \rowscols{\mat{D}_\mora}{:}{i+1:i+k}\mat{D}_\morb
                   \end{bmatrix}
                \rangle.
            \end{aligned}
        \end{equation*}
        With these intermediate steps, now we can look at (2).
        We have:
        \begin{equation*}
            \projA((2))=\begin{bmatrix} \prstart_\morb \\ \prstart_\mora \end{bmatrix}.
        \end{equation*}
        Furthermore, we have:
        %
        \begin{widepar}
            \begin{equation*}
                \begin{aligned}
                    \projB((2)) & =\begin{bmatrix}\mat{A}_{\morb}                                   & \mat{0}         \\
               \rowscols{\mat{B}_\mora}{:}{i+1:i+k}\mat{C}_\morb & \mat{A}_{\mora}\end{bmatrix}
                    +\begin{bmatrix}
                         \mat{B}_\morb \\
                         \rowscols{\mat{B}_\mora}{:}{i+1:i+k}\mat{D}_\morb
                     \end{bmatrix}\mat{E}\rowscols{\begin{bmatrix}
                                                           \rowscols{\mat{D}_\mora}{:}{i+1:i+k}\mat{C}_\morb & \mat{C}_\mora
                                                       \end{bmatrix}}{j+1:j+k}{:} \\
                                & =\begin{bmatrix}
                                       \mat{A}_\morb +\mat{B}_\morb \mat{E}\rowscols{\begin{bmatrix}\rowscols{\mat{D}_\mora}{:}{i+1:i+k}\mat{C}_\morb\end{bmatrix}}{j+1:j+k}{:}
                                        & \mat{B}_\morb \mat{E}\rowscols{\mat{C}_\mora}{j+1:j+k}{:}                                                    \\
                                       \rowscols{\mat{B}_\mora}{:}{i+1:i+k}\mat{C}_\morb+ \rowscols{\mat{B}_\mora}{:}{i+1:i+k}\mat{D}_\morb \mat{E}\rowscols{\begin{bmatrix}\rowscols{\mat{D}_\mora}{:}{i+1:i+k}\mat{C}_\morb\end{bmatrix}}{j+1:j+k}{:}
                                        & \mat{A}_\mora +\rowscols{\mat{B}_\mora}{:}{i+1:i+k}\mat{D}_\morb \mat{E}\rowscols{\mat{C}_\mora}{j+1:j+k}{:}
                                   \end{bmatrix}
                \end{aligned}
            \end{equation*}
        \end{widepar}
        where
        \begin{equation*}
            \mat{E}=\left(\idmat - \rowscols{\begin{bmatrix}
                    \rowscols{\mat{D}_\mora}{:}{i+1:i+k}\mat{D}_\morb
                \end{bmatrix}}{j+1:j+m}{i+1:i+m}\right)^{-1}.
        \end{equation*}

        \todo{type rest of the analytical derivation}
        The axiom does not hold on the nose.
        However, it works up to invertible linear transformations.
        For instance, consider the system $\mora \colon 2\mto 2$ with
        \begin{equation*}
            \mat{A}_\mora = \begin{bmatrix} 1&1\\ 1&1\end{bmatrix},\quad
            \mat{B}_\mora = \begin{bmatrix} 1&1\\ 1&1\end{bmatrix},\quad
            \mat{C}_\mora = \begin{bmatrix} 1&1\\ 1&1\end{bmatrix},\quad
            \mat{D}_\mora = \begin{bmatrix} 1&1\\ 1&0\end{bmatrix},\quad
        \end{equation*}
        and the system $\morb\colon 1\mto 1$ with
        \begin{equation*}
            \mat{A}_\morb=\mat{B}_\morb=\mat{C}_\morb=1,\quad \mat{D}_\morb=0.5.
        \end{equation*}
        In this case, we have:
        \begin{equation*}
            \projB((1))=\begin{bmatrix}1.5&1.5&1\\ 1.5&1.5&1\\ 1&1&1 \end{bmatrix}\neq
            \begin{bmatrix}1&1&1\\ 1&1.5&1.5\\ 1&1.5&1.5 \end{bmatrix}=\projB((2))
        \end{equation*}
        \textbf{Vanishing I:}
        Consider a LTI system $\mora\colon i\mto j$, given by $\genericlti{\mora}$ with state dimension $s\setin \natnumbers$.
        We need to check
        \begin{equation*}
            \Tr_{i,j}^0 (\mora)=\mora.
        \end{equation*}
        Using the trace definition, we can write:
        \begin{equation*}
            \begin{aligned}
                \Tr_{i,j}^0 (\mora)=\langle\prstart_{\mora}, & \mat{A}_\mora+\rowscols{\mat{B}_{\mora}}{:}{i+1:i+k}\mat{F}\rowscols{\mat{C}_{\mora}}{j+1:j+k}{:}, \\
                                                             & \rowscols{\mat{B}_{\mora}}{:}{1:i}+ \rowscols{\mat{B}_{\mora}}{:}{i+1:i+k}\mat{F}\rowscols{\mat{D}_{\mora}}{j+1:j+k}{1:i}, \\
                                                             & \rowscols{\mat{C}_{\mora}}{1:j}{:}+\rowscols{\mat{D}_{\mora}}{1:j}{i+1:i+k}\mat{F}\rowscols{\mat{C}_{\mora}}{j+1:j+k}{:}, \\
                                                             & \rowscols{\mat{D}_{\mora}}{1:j}{1:i}+ \rowscols{\mat{D}_{\mora}}{1:j}{i+1:i+k}\mat{F}\rowscols{\mat{D}_{\mora}}{j+1:j+k}{1:i}\rangle,
            \end{aligned}
        \end{equation*}
        where
        \begin{equation*}
            \mat{F}=\left(\idmat - \rowscols{\mat{D}_{\mora}}{j+1:j+k}{i+1:i+k}\right)^{-1}.
        \end{equation*}
        However, we have:
        \begin{equation*}
            \mat{B}_\mora=\begin{bmatrix}
                \rowscols{\mat{B}_{\mora}}{:}{1:i} & \mat{0}^{s\times 0}
            \end{bmatrix},\quad
            \mat{C}_\mora=\begin{bmatrix}
                \rowscols{\mat{C}_{\mora}}{1:j}{:} \\ \mat{0}^{0\times s}
            \end{bmatrix},
            \quad
            \mat{D}_\mora= \rowscols{\mat{D}_\mora}{1:j}{1:i}
        \end{equation*}
        and therefore
        \begin{equation*}
            \Tr_{i,j}^0 (\mora)=\tup{\prstart{\mora}, \mat{A}_\mora, \mat{B}_\mora, \mat{C}_\mora, \mat{D}_\mora}=\mora.
        \end{equation*}

        \textbf{Vanishing II:}
        Consider a LTI system $\mora\colon i+k+o\mto j+k+o$, given by $\genericlti{\mora}$ with state dimension $s\setin \natnumbers$.
        We need to check
        \begin{equation*}
            \underbrace{\Tr_{i,j}^{k+o} (\mora)}_{(1)}=\underbrace{\Tr_{i,j}^{k}\left(\Tr_{i+k,j+k}^o(\mora)\right)}_{(2)}.
        \end{equation*}
        \todo{continue}
        We now look at the general formulas to see in which cases this does not work.
        Let's start with the left-hand side of the statement.
        We have:
        \begin{equation}
            \label{eq:vanishing-ii-left-lti}
            \begin{aligned}
                \Tr_{i,j}^{k+o} (\mora)=\langle\prstart,\mat{A}_\mora + & \rowscols{\mat{B}_\mora}{:}{i+1:i+k+o}\mat{E} \rowscols{\mat{C}_\mora}{j+1:j+k+o}{:}, \\
                                                                        & \rowscols{\mat{B}_\mora}{:}{1:i}+\rowscols{\mat{B}_\mora}{:}{i+1:i+k+o}\mat{E}\rowscols{\mat{D}_\mora}{j+1:j+k+o}{1:i}, \\
                                                                        & \rowscols{\mat{C}_\mora}{1:j}{:}+\rowscols{\mat{D}_\mora}{1:j}{i+1:i+k+o}\mat{E}\rowscols{\mat{C}_\mora}{j+1:j+k+o}{:}, \\
                                                                        & \rowscols{\mat{D}_\mora}{1:j}{1:i}+\rowscols{\mat{D}_\mora}{1:j}{i+1:i+k+o}\mat{E} \rowscols{\mat{D}_\mora}{j+1:j+k+o}{1:i}\rangle,
            \end{aligned}
        \end{equation}
        where $\mat{E}=(\idmat-\rowscols{\mat{D}_\mora}{j+1:j+k+o}{i+1:i+k+o})^{-1}$.
        Let's look at the right-hand side.
        We have:
        \begin{equation}
            \label{eq:vanishing-ii-left-lti-bis}
            \begin{aligned}
                \Tr_{i+k,j+k}^{o}(\mora)=\langle & \prstart,\mat{A}_\mora + \rowscols{\mat{B}_\mora}{:}{i+k+1:i+k+o}\mat{F} \rowscols{\mat{C}_\mora}{j+k+1:j+k+o}{:}, \\
                                                 & \rowscols{\mat{B}_\mora}{:}{1:i+k}+\rowscols{\mat{B}_\mora}{:}{i+k+1:i+k+o}\mat{F}\rowscols{\mat{D}_\mora}{j+k+1:j+k+o}{1:i+k}, \\
                                                 & \rowscols{\mat{C}_\mora}{1:j+k}{:}+\rowscols{\mat{D}_\mora}{1:j+k}{i+k+1:i+k+o}\mat{F}\rowscols{\mat{C}_\mora}{j+k+1:j+k+o}{:}, \\
                                                 & \rowscols{\mat{D}_\mora}{1:j+k}{1:i+k}+\rowscols{\mat{D}_\mora}{1:j+k}{i+k+1:i+k+o}\mat{F} \rowscols{\mat{D}_\mora}{j+k+1:j+k+o}{1:i+k}\rangle,
            \end{aligned}
        \end{equation}
        where $\mat{F}=(\idmat-\rowscols{\mat{D}_\mora}{j+k+1:j+k+o}{i+k+1:i+k+o})^{-1}$.
        which we can leverage to express $\Tr_{i,j}^{k}\left(\Tr_{i+k,j+k}^o(\mora)\right)=\ast$.
        Clearly, $\projA(\ast)=\prstart$.
        Furthermore:
        \begin{equation*}
            \begin{aligned}
                \projB(\ast)= & .
                ..
            \end{aligned}
        \end{equation*}
        \todo{finish, this not light}

        \textbf{Superposing:}
        Consider LTI systems $\mora\colon i+k\mto j+k$ and $\morb\colon m\mto l$.
        We need to check:
        \begin{equation*}
            \underbrace{\Tr_{m+i,l+j}^{k}(\morb \mtimescatmor \mora)}_{(1)}=\morb \mtimescatmor \Tr_{i,j}^k(\mora).
        \end{equation*}
        We start from the left-hand side, which we refer to as (1), and write:
        \begin{equation*}
            \morb \mtimescatmor \mora = \tup{\begin{bmatrix}\prstart_\morb\\ \prstart_\mora \end{bmatrix}, \begin{bmatrix} \mat{A}_\morb&\mat{0}\\ \mat{0}&\mat{A}_\mora\end{bmatrix},
                \begin{bmatrix} \mat{B}_\morb&\mat{0}\\ \mat{0}&\mat{B}_\mora\end{bmatrix},\begin{bmatrix} \mat{C}_\morb&\mat{0}\\ \mat{0}&\mat{C}_\mora\end{bmatrix},
                \begin{bmatrix} \mat{D}_\morb&\mat{0}\\ \mat{0}&\mat{D}_\mora\end{bmatrix}}.
        \end{equation*}
        Therefore, we have
        \begin{equation*}
            \projA((1))=\begin{bmatrix} \prstart_\morb \\ \prstart_\mora \end{bmatrix}.
        \end{equation*}
        Furthermore, we have:
        \begin{equation*}
            \begin{aligned}
                \projB((1)) & =\begin{bmatrix} \mat{A}_\morb&\mat{0}\\ \mat{0}&\mat{A}_\mora\end{bmatrix}+
                \begin{bmatrix} \mat{0}\\ \rowscols{\mat{B}_\mora}{:}{i+1:i+k} \end{bmatrix}\mat{F} \begin{bmatrix} \mat{0}&\rowscols{\mat{C}_\mora}{j+1:j+k}{:}\end{bmatrix} \\
                            & =\begin{bmatrix}\mat{A}_\morb & \mat{0}                                                                                         \\
               \mat{0}       & \mat{A}_\mora + \rowscols{\mat{B}_\mora}{:}{i+1:i+k}\mat{F}\rowscols{\mat{C}_\mora}{j+1:j+k}{:}
                               \end{bmatrix}
            \end{aligned}
        \end{equation*}
        where
        \begin{equation*}
            \mat{F}=\left(\idmat - \rowscols{\mat{D}_\mora}{j+1:j+k}{i+1:i+k}\right)^{-1}.
        \end{equation*}
        Furthermore:
        \begin{equation*}
            \begin{aligned}
                \projC((1)) & =\begin{bmatrix} \mat{B}_\morb&\mat{0}\\ \mat{0}&\rowscols{\mat{B}_\mora}{:}{1:i}\end{bmatrix}
                + \begin{bmatrix} \mat{0}\\ \rowscols{\mat{B}_\mora}{:}{i+1:i+k} \end{bmatrix}\mat{F}\begin{bmatrix} \mat{0}&\rowscols{\mat{D}_\mora}{j+1:j+k}{1:i}\end{bmatrix} \\
                            & =\begin{bmatrix}
                                   \mat{B}_\morb & \mat{0}                                                                                                             \\
                                   \mat{0}       & \rowscols{\mat{B}_\mora}{:}{1:i}+ \rowscols{\mat{B}_\mora}{:}{i+1:i+k}\mat{F}\rowscols{\mat{D}_\mora}{j+1:j+k}{1:i}
                               \end{bmatrix},
            \end{aligned}
        \end{equation*}
        %
        \begin{equation*}
            \begin{aligned}
                \projD((1)) & =\begin{bmatrix} \mat{C}_\morb&\mat{0}\\ \mat{0}&\rowscols{\mat{C}_\mora}{1:j}{:}\end{bmatrix}
                + \begin{bmatrix}\mat{0}\\ \rowscols{\mat{D}_\mora}{1:j}{i+1:i+k}\end{bmatrix}\mat{F}\begin{bmatrix} \mat{0}&\rowscols{\mat{C}_\mora}{j+1:j+k}{:}\end{bmatrix} \\
                            & =\begin{bmatrix}
                                   \mat{C}_\morb & \mat{0}                                                                                                             \\
                                   \mat{0}       & \rowscols{\mat{C}_\mora}{1:j}{:}+\rowscols{\mat{D}_\mora}{1:j}{i+1:i+k}\mat{F}\rowscols{\mat{C}_\mora}{j+1:j+k}{:},
                               \end{bmatrix}
            \end{aligned}
        \end{equation*}
        and
        \begin{equation*}
            \begin{aligned}
                \projE((1)) & =\begin{bmatrix} \mat{D}_\morb&\mat{0}\\ \mat{0}&\rowscols{\mat{D}_\mora}{1:j}{1:i}\end{bmatrix}
                +\begin{bmatrix}\mat{0}\\ \rowscols{\mat{D}_\mora}{1:j}{i+1:i+k}\end{bmatrix}\mat{F}\begin{bmatrix} \mat{0}&\rowscols{\mat{D}_\mora}{:}{1:i}\end{bmatrix} \\
                            & =\begin{bmatrix}
                                   \mat{D}_\morb & \mat{0}                                                                                                          \\
                                   \mat{0}       & \rowscols{\mat{D}_\mora}{1:j}{1:i}+\rowscols{\mat{D}_\mora}{1:j}{i+1:i+k}\mat{F}\rowscols{\mat{D}_\mora}{:}{1:i}
                               \end{bmatrix}.
            \end{aligned}
        \end{equation*}
        We are now ready to look at the right-hand side of the axiom, referred to as (2).
        First, we have:
        \begin{equation*}
            \begin{aligned}
                \Tr_{i,j}^{k}(\mora)=\langle\prstart_\mora, & \mat{A}_\mora + \rowscols{\mat{B}_\mora}{:}{i+1:i+k}\mat{F}\rowscols{\mat{C}_\mora}{j+1:j+k}{:}, \\
                                                            & \rowscols{\mat{B}_\mora}{:}{1:i}+\rowscols{\mat{B}_\mora}{:}{i+1:i+k}\mat{F}\rowscols{\mat{D}_\mora}{j+1:j+k}{1:i}, \\
                                                            & \rowscols{\mat{C}_\mora}{1:j}{:}+ \rowscols{\mat{D}_\mora}{1:j}{i+1:i+k} \mat{F} \rowscols{\mat{C}_\mora}{j+1:j+k}{:}, \\
                                                            & \rowscols{\mat{D}_\mora}{1:j}{1:i}+ \rowscols{\mat{D}_\mora}{1:j}{i+1:i+k}\mat{F}\rowscols{\mat{D}_\mora}{j+1:j+k}{1:i}\rangle.
            \end{aligned}
        \end{equation*}
        Clearly:
        \begin{equation*}
            \projA((2))=\begin{bmatrix} \prstart_\morb \\ \prstart_\mora \end{bmatrix}=\projA((1)).
        \end{equation*}
        Furthermore:
        \begin{equation*}
            \begin{aligned}
                \projB((2)) & =\begin{bmatrix}\mat{A}_\morb&\mat{0}\\ \mat{0}& \mat{A}_\mora + \rowscols{\mat{B}_\mora}{:}{i+1:i+k}\mat{F}\rowscols{\mat{C}_\mora}{j+1:j+k}{:}\end{bmatrix}=\projB((1)),
            \end{aligned}
        \end{equation*}
        %
        \begin{equation*}
            \begin{aligned}
                \projC((2)) & =\begin{bmatrix}\mat{B}_\morb&\mat{0}\\ \mat{0}& \rowscols{\mat{B}_\mora}{:}{1:i}+\rowscols{\mat{B}_\mora}{:}{i+1:i+k}\mat{F}\rowscols{\mat{D}_\mora}{j+1:j+k}{1:i}\end{bmatrix}=\projC((1)),
            \end{aligned}
        \end{equation*}
        %
        \begin{equation*}
            \begin{aligned}
                \projD((2)) & =\begin{bmatrix}\mat{C}_\morb&\mat{0}\\ \mat{0}& \rowscols{\mat{C}_\mora}{1:j}{:}+ \rowscols{\mat{D}_\mora}{1:j}{i+1:i+k} \mat{F} \rowscols{\mat{C}_\mora}{j+1:j+k}{:}\end{bmatrix}=\projC((1)),
            \end{aligned}
        \end{equation*}
        and
        \begin{equation*}
            \begin{aligned}
                \projE((2)) & =\begin{bmatrix}\mat{D}_\morb&\mat{0}\\ \mat{0}& \rowscols{\mat{D}_\mora}{1:j}{1:i}+ \rowscols{\mat{D}_\mora}{1:j}{i+1:i+k}\mat{F}\rowscols{\mat{D}_\mora}{j+1:j+k}{1:i}\end{bmatrix}=\projC((1)),
            \end{aligned}
        \end{equation*}
        proving the axiom.

    \end{example}}
