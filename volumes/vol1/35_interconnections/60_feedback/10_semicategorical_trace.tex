% !TEX root = chapter-standalone.tex

\section{Feedback for symmetric strict monoidal semicategories}
\todo{fill this section}

\begin{ctdefinition}[Traced symmetric strict monoidal semicategory]
    \label{def:traced-fun-stack-scat}
    We say that a symmetric strict monoidal semicategory~$\tup{\CatC,\mtimescat, \idmoncat, \sourceperm{\sigma} , \targetperm{\sigma} }$ is \emph{traced} if it is equipped with a family of functions
    \begin{equation}
        \Tr_{\Obja,\Objb}^\Objc\colon \HomSet{\CatC}{\Obja \mtimescat \Objc}{\Objb\mtimescat \Objc}\to \HomSet{\CatC}{\Obja}{\Objb},
    \end{equation}
    satisfying the following axioms:
    \begin{enumerate}

        \item \emph{Naturality in $\Obja$:} For any morphisms $\mora\colon \Obja\mtimescat \Objc \mto \Objb\mtimescat \Objc$ and $\morb : \Obja' \mto \Obja$, if $\Objc$ has an identity $\catidat\Objc$, then
              \begin{equation}
                  \Tr_{\Obja', \Objb}^\Objc ( (\morb \mtimescat \catidat\Objc) \mthen \mora) = \morb \mthen \Tr_{\Obja, \Objb}^\Objc (\mora).
              \end{equation}

        \item \emph{Naturality in $\Objb$:}
              For any morphisms $\mora\colon \Obja\mtimescat \Objc \mto \Objb\mtimescat \Objc$ and $\morb : \Objb \mto \Objb'$, if $\Objc$ has an identity $\catidat\Objc$, then
              \begin{equation}
                  \Tr_{\Obja, \Objb'}^\Objc ( \mora \mthen (\morb \mtimescat \catidat\Objc) ) =  \Tr_{\Obja, \Objb}^\Objc (\mora) \mthen \morb.
              \end{equation}

        \item \emph{Dinaturality in $\Objc$:}
              For any morphisms $\mora \colon \Obja \mtimescat \Objc \mto \Objb \mtimescat \Objc'$ and $\morb \colon \Objc' \mto \Objc$, if $\catidat\Obja$ and $\catidat\Objb$ exist, then
              \begin{equation}
                  \Tr_{\Obja, \Objb}^{\Objc} (\mora \mthen (\catidat\Objb \mtimescat \morb)) = \Tr_{\Obja, \Objb}^{\Objc'}((\catidat\Obja \mtimescat \morb) \mthen \mora).
              \end{equation}

        \item \emph{Vanishing I:}
              For any morphism~$\mora \colon \Obja\mto \Objb$ in \CatC,
              \begin{equation}
                  \label{eq:semicat-trace-vanishing_1}
                  \Tr_{\Obja,\Objb}^\idmoncat (\mora)=\mora.
              \end{equation}

        \item \emph{Vanishing II:}
              For any morphism~$\mora \colon \Obja\mtimescat \Objc \mtimescat \Objd \mto \Objb\mtimescat \Objc \mtimescat \Objd$ in \CatC,
              \begin{equation}
                  \label{eq:semicat-trace-vanishing_2}
                  \Tr_{\Obja,\Objb}^{\Objc\mtimescat \Objd}(\mora)=\Tr_{\Obja,\Objb}^\Objc\left(
                  \Tr_{\Obja \mtimescat \Objc , \Objb \mtimescat \Objc}^\Objd(\mora)\right).
              \end{equation}

        \item \emph{Superposing:}
              For any morphisms~$\mora\colon \Obja\mtimescat \Objc \mto \Objb\mtimescat \Objc$ and $\morb : \Obje \mto \Objf$ in \CatC,
              \begin{equation}
                  \label{eq:semicat-trace-superposing}
                  \Tr_{\Obje\mtimescat \Obja,\Objf\mtimescat \Objb}^{\Objc}(\morb \mtimescat \mora)=\morb \mtimescat \Tr_{\Obja,\Objb}^\Objc(\mora).
              \end{equation}

        \item \emph{Yanking:}
              For any object $\Objc$, if $\catidat\Objc$ exists, then
              \begin{equation}
                  \label{eq:semicat-trace-yanking-I}
                  (\sourceperm{\pi}   \mthen \Tr_{\Objc,\Objc}^\Objc)\left(\catidat\Objc \mtimescat \catidat\Objc \right)=\catidat\Objc
              \end{equation}
              and
              \begin{equation}
                  \label{eq:semicat-trace-yanking-II}
                  (\targetperm{\pi}  \mthen \Tr_{\Objc,\Objc}^\Objc)\left(\catidat\Objc \mtimescat \catidat\Objc \right)=\catidat\Objc,
              \end{equation}
              where $\pi \setin S_2$ is the odd permutation corresponding to swapping the two copies of $\Objc$.
    \end{enumerate}
\end{ctdefinition}

\todotext{J: the above defines what might more precisely be called a \emph{right} trace... in a symmetric monoidal category this will, I think, automatically also give a left trace, because the axioms work nicely with the symmetric braiding... in the above I'm not sure if this works out still all nicely... perhaps or perhaps not... }

\begin{figure}[h!]
    \centering
    \includegraphics[scale=0.35]{feedback_axiomatic-semicat-trace-nat-X}
    \caption{Naturality in $\Obja$.}
    \label{fig:axiomatic-semicat-trace-nat-X}
\end{figure}

\begin{figure}[h!]
    \centering
    \includegraphics[scale=0.15]{feedback_axiomatic-semicat-trace-vanishing-I}
    \caption{Vanishing I.}
    \label{fig:axiomatic-semicat-trace-vanishing-I}
\end{figure}

\begin{figure}[h!]
    \centering
    \includegraphics[scale=0.2]{feedback_axiomatic-semicat-trace-superposing}
    \caption{Superposing.}
    \label{fig:axiomatic-semicat-trace-superposing}
\end{figure}

