% !TEX root = chapter-standalone.tex

\section{Feedback for symmetric strict monoidal semicategories}
\todo{fill this section}
\begin{widepar}
    \begin{ctdefinition}[Traced symmetric strict monoidal semicategory]
        \label{def:traced-fun-stack-scat}
        We say that a symmetric strict monoidal \SY{semicategory}~$\tup{\CatC, \mtimescat, \idmoncat, \sourceperm{\sigma}, \targetperm{\sigma} }$ is \emph{traced} if it is equipped with a family of functions
        \begin{equation}
            \Tr_{\Obja,\Objb}^\Objc\colon \HomSet{\CatC}{\Obja \mtimescatob \Objc}{\Objb\mtimescatob \Objc}\to \HomSet{\CatC}{\Obja}{\Objb},
        \end{equation}
        satisfying the following axioms:
        \begin{enumerate}

            \item \emph{Naturality in $\Obja$:} For any morphisms $\mora\colon \Obja\mtimescatob \Objc \mto \Objb\mtimescatob \Objc$ and $\morb \colon \Obja' \mto \Obja$, if $\Objc$ has an identity $\catidat\Objc$, then
                  \begin{equation}
                      \Tr_{\Obja', \Objb}^\Objc ( (\morb \mtimescatmor \catidat\Objc) \mthen \mora) = \morb \mthen \Tr_{\Obja, \Objb}^\Objc (\mora).
                  \end{equation}

            \item \emph{Naturality in $\Objb$:}
                  For any morphisms $\mora\colon \Obja\mtimescatob \Objc \mto \Objb\mtimescatob \Objc$ and $\morb \colon \Objb \mto \Objb'$, if $\Objc$ has an identity $\catidat\Objc$, then
                  \begin{equation}
                      \Tr_{\Obja, \Objb'}^\Objc ( \mora \mthen (\morb \mtimescatmor \catidat\Objc) ) = \Tr_{\Obja, \Objb}^\Objc (\mora) \mthen \morb.
                  \end{equation}

            \item \emph{Dinaturality in $\Objc$:}
                  For any morphisms $\mora \colon \Obja \mtimescatob \Objc \mto \Objb \mtimescatob \Objc'$ and $\morb \colon \Objc' \mto \Objc$, if $\catidat\Obja$ and $\catidat\Objb$ exist, then
                  \begin{equation}
                      \Tr_{\Obja, \Objb}^{\Objc} (\mora \mthen (\catidat\Objb \mtimescatmor \morb)) = \Tr_{\Obja, \Objb}^{\Objc'}((\catidat\Obja \mtimescatmor \morb) \mthen \mora).
                  \end{equation}

            \item \emph{Vanishing I:}
                  For any morphism~$\mora \colon \Obja\mto \Objb$ in \CatC,
                  \begin{equation}
                      \label{eq:semicat-trace-vanishing_1}
                      \Tr_{\Obja,\Objb}^\idmoncat (\mora)=\mora.
                  \end{equation}

            \item \emph{Vanishing II:}
                  For any morphism~$\mora \colon \Obja\mtimescatob \Objc \mtimescatob \Objd \mto \Objb\mtimescatob \Objc \mtimescatob \Objd$ in \CatC,
                  \begin{equation}
                      \label{eq:semicat-trace-vanishing_2}
                      \Tr_{\Obja,\Objb}^{\Objc\mtimescatob \Objd}(\mora)=\Tr_{\Obja,\Objb}^\Objc\pars{
                          \Tr_{\Obja \mtimescatob \Objc, \Objb \mtimescatob \Objc}^\Objd(\mora)}.
                  \end{equation}

            \item \emph{Superposing:}
                  For any morphisms~$\mora\colon \Obja\mtimescatob \Objc \mto \Objb\mtimescatob \Objc$ and $\morb \colon \Obje \mto \Objf$ in \CatC,
                  \begin{equation}
                      \label{eq:semicat-trace-superposing}
                      \Tr_{\Obje\mtimescatob \Obja,\Objf\mtimescatob \Objb}^{\Objc}(\morb \mtimescatmor \mora)=\morb \mtimescatmor \Tr_{\Obja,\Objb}^\Objc(\mora).
                  \end{equation}

            \item \emph{Yanking:}
                  For any object $\Objc$, if $\catidat\Objc$ exists, then
                  \begin{equation}
                      \label{eq:semicat-trace-yanking-I}
                      (\sourceperm{\pi} \mthen \Tr_{\Objc,\Objc}^\Objc)\pars{\catidat\Objc \mtimescatmor \catidat\Objc }=\catidat\Objc
                  \end{equation}
                  and
                  \begin{equation}
                      \label{eq:semicat-trace-yanking-II}
                      (\targetperm{\pi} \mthen \Tr_{\Objc,\Objc}^\Objc)\pars{\catidat\Objc \mtimescatmor \catidat\Objc }=\catidat\Objc,
                  \end{equation}
                  where $\pi \setin S_2$ is the odd permutation corresponding to swapping the two copies of $\Objc$.
        \end{enumerate}
    \end{ctdefinition}
\end{widepar}

\todotext{J: the above defines what might more precisely be called a \emph{right} trace... in a \SY{symmetric monoidal category} this will, I think, automatically also give a left trace, because the axioms work nicely with the symmetric braiding... in the above I'm not sure if this works out still all nicely... perhaps or perhaps not... }

    \begin{example}
        Consider the subsemicategory of \LTI only allowing proper \LTI systems as morphisms with the trace defined in \cref{def:trace-lti-prop}.
        The trace axioms are satified.

        We prove the statements not involving identities one by one.
        \paragraph*{Vanishing I}
        Consider a proper LTI system $\mora\colon i\mto j$, given by $\genericplti{\mora}$ with state dimension $s\setin \natnumbers$.
        We need to check
        \begin{equation*}
            \Tr_{i,j}^0 (\mora)=\mora.
        \end{equation*}
        Using the trace definition, we can write:
        \begin{equation*}
            \Tr_{i,j}^0 (\mora)=\tup{\prstart{\mora}, \mat{A}_\mora+\rowscols{\mat{B}_{\mora}}{:}{i+1:i+k}\rowscols{\mat{C}_{\mora}}{j+1:j+k}{:}, \rowscols{\mat{B}_{\mora}}{:}{1:i}, \rowscols{\mat{C}_{\mora}}{1:j}{:}}.
        \end{equation*}
        However, we have:
        \begin{equation*}
            \mat{B}_\mora=\begin{bmatrix}
                \rowscols{\mat{B}_{\mora}}{:}{1:i} & \mat{0}^{s\times 0}
            \end{bmatrix},\quad
            \mat{C}_\mora=\begin{bmatrix}
                \rowscols{\mat{C}_{\mora}}{1:j}{:} \\ \mat{0}^{0\times s}
            \end{bmatrix},
        \end{equation*}
        and therefore
        \begin{equation*}
            \Tr_{i,j}^0 (\mora)=\tup{\prstart{\mora}, \mat{A}_\mora, \mat{B}_\mora, \mat{C}_\mora}=\mora.
        \end{equation*}

        \paragraph*{Vanishing II}
        Consider a proper LTI system $\mora\colon i+k+o\mto j+k+o$, given by $\genericplti{\mora}$ with state dimension $s\setin \natnumbers$.
        We need to check
        \begin{equation*}
            \Tr_{i,j}^{k+o} (\mora)=\Tr_{i,j}^{k}\left(\Tr_{i+k,j+k}^o(\mora)\right).
        \end{equation*}
        Let's start with the left-hand side of the statement.
        We have:
        \begin{equation}
            \label{eq:vanishing2-left}
            \Tr_{i,j}^{k+o} (\mora)=\tup{\prstart, \mat{A}_\mora+\rowscols{\mat{B}_\mora}{:}{i+1:i+k+o}\rowscols{\mat{C}_\mora}{j+1:j+k+o}{:}, \rowscols{\mat{B}_\mora}{:}{1:i},\rowscols{\mat{C}_\mora}{1:j}{:}}.
        \end{equation}
        Let's look at the right-hand side.
        We have:
        \begin{widepar}
            \begin{equation*}
                \Tr_{i+k,j+k}^{o} (\mora)=\tup{\prstart, \mat{A}+\rowscols{\mat{B}_\mora}{:}{i+k+1:i+k+o} \rowscols{\mat{C}_\mora}{j+k+1:j+k+o}{:}, \underbrace{\rowscols{\mat{B}_\mora}{:}{1:i+k}}_{\mat{B}_{\mora,\star}},\underbrace{\rowscols{\mat{C}_\mora}{1:j+k}{:}}_{\mat{C}_{\mora,\star}}},
            \end{equation*}
        \end{widepar}
        which we can leverage to express $\Tr_{i,j}^{k}\left(\Tr_{i+k,j+k}^o(\mora)\right)=\ast$.
        Clearly, $\projA(\ast)=\prstart$.
        Furthermore:
        \begin{equation*}
            \begin{aligned}
                \projB(\ast) & =\mat{A}_\mora + \rowscols{\mat{B}_\mora}{:}{i+k+1:i+k+o}\rowscols{\mat{C}_\mora}{j+k+1:j+k+o}{:}+\rowscols{\mat{B}_{\mora,\star}}{:}{i+1:i+k}\rowscols{\mat{C}_{\mora,\star}}{:}{j+1:j+k} \\
                             & =\mat{A}_\mora + \rowscols{\mat{B}_\mora}{:}{i+k+1:i+k+o}\rowscols{\mat{C}_\mora}{j+k+1:j+k+o}{:}+\rowscols{\mat{B}_{\mora}}{:}{i+1:i+k}\rowscols{\mat{C}_{\mora}}{:}{j+1:j+k} \\
                             & =\mat{A}_\mora + \begin{bmatrix}
                                                    \rowscols{\mat{B}_{\mora}}{:}{i+1:i+k} & \rowscols{\mat{B}_\mora}{:}{i+k+1:i+k+o}
                                                \end{bmatrix} \begin{bmatrix}
                                                                  \rowscols{\mat{C}_{\mora}}{:}{j+1:j+k} \\
                                                                  \rowscols{\mat{C}_\mora}{j+k+1:j+k+o}{:}
                                                              \end{bmatrix} \\
                             & =\mat{A}+\rowscols{\mat{B}_\mora}{:}{i+1:i+k+o}\rowscols{\mat{C}_\mora}{j+1:j+k+o}{:}.
            \end{aligned}
        \end{equation*}
        Furthermore, $\projC(\ast) = \rowscols{\mat{B}_\mora}{:}{1:i}$ and $\projD(\ast) = \rowscols{\mat{C}_\mora}{1:j}{:}$.
        Clearly, the found results correspond to \cref{eq:vanishing2-left}.

        \paragraph*{Superposing}
        Consider the proper LTI systems $\mora\colon i+k\mto j+k$, given by $\genericplti{\mora}$, and $\morb\colon l\mto m$, given by $\genericplti{\morb}$.
        We need to check:
        \begin{equation*}
            \Tr_{l+i,m+j}^k(\morb \mtimescatmor \mora)=\morb \mtimescatmor \Tr_{i,j}^k(\mora).
        \end{equation*}
        Let's start with the left-hand side.
        We have:
        \begin{equation*}
            \morb \mtimescatmor \mora=\tup{\begin{bmatrix}\prstart_\morb\\ \prstart_\mora \end{bmatrix}, \begin{bmatrix} \mat{A}_\morb&\mat{0}\\ \mat{0}&\mat{A}_\mora\end{bmatrix},
                \begin{bmatrix} \mat{B}_\morb&\mat{0}\\ \mat{0}&\mat{B}_\mora\end{bmatrix},\begin{bmatrix} \mat{C}_\morb&\mat{0}\\ \mat{0}&\mat{C}_\mora\end{bmatrix}}
        \end{equation*}
        Now observe that $\mat{B}_\morb$ has $l$ columns, $\mat{B}_\mora$ has $i+k$ columns, $\mat{C}_\morb$ has $m$ rows, and $\mat{C}_\mora$ has $j+k$ rows.
        Let's denote $\Tr_{l+i,m+j}^k(\morb \mtimescatmor \mora)$ by $\ast$.
        Clearly:
        \begin{equation*}
            \projA(\ast)=\begin{bmatrix}\prstart_\morb &\prstart_\mora\end{bmatrix}.
        \end{equation*}
        We have:
        \begin{equation*}
            \begin{aligned}
                \projB(\ast) & =\begin{bmatrix} \mat{A}_\morb&\mat{0}\\ \mat{0}&\mat{A}_\mora\end{bmatrix}+\rowscols{\begin{bmatrix} \mat{B}_\morb&\mat{0}\\ \mat{0}&\mat{B}_\mora\end{bmatrix}}{:}{l+i+1:l+i+k}\rowscols{\begin{bmatrix} \mat{C}_\morb&\mat{0}\\ \mat{0}&\mat{C}_\mora\end{bmatrix}}{m+j+1:m+j+k}{:} \\
                             & =\begin{bmatrix} \mat{A}_\morb&\mat{0}\\ \mat{0}&\mat{A}_\mora\end{bmatrix}+\begin{bmatrix} \mat{0}\\ \rowscols{\mat{B}_\mora}{:}{i+1:i+k}\end{bmatrix}
                \begin{bmatrix}
                    \mat{0} & \rowscols{\mat{C}_\mora}{j+1:j+k}{:}
                \end{bmatrix} \\
                             & =\begin{bmatrix} \mat{A}_\morb&\mat{0}\\ \mat{0}&\mat{A}_\mora\end{bmatrix}
                \begin{bmatrix}
                    \mat{0} & \mat{0}                                                                  \\
                    \mat{0} & \rowscols{\mat{B}_\mora}{:}{i+1:i+k}\rowscols{\mat{C}_\mora}{j+1:j+k}{:}
                \end{bmatrix}
            \end{aligned}
        \end{equation*}
        Furthermore, we have
        \begin{equation*}
            \projC(\ast)=\begin{bmatrix}
                \mat{B}_\morb & \mat{0}                          \\
                \mat{0}       & \rowscols{\mat{B}_\mora}{:}{1:i}
            \end{bmatrix}
        \end{equation*}
        and
        \begin{equation*}
            \projD(\ast)=\begin{bmatrix}
                \mat{C}_\morb & \mat{0}                          \\
                \mat{0}       & \rowscols{\mat{C}_\mora}{1:j}{:}
            \end{bmatrix}
        \end{equation*}
        We can now look at the right-hand side.
        First:
        \begin{equation*}
            \Tr_{i,j}^{k} (\mora)=\tup{\prstart, \mat{A}_\mora+\rowscols{\mat{B}_\mora}{:}{i+1:i+k} \rowscols{\mat{C}_\mora}{j+1:j+k}{:}, \rowscols{\mat{B}_\mora}{:}{1:i},\rowscols{\mat{C}_\mora}{1:j}{:}},
        \end{equation*}
        Therefore, we have:
        \begin{equation*}
            \projA(\morb \mtimescatmor \Tr_{i,j}^k(\mora))=\begin{bmatrix}\prstart_\morb &\prstart_\mora\end{bmatrix},
        \end{equation*}
        \begin{equation*}
            \projB(\morb \mtimescatmor \Tr_{i,j}^k(\mora))=\begin{bmatrix}
                \mat{A}_\morb & \mat{0}                                                                                    \\
                \mat{0}       & \mat{A}_\mora + \rowscols{\mat{B}_\mora}{:}{i+1:i+k} \rowscols{\mat{C}_\mora}{j+1:j+k}{:},
            \end{bmatrix}
        \end{equation*}
        %
        \begin{equation*}
            \projC(\morb \mtimescatmor \Tr_{i,j}^k(\mora))=\begin{bmatrix}
                \mat{B}_\morb & \mat{0}                          \\
                \mat{0}       & \rowscols{\mat{B}_\mora}{:}{1:i}
            \end{bmatrix},
        \end{equation*}
        and
        \begin{equation*}
            \projD(\morb \mtimescatmor \Tr_{i,j}^k(\mora))=
            \begin{bmatrix}
                \mat{C}_\morb & \mat{0}                          \\
                \mat{0}       & \rowscols{\mat{C}_\mora}{1:j}{:}
            \end{bmatrix},
        \end{equation*}
        which clearly correspond to the left-hand side.
    \end{example}

\devel{
\begin{example}
    \LTI with the trace as in \cref{def:trace-lti} satifies the trace axioms.


        We prove the statements one by one.

        \paragraph*{Naturality in $i$}
        Consider the LTI systems $\mora\colon i+k\mto j+k$ and $\morb\colon m\mto i$.
        We need to check:
        \begin{equation*}
            \underbrace{\Tr_{m,j}^k((\morb \mtimescatmor \catid_k)\mthen \mora)}_{(1)}=\morb \mthen \Tr_{i,j}^k(\mora).
        \end{equation*}

        We start developing the left-hand side of the equation, referred to as (1).
        To do so, we first write $\morb \mtimescatmor \catid_k\colon m+k\mto i+k$.
        \begin{equation*}
            \morb \mtimescatmor \catid_k=\tupp{\prstart_\morb, \mat{A}_\morb, \begin{bmatrix} \mat{B}_\morb & \mat{0}\end{bmatrix},\begin{bmatrix} \mat{C}_\morb \\ \mat{0}\end{bmatrix}, \begin{bmatrix} \mat{D}_\morb & \mat{0}\\ \mat{0}&\idmat^{k\times k}\end{bmatrix}}.
        \end{equation*}

        Furthermore, we have:
        \begin{widepar}
            \begin{equation*}
                (\morb \mtimescatmor \catid_k)\mthen \mora =\tupp{\begin{bmatrix}\prstart_\morb\\ \prstart_\mora \end{bmatrix},
                    \begin{bmatrix} \mat{A}_\morb&\mat{0}\\ \mat{B}_\mora \begin{bmatrix} \mat{C}_\morb \\ \mat{0}\end{bmatrix}&\mat{A}_\mora \end{bmatrix},
                    \begin{bmatrix}\mat{B}_\morb & \mat{0}\\\rowscols{\mat{B}_\mora}{:}{1:i}\mat{D}_\morb &\rowscols{\mat{B}_\mora}{:}{i+1:i+k} \end{bmatrix},
                    \begin{bmatrix}\rowscols{\mat{D}_\mora}{:}{1:i}\mat{C}_\morb&\mat{C}_\mora\end{bmatrix},
                    \begin{bmatrix} \rowscols{\mat{D}_\mora}{:}{1:i}\mat{D}_\morb & \rowscols{\mat{D}_\mora}{:}{i+1:i+k}\end{bmatrix}}.
            \end{equation*}
        \end{widepar}

        With these intermediate calculations, we can start looking at (1) using the trace formula, component by component.
        Clearly, we have
        \begin{equation*}
            \projA ((1))=\begin{bmatrix}\prstart_\morb\\ \prstart_\mora \end{bmatrix}.
        \end{equation*}
        Furthermore, we have
        \begin{widepar}
            \begin{equation}
                \label{eq:lti-nat-1}
                \begin{aligned}
                    \projB ((1)) & =\begin{bmatrix} \mat{A}_\morb&\mat{0}\\ \mat{B}_\mora \begin{bmatrix} \mat{C}_\morb \\ \mat{0}\end{bmatrix}&\mat{A}_\mora \end{bmatrix}
                    +\rowscols{\begin{bmatrix}\mat{B}_\morb & \mat{0}\\\rowscols{\mat{B}_\mora}{:}{1:i}\mat{D}_\morb &\rowscols{\mat{B}_\mora}{:}{i+1:i+k} \end{bmatrix}}{:}{m+1:m+k}
                    \mat{F}
                    \rowscols{\begin{bmatrix}\rowscols{\mat{D}_\mora}{:}{1:i}\mat{C}_\morb&\mat{C}_\mora\end{bmatrix}}{j+1:j+k}{:},
                \end{aligned}
            \end{equation}
        \end{widepar}
        where we parametrize
        \begin{equation*}
            \mat{F} = \left(\idmat-\rowscols{\begin{bmatrix} \rowscols{\mat{D}_\mora}{:}{1:i}\mat{D}_\morb & \rowscols{\mat{D}_\mora}{:}{i+1:i+k}\end{bmatrix}}{j+1:j+k}{m+1:m+k}\right)^{-1},
        \end{equation*}
        for brevity.
        We can simplify \cref{eq:lti-nat-1} by observing that $\mat{B}_\morb$ has $m$ columns, $\mat{B}_\mora$ has $i+k$ columns, $\mat{D}_\mora$ has $j+k$ roes and $i+k$ columns, and $\mat{C}_\morb$ has $i$ rows.
        We get:
        %\begin{widepar}
        \begin{equation}
            \label{eq:lti-nat-2}
            \begin{aligned}
                 & \projB ((1)) \\
                 & =\begin{bmatrix} \mat{A}_\morb&\mat{0}\\ \mat{B}_\mora \begin{bmatrix} \mat{C}_\morb \\ \mat{0}\end{bmatrix}&\mat{A}_\mora \end{bmatrix}
                +\begin{bmatrix}\mat{0}\\ \rowscols{\mat{B}_\mora}{:}{i+1:i+k} \end{bmatrix}
                \mat{F}
                \rowscols{\begin{bmatrix}\rowscols{\mat{D}_\mora}{:}{1:i}\mat{C}_\morb&\mat{C}_\mora\end{bmatrix}}{j+1:j+k}{:} \\
                 & =\begin{bmatrix} \mat{A}_\morb&\mat{0}\\ \rowscols{\mat{B}_\mora}{:}{1:i}\mat{C}_\morb &\mat{A}_\mora \end{bmatrix}
                +\begin{bmatrix}\mat{0}\\ \rowscols{\mat{B}_\mora}{:}{i+1:i+k} \end{bmatrix}
                \mat{F}
                \rowscols{\begin{bmatrix}\rowscols{\mat{D}_\mora}{:}{1:i}\mat{C}_\morb&\mat{C}_\mora\end{bmatrix}}{j+1:j+k}{:} \\
                 & =\begin{bmatrix} \mat{A}_\morb                                 & \mat{0}       \\
                \rowscols{\mat{B}_\mora}{:}{1:i}\mat{C}_\morb & \mat{A}_\mora\end{bmatrix}
                +\begin{bmatrix}\mat{0}\\ \rowscols{\mat{B}_\mora}{:}{i+1:i+k} \end{bmatrix}
                \mat{F'}
                \begin{bmatrix}\rowscols{\mat{D}_\mora}{j+1:j+k}{1:i}\mat{C}_\morb&\rowscols{\mat{C}_\mora}{j+1:j+k}{:}\end{bmatrix} \\
                 & =\begin{bmatrix}
                        \mat{A}_\morb & \mat{0}                                                                                        \\
                        \rowscols{\mat{B}_\mora}{:}{1:i}\mat{C}_\morb+
                        \rowscols{\mat{B}_\mora}{:}{i+1:i+k}\mat{F'} \rowscols{\mat{D}_\mora}{j+1:j+k}{1:i}\mat{C}_\morb
                                      & \mat{A}_\mora+\rowscols{\mat{B}_\mora}{:}{i+1:i+k}\mat{F'}\rowscols{\mat{C}_\mora}{j+1:j+k}{:}
                    \end{bmatrix},
            \end{aligned}
        \end{equation}
        %\end{widepar}
        with
        \begin{equation*}
            \mat{F'} = \left(\idmat-\rowscols{\mat{D}_\mora}{j+1:j+k}{i+1:i+k}\right)^{-1}.
        \end{equation*}
        Furthermore, we have
        \begin{equation*}
            \begin{aligned}
                \projC((1)) & =
                \begin{bmatrix} \mat{B}_\morb \\ \rowscols{\mat{B}_\mora}{:}{1:i} \mat{D}_\morb \end{bmatrix}
                + \begin{bmatrix} \mat{0}\\ \rowscols{\mat{B}_\mora}{:}{i+1:i+k}\end{bmatrix}
                \mat{F'}\rowscols{\mat{D}_\mora}{j+1:j+k}{1:i}\mat{D}_\morb,
            \end{aligned}
        \end{equation*}
        %
        \begin{equation*}
            \begin{aligned}
                 & \projD((1)) \\
                 & =
                \begin{bmatrix} \rowscols{\mat{D}_\mora}{1:j}{1:i}\mat{C}_\morb&\rowscols{\mat{C}_\mora}{1:j}{:} \end{bmatrix}
                +\rowscols{\mat{D}_\mora}{1:j}{i+1:i+k}\mat{F'}
                \begin{bmatrix}
                    \rowscols{\mat{D}_\mora}{j+1:j+k}{1:i}\mat{C}_\morb & \rowscols{\mat{C}_\mora}{j+1:j+k}{:}
                \end{bmatrix},
            \end{aligned}
        \end{equation*}
        and
        \begin{equation*}
            \begin{aligned}
                \projE((1)) & =\rowscols{\mat{D}_\mora}{:}{1:i}\mat{D}_\morb + \rowscols{\mat{D}_\mora}{:}{i+1:i+k}\mat{F'}\rowscols{\mat{D}_\mora}{j+1:j+k}{i+1:i+k}.
            \end{aligned}
        \end{equation*}

        We are now ready to look at the right-hand side of the axiom to prove, which we refer to as (2).
        First, applying the definition of trace, we have:
        \begin{equation*}
            \begin{aligned}
                \Tr_{i,j}^{k}(\mora)=\langle\prstart_\mora, & \mat{A}_\mora +\rowscols{\mat{B}_\mora}{:}{i+1:i+k}\mat{F'}\rowscols{\mat{C}_\mora}{j+1:j+k}{:}, \\
                                                            & \rowscols{\mat{B}_\mora}{:}{1:i}+\rowscols{\mat{B}_\mora}{:}{i+1:i+k}\mat{F'}\rowscols{\mat{D}_\mora}{j+1:j+k}{1:i}, \\
                                                            & \rowscols{\mat{C}_\mora}{1:j}{:}+ \rowscols{\mat{D}_\mora}{1:j}{i+1:i+k} \mat{F'} \rowscols{\mat{C}_\mora}{j+1:j+k}{:}, \\
                                                            & \rowscols{\mat{D}_\mora}{1:j}{1:i}+ \rowscols{\mat{D}_\mora}{1:j}{i+1:i+k}\mat{F'}\rowscols{\mat{D}_\mora}{j+1:j+k}{i+1:i+k}\rangle
            \end{aligned}
        \end{equation*}
        Now, we can compute (2) component by component.
        We have:
        \begin{equation*}
            \projA ((2))=\begin{bmatrix}\prstart_\morb\\ \prstart_\mora \end{bmatrix}= \projA((1)).
        \end{equation*}
        Furthermore, we have
        \begin{equation*}
            \begin{aligned}
                 & \projB((2)) \\
                 & =\begin{bmatrix}
                        \mat{A}_\morb & \mat{0}                                                                                        \\
                        \rowscols{\mat{B}_\mora}{:}{1:i}\mat{C}_\morb+
                        \rowscols{\mat{B}_\mora}{:}{i+1:i+k}\mat{F'} \rowscols{\mat{D}_\mora}{j+1:j+k}{1:i}\mat{C}_\morb
                                      & \mat{A}_\mora+\rowscols{\mat{B}_\mora}{:}{i+1:i+k}\mat{F'}\rowscols{\mat{C}_\mora}{j+1:j+k}{:}
                    \end{bmatrix} \\
                 & =\projB((1)).
            \end{aligned}
        \end{equation*}

        \begin{equation*}
            \begin{aligned}
                \projC((2)) & =
                \begin{bmatrix} \mat{B}_\morb \\ \rowscols{\mat{B}_\mora}{:}{1:i} \mat{D}_\morb \end{bmatrix}
                + \begin{bmatrix} \mat{0}\\ \rowscols{\mat{B}_\mora}{:}{i+1:i+k}\end{bmatrix}
                \mat{F'}\rowscols{\mat{D}_\mora}{j+1:j+k}{1:i}\mat{D}_\morb \\
                            & =\projC((1)),
            \end{aligned}
        \end{equation*}
        %
        \begin{equation*}
            \begin{aligned}
                 & \projD((2)) \\
                 & =
                \begin{bmatrix} \rowscols{\mat{D}_\mora}{1:j}{1:i}\mat{C}_\morb&\rowscols{\mat{C}_\mora}{1:j}{:} \end{bmatrix}
                +\rowscols{\mat{D}_\mora}{1:j}{i+1:i+k}\mat{F'}
                \begin{bmatrix}
                    \rowscols{\mat{D}_\mora}{j+1:j+k}{1:i}\mat{C}_\morb & \rowscols{\mat{C}_\mora}{j+1:j+k}{:}
                \end{bmatrix} \\
                 & =\projD((2)),
            \end{aligned}
        \end{equation*}
        and
        \begin{equation*}
            \begin{aligned}
                \projE((2)) & =\rowscols{\mat{D}_\mora}{:}{1:i}\mat{D}_\morb + \rowscols{\mat{D}_\mora}{:}{i+1:i+k}\mat{F'}\rowscols{\mat{D}_\mora}{j+1:j+k}{i+1:i+k} \\
                            & =\projE((1)),
            \end{aligned}
        \end{equation*}
        proving naturality in $i$.

        \paragraph*{Naturality in $j$}
        Consider the LTI systems $\mora\colon i+k\mto j+k$ and $\morb\colon j\mto m$.
        We need to check:
        \begin{equation*}
            \underbrace{\Tr_{i,m}^k(\mora \mthen (\morb \mtimescatmor \catid_k))}_{(1)}=\Tr_{i,j}^k(\mora)\mthen \morb.
        \end{equation*}
        We start developing the left-hand side of the equation, referred to as (1).
        To do so, we first write $\morb \mtimescatmor \catid_k\colon j+k\mto m+k$:
        \begin{equation*}
            \morb \mtimescatmor \catid_k=\tupp{\prstart_\morb, \mat{A}_\morb, \begin{bmatrix} \mat{B}_\morb & \mat{0}\end{bmatrix},\begin{bmatrix} \mat{C}_\morb \\ \mat{0}\end{bmatrix}, \begin{bmatrix} \mat{D}_\morb & \mat{0}\\ \mat{0}&\idmat^{k\times k}\end{bmatrix}}.
        \end{equation*}
        Furthermore, we have:
        \begin{equation*}
            \begin{aligned}
                \mora \mthen (\morb \mtimescatmor \catid_k) =\langle\begin{bmatrix}\prstart_\mora\\ \prstart_\morb \end{bmatrix},
                 & \begin{bmatrix} \mat{A}_\mora&\mat{0}\\ \mat{B}_\morb \rowscols{\mat{C}_\mora}{1:j}{:} &\mat{A}_\morb \end{bmatrix},
                \begin{bmatrix}\mat{B}_\mora \\ \mat{B}_\morb \rowscols{\mat{D}_\mora}{1:j}{:} \end{bmatrix}, \\
                 & \begin{bmatrix}\begin{bmatrix}\mat{D}_\morb \rowscols{\mat{C}_\mora}{1:j}{:}\\ \rowscols{\mat{C}_\mora}{j+1:j+k}{:}\end{bmatrix}& \begin{bmatrix} \mat{C}_\morb \\ \mat{0}\end{bmatrix} \end{bmatrix},
                \begin{bmatrix}\mat{D}_\morb \rowscols{\mat{D}_\mora}{1:j}{:}\\ \rowscols{\mat{D}_\mora}{j+1:j+k}{:}\end{bmatrix}
                \rangle
            \end{aligned}
        \end{equation*}
        With these intermediate calculations, we can start looking at (1) using the trace formula, component by component.
        Clearly, we have
        \begin{equation*}
            \projA ((1))=\begin{bmatrix}\prstart_\mora\\ \prstart_\morb \end{bmatrix}.
        \end{equation*}
        Furthermore, we have
        \begin{equation}
            \label{eq:lti-natbis-1}
            \begin{aligned}
                \projB ((1)) & =\begin{bmatrix} \mat{A}_\mora&\mat{0}\\ \mat{B}_\morb \rowscols{\mat{C}_\mora}{1:j}{:} &\mat{A}_\morb \end{bmatrix}
                + \begin{bmatrix}\rowscols{\mat{B}_\mora}{:}{i+1:i+k} \\ \rowscols{\begin{bmatrix}\mat{B}_\morb \rowscols{\mat{D}_\mora}{1:j}{:}\end{bmatrix}}{:}{i+1:i+k} \end{bmatrix} \mat{F}\begin{bmatrix} \rowscols{\mat{C}_\mora}{j+1:j+k}{:}&\mat{0}\end{bmatrix} \\
                             & =\begin{bmatrix}
                                    \mat{A}_\mora +   \rowscols{\mat{B}_\mora}{:}{i+1:i+k}\mat{F}\rowscols{\mat{C}_\mora}{j+1:j+k}{:}                                                                                                            & \mat{0}        \\
                                    \mat{B}_\morb \rowscols{\mat{C}_\mora}{1:j}{:}+ \rowscols{\begin{bmatrix}\mat{B}_\morb \rowscols{\mat{D}_\mora}{1:j}{:}\end{bmatrix}}{:}{i+1:i+k}\mat{F}\rowscols{\mat{C}_\mora}{j+1:j+k}{:} & \mat{A}_\morb,
                                \end{bmatrix}
            \end{aligned}
        \end{equation}
        where we parametrize
        \begin{equation*}
            \mat{F} = \left(\idmat-\rowscols{\mat{D}_\mora}{j+1:j+k}{i+1:i+k}\right)^{-1},
        \end{equation*}
        for brevity.

        Furthermore, we have
        \begin{equation*}
            \begin{aligned}
                \projC((1)) & = \begin{bmatrix}\rowscols{\mat{B}_\mora}{:}{1:i} \\ \rowscols{\begin{bmatrix}\mat{B}_\morb \rowscols{\mat{D}_\mora}{1:j}{:}\end{bmatrix}}{:}{1:i} \end{bmatrix}+ \begin{bmatrix}\rowscols{\mat{B}_\mora}{:}{i+1:i+k} \\ \rowscols{\begin{bmatrix}\mat{B}_\morb \rowscols{\mat{D}_\mora}{1:j}{:}\end{bmatrix}}{:}{i+1:i+k} \end{bmatrix}\mat{F}\rowscols{\begin{bmatrix}\mat{D}_\morb \rowscols{\mat{D}_\mora}{1:j}{:}\\ \rowscols{\mat{D}_\mora}{j+1:j+k}{:}\end{bmatrix}}{m+1:m+k}{1:i} \\
                            & =\begin{bmatrix}\rowscols{\mat{B}_\mora}{:}{1:i} \\ \rowscols{\begin{bmatrix}\mat{B}_\morb \rowscols{\mat{D}_\mora}{1:j}{:}\end{bmatrix}}{:}{1:i} \end{bmatrix}+ \begin{bmatrix}\rowscols{\mat{B}_\mora}{:}{i+1:i+k} \\ \rowscols{\begin{bmatrix}\mat{B}_\morb \rowscols{\mat{D}_\mora}{1:j}{:}\end{bmatrix}}{:}{i+1:i+k} \end{bmatrix}\mat{F}\rowscols{\mat{D}_\mora}{j+1:j+k}{1:i}
            \end{aligned}
        \end{equation*}
        %
        \begin{equation*}
            \begin{aligned}
                \projD((1)) & = \begin{bmatrix} \mat{D}_\morb\rowscols{\mat{C}_\mora}{1:j}{:}&\mat{C}_\morb\end{bmatrix}+\rowscols{\begin{bmatrix}\mat{D}_\morb \rowscols{\mat{D}_\mora}{1:j}{:}\end{bmatrix}}{:}{i+1:i+k} \mat{F}\begin{bmatrix} \rowscols{\mat{C}_\mora}{j+1:j+k}{:}&\mat{0}\end{bmatrix},
            \end{aligned}
        \end{equation*}
        and
        \begin{equation*}
            \begin{aligned}
                \projE((1)) & = \rowscols{\begin{bmatrix}\mat{D}_\morb \rowscols{\mat{D}_\mora}{1:j}{:}\end{bmatrix}}{:}{1:i}+\rowscols{\begin{bmatrix}\mat{D}_\morb \rowscols{\mat{D}_\mora}{1:j}{:}\end{bmatrix}}{:}{i+1:i+k}\mat{F}\rowscols{\mat{D}_\mora}{j+1:j+k}{1:i}.
            \end{aligned}
        \end{equation*}
        We are now ready to look at the right-hand side of the axiom to prove, which we refer to as (2).
        First, applying the definition of trace, we have:
        \begin{equation*}
            \begin{aligned}
                \Tr_{i,j}^{k}(\mora)=\langle\prstart_\mora, & \mat{A}_\mora + \rowscols{\mat{B}_\mora}{:}{i+1:i+k}\mat{F}\rowscols{\mat{C}_\mora}{j+1:j+k}{:}, \\
                                                            & \rowscols{\mat{B}_\mora}{:}{1:i}+\rowscols{\mat{B}_\mora}{:}{i+1:i+k}\mat{F}\rowscols{\mat{D}_\mora}{j+1:j+k}{1:i}, \\
                                                            & \rowscols{\mat{C}_\mora}{1:j}{:}+ \rowscols{\mat{D}_\mora}{1:j}{i+1:i+k} \mat{F} \rowscols{\mat{C}_\mora}{j+1:j+k}{:}, \\
                                                            & \rowscols{\mat{D}_\mora}{1:j}{1:i}+ \rowscols{\mat{D}_\mora}{1:j}{i+1:i+k}\mat{F}\rowscols{\mat{D}_\mora}{j+1:j+k}{i+1:i+k}\rangle,
            \end{aligned}
        \end{equation*}
        Now, we can compute (2) component by component.
        Clearly we have:
        \begin{equation*}
            \projA((2))=\begin{bmatrix} \prstart_\mora \\ \prstart_\morb\end{bmatrix}=\projA((1)).
        \end{equation*}
        Furthermore, we have
        \begin{equation*}
            \begin{aligned}
                \projB((2)) & =\begin{bmatrix}
                                   \mat{A}_\mora + \rowscols{\mat{B}_\mora}{:}{i+1:i+k}\mat{F}\rowscols{\mat{C}_\mora}{j+1:j+k}{:}                                                   & \mat{0}       \\
                                   \mat{B}_\morb \left( \rowscols{\mat{C}_\mora}{1:j}{:}+ \rowscols{\mat{D}_\mora}{1:j}{i+1:i+k} \mat{F} \rowscols{\mat{C}_\mora}{j+1:j+k}{:}\right) & \mat{A}_\morb
                               \end{bmatrix} \\
                            & =\projB((1)),
            \end{aligned}
        \end{equation*}
        %
        \begin{equation*}
            \begin{aligned}
                \projC((2)) & =\begin{bmatrix} \rowscols{\mat{B}_\mora}{:}{1:i}+\rowscols{\mat{B}_\mora}{:}{i+1:i+k}\mat{F}\rowscols{\mat{D}_\mora}{j+1:j+k}{1:i} \\ \mat{B}_\morb \left( \rowscols{\mat{D}_\mora}{1:j}{1:i}+ \rowscols{\mat{D}_\mora}{1:j}{i+1:i+k}\mat{F}\rowscols{\mat{D}_\mora}{j+1:j+k}{i+1:i+k}\right)\end{bmatrix} \\
                            & =\projC((1)),
            \end{aligned}
        \end{equation*}
        %
        \begin{equation*}
            \begin{aligned}
                \projD((2)) & =\begin{bmatrix}\mat{D}_\morb \left(\rowscols{\mat{C}_\mora}{1:j}{:}+ \rowscols{\mat{D}_\mora}{1:j}{i+1:i+k} \mat{F} \rowscols{\mat{C}_\mora}{j+1:j+k}{:}\right)&\mat{C}_\morb \end{bmatrix} \\
                            & =\projD((1)),
            \end{aligned}
        \end{equation*}
        and
        \begin{equation*}
            \begin{aligned}
                \projE((2)) & =\mat{D}_\morb\left(\rowscols{\mat{D}_\mora}{1:j}{1:i}+ \rowscols{\mat{D}_\mora}{1:j}{i+1:i+k}\mat{F}\rowscols{\mat{D}_\mora}{j+1:j+k}{i+1:i+k} \right) \\
                            & =\projE((1)),
            \end{aligned}
        \end{equation*}
        proving naturality in $j$.
        \paragraph*{Dinaturality in $k$}
        Consider LTI systems $\mora\colon i+k\mto j+m$ and $\morb\colon m\mto k$.
        We need to show:
        \begin{equation*}
        \underbrace{\Tr_{i, j}^{k} (\mora \mthen (\catidat j \mtimescatmor \morb))}_{(1)} = \Tr_{i, j}^{m}((\catidat i \mtimescatmor \morb) \mthen \mora).
        \end{equation*}
        We start developing the left-hand side of the equation, referred to as (1).
        To do so, we first write $\catid_j \mtimescatmor \morb \colon j+m\mto j+k$.
        \begin{equation*}
        \catid_j \mtimescatmor \morb =\tupp{\prstart_\morb, \mat{A}_\morb, \begin{bmatrix} \mat{0}&\mat{B}_\morb\end{bmatrix},\begin{bmatrix}  \mat{0}\\ \mat{C}_\morb\end{bmatrix}, \begin{bmatrix}\idmat^{j\times j}  & \mat{0}\\ \mat{0}&\mat{D}_\morb \end{bmatrix}}.
        \end{equation*}
        Furthermore, we have:
        \begin{widepar}
        \begin{equation*}
        \mora \mthen (\catid_j \mtimescatmor \morb )=\tup{\begin{bmatrix} \prstart_\mora \\ \prstart_\morb \end{bmatrix}, 
        \begin{bmatrix} \mat{A}_\mora & \mat{0}\\
        \mat{B}_\morb\rowscols{\mat{C}_\mora}{j+1:j+m}{:}&\mat{A}_\morb
        \end{bmatrix},
        \begin{bmatrix}
        \mat{B}_\mora\\
        \mat{B}_\morb \rowscols{\mat{D}_\mora}{j+1:j+m}{:}
        \end{bmatrix},
        \begin{bmatrix}
        \rowscols{\mat{C}_\mora}{1:j}{:}&\mat{0}\\
        \mat{D}_\morb \rowscols{\mat{C}_\mora}{j+1:j+m}{:}&\mat{C}_\morb
        \end{bmatrix},
        \begin{bmatrix}
        \rowscols{\mat{D}_\mora}{1:j}{:}\\
        \mat{D}_\morb \rowscols{\mat{D}_\mora}{j+1:j+m}{:}
        \end{bmatrix}}
        \end{equation*}
        \end{widepar}
        With these intermediate calculations, we can start looking at (1) using the trace formula, component by component.
        Clearly, we have 
        \begin{equation*}
        \projA((1))=\begin{bmatrix} \prstart_\mora \\ \prstart_\morb \end{bmatrix}.
        \end{equation*}
        Furthermore, we have
        \begin{equation*}
        \begin{aligned}
        \projB((1))&=\begin{bmatrix} \mat{A}_\mora & \mat{0}\\
            \mat{B}_\morb\rowscols{\mat{C}_\mora}{j+1:j+m}{:}&\mat{A}_\morb
            \end{bmatrix} + \begin{bmatrix}
            \rowscols{\mat{B}_\mora}{:}{i+1:i+k}\\
            \rowscols{\begin{bmatrix}\mat{B}_\morb \rowscols{\mat{D}_\mora}{j+1:j+m}{:}\end{bmatrix}}{:}{i+1:i+k}
            \end{bmatrix}\mat{F}
            \begin{bmatrix}
                \mat{D}_\morb \rowscols{\mat{C}_\mora}{j+1:j+m}{:}&\mat{C}_\morb
                \end{bmatrix}\\
                &=TODO,
        \end{aligned}
        \end{equation*}
        where
        \begin{equation*}
        \mat{F}=\left(\idmat - \rowscols{\begin{bmatrix}
            \mat{D}_\morb \rowscols{\mat{D}_\mora}{j+1:j+m}{:}
            \end{bmatrix}}{:}{i+1:i+k}\right)^{-1}.
        \end{equation*}
        %
        Furthermore, we have:
        \begin{equation*}
        \begin{aligned}
        \projC((1))&=\begin{bmatrix}
            \rowscols{\mat{B}_\mora}{:}{1:i}\\
            \rowscols{\begin{bmatrix}\mat{B}_\morb \rowscols{\mat{D}_\mora}{j+1:j+m}{:}\end{bmatrix}}{:}{1:i}
            \end{bmatrix}+ \begin{bmatrix}
            \rowscols{\mat{B}_\mora}{:}{i+1:i+k}\\
            \rowscols{\begin{bmatrix}\mat{B}_\morb \rowscols{\mat{D}_\mora}{j+1:j+m}{:}\end{bmatrix}}{:}{i+1:i+k}
            \end{bmatrix}\mat{F}
            \rowscols{\begin{bmatrix}
                \mat{D}_\morb \rowscols{\mat{D}_\mora}{j+1:j+m}{:}
                \end{bmatrix}}{:}{1:i},
        \end{aligned}
        \end{equation*}
        %
        \begin{equation*}
        \begin{aligned}
        \projD((1))&=\begin{bmatrix}
            \rowscols{\mat{C}_\mora}{1:j}{:}&\mat{0}
            \end{bmatrix}+
            \rowscols{\mat{D}_\mora}{1:j}{i+1:i+k}\mat{F} 
            \begin{bmatrix}
                \mat{D}_\morb \rowscols{\mat{C}_\mora}{j+1:j+m}{:}&\mat{C}_\morb
                \end{bmatrix},
        \end{aligned}
        \end{equation*}
        and
        \begin{equation*}
        \begin{aligned}
        \projE((1))&=\rowscols{\mat{D}_\mora}{1:j}{1:i}+\rowscols{\mat{D}_\mora}{1:j}{i+1:i+k}\mat{F}\rowscols{\begin{bmatrix}
            \mat{D}_\morb \rowscols{\mat{D}_\mora}{j+1:j+m}{:}
            \end{bmatrix}}{:}{1:i}
        \end{aligned}
        \end{equation*}

        We are now ready to look at the right-hand side of the axiom to prove, which we refer to as (2).
        First, we have:
        \begin{equation*}
        \catid_i \mtimescatmor \morb =\tupp{\prstart_\morb, \mat{A}_\morb, \begin{bmatrix} \mat{0}&\mat{B}_\morb\end{bmatrix},\begin{bmatrix}  \mat{0}\\ \mat{C}_\morb\end{bmatrix}, \begin{bmatrix}\idmat^{i\times i}  & \mat{0}\\ \mat{0}&\mat{D}_\morb \end{bmatrix}}.
        \end{equation*}
        Furthermore, we have:
        \begin{equation*}
        \begin{aligned}
        (\catid_i \mtimescatmor \morb)\mthen \mora &= 
        \langle\begin{bmatrix} \prstart_\morb \\ \prstart_\mora \end{bmatrix}, \begin{bmatrix}\mat{A}_{\morb} & \mat{0}\\ 
        \rowscols{\mat{B}_\mora}{:}{i+1:i+k}\mat{C}_\morb &\mat{A}_{\mora}\end{bmatrix},
        \begin{bmatrix}
        \begin{bmatrix} \mat{0}&\mat{B}_\morb\end{bmatrix}\\
        a
        \end{bmatrix}
        \rangle
        \end{aligned}
        \end{equation*}
        \todo{do}
        \paragraph*{Vanishing I}
        Consider a LTI system $\mora\colon i\mto j$, given by $\genericlti{\mora}$ with state dimension $s\setin \natnumbers$.
        We need to check
        \begin{equation*}
            \Tr_{i,j}^0 (\mora)=\mora.
        \end{equation*}
        Using the trace definition, we can write:
        \begin{equation*}
            \begin{aligned}
                \Tr_{i,j}^0 (\mora)=\langle\prstart_{\mora}, & \mat{A}_\mora+\rowscols{\mat{B}_{\mora}}{:}{i+1:i+k}\mat{F}\rowscols{\mat{C}_{\mora}}{j+1:j+k}{:}, \\
                                                             & \rowscols{\mat{B}_{\mora}}{:}{1:i}+ \rowscols{\mat{B}_{\mora}}{:}{i+1:i+k}\mat{F}\rowscols{\mat{D}_{\mora}}{j+1:j+k}{1:i}, \\
                                                             & \rowscols{\mat{C}_{\mora}}{1:j}{:}+\rowscols{\mat{D}_{\mora}}{1:j}{i+1:i+k}\mat{F}\rowscols{\mat{C}_{\mora}}{j+1:j+k}{:}, \\
                                                             & \rowscols{\mat{D}_{\mora}}{1:j}{1:i}+ \rowscols{\mat{D}_{\mora}}{1:j}{i+1:i+k}\mat{F}\rowscols{\mat{D}_{\mora}}{j+1:j+k}{1:i}\rangle,
            \end{aligned}
        \end{equation*}
        where
        \begin{equation*}
            \mat{F}=\left(\idmat - \rowscols{\mat{D}_{\mora}}{j+1:j+k}{i+1:i+k}\right)^{-1}.
        \end{equation*}
        However, we have:
        \begin{equation*}
            \mat{B}_\mora=\begin{bmatrix}
                \rowscols{\mat{B}_{\mora}}{:}{1:i} & \mat{0}^{s\times 0}
            \end{bmatrix},\quad
            \mat{C}_\mora=\begin{bmatrix}
                \rowscols{\mat{C}_{\mora}}{1:j}{:} \\ \mat{0}^{0\times s}
            \end{bmatrix},
            \quad
            \mat{D}_\mora= \rowscols{\mat{D}_\mora}{1:j}{1:i}
        \end{equation*}
        and therefore
        \begin{equation*}
            \Tr_{i,j}^0 (\mora)=\tup{\prstart{\mora}, \mat{A}_\mora, \mat{B}_\mora, \mat{C}_\mora, \mat{D}_\mora}=\mora.
        \end{equation*}

        \paragraph*{Vanishing II}
        Consider a LTI system $\mora\colon i+k+o\mto j+k+o$, given by $\genericlti{\mora}$ with state dimension $s\setin \natnumbers$.
        We need to check
        \begin{equation*}
            \Tr_{i,j}^{k+o} (\mora)=\Tr_{i,j}^{k}\left(\Tr_{i+k,j+k}^o(\mora)\right).
        \end{equation*}
        Let's start with the left-hand side of the statement.
        We have:
        \begin{equation}
            \label{eq:vanishing-ii-left-lti}
            \begin{aligned}
                \Tr_{i,j}^{k+o} (\mora)=\langle\prstart,\mat{A}_\mora + & \rowscols{\mat{B}_\mora}{:}{i+1:i+k+o}\mat{E} \rowscols{\mat{C}_\mora}{j+1:j+k+o}{:}, \\
                                                                        & \rowscols{\mat{B}_\mora}{:}{1:i}+\rowscols{\mat{B}_\mora}{:}{i+1:i+k+o}\mat{E}\rowscols{\mat{D}_\mora}{j+1:j+k+o}{1:i}, \\
                                                                        & \rowscols{\mat{C}_\mora}{1:j}{:}+\rowscols{\mat{D}_\mora}{1:j}{i+1:i+k+o}\mat{E}\rowscols{\mat{C}_\mora}{j+1:j+k+o}{:}, \\
                                                                        & \rowscols{\mat{D}_\mora}{1:j}{1:i}+\rowscols{\mat{D}_\mora}{1:j}{i+1:i+k+o}\mat{E} \rowscols{\mat{D}_\mora}{j+1:j+k+o}{1:i}\rangle,
            \end{aligned}
        \end{equation}
        where $\mat{E}=(\idmat-\rowscols{\mat{D}_\mora}{j+1:j+k+o}{i+1:i+k+o})^{-1}$.
        Let's look at the right-hand side.
        We have:
        \begin{equation}
            \label{eq:vanishing-ii-left-lti}
            \begin{aligned}
                \Tr_{i+k,j+k}^{o}(\mora)=\langle & \prstart,\mat{A}_\mora + \rowscols{\mat{B}_\mora}{:}{i+k+1:i+k+o}\mat{F}_\mora \rowscols{\mat{C}_\mora}{j+k+1:j+k+o}{:}, \\
                                                 & \rowscols{\mat{B}_\mora}{:}{1:i+k}+\rowscols{\mat{B}_\mora}{:}{i+k+1:i+k+o}\mat{F}\rowscols{\mat{D}_\mora}{j+k+1:j+k+o}{1:i+k}, \\
                                                 & \rowscols{\mat{C}_\mora}{1:j+k}{:}+\rowscols{\mat{D}_\mora}{1:j+k}{i+k+1:i+k+o}\mat{F}\rowscols{\mat{C}_\mora}{j+k+1:j+k+o}{:}, \\
                                                 & \rowscols{\mat{D}_\mora}{1:j+k}{1:i+k}+\rowscols{\mat{D}_\mora}{1:j+k}{i+k+1:i+k+o}\mat{F} \rowscols{\mat{D}_\mora}{j+k+1:j+k+o}{1:i+k}\rangle,
            \end{aligned}
        \end{equation}
        where $\mat{F}=(\idmat-\rowscols{\mat{D}_\mora}{j+k+1:j+k+o}{i+k+1:i+k+o})^{-1}$.
        which we can leverage to express $\Tr_{i,j}^{k}\left(\Tr_{i+k,j+k}^o(\mora)\right)=\ast$.
        Clearly, $\projA(\ast)=\prstart$.
        Furthermore:
        \begin{equation*}
            \begin{aligned}
                \projB(\ast)= & .
                ..
            \end{aligned}
        \end{equation*}
        \todo{finish, this not light}

    \end{example}}

\begin{figure}[h!]
    \centering
    \includegraphics[scale=0.35]{feedback_axiomatic-semicat-trace-nat-X}
    \caption{Naturality in $\Obja$.}
    \label{fig:axiomatic-semicat-trace-nat-X}
\end{figure}

\begin{figure}[h!]
    \centering
    \includegraphics[scale=0.15]{feedback_axiomatic-semicat-trace-vanishing-I}
    \caption{Vanishing I.}
    \label{fig:axiomatic-semicat-trace-vanishing-I}
\end{figure}

\begin{figure}[h!]
    \centering
    \includegraphics[scale=0.2]{feedback_axiomatic-semicat-trace-superposing}
    \caption{Superposing.}
    \label{fig:axiomatic-semicat-trace-superposing}
\end{figure}

