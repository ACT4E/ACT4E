% !TEX root = chapter-standalone.tex
\section{Feedback for symmetric strict monoidal semicategories}
\todo{fill this section}
\begin{widepar}
    \begin{ctdefinition}[Traced symmetric strict monoidal semicategory]
        \label{def:traced-fun-stack-scat}
        We say that a symmetric strict monoidal \SY{semicategory}~$\tup{\CatC, \mtimescat, \idmoncat, \sourceperm{\sigma}, \targetperm{\sigma} }$ is \emph{traced} if it is equipped with a family of functions
        \begin{equation}
            \Tr_{\Obja,\Objb}^\Objc\colon \HomSet{\CatC}{\Obja \mtimescatob \Objc}{\Objb\mtimescatob \Objc}\to \HomSet{\CatC}{\Obja}{\Objb},
        \end{equation}
        satisfying the following axioms:
        \begin{enumerate}

            \item \emph{Naturality in $\Obja$:} For any morphisms $\mora\colon \Obja\mtimescatob \Objc \mto \Objb\mtimescatob \Objc$ and $\morb \colon \Obja' \mto \Obja$, if $\Objc$ has an identity $\catidat\Objc$, then
                  \begin{equation}
                      \Tr_{\Obja', \Objb}^\Objc ( (\morb \mtimescatmor \catidat\Objc) \mthen \mora) = \morb \mthen \Tr_{\Obja, \Objb}^\Objc (\mora).
                  \end{equation}

            \item \emph{Naturality in $\Objb$:}
                  For any morphisms $\mora\colon \Obja\mtimescatob \Objc \mto \Objb\mtimescatob \Objc$ and $\morb \colon \Objb \mto \Objb'$, if $\Objc$ has an identity $\catidat\Objc$, then
                  \begin{equation}
                      \Tr_{\Obja, \Objb'}^\Objc ( \mora \mthen (\morb \mtimescatmor \catidat\Objc) ) = \Tr_{\Obja, \Objb}^\Objc (\mora) \mthen \morb.
                  \end{equation}

            \item \emph{Dinaturality in $\Objc$:}
                  For any morphisms $\mora \colon \Obja \mtimescatob \Objc \mto \Objb \mtimescatob \Objc'$ and $\morb \colon \Objc' \mto \Objc$, if $\catidat\Obja$ and $\catidat\Objb$ exist, then
                  \begin{equation}
                      \Tr_{\Obja, \Objb}^{\Objc} (\mora \mthen (\catidat\Objb \mtimescatmor \morb)) = \Tr_{\Obja, \Objb}^{\Objc'}((\catidat\Obja \mtimescatmor \morb) \mthen \mora).
                  \end{equation}

            \item \emph{Vanishing I:}
                  For any morphism~$\mora \colon \Obja\mto \Objb$ in \CatC,
                  \begin{equation}
                      \label{eq:semicat-trace-vanishing_1}
                      \Tr_{\Obja,\Objb}^\idmoncat (\mora)=\mora.
                  \end{equation}

            \item \emph{Vanishing II:}
                  For any morphism~$\mora \colon \Obja\mtimescatob \Objc \mtimescatob \Objd \mto \Objb\mtimescatob \Objc \mtimescatob \Objd$ in \CatC,
                  \begin{equation}
                      \label{eq:semicat-trace-vanishing_2}
                      \Tr_{\Obja,\Objb}^{\Objc\mtimescatob \Objd}(\mora)=\Tr_{\Obja,\Objb}^\Objc\pars{
                          \Tr_{\Obja \mtimescatob \Objc, \Objb \mtimescatob \Objc}^\Objd(\mora)}.
                  \end{equation}

            \item \emph{Superposing:}
                  For any morphisms~$\mora\colon \Obja\mtimescatob \Objc \mto \Objb\mtimescatob \Objc$ and $\morb \colon \Obje \mto \Objf$ in \CatC,
                  \begin{equation}
                      \label{eq:semicat-trace-superposing}
                      \Tr_{\Obje\mtimescatob \Obja,\Objf\mtimescatob \Objb}^{\Objc}(\morb \mtimescatmor \mora)=\morb \mtimescatmor \Tr_{\Obja,\Objb}^\Objc(\mora).
                  \end{equation}

            \item \emph{Yanking:}
                  For any object $\Objc$, if $\catidat\Objc$ exists, then
                  \begin{equation}
                      \label{eq:semicat-trace-yanking-I}
                      (\sourceperm{\pi} \mthen \Tr_{\Objc,\Objc}^\Objc)\pars{\catidat\Objc \mtimescatmor \catidat\Objc }=\catidat\Objc
                  \end{equation}
                  and
                  \begin{equation}
                      \label{eq:semicat-trace-yanking-II}
                      (\targetperm{\pi} \mthen \Tr_{\Objc,\Objc}^\Objc)\pars{\catidat\Objc \mtimescatmor \catidat\Objc }=\catidat\Objc,
                  \end{equation}
                  where $\pi \setin S_2$ is the odd permutation corresponding to swapping the two copies of $\Objc$.
        \end{enumerate}
    \end{ctdefinition}
\end{widepar}

\todotext{J: the above defines what might more precisely be called a \emph{right} trace... in a \SY{symmetric monoidal category} this will, I think, automatically also give a left trace, because the axioms work nicely with the symmetric braiding... in the above I'm not sure if this works out still all nicely... perhaps or perhaps not... }

\begin{example}
    Consider the semicategory \Moore of Moore machines.
    We show that when equipped with stacking and trace operations as previously defined, it fulfills the trace axioms presented in \cref{def:traced-fun-stack-scat}.

    \paragraph*{Vanishing I}
    Consider a Moore machine $\mora\colon \prin \mto \prout$ with
    \begin{equation*}
        \mora=\tup{\prin,\prst, \prout, \prdyn,\prreadout,\prstart}.
    \end{equation*}
    We define $\idmoncat$ for \Moore as
    \todo{This will be moved to the introduction of the general concept of $\idmoncat$ as an example, as soon as it exists.
    }
    \begin{equation*}
        \idmoncat=\tup{\cObj{},\cObj{},\cObj{},\prdyn_\idmoncat, \prreadout_\idmoncat, \tup{}},
    \end{equation*}
    where
    \begin{equation*}
        \defmapcomma{\prdyn_{\idmoncat}}
        {\cObj{} \cprod \cObj{}}
        {\sto}
        {\cObj{}}
        {\tup{}}
        {\tup{}}
    \end{equation*}
    and
    \begin{equation*}
        \defmapperiod{\prreadout_{\idmoncat}}
        {\cObj{}}
        {\sto}
        {\cObj{}}
        {\tup{}}
        {\tup{}}
    \end{equation*}
    We need to show
    \begin{equation*}
        \Tr_{\prin,\prout}^{\idmoncat}(\mora)=\mora.
    \end{equation*}
    Clearly, we have:
    \begin{equation*}
        \begin{aligned}
            \Tr_{\prin,\prout}^{\idmoncat}(\mora) & =\tup{\prin, \prst, \prout, \prdyn_{\Tr \mora}, \prreadout_{\Tr \mora},\prstart}, \\
                                                  & =\tup{\prin, \prst, \prout, \prdyn, \prreadout,\prstart} \\
                                                  & =\mora,
        \end{aligned}
    \end{equation*}
    where we used
    \begin{equation*}
        \defmapcomma{\prdyn_{\Tr \mora}}
        {\prin \cprod \prst}
        {\sto}
        {\prst}
        {\prinel\tupconcat \prstel}
        {\prdyn(\prinel \tupconcat \proj_\idmoncat (\prreadout(\prstel))\tupconcat \prstel)=\prdyn(\prinel \tupconcat \tup{} \tupconcat \prstel)=\prdyn(\prinel \tupconcat \prstel)},
    \end{equation*}
    and
    \begin{equation*}
        \defmapcomma{\prreadout_{\Tr \mora}}
        {\prst}
        {\sto}
        {\prout}
        {\prstel}
        {\proj_\prout(\prreadout(\prstel))=\prreadout(\prstel)},
    \end{equation*}
    which show the equivalences $\prdyn_{\Tr \mora}=\prdyn$ and $\prreadout_{\Tr \mora}=\prreadout$.

    \paragraph*{Vanishing II}
    Consider a Moore machine $\mora\colon \prObja \cprod \prObjc \cprod \prObjd \mto \prObjb \cprod \prObjc \cprod \prObjd$ with
    \begin{equation*}
        \mora=\tup{\prObja \cprod \prObjc \cprod \prObjd,\prst, \prObjb \cprod \prObjc \cprod \prObjd, \prdyn,\prreadout,\prstart}.
    \end{equation*}
    We need to show
    \begin{equation*}
        \Tr_{\prObja, \prObjb}^{\prObjc \cprod \prObjd}(\mora)=\Tr_{\prObja, \prObjb}^{\prObjc }\left(\Tr_{\prObja\cprod \prObjc,\prObjb\cprod \prObjc }^{\prObjd}(\mora)\right)
    \end{equation*}
    Let's start from the left-hand side.
    We have:
    \begin{equation*}
        \Tr_{\prObja, \prObjb}^{\prObjc \cprod \prObjd}(\mora)=\tup{\prObja, \prst, \prObjb, \prdyna, \prreadouta, \prstart},
    \end{equation*}
    with
    \begin{equation*}
        \defmapcomma{\prdyna}
        {\prObja \cprod \prst}
        {\sto}
        {\prst}
        {\prObjael\tupconcat \prstel}
        {\prdyn(\prObjael\tupconcat \proj_{\prObjc \cprod \prObjd}(\prreadout(\prstel))\tupconcat \prstel)}
    \end{equation*}
    and
    \begin{equation*}
        \defmapperiod{\prreadouta}
        {\prst}
        {\sto}
        {\prObjb}
        {\prstel}
        {\proj_\prObjb(\prreadout(\prstel))}
    \end{equation*}

    We can look at the right-hand side of the axiom in two steps.
    First, we have:
    \begin{equation*}
        \Tr_{\prObja\cprod \prObjc,\prObjb\cprod \prObjc }^{\prObjd}(\mora)=
        \tup{\prObja\cprod \prObjc, \prst, \prObjb\cprod \prObjc, \prdynb, \prreadoutb, \prstart}
    \end{equation*}
    with
    \begin{equation*}
        \defmapcomma{\prdynb}
        {\prObja \cprod \prObjc\cprod \prst}
        {\sto}
        {\prst}
        {\prObjael\tupconcat \prObjcel \tupconcat \prstel}
        {\prdyn(\prObjael\tupconcat \prObjcel \tupconcat \proj_\prObjd(\prreadout(\prstel))\tupconcat \prstel)}
    \end{equation*}
    and
    \begin{equation*}
        \defmapperiod{\prreadoutb}
        {\prst}
        {\sto}
        {\prObjb\cprod \prObjc}
        {\prstel}
        {\proj_{\prObjb\cprod \prObjc} (\prreadout(\prstel))}
    \end{equation*}
    Furthermore:
    \begin{equation*}
        \begin{aligned}
            \Tr_{\prObja, \prObjb}^{\prObjc }\left(\Tr_{\prObja\cprod \prObjc,\prObjb\cprod \prObjc }^{\prObjd}(\mora)\right)
             & =\tup{\prObja, \prst, \prObjb, \prdync,\prreadoutc,\prstart} \\
             & =\tup{\prObja, \prst, \prObjb, \prdyna, \prreadouta, \prstart} \\
             & =\Tr_{\prObja, \prObjb}^{\prObjc \cprod \prObjd}(\mora),
        \end{aligned}
    \end{equation*}
    where we used
    \begin{equation*}
        \begin{aligned}
            \prdync\colon \prObja\cprod \prst & \sto \prst \\
            \prObjael\tupconcat \prstel       & \mapsto
            \prdynb(\prObjael \tupconcat \proj_\prObjc(\prreadoutb(\prstel))\tupconcat \prstel) \\
                                              & =\prdyn(\prObjael \tupconcat \proj_\prObjc(\proj_{\prObjb\cprod \prObjc}(\prreadout(\prstel))) \tupconcat \proj_\prObjd(\prreadout(\prstel))\tupconcat \prstel) \\
                                              & =\prdyn(\prObjael \tupconcat \proj_\prObjc(\prreadout(\prstel)) \tupconcat \proj_\prObjd(\prreadout(\prstel))\tupconcat \prstel) \\
                                              & =\prdyn(\prObjael \tupconcat \proj_{\prObjc\cprod \prObjd}(\prreadout(\prstel)) \tupconcat \prstel)
        \end{aligned}
    \end{equation*}
    %
    and
    %
    \begin{equation*}
        \begin{aligned}
            \prreadoutc\colon \prst & \sto \prObjb \\
            \prstel                 & \mapsto
            \proj_\prObjb(\prreadoutb(\prstel)) \\
                                    & =\proj_\prObjb(\proj_{\prObjb\cprod \prObjc} (\prreadout(\prstel))) \\
                                    & =\proj_\prObjb(\prreadout(\prstel)),
        \end{aligned}
    \end{equation*}
    to show the equivalences $\prdyna=\prdync$ and $\prreadouta=\prreadoutc$.

    \paragraph*{Superposing}
    Consider Moore machines $\mora\colon \prObja \cprod \prObjc \mto \prObjb \cprod \prObjc$ and $\morb\colon \prObje\mto \prObjf$ with
    \begin{equation*}
        \begin{aligned}
            \mora & =\tup{\prObja \cprod \prObjc,\prst_\mora, \prObjb \cprod \prObjc, \prdyn_\mora,\prreadout_\mora,\prstart_\mora} \\
            \morb & =\tup{\prObje,\prst_\morb, \prObjf, \prdyn_\morb,\prreadout_\morb,\prstart_\morb}.
        \end{aligned}
    \end{equation*}
    We need to show:
    \begin{equation*}
        \Tr_{\prObje\cprod \prObja, \prObjf\cprod \prObjb}^{\prObjc}(\morb \mtimescatmor \mora)=\morb \mtimescatmor \Tr_{\prObja,\prObjb}^{\prObjc}(\mora).
    \end{equation*}
    Let's start from the left-hand side.
    We have:
    \begin{equation*}
        \morb\mtimescatmor \mora=\tup{\prObje\cprod \prObja \cprod \prObjc, \prst_\morb \cprod \prst_\mora, \prObjf\cprod \prObjb \cprod \prObjc, \prdyn_{\morb\mtimescatmor\mora},\prreadout_{\morb\mtimescatmor \mora},\prstart_\morb\tupconcat \prstart_\mora}
    \end{equation*}
    Furthermore:
    \begin{equation*}
        \Tr_{\prObje\cprod \prObja, \prObjf\cprod \prObjb}^{\prObjc}(\morb \mtimescatmor \mora)=
        \tup{\prObje\cprod \prObja, \prst_\morb \cprod \prst_\mora, \prObjf\cprod \prObjb, \prdyna,\prreadouta,\prstart_\morb\tupconcat \prstart_\mora}
    \end{equation*}
    with
    \begin{equation*}
        \begin{aligned}
            \prdyna\colon \prObje\cprod \prObja\cprod \prst_\morb \cprod \prst_\mora        & \sto \prst_\morb \cprod \prst_\mora \\
            \prObjeel\tupconcat \prObjael \tupconcat \prstel_\morb \tupconcat \prstel_\mora & \mapsto
            \prdyn_{\morb\mtimescatmor \mora}(\prObjeel\tupconcat \prObjael\tupconcat \proj_\prObjc(\prreadout_{\morb\mtimescatmor \mora}(\prstel_\morb \tupconcat \prstel_\mora))\tupconcat \prstel_\morb \tupconcat \prstel_\mora) \\
                                                                                            & =\prdyn_\morb(\prObjeel\tupconcat \prstel_\morb)\tupconcat \prdyn_\mora(\prObjael\tupconcat \proj_\prObjc(\prreadout_{\morb}(\prstel_\morb)\tupconcat \prreadout_\mora(\prstel_\mora)) \tupconcat \prstel_\mora) \\
                                                                                            & =\prdyn_\morb(\prObjeel\tupconcat \prstel_\morb)\tupconcat \prdyn_\mora(\prObjael\tupconcat \proj_\prObjc(\prreadout_\mora(\prstel_\mora)) \tupconcat \prstel_\mora)
        \end{aligned}
    \end{equation*}
    and
    \begin{equation*}
        \begin{aligned}
            \prreadouta\colon \prst_\morb \cprod \prst_\mora & \sto \prObjf\cprod \prObjb \\
            \prstel_\morb \tupconcat \prstel_\mora           & \mapsto \proj_{\prObjf\cprod \prObjb}(\prreadout_{\morb \mtimescatmor \mora}(\prstel_\morb \tupconcat \prstel_\mora)) \\
                                                             & =\proj_{\prObjf\cprod \prObjb}(\prreadout_{\morb}(\prstel_\morb)\tupconcat \prreadout_{\mora}(\prstel_\mora)) \\
                                                             & =\prreadout_\morb(\prstel_\morb)\tupconcat \proj_\prObjb(\prreadout_\mora(\prstel_\mora)).
        \end{aligned}
    \end{equation*}
    For the right-hand side, we have:
    \begin{equation*}
        \Tr_{\prObja,\prObjb}^{\prObjc}(\mora)=\tup{\prObja,\prst_\mora, \prObjb,\prdyn_{\Tr \mora},\prreadout_{\Tr \mora},\prstart_\mora},
    \end{equation*}
    and
    \begin{equation*}
        \begin{aligned}
            \morb \mtimescatmor \Tr_{\prObja,\prObjb}^{\prObjc}(\mora) & =\tup{\prObje\cprod \prObja,\prst_\morb \cprod \prst_\mora, \prObje\cprod \prObjb,\prdynb,\prreadoutb,\prstart_\morb \tupconcat \prstart_\mora}, \\
                                                                       & =\tup{\prObje\cprod \prObja,\prst_\morb \cprod \prst_\mora, \prObje\cprod \prObjb,\prdyna,\prreadouta,\prstart_\morb \tupconcat \prstart_\mora}
        \end{aligned}
    \end{equation*}
    where we used
    \begin{equation*}
        \begin{aligned}
            \prdynb\colon \prObje\cprod \prObja\cprod \prst_\morb \cprod \prst_\mora       & \sto \prst_\morb \cprod \prst_\mora \\
            \prObjeel\tupconcat \prObjael\tupconcat \prstel_\morb \tupconcat \prstel_\mora & \mapsto
            \prdyn_\morb(\prObjeel \tupconcat \prstel_\morb)\tupconcat \prdyn_{\Tr \mora}(\prObjael\tupconcat \prstel_\mora) \\
                                                                                           & =\prdyn_\morb(\prObjeel \tupconcat \prstel_\morb)\tupconcat \prdyn_\mora(\prObjael\tupconcat \proj_\prObjc(\prreadout(\prstel_\mora))\tupconcat \prstel_\mora)
        \end{aligned}
    \end{equation*}
    and
    \begin{equation*}
        \begin{aligned}
            \prreadoutb\colon \prst_\morb \cprod \prst_\mora & \sto \prObje\cprod \prObjb \\
            \prstel_\morb \tupconcat \prstel_\mora           & \mapsto
            \prreadout_\morb(\prstel_\morb)\tupconcat \prreadout_{\Tr \mora}(\prstel_\mora) \\
                                                             & =\prreadout_\morb(\prstel_\morb)\tupconcat \proj_\prObjb(\prreadout_\mora(\prstel_\mora))
        \end{aligned}
    \end{equation*}
    to show the equivalences $\prdyna=\prdynb$ and $\prreadouta=\prreadoutb$.
\end{example}
%
\begin{example}
    Consider the subsemicategory of \LTI only allowing proper LTI systems as morphisms with the trace defined in \cref{def:trace-lti-prop}.
    The trace axioms are satified.
    We prove the statements not involving identities one by one.
    \paragraph*{Vanishing I}
    Consider a proper LTI system $\mora\colon i\mto j$, given by $\genericplti{\mora}$ with state dimension $s\setin \natnumbers$.
    We need to check
    \begin{equation*}
        \Tr_{i,j}^0 (\mora)=\mora.
    \end{equation*}
    Using the trace definition, we can write:
    \begin{equation*}
        \Tr_{i,j}^0 (\mora)=\tup{\prstart{\mora}, \mat{A}_\mora+\rowscols{\mat{B}_{\mora}}{:}{i+1:i+k}\rowscols{\mat{C}_{\mora}}{j+1:j+k}{:}, \rowscols{\mat{B}_{\mora}}{:}{1:i}, \rowscols{\mat{C}_{\mora}}{1:j}{:}}.
    \end{equation*}
    However, we have:
    \begin{equation*}
        \mat{B}_\mora=\begin{bmatrix}
            \rowscols{\mat{B}_{\mora}}{:}{1:i} & \mat{0}^{s\times 0}
        \end{bmatrix},\quad
        \mat{C}_\mora=\begin{bmatrix}
            \rowscols{\mat{C}_{\mora}}{1:j}{:} \\ \mat{0}^{0\times s}
        \end{bmatrix},
    \end{equation*}
    and therefore
    \begin{equation*}
        \Tr_{i,j}^0 (\mora)=\tup{\prstart{\mora}, \mat{A}_\mora, \mat{B}_\mora, \mat{C}_\mora}=\mora.
    \end{equation*}
    \paragraph*{Vanishing II}
    Consider a proper LTI system $\mora\colon i+k+o\mto j+k+o$, given by $\genericplti{\mora}$ with state dimension $s\setin \natnumbers$.
    We need to check
    \begin{equation*}
        \Tr_{i,j}^{k+o} (\mora)=\Tr_{i,j}^{k}\left(\Tr_{i+k,j+k}^o(\mora)\right).
    \end{equation*}
    Let's start with the left-hand side of the statement.
    We have:
    \begin{equation}
        \label{eq:vanishing2-left}
        \Tr_{i,j}^{k+o} (\mora)=\tup{\prstart, \mat{A}_\mora+\rowscols{\mat{B}_\mora}{:}{i+1:i+k+o}\rowscols{\mat{C}_\mora}{j+1:j+k+o}{:}, \rowscols{\mat{B}_\mora}{:}{1:i},\rowscols{\mat{C}_\mora}{1:j}{:}}.
    \end{equation}
    Let's look at the right-hand side.
    We have:
    \begin{widepar}
        \begin{equation*}
            \Tr_{i+k,j+k}^{o} (\mora)=\tup{\prstart, \mat{A}+\rowscols{\mat{B}_\mora}{:}{i+k+1:i+k+o} \rowscols{\mat{C}_\mora}{j+k+1:j+k+o}{:}, \underbrace{\rowscols{\mat{B}_\mora}{:}{1:i+k}}_{\mat{B}_{\mora,\star}},\underbrace{\rowscols{\mat{C}_\mora}{1:j+k}{:}}_{\mat{C}_{\mora,\star}}},
        \end{equation*}
    \end{widepar}
    which we can leverage to express $\Tr_{i,j}^{k}\left(\Tr_{i+k,j+k}^o(\mora)\right)=\ast$.
    Clearly, $\projA(\ast)=\prstart$.
    Furthermore:
    \begin{equation*}
        \begin{aligned}
            \projB(\ast) & =\mat{A}_\mora + \rowscols{\mat{B}_\mora}{:}{i+k+1:i+k+o}\rowscols{\mat{C}_\mora}{j+k+1:j+k+o}{:}+\rowscols{\mat{B}_{\mora,\star}}{:}{i+1:i+k}\rowscols{\mat{C}_{\mora,\star}}{:}{j+1:j+k} \\
                         & =\mat{A}_\mora + \rowscols{\mat{B}_\mora}{:}{i+k+1:i+k+o}\rowscols{\mat{C}_\mora}{j+k+1:j+k+o}{:}+\rowscols{\mat{B}_{\mora}}{:}{i+1:i+k}\rowscols{\mat{C}_{\mora}}{:}{j+1:j+k} \\
                         & =\mat{A}_\mora + \begin{bmatrix}
                                                \rowscols{\mat{B}_{\mora}}{:}{i+1:i+k} & \rowscols{\mat{B}_\mora}{:}{i+k+1:i+k+o}
                                            \end{bmatrix} \begin{bmatrix}
                                                              \rowscols{\mat{C}_{\mora}}{:}{j+1:j+k} \\
                                                              \rowscols{\mat{C}_\mora}{j+k+1:j+k+o}{:}
                                                          \end{bmatrix} \\
                         & =\mat{A}+\rowscols{\mat{B}_\mora}{:}{i+1:i+k+o}\rowscols{\mat{C}_\mora}{j+1:j+k+o}{:}.
        \end{aligned}
    \end{equation*}
    Furthermore, $\projC(\ast) = \rowscols{\mat{B}_\mora}{:}{1:i}$ and $\projD(\ast) = \rowscols{\mat{C}_\mora}{1:j}{:}$.
    Clearly, the found results correspond to \cref{eq:vanishing2-left}.
    \paragraph*{Superposing}
    Consider the proper LTI systems $\mora\colon i+k\mto j+k$, given by $\genericplti{\mora}$, and $\morb\colon l\mto m$, given by $\genericplti{\morb}$.
    We need to check:
    \begin{equation*}
        \Tr_{l+i,m+j}^k(\morb \mtimescatmor \mora)=\morb \mtimescatmor \Tr_{i,j}^k(\mora).
    \end{equation*}
    Let's start with the left-hand side.
    We have:
    \begin{equation*}
        \morb \mtimescatmor \mora=\tup{\begin{bmatrix}\prstart_\morb\\ \prstart_\mora \end{bmatrix}, \begin{bmatrix} \mat{A}_\morb&\mat{0}\\ \mat{0}&\mat{A}_\mora\end{bmatrix},
            \begin{bmatrix} \mat{B}_\morb&\mat{0}\\ \mat{0}&\mat{B}_\mora\end{bmatrix},\begin{bmatrix} \mat{C}_\morb&\mat{0}\\ \mat{0}&\mat{C}_\mora\end{bmatrix}}
    \end{equation*}
    Now observe that $\mat{B}_\morb$ has $l$ columns, $\mat{B}_\mora$ has $i+k$ columns, $\mat{C}_\morb$ has $m$ rows, and $\mat{C}_\mora$ has $j+k$ rows.
    Let's denote $\Tr_{l+i,m+j}^k(\morb \mtimescatmor \mora)$ by $\ast$.
    Clearly:
    \begin{equation*}
        \projA(\ast)=\begin{bmatrix}\prstart_\morb &\prstart_\mora\end{bmatrix}.
    \end{equation*}
    We have:
    \begin{equation*}
        \begin{aligned}
            \projB(\ast) & =\begin{bmatrix} \mat{A}_\morb&\mat{0}\\ \mat{0}&\mat{A}_\mora\end{bmatrix}+\rowscols{\begin{bmatrix} \mat{B}_\morb&\mat{0}\\ \mat{0}&\mat{B}_\mora\end{bmatrix}}{:}{l+i+1:l+i+k}\rowscols{\begin{bmatrix} \mat{C}_\morb&\mat{0}\\ \mat{0}&\mat{C}_\mora\end{bmatrix}}{m+j+1:m+j+k}{:} \\
                         & =\begin{bmatrix} \mat{A}_\morb&\mat{0}\\ \mat{0}&\mat{A}_\mora\end{bmatrix}+\begin{bmatrix} \mat{0}\\ \rowscols{\mat{B}_\mora}{:}{i+1:i+k}\end{bmatrix}
            \begin{bmatrix}
                \mat{0} & \rowscols{\mat{C}_\mora}{j+1:j+k}{:}
            \end{bmatrix} \\
                         & =\begin{bmatrix} \mat{A}_\morb&\mat{0}\\ \mat{0}&\mat{A}_\mora\end{bmatrix}
            \begin{bmatrix}
                \mat{0} & \mat{0}                                                                  \\
                \mat{0} & \rowscols{\mat{B}_\mora}{:}{i+1:i+k}\rowscols{\mat{C}_\mora}{j+1:j+k}{:}
            \end{bmatrix}
        \end{aligned}
    \end{equation*}
    Furthermore, we have
    \begin{equation*}
        \projC(\ast)=\begin{bmatrix}
            \mat{B}_\morb & \mat{0}                          \\
            \mat{0}       & \rowscols{\mat{B}_\mora}{:}{1:i}
        \end{bmatrix}
    \end{equation*}
    and
    \begin{equation*}
        \projD(\ast)=\begin{bmatrix}
            \mat{C}_\morb & \mat{0}                          \\
            \mat{0}       & \rowscols{\mat{C}_\mora}{1:j}{:}
        \end{bmatrix}
    \end{equation*}
    We can now look at the right-hand side.
    First:
    \begin{equation*}
        \Tr_{i,j}^{k} (\mora)=\tup{\prstart, \mat{A}_\mora+\rowscols{\mat{B}_\mora}{:}{i+1:i+k} \rowscols{\mat{C}_\mora}{j+1:j+k}{:}, \rowscols{\mat{B}_\mora}{:}{1:i},\rowscols{\mat{C}_\mora}{1:j}{:}},
    \end{equation*}
    Therefore, we have:
    \begin{equation*}
        \projA(\morb \mtimescatmor \Tr_{i,j}^k(\mora))=\begin{bmatrix}\prstart_\morb &\prstart_\mora\end{bmatrix},
    \end{equation*}
    \begin{equation*}
        \projB(\morb \mtimescatmor \Tr_{i,j}^k(\mora))=\begin{bmatrix}
            \mat{A}_\morb & \mat{0}                                                                                    \\
            \mat{0}       & \mat{A}_\mora + \rowscols{\mat{B}_\mora}{:}{i+1:i+k} \rowscols{\mat{C}_\mora}{j+1:j+k}{:},
        \end{bmatrix}
    \end{equation*}
    %
    \begin{equation*}
        \projC(\morb \mtimescatmor \Tr_{i,j}^k(\mora))=\begin{bmatrix}
            \mat{B}_\morb & \mat{0}                          \\
            \mat{0}       & \rowscols{\mat{B}_\mora}{:}{1:i}
        \end{bmatrix},
    \end{equation*}
    and
    \begin{equation*}
        \projD(\morb \mtimescatmor \Tr_{i,j}^k(\mora))=
        \begin{bmatrix}
            \mat{C}_\morb & \mat{0}                          \\
            \mat{0}       & \rowscols{\mat{C}_\mora}{1:j}{:}
        \end{bmatrix},
    \end{equation*}
    which clearly correspond to the left-hand side.
\end{example}