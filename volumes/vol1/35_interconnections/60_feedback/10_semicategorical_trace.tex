% !TEX root = chapter-standalone.tex

\section{Feedback for symmetric strict monoidal semicategories}
\todo{fill this section}
\begin{widepar}
    \begin{ctdefinition}[Traced symmetric strict monoidal semicategory]
        \label{def:traced-fun-stack-scat}
        We say that a symmetric strict monoidal \SY{semicategory}~$\tup{\CatC, \mtimescat, \idmoncat, \sourceperm{\sigma}, \targetperm{\sigma} }$ is \emph{traced} if it is equipped with a family of functions
        \begin{equation}
            \Tr_{\Obja,\Objb}^\Objc\colon \HomSet{\CatC}{\Obja \mtimescatob \Objc}{\Objb\mtimescatob \Objc}\to \HomSet{\CatC}{\Obja}{\Objb},
        \end{equation}
        satisfying the following axioms:
        \begin{enumerate}

            \item \emph{Naturality in $\Obja$:} For any morphisms $\mora\colon \Obja\mtimescatob \Objc \mto \Objb\mtimescatob \Objc$ and $\morb \colon \Obja' \mto \Obja$, if $\Objc$ has an identity $\catidat\Objc$, then
                  \begin{equation}
                      \Tr_{\Obja', \Objb}^\Objc ( (\morb \mtimescatmor \catidat\Objc) \mthen \mora) = \morb \mthen \Tr_{\Obja, \Objb}^\Objc (\mora).
                  \end{equation}

            \item \emph{Naturality in $\Objb$:}
                  For any morphisms $\mora\colon \Obja\mtimescatob \Objc \mto \Objb\mtimescatob \Objc$ and $\morb \colon \Objb \mto \Objb'$, if $\Objc$ has an identity $\catidat\Objc$, then
                  \begin{equation}
                      \Tr_{\Obja, \Objb'}^\Objc ( \mora \mthen (\morb \mtimescatmor \catidat\Objc) ) = \Tr_{\Obja, \Objb}^\Objc (\mora) \mthen \morb.
                  \end{equation}

            \item \emph{Dinaturality in $\Objc$:}
                  For any morphisms $\mora \colon \Obja \mtimescatob \Objc \mto \Objb \mtimescatob \Objc'$ and $\morb \colon \Objc' \mto \Objc$, if $\catidat\Obja$ and $\catidat\Objb$ exist, then
                  \begin{equation}
                      \Tr_{\Obja, \Objb}^{\Objc} (\mora \mthen (\catidat\Objb \mtimescatmor \morb)) = \Tr_{\Obja, \Objb}^{\Objc'}((\catidat\Obja \mtimescatmor \morb) \mthen \mora).
                  \end{equation}

            \item \emph{Vanishing I:}
                  For any morphism~$\mora \colon \Obja\mto \Objb$ in \CatC,
                  \begin{equation}
                      \label{eq:semicat-trace-vanishing_1}
                      \Tr_{\Obja,\Objb}^\idmoncat (\mora)=\mora.
                  \end{equation}

            \item \emph{Vanishing II:}
                  For any morphism~$\mora \colon \Obja\mtimescatob \Objc \mtimescatob \Objd \mto \Objb\mtimescatob \Objc \mtimescatob \Objd$ in \CatC,
                  \begin{equation}
                      \label{eq:semicat-trace-vanishing_2}
                      \Tr_{\Obja,\Objb}^{\Objc\mtimescatob \Objd}(\mora)=\Tr_{\Obja,\Objb}^\Objc\pars{
                          \Tr_{\Obja \mtimescatob \Objc, \Objb \mtimescatob \Objc}^\Objd(\mora)}.
                  \end{equation}

            \item \emph{Superposing:}
                  For any morphisms~$\mora\colon \Obja\mtimescatob \Objc \mto \Objb\mtimescatob \Objc$ and $\morb \colon \Obje \mto \Objf$ in \CatC,
                  \begin{equation}
                      \label{eq:semicat-trace-superposing}
                      \Tr_{\Obje\mtimescatob \Obja,\Objf\mtimescatob \Objb}^{\Objc}(\morb \mtimescatmor \mora)=\morb \mtimescatmor \Tr_{\Obja,\Objb}^\Objc(\mora).
                  \end{equation}

            \item \emph{Yanking:}
                  For any object $\Objc$, if $\catidat\Objc$ exists, then
                  \begin{equation}
                      \label{eq:semicat-trace-yanking-I}
                      (\sourceperm{\pi} \mthen \Tr_{\Objc,\Objc}^\Objc)\pars{\catidat\Objc \mtimescatmor \catidat\Objc }=\catidat\Objc
                  \end{equation}
                  and
                  \begin{equation}
                      \label{eq:semicat-trace-yanking-II}
                      (\targetperm{\pi} \mthen \Tr_{\Objc,\Objc}^\Objc)\pars{\catidat\Objc \mtimescatmor \catidat\Objc }=\catidat\Objc,
                  \end{equation}
                  where $\pi \setin S_2$ is the odd permutation corresponding to swapping the two copies of $\Objc$.
        \end{enumerate}
    \end{ctdefinition}
\end{widepar}

\todotext{J: the above defines what might more precisely be called a \emph{right} trace... in a \SY{symmetric monoidal category} this will, I think, automatically also give a left trace, because the axioms work nicely with the symmetric braiding... in the above I'm not sure if this works out still all nicely... perhaps or perhaps not... }

\devel{
    \begin{lemma}
        Consider the subsemicategory of \LTI only allowing propert \LTI systems as morphisms.
        The trace axioms not including identities are valid.
    \end{lemma}

    \begin{proof}
        We prove the statements not involving identities one by one.
        \paragraph*{Vanishing I}
        Consider a proper LTI system $\mora\colon i\mto j$, given by $\genericplti{\mora}$ with state dimension $s\setin \natnumbers$.
        We need to check
        \begin{equation*}
            \Tr_{i,j}^0 (\mora)=\mora.
        \end{equation*}
        Using the trace definition, we can write:
        \begin{equation*}
            \Tr_{i,j}^0 (\mora)=\tup{\prstart{\mora}, \mat{A}_\mora+\rowscols{\mat{B}_{\mora}}{:}{i+1:i+k}\rowscols{\mat{C}_{\mora}}{j+1:j+k}{:}, \rowscols{\mat{B}_{\mora}}{:}{1:i}, \rowscols{\mat{C}_{\mora}}{1:j}{:}}.
        \end{equation*}
        However, we have:
        \begin{equation*}
            \mat{B}_\mora=\begin{bmatrix}
                \rowscols{\mat{B}_{\mora}}{:}{1:i} & \mat{0}^{s\times 0}
            \end{bmatrix},\quad
            \mat{C}_\mora=\begin{bmatrix}
                \rowscols{\mat{C}_{\mora}}{1:j}{:} \\ \mat{0}^{0\times s}
            \end{bmatrix},
        \end{equation*}
        and therefore
        \begin{equation*}
            \Tr_{i,j}^0 (\mora)=\tup{\prstart{\mora}, \mat{A}_\mora, \mat{B}_\mora, \mat{C}_\mora}=\mora.
        \end{equation*}

        \paragraph*{Vanishing II}
        Consider a proper LTI system $\mora\colon i+k+o\mto j+k+o$, given by $\genericplti{\mora}$ with state dimension $s\setin \natnumbers$.
        We need to check
        \begin{equation*}
            \Tr_{i,j}^{k+o} (\mora)=\Tr_{i,j}^{k}\left(\Tr_{i+k,j+k}^o(\mora)\right).
        \end{equation*}
        Let's start with the left-hand side of the statement.
        We have:
        \begin{equation}
            \label{eq:vanishing2-left}
            \Tr_{i,j}^{k+o} (\mora)=\tup{\prstart, \mat{A}+\rowscols{\mat{B}_\mora}{:}{i+1:i+k+o}\rowscols{\mat{C}_\mora}{j+1:j+k+o}{:}, \rowscols{\mat{B}_\mora}{:}{1:i},\rowscols{\mat{C}_\mora}{1:j}{:}}.
        \end{equation}
        Let's look at the right-hand side.
        We have:
        \begin{widepar}
            \begin{equation*}
                \Tr_{i+k,j+k}^{o} (\mora)=\tup{\prstart, \mat{A}+\rowscols{\mat{B}_\mora}{:}{i+k+1:i+k+o} \rowscols{\mat{C}_\mora}{j+k+1:j+k+o}{:}, \underbrace{\rowscols{\mat{B}_\mora}{:}{1:i+k}}_{\mat{B}_{\mora,\star}},\underbrace{\rowscols{\mat{C}_\mora}{1:j+k}{:}}_{\mat{C}_{\mora,\star}}},
            \end{equation*}
        \end{widepar}
        which we can leverage to express $\Tr_{i,j}^{k}\left(\Tr_{i+k,j+k}^o(\mora)\right)=\ast$.
        Clearly, $\projA(\ast)=\prstart$.
        Furthermore:
        \begin{equation*}
            \begin{aligned}
                \projB(\ast) & =\mat{A}_\mora + \rowscols{\mat{B}_\mora}{:}{i+k+1:i+k+o}\rowscols{\mat{C}_\mora}{j+k+1:j+k+o}{:}+\rowscols{\mat{B}_{\mora,\star}}{:}{i+1:i+k}\rowscols{\mat{C}_{\mora,\star}}{:}{j+1:j+k} \\
                             & =\mat{A}_\mora + \rowscols{\mat{B}_\mora}{:}{i+k+1:i+k+o}\rowscols{\mat{C}_\mora}{j+k+1:j+k+o}{:}+\rowscols{\mat{B}_{\mora}}{:}{i+1:i+k}\rowscols{\mat{C}_{\mora}}{:}{j+1:j+k} \\
                             & =\mat{A}_\mora + \begin{bmatrix}
                                                    \rowscols{\mat{B}_{\mora}}{:}{i+1:i+k} & \rowscols{\mat{B}_\mora}{:}{i+k+1:i+k+o}
                                                \end{bmatrix} \begin{bmatrix}
                                                                  \rowscols{\mat{C}_{\mora}}{:}{j+1:j+k} \\
                                                                  \rowscols{\mat{C}_\mora}{j+k+1:j+k+o}{:}
                                                              \end{bmatrix} \\
                             & =\mat{A}+\rowscols{\mat{B}_\mora}{:}{i+1:i+k+o}\rowscols{\mat{C}_\mora}{j+1:j+k+o}{:}.
            \end{aligned}
        \end{equation*}
        Furthermore, $\projC(\ast) = \rowscols{\mat{B}_\mora}{:}{1:i}$ and $\projD(\ast) = \rowscols{\mat{C}_\mora}{1:j}{:}$.
        Clearly, the found results correspond to \cref{eq:vanishing2-left}.

        \paragraph*{Superposing}
        Consider the proper LTI systems $\mora\colon i+k\mto j+k$, given by $\genericplti{\mora}$, and $\morb\colon l\mto m$, given by $\genericplti{\morb}$.
        We need to check:
        \begin{equation*}
            \Tr_{l+i,m+j}^k(\morb \mtimescatmor \mora)=\morb \mtimescatmor \Tr_{i,j}^k(\mora).
        \end{equation*}
        Let's start with the left-hand side.
        We have:
        \begin{equation*}
            \morb \mtimescatmor \mora=\tup{\begin{bmatrix}\prstart_\morb\\ \prstart_\mora \end{bmatrix}, \begin{bmatrix} \mat{A}_\morb&\mat{0}\\ \mat{0}&\mat{A}_\mora\end{bmatrix},
                \begin{bmatrix} \mat{B}_\morb&\mat{0}\\ \mat{0}&\mat{B}_\mora\end{bmatrix},\begin{bmatrix} \mat{C}_\morb&\mat{0}\\ \mat{0}&\mat{C}_\mora\end{bmatrix}}
        \end{equation*}
        Now observe that $\mat{B}_\morb$ has $l$ columns, $\mat{B}_\mora$ has $i+k$ columns, $\mat{C}_\morb$ has $m$ rows, and $\mat{C}_\mora$ has $j+k$ rows.
        Let's denote $\Tr_{l+i,m+j}^k(\morb \mtimescatmor \mora)$ by $\ast$.
        We have:
        \begin{equation*}
            \begin{aligned}
                \projB(\ast) & =\begin{bmatrix} \mat{A}_\morb&\mat{0}\\ \mat{0}&\mat{A}_\mora\end{bmatrix}+\rowscols{\begin{bmatrix} \mat{B}_\morb&\mat{0}\\ \mat{0}&\mat{B}_\mora\end{bmatrix}}{:}{l+i+1:l+i+k}\rowscols{\begin{bmatrix} \mat{C}_\morb&\mat{0}\\ \mat{0}&\mat{C}_\mora\end{bmatrix}}{m+j+1:m+j+k}{:} \\
                             & =\begin{bmatrix} \mat{A}_\morb&\mat{0}\\ \mat{0}&\mat{A}_\mora\end{bmatrix}+\begin{bmatrix} \mat{0}\\ \rowscols{\mat{B}_\mora}{:}{i+1:i+k}\end{bmatrix}
                \begin{bmatrix}
                    \mat{0} & \rowscols{\mat{C}_\mora}{j+1:j+k}{:}
                \end{bmatrix} \\
                             & =\begin{bmatrix} \mat{A}_\morb&\mat{0}\\ \mat{0}&\mat{A}_\mora\end{bmatrix}
                \begin{bmatrix}
                    \mat{0} & \mat{0}                                                                  \\
                    \mat{0} & \rowscols{\mat{B}_\mora}{:}{i+1:i+k}\rowscols{\mat{C}_\mora}{j+1:j+k}{:}
                \end{bmatrix}
            \end{aligned}
        \end{equation*}
    \end{proof}
}

\begin{figure}[h!]
    \centering
    \includegraphics[scale=0.35]{feedback_axiomatic-semicat-trace-nat-X}
    \caption{Naturality in $\Obja$.
    }
    \label{fig:axiomatic-semicat-trace-nat-X}
\end{figure}

\begin{figure}[h!]
    \centering
    \includegraphics[scale=0.15]{feedback_axiomatic-semicat-trace-vanishing-I}
    \caption{Vanishing I.}
    \label{fig:axiomatic-semicat-trace-vanishing-I}
\end{figure}

\begin{figure}[h!]
    \centering
    \includegraphics[scale=0.2]{feedback_axiomatic-semicat-trace-superposing}
    \caption{Superposing.}
    \label{fig:axiomatic-semicat-trace-superposing}
\end{figure}

