% !TEX root = chapter-standalone.tex

\section{Feedback examples}
\publictodomessage

\devel{
\subsection{Continuous LTI}
\todojira{196}{rough here, to be polished and add trace thing with Gr, finish example}
In the following, we present the category of continuous linear time-invariant dynamical systems.
Let $T=\nonNegReals$ represent time.

\paragraph{Objects}
The objects of the category are natural numbers $n\setin \natnumbers $.
These represent sequences $s\colon T\to \reals^n$, defining tuples in $\reals^n\times T$.

\paragraph{Morphisms}
A morphism in \LTI is an arrow $l\to m$, for $l,m\setin \natnumbers $.
The arrow describes the transformation of an input sequence $u\colon T\to \reals^l$ into an output sequence $y\colon T\to \reals^m$.
Such an arrow is given by a continuous time LTI system of the form
\begin{equation}
    \begin{aligned}
        \dot{x}_{t} & =A x_{t}+B u_{t} \\
        y_{t}       & =C x_{t} +D u_{t},
    \end{aligned}
\end{equation}
where $A\setin \reals^{n\times n}$, $B\setin \reals^{n\times l}$, $C\setin \reals^{m\times n}$, $D\setin \reals^{m\times l}$, $x_t\setin \reals^n$, $u_t\setin \reals^l$, and $y_t\setin \reals^m$.
For brevity, we refer to the LTI as the tuple $\tup{A,B,C,D}$, leaving the dimensions of the input/output implicit.

\paragraph{Identity morphism}
The identity morphism for $l\setin \Ob_\LTI$ is an arrow $l\to l$, parametrized by the system $\tup{0,0^{1\times l},0^{l\times 1},\mathbb{I}^{l\times l}}$.

\paragraph{Composition of morphisms}
Given two arrows $a\to b$ and $b\to c$, parametrized by the two LTI systems $\tup{A_1,B_1,C_1,D_1}$ and $\tup{A_2,B_2,C_2,D_2}$, their composition is an arrow $a\to c$, parametrized by the LTI system $\tup{A,B,C,D}$, where
\begin{equation}
    A=\begin{bmatrix}
        A_1    & 0   \\
        B_2C_1 & A_2
    \end{bmatrix},\quad
    B=\begin{bmatrix}
        B_1 \\
        B_2D_1
    \end{bmatrix},\quad
    C=\begin{bmatrix}
        D_2C_1 & C_2
    \end{bmatrix}, \quad
    D=D_2D_1.
\end{equation}
}

\subsection{Trace of a linear transformation}
\label{subsec:trace-linear}
Consider the category $\FinVect_\reals$ of finite dimensional real vector spaces, which has as objects finite dimensional vector spaces and as morphisms linear maps between them.
Using the tensor product $\otimes$ of real vector spaces as monoidal product, one can show $\FinVect_\reals$ is a monoidal category.
Consider a linear transformation $f\colon B\otimes D\to C\otimes D$, with $B,C,D$ vector spaces with bases $\{b_i\},\{c_j\}$, and $\{d_k\}$ respectively.
Here, the trace (also called ``partial trace'') is a linear function $\Tr_{B,C}
    ^D(f)\colon B\to C$, given by
\begin{equation}
    \left(\Tr_{B,C}
    ^D(f) \right)_{i,j}=\sum_{k}f_{i\otimes k,j\otimes k}
\end{equation}
\todojira{197}{@J: Explain notation and show the simple case of sum of diagonals.
    Follow JL's lecture}

\todojira{198}{@Gioele: Add figure using the graphical notations for tensor operations}
