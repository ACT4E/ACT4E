% !TEX root = chapter-standalone.tex

\section{Feedback for systems}
\label{sec:feedback-processes}

\todotext{write intro to this section}


\subsection{Moore machines}

\todotext{write it up}

%\begin{definition}[Moore machine, 3rd definition]
%    \label{def:moore_machine_3rd}
%    A Moore machine is a tuple
%    \begin{equation}
%        \label{eq:moore-tuple-improved-again}
%        \tup{\prinL,\prstL,\proutL,\prdyn,\prreadout,\prstart},
%    \end{equation}
%    where~$\prinL$,~$\prstL$, and~$\proutL$ are objects of~$\SetL$,~$\prdyn$ is a function of the type
%    \begin{equation}
%        \prdyn \colon \prinL \cprod \prstL \mto_{\SetL} \prstL,
%    \end{equation}
%    $\prreadout$ is a function of the type
%    \begin{equation}
%        \prreadout \colon \prstL \mto_{\SetL} \proutL,
%    \end{equation}
%    and~$\prstart \setin \prstL$.
%\end{definition}

\begin{definition}
Let $\mora = \tup{\prinL \cprod \Objc,\prstL,\proutL \cprod \Objc,\prdyn_\mora,\prreadout_\mora,\prstart}$ be a Moore machine. The \emph{trace of} $\mora$ \emph{over} $\Objc$ is the Moore machine 
\begin{equation}
\Tr_{\prinL,\proutL}^\Objc \mora = \tup{\prinL,\prstL,\proutL, \prdyn_{\Tr \mora},  \prreadout_{\Tr \mora} ,\prstart}
\end{equation}
where 
\begin{equation}
  \defmapperiodset{
        \prdyn_{\Tr \mora}
    }{
        \prinL \cprod \prstL 
    }{
        \prstL 
    }{
        \prinel \tupconcat \prstel
    }{
        \prdyn_\mora(\prinel \tupconcat \pi_\Objc  \prreadout (\prstel) \tupconcat \prstel)
    }
\end{equation}
and 
\begin{equation}
  \defmapperiodset{
        \prreadout_{\Tr \mora}
    }{
        \prstL 
    }{
        \proutL 
    }{
        \prstel
    }{
        \pi_\proutL \prreadout (\prstel)
    }
\end{equation}

Above, $\pi_\Objc$ is the projection $\prinL \cprod \Objc \to \Objc$ and $\pi_\proutL$ is the projection $\proutL \cprod \Objc \to \proutL$.
\end{definition}


\subsection{Proper LTI systems}

\todotext{write it up}


