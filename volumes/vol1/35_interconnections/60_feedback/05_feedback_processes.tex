% !TEX root = chapter-standalone.tex

\section{Feedback for systems}
\label{sec:feedback-processes}

\todotext{write intro to this section}

\subsection{Moore machines}

\todotext{write up this subsection}

%\begin{definition}[Moore machine, 3rd definition]
%    \label{def:moore-machine}
%    A Moore machine is a tuple
%    \begin{equation}
%        \label{eq:moore-tuple-improved-again}
%        \tup{\prinL,\prstL,\proutL,\prdyn,\prreadout,\prstart},
%    \end{equation}
%    where~$\prinL$,~$\prstL$, and~$\proutL$ are objects of~$\SetL$,~$\prdyn$ is a function of the type
%    \begin{equation}
%        \prdyn \colon \prinL \cprod \prstL \mtoin{\SetL} \prstL,
%    \end{equation}
%    $\prreadout$ is a function of the type
%    \begin{equation}
%        \prreadout \colon \prstL \mtoin{\SetL} \proutL,
%    \end{equation}
%    and~$\prstart \setin \prstL$.
%\end{definition}

\begin{definition}\label{def:trace-moore}
    Let $\mora = \tup{\prinL \cprod \prObjw,\prstL,\proutL \cprod \prObjw, \prdyn_\mora,\prreadout_\mora,\prstart}$ be a \SY{Moore machine}.
    The \emph{trace of} $\mora$ \emph{over} $\prObjw$ is the Moore machine
    \begin{equation}
        \Tr_{\prinL,\proutL}^\prObjw \mora = \tup{\prinL,\prstL,\proutL, \prdyn_{\Tr \mora}, \prreadout_{\Tr \mora},\prstart}
    \end{equation}
    where
    \begin{equation}
        \defmapperiodset{
            \prdyn_{\Tr \mora}
        }{
            \prinL \cprod \prstL
        }{
            \prstL
        }{
            \prinel \tupconcat \prstel
        }{
            \prdyn_\mora(\prinel \tupconcat \proj_\prObjw \prreadout (\prstel) \tupconcat \prstel)
        }
    \end{equation}
    and
    \begin{equation}
        \defmapperiodset{
            \prreadout_{\Tr \mora}
        }{
            \prstL
        }{
            \proutL
        }{
            \prstel
        }{
            \proj_\proutL \prreadout (\prstel)
        }
    \end{equation}
    %
    Above, $\pi_\prObjw$ is the projection $\prinL \cprod \prObjw \to \prObjw$ and $\proj_\proutL$ is the projection $\proutL \cprod \prObjw \to \proutL$.
\end{definition}

\subsection{Proper LTI systems}

\todotext{write up this subsection}

%\begin{ctdefinition}[Category \LTI]
%    The category  \iindex{\LTI} of LTI systems is defined by:
%    \begin{enumerate}
%        \item \emph{Objects}: natural numbers.
%        \item \emph{Morphisms}: A morphism in \LTI from~$\styleobj{l}\setin \natnumbers$ to~$\styleobj{m}\setin \natnumbers$ is a continuous time LTI system~$\genericlti{}$.
%              %where~$\mat{A}\setin \reals^{\ntimesn}$,~$\mat{B}\setin \reals^{n\times l}$,~$\mat{C}\setin \reals^{m\times n}$,~$\mat{D}\setin \reals^{m\times l}$.
%              Note that the dimensions of the input/output are implicit.
%        \item \emph{Composition}: Given morphisms~$\mora \colon \styleobj{a}\mto \styleobj{b}$ and~$\morb\colon \styleobj{b}\mto \styleobj{c}$, described by the LTI systems
%              \begin{equation}
%                  \begin{aligned}
%                       & \genericlti{\mora} \\
%                       & \genericlti{\morb},
%                  \end{aligned}
%              \end{equation}
%              their composition~$(\mora \mthen \morb)\colon \styleobj{a}\mto \styleobj{c}$ is the LTI system~$\genericlti{}$, where
%              \begin{equation}
%                  \begin{aligned}
%                      \prstart & =\begin{bmatrix}
%                                      \prstart_\mora \\
%                                      \prstart_\morb \\
%                                  \end{bmatrix},\quad
%                      \mat{A}=\begin{bmatrix}
%                                  \mat{A}_\mora              & \mat{0}       \\
%                                  \mat{B}_\morb\mat{C}_\mora & \mat{A}_\morb
%                              \end{bmatrix},\quad
%                      \mat{B}=\begin{bmatrix}
%                                  \mat{B}_\mora \\
%                                  \mat{B}_\morb\mat{D}_\mora
%                              \end{bmatrix}, \\
%                      \mat{C}  & =\begin{bmatrix}
%                                      \mat{D}_\morb\mat{C}_\mora & \mat{C}_\morb
%                                  \end{bmatrix}, \quad
%                      \mat{D}=\mat{D}_\morb\mat{D}_\mora.
%                  \end{aligned}
%              \end{equation}
%        \item \emph{Identities}: the identity for~$\styleobj{l}\setin \natnumbers$ is the system~$\tup{\mat{0}^{0\times 1},\mat{0}^{0\times 0},\mat{0}^{0\times l},\mat{0}^{l\times 0},\mathbb{I}^{l\times l}}$.
%
%    \end{enumerate}
%\end{ctdefinition}

\begin{definition}\label{def:trace-lti-prop}
    Let $\mora = \genericplti{}$ be a proper LTI system from $\styleobj{l}\setin \natnumbers$ to~$\styleobj{m}\setin \natnumbers$, and suppose we are given factorizations $\styleobj{l} = \styleobj{i} + \styleobj{k}$ and $\styleobj{m} = \styleobj{j} + \styleobj{k}$ of the dimension of the input and output spaces, $\styleobj{U}$ and $\styleobj{Y}$, respectively.
    Then we can think of $\styleobj{U} = \reals^l$ as an (internal) direct sum of the form $\reals^l = \reals^i \oplus \reals^k$, and similarly so for $\styleobj{U} = \reals^m = \reals^j \oplus \reals^k$.
    We will use the notation $\styleobj{U} = \styleobj{U_1} \oplus \styleobj{U_2}$ and $\styleobj{Y} = \styleobj{Y_1} \oplus \styleobj{Y_2}$, respectively for these factorizations.
    This induces corresponding factorizations of the matrices $\mat{B}$ and $\mat{C}$ as block matrices:
    \begin{equation}
        \mat{B} =
        \begin{bmatrix}
            \mat{B}_1 & \mat{B}_2
        \end{bmatrix}
        \qquad
        \mat{C} =
        \begin{bmatrix}
            \mat{C}_1 \\
            \mat{C}_2
        \end{bmatrix}.
    \end{equation}

    Now we define the proper LTI system $\Tr_{\styleobj{i},\styleobj{j}}^{\styleobj{k}}\mora$ (the trace of $\mora$ over the $\reals^k$-component of the input and output spaces) as
    \begin{equation}\label{eq:proper-LTI-trace-formula}
        \Tr_{\styleobj{i},\styleobj{j}}^{\styleobj{k}}\mora = \tupp{\prstart,\mat{A} + \mat{B}_2 \mat{C}_2, \mat{B}_1,\mat{C}_1}
    \end{equation}
    % 
    %    \begin{equation}
    %        \Tr_{\styleobj{i},\styleobj{j}}^{\styleobj{k}}\mora = \tupp{\prstart,\mat{A} + \rowscols{\mat{B}}{:}{i+1:i+k}\rowscols{\mat{C}}{j+1:j+k}{:},\rowscols{\mat{B}}{:}{1:i},\rowscols{\mat{C}}{1:j}{:}}
    %    \end{equation}
\end{definition}

\devel{

    \subsection{LTI systems}

    Let $\mora = \genericlti{}$ be a LTI system from $\styleobj{l}\setin \natnumbers$ to~$\styleobj{m}\setin \natnumbers$, and suppose we are given factorizations $\styleobj{l} = \styleobj{i} + \styleobj{k}$ and $\styleobj{m} = \styleobj{j} + \styleobj{k}$ of the dimension of the input and output spaces, $\styleobj{U}$ and $\styleobj{Y}$, respectively.
    Then we can think of $\styleobj{U} = \reals^l$ as an (internal) direct sum of the form $\reals^l = \reals^i \oplus \reals^k$, and similarly so for $\styleobj{U} = \reals^m = \reals^j \oplus \reals^k$.
    We will use the notation $\styleobj{U} = \styleobj{U_1} \oplus \styleobj{U_2}$ and $\styleobj{Y} = \styleobj{U_1} \oplus \styleobj{U_2}$, respectively for these factorizations.
    This induces corresponding factorizations of the matrices $\mat{B}$, $\mat{C}$, and $\mat{D}$ as block matrices:
    \begin{equation}
        \mat{B} =
        \begin{bmatrix}
            \mat{B}_1 & \mat{B}_2
        \end{bmatrix}
        \qquad
        \mat{C} =
        \begin{bmatrix}
            \mat{C}_1 \\
            \mat{C}_2
        \end{bmatrix}
        \qquad
        \mat{D} =
        \begin{bmatrix}
            \mat{D}_{11} & \mat{D}_{12} \\
            \mat{D}_{21} & \mat{D}_{22}
        \end{bmatrix}.
    \end{equation}

    \begin{definition}\label{def:trace-lti}
        Let an LTI system $\mora = \genericlti{}$ and factorizations $\styleobj{l} = \styleobj{i} + \styleobj{k}$ and $\styleobj{m} = \styleobj{j} + \styleobj{k}$ be given, and let
        \begin{equation}\label{eq:LTI-induced-block-forms}
            \mat{B} =
            \begin{bmatrix}
                \mat{B}_1 & \mat{B}_2
            \end{bmatrix}
            \qquad
            \mat{C} =
            \begin{bmatrix}
                \mat{C}_1 \\
                \mat{C}_2
            \end{bmatrix}
            \qquad
            \mat{D} =
            \begin{bmatrix}
                \mat{D}_{11} & \mat{D}_{12} \\
                \mat{D}_{21} & \mat{D}_{22}
            \end{bmatrix}
        \end{equation}
        be the corresponding factorizations of $\mat{B}$, $\mat{C}$, and $\mat{D}$. If the matrix $\mat{I} - \mat{D}_{22}$ is invertible, we define the LTI system $\Tr_{\styleobj{i},\styleobj{j}}^{\styleobj{k}}\mora$ as
        \begin{equation}
            \begin{aligned}
                \langle\prstart, & \mat{A} + \mat{B}_2 (\idmat- \mat{D}_{22})^{-1} \mat{C}_2 , B_1 + D_2(\idmat- \mat{D}_{22})^{-1}
                D_{21}, \\
                                 & C_1 + D_{12}(\idmat- \mat{D}_{22})^{-1}C_2, D_{11} + D_{12}(\idmat- \mat{D}_{22})^{-1}D_{21} \rangle.
            \end{aligned}
        \end{equation}
        %        \begin{equation}
        %            \begin{aligned}
        %                \langle\prstart,\mat{A} + \rowscols{\mat{B}}{:}{i+1:i+k}\mat{E} \rowscols{\mat{C}}{j+1:j+k}{:}, & \rowscols{\mat{C}}{1:j}{:}+\rowscols{\mat{D}}{1:j}{i+1:i+k}\mat{E}\rowscols{\mat{C}}{j+1:j+k}{:}, \\
        %                                                                                                                & \rowscols{\mat{D}}{1:j}{1:i}+\rowscols{\mat{D}}{1:j}{i+1:i+k}\mat{E} \rowscols{\mat{D}}{j+1:j+k}{1:i}\rangle,
        %            \end{aligned}
        %        \end{equation}
    \end{definition}

    \begin{example}
        Let's consider the simple signal-flow diagram reported in \cref{fig:lti-ex-signalflow}.
        Note that the represented signals are scalar.
        In basic engineering classes, you learn that you can find an expression of the output $y(t)$ as a function of the input $u(t)$, by following the diagram.
        In particular, one can write
        \begin{equation}
            \label{eq:lti-input-output-gain}
            \begin{aligned}
                K(u(t)-Cy(t))=y & \Leftrightarrow Ku(t)-KCy(t)=y \\
                                & \Leftrightarrow Ku(t)=y(t)+KCy(t) \\
                                & \Leftrightarrow Ku(t)=y(t)(1+KC) \\
                                & \Leftrightarrow y(t)=\frac{K}{1+KC}u(t).
            \end{aligned}
        \end{equation}
        Now, we want to get the same expression, but interpreting the presented system as a composition of LTI systems, and leveraging the newly introduced concept of trace.

        This can be visualized as in \cref{fig:lti-ex-signalflow-bis}.
        The systems are given by

        \begin{equation*}
            \begin{aligned}
                \mora & = \langle\mat{0}^{0\times 1}, \mat{0}^{0\times 0}, \mat{0}^{0\times 1}, \mat{0}^{1\times 0}, \begin{bmatrix} {1} & {-1}\end{bmatrix}\rangle \\
                \morb & =\tup{\mat{0}^{0\times 1}, \mat{0}^{0\times 0}, \mat{0}^{0\times 1}, \mat{0}^{1\times 0}, K} \\
                \morc & =\tup{\mat{0}^{0\times 1}, \mat{0}^{0\times 0}, \mat{0}^{0\times 1}, \mat{0}^{1\times 0}, \begin{bmatrix} 1\\ 1\end{bmatrix}} \\
                \mord & =\langle\mat{0}^{0\times 1}, \mat{0}^{0\times 0}, \mat{0}^{0\times 1}, \mat{0}^{1\times 0},
                \begin{bmatrix}
                    1 & 0 \\
                    0 & C
                \end{bmatrix}\rangle
            \end{aligned}
        \end{equation*}

        Intuitively, $\mora$ is acting as the subtraction, $\morb$ as the gain $K$, $\morc$ is splitting the signal in two identical copies, one of which is used by the controller, expressed via $\mord$.
        All of these LTI systems are described by their last component, and are therefore explicit input-output relationships.
        We can compose the LTI systems.

        We just look at the last component of the composition, given by:
        \begin{equation*}
            \begin{aligned}
                \mat{D} & =\mat{D}_\mord \mat{D}_\morc \mat{D}_\morb \mat{D}_\mora \\
                        & =\begin{bmatrix}1 & 0\\ 0 & {C}\end{bmatrix} \begin{bmatrix} 1\\ 1 \end{bmatrix} K \begin{bmatrix} 1 & -1 \end{bmatrix} \\
                        & =\begin{bmatrix}K \\ {KC}\end{bmatrix}\begin{bmatrix} 1 & -1 \end{bmatrix} \\
                        & =\begin{bmatrix}
                               K  & -K  \\
                               CK & -CK
                           \end{bmatrix}
            \end{aligned}
        \end{equation*}

        We can now apply the formula for the trace and we get:
        \begin{equation*}
            \begin{aligned}
                \projE(\Tr_{1,1}^{w}) & =K - K(1+CK)^{-1}
                CK \\
                                      & =\frac{K}{1+CK}.
            \end{aligned}
        \end{equation*}
        From this we get the LTI system from \cref{eq:lti-input-output-gain} (in other words, a direct input-output dependency).

        \begin{marginfigure}
            \begin{center}
                \includesag{signal-flow-trace}
            \end{center}
            \caption{Example with signal-flow diagram.}
            \label{fig:lti-ex-signalflow}
        \end{marginfigure}
    \end{example}

    \begin{figure}
        \centering
        \includesag{signal-flow-to-lti}
        \caption{Signal-flow diagram transformed into composition of LTI systems.}
        \label{fig:lti-ex-signalflow-bis}
    \end{figure}

}

