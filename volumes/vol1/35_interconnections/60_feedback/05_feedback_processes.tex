% !TEX root = chapter-standalone.tex

\section{Feedback for systems}
\label{sec:feedback-processes}

\todotext{write intro to this section}

\subsection{Moore machines}

\todotext{write up this subsection}

%\begin{definition}[Moore machine, 3rd definition]
%    \label{def:moore-machine}
%    A Moore machine is a tuple
%    \begin{equation}
%        \label{eq:moore-tuple-improved-again}
%        \tup{\prinL,\prstL,\proutL,\prdyn,\prreadout,\prstart},
%    \end{equation}
%    where~$\prinL$,~$\prstL$, and~$\proutL$ are objects of~$\SetL$,~$\prdyn$ is a function of the type
%    \begin{equation}
%        \prdyn \colon \prinL \cprod \prstL \mtoin{\SetL} \prstL,
%    \end{equation}
%    $\prreadout$ is a function of the type
%    \begin{equation}
%        \prreadout \colon \prstL \mtoin{\SetL} \proutL,
%    \end{equation}
%    and~$\prstart \setin \prstL$.
%\end{definition}

\begin{definition}\label{def:trace-moore}
    Let $\mora = \tup{\prinL \cprod \prObjw,\prstL,\proutL \cprod \prObjw, \prdyn_\mora,\prreadout_\mora,\prstart}$ be a \SY{Moore machine}.
    The \emph{trace of} $\mora$ \emph{over} $\prObjw$ is the Moore machine
    \begin{equation}
        \Tr_{\prinL,\proutL}^\prObjw \mora = \tup{\prinL,\prstL,\proutL, \prdyn_{\Tr \mora}, \prreadout_{\Tr \mora},\prstart}
    \end{equation}
    where
    \begin{equation}
        \defmapperiodset{
            \prdyn_{\Tr \mora}
        }{
            \prinL \cprod \prstL
        }{
            \prstL
        }{
            \prinel \tupconcat \prstel
        }{
            \prdyn_\mora(\prinel \tupconcat \pi_\prObjw \prreadout (\prstel) \tupconcat \prstel)
        }
    \end{equation}
    and
    \begin{equation}
        \defmapperiodset{
            \prreadout_{\Tr \mora}
        }{
            \prstL
        }{
            \proutL
        }{
            \prstel
        }{
            \pi_\proutL \prreadout (\prstel)
        }
    \end{equation}
    %
    Above, $\pi_\Objc$ is the projection $\prinL \cprod \prObjw \to \prObjw$ and $\pi_\proutL$ is the projection $\proutL \cprod \prObjw \to \proutL$.
\end{definition}

\subsection{Proper LTI systems}

\todotext{write up this subsection}

%\begin{ctdefinition}[Category \LTI]
%    The category  \iindex{\LTI} of LTI systems is defined by:
%    \begin{enumerate}
%        \item \emph{Objects}: natural numbers.
%        \item \emph{Morphisms}: A morphism in \LTI from~$\styleobj{l}\setin \natnumbers$ to~$\styleobj{m}\setin \natnumbers$ is a continuous time LTI system~$\genericlti{}$.
%              %where~$\mat{A}\setin \reals^{\ntimesn}$,~$\mat{B}\setin \reals^{n\times l}$,~$\mat{C}\setin \reals^{m\times n}$,~$\mat{D}\setin \reals^{m\times l}$.
%              Note that the dimensions of the input/output are implicit.
%        \item \emph{Composition}: Given morphisms~$\mora \colon \styleobj{a}\mto \styleobj{b}$ and~$\morb\colon \styleobj{b}\mto \styleobj{c}$, described by the LTI systems
%              \begin{equation}
%                  \begin{aligned}
%                       & \genericlti{\mora} \\
%                       & \genericlti{\morb},
%                  \end{aligned}
%              \end{equation}
%              their composition~$(\mora \mthen \morb)\colon \styleobj{a}\mto \styleobj{c}$ is the LTI system~$\genericlti{}$, where
%              \begin{equation}
%                  \begin{aligned}
%                      \prstart & =\begin{bmatrix}
%                                      \prstart_\mora \\
%                                      \prstart_\morb \\
%                                  \end{bmatrix},\quad
%                      \mat{A}=\begin{bmatrix}
%                                  \mat{A}_\mora              & \mat{0}       \\
%                                  \mat{B}_\morb\mat{C}_\mora & \mat{A}_\morb
%                              \end{bmatrix},\quad
%                      \mat{B}=\begin{bmatrix}
%                                  \mat{B}_\mora \\
%                                  \mat{B}_\morb\mat{D}_\mora
%                              \end{bmatrix}, \\
%                      \mat{C}  & =\begin{bmatrix}
%                                      \mat{D}_\morb\mat{C}_\mora & \mat{C}_\morb
%                                  \end{bmatrix}, \quad
%                      \mat{D}=\mat{D}_\morb\mat{D}_\mora.
%                  \end{aligned}
%              \end{equation}
%        \item \emph{Identities}: the identity for~$\styleobj{l}\setin \natnumbers$ is the system~$\tup{\mat{0}^{0\times 1},\mat{0}^{0\times 0},\mat{0}^{0\times l},\mat{0}^{l\times 0},\mathbb{I}^{l\times l}}$.
%
%    \end{enumerate}
%\end{ctdefinition}

\begin{definition}\label{def:trace-lti-prop}
    Let $\mora = \genericplti{}$ be a proper LTI system from $\styleobj{l}\setin \natnumbers$ to~$\styleobj{m}\setin \natnumbers$.
    Given factorizations $\styleobj{l} = \styleobj{i} + \styleobj{k}$ and $\styleobj{m} = \styleobj{j} + \styleobj{k}$, we define the proper LTI system $\Tr_{\styleobj{i},\styleobj{j}}^{\styleobj{k}}\mora$ as
    \begin{equation}
        \tupp{\prstart,\mat{A} + \rowscols{\mat{B}}{:}{i+1:k}\rowscols{\mat{C}}{j+1:j+k}{:},\rowscols{\mat{B}}{:}{1:i},\rowscols{\mat{C}}{1:j}{:}}
    \end{equation}
    where the notation $\rowscols{\mat{A}}{a:b}{c:d}$ means considering rows $a$ to $b$ and columns $c$ to $d$ of matrix $\mat{A}$.
    In case only $:$ is given, all rows/column are considered.

    \begin{comment}
    where $\mat{B}_\styleobj{i}$ and $\mat{B}_\styleobj{k}$ are such that
    \begin{equation}
        \mat{B} = \begin{bmatrix}
            \mat{B}_\styleobj{i} & \mat{B}_\styleobj{k}
        \end{bmatrix}
    \end{equation}
    and $\mat{C}_\styleobj{j}$ and $\mat{C}_\styleobj{k}$ are such that
    \begin{equation}
        \mat{C} = \begin{bmatrix}
            \mat{C}_\styleobj{j} \\ \mat{C}_\styleobj{k} \\
        \end{bmatrix}.
    \end{equation}
    \end{comment}
\end{definition}

\begin{definition}\label{def:trace-lti}
    Let $\mora = \genericlti{}$ be a LTI system from $\styleobj{l}\setin \natnumbers$ to~$\styleobj{m}\setin \natnumbers$.
    Given factorizations $\styleobj{l} = \styleobj{i} + \styleobj{k}$ and $\styleobj{m} = \styleobj{j} + \styleobj{k}$, we define the LTI system $\Tr_{\styleobj{i},\styleobj{j}}^{\styleobj{k}}\mora$ as
    \begin{equation}
        \begin{aligned}
            \langle\prstart,\mat{A} + \rowscols{\mat{B}}{:}{i+1:i+k}\mat{E} \rowscols{\mat{C}}{j+1:j+k}{:}, & \rowscols{\mat{C}}{1:j}{:}+\rowscols{\mat{D}}{1:j}{i+1:i+k}\mat{E}\rowscols{\mat{C}}{j+1:j+k}{:}, \\
                                                                                                            & \rowscols{\mat{D}}{1:j}{1:i}+\rowscols{\mat{D}}{1:j}{i+1:i+k}\mat{E} \rowscols{\mat{D}}{j+1:j+k}{1:i}\rangle,
        \end{aligned}
    \end{equation}
    where $\mat{E}=(\idmat-\rowscols{\mat{D}}{j+1:j+k}{i+1:i+k})^{-1}$ and $\mat{D}$ is chosen to make the expression invertible.
    \begin{comment}
    where $\mat{B}_\styleobj{i}$ and $\mat{B}_\styleobj{k}$ are such that
    \begin{equation}
        \mat{B} = \begin{bmatrix}
            \mat{B}_\styleobj{i} & \mat{B}_\styleobj{k}
        \end{bmatrix}
    \end{equation}
    and $\mat{C}_\styleobj{j}$ and $\mat{C}_\styleobj{k}$ are such that
    \begin{equation}
        \mat{C} = \begin{bmatrix}
            \mat{C}_\styleobj{j} \\ \mat{C}_\styleobj{k} \\
        \end{bmatrix}.
    \end{equation}
    \end{comment}
\end{definition}

\begin{example}
    Let's consider the simple signal-flow diagram reported in \cref{fig:lti-ex-signalflow}.
    Note that the represented signals are scalar.
    In basic engineering classes, you learn that you can find an expression of the output $y(t)$ as a function of the input $u(t)$, by following the diagram.
    In particular, one can write
    \begin{equation}
        \label{exa:lti-input-output-gain}
        \begin{aligned}
            K(u(t)-Cy(t))=y & \Leftrightarrow Ku(t)-KCy(t)=y \\
                            & \Leftrightarrow Ku(t)=y(t)+KCy(t) \\
                            & \Leftrightarrow Ku(t)=y(t)(1+KC) \\
                            & \Leftrightarrow y(t)=\frac{K}{1+KC}u(t).
        \end{aligned}
    \end{equation}
    Now, we want to get the same expression, but interpreting the presented system as a composition of LTI systems, and leveraging the newly introduced concept of trace.

    This can be visualized as in \cref{fig:lti-ex-signalflow-bis}.
    The systems are given by

    \begin{equation*}
        \begin{aligned}
            \mora & = \langle\mat{0}^{0\times 1}, \mat{0}^{0\times 0}, \mat{0}^{0\times 1}, \mat{0}^{1\times 0}, \begin{bmatrix} {1} & {-1}\end{bmatrix}\rangle \\
            \morb & =\tup{\mat{0}^{0\times 1}, \mat{0}^{0\times 0}, \mat{0}^{0\times 1}, \mat{0}^{1\times 0}, K} \\
            \morc & =\tup{\mat{0}^{0\times 1}, \mat{0}^{0\times 0}, \mat{0}^{0\times 1}, \mat{0}^{1\times 0}, \begin{bmatrix} 1\\ 1\end{bmatrix}} \\
            \mord & =\langle\mat{0}^{0\times 1}, \mat{0}^{0\times 0}, \mat{0}^{0\times 1}, \mat{0}^{1\times 0},
            \begin{bmatrix}
                1 & 0 \\
                0 & C
            \end{bmatrix}\rangle
        \end{aligned}
    \end{equation*}

    We can compose the LTI systems.
    For simplicity, we just look at the last component of the composition, given by:
    \begin{equation*}
        \begin{aligned}
            \mat{D} & =\mat{D}_\mord \mat{D}_\morc \mat{D}_\morb \mat{D}_\mora \\
                    & =\begin{bmatrix}1 & 0\\ 0 & {C}\end{bmatrix} \begin{bmatrix} 1\\ 1 \end{bmatrix}  K \begin{bmatrix} 1 & -1 \end{bmatrix} \\
                    & =\begin{bmatrix}K \\ {KC}\end{bmatrix}\begin{bmatrix} 1 & -1 \end{bmatrix} \\
                    & =\begin{bmatrix}
                           K  & -K  \\
                           CK & -CK
                       \end{bmatrix}
        \end{aligned}
    \end{equation*}
    We can now apply the formula for the trace and we get:
    \begin{equation*}
        \begin{aligned}
            \projE(\Tr_{1,1}^{w}) & =K - K(1+CK)^{-1}CK \\
                                  & =\frac{K}{1+CK}.
        \end{aligned}
    \end{equation*}
    From this we get the LTI system from \cref{exa:lti-input-output-gain} (in other words, a direct input-output dependency).

    \begin{marginfigure}
        \begin{center}
            \begin{tikzpicture}[auto,node distance=1.25cm]
                \node [input, name=input] {};
                \node [sum, right of=input] (sum){};
                \node [blockk, right of=sum] (gain) {$K$};
                \node [blockk, below of=gain] (cont) {$C$};
                \draw [->] (input) -- node[] {$u(t)$}(sum);
                \draw [->] (sum) -- (gain);
                \draw [->] ($(gain.east)+(0,0)$) -- ($(gain.east)+(1.5,0)$)-| ($(cont.east)+(1.5,0)$)|- ($(cont.east)+(0,0)$);
                \draw [->] ($(gain.east)+(0,0)$) -- node[] {$y(t)$} ($(gain.east)+(2.5,0)$);
                \draw [->] ($(cont.west)+(0,0)$) -- ($(cont.west)+(0,0)$)-| node[pos=0.9] {$-$} ($(sum.south)+(0,0)$);
            \end{tikzpicture}
            \caption{Standard feedback control system structure. \label{fig:std_ctr}}
        \end{center}
        \caption{Example with signal-flow diagram.}
        \label{fig:lti-ex-signalflow}
    \end{marginfigure}
\end{example}

\begin{figure}
    \begin{tikzpicture}
        \node[block] (minus) at (0,0) {$\mora$};
        \node[block, right=1cm of minus] (k) {$\morb$};
        \node[block, right=1cm of k] (delta) {$\morc$};
        \node[block] (cont) at (6.5,-0.5) {$\mord$};
        \draw[-Triangle, styleobjects] (minus) -- (k);
        \draw[-Triangle, styleobjects] (k) -- (delta);
        \draw[-Triangle, styleobjects] (delta) -- node[pos=0.5,below] {$y(t)$} (cont);
        \draw[-Triangle, styleobjects] ($(delta.east)+(0,0.2)$) -- node[pos=1,above] {$y(t)$} ($(delta.east)+(3.5,0.2)$);
        \draw[-Triangle, styleobjects] ($(minus.west)+(-1,0.2)$) -- node[pos=-0.2,above] {$u(t)$} ($(minus.west)+(0,0.2)$);
        \draw[-Triangle, styleobjects] ($(cont.east)+(0,0.0)$) -- node[pos=1,above] {$w(t)$} ($(cont.east)+(1.25,0.0)$)-| ($(cont.east)+(1.25,-1)$) -- ($(minus.west)+(-1,-1.5)$)|-($(minus.west)+(-1,-0.2)$)|- ($(minus.west)+(0,-0.2)$);
        \draw[draw=morphisms, thick] (-1.4,-1) rectangle ++(9.2,1.7);
    \end{tikzpicture}
    \caption{Todo.}
    \label{fig:lti-ex-signalflow-bis}
\end{figure}