% !TEX root = chapter-standalone.tex

  \section{Modeling behavior}

    We will develop here a general-purpose formalization of system behaviors.

    In the example above with Moore machines, we said its behavior was a function $\prin^\natnumbers \to \prout^\natnumbers$.
    Here the set~$\natnumbers$ played the role of time: each $k \setin \natnumbers$ corresponded to a time step.
    In the following, we allow as a model of time any totally ordered set $\Time = \tup{\Time, \leq}$.

    When we want to talk about discrete-time signals, natural choices for $\Time$ might be $\Time = \natnumbers$ or $\Time = \wnumbers$.
    Similarly, if we want to talk about continuous time systems, we might choose $\Time = \reals$.
    %And when we want to talk about event-based systems, we can choose $\Time$ to consist of a set of possible events, ordered by their sequence of occurence.

    Given a set $\Obja$, we consider the set $\Obja^\Time$ of functions $\Time \to \Obja$ as the set of time-dependent signals taking values in the set $\Obja$.
    Then a system behavior is, for example, a function $\styleobj{U}^\Time \to \styleobj{Y}^\Time$.
    For more flexibility in our formalism, we will also allow behaviors which are \emph{relations} $\styleobj{U}^\Time \to \styleobj{Y}^\Time$.
    In other words, in general, the externally observable behavior of a system is a relation between input signals and output signals.

    \subsection{Categories of behaviors}

    \begin{definition}
        Let~$\Obja$ and~$\Objb$ be sets.
        An \emph{input-output behavior}, with input space $\Obja$ and output space $\Objb$, is a relation
        \begin{equation*}
            \relA \subseteq \Obja^\Time \cartprod \Objb^\Time.
        \end{equation*}
    \end{definition}

    \begin{definition}
        Fix a totally ordered set~$\Time = \tup{\Time, \leq}$.
        The category~$\Beh_\Time$ of input-output behaviors is the category where:
        \begin{enumerate}
            \item Objects consist of all sets of the form~$\Obja^\Time$, with~$\Obja$ ranging over all sets.
            \item Morphisms from~$\Obja^\Time$ to~$\Objb^\Time$ are relations from~$\Obja^\Time$ to~$\Objb^\Time$:
                  \begin{equation}
                      \HomSet{\Beh_\Time}{\Obja^\Time}{\Objb^\Time} = \powerset (\Obja^\Time \cartprod \Objb^\Time).
                  \end{equation}
            \item Composition is the usual composition of relations.
            \item The identity morphism at an object~$\Obja^\Time$ is the usual identity relation from~$\Obja^\Time$ to itself.
        \end{enumerate}
    \end{definition}

    Sometimes we will wish to consider signal spaces $\Obja$ which are vector spaces over some field~$\field$.
    Then the set~$\Obja^\Time$ naturally inherits a~$\field$-vector space structure from~$\Obja$ (by defining addition and scalar multiplication of functions pointwise), and we may consider input-output behaviors
    \begin{equation*}
        \relA \subseteq \Obja^\Time \oplus \Objb^\Time
    \end{equation*}
    which are~$\field$-linear relations.

    \begin{definition}
        Fix a totally ordered set~$\Time = \tup{\Time, \leq}$.
        The category~$\LBeh_\Time$ of~$\field$-linear input-output behaviors is the category where:
        \begin{enumerate}
            \item Objects consist of all~$\field$-vector spaces of the form~$\Obja^\Time$, ranging over all~$\field$-vector spaces~$\Obja$.
            \item Morphisms from~$\Obja^\Time$ to $\Objb^\Time$ are linear subspaces of~$\Obja^\Time \oplus \Objb^\Time$.
            \item The identity morphism at an object~$\Obja^\Time$ is the usual identity relation from~$\Obja^\Time$ to itself (this is always~$\field$-linear).
            \item Composition is the usual composition of relations.
        \end{enumerate}
    \end{definition}

