% !TEX root = chapter-standalone.tex

  \section{Modeling behavior}

\todotext{the below needs editing/rewriting}

    We present here a general-purpose formalization of system behaviors, using relations between signals. This is certainly not the only way one might model system behaviors; we choose to introduce this model because it provides a flexible yet unified view on many different types of behaviors.  
    
\subsection{Models of time}

In order to describe a dynamical system, say, we typically use coordinate systems with some spatial or configuration-space coordinates, as well as a coordinate for keeping track of time. For dynamical systems, it is common to model time using the real numbers $\reals$. However, depending on the systems or situations at hand, various other models of time might be appropriate. 

For instance, in many real-life situations, time is measured at a certain granularity, and not, as the case with $\reals$, up to arbitrary precision. For example, flight delays might be reported with a precision up to units of 5 minutes, while delays in the local Zurich transport system are reported up to the minute. In these cases, the integers $\wnumbers$ might be a more fitting model of time, where the step from one integer to the next corresponds to the smallest unit of time being measured. 

In other situations, we might want to measure time as progressing only forward from a fixed starting moment; in these cases models of time such as $\reals_{\geq 0}$ or $\natnumbers$ might be appropriate. 

Or, if you are the main character from the movie ``Arrival'', then maybe your personal model of time is a circle. 

With the exception of this last example, all the models of time discussed above have the common feature of coming equipped with a total order structure ``$\leq$'': for any two elements $\ela$ and $\elb$, it is the case that either $\ela \leq \elb$ or $\elb \leq \ela$. 

In order to accommodate various models of time, in the following we will sometimes work with a generic model of time given by a totally ordered set $\Time = \tup{\Time, \leq}$. 

\subsection{Sequences of events}

When working in a frame of reference with both spatial and temporal coordinates, one might say that an event is something that happens at a specific place and at a specific time. Sometimes, however, we don't care about where and when events happen, we only care about the order in which events happen. For example, consider a simple model of a finite state machine and think of inputs to the machine as events. We usually care only about the order of the inputs and not at what precise time of day they are actually inputed. 

\todotext{@J: finish this subsection}


\subsection{Signals}

\todotext{@J: rewrite this subsectoin}


    In the example above with Moore machines, we said its behavior was a function $\prin^\natnumbers \to \prout^\natnumbers$.
    Here the set~$\natnumbers$ played the role of time: each $k \setin \natnumbers$ corresponded to a time step.
   

    When we want to talk about discrete-time signals, natural choices for $\Time$ might be $\Time = \natnumbers$ or $\Time = \wnumbers$.
    Similarly, if we want to talk about continuous time systems, we might choose $\Time = \reals$.
    %And when we want to talk about event-based systems, we can choose $\Time$ to consist of a set of possible events, ordered by their sequence of occurence.

    Given a set $\Obja$, we consider the set $\Obja^\Time$ of functions $\Time \to \Obja$ as the set of time-dependent signals taking values in the set $\Obja$.
    Then a system behavior is, for example, a function $\styleobj{U}^\Time \to \styleobj{Y}^\Time$.
    For more flexibility in our formalism, we will also allow behaviors which are \emph{relations} $\styleobj{U}^\Time \to \styleobj{Y}^\Time$.
    In other words, in general, the externally observable behavior of a system is a relation between input signals and output signals.

    \subsection{Categories of behaviors}

    \begin{definition}
        Let~$\Obja$ and~$\Objb$ be sets.
        An \emph{input-output behavior}, with input space $\Obja$ and output space $\Objb$, is a relation
        \begin{equation*}
            \relA \subseteq \Obja^\Time \cartprod \Objb^\Time.
        \end{equation*}
    \end{definition}

    \begin{definition}
        Fix a totally ordered set~$\Time = \tup{\Time, \leq}$.
        The category~$\Beh_\Time$ of input-output behaviors is the category where:
        \begin{enumerate}
            \item Objects consist of all sets of the form~$\Obja^\Time$, with~$\Obja$ ranging over all sets.
            \item Morphisms from~$\Obja^\Time$ to~$\Objb^\Time$ are relations from~$\Obja^\Time$ to~$\Objb^\Time$:
                  \begin{equation}
                      \HomSet{\Beh_\Time}{\Obja^\Time}{\Objb^\Time} = \powerset (\Obja^\Time \cartprod \Objb^\Time).
                  \end{equation}
            \item Composition is the usual composition of relations.
            \item The identity morphism at an object~$\Obja^\Time$ is the usual identity relation from~$\Obja^\Time$ to itself.
        \end{enumerate}
    \end{definition}

\todotext{J: @J: include a remark here about how to encode signal spaces of the form $\listsof{\Obja}$ in a way that matches the above definition}

\todotext{J: @J: explain how the different actions of Moore machines discussed so far fit into this picture}

\todotext{J: @J: include LTI action as an example}

\subsection{Linear behaviors}

    Sometimes we will wish to consider signal spaces $\Obja$ which are vector spaces over some field~$\field$.
    Then the set~$\Obja^\Time$ naturally inherits a~$\field$-vector space structure from~$\Obja$ (by defining addition and scalar multiplication of functions pointwise), and we may consider input-output behaviors
    \begin{equation*}
        \relA \subseteq \Obja^\Time \oplus \Objb^\Time
    \end{equation*}
    which are~$\field$-linear relations.

    \begin{definition}
        Fix a totally ordered set~$\Time = \tup{\Time, \leq}$.
        The category~$\LBeh_\Time$ of~$\field$-linear input-output behaviors is the category where:
        \begin{enumerate}
            \item Objects consist of all~$\field$-vector spaces of the form~$\Obja^\Time$, ranging over all~$\field$-vector spaces~$\Obja$.
            \item Morphisms from~$\Obja^\Time$ to $\Objb^\Time$ are linear subspaces of~$\Obja^\Time \oplus \Objb^\Time$.
            \item The identity morphism at an object~$\Obja^\Time$ is the usual identity relation from~$\Obja^\Time$ to itself (this is always~$\field$-linear).
            \item Composition is the usual composition of relations.
        \end{enumerate}
    \end{definition}
    
  \todotext{J: @J: remark that LTI action as an example that has linear behaviors}

