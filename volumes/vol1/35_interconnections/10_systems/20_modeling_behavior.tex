% !TEX root = chapter-standalone.tex

\section{Modeling behavior}

We present here a general-purpose formalization of system behaviors, using relations between signals.
This is certainly not the only way one might model system behaviors; we choose to introduce this model because it provides a flexible yet unified view on many types of behaviors.

\subsection{Models of time}

In order to describe a dynamical system, say, we typically use coordinate systems with some spatial or configuration-space coordinates, as well as a coordinate for keeping track of time.
For dynamical systems, it is common to model time using the real numbers \reals.
However, depending on the systems or situations at hand, various other models of time might be appropriate.

For instance, in many real-life situations, time is measured at a certain granularity, and not, as the case with \reals, up to arbitrary precision.
For example, flight delays might be reported with a precision up to units of 5 minutes, while delays in the local Zurich transport system are reported up-to-the-minute.
In these cases, the integers $\wnumbers$ might be a more fitting model of time, where the step from one integer to the next corresponds to the smallest unit of time being measured.

In other situations, we might want to measure time as progressing only forward from a fixed starting moment; in these cases models of time such as \nonNegReals or \natnumbers might be appropriate.

Or, if you are the main character from the movie ``Arrival'', then maybe your personal model of time is a circle.

\todotext{include in the margin a circular graphic from the movie arrival}

Except for this last example, all the models of time discussed above have the common feature of coming equipped with a totally ordered \SY{poset} structure ``$\leq$''.
\footnote{We recall that totally ordered means that for any two elements $\ela$ and $\elb$, it is the case that either $\ela \leq \elb$ or $\elb \leq \ela$.}  (Do you see what fails in the case of the circle?)

In order to accommodate various models of time, we will sometimes simply work with a totally ordered set $\Time = \tup{\Time, \posleq}$ as a generic model of time.
\todotext{\alphubel: @JL: Here the \SY{poset} and the set should have different colors.}

\subsection{Sequences of events}

When working in a frame of reference with both spatial and temporal coordinates, one might say that an event is something that happens in a specific place or context and at a specific time.

For example, in the case of discrete-time LTI systems, we considered them acting on elements of $\streamsof\prinL$.
Here, a sequence of events is elements of $\prinL$ indexed by a discrete time variable $t \setin \natnumbers$.
Or, in the case of continuous LTI systems, we considered actions on continuous streams of events, modeled by some variant of $\prinL^{\reals}$ -- elements of $\prinL$ indexed by a time variable $t \setin \reals$.

Sometimes, however, we don't care about \emph{when} events happen.
Sometimes we only care about \emph{which} events happen and in what \emph{order}.

For example, consider a simple model of a finite state machine and think of inputs to the machine as events.
We might care only about the order of the inputs and not at what precise time of day they are received.

In such cases, it may fit best to model the possible finite sequences of inputs by the set $\listsof{\prinL}$, where $\prinL$ is the set of possible inputs.
Or, if we want to consider infinite sequences of events, we can use $\streamsof\prinL$.
Note that here, the \natnumbers in $\streamsof\prinL$ is modeling the \emph{order} of events, while in the case of discrete-time LTI systems, the \natnumbers in $\streamsof\prinL$ may be modeling the actually \emph{timing} of events (and from this timing, we can also derive an order).

\subsection{Signals}

We will use the term ``signal'' as a unifying notion for the various types of things that systems might act upon.

When discussing Moore and \SY{More machines}, we considered these machines acting on signals given by sequences of inputs and outputs.
For instance, \emph{signals valued in} $\prinL$ might be modeled by $\listsof{\prinL}$, $\streamsof\prinL$, or $\prinL^{\wnumbers}$.

For LTI systems, on the other hand, signals valued in $\prinL$ might be modeled by $\streamsof\prinL$ or $\prinL^{\wnumbers}$ in the discrete-time case, or in the continuous-time case by variants of $\prinL^{\nonNegReals}$ or $\prinL^{\reals}$ -- such as $C^1(\reals, \prinL)$ --- which take into account such aspects as differentiability, \etc

All of these cases are different ways to take a set (or other type of space) $\prinL$ and build from it a set of signals valued in $\prinL$.
And in fact they all do so in a functorial manner.
We have already seen that
\begin{equation}
    \prinL \mapsto \listsof{\prinL}
\end{equation}
is part of a \SY{functor} $\Set \fto \Set$.
Or, for a fixed model of time $\Time$, the operation
\begin{equation}
    \prinL \mapsto \prinL^{\Time}
\end{equation}
extends to a \SY{functor} $\Set \fto \Set$ by defining its action on morphisms to be
\begin{equation}
    \defmapunnamed{
        \HomSet{\Set}{\Obja}{\Objb}
    }{
        \to
    }{
        \HomSet{\Set}{\Obja^\Time}{\Objb^\Time}
    }{
        \mora
    }{
        (\sigma \mapsto \sigma \mthen \mora).
    }
\end{equation}
%
Similarly,
%
\begin{equation}
    \prinL \mapsto C^1(\reals, \prinL)
\end{equation}
%
can be viewed, for example, as part of a \SY{functor} to \Set from a category whose objects are of the form $\reals^n$, for any $n \setin \natnumbers$, and whose morphisms are $C^1$ functions.
The action of this \SY{functor} on morphisms is
\begin{equation}
    \defmapunnamed{
        C^1(\styleobj{\reals^m},\styleobj{\reals^n})
    }{
        \to
    }{
        \HomSet{\Set}{\styleobj{C^1(\reals, \reals^m)}}{\styleobj{C^1(\reals, \reals^n)}}
    }{
        \mora
    }{
        (\gamma \mapsto \gamma \mthen \mora).
    }
\end{equation}

In general, we will work with signal spaces of the form $\signalsin{\prinL}$, where $\signals$ is some (semi)functor which encodes how to build a space of signals from an initial set of space $\prinL$.

\subsection{Behaviors}

\todotext{@JL: rewrite this section}

In the context of \SY{Moore machines}, we said that the behavior of a machine $\prinL \to \proutL$ might be modeled via an action as a function $\streamsof\prinL  \to \streamsof\proutL$ or $\listsof{\prinL} \to \listsof{\proutL}$.
More generally, we can consider actions which produce a function $\signalsin{\prinL} \to \signalsin{\proutL}$, where $\signals$ is some chosen construction of signal spaces.

In some cases, we will want to not only consider actions (which, by definition are \SY{functors} into the category \Set), but also other categories for modeling notions of behavior.
In the following we will use categories of relations as general-purpose settings for modeling behaviors of systems.

\begin{definition}\label{def:input-output-beh}
    Let $\signalsin{\Obja}$ and $\signalsin{\Objb}$ be sets which model signals with values in $\Obja$ and~$\Objb$, respectively.
    An \emph{input-output behavior}, with input space $\Obja$ and output space $\Objb$, is a relation
    \begin{equation}
        \relA \setsubseteq \signalsin{\Obja} \cartprod \signalsin{\Objb}.
    \end{equation}
\end{definition}

\begin{definition}
    \label{def:Beh}
    Given a \SY{semicategory} \CatC (modeling some systems) and a \SY{semifunctor} $\signals: \CatC \to \Set$ (modeling sets of signals), we define the category~$\Beh_\signals$ of input-output behaviors with respect to $\signals$ as the category where:
    \begin{enumerate}
        \item Objects consist of all sets of the form~$\signalsin{\Obja}$, with~$\Obja$ ranging over all objects of \CatC.
        \item Morphisms from~$\signalsin{\Obja}$ to~$\signalsin{\Objb}$ are relations from~$\signalsin{\Obja}$ to~$\signalsin{\Objb}$:
              \begin{equation}
                  \HomSet{\Beh_\signals}{\signalsin{\Obja}}{\signalsin{\Objb}} = \powerset (\signalsin{\Obja} \cartprod \signalsin{\Objb}).
              \end{equation}
        \item Composition is the usual composition of relations.
        \item The \SY{identity morphism} at an object~$\signalsin{\Obja}$ is the usual identity relation from~$\signalsin{\Obja}$ to itself.
    \end{enumerate}
\end{definition}

\todotext{J: @JL: explain how the different actions of \SY{Moore machines} discussed so far fit into this picture}

\todotext{J: @JL: include LTI action as an example}

\todotext{J: are there examples where we might want to consider a construction $\signals$ which is distinct from the action \SY{functor} itself?}

\todotext{J: include an example where indeterminacy is involved in the behavior}

\subsection{Linear behaviors}

Let us consider input/output spaces $\Obja$ which are \SY{vector spaces} over some field~$\field$ and consider the signal-space construction $\signalsin{\Obja} = \Obja^\Time$ for some model $\Time$ of time.
The set $\Obja^\Time$ naturally inherits a~$\field$-vector space structure from~$\Obja$ (by defining addition and scalar multiplication of functions point wise), and so we may consider input-output behaviors
\begin{equation}
    \relA \setsubseteq \Obja^\Time \oplus \Objb^\Time
\end{equation}
which are~$\field$-linear relations.

\begin{definition}\label{def:linear-io-behavior}
    Fix a totally ordered set~$\Time = \tup{\Time, {\posleq}}$.
    The category~$\LBeh_\Time$ of~$\field$-linear input-output behaviors is the category where:
    \begin{enumerate}
        \item Objects consist of all~$\field$-vector spaces of the form~$\Obja^\Time$, ranging over all~$\field$-vector spaces~$\Obja$.
        \item Morphisms from~$\Obja^\Time$ to $\Objb^\Time$ are linear subspaces of~$\Obja^\Time \oplus \Objb^\Time$.
        \item The \SY{identity morphism} at an object~$\Obja^\Time$ is the usual identity relation from~$\Obja^\Time$ to itself (this is always~$\field$-linear).
        \item Composition is the usual composition of relations.
    \end{enumerate}
\end{definition}

\todotext{J: @JL: remark that LTI action as an example that has linear behaviors}
