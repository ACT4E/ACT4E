% !TEX root = chapter-standalone.tex

\section{LTI systems}
\label{sec:LTI-systems}

We now define LTI systems.

\todotext{J: do we want to make a remark that the definition of LTI below is for continuous time, and that there is also a version for discrete time?}

\todo{Let's just do finite-dimensional systems with matrices}
\begin{definition}[LTI System]
    \label{def:LTI_syst}
    A \emph{linear time-invariant dynamical (LTI) system}, in a so-called state-space representation, is specified by real \SY{vector spaces}~$\prin=\reals^l$ (input space),~$\prout=\reals^m$ (output space), and~$\prst=\reals^n$ (state space), along with a system of equations of the form
    \begin{align}
        \dot{\mat{x}}(t) & = \mat{A}\mat{x}(t) + \mat{B}\mat{u}(t) \label{eq:LTI-system-dyn} \\
        \mat{y}(t)       & = \mat{C}\mat{x}(t) + \mat{D}\mat{u}(t) \label{eq:LTI-system-ro},
    \end{align}
    and an \emph{initial state}~$\prstart\setin \prst$, where~$t \setin \nonNegReals$,~$\mat{u}(t) \setin \prin$,~$\mat{y}(t) \setin \prout$,~$\mat{x}(t) \setin \prst$, and where~$\mat{A}$,~$\mat{B}$,~$\mat{C}$,~$\mat{D}$ are real matrices of appropriate dimension.
\end{definition}
\begin{remark}
    \cref{def:LTI_syst} includes the particular case of matrices with 0 dimension.
    For instance, in a system with $\prin=\reals^l$ and $\prout=\reals^m$, we could have $\mat{A}\setin \reals^{0\times 0}$ (in other words, no state).
    This would imply $\mat{B}\setin \reals^{0\times l}$, $\mat{C}\setin \reals^{m\times 0}$, and $\mat{D}\setin \reals^{m\times l}$.
    Matrices with zero rows and/or zero columns multiply exactly as other matrices.
    For instance, the multiplication of a $\reals^{0\times 0}$ matrix with a $\reals^{0\times l}$ matrix, will return a $\reals^{0\times l}$ matrix.
\end{remark}

We compactly refer to an LTI system by writing it as a tuple~$\genericlti{}$.

The equations \cref{eq:LTI-system-dyn} describe the dynamics.
The equations \cref{eq:LTI-system-ro} describe the output, or, one might say, the variables that are ``exposed'' or externally visible.

The matrix $\mat{D}$ is called the \emph{feedthrough term}.
We will call an LTI system \emph{proper} if~$\mat{D}=\mat{0}^{\matdim{l}{m}}$.

\begin{remark}
    When using state-space LTI systems as models, it is typical that the equations \cref{eq:LTI-system-dyn} are chosen using physical laws and first principles reasoning, while the matrices $\mat{C}$ and $\mat{D}$ in \cref{eq:LTI-system-ro} are rather chosen based on what information from our model is explicitly relevant or accessible.
\end{remark}

\begin{example}
    \label{exa:sping-mass-lti}
    Consider a mass $m$ lying on a frictionless surface, and attached to two springs as depicted in \cref{fig:LTI-example_two-springs-one-mass}.

    \begin{figure}[h]\label{fig:LTI-example_two-springs-one-mass}
        \centering
        \includesag{mass-spring-lti}
        %\includegraphics[width=0.4\linewidth]{LTI-example_two-springs-one-mass}
    \end{figure}

    The coordinate $q(t)$ describes position of the mass along one horizontal dimension, $F$ denotes a force that is applied to $m$ in that horizontal direction, and $k_1$ and $k_2$ are the spring constants of the respective springs.

    The dynamics of the position coordinate $q$ as a function of time is described by the differential equation
    \begin{equation}\label{eq:LTI-exa-double-spring-equation}
        (k_1 + k_2)q + m \ddot q = F.
    \end{equation}

    We may rewrite this as a proper LTI system in a state-space representation by choosing the state-variable to be
    \begin{equation}\label{eq:LTI-exa-double-spring-state}
        \mat{x} =
        \begin{bmatrix}
            q(t) \\
            \dot{q}(t)
        \end{bmatrix}.
    \end{equation}
    Then
    \begin{equation}\label{eq:LTI-exa-double-spring-dynamics}
        \dot{\mat{x}} =
        \begin{bmatrix}
            0                       & 1 \\
            \tfrac{-(k_1 + k_2)}{m} & 0
        \end{bmatrix}
        \mat{x}(t) +
        \begin{bmatrix}
            0 \\
            1
        \end{bmatrix}
        F
    \end{equation}
    describes the dynamics, and as output we might choose
    \begin{equation}\label{eq:LTI-exa-double-spring-readout}
        \mat{y} =
        \begin{bmatrix}
            1 & 0
        \end{bmatrix}
        \mat{x}(t) +
        \begin{bmatrix}
            0
        \end{bmatrix}
        F.
    \end{equation}
\end{example}

\todotext{J: the above example is Example 4.3.1 from the Polderman Willems books ``Introduction to mathematical systems theory''}

\devel{

    \begin{example}
        [INSERT EXAMPLE 4.3.2 from Polderman Willems]
    \end{example}

    \begin{example}
        [INSERT EXAMPLE regarding predatory prey models, take the one from Jaz Meyers book]
    \end{example}

}

\subsection{Equivalent LTI systems}

\begin{definition}[Equivalence of LTI systems]
    \label{def:equivalence_lti}
    \SYNDEF{equivalence of LTI systems}
    Two systems~$\genericlti{\mora}$ and~$\genericlti{\morb}$ are \emph{equivalent} if and only if there exists an invertible linear transformation~$\mat{x}_\morb(t)=\mat{T}\mat{x}_\mora(t)$ such that
    \begin{equation}
        \mat{A}_\morb = \mat{T}\mat{A}_\mora \mat{T}^{-1}, \ \mat{B}_\morb = \mat{T}\mat{B}_\mora, \ \mat{C}_\morb =\mat{C}_\mora\mat{T}^{-1},\ \mat{D}_\morb=\mat{D}_\mora, \ \prstart_\morb=\mat{T}\prstart_\mora.
    \end{equation}
    $\mat{T}$ is called an \emph{equivalence transformation}.
\end{definition}

We think of equivalent LTI-systems as different ways of specifying what is essentially ``the same system''.
What is different in each specification is only different by a change of coordinates.

\subsection{Category of LTI systems}

We define a category of LTI systems \LTI.

\begin{ctdefinition}[Category \LTI]
    \SYNDEF{category of finite-dimensional linear time-invariant systems}
    \label{def:LTICat}
    The category \LTI of LTI systems is defined by:
    \begin{enumerate}
        \item \emph{Objects}: natural numbers.
        \item \emph{Morphisms}: A morphism in \LTI from~$\styleobj{l}\setin \natnumbers$ to~$\styleobj{m}\setin \natnumbers$ is a continuous time LTI system~$\genericlti{}$.
              %where~$\mat{A}\setin \reals^{\ntimesn}$,~$\mat{B}\setin \reals^{n\times l}$,~$\mat{C}\setin \reals^{m\times n}$,~$\mat{D}\setin \reals^{m\times l}$.
              Note that the dimensions of the input/output are implicit.
        \item \emph{Composition}: Given morphisms~$\mora \colon \styleobj{a}\mto \styleobj{b}$ and~$\morb\colon \styleobj{b}\mto \styleobj{c}$, described by the LTI systems
              \begin{equation}
                  \begin{aligned}
                       & \genericlti{\mora} \\
                       & \genericlti{\morb},
                  \end{aligned}
              \end{equation}
              their composition~$(\morab)\colon \styleobj{a}\mto \styleobj{c}$ is the LTI system~$\genericlti{}$, where
              \begin{equation}
                  \begin{aligned}
                      \prstart & =\begin{bmatrix}
                                      \prstart_\mora \\
                                      \prstart_\morb \\
                                  \end{bmatrix},\quad
                      \mat{A}=\begin{bmatrix}
                                  \mat{A}_\mora              & \mat{0}       \\
                                  \mat{B}_\morb\mat{C}_\mora & \mat{A}_\morb
                              \end{bmatrix},\quad
                      \mat{B}=\begin{bmatrix}
                                  \mat{B}_\mora \\
                                  \mat{B}_\morb\mat{D}_\mora
                              \end{bmatrix}, \\
                      \mat{C}  & =\begin{bmatrix}
                                      \mat{D}_\morb\mat{C}_\mora & \mat{C}_\morb
                                  \end{bmatrix}, \quad
                      \mat{D}=\mat{D}_\morb\mat{D}_\mora.
                  \end{aligned}
              \end{equation}
        \item \emph{Identities}: the identity for~$\styleobj{l}\setin \natnumbers$ is the system~$\tup{\mat{0}^{\matdim{0}{1}},\mat{0}^{\matdim{0}{0}},\mat{0}^{\matdim{0}{l}},\mat{0}^{\matdim{m}{0}},\mathbb{I}^{\matdim{m}{l}}}$.
    \end{enumerate}
\end{ctdefinition}

\begin{remark}
    Again, all of this works with matrices with zero rows and/or zero columns.
    Let's see practically how via a simple instance.
    Consider the system $\mora\colon a \mto b$ given by $\genericlti{\mora}$ and the system $\morb\colon b\mto c$ given by $\tupp{\mat{0}^{0\times 1},\mat{0}^{0\times 0}, \mat{0}^{0\times b}, \mat{0}^{c\times 0}, \mat{0}^{c\times b}}$.
    Their composition $\mora\mthen \morb\colon a\mto c$ is a system $\genericlti{}$ with:
    \begin{equation*}
        \prstart=\begin{bmatrix} \prstart_\mora \\ \mat{0}^{0\times 1}\end{bmatrix}=\prstart_\mora.
    \end{equation*}
    How could we write the above equation?
    $\prstart_\mora$ has $s$ rows and 1 column, and $\mat{0}^{0\times 1}$ has 0 rows and 1 column (it has to have 1 column for us to be able to write the above block matrix), allowing us to write the expression as $\prstart_\mora$.
    We can now write:
    \begin{equation*}
        \mat{A}=\begin{bmatrix}
            \mat{A}_\mora                    & \mat{0}^{s\times 0} \\
            \mat{0}^{0\times b}\mat{C}_\mora & \mat{0}^{0\times 0}
        \end{bmatrix}=
        \begin{bmatrix}
            \mat{A}_\mora       & \mat{0}^{s\times 0} \\
            \mat{0}^{0\times s} & \mat{0}^{0\times 0}
        \end{bmatrix}=\begin{bmatrix}
            \mat{A}_\mora \\
            \mat{0}^{0\times s}\end{bmatrix}=\mat{A}_\mora,
    \end{equation*}
    The above equality teaches us how to manipulate matrices with zero rows and/or zero columns.
    First, since $\mat{C}_\mora$ has $b$ rows and $s$ columns, the multiplication $\mat{0}^{0\times b}\mat{C}_\mora$ is well defined and gives the zero matrix with 0 rows and $s$ columns.
    Once we realize this, we see that we end up with a matrix made of four block matrices.
    Given their zero rows/columns, one can then simplify as shown.
    Similar arguments can be made for the other matrices resulting from the composition.
\end{remark}

\begin{exercise}
    Prove that \LTI is indeed a category.
\end{exercise}
\begin{solution}
    We start with associativity.
    Consider morphism $\mora \colon \styleobj{l}\mto \styleobj{m}$ given by~$\genericlti{\mora}$, and $\morb\colon \styleobj{m}\mto \styleobj{n}$ given by~$\genericlti{\morb}$, and $\morc\colon \styleobj{n}\mto \styleobj{o}$ given by~$\genericlti{\morc}$.
    The morphism~$\morab$ is described by the LTI
    \begin{equation}
        \genericlti{\mora,\morb},
    \end{equation}
    where
    \begin{widepar}
        \begin{equation}
            \prstart_{\mora,\morb}=\begin{bmatrix}
                \prstart_\mora \\
                \prstart_\morb
            \end{bmatrix},\quad
            \mat{A}_{\mora,\morb}=\begin{bmatrix}
                \mat{A}_\mora              & \mat{0}       \\
                \mat{B}_\morb\mat{C}_\mora & \mat{A}_\morb
            \end{bmatrix},\
            \mat{B}_{1,2}=\begin{bmatrix}
                \mat{B}_\mora \\
                \mat{B}_\morb\mat{D}_\mora
            \end{bmatrix},\
            \mat{C}_{\mora,\morb}=\begin{bmatrix}
                \mat{D}_\morb\mat{C}_\mora & \mat{C}_\morb
            \end{bmatrix}, \
            \mat{D}_{\mora,\morb}=\mat{D}_\morb\mat{D}_\mora.
        \end{equation}
    \end{widepar}
    The morphism~$(\morab)\mthen \morc$ is described by
    \begin{equation}
        \genericlti{(\mora,\morb),\morc},
    \end{equation}
    where:
    \begin{widepar}
        \begin{equation}
            \label{eq:assoc_lti_1}
            \begin{aligned}
                \prstart_{(\mora,\morb),\morc} & =\begin{bmatrix}
                                                      \prstart_\mora \\
                                                      \prstart_\morb \\
                                                      \prstart_\morc
                                                  \end{bmatrix},\quad
                \mat{A}_{(\mora,\morb),\morc}=\begin{bmatrix}
                                                  \mat{A}_\mora                           & \mat{0}                    & \mat{0}       \\
                                                  \mat{B}_\morb\mat{C}_\mora              & \mat{A}_\morb              & \mat{0}       \\
                                                  \mat{B}_\morc\mat{D}_\morb\mat{C}_\mora & \mat{B}_\morc\mat{C}_\morb & \mat{A}_\morc
                                              \end{bmatrix},\
                \mat{B}_{(\mora,\morb),\morc}=\begin{bmatrix}
                                                  \mat{B}_\mora              \\
                                                  \mat{B}_\morb\mat{D}_\mora \\
                                                  \mat{B}_\morc\mat{D}_\morb\mat{D}_\mora
                                              \end{bmatrix}, \\
                \mat{C}_{(\mora,\morb),\morc}  & =\begin{bmatrix}
                                                      \mat{D}_\morc\mat{D}_\morb\mat{C}_\mora & \mat{D}_\morc\mat{C}_\morb & \mat{C}_\morc
                                                  \end{bmatrix}, \
                \mat{D}_{(\mora,\morb),\morc}=\mat{D}_\morc\mat{D}_\morb\mat{D}_\mora.
            \end{aligned}
        \end{equation}
    \end{widepar}
    On the other hand, the morphism~$\morb\mthen \morc$ is described by
    \begin{equation}
        \genericlti{\morb,\morc},
    \end{equation}
    where:
    \begin{widepar}
        \begin{equation}
            \prstart{\morb,\morc}=\begin{bmatrix}
                \prstart_\morb \\
                \prstart_\morc
            \end{bmatrix},\
            \mat{A}_{\morb,\morc}=\begin{bmatrix}
                \mat{A}_\morb              & \mat{0}       \\
                \mat{B}_\morc\mat{C}_\morb & \mat{A}_\morc
            \end{bmatrix},\
            \mat{B}_{\morb,\morc}=\begin{bmatrix}
                \mat{B}_\morb \\
                \mat{B}_\morc\mat{D}_\morb
            \end{bmatrix},\
            \mat{C}_{\morb,\morc}=\begin{bmatrix}
                \mat{D}_\morc\mat{C}_\morb & \mat{C}_\morc
            \end{bmatrix}, \
            \mat{D}_{\morb,\morc}=\mat{D}_\morc\mat{D}_\morb.
        \end{equation}
    \end{widepar}
    Furthermore, the morphism~$\mora\mthen (\morb\mthen \morc)$ is described by
    \begin{equation}
        \genericlti{\mora,(\morb,\morc)},
    \end{equation}
    where:
    \begin{widepar}
        \begin{equation}
            \begin{aligned}
                \label{eq:assoc_lti_2}
                \prstart_{(\mora,\morb),\morc} & =\begin{bmatrix}
                                                      \prstart_\mora \\
                                                      \prstart_\morb \\
                                                      \prstart_\morc
                                                  \end{bmatrix},\quad
                \mat{A}_{\mora,(\morb,\morc)}=\begin{bmatrix}
                                                  \mat{A}_\mora                           & \mat{0}                    & \mat{0}       \\
                                                  \mat{B}_\morb\mat{C}_\mora              & \mat{A}_\morb              & \mat{0}       \\
                                                  \mat{B}_\morc\mat{D}_\morb\mat{C}_\mora & \mat{B}_\morc\mat{C}_\morb & \mat{A}_\morc
                                              \end{bmatrix},\
                \mat{B}_{\mora,(\morb,\morc)}=\begin{bmatrix}
                                                  \mat{B}_\mora              \\
                                                  \mat{B}_\morb\mat{D}_\mora \\
                                                  \mat{B}_\morc\mat{D}_\morb\mat{D}_\mora
                                              \end{bmatrix}, \\
                \mat{C}_{\mora,(\morb,\morc)}  & =\begin{bmatrix}
                                                      \mat{D}_\morc\mat{D}_\morb\mat{C}_\mora & \mat{D}_\morc\mat{C}_\morb & \mat{C}_\morc
                                                  \end{bmatrix}, \
                \mat{D}_{\mora,(\morb,\morc)}=\mat{D}_\morc\mat{D}_\morb\mat{D}_\mora.
            \end{aligned}
        \end{equation}
    \end{widepar}
    Clearly, the matrices in \cref{eq:assoc_lti_1} and \cref{eq:assoc_lti_2} coincide, showing associativity.

    We now show unitality.
    Consider a morphism~$\mora\colon \styleobj{l}\mto \styleobj{m}$, described by~$\genericlti{}$.
    The morphism~$\catidat{\styleobj{l}} \mthen \mora$ is a morphism~$\styleobj{l}\mto \styleobj{m}$ still given by~$\genericlti{}$.
    Similarly, the morphism~$\mora\mthen \catidat{\styleobj{m}}$ is a morphism~$\styleobj{l}\mto \styleobj{m}$, given by~$\genericlti{}$.
    Therefore, \LTI is a category.
\end{solution}

\begin{example}
    Consider the LTI for the spring-mass system from \cref{exa:sping-mass-lti}, and define the LTI
    \begin{equation*}
        \begin{aligned}
            \dot{\mat{z}}(t) & =p(t)+Cq(t) \\
            \mat{w}(t)       & =p(t),
        \end{aligned}
    \end{equation*}
    taking as input the output produced by the spring-mass system (the position of the mass along the horizontal dimension) and transforms it by a factor $C$.
    We can compose the two systems, obtaining the system $\genericplti{}$ with
    \begin{equation*}
        \mat{A}=\begin{bmatrix}0                  & 1 & 0 \\
               -\frac{k_1+k_2}{m} & 0 & 0 \\
               C                  & 0 & 1\end{bmatrix},\quad
        \mat{B}=\begin{bmatrix}
            0 \\
            1 \\
            0
        \end{bmatrix},
        \quad
        \mat{C}=\begin{bmatrix}
            0 & 0 & 1
        \end{bmatrix}.
    \end{equation*}
    This can be visualized more intuitively with the explicit composed dynamics:
    \begin{equation*}
        \begin{aligned}
            \begin{bmatrix}
                \dot{q}(t)  \\
                \ddot{q}(t) \\
                \dot{p}(t)
            \end{bmatrix} & =
            \begin{bmatrix}0                  & 1 & 0 \\
               -\frac{k_1+k_2}{m} & 0 & 0 \\
               C                  & 0 & 1\end{bmatrix} \begin{bmatrix}
                                                                            {q}(t)     \\
                                                                            \dot{q}(t) \\
                                                                            {p}(t)
                                                                        \end{bmatrix}+
            \begin{bmatrix}
                0 \\
                1 \\
                0
            \end{bmatrix} F \\
            w(t)                            & =\begin{bmatrix}
                                                   0 & 0 & 1
                                               \end{bmatrix}\begin{bmatrix}
                                                                {q}(t)     \\
                                                                \dot{q}(t) \\
                                                                {p}(t)
                                                            \end{bmatrix}
        \end{aligned}
    \end{equation*}
    Given an input force, I will get a transformed horizontal position as an output.
\end{example}

\todotext{J: @J: do an example here of LTI composition -- perhaps predatory prey example would work? hmm maybe not actually, because i think it involves trace or re-wirings.
    Other examples? Perhaps look at the Guzzella book for examples.
}

\subsection{Standard action of LTI systems}

\begin{definition}[LTI standard action]
    \label{def:lti_cat_action}
    We define a standard action of \LTI via:
    \begin{itemize}
        \item A map
              \begin{equation}\label{eq:LTI-action-objects}
                  \defmapperiod{
                      \funobspace{\act}
                  }{
                      \Obof\LTI
                  }{
                      \to
                  }{
                      \Obof\Set
                  }{
                      \styleobj{n}
                  }{
                      \diffun(\nonNegReals, \reals^\styleobj{n})
                  }
              \end{equation}
        \item A map
              \begin{equation}\label{eq:LTI-action-morphisms}
                  \funmorspace{\act} \colon \HomSet{\LTI}{\styleobj{m}}{\styleobj{n}}
                  \to
                  \HomSet{\Set}{\diffun(\nonNegReals, \reals^\styleobj{m})}{\diffun(\nonNegReals, \reals^\styleobj{n})}
              \end{equation}
              where~$\funmorspace{\act}$ takes an LTI system~$\mora\colon \styleobj{m} \to \styleobj{n}$ given by
              \begin{equation}
                  \genericlti{\mora},
              \end{equation}
              and returns the function
              \begin{equation}\label{eq:LTI-action-morphisms-defined}
                  \defmapcomma{\funmorspace{\act}(\mora)}
                  {\diffun(\nonNegReals, \reals^\styleobj{m})}
                  {\to}
                  {\diffun(\nonNegReals, \reals^\styleobj{n})}
                  {\mat{u}(t)}
                  {\mat{C}_\mora\mat{s}_\mora(t) + \mat{D}_\mora\mat{u}(t) }
              \end{equation}
              where~$\mat{s}_\mora$ is the unique solution of the initial value problem
              \begin{align}\label{eq:LTI-action-morphisms-cauchy-problem}
                  \begin{cases}
                      \dot{\mat{x}}(t) & = \mat{A}_\mora \mat{x}(t) + \mat{B}_\mora\mat{u}(t) \\
                      \mat{x}(0)       & = \prstart_\mora.
                  \end{cases}
              \end{align}
    \end{itemize}
\end{definition}

\todotextjira{754}{\alphubel: @Gioele:  we need to check and remark on the special cases involving zero-dimensional spaces}

\begin{remark}
    The initial value problem
    \begin{align}\label{eq:LTI-Cauchy-problem}
        \begin{cases}
            \dot{\mat{x}}(t) & = \mat{A} \mat{x}(t) + \mat{B} \mat{u}(t) \\
            \mat{x}(0)       & = \prstart
        \end{cases}
    \end{align}
    is a system of linear first-order non-homogenous differential equations.
    It does \emph{not} have constant coefficients, because the inhomogenous term $\mat{B}_\mora\mat{u}(t)$ is not constant (it depends on the independent variable $t$).

    The theorem of Picard-Lindelöf guarantees that initial value problems of the form \cref{eq:LTI-Cauchy-problem} always have a unique solution.
    There is general formula for the solution $\mat{s}(t)$ of \cref{eq:LTI-Cauchy-problem}, namely
    \begin{equation}
        \mat{s}(t) = e^{\mat{A} t} \prstart + \int_{0}^{t} e^{\mat{A} (t - \tau)}\mat{B}\mat{u}(s)d\tau.
    \end{equation}

\end{remark}

\begin{lemma}\label{lem:lti_cat_action}
    \cref{def:lti_cat_action} indeed defines a category action.
\end{lemma}
\begin{proof}
    We need to prove that
    \begin{equation}
        \funmorspace{\act}(\mora \mthenof\LTI \morb)
        =
        \funmorspace{\act}(\mora) \mthenof\Set \funmorspace{\act}(\morb).
    \end{equation}
    Consider~$\mora \colon \styleobj{m}\mto \styleobj{n}$ and~$\morb\colon \styleobj{n}\mto \styleobj{o}$.
    We first look at the map
    \begin{equation}
        \defmapcomma{\funmorspace{\act}(\mora \mthenof \LTI \morb)}
        {\diffun(\nonNegReals,\reals^\styleobj{m})}
        {\mto}
        {\diffun(\nonNegReals,\reals^\styleobj{o})}
        {\mat{u}(t)}
        {\mat{C}_{\morab}\mat{s}_{\morab}(t)+\mat{D}_{\morab}\mat{u}(t)}
    \end{equation}
    where~$\mat{s}_{\morab}$ is the unique solution of the initial value problem
    \begin{equation}
        \label{eq:lti_action_1}
        \begin{aligned}
            \dot{\mat{x}}(t) & =\mat{A}_{\morab}\mat{x}(t)+\mat{B}_{\morab}\mat{u}(t), \\
            \mat{x}(0)       & = \prstart_{\morab},
        \end{aligned}
    \end{equation}
    and
    \begin{equation}
        \mat{C}_{\morab}=\begin{bmatrix}
            \mat{D}_\morb \mat{C}_\mora & \mat{C}_\morb
        \end{bmatrix}, \quad \mat{D}=\mat{D}_\morb \mat{D}_\mora.
    \end{equation}
    From the definition of composition of LTI systems (\cref{def:LTICat}), we know that we can expand \cref{eq:lti_action_1} into
    \begin{equation}
        \label{eq:lti_action_12}
        \begin{aligned}
            \dot{\mat{x}}(t)=\begin{bmatrix}
                                 \dot{\mat{x}}_\mora(t) \\
                                 \dot{\mat{x}}_\morb(t)
                             \end{bmatrix} & =\begin{bmatrix}
                                                  \mat{A}_\mora              & \mat{0}       \\
                                                  \mat{B}_\morb\mat{C}_\mora & \mat{A}_\morb
                                              \end{bmatrix}
            %
            \begin{bmatrix}
                \mat{x}_\mora(t) \\
                \mat{x}_\morb(t)
            \end{bmatrix}%
            +%
            \begin{bmatrix}
                \mat{B}_\mora \\
                \mat{B}_\morb\mat{D}_\mora
            \end{bmatrix}\mat{u}(t) \\
            %\mat{y}_{\mora \mthen \morb}(t)           & =\begin{bmatrix}
            %                                                 \mat{D}_\morb\mat{C}_\mora & \mat{C}_\morb
            %                                             \end{bmatrix} \begin{bmatrix}
            %                                                               \mat{x}_\mora(t) \\
            %                                                               \mat{x}_\morb(t)
            %                                                           \end{bmatrix}+ \mat{D}_\morb\mat{D}_\mora\mat{u}_{\mora \mthen \morb}(t) \\
            \prstart                                  & =
            %
            \begin{bmatrix}
                \prstart_\mora \\
                \prstart_\morb
            \end{bmatrix}.
        \end{aligned}
    \end{equation}
    By instead looking at~$\funmorspace{\act}(\mora)$ we have
    \begin{equation}
        \defmapcomma{\funmorspace{\act}(\mora)}
        {\diffun(\nonNegReals,\reals^\styleobj{m})}
        {\mto}
        {\diffun(\nonNegReals,\reals^\styleobj{n})}
        {\mat{u}}
        {\mat{C}_\mora \mat{s}_\mora + \mat{D}_\mora \mat{u}}
    \end{equation}
    where~$\mat{s}_\mora$ is the unique solution of the initial value problem
    \begin{equation}
        \begin{aligned}
            \dot{\mat{x}}(t) & =\mat{A}_{\mora}\mat{x}(t)+\mat{B}_{\mora}\mat{u}(t), \\
            \mat{x}(0)       & = \prstart_{\mora},
        \end{aligned}
    \end{equation}
    and by looking at~$\funmorspace{\act}(\morb)$ we have
    \begin{equation}
        \label{eq:lti_action_2}
        \defmapcomma{
            \funmorspace{\act}(\morb)
        }{
            \diffun(\nonNegReals,\reals^\styleobj{n})
        }{
            \mto
        }{
            \diffun(\nonNegReals,\reals^\styleobj{o})
        }{
            \mat{u}}{\mat{C}_\morb \mat{s}_\morb+\mat{D}_\morb \mat{u}
        }
    \end{equation}
    where~$\mat{s}_\morb$ is the unique solution of the initial value problem
    \begin{equation}
        \begin{aligned}
            \dot{\mat{x}}(t) & =\mat{A}_{\morb}\mat{x}(t)+\mat{B}_{\morb}\mat{u}(t), \\
            \mat{x}(0)       & = \prstart_{\morb}.
        \end{aligned}
    \end{equation}
    Clearly, considering~$\funmorspace{\act}(\mora)\mthen \funmorspace{\act}(\morb)$ is equivalent to considering~$\mat{u}_\morb=\funmorspace{\act}(\mora)(\mat{u}_\mora)$.
    By substitution into \cref{eq:lti_action_2} we obtain
    \begin{equation}
        \defmapcomma{
            \funmorspace{\act}(\mora) \mthen \funmorspace{\act}(\morb)
        }{
            \diffun(\nonNegReals,\reals^\styleobj{m})
        }{
            \mto
        }{
            \diffun(\nonNegReals,\reals^\styleobj{o})
        }{
            \mat{u}
        }{
            \begin{bmatrix}
                \mat{D}_\morb \mat{C}_\mora & \mat{C}_\morb
            \end{bmatrix}\begin{bmatrix}
                \mat{s}_\mora \\ \mat{s}_\morb
            \end{bmatrix}+ \mat{D}_\morb \mat{D}_\mora \mat{u}
        }
    \end{equation}
    proving the statement.
\end{proof}

\begin{lemma}
    \label{lem:equivsystequivact}
    Two equivalent systems have the same LTI category action.
\end{lemma}
\begin{proof}
    Consider two equivalent LTI systems
    \begin{equation}
        \genericlti{\mora},\quad \genericlti{\morb}.
    \end{equation}
    The initial value problem~$\genericlti{\mora}$ poses reads:
    \begin{equation}
        \label{eq:first_lti_equiv}
        \begin{aligned}
            \dot{\mat{x}}(t) & =\mat{A}_{\mora}\mat{x}(t)+\mat{B}_{\mora}\mat{u}(t), \\
            \mat{x}(0)       & = \prstart_\mora.
        \end{aligned}
    \end{equation}
    Let~$\mat{s}(t)$ be the solution of \cref{eq:first_lti_equiv}.
    Consider the equivalence transformation~$\mat{r}(t)\linebreak[0]=\mat{T}\mat{s}(t)$.
    Then, we have
    \begin{equation}
        \begin{aligned}
            \dot{\mat{r}}(t) & =\mat{T}\dot{\mat{s}}(t) \\
                             & =\mat{T}\mat{A}_\mora \mat{s}(t)+\mat{T}\mat{B}_\mora \mat{u}(t) \\
                             & =\mat{T}\mat{A}_\mora \mat{T}^{-1}\mat{r}(t)+\mat{T}\mat{B}_\mora \mat{u}(t) \\
                             & =\mat{A}_\morb\mat{r}(t)+\mat{B}_\morb \mat{u}(t),
        \end{aligned}
    \end{equation}
    and~$ \prstart_\morb=\mat{T} \prstart_\mora$.
    Therefore, the action of the system~$\genericlti{\morb}$ is:
    \begin{equation}
        \begin{aligned}
            \funmorspace{\act}(\morb)(\mat{u}(t)) & =\mat{C}_\morb \mat{r}(t)+ \mat{D}_\morb \mat{u}(t) \\
                                                  & =\mat{C}_\mora \mat{T}^{-1}\mat{T}\mat{s}+\mat{D}_\mora \mat{u}(t) \\
                                                  & =\mat{C}_\mora \mat{s}(t)+\mat{D}_\mora \mat{u}(t) \\
                                                  & =\funmorspace{\act}(\mora)(\mat{u}(t)).
        \end{aligned}
    \end{equation}
\end{proof}

\begin{remark}
    Two LTI systems with the same LTI category action are not necessarily equivalent.
\end{remark}
\begin{proof}
    For a simple counterexample, consider the LTI system
    \begin{equation}
        \begin{aligned}
            \dot{\mat{x}}(t) & =\begin{bmatrix}
                                    1 & 0 \\ 0& 1
                                \end{bmatrix} \mat{x}(t)+\begin{bmatrix}
                                                             1 \\ 0
                                                         \end{bmatrix} \mat{u}(t) \\
            \mat{y}(t)       & =\begin{bmatrix}
                                    1 & 0
                                \end{bmatrix} \mat{x}(t)+\mat{u}(t).
        \end{aligned}
    \end{equation}
    The LTI category action of this system will be the same as the one of any system
    \begin{equation}
        \begin{aligned}
            \dot{\mat{x}}(t) & =\begin{bmatrix}
                                    1 & 0 \\ 0& \alpha
                                \end{bmatrix} \mat{x}(t)+\begin{bmatrix}
                                                             1 \\ 0
                                                         \end{bmatrix} \mat{u}(t) \\
            \mat{y}(t)       & =\begin{bmatrix}
                                    1 & 0
                                \end{bmatrix} \mat{x}(t)+\mat{u}(t),
        \end{aligned}
    \end{equation}
    where~$\alpha \setin \reals$, because the output~$\mat{y}$ generation does not care about the second component of~$\mat{x}$.
    However, there is no linear, invertible transformation which relates these systems.
    Note that control theory has developed tools to deal with hidden parts of the dynamics.
    Examples are the notion of controllability (ability of an actuator/input to control all states of a system) and observability (ability to estimate all states of the system).
\end{proof}

\devel{
    \subsection{Descriptor systems}
    \todo{J:Complete this part with subspaces.}

    Descriptor systems (DLTI systems) are a generalization of LTI systems which differ only in that the defining system of equations has the form
    \begin{align}
        \mat{E} \dot{\mat{x}}(t) & = \mat{A}\mat{x}(t) + \mat{B}\mat{u}(t) \label{eq:DLTI-system-dyn} \\
        \mat{y}(t)               & = \mat{C}\mat{x}(t) + \mat{D}\mat{u}(t) \label{eq:DLTI-system-ro}
    \end{align}
    where~$\mat{E}\colon \prst \to \prst$ is a linear map (again identifying the tangent space of~$\prst$ with~$\prst$).
    Clearly, a DLTI system is an LTI system when~$\mat{E}$ is the identity map.

    We associate the equation \cref{eq:DLTI-system-dyn} now with the relation
    \begin{equation}
        \prdyn \setsubseteq  (\prin \cartprod \prst) \cartprod \prst
    \end{equation}
    %given by
    %\begin{equation}
    %   \prdyn = \{ \tup{\tup{\mat{u}, \mat{x}} \mat{v}} \mid \mat{E} \mat{v} = \mat{A}\mat{x} + \mat{B}\mat{u} \}.
    %\end{equation}

}
