% !TEX root = chapter-standalone.tex

\section{Introduction to systems}
\todojira{624}{fill the whole part}



\section{Specification verses behavior}

\todotext{Talk about the distinction between formal specifications or parametrizations of systems versus behaviors of systems}



\section{System behaviors}


Let $\Time = \tup{\Time, \leq}$ be a totally ordered set (a model of time). Given a set $\Obja$, we will think here of the set $\Obja^\Time$ of functions $\Time \to \Obja$ as the set of time-dependent signals taking values in the signal space $\Obja$. 

When want to talk about discrete-time signals, natural choices for $\Time$ might be $\Time = \natnumbers$ or $\Time = \wnumbers$. When we want to talk about continuous time systems, we might choose $\Time = \reals$. And when we want to talk about event-based systems, we can choose $\Time$ to consist of a set of possible events, ordered by their sequence of occurence. 

The externally observable behavior of an input-output system is nothing else than a relation between input signals and output signals. 

\begin{definition}
Let $\Obja$ and $\Objb$ be sets. An \emph{input-output behavior}, with input signal space $\Obja$ and output signal space $\Objb$, is a relation 
\begin{equation}
\relA \subseteq \Obja^\Time \times \Objb^\Time. 
\end{equation}
\end{definition}

\begin{definition}
Fix a totally ordered set $\Time = \tup{\Time, \leq}$.
The category $\textbf{Bh}_\Time$ of input-output behaviors is the category where: 
\begin{enumerate}
\item Objects consist of all sets of the form $\Obja^\Time$, ranging over all sets $\Obja$. 
\item Morphisms from $\Obja^\Time$ to $\Objb^\Time$ are relations from $\Obja^\Time$ to $\Objb^\Time$: 
\begin{equation}
\HomSet{\textbf{Bh}_\Time}{\Obja^\Time}{\Objb^\Time} = \powerset (\Obja^\Time \times \Objb^\Time). 
\end{equation}
\item The identity morphism at an object $\Obja^\Time$ is the usual identity relation from $\Obja^\Time$ to itself. 
\item Composition is the usual composition of relations. 
\end{enumerate} 
\end{definition}

Sometimes we will wish to consider signal spaces $\Obja$ which are vector spaces over some field $\field$. Then the set $\Obja^\Time$ naturally inherits a $\field$-vector space structure from $\Obja$, and we may consider input-output behaviors 
\begin{equation*}
\relA \subseteq \Obja^\Time \oplus \Objb^\Time
\end{equation*}
which are $\field$-linear relations. 

\begin{definition}
Fix a totally ordered set $\Time = \tup{\Time, \leq}$.
The category $\textbf{LBh}_\Time$ of $\field$-linear input-output behaviors is the category where: 
\begin{enumerate}
\item Objects consist of all $\field$-vector spaces of the form $\Obja^\Time$, ranging over all $\field$-vector spaces $\Obja$. 
\item Morphisms from $\Obja^\Time$ to $\Objb^\Time$ are linear subspaces of  $\Obja^\Time \oplus \Objb^\Time$. 
\item The identity morphism at an object $\Obja^\Time$ is the usual identity relation from $\Obja^\Time$ to itself (this is always $\field$-linear). 
\item Composition is the usual composition of relations. 
\end{enumerate} 
\end{definition}
