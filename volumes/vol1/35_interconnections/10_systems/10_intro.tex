% !TEX root = chapter-standalone.tex

\section{Introduction to systems}
\devel{
    \todojira{624}{fill the whole part}

    \todotext{Maybe include several examples throughout this intro to liven up the text and make it more concrete and engineering flavored}

    The term ``system'' is not a precise mathematical term for us, but rather an intuitive concept that relates directly to modeling in engineering.
    We will think of systems as having input and output ports with which it can interact with other systems (or its broader environment), and that a system in some way establishes a relationship between input and output signals.
    This relationship might be a deterministic, causal relationship -- inputs determining outputs -- or it might be another form of lawfulness.

    \subsection{External and internal points of view}

    Sometimes we will think of systems as if they are ``black boxes": that is, as if we don't know anything about their inner workings, and all we care about is their external behavior in terms of which inputs are related to which outputs.
    Other times we will focus on the inner workings of systems and different ways to model these mathematically.

    \subsection{Composing systems}

    No matter whether we are looking `inside' of systems or are just considering their external behavior, a key feature of our understanding of the term system is that systems make be composed together.
    First of all, for serial composition, this means that the output of one system is taken as the input for the next systems.
    Furthermore, we will also formalize notions of parallel composition of systems, and of feedback.dede
    \section{Specifying systems}

    Let's consider some different examples of systems and ways of specifying them.

    \subsection{Moore machines}

    Recall Moore machines from \cref{sec:moore-semicategorically}.
    ..

    \subsection{More machines}

    Recall More machines from \cref{sec:more-machines}.
    ..
}
\subsection{LTI systems}
\todo{Note for matrices with 0 dimension}
We now define LTI systems.

\begin{definition}[LTI System]
    \label{def:LTI_syst}
    A \emph{linear time-invariant dynamical (LTI) system}, in a so-called state-space description, is specified by real vector spaces~$\prin=\reals^l$ (input space),~$\prout=\reals^m$ (output space), and~$\prst=\reals^n$ (state space), along with a system of equations of the form
    \begin{align}
        \dot{\mat{x}}(t) & = \mat{A}\mat{x}(t) + \mat{B}\mat{u}(t) \label{eq:LTI-system-dyn} \\
        \mat{y}(t)       & = \mat{C}\mat{x}(t) + \mat{D}\mat{u}(t) \label{eq:LTI-system-ro},
    \end{align}
    and an \emph{initial state}~$\mat{x}(0)\in \prst$, where~$t \setin \reals_{\geq 0}$,~$\mat{u}(t) \setin \prin$,~$\mat{y}(t) \setin \prout$,~$\mat{x}(t) \setin \prst$, and where~$\mat{A}\colon \prst \to \prst$,~$\mat{B}\colon \prin \to \prst$,~$\mat{C}\colon \prst \to \prout$,~$\mat{D}\colon \prin \to \prout$ are~$\reals$-linear maps (we are identifying the tangent space of~$\prst$ with~$\prst$ itself).
\end{definition}

We compactly refer to a LTI system by writing it as a tuple~$\genericlti{}$.

We will call an LTI system \emph{proper} if in \cref{eq:LTI-system-ro}, the map~$\mat{D}=\mat{0}^{p \times m}$.

\begin{definition}[Equivalence of LTI systems]
    \label{def:equivalence_lti}
    Two LTI systems~$\genericlti{\mora}$ and~$\genericlti{\morb}$ are \emph{equivalent} if and only if there exists an invertible linear transformation~$\mat{x}_\morb(t)=T\mat{x}_\mora(t)$ such that
    \begin{equation*}
        \mat{A}_\morb = \mat{T}\mat{A}_\mora \mat{T}^{-1}, \ \mat{B}_\morb = \mat{T}\mat{B}_\mora, \ \mat{C}_\morb =\mat{C}_\mora\mat{T}^{-1},\ \mat{D}_\morb=\mat{D}_\mora, \ \mat{x}_\morb(0)=\mat{T}\mat{x}_\mora(0).
    \end{equation*}
    $\mat{T}$ is called an \emph{equivalence transformation}.
\end{definition}

\subsection{Category of LTI systems}
We now define the category of LTI systems \LTI.

\begin{ctdefinition}[Category \LTI]
    \label{def:LTICat}
    The category \LTI, of LTI systems \iindex{\LTI}, is defined by:
    \begin{enumerate}
        \item \emph{Objects}: natural numbers.
        \item \emph{Morphisms}: A morphism in \LTI from~$\styleobj{l}\setin \natnumbers$ to~$\styleobj{m}\setin \natnumbers$ is a continuous time LTI system~$\genericlti{}$.
              %where~$\mat{A}\setin \reals^{n\times n}$,~$\mat{B}\setin \reals^{n\times l}$,~$\mat{C}\setin \reals^{m\times n}$,~$\mat{D}\setin \reals^{m\times l}$.
              Note that the dimensions of the input/output are implicit.
        \item \emph{Composition}: Given morphisms~$\mora \colon \styleobj{a}\mto \styleobj{b}$ and~$\morb\colon \styleobj{b}\mto \styleobj{c}$, described by the LTI systems
              \begin{equation*}
                  \begin{aligned}
                       & \genericlti{\mora} \\
                       & \genericlti{\morb},
                  \end{aligned}
              \end{equation*}
              their composition~$(\mora \mthen \morb)\colon \styleobj{a}\mto \styleobj{c}$ is the LTI system~$\genericlti{}$, where
              \begin{equation}
                  \begin{aligned}
                      \mat{x}(0) & =\begin{bmatrix}
                                        \mat{x}_\mora(0) \\
                                        \mat{x}_\morb(0) \\
                                    \end{bmatrix},\quad
                      \mat{A}=\begin{bmatrix}
                                  \mat{A}_\mora              & \mat{0}       \\
                                  \mat{B}_\morb\mat{C}_\mora & \mat{A}_\morb
                              \end{bmatrix},\quad
                      \mat{B}=\begin{bmatrix}
                                  \mat{B}_\mora \\
                                  \mat{B}_\morb\mat{D}_\mora
                              \end{bmatrix}, \\
                      \mat{C}    & =\begin{bmatrix}
                                        \mat{D}_\morb\mat{C}_\mora & \mat{C}_\morb
                                    \end{bmatrix}, \quad
                      \mat{D}=\mat{D}_\morb\mat{D}_\mora.
                  \end{aligned}
              \end{equation}
        \item \emph{Identities}: The identity morphism for~$\styleobj{l}\setin \natnumbers$ is the system~$\tup{\mat{0}^{0\times 1},\mat{0}^{0\times 0},\mat{0}^{0\times l},\mat{0}^{l\times 0},\mathbb{I}^{l\times l}}$.

    \end{enumerate}
\end{ctdefinition}

\begin{exercise}
    Prove that \LTI is indeed a category.
\end{exercise}
\begin{solution}
    We start with associativity.

    Consider morphisms~$\mora \colon \styleobj{l}\mto \styleobj{m}$ given by~$\genericlti{\mora}$,~$\morb\colon \styleobj{m}\mto \styleobj{n}$ given by~$\genericlti{\morb}$, and~$\morc\colon \styleobj{n}\mto \styleobj{o}$ given by~$\genericlti{\morc}$.

    The morphism~$\mora\mthen \morb$ is described by the LTI~$\genericlti{\mora,\morb}$, where
    \begin{widepar}
        \begin{equation}
            \mat{x}_{\mora,\morb}(0)=\begin{bmatrix}
                \mat{x}_\mora(0) \\
                \mat{x}_\morb(0)
            \end{bmatrix},\quad
            \mat{A}_{\mora,\morb}=\begin{bmatrix}
                \mat{A}_\mora              & \mat{0}       \\
                \mat{B}_\morb\mat{C}_\mora & \mat{A}_\morb
            \end{bmatrix},\
            \mat{B}_{1,2}=\begin{bmatrix}
                \mat{B}_\mora \\
                \mat{B}_\morb\mat{D}_\mora
            \end{bmatrix},\
            \mat{C}_{\mora,\morb}=\begin{bmatrix}
                \mat{D}_\morb\mat{C}_\mora & \mat{C}_\morb
            \end{bmatrix}, \
            \mat{D}_{\mora,\morb}=\mat{D}_\morb\mat{D}_\mora.
        \end{equation}
    \end{widepar}
    The morphism~$(\mora\mthen \morb)\mthen \morc$ is described by~$\genericlti{(\mora,\morb),\morc}$, where:
    \begin{widepar}
        \begin{equation}
            \label{eq:assoc_lti_1}
            \begin{aligned}
                \mat{x}(0)_{(\mora,\morb),\morc} & =\begin{bmatrix}
                                                        \mat{x}_\mora(0) \\
                                                        \mat{x}_\morb(0) \\
                                                        \mat{x}_\morc(0)
                                                    \end{bmatrix},\quad
                \mat{A}_{(\mora,\morb),\morc}=\begin{bmatrix}
                                                  \mat{A}_\mora                           & \mat{0}                    & \mat{0}       \\
                                                  \mat{B}_\morb\mat{C}_\mora              & \mat{A}_\morb              & \mat{0}       \\
                                                  \mat{B}_\morc\mat{D}_\morb\mat{C}_\mora & \mat{B}_\morc\mat{C}_\morb & \mat{A}_\morc
                                              \end{bmatrix},\
                \mat{B}_{(\mora,\morb),\morc}=\begin{bmatrix}
                                                  \mat{B}_\mora              \\
                                                  \mat{B}_\morb\mat{D}_\mora \\
                                                  \mat{B}_\morc\mat{D}_\morb\mat{D}_\mora
                                              \end{bmatrix}, \\
                \mat{C}_{(\mora,\morb),\morc}    & =\begin{bmatrix}
                                                        \mat{D}_\morc\mat{D}_\morb\mat{C}_\mora & \mat{D}_\morc\mat{C}_\morb & \mat{C}_\morc
                                                    \end{bmatrix}, \
                \mat{D}_{(\mora,\morb),\morc}=\mat{D}_\morc\mat{D}_\morb\mat{D}_\mora.
            \end{aligned}
        \end{equation}
    \end{widepar}
    On the other hand, the morphism~$\morb\mthen \morc$ is described by~$\genericlti{\morb,\morc}$, where:
    \begin{widepar}
        \begin{equation}
            \mat{x}_{\morb,\morc}(0)=\begin{bmatrix}
                \mat{x}_\morb(0) \\
                \mat{x}_\morc(0)
            \end{bmatrix},\
            \mat{A}_{\morb,\morc}=\begin{bmatrix}
                \mat{A}_\morb              & \mat{0}       \\
                \mat{B}_\morc\mat{C}_\morb & \mat{A}_\morc
            \end{bmatrix},\
            \mat{B}_{\morb,\morc}=\begin{bmatrix}
                \mat{B}_\morb \\
                \mat{B}_\morc\mat{D}_\morb
            \end{bmatrix},\
            \mat{C}_{\morb,\morc}=\begin{bmatrix}
                \mat{D}_\morc\mat{C}_\morb & \mat{C}_\morc
            \end{bmatrix}, \
            \mat{D}_{\morb,\morc}=\mat{D}_\morc\mat{D}_\morb.
        \end{equation}
    \end{widepar}
    Furthermore, the morphism~$\mora\mthen (\morb\mthen \morc)$ is described by~$\genericlti{\mora,(\morb,\morc)}$, where:
    \begin{widepar}
        \begin{equation}
            \begin{aligned}
                \label{eq:assoc_lti_2}
                \mat{x}(0)_{(\mora,\morb),\morc} & =\begin{bmatrix}
                                                        \mat{x}_\mora(0) \\
                                                        \mat{x}_\morb(0) \\
                                                        \mat{x}_\morc(0)
                                                    \end{bmatrix},\quad
                \mat{A}_{\mora,(\morb,\morc)}=\begin{bmatrix}
                                                  \mat{A}_\mora                           & \mat{0}                    & \mat{0}       \\
                                                  \mat{B}_\morb\mat{C}_\mora              & \mat{A}_\morb              & \mat{0}       \\
                                                  \mat{B}_\morc\mat{D}_\morb\mat{C}_\mora & \mat{B}_\morc\mat{C}_\morb & \mat{A}_\morc
                                              \end{bmatrix},\
                \mat{B}_{\mora,(\morb,\morc)}=\begin{bmatrix}
                                                  \mat{B}_\mora              \\
                                                  \mat{B}_\morb\mat{D}_\mora \\
                                                  \mat{B}_\morc\mat{D}_\morb\mat{D}_\mora
                                              \end{bmatrix}, \\
                \mat{C}_{\mora,(\morb,\morc)}    & =\begin{bmatrix}
                                                        \mat{D}_\morc\mat{D}_\morb\mat{C}_\mora & \mat{D}_\morc\mat{C}_\morb & \mat{C}_\morc
                                                    \end{bmatrix}, \
                \mat{D}_{\mora,(\morb,\morc)}=\mat{D}_\morc\mat{D}_\morb\mat{D}_\mora.
            \end{aligned}
        \end{equation}
    \end{widepar}
    Clearly, the matrices in \cref{eq:assoc_lti_1} and \cref{eq:assoc_lti_2} coincide, showing associativity.

    Let's now show unitality.
    Consider a morphism~$\mora\colon \styleobj{l}\mto \styleobj{m}$, described by~$\genericlti{}$.
    The morphism~$\catid_\styleobj{l} \mthen \mora$ is a morphism~$\styleobj{l}\mto \styleobj{m}$ still given by~$\genericlti{}$.
    Similarly, the morphism~$\mora\mthen \catid_\styleobj{m}$ is a morphism~$\styleobj{l}\mto \styleobj{m}$, given by~$\genericlti{}$.
    Therefore, \LTI is a category.
\end{solution}

\todotext{J: i guess the subsection below will eventually be moved to the section/chapter on stacking? }

\subsection{LTI is stacking}
When considering LTI systems, we can define stacking operations and show that it forms a stacking semicategory (\cref{def:simple-stacking-semi-cat}).
The stacking operation on objects is defined as
\begin{equation*}
    \defmapperiod{\mtimescatob}{\Ob_\LTI \cartprod \Ob_\LTI}{\to}{\Ob_\LTI}{\tup{\styleobj{l},\styleobj{m}}}{\styleobj{l}+\styleobj{m}}
\end{equation*}
On the other hand, the operation on morphisms ``stacks'' the LTI systems onto each other.
\footnote{For the control engineers out there: the resulting LTI system will have a Relative Gain Array (RGA) matrix corresponding to the identity matrix.}
Formally:
\begin{equation*}
    \defmapcomma{\mtimescatmor}{\Mor_\LTI \cartprod \Mor_\LTI}{\to}{\Mor_\LTI}{\tup{\genericlti{\mora},\genericlti{\morb}}}{\genericlti{}}
\end{equation*}
with
\begin{widepar}
    \begin{equation*}
        \mat{x}(0)=\begin{bmatrix}
            \mat{x}_\mora(0) \\
            \mat{x}_\morb(0)
        \end{bmatrix},\
        \mat{A}=\begin{bmatrix}
            \mat{A}_\mora & \mat{0}       \\
            \mat{0}       & \mat{A}_\morb
        \end{bmatrix}, \
        \mat{B}=\begin{bmatrix}
            \mat{B}_\mora & \mat{0}       \\
            \mat{0}       & \mat{B}_\morb
        \end{bmatrix},\
        \mat{C}=\begin{bmatrix}
            \mat{C}_\mora & \mat{0}       \\
            \mat{0}       & \mat{C}_\morb
        \end{bmatrix},\
        \mat{D}=\begin{bmatrix}
            \mat{D}_\mora & \mat{0}       \\
            \mat{0}       & \mat{D}_\morb
        \end{bmatrix}
    \end{equation*}
\end{widepar}

Clearly, the morphism given by~$\genericlti{}$ satisifes the compatibility condition for stacking semicategories:
if~$\genericlti{\mora}$ is~$\styleobj{l}\mto \styleobj{m}$ and~$\genericlti{\morb}$ is~$\styleobj{n}\mto \styleobj{o}$,~$\genericlti{}$ is~$\styleobj{l}+\styleobj{n}\mto \styleobj{m}+\styleobj{o}$.
Given that the resulting LTI system is described by diagonal matrices, the defined operations are associative, and therefore \LTI is an associative stacking semicategory.

\begin{example}
    We want to show that \LTI as defined is almost a functiorial stacking semicategory, but not quite.
    Given morphisms~$\mora\colon \styleobj{l}\mto \styleobj{m}$,~$\morc\colon \styleobj{m}\mto \styleobj{n}$,~$\morb\colon \styleobj{o}\mto \styleobj{p}$,~$\mord\colon \styleobj{p}\mto \styleobj{q}$, we would need to have
    \begin{equation*}
        (\mora \mthen \morc)\mtimescat (\morb\mthen \mord)=(\mora\mtimescat \morb)\mthen (\morc\mtimescat \mord).
    \end{equation*}
    This, however, is not true.
    Let's see this by looking at the first component of the LTI system.
    On one hand we have:
    \begin{equation*}
        \mat{A}_{(\mora \mthen \morc)\mtimescat (\morb\mthen \mord)}=
        \begin{bmatrix}
            \mat{A}_\mora              & \mat{0}       & \mat{0}                    & \mat{0}       \\
            \mat{B}_\morc\mat{C}_\mora & \mat{A}_\morc & \mat{0}                    & \mat{0}       \\
            \mat{0}                    & \mat{0}       & \mat{A}_\morb              & \mat{0}       \\
            \mat{0}                    & \mat{0}       & \mat{B}_\mord\mat{C}_\morb & \mat{A}_\mord
        \end{bmatrix}
    \end{equation*}
    On the other hand we have:
    \begin{equation*}
        \mat{A}_{(\mora\mtimescat \morb)\mthen (\morc\mtimescat \mord)}=
        \begin{bmatrix}
            \mat{A}_\mora               & \mat{0}                     & \mat{0}       & \mat{0}       \\
            \mat{0}                     & \mat{A}_\morb               & \mat{0}       & \mat{0}       \\
            \mat{B}_\morc \mat{C}_\mora & \mat{0}                     & \mat{A}_\morc & \mat{0}       \\
            \mat{0}                     & \mat{B}_\mord \mat{C}_\morb & \mat{0}       & \mat{A}_\mord
        \end{bmatrix}
    \end{equation*}
    These two are different, and will therefore describe different systems.
    However, the two matrices just differ by two permutations, which can be expressed via an invertible linear transformation~$\mat{T}$ as follows:
    \begin{equation*}
        \mat{A}_{(\mora\mtimescat \morb)\mthen (\morc\mtimescat \mord)}=
        \underbrace{\begin{bmatrix}
                1 & 0 & 0 & 0 \\
                0 & 0 & 1 & 0 \\
                0 & 1 & 0 & 0 \\
                0 & 0 & 0 & 1
            \end{bmatrix}}_{\mat{T}}\cdot
        \mat{A}_{(\mora \mthen \morc)\mtimescat (\morb\mthen \mord)}
        \cdot
        \begin{bmatrix}
            1 & 0 & 0 & 0 \\
            0 & 0 & 1 & 0 \\
            0 & 1 & 0 & 0 \\
            0 & 0 & 0 & 1
        \end{bmatrix}.
    \end{equation*}
    The other system matrices also obey \cref{def:equivalence_lti}.
\end{example}

\begin{definition}[LTI standard action]
    \label{def:lti_cat_action}
    We define a standard action of \LTI via:
    \begin{compactitem}
        \item A map
              \begin{equation*}
                  \defmapperiod{\catacOb_\LTI}{\Ob_\LTI}{\to}{\Ob_\Set}{\styleobj{n}}{\diffun(\nonNegReals, \reals^\styleobj{n})}
              \end{equation*}
        \item A map
              \begin{equation*}
                  \catacMor_\LTI \colon \HomSet{\LTI}{\styleobj{m}}{\styleobj{n}}
                  \to
                  \HomSet{\Set}{\diffun(\nonNegReals, \reals^\styleobj{m})}{\diffun(\nonNegReals, \reals^\styleobj{n})}
              \end{equation*}
              where~$\catacMor_\LTI$ takes a LTI system $\mora: \styleobj{m} \to \styleobj{n}$ given by
              \begin{equation*}
                  \genericlti{\mora},
              \end{equation*}
              and returns the function
              \begin{equation*}
                  \defmapcomma{\catacMor_\LTI(\mora)}
                  {\diffun(\nonNegReals, \reals^\styleobj{m})}
                  {\to}
                  {\diffun(\nonNegReals, \reals^\styleobj{n})}
                  {\mat{u}}
                  {\mat{C}_\mora\mat{s}_\mora(t) + \mat{D}_\mora\mat{u}(t) }
              \end{equation*}
              where~$\mat{s}_\mora$ is the unique solution of the initial value problem
              \begin{align*}
                  \dot{\mat{x}}(t) & = \mat{A}_\mora \mat{x}(t) + \mat{B}_\mora\mat{u}(t) \\
                  \mat{x}(0)       & =\mat{x}_\mora(0).
              \end{align*}
    \end{compactitem}
\end{definition}

\begin{lemma}
    \cref{def:lti_cat_action} indeed defines a category action.
\end{lemma}
\begin{proof}
    We need to prove that
    \begin{equation*}
        \catacMor_\LTI(\mora \mthen_\LTI \morb)=\catacMor_\LTI(\mora) \mthen_\Set \catacMor_\LTI(\morb).
    \end{equation*}
    Let's consider~$\mora \colon \styleobj{m}\mto \styleobj{n}$ and~$\morb\colon \styleobj{n}\mto \styleobj{o}$.
    We first look at the map
    \begin{equation*}
        \defmapcomma{\catacMor_\LTI(\mora \mthen_\LTI \morb)}
        {\diffun(\nonNegReals,\reals^\styleobj{m})}
        {\mto}
        {\diffun(\nonNegReals,\reals^\styleobj{o})}
        {\mat{u}}
        {\mat{C}_{\mora\mthen \morb}\mat{s}_{\mora\mthen \morb}(t)+\mat{D}_{\mora \mthen \morb}\mat{u}(t)}
    \end{equation*}
    where~$\mat{s}_{\mora \mthen \morb}$ is the unique solution of the initial value problem
    \begin{equation}
        \label{eq:lti_action_1}
        \begin{aligned}
            \dot{\mat{x}}(t) & =\mat{A}_{\mora \mthen \morb}\mat{x}(t)+\mat{B}_{\mora \mthen \morb}\mat{u}(t), \\
            \mat{x}(0)       & =\mat{x}_{\mora \mthen \morb}(0),
        \end{aligned}
    \end{equation}
    and
    \begin{equation*}
        \mat{C}_{\mora\mthen \morb}=\begin{bmatrix} \mat{D}_\morb \mat{C}_\mora & \mat{C}_\morb\end{bmatrix}, \quad \mat{D}=\mat{D}_\morb \mat{D}_\mora.
    \end{equation*}
    From the definition of composition of LTI systems (\cref{def:LTICat}), we know that one can expand \cref{eq:lti_action_1} into
    \begin{equation}
        \label{eq:lti_action_12}
        \begin{aligned}
            \dot{\mat{x}}(t)=\begin{bmatrix}
                                 \dot{\mat{x}}_\mora(t) \\
                                 \dot{\mat{x}}_\morb(t)
                             \end{bmatrix} & =\begin{bmatrix}
                                                  \mat{A}_\mora              & \mat{0}       \\
                                                  \mat{B}_\morb\mat{C}_\mora & \mat{A}_\morb
                                              \end{bmatrix} \begin{bmatrix}
                                                                \mat{x}_\mora(t) \\
                                                                \mat{x}_\morb(t)
                                                            \end{bmatrix}+\begin{bmatrix}
                                                                              \mat{B}_\mora \\
                                                                              \mat{B}_\morb\mat{D}_\mora
                                                                          \end{bmatrix}\mat{u}(t) \\
            %\mat{y}_{\mora \mthen \morb}(t)           & =\begin{bmatrix}
            %                                                 \mat{D}_\morb\mat{C}_\mora & \mat{C}_\morb
            %                                             \end{bmatrix} \begin{bmatrix}
            %                                                               \mat{x}_\mora(t) \\
            %                                                               \mat{x}_\morb(t)
            %                                                           \end{bmatrix}+ \mat{D}_\morb\mat{D}_\mora\mat{u}_{\mora \mthen \morb}(t) \\
            \mat{x}(0)                                & =
            \begin{bmatrix}
                \mat{x}_\mora(0) \\
                \mat{x}_\morb(0)\end{bmatrix}.
        \end{aligned}
    \end{equation}
    By instead looking at~$\catacMor_\LTI(\mora)$ one has
    \begin{equation*}
        \defmapcomma{\catacMor_\LTI(\mora)}
        {\diffun(\nonNegReals,\reals^\styleobj{m})}
        {\mto}
        {\diffun(\nonNegReals,\reals^\styleobj{n})}
        {\mat{u}}
        {\mat{C}_\mora \mat{s}_\mora+\mat{D}_\mora \mat{u}}
    \end{equation*}
    where~$\mat{s}_\mora$ is the unique solution of the initial value problem
    \begin{equation}
        \begin{aligned}
            \dot{\mat{x}}(t) & =\mat{A}_{\mora}\mat{x}(t)+\mat{B}_{\mora}\mat{u}(t), \\
            \mat{x}(0)       & =\mat{x}_{\mora}(0),
        \end{aligned}
    \end{equation}
    and by looking at~$\catacMor_\LTI(\morb)$ one has
    \begin{equation}
        \label{eq:lti_action_2}
        \defmapcomma{\catacMor_\LTI(\morb)}
        {\diffun(\nonNegReals,\reals^\styleobj{n})}
        {\mto}
        {\diffun(\nonNegReals,\reals^\styleobj{o})}
        {\mat{u}}{\mat{C}_\morb \mat{s}_\morb+\mat{D}_\morb \mat{u}}
    \end{equation}
    where~$\mat{s}_\morb$ is the unique solution of the initial value problem
    \begin{equation}
        \begin{aligned}
            \dot{\mat{x}}(t) & =\mat{A}_{\morb}\mat{x}(t)+\mat{B}_{\morb}\mat{u}(t), \\
            \mat{x}(0)       & =\mat{x}_{\morb}(0).
        \end{aligned}
    \end{equation}
    Clearly, considering~$\catacMor_\LTI(\mora)\mthen \catacMor_\LTI(\morb)$ is equivalent to considering~$\mat{u}_\morb=\catacMor_\LTI(\mora)(\mat{u}_\mora)$.
    By substitution into \cref{eq:lti_action_2} one obtains
    \begin{equation*}
        \defmapcomma{\catacMor_\LTI(\mora) \mthen \catacMor_\LTI(\morb)}
        {\diffun(\nonNegReals,\reals^\styleobj{m})}
        {\mto}
        {\diffun(\nonNegReals,\reals^\styleobj{o})}
        {\mat{u}}
        {\begin{bmatrix} \mat{D}_\morb \mat{C}_\mora & \mat{C}_\morb\end{bmatrix}\begin{bmatrix} \mat{s}_\mora \\ \mat{s}_\morb \end{bmatrix}+ \mat{D}_\morb \mat{D}_\mora \mat{u}}
    \end{equation*}

    \cref{eq:lti_action_12}, proving the statement.
\end{proof}

\begin{lemma}
    Two equivalent systems have the same LTI category action.
\end{lemma}
\begin{proof}
    Consider two equivalent LTI systems
    \begin{equation*}
        \genericlti{\mora},\quad \genericlti{\morb}.
    \end{equation*}
    The initial value problem~$\genericlti{\mora}$ poses reads:
    \begin{equation}
        \label{eq:first_lti_equiv}
        \begin{aligned}
            \dot{\mat{x}}(t) & =\mat{A}_{\mora}\mat{x}(t)+\mat{B}_{\mora}\mat{u}(t), \\
            \mat{x}(0)       & =\mat{x}_\mora(0).
        \end{aligned}
    \end{equation}
    Let~$\mat{s}(t)$ be the solution of \cref{eq:first_lti_equiv}.
    Consider the equivalence transformation~$\mat{r}(t)=\mat{T}\mat{s}(t)$.
    Then, one has
    \begin{equation*}
        \begin{aligned}
            \dot{\mat{r}}(t) & =\mat{T}\dot{\mat{s}}(t) \\
                             & =\mat{T}\mat{A}_\mora \mat{s}(t)+\mat{T}\mat{B}_\mora \mat{u}(t) \\
                             & =\mat{T}\mat{A}_\mora \mat{T}^{-1}\mat{r}(t)+\mat{T}\mat{B}_\mora \mat{u}(t) \\
                             & =\mat{A}_\morb\mat{r}(t)+\mat{B}_\morb \mat{u}(t),
        \end{aligned}
    \end{equation*}
    and~$\mat{r}(0)=\mat{T}\mat{s}(0)=\mat{T}\mat{x}_\mora$.
    Therefore, the action of the system~$\genericlti{\morb}$ is:
    \begin{equation*}
        \begin{aligned}
            \catacMor_\LTI(\morb)(\mat{u}(t)) & =\mat{C}_\morb \mat{r}(t)+ \mat{D}_\morb \mat{u}(t) \\
                                              & =\mat{C}_\mora \mat{T}^{-1}\mat{T}\mat{s}+\mat{D}_\mora \mat{u}(t) \\
                                              & =\mat{C}_\mora \mat{s}(t)+\mat{D}_\mora \mat{u}(t) \\
                                              & =\catacMor_\LTI(\mora)(\mat{u}(t)).
        \end{aligned}
    \end{equation*}
\end{proof}

\begin{remark}
    Two LTI systems with the same LTI category action are not necessarily equivalent.
\end{remark}
\begin{proof}
    For a simple counterexample, consider the LTI system
    \begin{equation*}
        \begin{aligned}
            \dot{\mat{x}}(t) & =\begin{bmatrix} 1&0\\ 0&1 \end{bmatrix} \mat{x}(t)+\begin{bmatrix} 1\\ 0\end{bmatrix} \mat{u}(t) \\
            \mat{y}(t)       & =\begin{bmatrix} 1&0\end{bmatrix} \mat{x}(t)+\mat{u}(t).
        \end{aligned}
    \end{equation*}
    The LTI category action of this system will be the same as the one of any system
    \begin{equation*}
        \begin{aligned}
            \dot{\mat{x}}(t) & =\begin{bmatrix} 1&0\\ 0&\alpha \end{bmatrix} \mat{x}(t)+\begin{bmatrix} 1\\ 0\end{bmatrix} \mat{u}(t) \\
            \mat{y}(t)       & =\begin{bmatrix} 1&0\end{bmatrix} \mat{x}(t)+\mat{u}(t),
        \end{aligned}
    \end{equation*}
    where~$\alpha \in \reals$, because the output~$\mat{y}$ generation does not care about the second component of~$\mat{x}$.
    However, there is no linear, invertible transformation which relates these systems.
\end{proof}

\devel{
    \subsection{Descriptor systems}
    \todo{J:Complete this part with subspaces.}

    Descriptor systems (DLTI systems) are a generalization of LTI systems which differ only in that the defining system of equations has the form
    \begin{align}
        \mat{E} \dot{\mat{x}}(t) & = \mat{A}\mat{x}(t) + \mat{B}\mat{u}(t) \label{eq:DLTI-system-dyn} \\
        \mat{y}(t)               & = \mat{C}\mat{x}(t) + \mat{D}\mat{u}(t) \label{eq:DLTI-system-ro}
    \end{align}
    where~$\mat{E}\colon \prst \to \prst$ is a linear map (again identifying the tangent space of~$\prst$ with~$\prst$).
    Clearly, a DLTI system is an LTI system when~$\mat{E}$ is the identity map.

    We associate the equation \cref{eq:DLTI-system-dyn} now with the relation
    \begin{equation*}
        \prdyn \subseteq  (\prin \cartprod \prst) \cartprod \prst
    \end{equation*}
    %given by
    %\begin{equation*}
    %   \prdyn = \{ \tup{ \tup{\mat{u}, \mat{x}} \mat{v}} \mid \mat{E} \mat{v} = \mat{A}\mat{x} + \mat{B}\mat{u} \}.
    %\end{equation*}

    \section{System behaviors}

    \todotext{Talk here briefly about the distinction between formal specifications or parametrizations of systems versus behaviors of systems; perhaps use Moore machines as an example}

    Consider a Moore machine $\tup{\prin, \prout, \prst, \prdyn, \prreadout}$.
    In \cref{sec:action-of-a-category} we saw how, given an initial state $x_0 \setin \prst$, a Moore machine may act on sequences of inputs, mapping these to sequences of outputs.

    For example, given a sequence of inputs $u_0, u_1, u_2, .
        .. $, we may map this to a sequence of outputs $y_0, y_1, y_2, ... $ using the following recursion relations
    \begin{align}
        x_{k+1} & = \prdyn(x_k, u_k) \label{eq:state-recursion-1} \\
        y_{k}   & = \prreadout(x_{k}), \label{eq:state-recursion-2}
    \end{align}
    with $k \setin \natnumbers$.

    In other words, given $x_0 \setin \prst$, we may map the Moore machine \label{eq:moore-again} to a function
    \begin{equation}
        \prin^\natnumbers \to \prout^\natnumbers,  \quad (u_k)_{k \setin \natnumbers} \mapsto (y_k)_{k \setin \natnumbers}.
    \end{equation}
    We will think of this function as an `external behavior' of the Moore machine, because it encodes what is externally observable to us in terms of how the Moore machine is used to relate inputs to outputs.

    Note that in concrete application we might not actually know a precise model of a Moore machine in the form {eq:moore-specification-again}.
    But we can still observe what it does when we feed it inputs.
    And in some cases this external behavior is in fact all we really care about.

    \subsection{Many behaviors from one machine}

    Other kinds of behavior may be generated with the same Moore machine $\tup{\prin, \prout, \prst, \prdyn, \prreadout}$ as above (and using the same initial state).
    For example, instead of \cref{eq:state-recursion-1} and \cref{eq:state-recursion-2} we may use the set of equations
    \begin{align}
        \tilde x_{k} & = \prdyn(x_k, u_k) \\
        x_{k+1}      & = \prdyn(\tilde x_{k}, u_k) \\
        y_{k}        & = \prreadout(x_{k}),
    \end{align}
    where $\tilde x$ is a new `latent' variable that we've introduced.
    This illustrates that the recursion equations are part of what generates a behavior of the Moore machine; they are not part of the specification $\tup{\prin, \prout, \prst, \prdyn, \prreadout}$ of the Moore machine itself.

    \subsection{Modeling behavior}

    We will now develop a general-purpose formalization of system behaviors.

    In the example above with Moore machines, we said its behavior was a function $\prin^\natnumbers \to \prout^\natnumbers$.
    Here the set~$\natnumbers$ played the role of time: each $k \setin \natnumbers$ corresponded to a time step.
    In the following, we allow as a model of time any totally ordered set $\Time = \tup{\Time, \leq}$.

    When we want to talk about discrete-time signals, natural choices for $\Time$ might be $\Time = \natnumbers$ or $\Time = \wnumbers$.
    Similarly, if we want to talk about continuous time systems, we might choose $\Time = \reals$.
    %And when we want to talk about event-based systems, we can choose $\Time$ to consist of a set of possible events, ordered by their sequence of occurence.

    Given a set $\Obja$, we consider the set $\Obja^\Time$ of functions $\Time \to \Obja$ as the set of time-dependent signals taking values in the set $\Obja$.
    Then a system behavior is, for example, a function $\styleobj{U}^\Time \to \styleobj{Y}^\Time$.
    For more flexibility in our formalism, we will also allow behaviors which are \emph{relations} $\styleobj{U}^\Time \to \styleobj{Y}^\Time$.
    In other words, in general, the externally observable behavior of a system is a relation between input signals and output signals.

    \subsection{Categories of behaviors}

    \begin{definition}
        Let~$\Obja$ and~$\Objb$ be sets.
        An \emph{input-output behavior}, with input space $\Obja$ and output space $\Objb$, is a relation
        \begin{equation*}
            \relA \subseteq \Obja^\Time \cartprod \Objb^\Time.
        \end{equation*}
    \end{definition}

    \begin{definition}
        Fix a totally ordered set~$\Time = \tup{\Time, \leq}$.
        The category~$\Beh_\Time$ of input-output behaviors is the category where:
        \begin{enumerate}
            \item Objects consist of all sets of the form~$\Obja^\Time$, with~$\Obja$ ranging over all sets.
            \item Morphisms from~$\Obja^\Time$ to~$\Objb^\Time$ are relations from~$\Obja^\Time$ to~$\Objb^\Time$:
                  \begin{equation}
                      \HomSet{\Beh_\Time}{\Obja^\Time}{\Objb^\Time} = \powerset (\Obja^\Time \cartprod \Objb^\Time).
                  \end{equation}
            \item Composition is the usual composition of relations.
            \item The identity morphism at an object~$\Obja^\Time$ is the usual identity relation from~$\Obja^\Time$ to itself.
        \end{enumerate}
    \end{definition}

    Sometimes we will wish to consider signal spaces $\Obja$ which are vector spaces over some field~$\field$.
    Then the set~$\Obja^\Time$ naturally inherits a~$\field$-vector space structure from~$\Obja$, and we may consider input-output behaviors
    \begin{equation*}
        \relA \subseteq \Obja^\Time \oplus \Objb^\Time
    \end{equation*}
    which are~$\field$-linear relations.

    \begin{definition}
        Fix a totally ordered set~$\Time = \tup{\Time, \leq}$.
        The category~$\LBeh_\Time$ of~$\field$-linear input-output behaviors is the category where:
        \begin{enumerate}
            \item Objects consist of all~$\field$-vector spaces of the form~$\Obja^\Time$, ranging over all~$\field$-vector spaces~$\Obja$.
            \item Morphisms from~$\Obja^\Time$ to $\Objb^\Time$ are linear subspaces of~$\Obja^\Time \oplus \Objb^\Time$.
            \item The identity morphism at an object~$\Obja^\Time$ is the usual identity relation from~$\Obja^\Time$ to itself (this is always~$\field$-linear).
            \item Composition is the usual composition of relations.
        \end{enumerate}
    \end{definition}

    \section{From specifications to behaviors}

    \todotext{write some intro lines to this section}

    \subsection{Moore machines}

    We have seen that given a Moore machine $\tup{\prin, \prout, \prst, \prdyn, \prreadout}$ and an initial state $x_0$, we obtain an associated behavior $\prin^\natnumbers \to  \prout^\natnumbers$ given by
    \begin{equation}
        u_k \mapsto \prreadout(\prdyn(x_k, u_k)),
    \end{equation}
    using the equations
    \begin{align}
        x_{k+1} & = \prdyn(x_k, u_k) \\
        y_{k}   & = \prreadout(x_{k}).
    \end{align}

    \subsection{More machines}

    \section{Properties of behaviors}

    \subsection{Causal behaviors}

    Given a signal~$\ela \setin \Obja^\Time$, we set the notation
    \begin{equation}
        \ela \vert_{\leq t} \definedas  \ela \vert_{\{ s \setin \Time \mid s \leq t\}}.
    \end{equation}

    \begin{definition}
        An input-output behavior~$\relA \subseteq \Obja^\Time \times \Objb^\Time$ is \emph{causal} if for all~$\tup{\ela_1, \elb_1}$, $\tup{\ela_2, \elb_2} \setin \relA$ holds:
        \begin{equation*}
            \forall t \setin \Time\colon \ela_1 \vert_{ \leq t} = \ela_2 \vert_{ \leq t} \  \Rightarrow \ \elb_1 \vert_{ \leq t} = \elb_2 \vert_{ \leq t}
        \end{equation*}
    \end{definition}

    \todotext{Questions: If two composable input-output behaviors are causal, is their composition also causal?
        What if only one of them is causal?
        And various other related questions.
        .. }

    \subsection{Time-invariance}

    In this subsection we assume that $\Time = \tup{\Time, \leq}$ comes equipped with a semigroup structure $\tup{\Time, +}$ which is compatible with the total order: if $t \leq t'$ and $s \leq s'$, then $t + s \leq t' + s'$.

    Fix an $s \setin \Time$.
    For any signal space~$\Obja$, we define a shift function~$\delta_s\colon \Obja^\Time \to \Obja^\Time$ by
    \begin{equation*}
        (\delta_s \ela)(t) \definedas \ela(t + s).
    \end{equation*}

    \begin{definition}
        An input-output behavior $\relA \subseteq \Obja^\Time \times \Objb^\Time$ is \emph{time-invariant} if
        \begin{equation*}
            \forall s \setin \Time\colon \tup{\ela, \elb} \setin \relA  \ \implies \tup{\delta_s \ela, \delta_s \elb} \setin \relA.
        \end{equation*}
    \end{definition}

    \todotext{Questions: If two time-invariant input-output behaviors are causal, is their composition also causal?
        What if only one of them is causal?
        And various other related questions.
        .. }

}