% !TEX root = chapter-standalone.tex

\section{Properties of behaviors}

\subsection{Causal behaviors}

Given a signal~$\ela \setin \Obja^\Time$, we set the notation
\begin{equation}
    \ela \vert_{\leq t} \definedas  \ela \vert_{\{ s \setin \Time \mid s \leq t\}}.
\end{equation}

\begin{definition}
    An input-output behavior~$\relA \setsubseteq \Obja^\Time \times \Objb^\Time$ is \emph{causal} if for all~$\tup{\ela_1, \elb_1}$, $\tup{\ela_2, \elb_2} \setin \relA$ holds:
    \begin{equation*}
        \forall t \setin \Time\colon \ela_1 \vert_{ \leq t} = \ela_2 \vert_{ \leq t} \  \Rightarrow \ \elb_1 \vert_{ \leq t} = \elb_2 \vert_{ \leq t}
    \end{equation*}
\end{definition}

\todotext{Questions: If two composable input-output behaviors are causal, is their composition also causal?
    What if only one of them is causal?
    And various other related questions.
    .. }

\subsection{Time-invariance}

In this subsection we assume that $\Time = \tup{\Time, \leq}$ comes equipped with a semigroup structure $\tup{\Time, +}$ which is compatible with the total order: if $t \leq t'$ and $s \leq s'$, then $t + s \leq t' + s'$.

Fix an $s \setin \Time$.
For any signal space~$\Obja$, we define a shift function~$\delta_s\colon \Obja^\Time \to \Obja^\Time$ by
\begin{equation*}
    (\delta_s \ela)(t) \definedas \ela(t + s).
\end{equation*}

\begin{definition}
    An input-output behavior $\relA \setsubseteq \Obja^\Time \times \Objb^\Time$ is \emph{time-invariant} if
    \begin{equation*}
        \forall s \setin \Time\colon \tup{\ela, \elb} \setin \relA  \ \implies \tup{\delta_s \ela, \delta_s \elb} \setin \relA.
    \end{equation*}
\end{definition}

\todotext{Questions: If two time-invariant input-output behaviors are causal, is their composition also causal?
    What if only one of them is causal?
    And various other related questions.
    .. }

