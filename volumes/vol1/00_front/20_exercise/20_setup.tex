% !TEX root = chapter-standalone.tex

\sectionexercises{Overview}

\subsection{Prerequisites}

You will need the following skills:

\begin{itemize}
    \item Python programming.
          If you want to learn Python, we suggest you look at
          \href{https://www.edx.org/learn/python}{EdX's courses}.
    \item Git / GitHub usage.
          If you want to learn Git and GitHub, go to \href{https://try.github.io/}{try.github.io}.
\end{itemize}

Additionally, it might be useful to have some idea of what Docker is.
In theory, you don't need to know anything about Docker, if everything goes well in our autograding scripts.

\subsection{Option 1: Using GitHub Classroom}\label{subsec:with-classroom}

\begin{quote}
    \emph{
        If you are following a class where the instructors are using GitHub Classroom,
        this is the section for you.
        Otherwise, jump to the section below about ``independent use''.
    }
\end{quote}

\subsubsection{Sign up on GitHub}

Sign up on GitHub if you haven't already.

\subsubsection{Get invite link and accept assignment}
GitHub Classroom is a product that helps coordinate assignments for large classes.
%

If you are enrolled in a class, your instructors will send you an invitation link to jo{}in the classroom.

If you are not enrolled in an official class, then you can use this invitation link:

\url{https://classroom.github.com/a/IMtOmeOH}

In any case, do not pass invite links around: there are different ``classrooms''.
Do not use old links because the classroom is refreshed after each season.

\subsubsection{Clone the repository}

Once you go through the procedure, you will have created a repository
%
\begin{quote}
    \str{https://github.com/INSTRUCTORS-term-exercises-YOU}
\end{quote}
where \str{INSTRUCTORS} is the GitHub organization for the instructors, \str{term} is the current term (e.g., \str{spring2022}), and \str{YOU} is your username.

Clone the repository just created and jump into the directory.

\begin{marginfigure}
    \includesag{dirtree-setup}
    \caption{Original content of the exercise template repository.}
    \label{fig:dirtree-setup}
\end{marginfigure}

You should see the content as in \cref{fig:dirtree-setup}.

In the following we list two options for you to be able to use the ACT4E interfaces when you code.

\subsection{Option 2: Independent use without GitHub Classroom}
\label{subsec:without-classroom}

\begin{quote}
    \emph{
        If you are \emph{not} following a class where the instructors are using GitHub Classroom,
        this is the section for you.
        Otherwise, jump to \cref{subsec:with-classroom}.
    }
\end{quote}

In this section we assume that you are not part of a class.

In that case, you can fork directly the repository

\url{https://github.com/ACT4E/ACT4E-exercises-template}

Note that there might be different branches to choose from.
The instructions in this book correspond to the branch \texttt{alphubel-prod}.

You can do the exercises on your own by using this repository.
%
% In the following, ignore any other mention of GitHub Classroom.

\subsection{Using a Visual Studio Code development container}

The exercises are meant to be solved in a Visual Studio Code ``development container''.
This allows the experience to work on Windows, Mac, and Linux.

The way this works is that Visual Studio Code will create a Docker environment you can work in.
You will be able to see and edit files as usual, but the code will run in the container.

Please follow the instructions in the \files{README.md} file in the repository.

They will guide you through the installation of Docker, Visual Studio Code, and the development container.

Note that:

\begin{itemize}
    \item You can still \emph{edit} the files with an external editor.
          You do not have to use VS Code as an editor.
          However, you need to run the code from inside the development container.
    \item Even if you are an expert, do not try to run the code from outside the container--there are some Python dependencies that are only available for Linux (arm/amd) and Python 3.10.
\end{itemize}

\subsection{Updating the exercises packages}

The exercises themselves and the interfaces are stored in the repository
\begin{quote}
    \url{https://github.com/ACT4E/ACT4E-exercises.git},
\end{quote}
and it is published as the package \str{ACT4E-exercises}.

Sometimes there will be a need to update this package, as we improve it or fix bugs.

There are two ways to do this.
\begin{enumerate}
    \item You can update the package alone by running the following command \textbf{inside the container}:

          \begin{console}
        python -m pip install -U ACT4E-exercises
    \end{console}
    \item You can use the Visual Studio Code command \emph{Rebuild development container without cache}.
\end{enumerate}

%
% The next step is to make sure that you can use the ``interfaces repository''.
% That repository contains all the Python snippets you see in this book.
% Your IDE needs it so that it can help you write the correct code.
%
% For this, we offer two options.
% You can clone the ``interfaces'' repository (from within your repository):
%
%
%
% and in there do
%
%
% Alternatively, you can directly install the python package:
%
% \begin{console}
%     pip install ACT4E-exercises
% \end{console}
%
% In both cases, please pull (in case you clone the repository) or update (in case you install the package) weekly, to always use the most updated version.

\subsection{Without Visual Studio Code, from command line}

If you cannot use Visual Studio Code, you can still edit the files from your favorite editor and use Docker alone
to check the results.

Whenever you will be instructed to run
\begin{console}
    act4e-check EXERCISENAME
\end{console}
You should run the following command instead:
\begin{console}
    make docker-check-EXERCISENAME
\end{console}

\subsection{Using PyCharm + Docker environment}
\label{sec:pycharm-plus-docker}

You can configure PyCharm to obtain the same result as Visual Studio Code, albeit with a bit more configuration.

The idea is that we are going to configure a ``Python interpreter'' in PyCharm that corresponds to the interpreter \emph{inside} a Docker image.

First, go to the root folder and type
\begin{console}
    make build
\end{console}
This will create an image called \str{act4e-image}.
Make sure you can run it using:
\begin{console}
    docker run -it act4e-image python --version
\end{console}
Which should give a result like:
\begin{console}
    Python 3.10.2
\end{console}

Create a new PyCharm project in the root folder.

In Preferences - Python Interpreter - find the gear and choose ``add interpreter''.

Choose the ``Docker'' type.
Put \pystr{act4e-image} as the image.

Enable this interpreter for the project.

Now you should be able to open any of the template files and all symbols should be resolvable.
For example, you should be able to click \str{act4e_interfaces} and explore the contents of the package.

