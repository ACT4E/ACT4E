% !TEX root = chapter-standalone.tex
\section{Set up}

\subsection{Sign up on Github}

Sign up on Github if you haven't already.

\subsection{Option 1: Using Github Classroom}\label{subsec:with-classroom}

\begin{quote}
  \emph{
    If you are following a class where the instructors are using Github Classroom,
    this is the section for you. Otherwise, jump to \cref{sub:without-classroom}.}
\end{quote}

\subsubsection{Get invite link and accept assignment}
Github Classroom is a product that helps coordinate assignments for large classes.

Your instructors will send you an invite link to join the classroom. An invite link is similar to this:

\url{https://classroom.github.com/a/XXXXXXX}

If you are not enrolled in an official class, then you can get the invite link on Zulip.

In any case, do not pass invite links around: there are different ``classrooms''.
Do not use old links because the classroom is refreshed after each season.

\subsubsection{Clone the repository}

Once you go through the procedure, you will have created a repository

\begin{quote}
\str{https://github.com/INSTRUCTORS/solution-YOU}
\end{quote}
where \str{INSTRUCTORS} is the Github organization for the instructors, and \str{YOU} is your username.

Clone the repository just created and jump into the directory.

\begin{marginfigure}
\includesag{dirtree-setup}
\caption{Original content of the exercise template repository.}
\label{fig:dirtree-setup}
\end{marginfigure}

You should see the content as in \cref{fig:dirtree-setup}.

\subsubsection{Clone the interfaces repository}

The next step is to clone the ``interfaces'' repository.
That repository contains all the Python snippets you see in this book.
Your IDE needs it so that it can help you write the correct code.

Clone the following repository:

\url{https://github.com/ACT4E/ACT4E-exercises.git}

and in there do

\begin{console}
python setup.py develop
\end{console}


\subsection{Option 2: Independent use without Github Classroom}
\label{sub:without-classroom}

\begin{quote}
  \emph{
    If you are \emph{not} following a class where the instructors are using Github Classroom,
    this is the section for you. Otherwise, jump to \cref{subsec:with-classroom}.
  }
\end{quote}

In this section we assume that you are not part of a class.

In that case, you can fork directly the repository

\url{https://github.com/ACT4E/ACT4E-exercises-template}

Note that there might be different branches to choose from. The instructions in this book correspond to
the branch \texttt{spring2021}.

You can do the exercises on your own by using this repository.

In the following, ignore any other mention of Github Classroom.

\subsection{Docker installation}

Install Docker following the information at

\url{https://docs.docker.com/get-docker/}



