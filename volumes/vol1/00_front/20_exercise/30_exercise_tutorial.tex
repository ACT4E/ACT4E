% !TEX root = chapter-standalone.tex

\section{Exercise tutorial}
\label{sec:exercise-tutorial}

During the exercises, you are going to build a Python library to build and manipulate most of the concepts in the book.

The exercises will give you the \emph{interface} of an \emph{abstract class} to be generated.

The interfaces are defined in the package \str{act4e_interfaces} which is in the repository \str{ACT4E/ACT4E-interfaces}.
% This package can be installed using
%
% \begin{console}
% python3 -m pip install ACT4E-interfaces
% \end{console}

For example, \cref{lst:SimpleIntro} shows an interface \classname{SimpleIntro} to be implemented.

\classlisting{SimpleIntro}

We are going to solve this exercise together.

We assume that you have checked out the repository as explained in the setup section.

% Of all the files in the repository, we only need the ones displayed in \cref{fig:dirtree-relevant}.

We can start by checking if this is a valid solution.

In the development environment, type this:

\begin{console}
    act4e-check TestSimpleIntro
\end{console}

Here \exname{TestSimpleIntro} is the code name for the exercise.
You can check other exercises by changing the name.
Each exercise name is of the form \str{TestINTERFACE} where \str{INTERFACE} is the name of the Python class
providing the interface.

The command above will fail; it will complain saying that something is not implemented.

Go to the file called \files{intro.py} in the \files|src/act4e_solutions| directory.
This file contains the skeleton for the class implementation, as in \cref{lst:act4e_book_examples.intro_empty}.

\begin{longcode}
    \caption{}
    \visualizemodule{act4e_book_examples.intro_empty}{}

    \label{lst:act4e_book_examples.intro_empty}
\end{longcode}

Now modify the file to implement the right computation, as in \cref{lst:act4e_book_examples.intro_good}.

\begin{longcode}
    \caption{}
    \visualizemodule{act4e_book_examples.intro_good}{}
    \label{lst:act4e_book_examples.intro_good}
\end{longcode}

Trying again this:

\begin{console}
    act4e-check TestSimpleIntro
\end{console}

You will see a success result.

