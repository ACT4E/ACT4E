% !TEX root = chapter-standalone.tex

\section{Exercise tutorial}
\label{sec:exercise-tutorial}

During the exercises, you are going to build in Python a library to build and manipulate most of the concepts in the book.

The exercises will give you the \emph{interface} of an \emph{abstract class} to be generated.

The interfaces are defined in the package \str{act4e_interfaces} which is in the repository \str{ACT4E/ACT4E-interfaces}.
This package can be installed using

\begin{console}
    python3 -m pip install ACT4E-interfaces
\end{console}

For example, \cref{lst:SimpleIntro} shows an interface \classname{SimpleIntro} to be implemented.

\classlisting{SimpleIntro}

We are going to solve this exercise together.

We assume that you have checked out the repository as explained in the setup section.

Of all the files in the repository, we only need the ones displayed in \cref{fig:dirtree-relevant}.

We can start by checking if this is a valid solution.
There is a recipe for this in the \files{Makefile}.
Type this:

\begin{console}
    make check-TestSimpleIntro
\end{console}

Here \exname{TestSimpleIntro} is the code name for the exercise.
You can check other exercises by changing the name.

This will fail; it will complain saying that it didn't even find any code implementing the exercise.

Create a file called \files{first.py} in the \files|src/act4e_solutions| directory.
This file will contain the implementation of the classes.

For now we are just going to add non-functional code for the classes, as in \cref{lst:act4e_book_examples.intro_empty}.

\begin{longcode}
    \caption{}
    \visualizemodule{act4e_book_examples.intro_empty}{}

    \label{lst:act4e_book_examples.intro_empty}
\end{longcode}

Now modify \files|__init__.py| to import all the symbols from the \files{first} module.

\begin{minted}{python}
    from .first import *
\end{minted}

At this point you can check again by typing:

\begin{console}
    make check-TestSimpleIntro
\end{console}

The program will tell you that it found the code supposedly implementing the function, but that the tests failed.
All of our functions returned \pystr{None}.

At this point we need to implement the rest of the code.

For example, \cref{lst:act4e_book_examples.intro_good} is a valid implementation.

\begin{longcode}
    \caption{}
    \visualizemodule{act4e_book_examples.intro_good}{}
    \label{lst:act4e_book_examples.intro_good}
\end{longcode}

Trying again this:

\begin{console}
    make check-TestSimpleIntro
\end{console}

you will receive a success result.

