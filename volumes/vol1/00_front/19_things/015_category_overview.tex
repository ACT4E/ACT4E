% !TEX root = chapter-standalone.tex
\begin{figure*}[p]
    \centering
    %\includepdf[scale=0.8,pages={22-26},nup=1x3,frame,pagecommand={}]{ACT4E-09-design.pdf}
    \fbox{\includegraphics[trim=70 90 80 80,clip,height=0.9\textheight]{lego-patent-embed.pdf}}
    \caption{The 1961 Lego patent.}
\end{figure*}

\section{Stacking blocks}

%\linkvideo{spring2021-intro:why-cat-theory} % Why category theory?
\linkvideo{spring2021-intro:composition} % Composition

\cref{sec:brief-history}

The first encounter children have with composition is with toy blocks like Lego.
%
\iflabelexists{sec:brief-history}{
    It is a coincidence that there is a \emph{lego} in \emph{intellego} (as explained in \ref{sec:brief-history}) ; the \emph{lego} in Lego is a contraction from Danish \emph{leg godt}, which means \emph{to play well}.
}{}

Legos are compositional in this sense: when you put together two blocks, you can treat the ensemble as one block for the purpose of composing it with other blocks.

We are going to use the following graphical notation to talk about composition.
We draw a black bar, and we write the \emph{ingredients} at the top, and the \emph{results} at the bottom.
%
\begin{equation}
    \vmiddle{
        \prfperiod{
            \text{ingredient}
        }{
            \text{ingredient}
        }{
            \text{ingredient}
        }{
            \text{result} \quad \quad \text{result}
        }
    }
\end{equation}
%
Note that the order of the ingredients matters.
For instance, we can have the following recipes for the composition of red and white bricks.
We scan the list of ingredients from left to right and then place the bricks on top of what is already on the table.

Composing red and white produces a red-white brick:
%
\begin{equation}
    \label{eq:red-white}
    \middlesag{composing_lego_white_red}
\end{equation}
%
Composing white and red produces a white-red brick:
%
\begin{equation}
    \middlesag{composing_lego_red_white}
\end{equation}

We can compose more than one brick.
For example, red, white, blue, make a red-white-blue brick:
%
\begin{equation}
    \middlesag{composing_lego_red_white_blue}
\end{equation}

In Lego, we can also decompose.
If we have a red-white-blue brick, we can also recover the single bricks:
%
\begin{equation}
    \middlesag{decompose_legos}
\end{equation}

If you have 3 bricks on a table, you can also permute them:
%
\begin{equation}
    \middlesag{permute_legos}
\end{equation}

Consequently, if you have a red-white-blue brick, you can disassemble, permute, and reassemble to obtain a blue-white-red:
%
\begin{equation}
    \label{eq:recipe-full}
    \middlesag{compose_decompose_permute}\quad.
\end{equation}

The aforementioned recipe contains several concrete steps to go from the initial ingredient to the final result.
If we do not care about the detailed steps, we can summarize the recipe as follows, by eliding the intermediate steps and only remember the ingredient and the results:
%
\begin{equation}
    \label{eq:recipe-theorem}
    \middlesag{lego_recipe_theorem}
\end{equation}

Alternatively, you can think of \cref{eq:recipe-theorem} as the statement of a theorem, and of \cref{eq:recipe-full} as the proof of the theorem.

Sometimes we want to think about the transformations that are reversible.
For example, we can assemble 3 red bricks into a red-red-red brick:
%
\begin{equation}
    \middlesag{assemble_red_red_red}
\end{equation}
%
We can also do the opposite:
%
\begin{equation}
    \middlesag{dissemble_red_red_red}
\end{equation}
%
To describe the bi-directionality, we use a double line:
%
\begin{equation}
    \middlesag{ass_diss_red_red_red}
\end{equation}
%
The flat pieces of Lego we have looked are actually one third shorter than a ``regular'' piece:
%
\begin{equation}
    \middlesag{lego_pieces}
\end{equation}
%
What is the relation between a red-red-red assembly and a full red brick?
One point of view that will be very useful is thinking in terms of ``substitution'': if I have one of those, can I use it as if I had the other?
Lego bricks are very strong when assembled: a red-red-red assembly can certainly substitute a regular brick in terms of structural functionality.
Therefore, given a red-red-red we can treat it as a full block, but not vice versa:
%
\begin{equation}
    \middlesag{lego_melt}
\end{equation}

\section{Mixing colors}

\begin{marginfigure}
    \centering
    \subfloat[\label{fig:subtractive}
        Subtractive composition]{
        \adjustbox{width=4cm}{\fbox{\includegraphics[width=4cm]{subtractive}}}
    }\\
    \subfloat[\label{fig:additive} Additive composition]{
        \includegraphics[width=4cm]{additive}
    }
    \caption{Additive vs subtractive composition}
\end{marginfigure}

We now look at how we can compose colors.
In Denmark there is a small group of \textbf{Lego purists}: they are only able to conceive of Lego assemblies where all bricks have the same color.
For them, a blue, red, white brick, make a block of a color they call \emph{horrible}:
%
\begin{equation}
    \middlesag{lego_purists_horrible}
\end{equation}

If you ask a color purist, they will tell you that red and red make red:
%
\begin{equation}
    \middlesag{red_red_makes_red}
\end{equation}
%
Furthermore, white and white make white:
%
\begin{equation}
    \middlesag{white_white_makes_white}
\end{equation}
%
However, white and red make \emph{horrible}:
%
\begin{equation}
    \middlesag{white_and_red_makes_horrible}
\end{equation}
% If we start with 2 colors + \emph{horrible}, there are in total 9 combinations,
% which can be written in a table.

% \begin{center}
% \includesag{colors-purist}
% \end{center}

We can think of many other ways to compose colors.
For example, we can think of formalizing what happens when you \textbf{mix paint}.
Red and white in equal measure give pink.
By mixing and mixing we can obtain all the shades that go from red to white:
%
\begin{equation}
    \label{eq:colors-mixing}
    \middlesag{colors-mixing}
\end{equation}

Colors on a monitor mix in an \textbf{additive} way.
Two dark reds give a brighter red.
Red and white remains white:
%
\begin{equation}
    \label{eq:colors-additive}
    \middlesag{colors-additive}
\end{equation}

Green, red, blue additively make white:
%
\begin{equation}
    \label{eq:colors-additive-rgb}
    \middlesag{colors-additive-rgb}
\end{equation}

A different way to compose colors is by using the \textbf{subtractive} rules in the CMY (cyan, magenta, yellow) color space.
These rules formalize the physical process of offset printing: we produce colors by putting pigments that block the other colors:
%
\begin{equation}
    \label{eq:colors_subtractive}
    \middlesag{colors-subtractive}
\end{equation}

This is how you produce red, blue, green from CMY:
%
\begin{equation}
    \label{eq:colors-subtractive-rgb}
    \middlesag{colors-subtractive-rgb}
\end{equation}
%
Finally, we can think of a \textbf{paint-over-it} composition rule: the first color is replaced by the second:
%
\begin{equation}
    \label{eq:colors-paint-over}
    \middlesag{colors-paint-over}
\end{equation}
