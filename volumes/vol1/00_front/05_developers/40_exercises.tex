% !TEX root = chapter-standalone.tex

\sectionexercises{Overview of code exercises}

\subsection{Repositories}
There are 3 repositories involved:
\begin{enumerate}
    \item \str{ACT4E/ACT4E-exercises} - a \textbf{public} repository.
          This contains the Python interfaces, or ``problem statements''.
    \item \str{ACT4E/ACT4E-exercises-template} - a \textbf{public} repository.
          This is the repository that the students need to clone and work inside.
    \item \str{ACT4E/ACT4E-private} - a \textbf{private} repository.
          This contains the solutions to the exercises, plus all the test infrastructure that allows the students to test the solution.
\end{enumerate}

To do development, these repositories need to be checked out inside the book repository.

\subsection{Workflow for code}

\subsubsection{Workflow for students}

From the point of view of the student, this is the workflow:

\begin{enumerate}
    \item Fork \str{ACT4E/ACT4E-exercises-template}.
          If they use GitHub Classroom then this step will be done for them.
    \item Work on the exercises inside this repository.
    \item To evaluate the exercises, type
          \checkexercise{EXERCISENAME}
\end{enumerate}

The evaluation is done by Docker.
In particular these are the steps:
%
\begin{enumerate}
    \item The instructors have generated a certain Docker image \evalrepo and published to Docker Hub (explained later in \cref{{subsub:developers}}).
          This image contains only binaries, so the user cannot see the source code for the tests or our reference implementation.
    \item The \files{Dockerfile} inside \str{ACT-exercises-template} build from \evalrepo.
          It adds more dependencies that the user might have declared in \files{requirements.txt}.
    \item The original repo \evalrepo contains the tester \files{act4e-tests}.
          In the \files{Makefile}, you can see how we pass the group of tests to be called to that executable.
\end{enumerate}
\begin{minted}[firstline=2,lastline=5,fontsize=\footnotesize]{console}
    \begin{verbatim}
        check-%: build
        docker run -it --rm $(tag) act4e-tests \
            --module act4e_solutions \
            --group $*
    \end{verbatim}
\end{minted}
\subsubsection{Workflow for developers}\label{subsub:developers}

A complete rebuild of the code is done through these steps.

(At this point Andrea is the only one to be able to do this, because of various credentials issues.)

\begin{enumerate}
    \item Commit and push what is on \str{ACT4E-exercises} to the current branch.

    \item In \str{ACT4E-private} type \str{make build}.
          This creates and pushes the \evalrepo.
\end{enumerate}
%
%\subsection{Workflow for book creation}\label{sub:book-creation}
%
%There is now some smartness in how the book is created. For example, we are able
%to pull directly the Python code into the book. This needs external programs and some configuration.
%
%At this point this is something that only Andrea can do as the things are not well packaged.
%
%\begin{console}
%docker run --rm -v ${PWD}:${PWD} -w ${PWD} andreacensi/act4e:spring2021 act4e-tests --module act4e_solutions --group TestFiniteSetProperties
%\end{console}
