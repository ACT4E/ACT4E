% !TEX root = chapter-standalone.tex


\section{Compiling the book}

\subsection{Dependencies}

You need to install Pygments.

\subsection{Top level compilation}

A quick way to start is:

\begin{console}
  make ACT4E-vol1.pdf -B
\end{console}

\subsection{Partial compilation}

It is possible to do partial compilation of chapters and parts.


From a directory containing a file \files{chapter-standalone.tex}, give this command:

\begin{console}
  make chapter-once
\end{console}

You can use
%
\begin{console}
  make chapter-continuous
\end{console}
%
to make \str{latexmk} compile continuously and your pdf viewer to synchronize if you set it up as in \cref{sub:overleaf-experience}.

To compile entire parts, you can use \str{make part-once} or \str{make part-once} from any folder containing
\files{part-standalone.tex}.

You don't need to change into the directories.
From the root you can use commands like these:
%
\begin{console}
  make -C volumes/vol1/00_front/ part-continuous
  make -C volumes/vol1/00_front/05_developers/ chapter-continuous
\end{console}

\subsection{Getting the overleaf experience locally}
\label{sub:overleaf-experience}
It is suggested you use a program like Skim that allows to click and jump to the code.

Put this in a file \files{~/.latexmkrc}:

\begin{console}
  $pdf_previewer = 'open -a Skim'
\end{console}

\subsection{Flags}

We have the following flags to control compilation:
\begin{table*}
\begin{tabular}{ll}
  \str{instructors}   & Include text only for instructors (command \str{\instructors})                      \\
  \str{devel}         & Include text only for writers/developers (command \str{\devel}), like this section. \\
  \str{statuscolors}  & Show the status colors in the headers                                               \\
  \str{debugimages}   & Show the name of the tikz file                                                      \\
  \str{cachepdf}      & Use the pre-rendered figures.                                                       \\
  \str{codeexercises} & Include the code exercises.                                                         \\
  \str{bookversion}   & (Obsolete?)                                                                         \\
\end{tabular}
\end{table*}

\subsection{Top targets}

Relevant top files with various choices of flags:
\begin{table*}
\begin{tabular}{lcccccc}
  filename
  & \str{instructors}
  & \str{devel}
  & \str{statuscolors}
  & \str{debugimages}
  & \str{codeexercises}
  & \str{cachepdf}
  \\
  \files{ACT4E-vol1.tex} &
  yes & yes & yes & yes & yes & no \\
  \files{ACT4E-vol1-fast.tex} &
  yes & yes & yes & yes & yes & yes \\
  \files{ACT4E-vol1-final.tex} &
  no & no & no & no & yes & no \\
  \files{ACT4E-vol1-instructors.tex} &
  yes & no & no & no & yes & no \\
  \files{ACT4E-vol2.tex} &
  yes & yes & yes & yes & yes & no \\
  \files{ACT4E-vol2-final.tex} &
  no & no & no & no & yes & no \\
\end{tabular}
\end{table*}
\subsection{Details about compilation process}

For everything to run smoothly, there are a few steps to run outside latex.

There are some other steps that generate some material.

\subsubsection{Recreating the directory structure}

If you add a chapter/part, you should use the command

\begin{console}
  make recursive
\end{console}

This will create the various stubs files in the various subdirectories.
The files include:
\begin{itemize}
  \item \str{Makefile}
  \item \str{chapter-standalone.tex}
  \item \str{part-standalone.tex}
  \item \str{part-standalone.tex}
\end{itemize}
Also these links will be created:
\begin{itemize}
  \item \str{chapter-link-snippets}
  \item \str{chapter-link-minted}
  \item \str{part-link-snippets}
  \item \str{part-link-minted}
\end{itemize}
These links allow the partial compilation process to access the files in the root of the repo.

\subsubsection{Nomenclature and symbol table}

\todotext{@AC: document nomenclature }

\subsubsection{Python snippets from source code}

\todotext{@AC: document snippets }

\subsubsection{Prerender of figures}

\todotext{@AC: document prerender }
