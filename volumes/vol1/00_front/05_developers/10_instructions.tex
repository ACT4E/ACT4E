% !TEX root = chapter-standalone.tex

\section{Compiling the book using Docker}
\label{sec:compile-docker}

\subsection{Setup}

Install docker.

Run:
%
\begin{console}
    docker pull act4e/act4e-build:alphubel
\end{console}
%
Stop if any error.

\subsection{Clone the book repo}

Clone the root repo:
%
\begin{console}
    git clone --depth=1000 git@github.com:ACT4E/ACT4E.git
\end{console}

If you have access to the solutions repo, do the following.

Inside the resulting folder, clone the solutions repo:
%
\begin{console}
    git clone git@github.com:ACT4E/ACT4E-solutions.git
\end{console}

\subsection{Compilation}

In the repo root use:
%
\begin{console}
    make shell
\end{console}
%
and from the shell you can run the commands you would usually run.

For example, to compile the book in development mode, use:
%
\begin{console}
    make ACT4E-devel-slow.pdf
\end{console}
%
To compile a chapter:
%
\begin{console}
    make -C volumes/vol1/10_arrows/10_cats/ chapter-continuous
\end{console}

Alternatively, outside Docker, you can use
\begin{console}
    make docker-XXX
\end{console}
to make the target \str{XXX} inside Docker.
For example, you can use
\begin{console}
    make docker-ACT4E-devel-slow.pdf
\end{console}
%

\begin{remark}
    The \str{.aux} files and other temporary files are incompatible between the Docker latex and your system's latex.
    You will have to delete them when switching; or the first compilation will fail.
\end{remark}

\subsection{Regenerating nomenclature and other generated tables}

Now go to the book repo and type:
%
\begin{console}
    make magic
\end{console}
%
This will create nomenclature, table, standalone equations.

To compile the standalone equations, you should run (in or outside) Docker,
%
\begin{console}
    make -C equations
\end{console}

\subsection{Regenerating snippets for Python code}

For the next step you must have checked out in the book root the two repos \str{ACT4E-exercises} and \str{ACT4E-private}.

If everything is good you should be able to run:
%
\begin{console}
    make ultramagic
\end{console}
%
which does fancier things (extracts Python code).

\section{Known issues}

\subsection{Compiling without docker: latest LaTeX distribution needed}

If you are compiling the LaTeX files without Docker, you may run into issues if you are not using the lasted distribution of LaTeX, because some commands that are used are not supported by older LaTeX distributions.

\subsection{Pygmentize and TexShop}

If you are using TexShop on a Mac: there is an issue with Pygmentize not working properly with the minted package.
The following should resolve the issue:

\begin{console}
sudo ln -s "$(which pygmentize)" /Library/TeX/texbin/pygmentize
\end{console}

See: \url{https://tex.stackexchange.com/questions/279214/pygmentize-not-working-properly-with-minted-package-in-texshop-on-os-x}

\section{Details about compilation process}

\subsection{Partial compilation}

It is possible to do partial compilation of chapters and parts.

From a directory containing a file \files{chapter-standalone.tex}, give this command:
%
\begin{console}
    make chapter-once
\end{console}

You can use
%
\begin{console}
    make chapter-continuous
\end{console}
%
to make \str{latexmk} compile continuously and your PDF viewer to synchronize if you set it up as in \cref{subsec:overleaf-experience}.

To compile entire parts, you can use \str{make part-once} or \str{make part-once} from any folder containing \files{part-standalone.tex}.

You don't need to change into the directories.
From the root you can use commands like these:
%
\begin{console}
    make -C volumes/vol1/00_front/ part-continuous
    make -C volumes/vol1/00_front/05_developers/ chapter-continuous
\end{console}

\subsection{Getting the overleaf experience locally}
\label{subsec:overleaf-experience}

It is suggested you use a program like Skim that allows to click and jump to the code.

Put this in a file \files{~/.latexmkrc}:
%
\begin{console}
    \$pdf_previewer = 'open -a Skim'
\end{console}

\subsection{Flags}

The flags to control compilation are in \cref{tab:flags}.

\begin{table*}[h]
    \label{tab:flags}
    \begin{tabular}{ll}
        \str{instructors}   & Include text only for instructors (command \str{\instructors}). \\
        \str{devel}         & Include text only for writers/developers (command \str{\devel}), like this section. \\
        \str{statuscolors}  & Show the status colors in the headers. \\
        \str{debugimages}   & Show the name of the Tikz files. \\
        \str{cachepdf}      & Use the pre-rendered figures. \\
        \str{codeexercises} & Include the code exercises. \\
        \str{showslides}    & Show the slide-only material. \\
    \end{tabular}
\end{table*}

\subsection{Top targets}

Relevant top files with various choices of flags are shown in \cref{tab:top-targets}.

\begin{table*}[h]
    \label{tab:top-targets}
    \begin{tabular}{lccccccc}
        filename
                                           & \str{instructors}
                                           & \str{devel}
                                           & \str{statuscolors}
                                           & \str{debugimages}
                                           & \str{codeexercises}
                                           & \str{cachepdf}
                                           & \str{showslides} \\
        \files{ACT4E-devel-slow.tex}       &
        yes                                & yes                 & yes & yes & yes & no  & no \\
        \files{ACT4E-devel-fast.tex}       &
        yes                                & yes                 & yes & yes & yes & yes & no \\
        \files{ACT4E-public-slow.tex}      &
        no                                 & no                  & no  & no  & yes & no  & no \\
        \files{ACT4E-public-fast.tex}      &
        no                                 & no                  & no  & no  & yes & yes & no \\
        \files{ACT4E-instructors-slow.tex} &
        yes                                & no                  & no  & no  & yes & no  & no \\
        \files{ACT4E-instructors-fast.tex} &
        yes                                & no                  & no  & no  & yes & yes & no
    \end{tabular}
\end{table*}

\subsection{Magic}
For everything to run smoothly, there are a few steps to run outside latex.

There are some other steps that generate some material.

\subsubsection{Recreating the directory structure}

If you add a chapter/part, you should use the command
%
\begin{console}
    make recursive
\end{console}

This will create the various stubs files in the various subdirectories.
The files include:
\begin{itemize}
    \item \str{Makefile}
    \item \str{chapter-standalone.tex}
    \item \str{part-standalone.tex}
    \item \str{part-standalone.tex}
\end{itemize}
Also these links will be created:
\begin{itemize}
    \item \str{chapter-link-snippets}
    \item \str{chapter-link-minted}
    \item \str{part-link-snippets}
    \item \str{part-link-minted}
\end{itemize}
These links allow the partial compilation process to access the files in the root of the repo.

\subsubsection{Nomenclature and symbol table}

\todotextjira{286}{\someday: @Andrea: document nomenclature }

\subsubsection{Python snippets from source code}

\todotextjira{286}{\someday: @Andrea: document snippets }

\subsubsection{Prerender of figures}

\todotextjira{286}{\someday: @Andrea: document prerender }

\section{Latex autoformatter}
\label{sec:latex-autoformatter}

We can set up Git so that the latex code is always reformatted to a canonical form before committing using \str{pre-commit} and \str{latexindent}.

These instructions should work for Mac.

Install \str{pre-commit}:
%
\begin{console}
    pip install pre-commit
\end{console}
%

Install \str{latexindent} using Brew - the one in the TeX distribution didn't work well for Andrea.
%
\begin{console}
    brew install latexindent
\end{console}
%
Which latexindent is on the path?
%
\begin{console}
    which latexindent
\end{console}
%
The output should be \str{/usr/local/bin/latexindent}.

Install \str{pre-commit} in the repo:
%
\begin{console}
    pre-commit install
\end{console}
%
Now every time you commit, \str{latexindent} will reformat the latex files.
(If so, then you should commit a second time.)
