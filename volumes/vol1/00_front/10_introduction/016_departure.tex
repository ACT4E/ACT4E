% !TEX root = chapter-standalone.tex

\section[Departures from tradition]{Departures from traditional exposition [experts only]}

\textbf{This section is for experts only.
    Skip at a first reading.
}

This section describes the ``departures from tradition'' in our text.
We made several choices to streamline the traditional exposition of category theory, to make it more understandable and relevant to the engineering field.

\subsection{Induction vs deduction}

The greatest difference between this text and a mathematical text is the use of an \textbf{inductive exposition} rather than a deductive explanation.
In a typical mathematical exposition of category theory, one defines a general mathematical structure, and then give several specific examples~\cite{riehl2017category}.

Instead, here we first build up the examples as something that is interesting per se, and then we show how they can be all be instances of the same general concept.
In this way, the general concept is well motivated.
The path laid by the book is one of \emph{spiral learning}.

For example, we look at various constructions from specific to general:
%
\begin{equation}
    \text{set product} \rightarrow \text{poset product} \rightarrow \text{categorical product}.
\end{equation}
%
Similarly, we discuss
%
\begin{equation}
    \text{monoid morphisms} \rightarrow \text{category actions} \rightarrow \text{functors}.
\end{equation}

\subsection{Materials covered}
\begin{itemize}
    \item Certain topics (limits, Yoneda's lemma, \etc) that would be traditionally discussed relatively early, are not discussed in this volume.
          We ordered topics by usefulness in engineering.

    \item The main text uses traditional set theory.
          To ground the exercises, we use slightly more formal \textbf{type theory} foundations (setoids, \etc).
          It is in our plan to transition completely to type theory also in the main text.
          Please contact us if you can help!

\end{itemize}

\begin{margintable}
    \caption{Use of colors}
    \label{tab:use-colors}
    \begin{tabular}{p{2.5cm}l}
        sets                           & $\setA, \setB$ \\
        posets                         & $\posA, \posB$ \\
        categories                     & $\CatC, \CatD$ \\
        objects                        & $\Obja, \Objb$ \\
        morphisms                      & $\mora \colon \Obja \mto \Objb$ \\
        functors                       & $\funa \colon \CatC \fto \CatD$ \\
        natural \mbox{transformations} & $\ntrafoa \colon \funa \nto \funb$
    \end{tabular}
\end{margintable}
\subsection{Use of colors}
\begin{itemize}
    \item We \textbf{use colors} to aid in the parsing of formulas and diagrams (\cref{tab:use-colors}).
          We also color the composition operations.
          In this way it is easy to see the types at first glance:
          $\mora\mthen\morb$, $\funa\fthen\funb$, etc.
    \item Color is \emph{not} necessary to infer meaning.
          The choice of colors is \textbf{colorblind-friendly} for red-green color blindness.
          (One of the authors is colorblind.)
          Please let us know if this is not the case.
\end{itemize}

\subsection{Notation and conventions}
\begin{itemize}
    \item In general, we use diagrammatic notation $\morab$ (pronounced ``$\mora$ then $\morb$'') rather than $\morb \after \mora$ (pronounced ``$\morb$ after $\mora$'') for function and morphism composition.
    \item In the discussion of \SY{semigroups}, we use ``$\mtimes$'' rather than ``$\after$'' as the \SY{semigroup} composition operation.
          This is because for us a semigroup/monoid is a special (semi)category with only one object.
    \item In \cref{chap:actions} we discuss \emph{covariant} and \emph{contravariant} actions.
          We do not use the terms \emph{left} and \emph{right} actions because they are notation-dependent.
    \item We abundantly use \emph{semi}categories (\SY{semifunctors}, \etc).
          For us, \SY{semicategory} is the primitive definition.
          A category is a \SY{semicategory} with a particular property: having identities at each object.
\end{itemize}

\begin{margintable}
    \caption{Tuple \SY{subcategories} of well-known categories}
    \begin{tabular}{cc}
        original               & tuples \SY{subcategory} \\
        \Set (\cref{def:Set})  & \SetL  (\cref{def:SetL}) \\
        \Pos  (\cref{def:Pos}) & \PosL  (\cref{def:PosL}) \\
        \Rel  (\cref{def:Rel}) & \RelL  (\cref{def:RelL}) \\
    \end{tabular}
\end{margintable}
\subsection{Extensive use of tuples}

We extensively use tuples and tuples concatenation to work directly with \SY{strict monoidal categories}.

This is a list of standard categories together with their ``tupled'' definition.

\subsection{Treatment of monoidal categories}

We noticed that there is a step increase in difficulty associated to \SY{natural transformations}, without much immediate justification.
In our trajectory, \SY{natural transformations} appear first associated to \SY{monoidal categories}.
The role they play there is associated to very technical checks.
Nothing exciting!
We made the decision to provide a version of \SY{monoidal categories} that are strict, so that there is no need for \SY{natural transformations}.

We also noticed sever examples of interest (\eg proper LTI system) that are only \SY{semicategories}, but they still have a notion of trace.

Furthermore, we noticed that in applications that there are several interesting examples that have a notion of vertical composition but the monoidal structure is not functorial on the nose: for example, systems with states.

Because there is a large part of concrete code exercises, it was not convenient for us to just wave our hands and say things like ``let's just consider the strict / modulo isomorphism version''.

In conclusion, in \cref{chap:parallelism} we introduce several notions of ``stacking'' categories, which are defined for \SY{semicategories}, are strict in the vertical composition operation, and for which the functoriality of the monoidal structure is not a given.

We provide the traditional exposition of \SY{monoidal categories} in \cref{chap:generalization} after the introduction of \SY{natural transformations} in \cref{chap:naturality}.
