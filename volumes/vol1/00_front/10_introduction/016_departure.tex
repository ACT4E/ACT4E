% !TEX root = chapter-standalone.tex

\section{Departures from traditional exposition}

We made several choices to streamline the traditional exposition.

\paragraph{Notation and convention}
\begin{itemize}
    \item In general, we use diagrammatic notation $\mora \then \morb$ (``$\mora$ then $\morb$'') rather than $\morb \after \mora$ (``$\morb$ after $\mora$'') for function composition.
    \item In the discussion of semigroups, we use ``$\then$'' as the semigroup composition operation.
    This is because for us a semigroup/monoid is a special (semi)category with only one object.
    \item In \chrefplus{chap:actions} we discuss \emph{covariant} and \emph{contravariant} actions.
    We do not use the terms \emph{left} and \emph{right} actions because they are notation-dependent.
    \item We abundantly use semicategories (semifunctors, \etc).
    For us semicategory is the primitive definition.
    A category is a semicategory with a particular property: having identities at each object.
\end{itemize}

\paragraph{Use of colors}
\begin{itemize}
    \item We use colors to aid in the parsing of formulas and diagrams.
    However, color is not necessary to infer meaning.
    The choice of colors is colorblind-friendly.
    (One of the authors is colorblind.)
\end{itemize}

\paragraph{Materials covered}
\begin{itemize}
    \item Certain topics (limits, Yoneda's lemma, \etc) that would be traditionally discussed relatively soon, are not discussed in the current book.
    \item The text uses traditional set theory.
    In the exercises we use slightly more formal type theory foundations (setoids, apartdness, \etc).

\end{itemize}

\paragraph{Induction vs deduction}

The greatest difference between this text and a mathematical text is the use of an inductive exposition rather than a deductive explanation.

Instead, here we first build up the examples as something that is interesting per se, and then we show how they can be all be instances of the same general concept. In this way, the general concept is well motivated.

\section{For instructors}

\subsection{Instructor copy}
If you are an instructor, please contact us to obtain a copy of the book with the solutions to the graded exercises.


