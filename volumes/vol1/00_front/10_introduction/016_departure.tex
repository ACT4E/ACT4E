% !TEX root = chapter-standalone.tex

\section[Departures from tradition]{Departures from traditional exposition [experts only]}

\textbf{This section is for experts only. Skip at a first reading.}

We made several choices to streamline the traditional exposition of category theory, to make it more understandable and relevant to the engineering field.

\paragraph{Notation and convention}
\begin{itemize}
    \item In general, we use diagrammatic notation $\mora \then \morb$ (pronounced ``$\mora$ then $\morb$'') rather than $\morb \after \mora$ (pronounced ``$\morb$ after $\mora$'') for function composition.
    \item In the discussion of semigroups, we use ``$\then$'' rather than ``$\circ$'' as the semigroup composition operation.
          This is because for us a semigroup/monoid is a special (semi)category with only one object.
    \item In \cref{chap:actions} we discuss \emph{covariant} and \emph{contravariant} actions.
          We do not use the terms \emph{left} and \emph{right} actions because they are notation-dependent.
    \item We abundantly use \emph{semi}categories (semifunctors, \etc).
          For us semicategory is the primitive definition.
          A category is a semicategory with a particular property: having identities at each object.
\end{itemize}

\paragraph{Use of colors}
\begin{itemize}
    \item We use colors to aid in the parsing of formulas and diagrams.
          However, color is not necessary to infer meaning.
          The choice of colors is colorblind-friendly.
          (One of the authors is colorblind.)
          Please let us know if this is not the case.
\end{itemize}

\paragraph{Materials covered}
\begin{itemize}
    \item Certain topics (limits, Yoneda's lemma, \etc) that would be traditionally discussed relatively soon, are not discussed in the first volumes.

    \item The text uses traditional set theory.
          To ground the exercises, we use slightly more formal type theory foundations (setoids, apartness, \etc).

\end{itemize}

\paragraph{Treatment of monoidal categories}

We noticed that there is a step increase in difficulty associated to natural transformations, without much immediate justification.
In our trajectory, natural transformations appear first associated to monoidal categories.
The role they play there is associated to very technical checks.
Nothing exciting!
We made the decision to provide a version of monoidal categories that are strict, so that there is no need for natural transformations.

We also noticed sever examples of interest (\eg proper LTI system) that are only semicategories, but they still have a notion of trace.

Furthermore, we noticed that in applications that there are several interesting examples that have a notion of vertical composition but the monoidal structure is not functorial on the nose: for example, systems with states.

Because there is a large part of concrete code exercises, it was not convenient for us to just wave our hands and say things like ``let's just consider the strict / modulo isomorphism version''.

In conclusion, in \cref{chap:parallelism} we introduce several notions of ``stacking'' categories, which are defined for semicategories, are strict in the vertical composition operation, and for which the functoriality of the monoidal structure is not a given.

We provide the traditional exposition of monoidal categories in \cref{chap:generalization} after the introduction of natural transformations in \cref{chap:naturality}.

\paragraph{Induction vs deduction}

The greatest difference between this text and a mathematical text is the use of an inductive exposition rather than a deductive explanation.
In a typical mathematical exposition of category theory, one defines a general mathematical structure, and then give several specific examples~\cite{riehl2017category}.
Instead, here we first build up the examples as something that is interesting per se, and then we show how they can be all be instances of the same general concept.
In this way, the general concept is well motivated.
The path laid by the book is one of \emph{spiral learning}.

For example, we look at various constructions from specific to general:
%
\begin{equation}
    \text{set product}  \to \text{poset product} \to \text{categorical product}.
\end{equation}
%
Similarly, we discuss
%
\begin{equation}
    \text{monoid morphisms}  \to \text{category actions} \to \text{functors}.
\end{equation}

