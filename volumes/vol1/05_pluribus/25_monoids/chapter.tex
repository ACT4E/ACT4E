% !TEX root = chapter-standalone.tex


\chaptersecond{\statusdraft{E pluribus unum}}{placeholder}{}{
  This chapter introduces three of the most basic algebra structures: semigroups, monoids, and groups.
}
\label{ch:monoids}

Oftentimes we are going to studies certain \emph{structures} and then their \emph{refinements}.
By \emph{refinement} we mean another type of structure that has additional properties/constraints.

The simplest structure that has to do with composition is that of a \emph{magma}: it just assumes that there is a set with a binary operation defined on it.

\begin{definition}[Magma]
  \label{def:magma}
  A \emph{\iindex{magma}}~$\sgrpA$ is a set~$\sgrpAset$, together with a binary operation
  \begin{equation}
    \mtimes  \colon \sgrpAset \cartprod \sgrpAset \sto \sgrpAset.
  \end{equation}
\end{definition}

Given a finite set $\setA$, one way to specify a composition operation~$\mtimes$ on $\setA$ is simply by writing out what it does with each pair of elements of~$\setA$.
Since~$\mtimes$ is a function of two variables, this can be conveniently displayed as a table, sometimes called a \emph{multiplication table} or a \emph{Cayley table}. We will use the name \emph{composition table}. Consider for example the set $\setA = \{ \ela, \elb \}$. A composition operation $\mtimes$ is specified in \cref{comp-table}. 
\begin{table}[h!]
  \begin{center}
    \caption{Composition table.}
    \label{tab:comp-table}
    \begin{tabular}{c|cc}
      $\mtimes$ & $\ela$ & $\elb$ \\
      \hline
      \ela & \ela & \ela \\            
      \elb & \elb & \ela \\ 
    \end{tabular}
  \end{center}
\end{table}
We read it as saying
$$
\ela \mtimes \ela = \ela, \quad \ela \mtimes \elb = \ela, \quad \elb \mtimes \ela = \elb,  \quad \elb \mtimes \ela = \ela.
$$

Magmas are quite general and simplistic. We can build more interesting structures by considering, for example, properties that the composition operation might have.


% !TEX root = chapter-standalone.tex

\section{\statusdraft{Semigroups}}\label{sec:semigroups}


\begin{definition}[Semigroup]
  \label{def:semigroup}
  A \emph{\iindex{semigroup}} is a set~$\sgrpA$, together with a binary operation~$\mtimes \colon \sgrpA \cartprod \sgrpA \sto \monoidA$, called \emph{multiplication}, which satisfies:
  \begin{enumerate}
    \item Associative law:~$(x\mtimes y)\mtimes z=x\mtimes (y\mtimes z)$ for all $x , y, z \in \sgrpA$.
  \end{enumerate}
\end{definition}

\begin{remark}
Given a fixed set $\sgrpA$, there will in general be many different choices of multiplication operation which make $\sgrpA$ into a semigroup. So, technically, a semigroup is a pair $\tup{\sgrpA, \mtimes}$ consisting of a set $\sgrpA$ and a choice of multiplication $\mtimes$. The set $\sgrpA$ is the \emph{underlying set} of the semigroup. Often we will be slightly imprecise and refer to a semigroup simply by the name of its underlying set; this is practical when it is clear from context which multiplication operation we are considering, or when it is not necessary to refer to the multiplication explicitly.
\end{remark}

\begin{example}\label{natnum-semigroup}
  Consider~$\tup{\natnumbers,+}$. This is a semigroup, since, for all~$l,m, n \in \natnumbers$, we have
  \begin{equation*}
  (l+m)
    +n=l+(m+n).
  \end{equation*}
\end{example}

\begin{example}\label{string-semigroup}
Consider a finite set $A$, which we think of as an alphabet. For instance, consider $A = \{ L, R \}$.
Let $\sgrpA$ be the set of non-empty strings of elements of $A$. For example,
$$ LLRLRRRL $$
is a non-empty string of elements of $A$.

We may define a multiplication operation on $\sgrpA$ simply by concatenating strings. Given the strings
$$ LLRLRRRL  \quad \text{ and } \quad RRLR, $$
their concatenation is the string
$$ LLRLRRRLRRLR.$$
It is readily seen that concatenation satisfies the associative law, so $\sgrpA$, together with this multiplication, forms a semigroup.
\end{example}





When the underlying set of a semigroup $\tup{\sgrpA, \mtimes}$ is a finite set, one way to specify the multiplication $\mtimes$ is simply by writing out what it does with each pair of elements of $\sgrpA$. Since $\mtimes$ is a function of two variable, this can be conveniently displayed as a table, called a \emph{multiplication table}.

\todotext{JL: write out (and include graphic) of a simply multiplication table}

\begin{exercise}
Consider the set $\sgrpA$ of finite non-empty strings of symbols from the alphabet $A$, as in \cref{string-semigroup}. Can you think of other candidates for multiplication operations on $\sgrpA$, besides the straightforward concatenation of strings considered above? Do your candidates define semigroup multiplications -- that is, do they obey the associative law?

For example, one might consider the operation where, given an ordered pair of strings, one first doubles the last symbol of the first string, and then concatenates. Is this operation associative?
\end{exercise}
\begin{solution}
  \todotext{to write}
\end{solution}


\begin{example}\label{max-semigroup}
The function $\max : \natnumbers \times \natnumbers \sto \natnumbers$ defines a multiplication operation which equips $\natnumbers$ with the structure of a semigroup.
\end{example}

\begin{exercise}
Verify the statement made in \cref{max-semigroup}; that is, check that the associative law holds.

Does $\min : \natnumbers \times \natnumbers \sto \natnumbers$ also define a semigroup structure on $\natnumbers$?
\end{exercise}
\begin{solution}
  \todotext{to write}
\end{solution}


\begin{example}\label{plant-trafo-semigroup}
Consider the set $X = \{ \text{sprout}, \text{young}, \text{mature}, \text{old}, \text{dead} \}$ which describes five possible states of a plant. Let $T : X \sto X$ be the function that describes ``development'':
\begin{align*}
T(\text{sprout}) &=  \text{young} \\
T(\text{young}) &=  \text{mature} \\
T(\text{mature}) &=  \text{old} \\
T( \text{old}) &= \text{dead} \\
T (\text{dead}) &= \text{dead}
\end{align*}

In other words, we think of $T$ as the change of state of the plant during a given time interval (say, three months). Composing the function $T$ with itself corresponds to considering multiples of the given time interval. For example, the function
$$
T \then T \then T : X \sto X
$$
models the change over the course of nine months. In general, for the n-fold composition of $T$ with itself we write $T^n$. The set $\sgrpA = \{ T^n \mid n \in \natnumbers \}$ is a semigroup, with the multiplication given by the composition operation.

\end{example}

\begin{definition}[Semigroup homomorphism]
  \label{def:semigroup-mor}
Let $\sgrpA$ and $\sgrpB$ be semigroups. A homomorphism of semigroups from $\sgrpA$ to $\sgrpB$ is a function $\sgrpmorA: \sgrpA \sto \sgrpB$ such that
\begin{equation}\label{sgrp-mor-comp}
\sgrpmorA(\sgrpelAa \mtimes_{\sgrpA} \sgrpelAb) = \sgrpmorA (\sgrpelAa) \mtimes_{\sgrpB} \sgrpmorA(\sgrpelAb) \quad \quad \forall \sgrpelAa, \sgrpelAb \in \sgrpA
\end{equation}
We think of the equation \cref{sgrp-mor-comp} as a way of saying that the function of sets $\sgrpmorA: \sgrpA \sto \sgrpB$  is \emph{compatible} with the multiplication operations on $\sgrpA$ and $\sgrpB$, respectively.
\end{definition}


\begin{definition}[Identity homomorphism]
  \label{def:identity-sgrp-mor}
Let $\sgrpA$ be a semigroup. The identity function $\id_\sgrpA: \sgrpA \sto \sgrpA$ is always a morphism of semigroups. Indeed, the condition
\begin{equation}
\id (\sgrpelAa \mtimes_{\sgrpA} \sgrpelAb) = \id (\sgrpelAa) \mtimes_{\sgrpA} \id(\sgrpelAb) \quad \quad \forall \sgrpelAa, \sgrpelAb \in \sgrpA
\end{equation}
is satisfied. We call this the \emph{identity homomorphism} of $\sgrpA$.
\end{definition}

\begin{definition}[Semigroup isomorphism]
  \label{def:semigroup-iso}
Let $\sgrpA$ and $\sgrpB$ be semigroups. A homomorphism of semigroups $\sgrpmorA: \sgrpA \mto \sgrpB$ is called a \emph{semigroup isomorphism} if there exists a homomorphism of semigroups $\sgrpmorB: \sgrpB \mto \sgrpA$ such that
\begin{equation}\label{sgrp-iso-cond}
\sgrpmorA \then \sgrpmorB = \id_\sgrpA \quad \text{ and } \quad  \sgrpmorB \then \sgrpmorA = \id_\sgrpB.
\end{equation}
\end{definition}

\begin{exercise}
How many different non-isomorphic semigroups are there with precisely one element? How many with precisely two elements?
\end{exercise}
\begin{solution}
  \todotext{to write}
\end{solution}

\begin{exercise}
Let $\sgrpmorA: \sgrpA \mto \sgrpB$ be a morphism of semigroups. Prove that $\sgrpmorA$ is an isomorphism of semigroups if and only if the function $\sgrpmorA: \sgrpA \sto \sgrpB$ is bijective.
\end{exercise}
\begin{solution}
  \todotext{to write}
\end{solution}


Given a semigroup $\tup{S, \mtimes}$, for each $n \in \natnumbers$, we can define an induced n-ary multiplication operation
$$
S^n \sto S, \ \tup{s_1, s_2, ..., s_n} \mapsto s_1 \mtimes s_2 \dots \mtimes s_n.
$$
Thanks to the associative law, this is well-defined -- that is, we do not need to set parenthases. We'll say that an element $s \in S$ is an \emph{n-fold multiplication} if it is in the image of this n-ary multiplication operation. At times we may not wish to specify the arity of the multiplication, in which case we just speak of a \emph{multiplication}. 

\devel{\subsection{Code exercises}

\begin{remark}
  In this course we assume, unles otherwise noted, that you have done the previous exercises, so that you can build on top of the code that you already have. The first exercise and tutorial was in \cref{sec:exercise-tutorial}.
\end{remark}

\subsubsection*{Interface}

\classsource{FiniteSemigroup}{}

\subsubsection*{Representation}

The format is shown in \cref{fig:sm1}.

\datafilefig{sm1}{sm1.semigroup.yaml}{fig:sm1}


\begin{gradedexercise}[\exname{TestFiniteSemigroupRepresentation}]
  \label{ex:TestFiniteSemigroupRepresentation}
  Create a function to load the data.


  \classsource{FiniteSemigroupRepresentation}{}

\end{gradedexercise}

\subsubsection*{Constructors}

\begin{gradedexercise}
  \label{ex:FiniteSemigroupRepresentation}
  Given a finite set, construct the free semigroup.

  \methodsource{FiniteSemigroupConstruct}{free}{}
\end{gradedexercise}
}
% !TEX root = chapter-standalone.tex


\section{\statusdraft{Monoids}}
\label{sec:parallelism-monoids}


Algebraic structures are often defined in \emph{layers}. For example, in the definition of semigroup, we start with a set $\sgrpA$ as a basic building block, and we add a layer of structure to it, namely a multiplication operation $\mtimes \colon \sgrpA \cartprod \sgrpA \sto \monoidA$. The multiplication operation for semigroups was not only a new \emph{structure} that we added, but we also required that this structure obey a \emph{condition}, namely that it satisfy the associative law. One might also say that the multiplication operation was a new \emph{constituent} or a new \emph{datum}, and that satisfying the associative law is a \emph{property}. Mathematicians often use such words in an intuitive, non-rigorous way as a tool for structuring their thinking. We will do the same. For clarity, we will aim to stick with the words \emph{constituents} and \emph{conditions}. Roughly speaking, we think of constituents as building blocks, and we think of conditions as rules for how those blocks fit together and behave.

Using the constituent vs. condition distinction we will, in particular, present some definitions in the following succinct, list-like fashion:

\begin{definition}[Monoid]
  \label{def:monoid}
  A \emph{\iindex{monoid}} is:
  \begin{quote}
    \constit
    \begin{enumerate}
      \item a set~$\monoidA$;
      \item a binary operation~$\mtimes \colon \monoidA \cartprod \monoidA \sto \monoidA$, called \emph{multiplication};
      \item a specified element $\idmon\in \monoidA$, called \emph{neutral element}.
    \end{enumerate}
    \condit
    \begin{enumerate}
      \item Associative law:~$(x\mtimes y)\mtimes z=x\mtimes (y\mtimes z)$;
      \item Neutrality Laws:~$\idmon\mtimes x=x=x\mtimes \idmon$.
    \end{enumerate}
  \end{quote}
\end{definition}


\begin{remark}
  The way that we presented the definition of a monoid is certainly not unique. For example, we could have done the following.

  \begin{quote}
    A \emph{\iindex{monoid}} is:
    \begin{quote}
      \constit
      \begin{enumerate}
        \item a semigroup~$\tup{\monoidA, \mtimes}$;
        \item a specified element $\idmon\in \monoidA$, called \emph{neutral element}.
      \end{enumerate}
      \condit
      \begin{enumerate}
        \item Neutrality Laws:~$\idmon\mtimes x=x=x\mtimes \idmon$.
      \end{enumerate}
    \end{quote}
  \end{quote}
  In this version, two constituents and one condition from \cref{def:monoid} are ``compressed'' into the information that we are using here a semigroup as a constituent. This kind of ``compression'' has its pros and cons; depending on the context will use it to varing degrees.

  There is a similar dilemma when considering the software interfaces to describe these structure.
  In terms of software engineering, the two strategies are \emph{composition} (a monoid has a semigroup as a constituent)
  and \emph{inheritance} (a monoid \emph{is} a semigroup with additional data). 

\end{remark}




\begin{example}
  Consider~$\tup{\reals,+,0}$. This is a monoid, since, for all~$x,y, z \in \reals$, we have
  \begin{equation*}
  (x+y)
    +z=x+(y+z),
  \end{equation*}
  and
  \begin{equation*}
    x+0=x=0+x.
  \end{equation*}
\end{example}

\begin{example}
  The set~$\wnumbers$, together with the operation of multiplication of whole numbers, forms a monoid. The neutral element is the number $1$.
\end{example}

\begin{lemma}
  \label{neut-el-unique}
  Let $\tup{\sgrpA, \mtimes}$ be a semigroup. If there exists elements $1 \in M$ and $1' \in M$ such that $\tup{\sgrpA, \mtimes, 1}$ and $\tup{\sgrpA, \mtimes, 1'}$ are each monoids, then $1 = 1'$ must hold. In other words, the neutral element of a monoid is uniquely determined by the underlying semigroup structure.
\end{lemma}

\begin{exercise}
  Prove \cref{neut-el-unique}.
\end{exercise}

\begin{definition}[Monoid homomorphism]
  \label{def:monoid-mor}
  Let $\tup{\monoidA, \mtimes_\monoidA, 1_\monoidA}$ and $\tup{\monoidB, \mtimes_\monoidB, 1_\monoidB}$ be monoids. A homomorphism of monoids from $\monoidA$ to $\monoidB$ is a function $\mapa : \monoidA \sto \monoidB$ such that
  \begin{equation}
    \label{mon-mor-comp}
    \mapa (\monelAa \mtimes_{\monoidA} \monelAb) = \mapa (\monelAa) \mtimes_{\monoidB}  \mapa(\monelAb) \quad \quad \forall \monelAa, \monelAb \in \monoidA
  \end{equation}
  and
  \begin{equation}
    \label{mon-id-comp}
    \mapa (1_\monoidA) = 1_{\monoidB}
  \end{equation}
\end{definition}

\begin{example}
  The set~$\monoidA = \{ -1, 0, 1 \}$, together with multiplication of whole numbers and with $1$ as neutral element, forms a monoid. The inclusion map $ \monoidA \sto \wnumbers$ is a homomorphism of monoids.
\end{example}

\todo{Nice example: the map from sequence of characters to sequences
of sounds is monoidal in certain languages (Korean, Japanese, almost in Italian.) and
also invertible.}


\begin{definition}[Identity homomorphism]
  \label{def:identity-mon-mor}
  Let $\monoidA$ be a monoid. Similar to the case of semigroups, the identity function $\id_\monoidA: \monoidA \sto \monoidA$ is also a homomorphisms of monoids.
\end{definition}



\begin{definition}[Monoid isomorphism]
  \label{def:monoid-iso}
  A homomorphism of semigroups $\mora: \monoidA \mto \monoidB$ is called a \emph{monoid isomorphism} if there exists a homomorphism of monoids $\morb: \monoidB \sto \monoidA$ such that
  \begin{equation}
    \label{sgrp-iso-cond}
    \mora \then \morb = \id_\monoidA \quad \text{ and } \quad  \morb \then \mora = \id_\monoidB.
  \end{equation}
\end{definition}


\begin{exercise}
  Prove: a morphism of monoids $\mora: \monoidA \mto \monoidB$ is an isomorphism of monoids if and only if the function $\mora: \sgrpA \sto \sgrpB$ is bijective.
\end{exercise}

\

\

\



\

\

\


\todotext{JL: I would change the example below; it feels slightly misleading to use ``0''. This construction works for any closed ray open toward + $\infty$}

\begin{example}
  Consider~$\tup{\nonNegReals,\max,0}$. This is a monoid, since, for all~$x,y\in \nonNegReals$, we have:
  \begin{equation*}
    \max(\max(x,y),z)=\max(x,\max(y,z)),
  \end{equation*}
  and
  \begin{equation*}
    \max(x,0)=x=\max(0,x).
  \end{equation*}
\end{example}



\

\

\



\

\

\


\todotext{JL: We should edit / change the example below (for coherence with the string example for semigroups, and to change the definition of sequence)}
\begin{example}[Sequences]
  A sequence is a function whose domain is a subset of~$\natnumbers$ \JL{this is a strange and non-standard definition of sequence; I would avoid}, and are called \emph{finite} if the domain of the function is finite. Often finite sequences are referred to as \emph{lists}. Given a set~$S$, we denote the set of all lists on~$S$ by $S^\ast$. This can be made into a monoid, by considering \emph{concatenation} as the operation, and the empty list as the neutral element. Specifically, a list is an element $s\in S^\ast$ consists of a $n\in \natnumbers$ and a function~$f\colon [n]\to S$, where~$[n]=\{i\colon \natnumbers\mid i<n\}\subseteq \natnumbers$. The empty list, denoted~$()$, is the unique list of length 0. Given~$n>0$, we can write the list which assigns~$0,\ldots,n-1$ to~$s_0,s_1,\ldots,s_{n-1}$ as $(s_0,s_1,\ldots,s_{n-1})$. Given a list~$x\in S^\ast$ of length~$m$ and a list~$y\in S^\ast$ of length~$n$, we can define their \emph{concatenation}~$x*y$ as list of length~$m+n$ with:
  \begin{equation*}
    i\mapsto
    \begin{cases}
      x_i&\text{if }i<m\\
      y_{i-m}&\text{if }i\geq m.
    \end{cases}
  \end{equation*}
  Clearly, this definition of concatenation satisfies associativity and unitality, making this construction a monoid. This is often referred to as the \emph{free monoid on~$S$}.
\end{example}


\todostructure{JL: The example below needs to be moved}
\begin{example}
  Given any category~\CatC, and any object~$\Obja\in \CatC$, the set of \emph{endomorphisms}~$\Hom_{\CatC}(\Obja,\Obja)$ is a monoid. The category depicted in \cref{fig:monoid_endomorphisms} has three objects~$\Obja,\Objb,\Objc$ and several morphisms.~$\Obja$ has four endomorphisms,~$\Objb$ two, and~$\Objc$ three (including identity morphisms). Let's now take the binary operation~$\mtimes$ to be the composition~$\then$ in~\CatC, and the neutral element to be the identity~$\catid_\Obja$. The associativity and unitality laws of the category~\CatC coincide with the ones of the monoid's definition, and are satisfied. Therefore, we can identity a monoid as a one-object category.
\end{example}

\begin{figure}[h!]
  \begin{center}
    \includesag{043_monoid_endomorphisms}
    \caption{}
    \label{fig:monoid_endomorphisms}
  \end{center}
\end{figure}


%\section{Dynamical systems and monoids}

\AC{in the end I would make this only a simple example of monoid - no introduction of group etc.}
\JL{inserting this here as an un-baked idea for a subsection. maybe it could be the first subsection of this chapter; that way idendity laws and associative laws can be introduced before talking about categories}
\gray{
  What are the simplest kinds of mathematical models of a dynamical system that we can think of?

  One possible answer is something like this: we can describe a dynamical system as a set $S$ of possible states, together with a description of how states change over time. For the latter, consider time to be labeled by distinct ``points in time''. Then, we can just think in terms of time-steps, \eg  seconds, or we can think of points in time where \eg  an action is triggered and the system passes to a new state.

  One thing we want to describe is how the state of our system changes over time, and in particular from one moment in time to the next. For any time step, we will not assume that we know what specific state the system is in, but rather we will describe, at once, all possible evolutions during that time step, \ie  we consider all possible initial conditions at once. Given two consecutive moments in time, we might describe the possible changes in the system by a function $T : S \rightarrow S$, which maps each state $s \in S$ to a next state $T(s) \in S$. This is a deterministic change of state: given $s$, the function $T$ determines the next state $T(s)$. The function $T$ is like a rule. Let's call $T$ an ``evolution operator'', because it describes how the system states might evolve over a time step.

  We might want to consider various possible evolution operators. We could consider functions $T_a$, $T_b$, $T_c$, \etc. We can also compose these functions: given $T_a$ and $T_b$, we might have, over the course of two time steps, the change described by $T_a \circ T_b$. For simplicity, let's suppose we work with three evolution operations $T_a$, $T_b$, and $T_c$.

  \

  -> introduce semigroups (implicitly or explicitly)

  \


  -> introduce monoids
}

\AC{
  For me the basic example of monoid with dynamical systems is taking the transition function.

  Let $E^s_t: X \to X$ be the evolution function from $s$ to $t$. Then states evolve like this: $x_t = E^s_t (x_s)$.

  If you assume that the system is time invariant, then the evolution only depends on the difference $\delta = s-t$.
  You have now a communative monoid of transition functions $T_\delta$ where $T_0 = \text{identity}$.

  (No need to do semigroups.)

}

\devel{% !TEX root = chapter-standalone.tex
\subsection{Code exercises}

\subsubsection*{Interface}

\classsource{Monoid}{}

\classsource{FiniteMonoid}{}

\subsubsection*{Representation}


\margindatafilefig{mon_min5}{Simple monoid}{fig:mon_min5}

The format is shown in \cref{fig:mon_min5}.



\begin{gradedexercise}[Representation]
  Create a function to load the data.
%
  \classsource{FiniteMonoidRepresentation}{}

\end{gradedexercise}

}
\devel{
% !TEX root = chapter-standalone.tex


\section{Rope Goldberg Machines}

The inventions of Professor Butts, transcribed by Rube Goldberg, are prodigious machines that provide practical ways to perform everyday tasks (\cref{fig:Rube3}). They also highlight the power of compositionality.

\begin{figure}[h]
  \includegraphics[width=\textwidth]{Rube3}
  \caption{\emph{Life’s Little Jokes \#59,380}, Rube Goldberg}
  \label{fig:Rube3}
\end{figure}



We are going to assemble similar machines. We start by introducing some of the essential elements:

\begin{itemize}
  \item Rope
  \item Elastic
  \item Brick
  \item Glass
  \item Spring
  \item Rubber
\end{itemize}

\todographics{work on original icons}

\begin{center}
\begin{tabular}{ccc}
  \rope{ropecola}{ropecolb}{2}{0}{0}{1}{}&
  \includegraphics[width=2cm, height=2cm]{elastic_rope}&
  \includegraphics[width=2cm, height=2cm]{brick}\\
  rope&elastic&brick\\
  \includegraphics[width=2cm, height=2cm]{empty_glass}&
  \includegraphics[width=2cm, height=2cm]{spring}&
  \includegraphics[width=2cm, height=2cm]{rubber}\\
  glass&spring&rubber
\end{tabular}
\end{center}

These different elements can be classified by their elasticity properties, and hence by the way in which they deform.
Given any material, it reacs to the application of specific forces. The force applied per unit area is referred to as \emph{stress}, and the stretching/compression produced as a response from the material is called \emph{strain}.
The strain~$\strain$ is usually written as the ratio of the difference in length along the direction of the stress~$\Delta L$, and the original length of the material~$L_0$:
\begin{equation*}
  \strain = \frac{\Delta L}{L_0}
\end{equation*}

When stress is applied, each material experiences a specific strain, which depends on the chemical bonds creating the material. Whether the material returns to its original shape when stress is removed, depends on the entity of deformation.
One usually differentiates between \emph{elastic} deformation (when removing the stress makes the material return to its original shape), and \emph{plastic} deformation (when the material deforms irreversibly): see \cref{fig:stress_strain}.
If one observes the stress vs. strain curve, the first region behaves linearly, meaning that the force required to deform an elastic material is directly proportional to its deformation.
This is commonly known as the Hooke's law, which is due to the physicist Robert Hooke:
\begin{equation*}
  \force=-\springconst \deformation,
\end{equation*}
where~$\force$ represents the force,~$\deformation$ the deformation of the material, and~$\springconst$ the so-called \emph{spring constant} (in \unit[]{[N/m]}).

\begin{figure}[h]
\begin{center}
\includegraphics[width=10cm]{strain_stress}
\end{center}
  \caption{\todo{do our one}. \label{fig:stress_strain}}
\end{figure}

\begin{example}
  You are skiing in the beautiful Swiss alps. Your friend weighs \unit[100]{kg} and sits on a chairlift with spring constant \unit[4,000]{N/m}. The nominal extension of the chairlift is \unit[0.25]{m}. The weight of your friend will create an extension of the chairlift, which can be computed as:
  \begin{equation*}
  \begin{aligned}
    \deformation &=\frac{\force}{\springconst}\\
    &=\frac{mg}{\springconst}\\
    &\approx \unit[0.15]{m}.
  \end{aligned}
  \end{equation*}
  One can therefore compute the extension as~$L=L_0-\deformation=\unit[0.1]{m}$.
\end{example}

The ability of a material to being deformed elastically is described by the so-called Young's \emph{modulus}, name of which is due to the physicist Thomas Young. This modulus, usually named~$\youngmod$ is defined as
\begin{equation*}
  \youngmod=\frac{\text{stress}}{\text{strain}}.
\end{equation*}
Note that~$\youngmod$~is a constant if computed in the domain of validity of Hooke's law. In this domain, we can obtain the spring constant~$\springconst$ by using the formula:
\begin{equation*}
  \springconst=\youngmod \frac{A}{L_0},
\end{equation*}
where~$L_0$ is the nominal length of the material, and~$A$ is the area over which the force is applied.

It is interesting to see what happens when one combines multiple materials. The first two elastic material combinations which come to mind are \emph{series} and \emph{parallel}. If you take two materials with the same spring constant~$\springconst$, you concatenate them, and you hang a weight at their lower end (\cref{fig:series_spring}), they will be equivalent to one single spring with double the length.
The effective spring constant of the combination must therefore be halved to~$\springconst/2$.
In case one puts the two materials in parallel (\cref{fig:parallel}), the length remains the same, and the resulting spring constant doubles to~$2\springconst$.

\todo{This is the exact same reasoning for capacitors in electrical circuits (isomorphic example for ITET people:) )}
\begin{tabular}{cc}
\includesag{30_spring_series}&\includesag{30_spring_parallel}
\end{tabular}

\todotext{Describe here the properties of these components (elasticity, etc.)}


These are the classes that define the data.

\classsource{Component}{max_size=25}
\classsource{Rope}{max_size=25}
\classsource{Brick}{max_size=25}
\classsource{Elastic}{max_size=25}

\begin{gradedexercise}[\exname{TestRopeGoldberg}]
  In this exercise you will compute the behavior of a chain of those components.

  The interface you need to implement is:

  \classsource{RupeGoldbergSolver}{}


  \begin{enumerate}
    \item The first function asks what would happen if you were to hang this chain on one end, and let it hang (\cref{fig:hang}).
    Would it break? If not, what would be the extended length?
   \GZ{The ```would it break'' question is not as smooth as it seems. The strain-stress diagrams have nonlinearities and maybe we end up in oversimplifying.}
    \item The second question asks what would happen if you were to let the chain rest horizontally (\cref{fig:push}). Then fix one end to an \emph{immovable wall} and push to the other end. What would happen? If the push destroys some of the objects, consider
    them destroyed and of length 0.
    \item  The third question asks what would happen if you were to pull with a certain force  (\cref{fig:pull}). Either you will break the chain, or there will be an equilibrium. What would be the length?
  \end{enumerate}

\end{gradedexercise}

\begin{figure}
  \hfill
  \subfloat[\label{fig:hang}Hanging the chain]{
    \includegraphics[width=3cm, height=4cm]{placeholder}
  }
  \hfill
  \subfloat[\label{fig:push}Horizontal compression]{
  \includegraphics[width=3cm, height=4cm]{placeholder}
  }
  \hfill
  \subfloat[\label{fig:pull}Horizontal extension]{
  \includegraphics[width=3cm, height=4cm]{placeholder}
  }
  \hfill
  \caption{Visualization of the three scenarios.
  \todographics{Make graphics. }}
\end{figure}


We will walk you through the theoretical part, but we will leave the implementation to you.

We should think \emph{compositionally}. We should first see what happens in the specific cases,
and see if the answer can be generalized in a way that the solution is compositional.

\todotext{Go through the specific cases}


\todotext{Give breakthrough observation that all the behavior can be described by one relation between force and length. Hence what we are talking about is really composition in a certain monoid.}

\todotext{The solution is to create for each component this relation, and then compose the relations.}
}
% !TEX root = chapter-standalone.tex

\section{Generators and relations}

\todotext{\bernina: @JL: add a sentence or two to intro this section.}

\subsection{Generating subsets}

In \cref{exa:plant-trafo-semigroup} we considered a set of states
%
\begin{equation}
    \setA= \makeset{ \sprout, \yng, \mature, \old, \dead },
\end{equation}
%
a function~$\mapa \colon \setA \sto \setA$, and the semigroup
%
\begin{equation}
    \label{eq:trafo-sgrp-one-gen}
    \sgrpA = \makeset{ \mapa^n \mid n \setin \natnumbers }.
\end{equation}
%
Note that~$\sgrpA$ has a special form: all of its elements can be expressed in terms one of its elements,~$\mapa$, and the multiplication operation (which, in this case, is function composition).
To describe this state of affairs we say that~$\sgrpA$ is \emph{generated} by the element~$\mapa$.

\begin{ctdefinition}[Generating subsets]
    \label{def:gen-semigrp}
    Let~$\sgrpA = \tup{\sgrpAset, \mtimes}$ be a  \SY{semigroup}, and let~$\setA \setsubseteq \sgrpA$ be a subset.
    We say that~$\sgrpA$ is \emph{generated} by~\setA if every element of~$\sgrpA$ can be expressed as a finite composition of elements of~\setA.
\end{ctdefinition}

\begin{remark}
    Mutatis mutandis, the same definition also holds for \SY{monoids}.
    For \SY{groups}, we say \setA generates the group if every element of the group can be expressed as a finite composition of elements of \setA or their inverses.
\end{remark}

\begin{example}
    Consider \cref{exa:string-semigroup}, where elements of the \SY{semigroup}~$\sgrpA$ were non-empty lists built using the elements of the ``alphabet'' set~$\setA = \makeset{ \alphabeta, \alphabetb}$.
    In this case,~$\sgrpA$ is generated by~\setA.
\end{example}

\begin{example}
    Consider the natural numbers (without zero) as a   \SY{semigroup}, where addition is the \SY{semigroup} composition operation (see \cref{exa:natnum-semigroup}).
    This \SY{semigroup} is generated by the subset~$\makeset{1 }$.
\end{example}

\subsection{Relations}

Let us return to the \SY{semigroup} \cref{eq:trafo-sgrp-one-gen}.
Recall that~$\mapa$ was defined by
%
\begin{align}
    \mapa(\sprout) & =  \yng, \\
    \mapa(\yng)    & =  \mature, \\
    \mapa(\mature) & =  \old, \\
    \mapa( \old)   & = \dead, \\
    \mapa(\dead)   & = \dead.
\end{align}

Observe that the function~$\mapa^4$ will map all elements of~\setA to the element ``\dead''.
For example, if we start with the element ``\sprout'', the result of applying~$\mapa$ four times is
%
\begin{equation}
    \sprout \overset{\mapa}{\sto} \yng \overset{\mapa}{\sto} \mature \overset{\mapa}{\sto} \old \overset{\mapa}{\sto} \dead.
\end{equation}

Note also that for \emph{any}~$n \geq 4$, the function~$\mapa^n$ will map all elements of~\setA to the element ``\dead''.
If we consider~$\mapa^6$, for example, then, for any~$\ela \setin \setA$,
%
\begin{equation}
    \mapa^6(\ela) = \mapa^2(\mapa^4(\ela)) = \mapa^2(\dead) = \mapa(\mapa(\dead)) = \mapa(\dead) = \dead.
\end{equation}
%
It follows that all~$\mapa^n$, for~$n \geq 4$, are actually \emph{all the same map}: the one that sends every state to the dead state.
Thus, ~$\sgrpA = \makeset{ \mapa^n \mid n \setin \natnumbers }$ actually only has at most \emph{four} elements: $\mapa$,~$\mapa^2$,~$\mapa^3$, and~$\mapa^4$.

\begin{gradedexercise}[\exname{CheckRelations}]
    \label{ex:CheckRelations}
    Are any of the four maps~$\mapa$,~$\mapa^2$,~$\mapa^3$, and~$\mapa^4$ actually equal?
    Justify your answer by argumentation or by explicitly checking via calculation.
\end{gradedexercise}
\solutionof{CheckRelations}

When two elements which a priori could be distinct from each other (such as~$\mapa^6$ and~$\mapa^4$ above, for example) turn out to be equal, we call this a \emph{relation} between the elements of~$\sgrpA$.

\begin{ctdefinition}\label{def:relation-on-semigroup}
    \SYNDEF{relation on a semigroup}
    A \emph{relation} on a \SY{semigroup}~$\tup{\sgrpA, \mtimes}$ is an equation between compositions of elements of~$\sgrpA$.
\end{ctdefinition}

\begin{remark}
    Again, we have analogous definitions for \SY{monoids} and \SY{groups}.
    In these cases, we interpret the \SY{neutral element} $\idgrp$ as a ``zero-fold'' multiplication, so it can also be part of equations that express relations.
\end{remark}

\begin{remark}
    This is not the same notion as that of a (binary) relation, which was the topic of \cref{chap:relation} and takes up a much more important role in this book than the notion that we are discussing here.
\end{remark}

\begin{example}
    For the \SY{semigroup} \cref{eq:trafo-sgrp-one-gen}, the relations $\mapa^5 = \mapa^4$, $\mapa^6 = \mapa^5$, and $\mapa^6 = \mapa^4$, \etc are satisfied.
    However, the relation $\mapa^3 = \mapa$ is not satisfied, for example.

\end{example}

\begin{example}
    Consider the \SY{semigroup}~$\tup{\natnumbers, +}$.
    The equation~$l + k = k + l$ is an example of a relation that holds for all $l, k \setin \natnumbers$.
\end{example}

\begin{example}
    Consider the group $\grpA$ discussed in \cref{grp-order-three}, where
    %
    \begin{equation}
        \grpAset = \makesetBig{ 1, e^{\frac{1}{3}2\pi i}, e^{\frac{2 }{3}2\pi i} } \setsubseteq \cnumbers
    \end{equation}
    and the composition operation is multiplication of complex numbers.
    The element $\ela \definedas e^{\frac{1}{3}2\pi i}$ satisfies the relation $\ela^3 = \idgrp_\grpA$.
\end{example}

\begin{example}
    Consider the group $\grpA$ given in \cref{exa:grp-Klein4} which describe symmetries of a rectangle.
    We had
    \begin{equation}
        \grpAset = \makeset{ I, V, H, R}
    \end{equation}
    where $I = \idgrp$ corresponds to ``doing nothing'', $V$ is reflecting the rectangle along its long axis, $H$ is reflecting on the short axis, and $H$ is rotation by 180 degrees.

    In this group, the relations $V^2 = I$, $H^2 = I$, $H^2 = I$ are satisfied, for example.
\end{example}

\subsection{Freeness}

When, in \cref{def:gen-semigrp} we spoke about the \SY{semigroup}~$\sgrpA = \tup{\sgrpAset, \mtimes}$  being generated by a subset~$\setA \setsubseteq \sgrpAset$, we supposed that we already had a \SY{semigroup}~$\sgrpA$ to work with.
However, if we start with just a set, say~$\setA = \makeset{ \alphabeta, \alphabetb}$, then we saw in \cref{string-sgrp} that we can build a \SY{semigroup} from this set by considering lists of elements of~\setA, with concatenation as the composition operation.
The resulting \SY{semigroup} in that example has a special characteristic: its elements do not satisfy any relations other than the ones that are required by the definition of a   \SY{semigroup}, namely those relations dictated by the \SY{associative law}.
Such a \SY{semigroup} is called \emph{free}.
If we think of relations as ``constraints'' (they are equations) between the elements of a  \SY{semigroup}, then \SY{free semigroups} are ``free of constraints''.

For a given set, say~$\setA = \makeset{ \alphabeta, \alphabetb}$, there will in general be different ways of formally building a \SY{free semigroup} from it.
For instance, instead of considering lists of elements of~\setA
\begin{equation}
    \makelist{\alphabeta,\alphabeta, \alphabetb, \alphabeta, \alphabetb, \alphabetb, \alphabetb, \alphabeta},
\end{equation}
we could instead consider strings of elements
\begin{equation}
    \alphabeta\alphabeta \alphabetb \alphabeta \alphabetb \alphabetb \alphabetb \alphabeta,
\end{equation}
or tuples of elements
\begin{equation}
    \tup{\alphabeta, \alphabeta,  \alphabetb,  \alphabeta,  \alphabetb,  \alphabetb,  \alphabetb,  \alphabeta},
\end{equation}
both of which could also be composed in a way which is analogous to concatenation.

A common feature of all three of these formalizations is that we are writing a finite sequence of elements of~\setA, keeping account of the ordering.
Both approaches will in fact build a  \SY{semigroup} from the set~\setA which is \emph{free}.
And in both cases there is a natural way of seeing~\setA as \emph{generating} the resulting \SY{semigroup}.
However, the two set-ups are \emph{formally} distinct because we are using a different way of writing things down with symbols.
We will see in a later chapter that the resulting \SY{semigroups} are essentially ``the same'' (they are isomorphic) and their ``freeness'' can be given an elegant characterization.

Because in fact \emph{all} \SY{free semigroups} constructed from~\setA are ``the same'', independent of the formal symbolic specifics of how they are constructed, we refer to them all as \emph{the} free \SY{semigroup} generated by~\setA.
Furthermore, if we were to work with a set~$\setB = \makeset{ \setAel, \setBel }$ instead of~$\setA = \makeset{ \alphabeta, \alphabetb}$, and generate a free \SY{semigroup} from~\setB, then this would also produce a \SY{semigroup} which is ``the same''.
Therefore, sometimes one speaks simply of ``the free group on two generators'', in view of the fact that the essential feature is that both~\setA and~\setB have two elements.

Note that although~\setA and~\setB each only have two elements, the \SY{free semigroups} that they generate will have infinitely many elements.
Indeed, there are infinitely many lists that we can build from the elements of~\setA.
As we concatenate lists, the resulting compositions grow longer and longer, and there are no relations which would allow us to ``simplify'' a string to one which is shorter.



\devel{\sectionexercises{Morphisms}

We define an interface \classname{SemigroupMorphism} as follows.
The methods \funcname{source} and \funcname{target} give the two semigroups linked by the morphism.
The method \funcname{mapping} returns the mapping between the sets.
This mapping is assumed to satisfy the conditions of a semigroup morphism.

\classlisting{SemigroupMorphism}

The derived interfaces \classname{MonoidMorphism} and \classname{GroupMorphism} are defined in the obvious way.

\classlisting{MonoidMorphism}
\classlisting{GroupMorphism}

We also define the corresponding finite versions as \classname{FiniteSemigroupMorphism}, \classname{FiniteMonoidMorphism}, \classname{FiniteGroupMorphism}.

\todotextjira{290}{@Andrea: exercises morphisms}

\subsection{Representation}

The concrete representation for morphisms is
\begin{minted}{yaml}
source: ...
target: ...
mapping: ...
\end{minted}
where \fieldname{source} and \fieldname{target} are representations of semigroups/monoid/groups,
and \fieldname{mapping} is the representation of a \classname{FiniteMap}.

The following are three very similar exercises for loading the three types of exercises.

\codeboilerplate{FiniteSemigroupMorphismRepresentation}{

}

\begin{widepar}
    \aligninner{%
        \begin{minipage}{16cm}
            \classlisting{FiniteSemigroupMorphismRepresentation}
        \end{minipage}
    }
\end{widepar}

\codeboilerplate{FiniteMonoidMorphismRepresentation}{

}

\begin{widepar}
    \aligninner{%
        \begin{minipage}{16cm}
            \classlisting{FiniteMonoidMorphismRepresentation}
        \end{minipage}
    }
\end{widepar}

\codeboilerplate{FiniteGroupMorphismRepresentation}{

}

\begin{widepar}
    \aligninner{%
        \begin{minipage}{16cm}
            \classlisting{FiniteGroupMorphismRepresentation}
        \end{minipage}
    }
\end{widepar}

\clearpage
\subsection{Checking properties}

\codeboilerplate{FiniteSemigroupMorphismsChecks}{

}

\begin{widepar}
    \aligninner{%
        \begin{minipage}{17cm}
            \classlisting{FiniteSemigroupMorphismsChecks}
        \end{minipage}
    }
\end{widepar}

\clearpage
\subsection{Composition}

\codeboilerplate{FiniteSemigroupMorphismsOperations}{

}

\begin{widepar}
    \aligninner{%
        \begin{minipage}{17cm}
            \classlisting{FiniteSemigroupMorphismsOperations}
        \end{minipage}
    }
\end{widepar}}
% !TEX root = chapter-standalone.tex
\section{\statusdraft{Groups}}\label{sec:groups}
\begin{ctdefinition}[Group]
  \label{def:group}
  A \emph{\iindex{group}} is a monoid together with an ``inverse'' operation.
  In more detail, a group $\grpA$ is
  \begin{quote}
    \constit
    \begin{enumerate}
      \item a set~$\grpAset$;
      \item a binary operation~$\gtimes \colon \grpAset \cartprod \grpAset \sto \grpAset$, called \emph{composition};
      \item a specified element~$\idgrp \in \grpAset$;
      \item a map $\ginv: \grpAset \to \grpAset$, called \emph{inverse}.
    \end{enumerate}
    \condit
    \begin{enumerate}
      \item Associative law:~$(\grpela\mtimes \grpelb)\mtimes \grpelc=\grpela\mtimes (\grpelb\mtimes \grpelc) \hfill \forall \  \grpela, \grpelb, \grpelc \in \grpAset$;
       \item Neutrality laws:~$\idmon\mtimes \grpela=\grpela=\grpela \gtimes \idgrp \hfill \forall \  \grpela \in \grpAset$;
       \item Inverse laws:
       \begin{equation}\label{eq:group-inverse-law}
         \ginv(\monela) \mtimes \monela = \idmon = \ginv(\monela) \mtimes \monela \quad \quad \forall \  \grpela  \in \grpAset.
       \end{equation}
     \end{enumerate}
   \end{quote}
\end{ctdefinition}
\begin{example}
The following is a group: the set $\wnumbers$, together with addition as the composition operation, the element $0$ as neutral element, and ``taking the negative'' as the inverse operation:
\begin{equation}
\ginv(x) \definedas -x \quad \quad \forall \ x \in \wnumbers.
\end{equation}
\end{example}
\begin{example}
The monoid $\tup{\reals_{\backslash \{ 0 \}}, \cdot, 1}$ becomes a group when equipped with the inverse operation defined by
\begin{equation}
\ginv(x) \definedas \frac{1}{x} \quad \quad \forall \ x \in \reals.
\end{equation}
\end{example}
\begin{example}
    The monoid~$\tup{B,\wedge, \true}$ from \cref{ex:bool_monoid} cannot become a group, because there cannot by an inverse for~$\false$. In other words, there is no inverse for which~$\ginv(\false)\wedge \false=\true$.
\end{example}
\begin{example}\label{grp-order-three}
Consider the set $\grpAset \definedas \{ 1, e^{\frac{1}{3} 2 \pi}, e^{\frac{2 }{3} 2 \pi}  \} \subseteq \cnumbers$. Taking the usual multiplication of complex numbers as the composition operation, these three elements from a group.
\end{example}
\begin{exercise}[Group with three elements]
In \cref{grp-order-three}, what is the neutral element? What is the inverse operation?
Draw the multiplication table for this group.
\end{exercise}
\todotext{JL: insert examples of groups of order 1, 2, and 3 respectively}
\todotext{JL: insert example showing multiplication tables of the two possible group structures on a set of four elements}
\begin{example}[Square matrices with full rank]\label{exa:square-full}
Fix an integer $n\geq1 $ and consider the set of square matrices with full rank
\begin{equation} \label{eq:group-fullrank-issquare}
    \mathbf{A} \in \reals^{n\times n}
\end{equation}
\begin{equation} \label{eq:group-fullrank-isfullrank}
   \mdet(\mathbf{A}) \neq 0
\end{equation}
\begin{equation} \label{eq:group-fullrank-composition}
   \mathbf{A} \mtimes \mathbf{B}  \definedas \mathbf{A}\,\mathbf{B}
\end{equation}
\begin{equation} \label{eq:group-fullrank-composition-det}
   \mdet(\mathbf{A} \, \mathbf{B}) = \mdet(\mathbf{A}) \cdot  \mdet(\mathbf{B})
\end{equation}
\begin{equation} \label{eq:group-fullrank-inverse}
   \ginv(\mathbf{A}) \definedas {\mathbf{A}^{-1}}
\end{equation}
\begin{equation} \label{eq:group-fullrank-inverse-det}
   \mdet(\mathbf{A}^{-1}) = \frac{1}{\mdet(\mathbf{A})}
\end{equation}
\begin{equation} \label{eq:group-fullrank-inverse-det2}
   \mdet(\mathbf{A}^{-1}) = \left( \mdet(\mathbf{A}) \right) ^ {-1}
\end{equation}
\begin{equation} \label{eq:group-fullrank-inverse-comp}
   (\mathbf{A}\,\mathbf{B})^{-1} = {\mathbf{B}^{-1} \, \mathbf{A}^{-1}}
\end{equation}
\end{example}
\begin{lemma}\label{lemma-inv-op-properties}
Let $\tup{\grpAset, \gtimes, \idgrp, \ginv}$ be a group. Then
\begin{enumerate}
\item\label{eq:group-neutral-invariant} $\ginv(\idmon) = \idmon$;
\item\label{eq:group-inverse-inverse}  $ \ginv(\ginv(\grpela)) = \grpela \quad \quad \forall \  \grpela \in \grpAset$;
\item\label{eq:group-inverse-of-composition} $\ginv(\grpela \gtimes \grpelb ) = \ginv(\grpelb) \gtimes \ginv(\grpela) \quad \quad \forall \ \grpela, \grpelb \in \grpAset$.
\end{enumerate}
\end{lemma}
\begin{gradedexercise}[Properties of group inverse operation]
Prove \cref{lemma-inv-op-properties}.
\end{gradedexercise}
\todotext{Example-corollary: rototranslations SE(2), SE(3) }

\devel{\subsection{Code exercises}

\classlisting{Group}{}

\classlisting{FiniteGroup}{}

\subsubsection*{Representation}
The file format for groups is a extension of the one for monoids (which was an extension of the one for semigroups).
There is an extra field \fieldname{inverse} which gives you the inverse map.


\todo{write down file format}
\begin{codeexercise}[\exname{TestFiniteGroupRepresentation}]
  Create a function to load the data.
%
\end{codeexercise}

\classlisting{FiniteGroupRepresentation}{}

\todotext{to write}
}


