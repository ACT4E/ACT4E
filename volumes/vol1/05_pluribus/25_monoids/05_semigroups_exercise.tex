\subsection{Code exercises}

\begin{remark}
  In this course we assume, unless otherwise noted, that you have done the previous exercises, so that you can build on top of the code that you already have. The first exercise and tutorial was in \cref{sec:exercise-tutorial}.
\end{remark}

\begin{marginfigure}\label{fig:uml_semigroup}
  \includesag{uml_semigroup}
\end{marginfigure}

We define two classes: \Semigroup and \FiniteSemigroup.
Their relations are shown in \cref{fig:uml_semigroup}.

A \Semigroup is something that \emph{has} a carrier set, which is a \Setoid.
It also has a compose function.

\classsource{Semigroup}{min_size=0}

A \FiniteSemigroup is a special \Semigroup whose carrier is a \FiniteSet.

\classsource{FiniteSemigroup}{min_size=0}

\margindatafilefig{sm1}{sm1.semigroup.yaml}{fig:sm1}

The data format to define a finite semigroup is given in \cref{fig:sm1}.


\begin{gradedexercise}[\exname{TestFiniteSemigroupRepresentation}]
  \label{ex:TestFiniteSemigroupRepresentation}
  Create a function to load the data, with the interface shown in \cref{fig:FiniteSemigroupRepresentation}.
  Hint: you want to reuse the code you already have to load a \FiniteSet.
\end{gradedexercise}


\begin{figure*}\label{fig:FiniteSemigroupRepresentation}
\classsource{FiniteSemigroupRepresentation}{}
\end{figure*}

\subsubsection*{Constructors}

\begin{gradedexercise}
  \label{ex:FiniteSemigroupRepresentation}
  Given a finite set, construct the free semigroup.

  \methodsource{FiniteSemigroupConstruct}{free}{}
\end{gradedexercise}
