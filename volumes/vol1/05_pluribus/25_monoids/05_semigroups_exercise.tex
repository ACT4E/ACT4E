% !TEX root = chapter-standalone.tex
\sectionexercises{Semigroups}

\begin{remark}
    In this course we assume, unless otherwise noted, that you have done the previous exercises, so that you can build on top of the code that you already have.
    The first exercise and tutorial was in \cref{sec:exercise-tutorial}.
\end{remark}

\begin{figure*}[b]
    \includesag{uml_semigroup}
    \caption{}
    \label{fig:uml_semigroup}
\end{figure*}

We define two classes: \Semigroup and \FiniteSemigroup.
\Cref{fig:uml_semigroup} shows their relation.

A \Semigroup is something that \emph{has} a carrier set, which is a \Setoid.
It also has a compose function.

\classlisting{Semigroup}{}

A \FiniteSemigroup is a special \Semigroup whose carrier is a \FiniteSet.

\classlisting{FiniteSemigroup}{}

\ifextraspace{\clearpage\vfill}

\subsection{Semigroup representation}

\begin{marginfigure}%
    \datafile{sm_min0}{}
    \caption{}%
    \label{fig:sm_min3}%
\end{marginfigure}%

\Cref{fig:sm_min3} shows the data format to define a finite \SY{semigroup}.

The data structure contains two fields:
\begin{enumerate}
    \item The field \fieldname{carrier} is the serialization of a \FiniteSet \setA.
    \item The field \fieldname{composition} is the serialization of a \FiniteMap.
          The \str{source} for this map must be a product of sets ($\setA \cartprod \setA$), and the target must be equal to \setA.
\end{enumerate}

\codeboilerplate{FiniteSemigroupRepresentation}{
    Create a function to load the data, with the interface shown in \cref{lst:FiniteSemigroupRepresentation}.
}

\classlisting{FiniteSemigroupRepresentation}{}

\begin{hint}
    You want to reuse the code you already have to load a \FiniteSet and a \FiniteMap.
\end{hint}

\sectionexercises{Free semigroup}
\label{sec:free-semigroup}

\codeboilerplate{FiniteSemigroupConstruct}{
    Given a finite set, construct the free \SY{semigroup}.
}
% \begin{codeexercise}[\exname{TestFiniteSemigroupConstruct}]
%     \label{ex:TestFiniteSemigroupConstruct}
%     Given a finite set, construct the free \SY{semigroup}.
%     Implement the interface in \cref{lst:FiniteSemigroupConstruct}.
% \end{codeexercise}

\classlisting{FiniteSemigroupConstruct}{}

A \FreeSemigroup is a \SY{semigroup} with the additional method \funcname{unit} that converts an element of the carrier set to an element of the free group (\eg, from element to a list with one element.)

\classlisting{FreeSemigroup}{}
