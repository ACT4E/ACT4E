% !TEX root = chapter-standalone.tex


\section{\statusdraft{Generators and relations}}


In \cref{plant-trafo-semigroup} we considered a set of states~$X = \{ \sprout, \yng, \mature, \old, \dead \}$, a function~$T\colon X \sto X$, and the semigroup
\begin{equation}
\sgrpA = \{ T^n \mid n \in \natnumbers \}.
\end{equation}
Note that~$\sgrpA$ has a special form: all of its elements can be expressed in terms one of its elements,~$T$, and the multiplication operation (which, in this case, is function composition).
To describe this state of affairs we say that~$\sgrpA$ is \emph{generated} by the element~$T$.


Recall that~$T$ was defined by
\begin{align*}
T(\sprout) &=  \yng \\
T(\yng) &=  \mature \\
T(\mature) &=  \old \\
T( \old) &= \dead \\
T (\dead) &= \dead.
\end{align*}
Observe that the function~$T^4$ will map all elements of~$X$ to the element ``\dead''. For example, if we start with the element ``\sprout'', the result of applying~$T$ four times is
\begin{equation*}
\sprout \overset{T}{\sto} \yng \overset{T}{\sto} \mature \overset{T}{\sto} \old \overset{T}{\sto} \dead.
\end{equation*}

Note also that for \emph{any}~$n \geq 4$, the function~$T^n$ will map all elements of~$X$ to the element ``\dead''.
If we consider~$T^6$, for example, then, for any~$x \in X$,
\begin{equation*}
T^6(x) = T^2(T^4(x)) = T^2(\dead) = T(T(\dead)) = T(\dead) = \dead.
\end{equation*}
It follows that all~$T^n$, for~$n \geq 4$, are actually \emph{all the same map}: the one that sends every state to the dead state.
Thus~$\sgrpA = \{ T^n \mid n \in \natnumbers \}$ actually only has at most four elements! Namely~$T$,~$T^2$,~$T^3$, and~$T^4$.

\begin{exercise}[Checking for relations]
Are any of the four maps $T$,~$T^2$,~$T^3$, and~$T^4$ actually equal? Justify your answer by argumentation or explicitly checking by calculation. 
\end{exercise}

When two elements which a priori could be distinct from each other (such as~$T^6$ and~$T^4$ above) turn out to be equal, we call this a \emph{relation} between the elements of~$\sgrpA$.

\begin{definition}[Generated semigroups]
Let~$\tup{\sgrpA, \mtimes}$ be a semigroup, and let~$\setA \subseteq \sgrpA$ be a subset.
We say that~$\sgrpA$ is \emph{generated} by~$\setA$ if every element of~$\sgrpA$ can be expressed as a finite multiplication of elements of~$\setA$.
\end{definition}

\begin{example}
Consider \cref{string-semigroup}, where elements of the semigroup~$\sgrpA$ were non-empty strings built using the elements of the ``alphabet'' set~$\setA = \{ \alphabeta, \alphabetb\}$.
In this case,~$\sgrpA$ is generated by~$\setA$.
\end{example}


\begin{example}
Consider the natural numbers (without zero) as a semigroup, where addition is the semigroup multiplication (see \cref{natnum-semigroup}). This semigroup is generated by the subset~$\{1 \}$.
\end{example}

\begin{definition}
A \emph{relation} on a semigroup~$\tup{\sgrpA, \mtimes}$ is an equation between two multiplications of elements of~$\sgrpA$.
\end{definition}

\begin{example}
Consider the semigroup~$\tup{\natnumbers, +}$, and consider~$l, k \in \natnumbers$. The equation~$l + k = k + l$ is an example of  a relation.
\end{example}

\todotext{say that similar notions/defintions also apply for monoids}
\todotext{introduce free semigroups and free monoids}

\todotext{introduce and explain presentations of semigroups and monoids via generators and relations}
