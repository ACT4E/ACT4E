% !TEX root = chapter-standalone.tex

\section{Vector spaces}
\label{sec:vector-spaces}

\todotext{J: @J: add material}

\todotext{@J: define vector spaces}

\begin{ctdefinition}[Real vector space]
    \label{def:real-vector-space}
    A \maindef{real vector space} is
    \begin{body}
        \constit
        \begin{enumerate}
            \item a set~$\vecspBset$, the elements of which are called \emph{vectors};
            \item a binary operation~$+ \colon \vecspBset \cartprod \vecspBset \sto \vecspBset$, called \emph{vector addition};
            \item an element $0 \setin \vecspBset$;
            \item an operation ~$\cdot \colon \reals \cartprod \vecspBset \sto \vecspBset$, called \emph{scalar multiplication}.
        \end{enumerate}
        \condit
        \begin{enumerate}
            \item $\tup{\vecspBset, +}$ is a commutative group, with neutral element $0$;
            \item Scalar multiplication is an action:
                  \begin{enumerate}
                      \item $(\lambda \mu) \cdot \ela = \lambda \cdot( \mu \cdot \ela) \quad \forall \lambda, \mu \setin \reals, \ \forall \ela \setin \vecspA$;
                      \item $1 \cdot \ela = \ela \quad \forall \ela \setin \vecspAset$;
                      \item $(\lambda + \mu) \cdot \ela = (\lambda \cdot \mu) + (\mu \cdot \ela) \quad \forall \lambda, \mu \setin \reals, \ \forall \ela \setin \vecspBset$;
                      \item $\lambda \cdot (\ela + \elb) = (\lambda \cdot \ela) + (\lambda \cdot \elb), \quad \forall \lambda \setin \reals, \ \forall \ela, \elb \setin \vecspBset$.
                  \end{enumerate}
        \end{enumerate}
    \end{body}
\end{ctdefinition}

\begin{remark}
    The general definition of a vector space is obtained by replacing $\reals$ in the above definition with an arbitrary field $\fieldA$.
\end{remark}

\begin{example}
    $\tup{\reals^n, +, \cdot}$
\end{example}

\todotext{@J: example: $R^n$}

\todotext{@J: example: polynomials (is there also a simple/neat applied interpretation/example here?)}

\begin{gradedexercise}[\exname{RealPolynomials}]
    \label{ex:real-polynomials}
    Let $\setA$ denote the set of polynomials in one variable and with coefficients in $\reals$.
    In other words, elements of $\setA$ are polynomials of the form
    \begin{equation}
        p(X) = a_nX^n + a_{n-1}
        X^{n-1} + \dots + a_1X + a_0,
    \end{equation}
    where $n \setin \natnumbers$ and $a_n, a_{n-1}, \dots, a_1, a_0 \setin \reals$ may vary.

    \begin{enumerate}
        \item Check (prove) that $\setA$ forms a commutative monoid when equipped with with the usual addition operation for polynomials and the zero polynomial $p(X) = 0$ as neutral element.
        \item Is the commutative monoid $\tup{\setA, +, 0}$ in fact a group?
        \item Check that $\tup{\setA, +, 0}$ forms a real vector space when we equip it with the following (usual) scalar multiplication:
              \begin{equation}
                  \lambda \cdot (a_nX^n + a_{n-1}
                  X^{n-1} + \dots + a_1X + a_0) \definedas \lambda a_nX^n + \lambda a_{n-1}X^{n-1} + \dots + \lambda a_1X + \lambda a_0
              \end{equation}
              for any $\lambda \setin \reals$.
    \end{enumerate}
\end{gradedexercise}

\solutionof{RealPolynomials}

\todotext{@J: example: stuff related to fourier analysis}