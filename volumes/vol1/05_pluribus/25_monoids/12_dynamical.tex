\section[Dynamical systems]{Dynamical systems and transition functions}

In this section we discuss an example of how \SY{monoids} are related to the evolution of a dynamical system in time.

\label{exa:transition-functions}
\begin{definition}[Continuous-time dynamical system]
    \label{def:ct-dynsyst}
    A dynamical system on~$\reals^n$ may be defined by a function
    \begin{equation}
        \mapa \colon \reals^n \sto \reals^n.
    \end{equation}
    A trajectory of a dynamical system is a function~$\stylemaps{x}\colon \reals \sto \reals^n$ such that
    \begin{equation}
        \dot{\stylemaps{x}}(t) = \mapa(\stylemaps{x}(t)).
    \end{equation}
    We use the notation~$\dot{\stylemaps{x}}$ to abbreviate~$\dd \stylemaps{x} / \dd t$.
\end{definition}

Suppose that we are dealing with dynamical systems such that for any point~$x_0$, there is exactly one trajectory beginning at~$x_0$.
(For this we need to put reasonable constraints on the function~$\mapa$.)
\todotextjira{554}{\alphubel: @Andrea: spell out some simple sufficient constraint (smooth?)}

We can then ask the following: given a point~$x$, where would its trajectory be after~$\delta$?
This question induces a family of functions~$\stylemaps{T_\delta}$, called transition functions.
For each particular~$\delta$, we have a function
\begin{equation}
    \stylemaps{T_{\delta}}\colon \reals^n \sto \reals^n
\end{equation}
that maps a point to its position~$\delta$ in the future.

We can spot here a \SY{semigroup} structure.
Suppose we want to know the position of a point~$\delta_1 + \delta_2$
in the future.
We can take~$\stylemaps{T_{\delta_1}}$ and compose it with~$\stylemaps{T_{\delta_2}}$; or take directly~$\stylemaps{T_{\delta_1 + \delta_2}}$.
By construction, we will have that
\begin{equation}
    \label{eq:transition-property}
    \stylemaps{T_{\delta_1 + \delta_2}} = \stylemaps{T_{\delta_1}} \mtimes \stylemaps{T_{\delta_2}}.
\end{equation}
We can also easily prove associativity:
\begin{equation}
    \begin{aligned}
        \stylemaps{T_{\delta_1}}\mtimes (\stylemaps{T_{\delta_2}}\mtimes \stylemaps{T_{\delta_3}}) & =\stylemaps{T_{\delta_1}}\mtimes \stylemaps{T_{\delta_2+\delta_3}} \\
                                                                                                   & =\stylemaps{T_{\delta_1+\delta_2+\delta_3}} \\
                                                                                                   & =\stylemaps{T_{\delta_1+\delta_2}}\mtimes \stylemaps{T_{\delta_3}} \\
                                                                                                   & =(\stylemaps{T_{\delta_1}}\mtimes \stylemaps{T_{\delta_2}})\mtimes \stylemaps{T_{\delta_3}}.
    \end{aligned}
\end{equation}
%
This shows that the set of transition functions for a particular system with the operation of function composition form a \SY{semigroup}.

This \SY{semigroup} is a \SY{monoid} because there is an identity.
The identity is~$\stylemaps{T_0}$, the map that tells us what happens after~$0$ seconds.
That is~$\idmon_{\reals^n}$, the identity on~$\reals^n$.
To show that~$\stylemaps{T_0} = \idmon_{\reals^n}$ is an identity, we can fix any~$\delta$ and substituting in \cref{eq:transition-property} we have
\begin{equation}
    \begin{aligned}
        \stylemaps{T_{\delta + 0}} & = \stylemaps{T_{\delta}} \mtimes \stylemaps{T_{0}} \\
        =                          & \stylemaps{T_{\delta}}.
    \end{aligned}
\end{equation}
