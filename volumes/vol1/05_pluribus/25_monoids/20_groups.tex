% !TEX root = chapter-standalone.tex


\section{Groups}
\label{sec:groups}
\linkvideo{spring2021-semi-mon-gro:groups}
\todotext{Add intro}

\begin{ctdefinition}[Group]
    \label{def:group}
    A \emph{\iindex{group}} is a monoid together with an ``inverse'' operation.
    In more detail, a group~$\grpA$ is
    \begin{body}
        \constit
        \begin{enumerate}
            \item a set~$\grpAset$;
            \item a binary operation~$\gtimes \colon \grpAset \cartprod \grpAset \sto \grpAset$, called \emph{composition};
            \item a specified element~$\idgrp \in \grpAset$;
            \item a map~$\ginv\colon \grpAset \to \grpAset$, called \emph{inverse}.
        \end{enumerate}
        \condit
        \begin{enumerate}
            \item Associative law:~$(\grpela\mtimes \grpelb)\mtimes \grpelc=\grpela\mtimes (\grpelb\mtimes \grpelc), \quad \forall  \grpela, \grpelb, \grpelc \in \grpAset$;
            \item Neutrality laws:~$\idmon\mtimes \grpela=\grpela=\grpela \gtimes \idgrp, \quad \forall  \grpela \in \grpAset$;
            \item Inverse laws:
            \begin{equation}
                \label{eq:group-inverse-law}
                \ginv(\monela) \mtimes \monela = \idmon = \monela \mtimes \ginv(\monela), \quad \forall  \grpela  \in \grpAset.
            \end{equation}
        \end{enumerate}
    \end{body}
\end{ctdefinition}
\begin{example}
    The following is a group: the set~$\wnumbers$, together with addition as the composition operation, the element $0$ as neutral element, and ``taking the negative'' as the inverse operation:
    \begin{equation}
        \ginv(\grpela) \definedas -\grpela, \quad \forall \ \grpela \in \wnumbers.
    \end{equation}
\end{example}
\begin{example}
    The monoid~$\tup{\reals_{\backslash \{ 0 \}}, \cdot, 1}$ becomes a group when equipped with the inverse operation defined by
    \begin{equation}
        \ginv(\grpela) \definedas \frac{1}{\grpela}, \quad \forall \grpela \in \reals.
    \end{equation}
\end{example}

\begin{example}
    For the monoids~$\tup{\natnumbers, +,0}$ and~$\tup{\natnumbers,\cdot ,1}$ one cannot find an inverse operation to form a group.
\end{example}

\begin{exercise}
    Can one find an inverse operation for the monoid~$\tup{\natnumbers,\max,0}$?
\end{exercise}
%
\begin{solution}
    No. Consider the condition~$0=\max(\ginv(\grpela),\grpela)$. In general, this is true only if $\grpela=\ginv(\grpela)=0$.
\end{solution}

\begin{example}
    The monoid~$\tup{\stylesets{B},\wedge, \true}$ from \cref{ex:bool_monoid} cannot become a group, because there cannot by an inverse for~$\false$.
    In other words, there is no inverse for which~$\ginv(\false)\wedge \false=\true$.
\end{example}
\begin{example}
    \label{grp-order-three}
    Consider the set
    \begin{equation*}
        \grpAset \definedas \left\{ 1, e^{\frac{1}{3}2\pi i}, e^{\frac{2 }{3}2\pi i}  \right\} \subseteq \cnumbers.
    \end{equation*}
    Taking the usual multiplication of complex numbers as the composition operation, these three elements from a group.
\end{example}

\begin{gradedexercise}[\exname{GroupWithThreeElements}]
    \label{ex:GroupWithThreeElements}
    In \cref{grp-order-three}, what is the neutral element? What is the inverse operation?
    Draw the composition table for this group.
\end{gradedexercise}

\solutionof{GroupWithThreeElements}
\todotextjira{12}{@J: Insert examples of groups of order 1, 2, and 3 respectively}
\todotextjira{13}{@J: insert example showing multiplication tables of the two possible group structures on a set of four elements}

\begin{lemma}
    \label{lem:inv-op-unique}
    Let~$\grpA = \tup{ \grpAset, \gtimes, \idgrp, \ginv }$ be a group and let~$\grpela, \grpelb \in \grpAset$. If~$\grpela$ and~$\grpelb$ satisfy the equation
    \begin{equation}
        \grpela \gtimes \grpelb = \idgrp.
    \end{equation}
    Then~$\grpelb = \ginv( \grpela )$ and~$\grpela = \ginv(\grpelb)$.
\end{lemma}

\begin{proof}
    If~$\grpela \gtimes \grpelb = \idgrp$, then, by composing both sides of this equation with~$\ginv(\grpela)$ from the left, and using associativity to remove brackets, we find~$\ginv(\grpela) \gtimes \grpela \gtimes \grpelb = \idgrp \gtimes \ginv(\grpela)$.
    Applying the inverse laws on the left-hand side, we obtain~$\idgrp \gtimes \grpelb = \idgrp \gtimes \ginv(\grpela)$, and using the neutrality laws on both side of this equation yields~$\grpelb =  \ginv(\grpela)$.
    The fact that~$\grpela = \ginv(\grpelb)$ may be proved similarly.
\end{proof}


\begin{corollary}
    \label{cor:inv-op-unique}
    Let~$\monoidA = \tup{\monoidAset, \mtimes,  \idmon }$ be a monoid.
    If~$\ginv$ and~$\widetilde \ginv$ are both operations of inverse which make~$\monoidA$ into a group, then~$\ginv = \widetilde \ginv$ holds.
    In other words, if a monoid can be made into a group by adding an inverse operation, then the resulting group is uniquely determined by the underlying monoid.
\end{corollary}

\begin{proof}
    Let~$\monela \in \monoidAset$.
    Since $\tup{\monoidAset, \mtimes,  \idmon, \widetilde \ginv }$ is a group, we have~$\monela \mtimes  \widetilde \ginv(\monela) = \idmon$.
    Also~$\tup{\monoidAset, \mtimes,  \idmon, \ginv }$ is a group, and in terms of this group, the equation~$\monela \mtimes  \widetilde \ginv(\monela) = \idmon$ implies, by \cref{lem:inv-op-unique}, that~$\widetilde \ginv(\monela) = \ginv(\grpela)$, by thinking of~$\widetilde \ginv(\monela)$ in the role of~$\grpelb$ from \cref{lem:inv-op-unique}.
    Since~$\grpela \in \monoidAset$ was arbitrary,~$\widetilde \ginv = \ginv$ as functions on~$\monoidAset$.
\end{proof}


\begin{lemma}
    \label{lem:inv-op-properties}
    Let~$\tup{\grpAset, \gtimes, \idgrp, \ginv}$ be a group. Then
    \begin{enumerate}
        \item\label{eq:group-neutral-invariant}~$\ginv(\idgrp) = \idgrp$;
        \item\label{eq:group-inverse-inverse}~$ \ginv(\ginv(\grpela)) = \grpela, \quad \forall \grpela \in \grpAset$;
        \item\label{eq:group-inverse-of-composition}~$\ginv(\grpela \gtimes \grpelb ) = \ginv(\grpelb) \gtimes \ginv(\grpela), \quad \forall \grpela, \grpelb \in \grpAset$.
    \end{enumerate}
\end{lemma}

\begin{gradedexercise}[\exname{GroupInverseProperties}]
    \label{ex:GroupInverseProperties}
    Prove \cref{lem:inv-op-properties}.
\end{gradedexercise}

\solutionof{GroupInverseProperties}

\begin{example}[Square matrices with full rank]
    \label{exa:square-full}
    Fix an integer~$n\geq1 $ and consider the set of square matrices with full rank~$\mat{A} \in \reals^{n\times n}$.
    This set, equipped with the usual matrix multiplication as the binary operation ($\mat{A}\mtimes \mat{B}\definedas \mat{AB}$), the identity matrix~$\idmat$ as the neutral element, and matrix inverse as the inverse ($\ginv(\mat{A})\definedas \mat{A}^{-1}$), forms a group.
    Furthermore, note that for this type of matrices, one has the properties:
    \begin{compactenum}
        \item $\mdet(\mat{A}) \neq 0$;
        \item $\mdet(\mat{A} \, \mat{B}) = \mdet(\mat{A}) \cdot  \mdet(\mat{B})$;
        % \item $\mdet(\mat{A}^{-1}) = \frac{1}{\mdet(\mat{A})}$;
        \item $\mdet(\mat{A}^{-1}) = \left( \mdet(\mat{A}) \right) ^ {-1}$;
        \item $(\mat{A}\,\mat{B})^{-1} = {\mat{B}^{-1} \, \mat{A}^{-1}}$.
    \end{compactenum}
    \todotextjira{14}{Useless to just mention these equations without discussing why they are important.}
\end{example}

\showslides{
    \begin{forslides}
        \begin{equation}
            \label{eq:group-fullrank-isfullrank}
            \mdet(\mathbf{A}) \neq 0
        \end{equation}
        \begin{equation}
            \label{eq:group-fullrank-composition}
            \mathbf{A} \mtimes \mathbf{B}  \definedas \mathbf{A}\,\mathbf{B}
        \end{equation}
        \begin{equation}
            \label{eq:group-fullrank-composition-det}
            \mdet(\mathbf{A} \, \mathbf{B}) = \mdet(\mathbf{A}) \cdot  \mdet(\mathbf{B})
        \end{equation}
        \begin{equation}
            \label{eq:group-fullrank-inverse}
            \ginv(\mathbf{A}) \definedas {\mathbf{A}^{-1}}
        \end{equation}
        \begin{equation}
            \label{eq:group-fullrank-inverse-det}
            \mdet(\mathbf{A}^{-1}) = \frac{1}{\mdet(\mathbf{A})}
        \end{equation}
        \begin{equation}
            \label{eq:group-fullrank-inverse-det2}
            \mdet(\mathbf{A}^{-1}) = \left( \mdet(\mathbf{A}) \right) ^ {-1}
        \end{equation}
        \begin{equation}
            \label{eq:group-fullrank-inverse-comp}
            (\mathbf{A}\,\mathbf{B})^{-1} = {\mathbf{B}^{-1} \, \mathbf{A}^{-1}}
        \end{equation}
    \end{forslides}
}%\devel

\begin{example}[Orthogonal matrices]
    Fix an integer~$n\geq 1$ and consider the set of \emph{orthogonal} matrices~$\mat{A}\in \reals^{n\cartprod n}$. Orthogonal matrices are real, square matrices with orthonormal columns and rows. One was to express orthogonality of a matrix is:
    \begin{equation*}
        \mat{A}^\intercal \mat{A}=\mat{A}\mat{A}^\intercal = \idmat,
    \end{equation*}
    where~$\idmat$ is the identity matrix. The set~$\reals^{n\cartprod n}$ equipped with matrix multiplication as a binary operation, the identity matrix as the neutral element, and the transposition~$(\cdot)^\intercal$ (which for this specific type of matrices corresponds to the inverse) forms a group, usually denoted~$\text{O}(n)$.
    Any orthogonal matrix~$\mat{A}$ has the property~$\deter(\mat{A})\in \{-1,1\}$. The subset of orthogonal~$n\cartprod n$ matrices with determinant 1 forms the so-called \emph{special} orthogonal group~$\mgSOn$.
\end{example}

