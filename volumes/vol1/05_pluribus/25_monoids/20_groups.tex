% !TEX root = chapter-standalone.tex

\section{Groups}
\label{sec:groups}
\linkvideo{spring2021-semi-mon-gro:groups}

\SY{Groups} appear in many areas of mathematics, both pure and applied.

One important use of \SY{groups} is to describe symmetries.
Roughly speaking, a symmetry is an invertible transformation or reconfiguration of a figure (or object) that leaves the essential features of that figure invariant.

For example, consider a perfectly square sheet of monochrome paper lying on a table-top.
If we rotate the piece of paper by 90 degrees around its center, how it appears to us after this rotation will be essentially indistinguishable from before it was rotated.
Thus, this rotation is a symmetry.

\SY{Groups of symmetries} play an important role in physics and chemistry, for example when studying the repeating patterns of \SY{lattices} of atoms or molecules in materials, or when studying the geometric patterns of atoms and molecules themselves.

For engineering, a particularly important class of \SY{groups} are \SY{matrix groups}, in particular those that describe various types of motion in space.
These are discussed in \cref{sec:matrix-grps}.

\begin{ctdefinition}[Group]
    \label{def:group}
    A \maindef{group} is a \SY{monoid} together with an ``inverse'' operation.
    In more detail, a group~$\grpA$ is
    \begin{body}
        \constit
        \begin{enumerate}
            \item a set~$\grpAset$;
            \item a binary operation~$\gtimes \colon \grpAset \cartprod \grpAset \sto \grpAset$, called \emph{composition};
            \item a specified element~$\idgrp \setin \grpAset$;
            \item a map~$\ginv\colon \grpAset \sto \grpAset$, called \emph{inverse}.
        \end{enumerate}
        \condit
        \begin{enumerate}
            \item Associative law:~$(\grpela\mtimes \grpelb)\mtimes \grpelc=\grpela\mtimes (\grpelb\mtimes \grpelc), \quad \forall  \grpela, \grpelb, \grpelc \setin \grpAset$;
            \item Neutrality laws:~$\idmon\mtimes \grpela=\grpela=\grpela \gtimes \idgrp, \quad \forall  \grpela \setin \grpAset$;
            \item Inverse laws:
                  \begin{equation}
                      \label{eq:group-inverse-law}
                      \ginv(\monela) \mtimes \monela = \idmon = \monela \mtimes \ginv(\monela), \quad \forall  \grpela  \setin \grpAset.
                  \end{equation}
        \end{enumerate}
    \end{body}
\end{ctdefinition}


\begin{marginfigure}
    \centering
    \includesag{group-diagram}
    \caption{Group Diagram}
    \label{fig:group-diagram}
\end{marginfigure}


\begin{ctdefinition}[Commutative group]\label{def:commutative-group}
    \SYNDEF{commutative group}
    A \SY{group} is called \emph{commutative} (or: \emph{Abelian}) if its underlying \SYN{commutative semigroup}{semigroup is commutative}.
\end{ctdefinition}

\begin{remark}
    The size of the underlying set of a group is often called the \emph{order} or cardinality of the group.
    We'll sometimes use this terminology, or just call it the size of the group.
\end{remark}

\subsection{Some examples}

\begin{example}
    The following is a group: the set~$\wnumbers$, together with addition as the composition operation, the element $0$ as \SY{neutral element}, and ``taking the negative'' as the inverse operation:
    \begin{equation}
        \ginv(\grpela) \definedas -\grpela, \quad \forall \ \grpela \setin \wnumbers.
    \end{equation}
\end{example}
\begin{example}
    The \SY{monoid}~$\tup{\reals_{\setcomplement \makeset{0}}, \cdot, 1}$ becomes a group when equipped with the inverse operation defined by
    \begin{equation}
        \ginv(\grpela) \definedas \frac{1}{\grpela}, \quad \forall \grpela \setin \reals.
    \end{equation}
\end{example}

\begin{example}
    For the \SY{monoids}~$\tup{\natnumbers, +, 0}$ and~$\tup{\natnumbers, \cdot, 1}$ we cannot find an inverse operation that would turn these \SY{monoids} into \SY{groups}.
\end{example}

\begin{exercise}
    Can one find an inverse operation for the \SY{monoid}~$\tup{\natnumbers,\max,0}$?
\end{exercise}
%
\begin{solution}
    No.
    Consider the condition~$0=\max(\ginv(\grpela),\grpela)$.
    In general, this is true only if~$\grpela=\ginv(\grpela)=0$.
\end{solution}

\begin{example}
    The \SY{monoid}~$\tup{\boolset,\booland, \true}$ from \cref{ex:bool_monoid} cannot become a group, because there cannot be an inverse for~$\false$: there is no possible choice for $\ginv(\false)$ such that $\ginv(\false)\booland \false=\true$.
\end{example}

\begin{example}
    Given a set~\setA, an invertible function $\setA \sto \setA$ is called an \emph{automorphism} of~\setA.
    There is a ``naturally given'' group structure on the set $\Autof\setA$ of automorphisms of~\setA:
    we can take the composition operation to be the composition of functions, the \SY{neutral element} is the identity function on~\setA, and inverses are given by the inverses of functions.

    As a sub-example, consider the set
    \begin{equation}
        \setA = \makeset{1, 2, 3, \dots, n-1, n},
    \end{equation}
    where~$n \setin \natnumbers$.
    Then~$\Autof\setA$ is the group of permutations of~$n$ elements.
    The usual notation for this group is~$\Perms_n$.
    Its size is~$n! = n \cdot (n-1) \cdot \cdots \cdot 3 \cdot 2 \cdot 1$.

\end{example}

\begin{example}
    \label{exa:grp-order-two}
    Consider the set~$\makeset{0, 1}$, equipped with the composition operation~$\gtimes$ defined to be ``addition modulo 2''.
    The composition table for this is \cref{tab:comp-table-add-mod-2}
    \begin{margintable}
        \caption{
            Addition modulo 2 on the set~$\makeset{0, 1}$.
        }
        \label{tab:comp-table-add-mod-2}
        \centering
        \begin{tabular}{c|cc}
            $+$ & $0$ & $1$ \\
            \hline
            $0$ & $0$ & $1$ \\
            $1$ & $1$ & $0$
        \end{tabular}
    \end{margintable}

    Choose~$0$ as the \SY{neutral element}, and let~$\ginv(0) = 0$ and~$\ginv(1) = 1$.
    Then $\tupp{\makeset{0, 1}, \gtimes, \idgrp, \ginv}$ is a group.
\end{example}

\begin{example}
    \label{grp-order-three}
    Consider the following set of complex numbers:
    %
    \begin{equation}
        \makesetBig{ 1, e^{\frac{1}{3}2\pi i}, e^{\frac{2 }{3}2\pi i} } \setsubseteq \cnumbers.
    \end{equation}
    %
    Taking the usual multiplication of complex numbers as the composition operation, these three numbers form a group.
\end{example}

\begin{margintable}
    \caption{Cyclic group of order 4.}
    \label{tab:comp-table-Z4}
    \centering
    \begin{tabular}{c|cccc}
        $+$ & $0$ & $1$ & $2$ & $3$ \\
        \hline
        $0$ & $0$ & $1$ & $2$ & $3$ \\
        $1$ & $1$ & $2$ & $3$ & $0$ \\
        $2$ & $2$ & $3$ & $0$ & $1$ \\
        $3$ & $3$ & $0$ & $1$ & $2$
    \end{tabular}
\end{margintable}

\begin{example}
    \label{exa:grp-Z4}
    The set~$\makeset{0, 1, 2, 3 }$ may be equipped with a group structure whose composition operation is addition modulo 4, and where~$0$ is the \SY{neutral element}.
    The composition table is \cref{tab:comp-table-Z4}.
\end{example}

\begin{margintable}
    \caption{The Klein four group}
    \label{tab:comp-table-Klein4}
    \centering
    \begin{tabular}{c|cccc}
        $+$ & $I$ & $V$ & $H$ & $R$ \\
        \hline
        $I$ & $I$ & $V$ & $H$ & $R$ \\
        $V$ & $V$ & $I$ & $R$ & $H$ \\
        $H$ & $H$ & $R$ & $I$ & $V$ \\
        $R$ & $R$ & $H$ & $V$ & $I$
    \end{tabular}
\end{margintable}
\begin{example}
    \label{exa:grp-Klein4}
    Consider the shape of a rectangle in the plane, oriented vertically, and assume the rectangle is not a square.
    Then there are four symmetries of this shape:
    \begin{enumerate}
        \item doing nothing (leaving everything in place);
        \item reflecting the shape along the vertical axis;
        \item reflecting the shape along the horizontal axis;
        \item rotating the shape by 180 degrees.
    \end{enumerate}
    Let us call these symmetries $I$, $V$, $H$, and $R$, respectively.
    We can model these for instance using \SY{bijective} functions of the plane $\reals^2$.
    The set $\makeset{I, V, H, R}$ can be given the structure of a group, with $I$ being its \SY{neutral element}, and with composition and inverses given as in \cref{tab:comp-table-Klein4}.
    We think of composition meaning that we perform one symmetry transformation and then the other (function composition), and inverses correspond to taking the inverse transformation.
    This group is known as the Klein four-group, named after the mathematician Felix Klein.
\end{example}

\begin{marginfigure}
    \centering
    \includegraphics[height=2cm]{tetrahedron} \\
    \includegraphics[height=2cm]{square}
    \includegraphics[height=2cm]{octahedron} \\
    \includegraphics[height=2cm]{dodecahedron}
    \includegraphics[height=2cm]{icosahedron}
    \caption{The Platonic solids}
    \label{fig:platonic-solids}
\end{marginfigure}

\todotextjira{726}{\bernina: Reference for images of platonic solids: wikipedia page ``platonic solid''}

\begin{example}
    Similar to the previous example, all sorts of geometric shapes have \SY{groups} of symmetries associated with them.

    For example, each of the five Platonic solids (\cref{fig:platonic-solids}) have a group of symmetries associated with them.

    It turns out that the cube and the octahedron have the same group of symmetries, and similarly the dodecahedron and the icosahedron have the same symmetry group.
    This has to do with the fact that these Platonic solids are dual to each other, respectively, in the following sense.
    An octahedron can be obtained from a cube by replacing faces with vertices and vertices with faces, and the same goes for the dodecahedron and the icosahedron.

    The symmetry group of the tetrahedron has order 24, the symmetry group of the cube and octahedron has order 48, and the symmetry group of the dodecahedron and icosahedron has order 120.
\end{example}

\begin{marginfigure}
    \centering
    \includegraphics[height=4cm]{rubiks} \\
    \caption{A Rubik's cube}
    \label{fig:rubiks}
\end{marginfigure}

\begin{example}
    There is a group that describes all the possible manipulations of a Rubik's cube (\cref{fig:rubiks}).
    The size of this group is
    \begin{equation}
        43,252,003,274,489,856,000 = 12! \cdot 2^{10} \cdot 8! \cdot 3^7.
    \end{equation}
\end{example}

\todotextjira{726}{\bernina: Reference for images Rubik's cube: wikipedia page for rubik's cube group}


\subsection{Matrix groups}
\label{sec:matrix-grps}

%\linkvideo{spring2021-actions:matrix-groups} % Matrix groups

There are various \emph{matrix groups} (\cref{tab:matrix-groups}) that represent linear transformations having special properties. Here we consider only matrices with entries from the real number field~\reals.

\begin{definition}
    [General linear group~$\mgGLn$]
    \label{def:general-linear-group}
    The (real) general linear group of order~$n$, written~$\mgGLn$, is the group of $\ntimesn$ invertible matrices with entries in \reals.
\end{definition}

\begin{definition}
    [Orthogonal group~$\mgOn$]
    \label{def:general-orthogonal-group}
    The (real) orthogonal group of order~$n$, written~$\mgOn$, is the group of~$\ntimesn$ real matrices that satisfy
    \begin{equation}
        \mat{M} \mat{M}\mattransp = \mat{M}\mattransp \mat{M} = \idmat.
    \end{equation}
\end{definition}

\begin{definition}
    Special linear group~$\mgSLn$]
    \label{def:special-linear-group}
    The (real) special linear group of order~$n$, written~$\mgSLn$, is the group of~$\ntimesn$ invertible real matrices with determinant equal to~1.
\end{definition}
\begin{definition}
    Special orthogonal group~$\mgSOn$]
    \label{def:special-orthogonal-group}
    The (real) special orthogonal group of order~$n$, written~$\mgSOn$, is the group of~$\ntimesn$ real matrices that satisfy
    \begin{equation}
        \mat{M} \mat{M}\mattransp = \mat{M}\mattransp \mat{M} = \idmat,
    \end{equation}
    and~$\deter(\mat{M}) = 1$.
\end{definition}

\todojira{610}{\bernina: There's lots of definitions but not much synthesis.
    Create picture with hierarchy of matrix \SY{groups}.
}


\subsection{Properties of groups}

\begin{lemma}
    \label{lem:inv-op-properties}
    Let~$\tup{\grpAset, \gtimes, \idgrp, \ginv}$ be a group.
    Then
    \begin{enumerate}
        \item\label{eq:group-neutral-invariant}~$\ginv(\idgrp) = \idgrp$;
        \item\label{eq:group-inverse-inverse}~$ \ginv(\ginv(\grpela)) = \grpela, \quad \forall \grpela \setin \grpAset$;
        \item\label{eq:group-inverse-of-composition}~$\ginv(\grpela \gtimes \grpelb ) = \ginv(\grpelb) \gtimes \ginv(\grpela), \quad \forall \grpela, \grpelb \setin \grpAset$.
    \end{enumerate}
\end{lemma}

\begin{lemma}
    \label{lem:inv-op-unique}
    Let~$\grpA = \tup{\grpAset, \gtimes, \idgrp, \ginv }$ be a group and let~$\grpela, \grpelb \setin \grpAset$.
    If~$\grpela$ and~$\grpelb$ satisfy the equation
    \begin{equation}
        \grpela \gtimes \grpelb = \idgrp,
    \end{equation}
    then~$\grpelb = \ginv( \grpela )$ and~$\grpela = \ginv(\grpelb)$.
\end{lemma}

\begin{proof}
    If~$\grpela \gtimes \grpelb = \idgrp$, then, by composing both sides of this equation with $\ginv(\grpela)$ from the left, and using associativity to remove brackets, we find~$\ginv(\grpela) \gtimes \grpela \gtimes \grpelb = \idgrp \gtimes \ginv(\grpela)$.
    Applying the inverse laws on the left-hand side, we obtain~$\idgrp \gtimes \grpelb = \idgrp \gtimes \ginv(\grpela)$, and using the neutrality laws on both side of this equation yields~$\grpelb = \ginv(\grpela)$.
    The fact that~$\grpela = \ginv(\grpelb)$ may be proved similarly.
\end{proof}

\begin{corollary}
    \label{cor:inv-op-unique}
    Let~$\monoidAdefinition$ be a \SY{monoid}.
    If~$\ginv_1$ and~$\ginv_2$ are both operations of inverse which make~$\monoidA$ into a group, then~$\ginv_1 =\ginv_2$ holds.
    In other words, if a \SY{monoid} can be made into a group by adding an inverse operation, then the resulting group is uniquely determined by the underlying \SY{monoid}.
\end{corollary}

\begin{proof}
    Let~$\monela \setin \monoidAset$.
    Since~$\tup{\monoidAset, \mtimes, \idmon, \ginv_2 }$ is a group, we have~$\monela \mtimes \ginv_2(\monela) = \idmon$.
    Also~$\tup{\monoidAset, \mtimes, \idmon, \ginv_1}$ is a group, and in terms of this group, the equation~$\monela \mtimes \ginv_2(\monela) = \idmon$ implies, by \cref{lem:inv-op-unique}, that~$\ginv_2(\monela) = \ginv_1(\grpela)$, by thinking of~$\ginv_2(\monela)$ in the role of~$\grpelb$ from \cref{lem:inv-op-unique}.
    Since~$\grpela \setin \monoidAset$ was arbitrary,~$\ginv_2 = \ginv_1$ as functions on~$\monoidAset$.
\end{proof}

\todotextjira{727}{\bernina: @JL: make comment about common other definition of a group}

\begin{example}[Orthogonal matrices]

    Fix an integer~$n\geq 1$ and consider the set of real \emph{orthogonal} matrices~$\mat{A}\setin \reals^{\ntimesn}$:
    real, square matrices with orthonormal columns and rows.
    One way to express orthogonality of a matrix is:
    \begin{equation}
        \mat{A}\mattransp \mat{A}=\mat{A}\mat{A}\mattransp = \idmat,
    \end{equation}
    where~$\idmat$ is the \maindef{identity matrix}.
    The set~$\reals^{\ntimesn}$ equipped with matrix multiplication as a binary operation, the identity matrix as the \SY{neutral element}, and the transposition~$(\cdot)\mattransp$ (which for this specific type of matrices corresponds to the inverse) forms a group, usually denoted~$\text{O}(n)$.
    Any orthogonal matrix~$\mat{A}$ has the property~$\deter(\mat{A})\setin \makeset{{{\minusone}},{{\plusone}}}$.
    The subset of orthogonal~$\ntimesn$ matrices with determinant 1 forms the so-called \emph{special} orthogonal group~$\mgSOn$.
\end{example}

\vfill%exercises

\begin{gradedexercise}[\exname{GroupWithThreeElements}]
    \label{ex:GroupWithThreeElements}
    In \cref{grp-order-three}, what is the neutral element?
    What is the inverse operation?
    Draw the composition table for this group.
\end{gradedexercise}

\solutionof{GroupWithThreeElements}

\begin{gradedexercise}[\exname{GroupInverseProperties}]
    \label{ex:GroupInverseProperties}
    Prove \cref{lem:inv-op-properties}.
\end{gradedexercise}

\solutionof{GroupInverseProperties}

\begin{remark}
    Just like for semigroups and monoids, any group has an opposite.
    It is defined analogously.
\end{remark}

\subsection{Subgroups}

\todotextjira{454}{\bernina: Add more to subgroups subsection}

\begin{definition}[Subgroup]\label{def:subgroup}
    Let~$\grpA = \tup{\grpAset, \gtimes, \idgrp, \ginv}$ be a group.
    A \maindef{subgroup} of~$\grpA$ is:

    \constit

    \begin{enumerate}
        \item A subset~$\grpBset \setsubseteq \grpAset$.
    \end{enumerate}

    \condit

    \begin{enumerate}
        \item The set~$\grpBset$ is closed under the composition operation~$\gtimes$ from~$\grpA$:
              \begin{equation}
                  \prfsemi{
                      \grpela \setin \grpBset \quad \grpelb \setin \grpBset
                  }{
                      \grpela \gtimes \grpelb \setin \grpBset
                  }
              \end{equation}

        \item The \SY{neutral element}~$\idgrp \setin \grpAset$ is an element of~$\grpBset$;

        \item The set~$\grpBset$ is closed under the inverse operation~$\ginv$ from~$\grpA$:
              \begin{equation}
                  \prfperiod{
                      \grpela \setin \grpBset
                  }{
                      \ginv(\grpela) \setin \grpBset
                  }
              \end{equation}
    \end{enumerate}
\end{definition}
