% !TEX root = chapter-standalone.tex

\section{Groups}
\label{sec:groups}
\linkvideo{spring2021-semi-mon-gro:groups}

Groups appear in many areas of mathematics, both pure and applied.

One important use of groups is to describe symmetries.
Roughly speaking, a symmetry is an invertible transformation or reconfiguration of a figure (or object) that leaves the essential features of that figure invariant.

For example, consider a perfectly square sheet of monochrome paper lying on a table top.
If we rotate the piece of paper by 90 degrees around its center, how it appears to us after this rotation will be essentially indistinguishable from before it was rotated.
Thus this rotation is a symmetry.

Groups of symmetries play an important role in physics and chemistry, for example when studying the repeating patterns of lattices of atoms or molecules in materials, or when studying the geometric patterns of atoms and molecules themselves.

For engineering, a particularly important class of groups are matrix groups, in particular those that describe various types of motion in space. These are discussed in \cref{sec:matrix-grps}.

\begin{ctdefinition}[Group]
    \label{def:group}
    A \emph{\iindex{group}} is a monoid together with an ``inverse'' operation.
    In more detail, a group~$\grpA$ is
    \begin{body}
        \constit
        \begin{enumerate}
            \item a set~$\grpAset$;
            \item a binary operation~$\gtimes \colon \grpAset \cartprod \grpAset \sto \grpAset$, called \emph{composition};
            \item a specified element~$\idgrp \setin \grpAset$;
            \item a map~$\ginv\colon \grpAset \to \grpAset$, called \emph{inverse}.
        \end{enumerate}
        \condit
        \begin{enumerate}
            \item Associative law:~$(\grpela\mtimes \grpelb)\mtimes \grpelc=\grpela\mtimes (\grpelb\mtimes \grpelc), \quad \forall  \grpela, \grpelb, \grpelc \setin \grpAset$;
            \item Neutrality laws:~$\idmon\mtimes \grpela=\grpela=\grpela \gtimes \idgrp, \quad \forall  \grpela \setin \grpAset$;
            \item Inverse laws:
                  \begin{equation}
                      \label{eq:group-inverse-law}
                      \ginv(\monela) \mtimes \monela = \idmon = \monela \mtimes \ginv(\monela), \quad \forall  \grpela  \setin \grpAset.
                  \end{equation}
        \end{enumerate}
    \end{body}
\end{ctdefinition}

\begin{remark}
A group is called \emph{commutative} (or: \emph{abelian}) if its underlying semigroup is commutative.
\end{remark}

\begin{remark}
The size of the underlying set of a group is often called the \emph{order} or cardinality of the group. We'll sometimes use this terminology, or just call it the size of the group.  
\end{remark}

\subsection{Some examples}

\begin{example}
    The following is a group: the set~$\wnumbers$, together with addition as the composition operation, the element $0$ as neutral element, and ``taking the negative'' as the inverse operation:
    \begin{equation}
        \ginv(\grpela) \definedas -\grpela, \quad \forall \ \grpela \setin \wnumbers.
    \end{equation}
\end{example}
\begin{example}
    The monoid~$\tup{\reals_{\backslash \{ 0 \}}, \cdot, 1}$ becomes a group when equipped with the inverse operation defined by
    \begin{equation}
        \ginv(\grpela) \definedas \frac{1}{\grpela}, \quad \forall \grpela \setin \reals.
    \end{equation}
\end{example}

\begin{example}
    For the monoids~$\tup{\natnumbers, +,0}$ and~$\tup{\natnumbers,\cdot ,1}$ one cannot find an inverse operation that would turn these monoids into groups.
\end{example}

\begin{exercise}
    Can one find an inverse operation for the monoid~$\tup{\natnumbers,\max,0}$?
\end{exercise}
%
\begin{solution}
    No.
    Consider the condition~$0=\max(\ginv(\grpela),\grpela)$.
    In general, this is true only if $\grpela=\ginv(\grpela)=0$.
\end{solution}

\begin{example}
    The monoid~$\tup{\stylesets{B},\booland, \true}$ from \cref{ex:bool_monoid} cannot become a group, because there cannot by an inverse for~$\false$: there is no possible choice for $\ginv(\false)$ such that $\ginv(\false)\wedge \false=\true$.
\end{example}

\begin{example}
Given a set $\setA$, an invertible function $\setA \to \setA$ is called an \emph{automorphism} of $\setA$. There is a `naturally given' group structure on the set $\Autof\setA$ of automorphisms of $\setA$: we can take the composition operation to be the composition of functions, the neutral element is the identity function on $\setA$, and inverses are given by the inverses of functions.
\end{example}

\begin{example}\label{exa:grp-order-two}
    Consider the set $\makeset{ 0, 1 }$, equipped with the composition operation $\gtimes$ defined to be `addition modulo 2'. The composition table for this is \cref{tab:comp-table-add-mod-2}
    \begin{margintable}
    \caption{Addition modulo 2 on the set $\{ 0, 1 \}$.}
    \label{tab:comp-table-add-mod-2}
    \centering
    \begin{tabular}{c|cc}
        $+$ & $0$ & $1$ \\
        \hline
        $0$   & $0$ & $1$ \\
        $1$    & $1$ & $0$
    \end{tabular}
\end{margintable}

Let's choose $0$ as the neutral element, and let $\ginv(0) = 0$ and $\ginv(1) = 1$. Then $\tup{\{ 0, 1 \}, \gtimes, \idgrp, \ginv}$ is a group.
\end{example}


\begin{example}
    \label{grp-order-three}
    Consider the following set of complex numbers
    \begin{equation*}
        \Big\{ 1, e^{\frac{1}{3}2\pi i}, e^{\frac{2 }{3}2\pi i}  \Big\} \subseteq \cnumbers.
    \end{equation*}
    Taking the usual multiplication of complex numbers as the composition operation, these three numbers form a group.
\end{example}

\begin{gradedexercise}[\exname{GroupWithThreeElements}]
    \label{ex:GroupWithThreeElements}
    In \cref{grp-order-three}, what is the neutral element?
    What is the inverse operation?
    Draw the composition table for this group.
\end{gradedexercise}

\solutionof{GroupWithThreeElements}

\begin{margintable}
    \caption{The Klein four group}
    \label{tab:comp-table-Klein4}
    \centering
    \begin{tabular}{c|cccc}
        $+$ & $I$ & $V$ & $H$ & $R$ \\
        \hline
        $I$ & $I$ & $V$ & $H$ & $R$ \\
        $V$ & $V$ & $I$ & $R$ & $H$ \\
        $H$ & $H$ & $R$ & $I$ & $V$ \\
        $R$ & $R$ & $H$ & $V$ & $I$
    \end{tabular}
\end{margintable}

\begin{example}
    \label{exa:grp-Z4}
The set $\makeset{0, 1, 2, 3 }$ may be equipped with a group structure whose composition operation is addition modulo 4, and where $0$ is the neutral element. The composition table is $\cref{tab:comp-table-Z4}$.
    \begin{margintable}
    \caption{Cyclic group of order 4.}
    \label{tab:comp-table-Z4}
    \centering
    \begin{tabular}{c|cccc}
        $+$ & $0$ & $1$ & $2$ & $3$ \\
        \hline
        $0$ & $0$ & $1$ & $2$ & $3$ \\
        $1$ & $1$ & $2$ & $3$ & $0$ \\
        $2$ & $2$ & $3$ & $0$ & $1$ \\
        $3$ & $3$ & $0$ & $1$ & $2$
    \end{tabular}
\end{margintable}
\end{example}


\begin{example}
\label{exa:grp-Klein4}
Consider the shape of a rectangle in the plane, oriented vertically, and let's assume the rectangle is not a square. Then there are four symmetries of this shape:
\begin{enumerate}
\item doing nothing (leaving everything in place);
\item reflecting the shape along the vertical axis;
\item reflecting the shape along the horizontal axis;
\item rotating the shape by 180 degrees.
\end{enumerate}
Let us call these symmetries $I$, $V$, $H$, and $R$, respectively. We can model these for instance using bijective functions of the plane $\reals^2$. The set $\makeset{I, V, H, R}$ can be given the structure of a group, with $I$ being its neutral element, and with composition and inverses given as in \cref{tab:comp-table-Klein4}. We think of composition meaning that we perform one symmetry transformation and then the other (function composition), and inverses correspond to taking the inverse transformation. This group is known as the Klein four group, named after the mathematician Felix Klein.
\end{example}

\begin{figure*}
\centering
\includegraphics[height=2cm]{tetrahedron}
%\includegraphics{square}
%\includegraphics{octahedron}
%\includegraphics{icosahedron}
%\includegraphics{dodecahedron}
%    \caption{Platonic Solids}
%    \label{fig:platonic-solids}
\end{figure*}

\begin{example}
Similar to the previous example, all sorts of geometric shapes have groups of symmetries associated with them. 

For example, each of the five platonic solids have a group of symmetries associated with them. 

It turns out that the cube and the octahedron have the same group of symmetries, and similarly the icosahedron and the dodecahedron have the same symmetry group. This has to do with the fact that these platonic solids are dual to each other, respectively, in the following sense. An octahedron can be obtained from a cube by replacing faces with vertices and vertices with faces, and the same goes for the icosahedron and the dodecahedron.

The symmetry group of the tetrahedron has order 24, the symmetry group of the cube and octahedron has order 48, and the symmetry group of the icosahedron and dodecahedron has order 120.
\end{example}

\todotext{POST-ALPHUBEL include images/figures of the platonic solids}

\


\subsection{An alternative definition}

Many times groups are defined in a slightly different way than we have done. Namely as follows. 
 \begin{quote}
A group is:
 \begin{body}
        \constit
        \begin{enumerate}
            \item a set~$\grpAset$;
            \item a binary operation~$\gtimes \colon \grpAset \cartprod \grpAset \sto \grpAset$, called \emph{composition};
            \item a specified element~$\idgrp \setin \grpAset$.
        \end{enumerate}
        \condit
        \begin{enumerate}
            \item Associative law:~$(\grpela\mtimes \grpelb)\mtimes \grpelc=\grpela\mtimes (\grpelb\mtimes \grpelc), \quad \forall  \grpela, \grpelb, \grpelc \setin \grpAset$;
            \item Neutrality laws:~$\idmon\mtimes \grpela=\grpela=\grpela \gtimes \idgrp, \quad \forall  \grpela \setin \grpAset$;
            \item Inverses: for every $\grpela \in \grpAset$ there exists a $\grpelb \in \grpAset$ such that
                  \begin{equation}
                      \grpelb \mtimes \grpela = \idmon = \grpela \mtimes \grpelb.
                  \end{equation}
        \end{enumerate}
    \end{body}
    \end{quote}
Note that in this definition we no longer have an `inverse' map $\ginv$ as part of the constituents, and instead we have the existence of inverse elements as part of the conditions.  

It is possible to show that the two different definitions are equivalent in the sense that one may be derived from the other. The core of this is captured in the lemma and corollary below. The corollary says that if a monoid can be made into a group by adding an inverse operation, then the resulting group is uniquely determined by the underlying monoid. In other words, we can alternatively think of groups as special kinds of monoids which happen to have the \emph{property} of `being a group'.


\begin{lemma}
    \label{lem:inv-op-unique}
    Let~$\grpA = \tup{ \grpAset, \gtimes, \idgrp, \ginv }$ be a group and let~$\grpela, \grpelb \setin \grpAset$.
    If~$\grpela$ and~$\grpelb$ satisfy the equation
    \begin{equation}
        \grpela \gtimes \grpelb = \idgrp.
    \end{equation}
    Then~$\grpelb = \ginv( \grpela )$ and~$\grpela = \ginv(\grpelb)$.
\end{lemma}

\begin{proof}
    If~$\grpela \gtimes \grpelb = \idgrp$, then, by composing both sides of this equation with $\ginv(\grpela)$ from the left, and using associativity to remove brackets, we find~$\ginv(\grpela) \gtimes \grpela \gtimes \grpelb = \idgrp \gtimes \ginv(\grpela)$.
    Applying the inverse laws on the left-hand side, we obtain~$\idgrp \gtimes \grpelb = \idgrp \gtimes \ginv(\grpela)$, and using the neutrality laws on both side of this equation yields~$\grpelb =  \ginv(\grpela)$.
    The fact that~$\grpela = \ginv(\grpelb)$ may be proved similarly.
\end{proof}

\begin{corollary}
    \label{cor:inv-op-unique}
    Let~$\monoidA = \tup{\monoidAset, \mtimes,  \idmon }$ be a monoid.
    If~$\ginv_1$ and~$\ginv_2$ are both operations of inverse which make~$\monoidA$ into a group, then~$\ginv_1 =\ginv_2$ holds.
\end{corollary}

\begin{proof}
    Let~$\monela \setin \monoidAset$.
    Since~$\tup{\monoidAset, \mtimes,  \idmon, \ginv_2 }$ is a group, we have~$\monela \mtimes \ginv_2(\monela) = \idmon$.
    Also~$\tup{\monoidAset, \mtimes,  \idmon, \ginv_1}$ is a group, and in terms of this group, the equation~$\monela \mtimes \ginv_2(\monela) = \idmon$ implies, by \cref{lem:inv-op-unique}, that~$\ginv_2(\monela) = \ginv_1(\grpela)$, by thinking of~$\ginv_2(\monela)$ in the role of~$\grpelb$ from \cref{lem:inv-op-unique}.
    Since~$\grpela \setin \monoidAset$ was arbitrary,~$\ginv_2 = \ginv_1$ as functions on~$\monoidAset$.
\end{proof}


\showslides{
    \begin{forslides}
        \begin{equation}
            \label{eq:group-fullrank-isfullrank}
            \mdet(\mathbf{A}) \neq 0
        \end{equation}
        \begin{equation}
            \label{eq:group-fullrank-composition}
            \mathbf{A} \mtimes \mathbf{B}  \definedas \mathbf{A}\,\mathbf{B}
        \end{equation}
        \begin{equation}
            \label{eq:group-fullrank-composition-det}
            \mdet(\mathbf{A} \, \mathbf{B}) = \mdet(\mathbf{A}) \cdot  \mdet(\mathbf{B})
        \end{equation}
        \begin{equation}
            \label{eq:group-fullrank-inverse}
            \ginv(\mathbf{A}) \definedas {\mathbf{A}^{-1}}
        \end{equation}
        \begin{equation}
            \label{eq:group-fullrank-inverse-det}
            \mdet(\mathbf{A}^{-1}) = \frac{1}{\mdet(\mathbf{A})}
        \end{equation}
        \begin{equation}
            \label{eq:group-fullrank-inverse-det2}
            \mdet(\mathbf{A}^{-1}) = \left( \mdet(\mathbf{A}) \right) ^ {-1}
        \end{equation}
        \begin{equation}
            \label{eq:group-fullrank-inverse-comp}
            (\mathbf{A}\,\mathbf{B})^{-1} = {\mathbf{B}^{-1} \, \mathbf{A}^{-1}}
        \end{equation}
    \end{forslides}
}%\devel



\begin{gradedexercise}[\exname{GroupInverseProperties}]
    \label{ex:GroupInverseProperties}
    \label{lem:inv-op-properties}
    Let~$\tup{\grpAset, \gtimes, \idgrp, \ginv}$ be a group.
    Prove the following: 
    \begin{enumerate}
        \item\label{eq:group-neutral-invariant}~$\ginv(\idgrp) = \idgrp$;
        \item\label{eq:group-inverse-inverse}~$ \ginv(\ginv(\grpela)) = \grpela, \quad \forall \grpela \in \grpAset$;
        \item\label{eq:group-inverse-of-composition}~$\ginv(\grpela \gtimes \grpelb ) = \ginv(\grpelb) \gtimes \ginv(\grpela), \quad \forall \grpela, \grpelb \in \grpAset$.
    \end{enumerate}
\end{gradedexercise}

\solutionof{GroupInverseProperties}


\subsection{Subgroups}

\todotext{ Add some text and some examples to this subsection}
\todojira{454}{POST-ALPHUBEL: Add more to subgroups subsection}


\begin{definition}
Let $\grpA = \tup{\grpAset, \gtimes, \idgrp, \ginv}$ be a group. A \emph{subgroup} of $\grpA$ is:

\constit

\begin{enumerate}
\item A subset $\grpBset \subseteq \grpAset$.
\end{enumerate}

\condit

\begin{enumerate}
\item The set $\grpBset$ is closed under the composition operation $\gtimes$ from $\grpA$:
\begin{equation}
\prfsemi{\grpela \in \grpBset \quad \grpelb \in \grpBset}{\grpela \gtimes \grpelb \in \grpBset}
\end{equation}

\item The neutral element $\idgrp \in \grpAset$ is an element of $\grpBset$.

\item The set $\grpBset$ is closed under the inverse operation $\ginv$ from $\grpA$:
\begin{equation}
\prfperiod{\grpela \in \grpBset}{\ginv(\grpela) \in \grpBset}
\end{equation}
\end{enumerate}
\end{definition}



\begin{example}[Orthogonal matrices]

    Fix an integer~$n\geq 1$ and consider the set of real \emph{orthogonal} matrices~$\mat{A}\in \reals^{n\cartprod n}$:
   real, square matrices with orthonormal columns and rows.
    One way to express orthogonality of a matrix is:
    \begin{equation*}
        \mat{A}^\intercal \mat{A}=\mat{A}\mat{A}^\intercal = \idmat,
    \end{equation*}
    where~$\idmat$ is the identity matrix.
    The set~$\reals^{n\cartprod n}$ equipped with matrix multiplication as a binary operation, the identity matrix as the neutral element, and the transposition~$(\cdot)^\intercal$ (which for this specific type of matrices corresponds to the inverse) forms a group, usually denoted~$\text{O}(n)$.
    Any orthogonal matrix~$\mat{A}$ has the property~$\deter(\mat{A})\setin \{-1,1\}$.
    The subset of orthogonal~$n\cartprod n$ matrices with determinant 1 forms the so-called \emph{special} orthogonal group~$\mgSOn$.
\end{example}

\todotext{J: this is maybe redundant with material in the "matrix groups" section}
