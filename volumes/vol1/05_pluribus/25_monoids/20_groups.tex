% !TEX root = chapter-standalone.tex
\section{\statusdraft{Groups}}\label{sec:groups}
\begin{ctdefinition}[Group]
  \label{def:group}
  A \emph{\iindex{group}} is a monoid together with an ``inverse'' operation.
  In more detail, a group $\grpA$ is
  \begin{quote}
    \constit
    \begin{enumerate}
      \item a set~$\grpAset$;
      \item a binary operation~$\gtimes \colon \grpAset \cartprod \grpAset \sto \grpAset$, called \emph{composition};
      \item a specified element~$\idgrp \in \grpAset$;
      \item a map $\ginv: \grpAset \to \grpAset$, called \emph{inverse}.
    \end{enumerate}
    \condit
    \begin{enumerate}
      \item Associative law:~$(\grpela\mtimes \grpelb)\mtimes \grpelc=\grpela\mtimes (\grpelb\mtimes \grpelc) \hfill \forall \  \grpela, \grpelb, \grpelc \in \grpAset$;
       \item Neutrality laws:~$\idmon\mtimes \grpela=\grpela=\grpela \gtimes \idgrp \hfill \forall \  \grpela \in \grpAset$;
       \item Inverse laws:
       \begin{equation}\label{eq:group-inverse-law}
         \ginv(\monela) \mtimes \monela = \idmon = \ginv(\monela) \mtimes \monela \quad \quad \forall \  \grpela  \in \grpAset.
       \end{equation}
     \end{enumerate}
   \end{quote}
\end{ctdefinition}
\begin{example}
The following is a group: the set $\wnumbers$, together with addition as the composition operation, the element $0$ as neutral element, and ``taking the negative'' as the inverse operation:
\begin{equation}
\ginv(x) \definedas -x \quad \quad \forall \ x \in \wnumbers.
\end{equation}
\end{example}
\begin{example}
The monoid $\tup{\reals_{\backslash \{ 0 \}}, \cdot, 1}$ becomes a group when equipped with the inverse operation defined by
\begin{equation}
\ginv(x) \definedas \frac{1}{x} \quad \quad \forall \ x \in \reals.
\end{equation}
\end{example}
\begin{example}
    The monoid~$\tup{B,\wedge, \true}$ from \cref{ex:bool_monoid} cannot become a group, because there cannot by an inverse for~$\false$. In other words, there is no inverse for which~$\ginv(\false)\wedge \false=\true$.
\end{example}
\begin{example}\label{grp-order-three}
Consider the set $\grpAset \definedas \{ 1, e^{\frac{1}{3} 2 \pi}, e^{\frac{2 }{3} 2 \pi}  \} \subseteq \cnumbers$. Taking the usual multiplication of complex numbers as the composition operation, these three elements from a group.
\end{example}
\begin{exercise}[Group with three elements]
In \cref{grp-order-three}, what is the neutral element? What is the inverse operation?
Draw the multiplication table for this group.
\end{exercise}
\todotext{JL: insert examples of groups of order 1, 2, and 3 respectively}
\todotext{JL: insert example showing multiplication tables of the two possible group structures on a set of four elements}
\begin{example}[Square matrices with full rank]\label{exa:square-full}
Fix an integer $n\geq1 $ and consider the set of square matrices with full rank
\begin{equation} \label{eq:group-fullrank-issquare}
    \mathbf{A} \in \reals^{n\times n}
\end{equation}
\begin{equation} \label{eq:group-fullrank-isfullrank}
   \mdet(\mathbf{A}) \neq 0
\end{equation}
\begin{equation} \label{eq:group-fullrank-composition}
   \mathbf{A} \mtimes \mathbf{B}  \definedas \mathbf{A}\,\mathbf{B}
\end{equation}
\begin{equation} \label{eq:group-fullrank-composition-det}
   \mdet(\mathbf{A} \, \mathbf{B}) = \mdet(\mathbf{A}) \cdot  \mdet(\mathbf{B})
\end{equation}
\begin{equation} \label{eq:group-fullrank-inverse}
   \ginv(\mathbf{A}) \definedas {\mathbf{A}^{-1}}
\end{equation}
\begin{equation} \label{eq:group-fullrank-inverse-det}
   \mdet(\mathbf{A}^{-1}) = \frac{1}{\mdet(\mathbf{A})}
\end{equation}
\begin{equation} \label{eq:group-fullrank-inverse-det2}
   \mdet(\mathbf{A}^{-1}) = \left( \mdet(\mathbf{A}) \right) ^ {-1}
\end{equation}
\begin{equation} \label{eq:group-fullrank-inverse-comp}
   (\mathbf{A}\,\mathbf{B})^{-1} = {\mathbf{B}^{-1} \, \mathbf{A}^{-1}}
\end{equation}
\end{example}
\begin{lemma}\label{lemma-inv-op-properties}
Let $\tup{\grpAset, \gtimes, \idgrp, \ginv}$ be a group. Then
\begin{enumerate}
\item\label{eq:group-neutral-invariant} $\ginv(\idmon) = \idmon$;
\item\label{eq:group-inverse-inverse}  $ \ginv(\ginv(\grpela)) = \grpela \quad \quad \forall \  \grpela \in \grpAset$;
\item\label{eq:group-inverse-of-composition} $\ginv(\grpela \gtimes \grpelb ) = \ginv(\grpelb) \gtimes \ginv(\grpela) \quad \quad \forall \ \grpela, \grpelb \in \grpAset$.
\end{enumerate}
\end{lemma}
\begin{gradedexercise}[Properties of group inverse operation]
Prove \cref{lemma-inv-op-properties}.
\end{gradedexercise}
\todotext{Example-corollary: rototranslations SE(2), SE(3) }
