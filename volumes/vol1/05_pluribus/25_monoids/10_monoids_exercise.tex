% !TEX root = chapter-standalone.tex


\section{Code exercises}



\classlisting{Monoid}{}

\classlisting{FiniteMonoid}{}



\subsubsection*{Representation}
%
%\begin{longcode}
%\datafile{mon_min2}{boxit=False}
%\caption{}
%\label{lst:mon_min2}
%\end{longcode}

The file format for monoids is an extension of those for semigroups.
There is an extra field \fieldname{neutral} which gives you the neutral element.
\Cref{lst:monoid1} shows an example.

\begin{longcode}
    \caption{}
    \label{lst:monoid1}
    \begin{minted}{yaml}
carrier:
    elements: [0, 1]
composition:
    source:
        product:
        - elements: [0, 1]
        - elements: [0, 1]
    target:
        - elements: [0, 1]
    values:
       - [[0, 0], 0]
       - [[0, 1], 0]
       - [[1, 0], 0]
       - [[1, 1], 1]
neutral: 1
    \end{minted}
\end{longcode}



\begin{codeexercise}[\exname{TestFiniteMonoidRepresentation}]
    Create a function to load the data.
    %
\end{codeexercise}

\classlisting{FiniteMonoidRepresentation}{}

