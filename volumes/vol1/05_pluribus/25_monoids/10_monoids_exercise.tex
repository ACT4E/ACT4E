% !TEX root = chapter-standalone.tex

\section{\usebox{\chaptergear}
  Code exercises}

A \Monoid is a \Semigroup with an identity.

\classlisting{Monoid}{}


A \FiniteMonoid is a particular \FiniteSemigroup.

\classlisting{FiniteMonoid}{}

\subsubsection*{Representation}
%
%\begin{longcode}
%\datafile{mon_min2}{boxit=False}
%\caption{}
%\label{lst:mon_min2}
%\end{longcode}

The file format for monoids is an extension of those for semigroups.
There is an extra field \fieldname{neutral} which gives you the neutral element.
\Cref{fig:monoid2} shows an example.


\begin{marginfigure}%
    \yamldatafile{monoid2.monoid.yaml}{}
    \caption{The simple monoid $\{0,+1\}$ with multiplication}%
    \label{fig:monoid2}%
\end{marginfigure}%

% \begin{longcode}
%     \caption{}
%     \label{lst:monoid1}
%     \begin{minted}{yaml}
% carrier:
%     elements: [0, 1]
% composition:
%     source:
%         product:
%         - elements: [0, 1]
%         - elements: [0, 1]
%     target:
%         - elements: [0, 1]
%     values:
%        - [[0, 0], 0]
%        - [[0, 1], 0]
%        - [[1, 0], 0]
%        - [[1, 1], 1]
% neutral: 1
%     \end{minted}
% \end{longcode}

\begin{codeexercise}[\exname{TestFiniteMonoidRepresentation}]
    Create a function to load the data implementing the 
    interface in \cref{lst:FiniteMonoidRepresentation}.
    %
\end{codeexercise}

\classlisting{FiniteMonoidRepresentation}{}

