% !TEX root = chapter-standalone.tex

\section{Rings, fields}
\label{sec:rings-fields}

\todotext{J: @J: add material}

\todotext{@J: define rings}

\begin{ctdefinition}[Ring]
    \label{def:ring}
    A \maindef{ring} is
    \begin{body}
        \constit
        \begin{enumerate}
            \item a set~$\rngAset$;
            \item a binary operation~$\cdot \colon \rngAset \cartprod \rngAset \sto \rngAset$, called \emph{multiplication};
            \item a binary operation~$+ \colon \rngAset \cartprod \rngAset \sto \rngAset$, called \emph{addition};
            \item a specified element~$1 \setin \rngAset$ called \emph{one} (or \emph{unit});
            \item a specified element~$0 \setin \rngAset$ called \emph{zero}.
        \end{enumerate}
        \condit
        \begin{enumerate}
            \item $\tup{\rngAset, \cdot}$ is a monoid, with neutral element $1$;
            \item $\tup{\rngAset, +}$ is a commutative group, with neutral element $0$;
            \item Distributive law:
                  \begin{equation}\label{eq:distributive-law-rings}
                      \ela \cdot (\elb + \elc) = (\ela \cdot \elb) + (\ela \cdot \elc), \quad \forall \ela, \elb, \elc \setin \rngAset.
                  \end{equation}
        \end{enumerate}
    \end{body}
\end{ctdefinition}

\todotext{@J: example: integers}

\todotext{@J: example: rings of real valued functions}

\todotext{@J: example: polynomials}

\todotext{@J: example: tropical rings? applied examples of this? }

\todotext{@J: define fields}

\begin{definition}[Field]\label{def:field}
    A field is a ring $\tup{\fieldAset, \cdot, +, \rngunit, \rngzero}$ such that $\tup{\fieldAset \backslash \makeset{\rngzero}, \cdot}$ is a commutative group.
\end{definition}

\todotext{@J: example: reals, complex numbers}

\todotext{@J: example: rational functions}

\todotext{@J: example (skewfields): quaternions (include example about computer graphsics?? classical mechanics??)}