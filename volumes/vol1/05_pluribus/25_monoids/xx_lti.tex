\section{Discrete-time linear systems}
\begin{definition}[Discrete-time linear systems]
	\label{def:discrete-time-linear-system}
	A \emph{discrete-time linear time-invariant proper open system} is defined by three matrices~$\mat{A},\mat{B},\mat{C}$.
	Together they give a recurrence of the type
	\begin{equation}
		\label{eq:discrete-time-dynamics}
		\begin{aligned}
			\mat{x}_{k+1} & =  \mat{A} \mat{x}_k  + \mat{B} \mat{u}_k, \\
			\mat{y}_{k}   & =  \mat{C} \mat{x}_k.
		\end{aligned}
	\end{equation}
	If~$\mat{x}$ has dimension~$n\geq1$,~$\mat{u}$ dimension~$m\geq1$ and~$\mat{y}$ dimension~$p\geq1$, then~$\mat{A}$ has dimension~$n \times n$,~$\mat{B}$ has dimension~$n \times m$, and~$\mat{C}$ has dimension~$p \times n$.
\end{definition}

\begin{marginfigure}
	\centering
	\prftree{\includesag{20_dyn_1}}{\includesag{20_dyn_2}}{\includesag{20_dyn_1_2}}
	\caption{Composition of discrete-time linear systems.}
	\label{fig:comp_dyn_syst}
\end{marginfigure}
\todo{J: Is it intentional that there is no mention of a start state in the above?
	These systems are special cases of Moore machines, but for the latter we define them as coming equipped with a start state.
}
Consider now the systems where~$m=p$ (but possibly different~$n$):
these are systems with input and output of the same size.
Hence, we can compose them in series.
Series composition is the composition in which the output of a system is the input of another system (\cref{fig:comp_dyn_syst}).
The composition forms a new discrete-time linear system, which has the input of the first system and the output of the second system.
Let's consider the first system~$\dtsysa$:
%
\begin{equation*}
	\label{eq:dlin-1}
	\begin{aligned}
		\mat{x}_{k+1} & = \mat{A} \mat{x}_k + \mat{B} u_k, \\
		\mat{y}_{k}   & = \mat{C} \mat{x}_k,
	\end{aligned}
\end{equation*}
%
and the second system (which has~$\mat{y}$ as input)~$\dtsysb$:
%
\begin{equation*}
	\label{eq:dlin-2}
	\begin{aligned}
		\mat{z}_{k+1} & = \mat{E} \mat{z}_k + \mat{F} \mat{y}_k, \\
		\mat{v}_{k}   & = \mat{G} \mat{z}_k.
	\end{aligned}
\end{equation*}

We can write their composition~$\dtsysa\mtimes \dtsysb$ compactly as a discrete linear system with input~$\mat{u}$ and output~$\mat{v}$:
%
\begin{equation*}
	\label{eq:dlin-3}
	\begin{aligned}
		\begin{pmatrix}
			\mat{x}_{k+1} \\
			\mat{z}_{k+1}
		\end{pmatrix} & =
		\begin{pmatrix}
			\mat{A}  & \mat{0} \\
			\mat{FC} & \mat{E}
		\end{pmatrix}
		\begin{pmatrix}
			\mat{x}_k \\ \mat{z}_k
		\end{pmatrix}
		+
		\begin{pmatrix}
			\mat{B} \\ \mat{0}
		\end{pmatrix}\mat{u}_k, \\
		\mat{v}_k                 & =
		\begin{pmatrix}
			\mat{0} & \mat{G}
		\end{pmatrix}
		\begin{pmatrix}
			\mat{x}_k \\ \mat{z}_k
		\end{pmatrix}.
	\end{aligned}
\end{equation*}

We now want to show that the composition of discrete linear systems is associative (graphically reported in \cref{fig:ass_dyn_syst}).
To do so, we need a third system,~$\dtsysc$, with input~$\mat{v}$ and output~$\mat{w}$:
\begin{equation*}
	\label{eq:dlin-4}
	\begin{aligned}
		\mat{s}_{k+1} & = \mat{H} \mat{s}_k + \mat{I} \mat{v}_k, \\
		\mat{w}_{k}   & = \mat{J} \mat{s}_k.
	\end{aligned}
\end{equation*}

Furthermore, we want to show that~$(\dtsysa\mtimes \dtsysb)\mtimes \dtsysc=\dtsysa\mtimes (\dtsysb\mtimes \dtsysc)$.
\todo{J: This equality is maybe not formally true on the nose... but maybe yes... i guess we should try to make a more precise definition?
	If we want to see these as a special case of Moore machines (which seems desireable), then we won't have equality on the nose unless we use the trick AC suggested, using \textbf{SetL}.
}
Proceeding as we did before, we can write the composition~$(\dtsysa\mtimes \dtsysb)\mtimes \dtsysc$ as:

\begin{equation*}
	\label{eq:dlin-5}
	\begin{aligned}
		\begin{pmatrix}
			\mat{x}_{k+1} \\
			\mat{z}_{k+1} \\
			\mat{s}_{k+1}
		\end{pmatrix} & =
		\begin{pmatrix}
			\mat{A}  & \mat{0}  & \mat{0} \\
			\mat{FC} & \mat{E}  & \mat{0} \\
			\mat{0}  & \mat{IG} & \mat{H}
		\end{pmatrix}
		\begin{pmatrix}
			\mat{x}_k \\ \mat{z}_k\\ \mat{s}_k
		\end{pmatrix}+
		\begin{pmatrix}
			\mat{B} \\ \mat{0}\\ \mat{0}
		\end{pmatrix}\mat{u}_k, \\
		\mat{w}_k                  & =
		\begin{pmatrix}
			\mat{0} & \mat{0} & \mat{J}
		\end{pmatrix}
		\begin{pmatrix}
			\mat{x}_k \\ \mat{z}_k\\ \mat{s}_k
		\end{pmatrix}.
	\end{aligned}
\end{equation*}
We now want to write the composition~$\dtsysa\mtimes (\dtsysb\mtimes \dtsysc)$.
To do so, we start by writing~$(\dtsysb\mtimes \dtsysc)$ as:
\begin{equation*}
	\label{eq:dlin-6}
	\begin{aligned}
		\begin{pmatrix}
			\mat{z}_{k+1} \\
			\mat{s}_{k+1}
		\end{pmatrix} & =
		\begin{pmatrix}
			\mat{E}  & \mat{0} \\
			\mat{IG} & \mat{H}
		\end{pmatrix}
		\begin{pmatrix}
			\mat{z}_k \\ \mat{s}_k
		\end{pmatrix}+
		\begin{pmatrix}
			\mat{F} \\ \mat{0}
		\end{pmatrix}\mat{y}_k, \\
		\mat{w}_k                  & =
		\begin{pmatrix}
			\mat{0} & \mat{J}
		\end{pmatrix}
		\begin{pmatrix}
			\mat{z}_k \\ \mat{s}_k
		\end{pmatrix}.
	\end{aligned}
\end{equation*}
From this we can write~$\dtsysa\mtimes (\dtsysb\mtimes \dtsysc)$ as:
\begin{equation*}
	\label{eq:dlin-7}
	\begin{aligned}
		\begin{pmatrix}
			\mat{x}_{k+1} \\
			\mat{z}_{k+1} \\
			\mat{s}_{k+1}
		\end{pmatrix} & =
		\begin{pmatrix}
			\mat{A}  & \mat{0}  & \mat{0} \\
			\mat{FC} & \mat{E}  & \mat{0} \\
			\mat{0}  & \mat{IG} & \mat{H}
		\end{pmatrix}
		\begin{pmatrix}
			\mat{x}_k \\ \mat{z}_k\\ \mat{s}_k
		\end{pmatrix}+
		\begin{pmatrix}
			\mat{B} \\ \mat{0}\\ \mat{0}
		\end{pmatrix}\mat{u}_k, \\
		\mat{w}_k                  & =
		\begin{pmatrix}
			\mat{0} & \mat{0} & \mat{J}
		\end{pmatrix}
		\begin{pmatrix}
			\mat{x}_k \\ \mat{z}_k\\ \mat{s}_k
		\end{pmatrix},
	\end{aligned}
\end{equation*}
showing associativity.

\begin{figure}[tbh]
	\centering
	\prflinepadbefore=5pt
	\prflinepadafter=5pt
	\prfdouble{\prftree{\prftree{\includesag{20_dyn_1}}{\includesag{20_dyn_2}}{\includesag{20_dyn_1_2}}}{\prftree{\includesag{20_dyn_3}}{\includesag{20_dyn_3}}}{\includesag{20_dyn_12_3}}
	}
	{\prftree{\prftree{\includesag{20_dyn_1}}{\includesag{20_dyn_1}}}{\prftree{\includesag{20_dyn_2}}{\includesag{20_dyn_3}}{\includesag{20_dyn_2_3}}}{\includesag{20_dyn_1_23}}
	}
	\caption{Associativity law for the composition of discrete-time linear systems. }
	\label{fig:ass_dyn_syst}
\end{figure}
