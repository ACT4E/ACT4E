% !TEX root = chapter-standalone.tex

\section{Semigroups}
\label{sec:semigroups}

\linkvideo{spring2021-semi-mon-gro:mag-semi}
\begin{ctdefinition}[Semigroup]
    \label{def:semigroup}
    A \emph{\iindex{semigroup}}~$\sgrpA$ is:
    \begin{body}
        \constit
        \begin{enumerate}
            \item A set~$\sgrpAset$;
            \item A binary operation~$\mtimes  \colon \sgrpAset \cartprod \sgrpAset \sto \sgrpAset$ called \emph{composition}.
        \end{enumerate}
        \condit
        \begin{enumerate}
            \item Associative law
                  \begin{equation}
                      \label{eq:semigroup-associative}
                      (\sgrpela \mtimes   \sgrpelb)\mtimes   \sgrpelc
                      = \sgrpela \mtimes   (\sgrpelb \mtimes  \sgrpelc),
                  \end{equation}
                  for all~$\sgrpela , \sgrpelb, \sgrpelc \in \sgrpAset$.
        \end{enumerate}
    \end{body}
\end{ctdefinition}

\begin{remark}
    Given a fixed set~$\sgrpAset$, there will in general be many different choices of composition operation which make~$\sgrpAset$ into a semigroup.
    Therefore, technically, a semigroup $\sgrpA$ is a pair~$\tup{\sgrpAset, \mtimes}$ consisting of a set~$\sgrpAset$ and a choice of composition~$\mtimes.
    $
    The set~$\sgrpAset$ is the \emph{underlying set} of the semigroup.
    Often we will be slightly imprecise and refer to a semigroup simply by the name of its underlying set;
    this is practical when it is clear from context which multiplication operation we are considering, or when it is not necessary to refer to the multiplication explicitly.
\end{remark}

\begin{example}
    Pair-wise average on $\reals$,
    \begin{equation}
        \label{eq:pairwise}
        \monela \mtimes \monelb \definedas  \frac{\monela + \monelb}{2},
    \end{equation}
    does not define semigroup composition, because it is not associative.

    For example:
    \begin{equation}
        \label{eq:pairwise-counterexample}
        (4 \mtimes 8) \mtimes 16 = 11 \neq  8 = 4 \mtimes (8 \mtimes 16).
    \end{equation}
\end{example}

\begin{example}
    \label{exa:natnum-semigroup}
    Consider~$\tup{\natnumbers,+}$, defined as:
    \begin{equation*}
        \label{eq:natnum-semigroup-definition}
        \monela \mtimes \monelb \definedas  \monela + \monelb.
    \end{equation*}

    This is a semigroup, since, for all~$l,m, n \in \natnumbers$, we have
    \begin{equation}
        (l+m)
        +n = l+(m+n).
    \end{equation}
\end{example}

\showslides{
    \begin{forslides}
        \begin{equation}
            \label{eq:natnum-product}
            \monela \mtimes \monelb \definedas  \monela \cdot \monelb
        \end{equation}
        \begin{equation}
            \label{eq:natnum-min}
            \monela \mtimes \monelb \definedas  \min(\monela, \monelb)
        \end{equation}
        \begin{equation}
            \label{eq:natnum-max}
            \monela \mtimes \monelb \definedas  \max(\monela, \monelb)
        \end{equation}
    \end{forslides}
}

\begin{margintable}
    \caption{Composition table for booleans.}
    \label{tab:comp-table-bool}
    \centering
    \begin{tabular}{c|cc}
        $\booland$ & $\false$ & $\true$ \\
        \hline
        $\false$   & $\false$ & $\false$ \\
        $\true$    & $\false$ & $\true$
    \end{tabular}
\end{margintable}

\begin{example}[Booleans]
    \label{exa:booleans-table}
    \label{ex:bool_semigroup}
    Consider the set~$\stylesets{B}=\{\false, \true\}$, and~$\tup{\stylesets{B},\booland}$, where the operation~$\booland$ is defined via \cref{tab:comp-table-bool}.

    This forms a semigroup, given the associativity of~$\booland$.
\end{example}

\begin{gradedexercise}[\exname{CompositionTable}]
    \label{ex:CompositionTable}
    Consider the composition presented in \cref{tab:comp-table}.
    %    \begin{comment}
    %        \begin{center}
    %            \begin{tabular}{c|cc}
    %                $\mtimes$       & $\alphabetasymba$ & $\alphabetasymbb$ \\
    %                \hline
    %                \alphabetasymba & \alphabetasymba   & \alphabetasymba \\
    %                \alphabetasymbb & \alphabetasymbb   & \alphabetasymba
    %            \end{tabular}
    %        \end{center}
    %    \end{comment}
    Does this composition operation define a semigroup?
\end{gradedexercise}
\solutionof{CompositionTable}

\begin{exercise}[Cross-product]
    \label{ex:cross_prod}
    Consider~$\sgrpAset=\reals^3$ and the operation usually referred to as the ``cross-product'':
    \begin{equation*}
        \begin{pmatrix}
            a \\ b\\ c
        \end{pmatrix}
        \mtimes
        \begin{pmatrix}
            x \\ y \\ z
        \end{pmatrix}
        \definedas
        \begin{pmatrix}
            bz-cy \\
            cx-az \\
            ay-bx
        \end{pmatrix}.
    \end{equation*}
    This is a binary operation and therefore~$\tupp{\reals^3,\mtimes}$ forms a magma.
    Show that the this does not form a semigroup.
\end{exercise}
%
\begin{solution}
    Let's consider a counterexample.
    Clearly one has
    \begin{equation}
        \left(
        \begin{pmatrix}
            0 \\ 2\\ 1
        \end{pmatrix}\mtimes
        \begin{pmatrix}
            1 \\ 0 \\ 1
        \end{pmatrix}
        \right)
        \mtimes
        \begin{pmatrix}
            1 \\ 0 \\ 0
        \end{pmatrix}
        =
        \begin{pmatrix}
            0 \\-1\\-2
        \end{pmatrix}.
    \end{equation}
    However,
    \begin{equation}
        \begin{pmatrix}
            0 \\ 2\\ 1
        \end{pmatrix}\mtimes \left(\begin{pmatrix}
            1 \\ 0 \\ 1
        \end{pmatrix}\mtimes
        \begin{pmatrix}
            1 \\ 0 \\ 0
        \end{pmatrix}\right)
        =\begin{pmatrix}
            -2 \\0\\0
        \end{pmatrix},
    \end{equation}
    violating the associative law.
\end{solution}

\begin{example}
    \label{string-sgrp}
    \label{exa:string-semigroup}
    Consider a finite set~$\setA$, which we think of as an alphabet.
    For instance, consider
    %
    \begin{equation}
        \label{eq:string-semigroup-alphabet}
        \setA = \{ \alphabeta, \alphabetb \}.
    \end{equation}
    %
    Let~$\sgrpA$ be the set of non-empty strings of elements of~$\setA$.
    For example,
    %
    \begin{equation}
        \label{eq:string-semigroup-word}
        \alphabeta\alphabeta \alphabetb \alphabeta \alphabetb \alphabetb \alphabetb \alphabeta
    \end{equation}
    %
    is a non-empty string of elements of~$\setA$.

    We may define a multiplication operation on~$\sgrpA$ simply by concatenating strings.
    Given the strings
    \begin{equation*}
        \worda \text{ and } \wordb,
    \end{equation*}
    their concatenation
    \begin{equation*}
        \label{eq:string-semigroup-wordab}
        \worda \mtimes  \wordb
    \end{equation*}
    is the string
    \begin{equation*}
        \label{eq:string-semigroup-wordab-concat}
        \worda \wordb.
    \end{equation*}
    It is readily seen that concatenation satisfies the associative law, so~$\sgrpA$, together with this multiplication, forms a semigroup, often called \emph{free semigroup}.
\end{example}
\showslides{
    \begin{forslides}
        \begin{equation}
            \label{eq:string-semigroup-empty-string}
            \alphabeta\alphabeta \alphabetb \mtimes \tup{} =  \alphabeta\alphabeta \alphabetb
        \end{equation}
    \end{forslides}
}

\begin{gradedexercise}[\exname{VariationsOnConcatenation}]
    \label{ex:VariationsOnConcatenation}
    \label{ex:alphabet}
    Consider the set~$\sgrpA$ of finite non-empty strings of symbols from the alphabet~$\setA$, as in \cref{exa:string-semigroup}.

    Can you think of other candidates for multiplication operations on~$\sgrpA$, besides the straightforward concatenation of strings considered above?
    Do your candidates define semigroup multiplications -- that is, do they obey the associative law?

    For example, one might consider the operation where, given an ordered pair of strings, one first doubles the last symbol of the first string, and then concatenates.
    Is this operation associative?
    Justify your answers.
\end{gradedexercise}
\solutionof{VariationsOnConcatenation}

\begin{example}
    \label{max-semigroup}

    The function~$\max \colon \natnumbers \cartprod \natnumbers \sto \natnumbers$ defines a multiplication operation which equips~$\natnumbers$ with the structure of a semigroup.
    It is easy to show that it satisfies associativity.
    Given~$\sgrpela,\sgrpelb,\sgrpelc\in \natnumbers$, we have:

    \begin{equation*}
        \label{eq:max-assoc}
        \max(\max(\sgrpela,\sgrpelb),\sgrpelc)
        =
        \max(\sgrpela,\max(\sgrpelb,\sgrpelc)).
    \end{equation*}
\end{example}

\begin{exercise}
    \label{ex:max-semigroup}
    Verify the statement made in \cref{max-semigroup}; that is, check that the associative law holds.

    Does~$\min \colon \natnumbers \cartprod \natnumbers \sto \natnumbers$ also define a semigroup structure on~$\natnumbers$?
\end{exercise}
\begin{solution}
    Given~$\sgrpela,\sgrpelb,\sgrpelc\in \natnumbers$, we have:
    \begin{equation*}
        \label{eq:min-assoc}
        \min(\min(\sgrpela,\sgrpelb),\sgrpelc)=\min(\sgrpela,\min(\sgrpelb,\sgrpelc)).
    \end{equation*}
\end{solution}

\begin{marginfigure}
    \includesag{05_plants}
    \caption{Graphical representation of plant transitions.}
    \label{fig:plants_transitions}
\end{marginfigure}

\begin{example}
    \label{exa:plant-trafo-semigroup}
    \todographicsjira{9}{@Gioele: work on original icons}
    Consider the set~$\setA = \{ \sprout, \yng, \mature, \old, \dead \}$ which describes five possible states of a plant.
    Let~$\mapa \colon \setA \sto \setA$ be the function that describes ``development'' (\cref{fig:plants_transitions}):
    \begin{align*}
        \mapa(\sprout) & =  \yng, \\
        \mapa(\yng)    & =  \mature, \\
        \mapa(\mature) & =  \old, \\
        \mapa( \old)   & = \dead, \\
        \mapa (\dead)  & = \dead.
    \end{align*}
    % \begin{marginfigure}
    %     \includesag{05_plants}
    %     \caption{Graphical representation of plant transitions.}
    %     \label{fig:plants_transitions}
    % \end{marginfigure}
    In other words, we think of~$\mapa$ as the change of state of the plant during a given time interval (say, three months).
    Composing the function~$\mapa$ with itself corresponds to considering multiples of the given time interval.
    For example, the function
    \begin{equation*}
        (\mapa \mtimes \mapa \mtimes \mapa)
        \colon \setA \sto \setA
    \end{equation*}
    models the change over the course of nine months.
    In general, for the n-fold composition of~$\mapa$ with itself we write~$\mapa^n$.
    The set~$\sgrpA = \{ \mapa^n \mid n \in \natnumbers \}$, together with the multiplication given by the composition operation, forms a semigroup.
\end{example}
%
% \begin{example}
%     \label{exa:plant-trafo-semigroup}
%     \todographicsjira{9}{@Gioele: work on original icons}
%     Consider the set~$\setA = \{ \sprout, \yng, \mature, \old, \dead \}$ which describes five possible states of a plant.
%     Let~$\mapa \colon \setA \sto \setA$ be the function that describes ``development'' (\cref{fig:plants_transitions}):
%     \begin{align*}
%         \mapa(\sprout) & =  \yng, \\
%         \mapa(\yng)    & =  \mature, \\
%         \mapa(\mature) & =  \old, \\
%         \mapa( \old)   & = \dead, \\
%         \mapa (\dead)  & = \dead.
%     \end{align*}
%     In other words, we think of~$\mapa$ as the change of state of the plant during a given time interval (say, three months).
%     Composing the function~$\mapa$ with itself corresponds to considering multiples of the given time interval.
%     For example, the function
%     \begin{equation*}
%         \mapa \mtimes \mapa \mtimes \mapa \colon \setA \sto \setA
%     \end{equation*}
%     models the change over the course of nine months.
%     In general, for the n-fold composition of~$\mapa$ with itself we write~$\mapa^n$.
%     The set~$\sgrpA = \{ \mapa^n \mid n \in \natnumbers \}$, together with the multiplication given by the composition operation, forms a semigroup.
% \end{example}
\todotext{It looks premature to bring out the big guns here with the full definition of LTI
    systems - which is a category.
    Maybe skip until later.
    Or do the one-dimensional case, perhaps with ODEs.
    Morphisms of the form
    \begin{equation}
        \begin{aligned}
            \dot{x} & =  a\, x + b u, \\
            y       & =  c\,x.
        \end{aligned}
    \end{equation}
}
\begin{definition}[Discrete-time linear systems]
    \label{def:discrete-time-linear-system}
    A \emph{discrete-time linear time-invariant proper open system} is defined by three matrices~$\mat{A},\mat{B},\mat{C}$.
    Together they give a recurrence of the type
    \begin{equation}
        \label{eq:discrete-time-dynamics}
        \begin{aligned}
            \mat{x}_{k+1} & =  \mat{A} \mat{x}_k  + \mat{B} \mat{u}_k, \\
            \mat{y}_{k}   & =  \mat{C} \mat{x}_k.
        \end{aligned}
    \end{equation}
    If~$\mat{x}$ has dimension~$n\geq1$,~$\mat{u}$ dimension~$m\geq1$ and~$\mat{y}$ dimension~$p\geq1$, then~$\mat{A}$ has dimension~$n \times n$,~$\mat{B}$ has dimension~$n \times m$, and~$\mat{C}$ has dimension~$p \times n$.
\end{definition}

\begin{marginfigure}
    \centering
    \prftree{\includesag{20_dyn_1}}{\includesag{20_dyn_2}}{\includesag{20_dyn_1_2}}
    \caption{Composition of discrete-time linear systems.}
    \label{fig:comp_dyn_syst}
\end{marginfigure}
\todo{J: Is it intentional that there is no mention of a start state in the above?
    These systems are special cases of Moore machines, but for the latter we define them as coming equipped with a start state.
}
Consider now the systems where~$m=p$ (but possibly different~$n$):
these are systems with input and output of the same size.
Hence, we can compose them in series.
Series composition is the composition in which the output of a system is the input of another system (\cref{fig:comp_dyn_syst}).
The composition forms a new discrete-time linear system, which has the input of the first system and the output of the second system.
Let's consider the first system~$\dtsysa$:
%
\begin{equation*}
    \label{eq:dlin-1}
    \begin{aligned}
        \mat{x}_{k+1} & = \mat{A} \mat{x}_k + \mat{B} u_k, \\
        \mat{y}_{k}   & = \mat{C} \mat{x}_k,
    \end{aligned}
\end{equation*}
%
and the second system (which has~$\mat{y}$ as input)~$\dtsysb$:
%
\begin{equation*}
    \label{eq:dlin-2}
    \begin{aligned}
        \mat{z}_{k+1} & = \mat{E} \mat{z}_k + \mat{F} \mat{y}_k, \\
        \mat{v}_{k}   & = \mat{G} \mat{z}_k.
    \end{aligned}
\end{equation*}

We can write their composition~$\dtsysa\mtimes \dtsysb$ compactly as a discrete linear system with input~$\mat{u}$ and output~$\mat{v}$:
%
\begin{equation*}
    \label{eq:dlin-3}
    \begin{aligned}
        \begin{pmatrix}
            \mat{x}_{k+1} \\
            \mat{z}_{k+1}
        \end{pmatrix} & =
        \begin{pmatrix}
            \mat{A}  & 0       \\
            \mat{FC} & \mat{E}
        \end{pmatrix}
        \begin{pmatrix}
            \mat{x}_k \\ \mat{z}_k
        \end{pmatrix}
        +
        \begin{pmatrix}
            \mat{B} \\ 0
        \end{pmatrix}\mat{u}_k, \\
        \mat{v}_k        & =
        \begin{pmatrix}
            0 & \mat{G}
        \end{pmatrix}
        \begin{pmatrix}
            \mat{x}_k \\ \mat{z}_k
        \end{pmatrix}.
    \end{aligned}
\end{equation*}

We now want to show that the composition of discrete linear systems is associative (graphically reported in \cref{fig:ass_dyn_syst}).
To do so, we need a third system,~$\dtsysc$, with input~$\mat{v}$ and output~$\mat{w}$:
\begin{equation*}
    \label{eq:dlin-4}
    \begin{aligned}
        \mat{s}_{k+1} & = \mat{H} \mat{s}_k + \mat{I} \mat{v}_k, \\
        \mat{w}_{k}   & = \mat{J} \mat{s}_k.
    \end{aligned}
\end{equation*}

Furthermore, we want to show that~$(\dtsysa\mtimes \dtsysb)\mtimes \dtsysc=\dtsysa\mtimes (\dtsysb\mtimes \dtsysc)$.
\todo{J: This equality is maybe not formally true on the nose... but maybe yes... i guess we should try to make a more precise definition?
    If we want to see these as a special case of Moore machines (which seems desireable), then we won't have equality on the nose unless we use the trick AC suggested, using \textbf{SetL}.
}
Proceeding as we did before, we can write the composition~$(\dtsysa\mtimes \dtsysb)\mtimes \dtsysc$ as:
\todotext{Below there should be "fat zeros".
}
\begin{equation*}
    \label{eq:dlin-5}
    \begin{aligned}
        \begin{pmatrix}
            \mat{x}_{k+1} \\
            \mat{z}_{k+1} \\
            \mat{s}_{k+1}
        \end{pmatrix} & =
        \begin{pmatrix}
            \mat{A}  & 0        & 0       \\
            \mat{FC} & \mat{E}  & 0       \\
            0        & \mat{IG} & \mat{H}
        \end{pmatrix}
        \begin{pmatrix}
            \mat{x}_k \\ \mat{z}_k\\ \mat{s}_k
        \end{pmatrix}+
        \begin{pmatrix}
            \mat{B} \\ 0\\ 0
        \end{pmatrix}\mat{u}_k, \\
        \mat{w}_k                         & =
        \begin{pmatrix}
            0 & 0 & \mat{J}
        \end{pmatrix}
        \begin{pmatrix}
            \mat{x}_k \\ \mat{z}_k\\ \mat{s}_k
        \end{pmatrix}.
    \end{aligned}
\end{equation*}
We now want to write the composition~$\dtsysa\mtimes (\dtsysb\mtimes \dtsysc)$.
To do so, we start by writing~$(\dtsysb\mtimes \dtsysc)$ as:
\begin{equation*}
    \label{eq:dlin-6}
    \begin{aligned}
        \begin{pmatrix}
            \mat{z}_{k+1} \\
            \mat{s}_{k+1}
        \end{pmatrix} & =
        \begin{pmatrix}
            \mat{E}  & 0       \\
            \mat{IG} & \mat{H}
        \end{pmatrix}
        \begin{pmatrix}
            \mat{z}_k \\ \mat{s}_k
        \end{pmatrix}+
        \begin{pmatrix}
            \mat{F} \\ 0
        \end{pmatrix}\mat{y}_k, \\
        \mat{w}_k        & =
        \begin{pmatrix}
            0 & \mat{J}
        \end{pmatrix}
        \begin{pmatrix}
            \mat{z}_k \\ \mat{s}_k
        \end{pmatrix}.
    \end{aligned}
\end{equation*}
From this we can write~$\dtsysa\mtimes (\dtsysb\mtimes \dtsysc)$ as:
\begin{equation*}
    \label{eq:dlin-7}
    \begin{aligned}
        \begin{pmatrix}
            \mat{x}_{k+1} \\
            \mat{z}_{k+1} \\
            \mat{s}_{k+1}
        \end{pmatrix} & =
        \begin{pmatrix}
            \mat{A}  & 0        & 0       \\
            \mat{FC} & \mat{E}  & 0       \\
            0        & \mat{IG} & \mat{H}
        \end{pmatrix}
        \begin{pmatrix}
            \mat{x}_k \\ \mat{z}_k\\ \mat{s}_k
        \end{pmatrix}+
        \begin{pmatrix}
            \mat{B} \\ 0\\ 0
        \end{pmatrix}\mat{u}_k, \\
        \mat{w}_k                         & =
        \begin{pmatrix}
            0 & 0 & \mat{J}
        \end{pmatrix}
        \begin{pmatrix}
            \mat{x}_k \\ \mat{z}_k\\ \mat{s}_k
        \end{pmatrix},
    \end{aligned}
\end{equation*}
showing associativity.

\begin{figure}[tbh]
    \centering
    \prflinepadbefore=5pt
    \prflinepadafter=5pt
    \prfdouble{\prftree{\prftree{\includesag{20_dyn_1}}{\includesag{20_dyn_2}}{\includesag{20_dyn_1_2}}}{\prftree{\includesag{20_dyn_3}}{\includesag{20_dyn_3}}}{\includesag{20_dyn_12_3}}
    }
    {\prftree{\prftree{\includesag{20_dyn_1}}{\includesag{20_dyn_1}}}{\prftree{\includesag{20_dyn_2}}{\includesag{20_dyn_3}}{\includesag{20_dyn_2_3}}}{\includesag{20_dyn_1_23}}
    }
    \caption{Associativity law for the composition of discrete-time linear systems. }
    \label{fig:ass_dyn_syst}
\end{figure}

\subsection{Induced $n$-ary multiplication}
Given a semigroup~$\tup{\stylesets{S}, \mtimes}$, for each~$n \in \natnumbers$, we can define an induced $n$-ary multiplication operation
% 
\begin{equation*}
    \defmapperiodset{\mtimes^n}{\stylesets{S}^n}{\stylesets{S}}{\tup{s_1, s_2, \dots, s_n}}{ s_1 \mtimes s_2 \dots \mtimes s_n}
\end{equation*}

\todotextjira{10}{@J: insert something introducing a notion of “simplifying" or "computing ?
    insert exercise about computing/simplifying a multiplication using a multiplication table?
    Rewrite the above in function definition extended form; give a name.
}

Thanks to the associative law, this is well-defined -- that is, we do not need to use parentheses.
We will say that an element~$s \in \stylesets{S}$ is an \emph{n-fold multiplication} if it is in the image of this $n$-ary multiplication operation.
At times, we may not wish to specify the arity of the multiplication, in which case we just speak of a \emph{multiplication}.

\showslides{
    \begin{forslides}
        \begin{equation}
            \label{eq:sg-mora}
            \mora
        \end{equation}
        \begin{equation}
            \label{eq:sg-morb}
            \morb
        \end{equation}
        \begin{equation}
            \label{eq:sg-morab}
            \mora\then\morb
        \end{equation}
    \end{forslides}
}
