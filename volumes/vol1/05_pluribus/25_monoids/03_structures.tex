% !TEX root = chapter-standalone.tex

\section{Magmas}
\label{sec:structures}

Oftentimes we are going to study certain \emph{structures} and then their \emph{refinements}.
By \emph{refinement} we mean another type of structure that has additional properties/constraints.

The simplest algebraic structure that has to do with composition is that of a \emph{magma}: it just assumes that there is a set with a binary operation defined on it.

\begin{ctdefinition}[Magma]
    \label{def:magma}
    A \emph{\iindex{magma}}~$\sgrpA$ is a set~$\sgrpAset$, together with a binary operation
    \begin{equation}
        \mtimes  \colon \sgrpAset \cartprod \sgrpAset \sto \sgrpAset.
    \end{equation}
\end{ctdefinition}

\begin{margintable}
    \centering
    \caption{Composition table.}
    \label{tab:comp-table}
    \begin{tabular}{c|cc}
        $\mtimes$         & \stain{staincola} & \stain{white} \\
        \hline
        \stain{staincola} & \stain{staincola} & \stain{white} \\
        \stain{white}     & \stain{staincola} & \stain{white}
    \end{tabular}
\end{margintable}

Given a finite set~$\setA$, one way to specify a composition operation~$\mtimes$ on~$\setA$ is simply by writing out what it does with each pair of elements of~$\setA$.
Since~$\mtimes$ is a function of two variables, this can be conveniently displayed as a table, sometimes called a \emph{multiplication table} or a \emph{Cayley table}.
We will use the name \emph{composition table}.
Consider for example the set
\begin{equation*}
    \setA = \left\{ \stain{staincola}, \stain{white}\right\},
\end{equation*}
representing painting colors.
A composition operation~$\mtimes$ is specified in \cref{tab:comp-table}.
The rule describes the process of ``painting over'' another color, meaning that the last color that has been used to paint is the dominant one.
We read it as saying
\begin{equation*}
    \begin{aligned}
        \stain{staincola} \mtimes \stain{staincola} & = \stain{staincola}, \quad \stain{staincola} \mtimes \stain{white} = \stain{white}, \\
        \stain{white} \mtimes \stain{staincola}     & = \stain{staincola},  \quad \stain{white} \mtimes \stain{white} = \stain{white}.
    \end{aligned}
\end{equation*}

Magmas are quite general and simplistic.
There is not much to say about them.

We can build more interesting structures by considering, for example, properties that the composition operation might have.

In the next sections we will study:

\begin{itemize}
    \item \emph{semigroups}, a magma in which the operation is associative;
    \item \emph{monoids}, a semigroup with an identity;
    \item \emph{groups}, a monoid with an inverse operation.
\end{itemize}