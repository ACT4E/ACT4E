% !TEX root = chapter-standalone.tex

\section{Structures}
\label{sec:structures}

Oftentimes we are going to study certain \emph{structures} and then their \emph{refinements}.
By \emph{refinement} we mean another type of structure that has additional properties/constraints.

The simplest structure that has to do with composition is that of a \emph{magma}: it just assumes that there is a set with a binary operation defined on it.

\begin{ctdefinition}[Magma]
    \label{def:magma}
    A \emph{\iindex{magma}}~$\sgrpA$ is a set~$\sgrpAset$, together with a binary operation
    \begin{equation}
        \mtimes  \colon \sgrpAset \cartprod \sgrpAset \sto \sgrpAset.
    \end{equation}
\end{ctdefinition}

\begin{margintable}
    \centering
    \caption{Composition table.}
    \label{tab:comp-table}
    \begin{tabular}{c|cc}
        $\mtimes$ & $\stea$ & $\smilk$ \\
        \hline
        \stea     & \stea   & \stea \\
        \smilk    & \smilk  & \stea
    \end{tabular}
\end{margintable}

Given a finite set~$\setA$, one way to specify a composition operation~$\mtimes$ on~$\setA$ is simply by writing out what it does with each pair of elements of~$\setA$.
Since~$\mtimes$ is a function of two variables, this can be conveniently displayed as a table, sometimes called a \emph{multiplication table} or a \emph{Cayley table}.
We will use the name \emph{composition table}.
Consider for example the set~$\setA = \{ \stea, \smilk\}$.
A composition operation~$\mtimes$ is specified in \cref{tab:comp-table}.
We read it as saying
\begin{equation*}
    \stea \mtimes \stea = \stea, \quad \stea \mtimes \smilk = \stea, \quad \smilk \mtimes \stea = \smilk,  \quad \smilk \mtimes \smilk = \stea.
\end{equation*}

Magmas are quite general and simplistic.
There is not much to say about them.

We can build more interesting structures by considering, for example, properties that the composition operation might have.

In the next sections we will study:

\begin{itemize}
    \item \emph{semigroups}, a magma in which the operation is associative;
    \item \emph{monoids}, a semigroup with an identity;
    \item \emph{groups}, a monoid with an inverse operation.
\end{itemize}