% !TEX root = chapter-standalone.tex

\section{Magmas}
\label{sec:structures}

Oftentimes we are going to study certain \emph{structures} and then their \emph{refinements}.
By \emph{refinement}, we mean another type of structure that has additional properties or constraints.

The simplest algebraic structure that has to do with composition is that of a \emph{magma}: it just assumes that there is a set with a binary operation defined on it.

\begin{ctdefinition}[Magma]
    \label{def:magma}
    A \maindef{magma}~$\sgrpA$ is a set~$\sgrpAset$, together with a binary operation
    \begin{equation}
        \mtimes \colon \sgrpAset \cartprod \sgrpAset \sto \sgrpAset,
    \end{equation}
    which we refer to as ``composition''.
\end{ctdefinition}

\begin{margintable}
    \centering
    \caption{Composition table.}
    \label{tab:comp-table}
    \begin{tabular}{c|cc}
        $\mtimes$         & \stain{staincola} & \stain{white} \\
        \hline
        \stain{staincola} & \stain{staincola} & \stain{white} \\
        \stain{white}     & \stain{staincola} & \stain{white}
    \end{tabular}
\end{margintable}

Given a finite set~\setA, one way to specify a composition operation~$\mtimes$ on~\setA is simply by writing out what it does with each pair of elements of~\setA.
Since~$\mtimes$ is a function of two variables, this can be conveniently displayed as a table, sometimes called a \emph{multiplication table} or a \emph{Cayley table}.
We will use the name \emph{composition table}.

\begin{example}
    Consider the set
    %vmiddle not adequate for this case
    \begin{equation}
        \setA = \makesett{\hspace{0.05cm} \vstain{staincola}\hspace{0.05cm},\hspace{0.05cm} \vstain{white}\hspace{0.05cm}},
    \end{equation}
    representing painting colors.
    A composition operation~$\mtimes$ is specified in \cref{tab:comp-table}.
    The rule describes the process of ``painting over'' another color, meaning that the last color that has been used to paint is the dominant one.
    We read it as saying
    \begin{center}
        \setlength{\tabcolsep}{20pt}
        \begin{tabular}{cc}
            $\vstain{staincola} \mtimes \vstain{staincola} = \vstain{staincola}$ & $\vstain{staincola} \mtimes \vstain{white} = \vstain{white}$ \\
            $\vstain{white} \mtimes \vstain{staincola} = \vstain{staincola}$     & $\vstain{white} \mtimes \vstain{white} = \vstain{white}$
        \end{tabular}
    \end{center}
\end{example}

\begin{definition}
    \label{def:magma-commutativity}
    \SYNDEF{commutative magma}
    A \SY{magma}~$\sgrpA = \tup{\sgrpAset, \mtimes}$ is called \emph{commutative} (or: Abelian) if
    \begin{equation}
        \label{eq:semigroup-commutativity}
        \sgrpela \mtimes \sgrpelb = \sgrpelb \mtimes \sgrpela,
    \end{equation}
    for all~$\sgrpela, \sgrpelb \setin \sgrpAset$.
\end{definition}

\begin{exercise}
    Is the \SY{magma} specified in \cref{tab:comp-table} commutative?
\end{exercise}
\begin{solution}
    No.
\end{solution}

Magmas are quite general and simplistic.
There is not all much to say about them.

We can build more interesting structures by considering, for example, properties that the composition operation might have.

In the next sections we will study:

\begin{itemize}
    \item \emph{\SY{semigroups}}, \SY{magmas} in which the operation is associative;
    \item \emph{\SY{monoids}}, \SY{semigroups} with an identity;
    \item \emph{\SY{groups}}, \SY{monoids} with an inverse operation.
\end{itemize}
