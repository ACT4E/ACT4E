% !TEX root = chapter-standalone.tex


\section{Structures}
\label{sec:structures}

Oftentimes we are going to study certain \emph{structures} and then their \emph{refinements}.
By \emph{refinement} we mean another type of structure that has additional properties/constraints.

The simplest structure that has to do with composition is that of a \emph{magma}: it just assumes that there is a set with a binary operation defined on it.

\begin{definition}[Magma]
    \label{def:magma}
    A \emph{\iindex{magma}}~$\sgrpA$ is a set~$\sgrpAset$, together with a binary operation
    \begin{equation}
        \mtimes  \colon \sgrpAset \cartprod \sgrpAset \sto \sgrpAset.
    \end{equation}
\end{definition}


\begin{margintable}
    \caption{Composition table.}
    \label{tab:comp-table}
    \begin{tabular}{c|cc}
        $\mtimes$       & $\alphabetasymba$ & $\alphabetasymbb$ \\
        \hline
        \alphabetasymba & \alphabetasymba   & \alphabetasymba   \\
        \alphabetasymbb & \alphabetasymbb   & \alphabetasymba
    \end{tabular}
\end{margintable}


Given a finite set~$\setA$, one way to specify a composition operation~$\mtimes$ on~$\setA$ is simply by writing out what it does with each pair of elements of~$\setA$.
Since~$\mtimes$ is a function of two variables, this can be conveniently displayed as a table, sometimes called a \emph{multiplication table} or a \emph{Cayley table}.
We will use the name \emph{composition table}.
Consider for example the set~$\setA = \{ \alphabetasymba, \alphabetasymbb\}$.
A composition operation~$\mtimes$ is specified in \cref{tab:comp-table}.
We read it as saying
\begin{equation*}
    \alphabetasymba \mtimes \alphabetasymba = \alphabetasymba, \quad \alphabetasymba \mtimes \alphabetasymbb = \alphabetasymba, \quad \alphabetasymbb \mtimes \alphabetasymba = \alphabetasymbb,  \quad \alphabetasymbb \mtimes \alphabetasymbb = \alphabetasymba.
\end{equation*}


Magmas are quite general and simplistic.
We can build more interesting structures by considering, for example, properties that the composition operation might have.
