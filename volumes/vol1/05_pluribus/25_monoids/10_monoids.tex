% !TEX root = chapter-standalone.tex


\section{\statusdraft{Monoids}}
\label{sec:parallelism-monoids}


Algebraic structures are often defined in \emph{layers}.
For example, in the definition of semigroup, we start with a set~$\sgrpA$ as a basic building block, and we add a layer of structure to it, namely a multiplication operation~$\mtimes \colon \sgrpA \cartprod \sgrpA \sto \monoidA$.
The multiplication operation for semigroups was not only a new \emph{structure} that we added, but we also required that this structure obey a \emph{condition}, namely that it satisfy the associative law.
One might also say that the multiplication operation was a new \emph{constituent} or a new \emph{datum}, and that satisfying the associative law is a \emph{property}.
Mathematicians often use such words in an intuitive, non-rigorous way as a tool for structuring their thinking.
We will do the same.
For clarity, we will aim to stick with the words \emph{constituents} and \emph{conditions}.
Roughly speaking, we think of constituents as building blocks, and we think of conditions as rules for how those blocks fit together and behave.

Using the constituent vs. condition distinction we will, in particular, present some definitions in the following succinct, list-like fashion:

\begin{ctdefinition}[Monoid]
  \label{def:monoid}
  A \emph{\iindex{monoid}}  $\monoidA$  is:
  \begin{quote}
    \constit
    \begin{enumerate}
      \item a set~$\monoidAset$;
      \item a binary operation~$\mtimes  \colon \monoidAset \cartprod \monoidAset \sto \monoidAset$;
      \item a specified element~$\idmon \in \monoidAset$, called \emph{neutral element}.
    \end{enumerate}
    \condit
    \begin{enumerate}
      \item Associative law:~$(\monela\mtimes  \monelb)\mtimes  \monelc=
      \monela\mtimes  (\monelb\mtimes  \monelc) \hfill  \forall \  \monela, \monelb, \monelc \in \monoidAset$;
      \item Neutrality Laws:~$\idmon\mtimes \monela=\monela=\monela\mtimes  \idmon \hfill  \forall \ \monela \in \monoidAset $.
    \end{enumerate}
  \end{quote}
\end{ctdefinition}


\begin{remark}
  The way that we presented the definition of a monoid is certainly not unique. For example, we could have done the following.

  \begin{quote}
    A \emph{\iindex{monoid}} is:
    \begin{quote}
      \constit
      \begin{enumerate}
        \item a semigroup~$\tup{\monoidAset, \mtimes}$;
        \item a specified element~$\idmon\in \monoidAset$, called \emph{neutral element}.
      \end{enumerate}
      \condit
      \begin{enumerate}
        \item Neutrality laws:~$\idmon\mtimes \monela= \monela= \monela \mtimes \idmon$.
      \end{enumerate}
    \end{quote}
  \end{quote}
  In this version, two constituents and one condition from \cref{def:monoid} are ``compressed'' into the information that we are using here a semigroup as a constituent. This kind of ``compression'' has its pros and cons; depending on the context will use it to varing degrees.

  There is a similar dilemma when considering the software interfaces to describe these structure.
  In terms of software engineering, the two strategies are \emph{composition} (a monoid has a semigroup as a constituent)
  and \emph{inheritance} (a monoid \emph{is} a semigroup with additional data).

\end{remark}



\begin{example}
  Consider~$\tup{\reals,+,0}$.

  This is a monoid, since, for all~$x,y, z \in \reals$, we have
  \begin{equation*}
  (x+y)
    +z=x+(y+z),
  \end{equation*}
  and
  \begin{equation*}
    x+0=x=0+x.
  \end{equation*}
\end{example}

\begin{example}
  The set~$\wnumbers$, together with the operation of multiplication of whole numbers, forms a monoid. The neutral element is the number $1$.
\end{example}

\begin{example}
  \label{ex:bool_monoid}
  Consider~$\tup{B,\wedge}$ as in \cref{ex:bool_semigroup}, and consider~$\true$ as neutral element. This forms a monoid, since~$b\wedge \true=b=\true \wedge b$, for all~$b\in B$.
\end{example}

\begin{lemma}
  \label{neut-el-unique}
  Let~$\tup{\sgrpA, \mtimes}$ be a semigroup. If there exists elements~$1 \in M$ and~$1' \in M$ such that~$\tup{\sgrpA, \mtimes, 1}$ and $\tup{\sgrpA, \mtimes, 1'}$ are each monoids, then~$1 = 1'$ must hold.
  In other words, the neutral element of a monoid is uniquely determined by the underlying semigroup structure.
\end{lemma}

\begin{gradedexercise}[Neutral elements of monoids are unique]
  Prove \cref{neut-el-unique}.
\end{gradedexercise}
\begin{solution}
  \todo{to write solution}
\end{solution}

\begin{example}
  Consider~$\tup{\nonNegReals,\max,0}$. This is a monoid, since, for all~$x,y\in \nonNegReals$, we have:
  \begin{equation*}
    \max(\max(x,y),z)=\max(x,\max(y,z)),
  \end{equation*}
  and
  \begin{equation*}
    \max(x,0)=x=\max(0,x).
  \end{equation*}
\end{example}

\begin{remark}
Note that in the above example, we could have just as well instead considered the set $\reals_{\geq 7.5}$ of real numbers greater than $7.5$, together with ``$\max$'' as composition and $7.5$ as neutral element. In other words, we can choose any real number $a \in \reals$ and obtain a monoid $\tup{\reals_{\geq a},\max, a}$.
\end{remark}




\begin{example}
We can extend the semigroup introduced in \cref{string-semigroup} by defining the notion of an empty string~$\tup{}$. This functions as a neutral element, so we obtain a monoid. 
 \end{example}


%\section{Dynamical systems and monoids}
%
%\AC{in the end I would make this only a simple example of monoid - no introduction of group etc.}
%\JL{inserting this here as an un-baked idea for a subsection. maybe it could be the first subsection of this chapter; that way idendity laws and associative laws can be introduced before talking about categories}
%\gray{
%  What are the simplest kinds of mathematical models of a dynamical system that we can think of?
%
%  One possible answer is something like this: we can describe a dynamical system as a set $S$ of possible states, together with a description of how states change over time. For the latter, consider time to be labeled by distinct ``points in time''. Then, we can just think in terms of time-steps, \eg  seconds, or we can think of points in time where \eg  an action is triggered and the system passes to a new state.
%
%  One thing we want to describe is how the state of our system changes over time, and in particular from one moment in time to the next. For any time step, we will not assume that we know what specific state the system is in, but rather we will describe, at once, all possible evolutions during that time step, \ie  we consider all possible initial conditions at once. Given two consecutive moments in time, we might describe the possible changes in the system by a function $T : S \rightarrow S$, which maps each state $s \in S$ to a next state $T(s) \in S$. This is a deterministic change of state: given $s$, the function $T$ determines the next state $T(s)$. The function $T$ is like a rule. Let's call $T$ an ``evolution operator'', because it describes how the system states might evolve over a time step.
%
%  We might want to consider various possible evolution operators. We could consider functions $T_a$, $T_b$, $T_c$, \etc. We can also compose these functions: given $T_a$ and $T_b$, we might have, over the course of two time steps, the change described by $T_a \circ T_b$. For simplicity, let's suppose we work with three evolution operations $T_a$, $T_b$, and $T_c$.
%
%  \
%
%  -> introduce semigroups (implicitly or explicitly)
%
%  \
%
%
%  -> introduce monoids
%}

\begin{example}[Transition functions]
\label{exa:transition-functions}
  \begin{definition}[Continuous-time dynamical system]
    \label{def:ct-dynsyst}
    A dynamical system on~$\reals^n$ may be defined by a function
    \begin{equation}
      f\colon \reals^n \to \reals^n.
    \end{equation}
    A trajectory of a dynamical system is a function~$x\colon \reals \to \reals^n$ such that
    \begin{equation}
      \dot{x}_t = f(x_t).
    \end{equation}
    We use the notation~$\dot{x}$ to abbreviate~$\dd x / \dd t$.
  \end{definition}

  Suppose that we are dealing with dynamical systems such that for any point~$x_0$, there
  is exactly one trajectory beginning at~$x_0$. (For this we need to put reasonable constraints
  on the function~$f$.)

  We can then ask the following: given a point~$x$, where would its trajectory be after~$\delta$?
  This question induces a family of functions~$T_\delta$, called transition functions.
  For each particular~$\delta$, we have a function
  \begin{equation*}
    T_{\delta}\colon \reals^n \to  \reals^n
  \end{equation*}
  that maps a point to its position~$\delta$ in the future.

  We can spot here a semigroup structure. Suppose we want to know the position of a point~$\delta_1 + \delta_2$
  in the future. We can take~$T_{\delta_1}$ and compose it with~$T_{\delta_2}$; or take directly~$T_{\delta_1 + \delta_2}$.
  By construction we will have that
  \begin{equation}
    \label{eq:transition-property}
    T_{\delta_1 + \delta_2} = T_{\delta_1} \then T_{\delta_2}
  \end{equation}
  We can also easily prove associativity:
  \begin{equation}
  \begin{aligned}
    T_{\delta_1}\then (T_{\delta_2}\then T_{\delta_3})&=T_{\delta_1}\then T_{\delta_2+\delta_3}\\
    &=T_{\delta_1+\delta_2+\delta_3}\\
    &=T_{\delta_1+\delta_2}\then T_{\delta_3}\\
    &=(T_{\delta_1}\then T_{\delta_2})\then T_{\delta_3}.
\end{aligned}
\end{equation}

  This shows that the set of transition functions for a particular system with the operation of function composition form a semigroup.

  This semigroup is a monoid because there is an identity. The identity is~$T_0$, the map that tells us
  what happens after~$0$ seconds. That is~$\id_{\reals^n}$, the identity on~$\reals^n$.
  To show that~$T_0 = \id_{\reals^n}$ is an identity, we can fix any~$\delta$ and substituting in \cref{eq:transition-property} we have
  \begin{equation*}
    \begin{aligned}
      T_{\delta + 0}&= T_{\delta} \then T_{0} \\
      = & T_{\delta}
    \end{aligned}
  \end{equation*}

\end{example}


\begin{exercise}[Discrete-time linear systems]
  In \cref{ex:discrete-time-linear} we have constructed the semigroup of linear discrete-time dynamical systems.

  \begin{enumerate}
    \item Show that it is not a monoid, by showing that one cannot find an identity.
    \item Suggest an extension to \cref{def:dicrete-time-linear-system} so that you can obtain a monoid.
  \end{enumerate}
\end{exercise}
\begin{solution}
  \todotext{to complete}
  Consider this formalization
   \begin{equation} \label{eq:discrete-time-dynamics-D}
      \begin{aligned}
        x_{k+1} &=& \mathbf{A} x_k & + \mathbf{B} u_k \\
        y_{k}   &=& \mathbf{C} x_k & + \mathbf{D} u_k
      \end{aligned}
    \end{equation}
  This introduces a direct I/O link that allows to not have delay.


  \AC{The extension needs to introduce the matrix $D$ to have a direct i-o signal.
  ALSO, we need to change the formalization to allow $n=0$, otherwise you cannot make an identity
  because the state space grows by at least 1 when you compose with it.}
\end{solution}

\begin{comment}
\begin{equation} \label{eq:example-reals-plus}
\tupp{\reals, +}
\end{equation}

\begin{equation} \label{eq:example-reals-mult}
\tupp{\reals, \cdot}
\end{equation}

\begin{equation} \label{eq:example-reals-min}
\tupp{\reals, \min}
\end{equation}

\begin{equation} \label{eq:example-reals-max}
\tupp{\reals, \max}
\end{equation}

\begin{equation} \label{eq:example-nats-plus}
\tupp{\natnumbers, +}
\end{equation}

\begin{equation} \label{eq:example-nats-mult}
\tupp{\natnumbers, \cdot}
\end{equation}

\begin{equation} \label{eq:example-nats-min}
\tupp{\natnumbers, \min}
\end{equation}

\begin{equation} \label{eq:example-nats-max}
\tupp{\natnumbers, \max}
\end{equation}

\begin{equation} \label{eq:example-nats21-plus}
\tupp{ \{x\in\natnumbers: x \geq 2021 \}, +}
\end{equation}
\begin{equation} \label{eq:example-nats21-mult}
\tupp{ \{x\in\natnumbers: x \geq 2021 \}, \cdot}
\end{equation}
\begin{equation} \label{eq:example-nats21-min}
\tupp{ \{x\in\natnumbers: x \geq 2021 \}, \min}
\end{equation}
\begin{equation} \label{eq:example-nats21-max}
\tupp{ \{x\in\natnumbers: x \geq 2021 \}, \max}
\end{equation}


\end{comment}


\begin{comment}
\begin{equation} \label{eq:example-monoid-reals-plus-zero}
\tupp{\reals, +, 0}
\end{equation}

\begin{equation} \label{eq:example-monoid-reals-mult-one}
\tupp{\reals, \cdot, 1}
\end{equation}


\begin{equation} \label{eq:example-monoid-nats-plus-zero}
\tupp{\natnumbers, +, 0}
\end{equation}

\begin{equation} \label{eq:example-monoid-nats-mult-one}
\tupp{\natnumbers, \cdot, 1}
\end{equation}


\begin{equation} \label{eq:example-monoid-nats-max-zero}
\tupp{\natnumbers, \max, 0}
\end{equation}
\end{comment}
