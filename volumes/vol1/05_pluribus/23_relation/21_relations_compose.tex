
\section{Composing relations}

\linkvideo{spring2021-relations:relations:comp-rel} % Composing relations

The visualization in \cref{fig:example_rel} hints at the fact that we can compose relations
if they have compatible source and target.

To illustrate the composition rule in \cref{eq:RelCompRule} for relations, let us consider a simple example, involving sets~$\setA$,~$\setB$, and~$\setC$, and relations~$\relA \colon \setA \to \setB$ and~$\relB \colon \setB \to \setC$, as depicted graphically below in \cref{fig:example_rel_composable}.
%
\begin{figure}[h!]
    \centering
    \subfloat[Relations compatible for composition. \label{fig:example_rel_composable}]{
        \includesag{30_rel_2}}\\
    \subfloat[Composition of relations. \label{fig:example_rel_composed}]{
        \includesag{30_rel_3}
    }

    \caption{Illustrations for relations composition.}
    \todographics{Why red composition symbol?}
\end{figure}
%
Now, according to the rule in \cref{eq:RelCompRule}, the composition~$\relA \mthen \relB \colon \setA \mto \setC$ will be such that~$\inrel \ela {(\relA \mthen \relB)} \elc$ if and only if there exists some~$\elb \in \setB$ such that~$\inrel \ela \relA \elb $ and~$\inrel \elb \relB{\elc}$, which, graphically, means that for~$\tup{\ela,\elc}$ to be an element of the relation~$\relA \mthen \relB$,~$\ela$ and~$\elb$ need to be connected by at least one sequence of two arrows such that the target of the first arrow is the source of the second.
For example, in \cref{fig:example_rel_composable}, there is an arrow from~$\sfondue$ to~$\swater$, and from there on to~$\sapple$, and therefore, in the composition~$\relA \mthen \relB$ depicted in~\cref{fig:example_rel_composed}, there is an arrow from~$\sfondue$ to~$\sapple$.
