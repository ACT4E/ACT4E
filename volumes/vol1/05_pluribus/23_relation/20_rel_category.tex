
\section{The category of relations \Rel}
\todostructure{Obviously move later.}
\linkvideo{spring2021-relations:relations:cat-rel} % Category of relations

Now that we know how relations compose, it is natural to think of a relation~$\relA \subseteq \setA \cartprod \setB$ as a \emph{morphism} from~$\setA$ to~$\setB$.

\begin{ctdefinition}[Category \Rel]
    \label{def:Rel}
    The category of relations \iindex{\Rel}  is given by:
    \begin{enumerate}
        \item \emph{Objects}: The objects of this category are all sets.
        \item \emph{Morphisms}: Given sets~$\Obja, \Objb$, the homset~$\HomSet{\Rel}{\Obja}{\Objb}$ consists of all relations~$\relA\subseteq \Obja\cartprod \Objb$.
        \item \emph{Identity morphisms}: Given a set~$\Obja$, its identity morphism is
              \begin{equation}
                  \catid_\Obja \definedas \{ \tupp{\ela,\elb} \in \Obja \cartprod \Obja \mid  \ela = \elb \}.
              \end{equation}
        \item \emph{Composition}: Given relations~$\relA \colon \Obja\mto \Objb$,~$\relB \colon \Objb\mto \Objc$, their composition is given by
              \begin{equation}
                  \label{eq:RelCompRule}
                  \relA \mthen \relB \definedas \{\tup{\ela,\elc} \in \Obja \cartprod \Objc \mid  \exists \elb \in \Objb \colon \left(\inrel{\ela}{\relA}{\elb} \right) \booland \left(\inrel{\elb}{\relB}{\elc}\right)\}.
              \end{equation}
    \end{enumerate}
\end{ctdefinition}

\begin{remark}
    Relations with the same source and target can be \emph{compared} via inclusion.
    Given~$\relA, \relB \colon \setA\mto \setB$  we can ask whether~$\relA\subseteq \relB$ or~$\relB\subseteq \relA$ (or neither).
\end{remark}
