

\section{\wHomos}

\begin{ctdefinition}[Semigroup \whomo]
  \label{def:semigroup-mor}
   A morphism between semigroups
  \begin{equation}
  \sgrpA = \tup{\sgrpAset, \mtimes_\sgrpA}
   \qqand
   \sgrpB = \tup{\sgrpBset, \mtimes_\sgrpB}.
   \end{equation}
     is a function~
     \begin{equation}
     \sgrpmorA\colon \sgrpAset \to \sgrpBset
      \end{equation}
      such that for all $\sgrpela, \sgrpelb \in \sgrpAset$,
  \begin{equation}
    \label{eq:sgrp-mor-comp}
    \sgrpmorA(\sgrpela \mtimes_{\sgrpA} \sgrpelb) = \sgrpmorA (\sgrpela) \mtimes_{\sgrpB} \sgrpmorA(\sgrpelb).
  \end{equation}

\end{ctdefinition}

\begin{forslides}
\begin{equation}\label{eq:sgrslides_xy}
\sgrpela, \sgrpelb \in \sgrpAset
\end{equation}
\begin{equation}\label{eq:sgrslides_xty}
\sgrpela \mthen_\sgrpA \sgrpelb \in \sgrpAset
\end{equation}
\begin{equation}\label{eq:sgrslides_fxfy}
\sgrpmorA(\sgrpela),\, \sgrpmorA( \sgrpelb) \in \sgrpBset
\end{equation}
\begin{equation}\label{eq:sgrslides_fxtfy}
\sgrpmorA(\sgrpela) \mthen_\sgrpB \sgrpmorA(\sgrpelb) \in \sgrpBset
\end{equation}
\begin{equation}\label{eq:sgrslides_fxty}
\sgrpmorA(\sgrpela  \mthen_\sgrpA  \sgrpelb) \in \sgrpBset
\end{equation}
 \begin{equation}\label{eq:sgrpA}
  \sgrpA = \tup{\sgrpAset, \mtimes_\sgrpA}
   \end{equation}
   \begin{equation}\label{eq:sgrpB}
   \sgrpB = \tup{\sgrpBset, \mtimes_\sgrpB}
   \end{equation}
\end{forslides}
We think of   \cref{eq:sgrp-mor-comp} as a way of saying that the function~$\sgrpmorA\colon \sgrpAset \sto \sgrpBset$  is \emph{compatible} with the multiplication operations on~$\sgrpA$ and~$\sgrpB$, respectively.

\begin{ctdefinition}[Identity \whomo]
  \label{def:identity-sgrp-mor}
  Let~$\sgrpA$ be a semigroup.
  The identity function~$\funid_\sgrpA\colon \sgrpAset \sto \sgrpAset$ is always a morphism of semigroups.
  Indeed, the condition
  \begin{equation}
    \funid (\sgrpela \mtimes_{\sgrpA} \sgrpelb) = \funid (\sgrpela) \mtimes_{\sgrpA} \funid(\sgrpelb)
  \end{equation}
  is satisfied for all $\sgrpelAa, \sgrpelAb \in \sgrpAset$. We call this the \emph{identity \whomo} of~$\sgrpA$.
\end{ctdefinition}



\begin{ctdefinition}[Semigroup isomorphism]
  \label{def:semigroup-iso}
  Let~$\sgrpA$ and~$\sgrpB$ be semigroups.
  A \whomo of semigroups~$\sgrpmorA\colon \sgrpA \mto \sgrpB$ is called a \emph{semigroup isomorphism} if there exists a \whomo of semigroups~$\sgrpmorB\colon \sgrpB \mto \sgrpA$ such that
  \begin{equation}
    \label{eq:sgrp-iso-cond}
    \sgrpmorA \then \sgrpmorB = \funid_\sgrpA  \qqand \sgrpmorB \then \sgrpmorA = \funid_\sgrpB.
  \end{equation}
\end{ctdefinition}


\begin{example}[Logarithms and exponentials]
    The positive reals with multiplication  $\tup{\posReals, \cdot}$ is a semigroup.
    The reals with addition $\tup{\reals, +}$ is a semigroup.

    Now consider as a bridge between the two the logarithmic function
    \begin{equation} \label{eq:log}
        \log :  \posReals \to \reals
    \end{equation}
    and its inverse
    \begin{equation}\label{eq:exp}
        \exp :  \reals \to \posReals
    \end{equation}
    We already know that these are inverse of each other:
    \begin{equation}
    \begin{aligned}
        \exp \then \log &= \id_{\reals} \\
        \log \then \exp &= \id_{\posReals}
        \end{aligned}
    \end{equation}
    We can verify that $\log$ is also a semigroup \whomo, because of this property of the logarithms:
    \begin{equation} \label{eq:log-property}
        \log(a \cdot b) = \log(a) + \log(b).
    \end{equation}
    Because $\log$ is a bijection and $\exp$ is its inverse, it already follows that $\exp$ is
    a \whomo in the opposite direction. Alternatively we can see that is the case because of the property of the exponentials:
    \begin{equation} \label{eq:exp-property}
        \exp(c + d) = \exp(c) \cdot \exp(d).
    \end{equation}
    \Cref{eq:log-property,eq:exp-property} are both \cref{eq:sgrp-mor-comp} in disguise.
\end{example}

\begin{forslides}
\begin{equation}\label{eq:moncomp1}
\prftree{\monela \in \sgrpAset}{\monelb \in \sgrpAset}{(\monela \mtimes \monelb) \in \sgrpAset}
\end{equation}
\begin{equation}\label{eq:moncomp2}
\prftree{\monela : \sgrpAset}{\monelb : \sgrpAset}{(\monela \mtimes \monelb) : \sgrpAset}
\end{equation}
\begin{equation}\label{eq:moncomp3}
\prftree{\monela \in \sgrpAset}{\monelb \in \sgrpAset}{\mtimes (\monela,\monelb) \in \sgrpAset}
\end{equation}
\begin{equation}\label{eq:moncomp4}
\prftree{\monela : \sgrpAset}{\monelb : \sgrpAset}{\mtimes (\monela ,\monelb) : \sgrpAset}
\end{equation}
\begin{equation}\label{eq:funccomp1}
\prftree{\mora : \setA \to \setB}{\morb : \setB \to \setC}{\mora \mthen \morb : \setB \to \setC}
\end{equation}
\begin{equation}\label{eq:funccomp2}
\prftree{\mora : \setA \to \setB}{\morb : \setB \to \setC}{\morb \after \mora : \setB \to \setC}
\end{equation}

\begin{equation}\label{eq:funccomp3}
\prftree{T_\alpha:  X \to X }{T_\beta : X \to X}{T_{\alpha+\beta} : X \to X}
\end{equation}
\begin{equation}\label{eq:explogid}
\exp \then \log = \id_{\reals}
\end{equation}
\begin{equation}\label{eq:logexpid}
\log \then \exp = \id_{\posReals}
\end{equation}
\end{forslides}


\begin{marginfigure}
\includegraphics[width=\textwidth]{ascii}
\caption{7-bit US-ASCII encoding }
\label{fig:ascii}
\end{marginfigure}

\begin{example}[ASCII code]
    ASCII encoding takes any alphanumerical characters and symbols into a number between 0 and 127 (\cref{fig:ascii}).
    Let's call $\alphanums$ the set of those 128 symbols.
    We can see ASCII encoding as a semigroup homorphism of $\alphanums^*$ to the free semigroup on the integers $[0,127]^*$.

    Because we can also go back, by finding the inverse function, ASCII encoding is also an isomorphisms of semigroups.
\end{example}
\begin{example}[ASCII code to binary]
    Currently, computers use binary to store data. (There were, in fact, \emph{trinary} computers.)
    In \cref{fig:ascii}, you can see represented also the binary encoding of each character.
    Therefore, we can see ASCII as a \whomo between $\alphanums^*$ and binary strings $\{0,1\}^*$.
\end{example}
\begin{exercise}
Show that the \whomo
\begin{equation}
\asciienc: \alphanums^* \to \{0,1\}^*
 \end{equation}
 is \emph{not} an isomorphism.
\end{exercise}
\begin{solution}
We can show that we cannot find an inverse morphism
 \begin{equation}
 g:  \{0,1\}^* \to \alphanums^*
 \end{equation}
 At first sight everything seems in order: if we could find an isomorphism into $[0,127]^*$ and
 we can express integers in binary, what could hold us back?

What fails here is something so simple it could go unnoticed: the hypothetical function $g$ is not well defined for all points of its domain. We know how to translate a binary string of length $7,14,21,\dots$ back to symbols; but what would be the output of $g$ on the string $111$?

The function $g$ is a left inverse for $\asciienc$, in the sense that
$\asciienc\then g = \id_{\alphanums^*}$, but it is not a right inverse.

%There is a way to elegantly fix the situation. Let $\alphanums^\omega$ be the set of \emph{infinite streams} on $\alphanums$. (Recall that $\alphanums^*$ is the set of all \emph{finite} sequences.)
%Then $\asciienc$ \emph{is} an isomorphism between
\end{solution}

\begin{margintable}
\footnotesize
\centering
\caption{Morse encoding}

\begin{tabular}{rl}
\mst A&	\morsedot \morsedash\\
\mst B&	\morsedash \morsedot \morsedot \morsedot\\
\mst C&	\morsedash \morsedot \morsedash \morsedot\\
\mst D&	\morsedash \morsedot \morsedot\\
\mst E&	\morsedot\\
\mst F&	\morsedot \morsedot \morsedash \morsedot\\
\mst G&	\morsedash \morsedash \morsedot\\
\mst H&	\morsedot \morsedot \morsedot \morsedot\\
\mst I&	\morsedot \morsedot\\
\mst J&	\morsedot \morsedash \morsedash \morsedash\\
\mst K&	\morsedash \morsedot \morsedash\\
\mst L&	\morsedot \morsedash \morsedot \morsedot\\
\mst M&	\morsedash \morsedash\\
\mst N&	\morsedash \morsedot\\
\mst O&	\morsedash \morsedash \morsedash\\
\mst P&	\morsedot \morsedash \morsedash \morsedot\\
\mst Q&	\morsedash \morsedash \morsedot \morsedash\\
\mst R&	\morsedot \morsedash \morsedot\\
\mst S&	\morsedot \morsedot \morsedot\\
\mst T&	\morsedash\\
\mst U&	\morsedot \morsedot \morsedash\\
\mst V&	\morsedot \morsedot \morsedot \morsedash\\
\mst W&	\morsedot \morsedash \morsedash\\
\mst X&	\morsedash \morsedot \morsedot \morsedash\\
\mst Y&	\morsedash \morsedot \morsedash \morsedash\\
\mst Z&	\morsedash \morsedash \morsedot \morsedot\\
\end{tabular}
\begin{tabular}{rl}
\mst 0&	\morsedash \morsedash \morsedash \morsedash \morsedash\\
\mst 1&	\morsedot \morsedash \morsedash \morsedash \morsedash\\
\mst 2&	\morsedot \morsedot \morsedash \morsedash \morsedash\\
\mst 3&	\morsedot \morsedot \morsedot  \morsedash \morsedash\\
\mst 4&	\morsedot \morsedot \morsedot \morsedot \morsedash\\
\mst 5&	\morsedot \morsedot \morsedot \morsedot \morsedot\\
\mst 6&	\morsedash \morsedot \morsedot \morsedot \morsedot\\
\mst 7&	\morsedash \morsedash \morsedot \morsedot \morsedot\\
\mst 8&	\morsedash \morsedash \morsedash  \morsedot \morsedot\\
\mst 9&	\morsedash \morsedash \morsedash \morsedash \morsedot\\
\end{tabular}


  \label{fig:morse}
\end{margintable}

\begin{example}[Morse code]
  Consider the Morse code:  a way to encode the letters and numerals to an alphabet
  of dots ($\morsedot$) and dashes ($\morsedash$). The encoding is shown in \cref{fig:morse}.
  Here, the alphabet $\mathcal{A}$ is the letters A--Z and the numbers 0--9.
  There is no difference between upper and lower case, and there are no punctuation marks.

  Transcribing a text in Morse code is not just a matter of creating the right sequence of dots and dashes. The standard also requires a certain timing of the events. If the length of $\morsedot$ is 1, then the length of $\morsedash$ must be 3. There must be an interval of $3$ between letters, and $7$ between words.

  Therefore, there are 5 symbols in the Morse alphabet (\cref{tab:morse5}); each representing a \emph{signal}.

    \begin{margintable}
    \caption{5 symbols for Morse encoding}
    \small
  \begin{tabular}{cll}
    \morsedot & beep of length $\ell$ & \Morsedot\\
    \morsedash & beep of length $3\ell$& \Morsedash \\
    \morsedsp & silence of length $\ell$& \Morsedsp  \\
    \morselsp & silence of length $3\ell$& \Morselsp \\
    \morsewsp & silence of length $7\ell$& \Morsewsp
  \end{tabular}
  \label{tab:morse5}
  \end{margintable}

  Define now the extended alphabet  $\mathcal{A}'$ to be the union of $\mathcal{A}$ and the set $\{\lettersp, \wordsp\}$, where $\lettersp$ is interletter space, and $\wordsp$ is inter-word space.

  Therefore, to encode the sentence
  \begin{equation}
  \text{``I am well''}
  \end{equation}
   we first transform it to upper case
   \begin{equation}
   \text{``I AM WELL''}
    \end{equation}
   Then we note the inter-letter space and the interword spaces:
  \begin{equation}
      \texttt{I} \wordsp \texttt{A} \lettersp \texttt{M} \wordsp \texttt{W} \lettersp
       \texttt{E} \lettersp \texttt{L} \lettersp \texttt{L}
  \end{equation}
  At this point we can substitute the Morse code to obtain
\begin{equation}
      \morseI \morsewsp \morseA \morselsp \morseM \morsewsp \morseW \morselsp \morseE \morselsp \morseL \morselsp \morseL
  \end{equation}
    In signal space---what somebody would hear---this becomes
\begin{equation}
      \MorseI \Morsewsp \MorseA \Morselsp \MorseM \Morsewsp \MorseW \Morselsp \MorseE \Morselsp \MorseL \Morselsp \MorseL
  \end{equation}
  With this representation it is clear that 5 symbols are redundant: if we have a 1-period beep and a 1-period silence, we can obtain the 3-period silence and beeps and the 7-period silence.

  In the end, the Morse alphabet is \emph{binary} in the sense that it all reduces to two symbols: not $\{\morsedot, \morsedash\}$ but rather the alphabet $\{\Morsedot, \Morsedsp\}$.

\end{example}

\begin{exercise}
We have seen that Morse code transforms a word in the alphabet
\begin{equation}
\mathcal{A} = (\texttt{A} \ \text{to}\ \texttt{Z}) \cup (0 \ \text{to}\ 9) \cup \{\wordsp\}
\end{equation}
to the alphabet
\begin{equation}
\mathcal{B} = \{\Morsedot, \Morsedsp\}
\end{equation}
Is this map a \whomo of semigroups?
\end{exercise}

\begin{solution}
The answer is \textbf{no} because the encoding is context dependent; I don't know if a single letter is followed by a space or another letter.
For example, take the string
    \begin{equation}
    I\wordsp AM\wordsp MAX
    \end{equation}
We can decompose it as follows
\begin{equation}
    I\wordsp A\mtimes M \mtimes \wordsp \mtimes M \mtimes AX
    \end{equation}
If Morse encoding was a \whomo $\sgrpmorA$ then we would be able to encode the string as follows:
\begin{equation}
    \sgrpmorA(I\wordsp A) \mtimes \sgrpmorA (M) \mtimes  \sgrpmorA(\wordsp) \mtimes  \sgrpmorA(M)
     \mtimes  \sgrpmorA(AX)
    \end{equation}
However, this cannot work, because in the second instance of $M$ we would need to output a letter separator, while in the first case we don't.

Can you find a way to fix it?

For example you can consider the alphabet
\begin{equation}
 \left( (\texttt{A} \ \text{to}\ \texttt{Z}) \cup (0 \ \text{to}\ 9) \right) \times \{\wordsp, \lettersp\}
\end{equation}
where we annotate if each symbol is followed by a letter or by a space.

In this representation, the string can be written as
\begin{equation}
 \tup{I, \wordsp} \tup{A, \lettersp} \tup{ M, \wordsp} \tup{M, \lettersp} \tup{A, \lettersp}
  \tup{X, \wordsp}.
\end{equation}
Based on this representation we can define context-independent rules that make a \whomo.
\end{solution}

\begin{example}[Phonetic languages]
  \todotext{Nice example: the map from sequence of characters to sequences
  of sounds is monoidal in certain languages (Korean, Japanese, almost in Italian.) and
  also invertible.}
\end{example}


\begin{example}[State dimension of discrete dynamical systems]
  \todotext{Example of the map $f: \text{DDS} \to \natnumbers$ that gives the size of the state.}
\end{example}


\begin{example}[Transition function, continuation of \cref{exa:transition-functions}]
  Consider the map $f: \nonNegReals \to (\reals^n \to \reals^n)$ that associates to a delta $\delta$
  its transition function $T_\delta$.
  Re-reading~\cref{eq:transition-property}, we can see that it is a \whomo between the semigroup $\tup{\nonNegReals,+}$ and the semigroup of endormorphisms of $\reals^n$.
\end{example}




\begin{exercise}
  \label{ex:non-isomorphic}
  How many different non-isomorphic semigroups are there with precisely one element?
  How many with precisely two elements?
\end{exercise}
\begin{solution}
  \todotext{Write solution of \cref{ex:non-isomorphic}.}
\end{solution}

\begin{exercise}
  \label{ex:semi-morph}
  Let~$\sgrpmorA\colon \sgrpA \mto \sgrpB$ be a morphism of semigroups.
  Prove that~$\sgrpmorA$ is an isomorphism of semigroups if and only if the function~$\sgrpmorA\colon \sgrpA \sto \sgrpB$ is bijective.
\end{exercise}
\begin{solution}
  \todotext{Write solution of \cref{ex:semi-morph}.}
\end{solution}


\section{Monoid \whomos}

%
%\begin{ctdefinition}[Semigroup \whomo]
%  \label{def:semigroup-mor}
%  Let~$\sgrpA$ and~$\sgrpB$ be semigroups. A \whomo of semigroups from~$\sgrpA$ to~$\sgrpB$ is a function~$\sgrpmorA\colon \sgrpAset \to \sgrpBset$ such that for all $\sgrpela, \sgrpelb \in \sgrpAset$,
%  \begin{equation}
%    \label{eq:sgrp-mor-comp}
%    \sgrpmorA(\sgrpela \mtimes_{\sgrpA} \sgrpelb) = \sgrpmorA (\sgrpela) \mtimes_{\sgrpB} \sgrpmorA(\sgrpelb)
%  \end{equation}
%
%\end{ctdefinition}
%

We have defined semigroup \whomo. A monoid \whomo has the same properties, and one additional one: the constraint that it transforms identities to identities.

\begin{ctdefinition}[Monoid \whomo]
  \label{def:monoid-mor}
  Consider two monoids
  \begin{equation}\label{eq:monoidA}
  \monoidA = \tup{\monoidAset, \mtimes_\monoidA, \idmon_\monoidA}
   \end{equation}
   and
   \begin{equation}\label{eq:monoidA}
   \monoidB = \tup{\monoidBset, \mtimes_\monoidB, \idmon_\monoidB}
   \end{equation}
  A \whomo of monoids from~$\monoidA$ to~$\monoidB$ is a function~$\sgrpmorA \colon \monoidA \sto \monoidB$ such that $ \forall \monela, \monelb \in \monoidA$,
  \begin{equation}
    \label{eq:mon-mor-comp}
    \sgrpmorA (\monela \mtimes_{\monoidA} \monelb) = \sgrpmorA (\monela) \mtimes_{\monoidB}  \sgrpmorA(\monelb)
  \end{equation}
  and
  \begin{equation}
    \label{eq:mon-id-comp}
    \sgrpmorA (\idmon_\monoidA) = \idmon_{\monoidB}
  \end{equation}
\end{ctdefinition}

\begin{example}
  The set~$\monoidA = \{ -1, 0, 1 \}$, together with multiplication of whole numbers and with~$1$ as neutral element, forms a monoid. The inclusion map~$\monoidA \sto \wnumbers$ is a \whomo of monoids.
\end{example}

\todo{Nice example: the map from sequence of characters to sequences
of sounds is monoidal in certain languages (Korean, Japanese, almost in Italian.) and
also invertible.}


\begin{definition}[Identity \whomo]
  \label{def:identity-mon-mor}
  Let~$\monoidA$ be a monoid. Similar to the case of semigroups, the identity function~$\id_\monoidA\colon \monoidA \sto \monoidA$ is also a \whomo of monoids.
\end{definition}



\begin{definition}[Monoid isomorphism]
  \label{def:monoid-iso}
  A \whomo of semigroups~$\mora\colon \monoidA \mto \monoidB$ is called a \emph{monoid isomorphism} if there exists a homomorphism of monoids~$\morb\colon \monoidB \sto \monoidA$ such that
  \begin{equation}
    \mora \then \morb = \id_\monoidA \qqand \morb \then \mora = \id_\monoidB.
  \end{equation}
\end{definition}


\begin{exercise}
  Prove: a morphism of monoids~$\mora\colon \monoidA \mto \monoidB$ is an isomorphism of monoids if and only if the function~$\mora\colon \sgrpA \sto \sgrpB$ is bijective.
\end{exercise}
\begin{solution}
  \todotext{Write solution}
\end{solution}


\section{Group \whomo}


\todotext{Show the inverse operation being compatible with group structure, commuting with \whomos. This is the simplest
example of a dagger category, to be explained later on.}


