% !TEX root = chapter-standalone.tex

\section{Actions}
\label{sec:actions}

\todotext{Expand this intro a bit to explain the conceptual idea of actions better, and mention the key term ``state space''. }

There are many examples in engineering of situations where there is a \SY{semigroup} whose elements can be thought of transformations that are applied to a second space.
This concept is formalized as an ``action''.
% There are two versions for actions: a ``covariant'' and a ``contravariant'' version, depending on the order of the transformations.
% \footnote{
%     Traditionally, people have used the names ``left'' and ``right'' to describe the two types.
%     However, those names are connected to a particular choice of notation.
%     We name the two versions by highlighting the properties of covariance and contravariance.
% }

%\linkvideo{spring2021-actions:semi-actions} % Semigroup actions
\begin{ctdefinition}
    [Semigroup covariant action (preliminary version)]\label{def:semigroup-cov-action-prelim}
    A \emph{covariant semigroup action} of a \SY{semigroup}~$\sgrpA$ on a set~\setA is a map
    \begin{equation}
        \label{eq:lact}
        \lact \colon \makecartprod{\sgrpAset, \setA} \sto \setA
    \end{equation}
    such that, for all~$\setAel \setin \setA$ and $\monela,\monelb\setin\sgrpAset$,
    \begin{equation}
        \label{eq:lact1cond}
        \lact(\monelb, \lact(\monela, \setAel)) = \lact(\monela \mtimesof\sgrpA \monelb, \setAel).
    \end{equation}
\end{ctdefinition}

We read \cref{eq:lact1cond} as saying that, if we first apply the action of $\monela$ to obtain $\lact(\monela, \setAel)$,
and then we apply the action of $\monelb$ $\lact(\monelb, -)$, it is the same thing as applying the action of $\monela\mtimes\monelb$.

% \begin{ctdefinition}
%     [Semigroup contravariant action (preliminary version)]\label{def:semigroup-contra-action-prelim}
%     A \emph{semigroup \textbf{contravariant} action} of a \SY{semigroup}~$\sgrpA$ on a set~$\setA$ is a map
%     \begin{equation}
%         \label{eq:ract}
%         \ract \colon \makecartprod{\sgrpAset, \setA} \to \setA
%     \end{equation}
%     such that, for all~$\setAel \setin \setA$,
%     \begin{equation}
%         \label{eq:ract1cond}
%         \ract(\monelb, \ract(\monela, \setAel)) = \ract(\monelb \mthenof\sgrpA  \monela, \setAel).
%     \end{equation}
% \end{ctdefinition}

This is standard way to formulate the definition. Another, equivalent way to formulate the definition is by ``currying'' the definition given above. Recall that there is a canonical isomorphism
\begin{equation}
\curry \colon \setC^{\setA \cartprod \setB} \to \pars{\setC^\setB}^\setA,
\end{equation}
which we called ``curry'', which transforms any function of the type
\begin{equation}
\setA \cartprod \setB \to \setC
\end{equation}
into one of the type
\begin{equation}
\setA \to (\setB \to \setC).
\end{equation}
Applying this to $\lact \colon \makecartprod{\sgrpAset, \setA} \sto \setA$ we obtain a function 
\begin{equation}
\curry(\lact) \colon \sgrpAset \to (\setA \to \setA). 
\end{equation}
We can tell the difference between $\lact$ and $\curry(\lact)$ based on their respective domains and codomains, so we will usually simply use the name $\lact$ for both functions. Furthermore, we will usually write
\begin{equation}
    \label{eq:act-to-endo}
    \lact \colon \sgrpAset \to \Endof\setA,
\end{equation}
using the notation $\Endof\setA$ for functions of type~$\setA \sto \setA$. These are the endomorphisms of~\setA.

Now let's take a second look at \cref{eq:lact1cond}:
%
\begin{equation}
    \label{eq:lact1cond-2}
    \lact(\monelb, \lact(\monela, \setAel)) = \lact(\monela \mtimesof\sgrpA \monelb, \setAel).
\end{equation}
%
If we rewrite it as an equality of functions, we obtain
%
\begin{equation}
    \label{eq:act-to-endo-properties}
    \lact(\monela) \mtimesof{\Endof\setA} \lact(\monelb) =_{\Endof\setA} \lact(\monela \mtimesof\sgrpA \monelb).
\end{equation}
%
Looking at \cref{eq:act-to-endo,eq:act-to-endo-properties} we recognize that together they mean that~$\lact$ is a \SY{semigroup morphism}~(\cref{def:semigroup-mor}).
This brings us to a more compact description of what a \SY{semigroup action} is.

\begin{ctdefinition}[Semigroup covariant action]
    \label{def:semigroup-cov-action}
    A \maindef{covariant semigroup action} of a \SY{semigroup}~$\sgrpA$ onto a set~\setA is a \SY{semigroup} morphism
    \begin{equation}
        \label{eq:act-semigroup}
        \lact \colon \sgrpA \fto \Endof\setA.
    \end{equation}
\end{ctdefinition}

As it turns out, what could look like a new notion, is actually a special case of a general notion we already encountered, namely the notion of \SY{semigroup morphism}.

For completeness, we also define \SY{monoid} actions and group actions.

\begin{ctdefinition}[Monoid covariant action]
    \label{def:monoid-cov-action}
    A \maindef{covariant monoid action} of a \SY{monoid}~$\monoidA$ onto a set~\setA is a \SY{monoid morphism}
    \begin{equation}
        \label{eq:act-monoid}
        \lact \colon \monoidA \fto \Endof\setA.
    \end{equation}
\end{ctdefinition}

The \SY{neutral element} of the \SY{monoid}~$\Endof\setA$ is the \SY{identity function}~$\mapidat{\setA}$.
Thus, a \SY{monoid action} must map the \SY{neutral element} of~$\monoidA$ to~$\mapidat{\setA}$.
%
%We could also say:
%\begin{quote}\itshape
%A \SY{monoid} action is a \SY{monoid} of endomorphisms.
%\end{quote}

For defining a group covariant action, we must introduce a slight variation.
The endomorphisms~$\Endof\setA$ are not a group, because they also contain non-invertible maps.
Recall that an invertible \SY{endomorphism} is called an \emph{automorphism}, and that~$\Autof\setA$, the set of \SY{automorphisms} of~\setA, comes naturally equipped with a \SY{group} structure.

\todotextjira{730}{\bernina: @JL: better would be to make examples in the monoid and group sections for End(A) and Aut(A), respectively, and then reference those here.}

\begin{ctdefinition}[Group covariant action]
    \label{def:group-cov-action}
    A \maindef{covariant group action} of a \SY{group}~$\grpA$ onto a set~\setA is a \SY{group morphism}
    \begin{equation}
        \label{eq:act-group}
        \lact \colon \grpA \fto \Autof\setA.
    \end{equation}
\end{ctdefinition}

\begin{lemma}
    \label{lem:uncurrying-group-actions}
    The data of a \SY{covariant group action}
    \begin{equation}
        \lact \colon \grpA \fto \Autof\setA
    \end{equation}
    is equivalent to the data of having a function
    \begin{equation}
        \stylemaps{\alpha} \colon \grpAset \cartprod \setA \sto \setA,
    \end{equation}
    which satisfies the conditions
    \begin{enumerate}
        \item $ \stylemaps{\alpha} (\idgrp_\grpA, \setAel) = \setAel \quad \forall \setAel \setin \setA$;
        \item $ \stylemaps{\alpha} (\grpelb, \stylemaps{\alpha}(\grpela, \setAel)) = \stylemaps{\alpha} (\grpela \mtimes \grpelb, \setAel) \quad \forall \setAel \setin \setA, \forall \grpela, \grpelb \setin \grpAset.
              $
    \end{enumerate}
\end{lemma}

\vfill%exercises

\todotextjira{26}{\bernina: @JL: add more exercises}

\begin{gradedexercise}[\exname{UncurryingGroupActions}]
    \label{ex:UncurryingGroupActions}
    Prove \cref{lem:uncurrying-group-actions}.
\end{gradedexercise}
\solutionof{UncurryingGroupActions}

\begin{gradedexercise}[\exname{MatrixMultAction}]
    \label{ex:MatrixMultAction}
    Let~$\setA = \reals^n$, and let~$\mgGLn$ be the group of invertible~$\ntimesn$ matrices.
    Let
    \begin{equation}
        \label{eq:action-matrix-mult}
        \defmapset{
            \alpha
        }{
            \mgGLn \cartprod \setA
        }{
            \setA
        }{
            \tup{\matmor{M}, \vectmor{v}}
        }{
            \matmor{M}\vectmor{v}
        }
    \end{equation}
    be the usual multiplication of matrices with vectors.
    Check that \cref{eq:action-matrix-mult} defines a covariant group action of~$\mgGLn$ on~\setA.
\end{gradedexercise}

\solutionof{MatrixMultAction}

\section{Contravariant actions}

There are two versions for actions: a ``covariant'' and a ``contravariant'' version, depending on the order of the transformations.
\footnote{
    Traditionally, people have used the names ``left'' and ``right'' to describe the two types.
    However, those names are connected to a particular choice of notation.
    We name the two versions by highlighting the properties of covariance and contravariance.
}

\begin{ctdefinition}
    [Semigroup contravariant action (preliminary version)]\label{def:semigroup-contra-action-prelim}
    A \emph{contravariant semigroup action} of a \SY{semigroup}~$\sgrpA$ on a set~\setA is a map
    \begin{equation}
        \label{eq:ract}
        \ract \colon \makecartprod{\sgrpAset, \setA} \sto \setA
    \end{equation}
    such that, for all~$\setAel \setin \setA$,
    \begin{equation}
        \label{eq:ract1cond}
        \ract(\monelb, \ract(\monela, \setAel)) = \ract(\monelb \mtimesof\sgrpA \monela, \setAel).
    \end{equation}
\end{ctdefinition}

In contrast to \cref{eq:lact1cond}, \cref{eq:ract1cond} says that, if we first apply the action of $\monela$ to obtain $\lact(\monela, \setAel)$,
and then we apply the action of $\monelb$ $\lact(\monelb, -)$, it is the same thing as applying the action of $\monelb\mtimes\monela$ (note the inverted order).

% And what about contravariant actions?
A \SY{contravariant semigroup action} can be defined as a covariant action of \emph{the opposite} \SY{semigroup}.
And similarly for \SY{monoids} and \SY{groups}.

\begin{ctdefinition}[Semigroup contravariant action]
    \label{def:semigroup-cont-action}
    A \maindef{contravariant semigroup action} of a \SY{semigroup}~$\sgrpA$ onto a set~\setA is a \SY{semigroup} morphism
    \begin{equation}
        \ract \colon \sgrpA\op \fto \Endof\setA.
    \end{equation}
\end{ctdefinition}

\todotext{\bernina: @JL: add example of vector spaces as actions of a field on an abelian group; mention concept of a module}
