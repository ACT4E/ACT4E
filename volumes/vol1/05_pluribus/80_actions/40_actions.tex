% !TEX root = chapter-standalone.tex


%\todotext{\cref{plant-trafo-semigroup} is an option for introducing the notion of action, or just as an example}
%
%\todotext{Make example of dynamical systems acting on sequences.}

%\todotext{Mention group actions and give some simple examples, e.g. in terms of symmetries}

\section{Actions}

There are actually 2 versions for actions: a left and a right version, depending on the order of the transformations.


\begin{ctdefinition}[Semigroup left action (preliminary version)]\label{def:semigroup-left-action-prelim}
  A \emph{semigroup \textbf{left} action} of a semigroup $\sgrpA$ onto a set $\setA$ is a map
  \begin{equation}\label{eq:lact}
    \lact \colon \sgrpAset \cartprod \setA \to \setA
  \end{equation}
  such that, for all $\setAel \in \setA$,
  \begin{equation}\label{eq:lact1cond}
    \lact(\monela, \lact(\monelb, \setAel)) = \lact(\monela \mthen_\sgrpA \monelb, \setAel).
  \end{equation}
\end{ctdefinition}


\begin{ctdefinition}[Semigroup right action (preliminary version)]\label{def:semigroup-right-action-prelim}
  A \emph{semigroup \textbf{right} action} of a semigroup $\sgrpA$ onto a set $\setA$ is a map
  \begin{equation}\label{eq:ract}
    \ract \colon \setA \cartprod \sgrpAset \to \setA
  \end{equation}
  such that, for all $\setAel \in \setA$,
  \begin{equation}\label{eq:ract1cond}
    \ract(\ract(\setAel, \monela), \monelb) = \ract(\setAel, \monela \mthen_\sgrpA \monelb).
  \end{equation}
\end{ctdefinition}


This definition is the standard algebraic definition.
We will compress it a bit using the language of category theory.

\begin{marginfigure}
  \caption{Haskell Curry (X--X)}
  \label{fig:haskell-curry}
Haskell Curry
\end{marginfigure}

We need to introduce Dr.~Haskell Curry (\cref{fig:haskell-curry}).
His first name,  Haskell, named the language.
His last name, Curry, named the operation of \emph{currying} that we are going to need.

Curry noticed that for describing the domain of a function, we do not need to have the Cartesian product.
If we have a function
\begin{equation}
f \colon \setA \cartprod \setB \to \setC
\end{equation}
we can rewrite it as a function of higher type
\begin{equation}
f \colon \setA \to ( \setB \to \setC).
\end{equation}
If you feel like a cool computer scientist, you can also drop the parenthesis and write
\begin{equation}
f \colon \setA \to \setB \to \setC,
\end{equation}
because the expression is not ambiguous, as~$\to$ associates from the left.



This specifies $f$ as a function that, given a~$\setA$, provides a function that, given a $\setB$,
provides a $\setC$; this is the same as a function that needs the two arguments~$\setA$ and $\setB$ before giving the~$\setC$.  To describe the isomorphism we can write it as
\begin{equation}
  \setA \to ( \setB \to \setC) \simeq  \setA \cartprod \setB \to \setC,
\end{equation}
or, more precisely, highlighting that these are morphism arrows in~\Set, we have
\begin{equation}
  \HomSet\Set\setA{\HomSet\Set\setB\setC} \simeq \HomSet\Set{\setA \cartprod \setB}\setC.
\end{equation}
Keep this in mind, it will show up later.

Now armed with currying, we can take a second look at~\cref{eq:lact} and realize that we can rewrite
\begin{equation}
\lact \colon \sgrpAset \cartprod \setA \to \setA
\end{equation}
as
\begin{equation}
  \lact \colon \sgrpAset \to (\setA \to \setA).
\end{equation}
We gave a name to functions of type~$\setA \to \setA$: these are the endomorphisms of $\setA$, written as~$\Endof\setA$.
Thus, we rewrite this as
\begin{equation}\label{eq:act-to-endo}
  \lact \colon \sgrpAset \to \Endof\setA.
\end{equation}
Now we take a second look at \cref{eq:lact1cond}
\begin{equation}
  \lact(\monela, \lact(\monelb, \setAel)) = \lact(\monela \mthen_\sgrpA \monelb, \setAel).
\end{equation}
If we rewrite it as an equality of functions, we obtain
\begin{equation} \label{eq:act-to-endo-properties}
  \lact(\monela) \then_{\Endof\setA}  \lact(\monelb) =_{\Endof\setA} \lact(\monela \mthen_\sgrpA \monelb).
\end{equation}

Looking at \cref{eq:act-to-endo,eq:act-to-endo-properties} we recognize that together
they indicated that~$\lact$ is a semigroup morphism~(\cref{def:semigroup-morphism}). This brings us to a more compact description of what is a semigroup action.

\begin{ctdefinition}[Semigroup left action]\label{def:semigroup-left-action}
  A \emph{semigroup \textbf{left} action} of a semigroup~$\sgrpA$ onto a set~$\setA$ is a semigroup morphism
  \begin{equation}\label{eq:act-semigroup}
    \lact \colon \sgrpA \fto \Endof\setA.
  \end{equation}
\end{ctdefinition}

As it turns out, what could look like a new notion, is actually a special case of a general notion we already encountered.

For completeness, we also define monoid action.

\begin{ctdefinition}[Monoid left action]\label{def:monoid-left-action}
  A \emph{monoid \textbf{left} action} of a monoid~$\monoidA$ onto a set~$\setA$ is a monoid morphism
  \begin{equation}\label{eq:act-monoid}
    \lact \colon \monoidA \fto \Endof\setA.
  \end{equation}
\end{ctdefinition}

The neutral element of the monoid~$\Endof\setA$ is the identity function~$\mapid_{\setA}$.
Thus, a monoid action must map the neutral element of~$\monoidA$ to~$\mapid_{\setA}$.
%
%We could also say:
%\begin{quote}\itshape
%A monoid action is a monoid of endomorphisms.
%\end{quote}


For defining a group left action, we must introduce a slight variation.
The endomorphisms~$\Endof\setA$ are not a group, because they also contain non-invertible maps.

We need to reference the \emph{automorphisms}~$\Autof\setA$, which is a group.

\begin{ctdefinition}[Group left action]\label{def:group-left-action}
  A \emph{group \textbf{left} action} of a group~$\grpA$ onto a set~$\setA$ is a group morphism
  \begin{equation}\label{eq:act-group}
    \lact \colon \grpA \fto \Autof\setA.
  \end{equation}
\end{ctdefinition}


Regarding the right version of these definitions, it is easy to see that a right semigroup action is a left action of the opposite semigroup.


\begin{ctdefinition}[Semigroup right action]\label{def:semigroup-right-action}
  A \emph{semigroup \textbf{right} action} of a semigroup~$\sgrpA$ onto a set~$\setA$ is a semigroup morphism
  \begin{equation}
    \ract \colon \sgrpA^{\op} \fto \Endof\setA.
  \end{equation}
\end{ctdefinition}

\todotext{Usual exercises}
