% !TEX root = chapter-standalone.tex

\section{A simplest dynamical system}
\label{sec:simplest-dyn-system}

\todotext{J: @J: write up material}

Dynamical systems are an important topic in engineering, and there are various ways to formulate this concept mathematically. At a very basic level, there are usually the following ingredients: 
\begin{enumerate}
\item a \emph{state space} which comprises the possible states of the system in question
\item a \emph{dynamics rule} which governs how states of the system may change
\end{enumerate}
Typically, there is a model of \emph{time} involved (e.g. either in discrete time-steps, or as continuous time) for describing how states change over time, or, instead of a model of time, one may use a set of \emph{events} as the triggers for changes of state. 

The following is a very simple example of something that might be called a dynamical system, using only very simple mathematical ingredients. Our aim here is to use simplicity to highlight the core conceptual ingredients; later we will consider more sophisticated formalizations, where this basic structure might be obfuscated by the formalism.

As a state space we consider the following set $\Obja$, whose elements we think of as some different possible states in the lifetime of a plant (sprout, young, mature, old, dead): 
\begin{equation}
\Obja = 
\end{equation}
