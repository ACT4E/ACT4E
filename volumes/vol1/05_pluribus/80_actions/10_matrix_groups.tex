
\section{Matrix groups}



There are many \emph{matrix groups} (\cref{tab:matrix-groups}) that represent linear transformations of a vector space that have some special properties.

\begin{definition}[General linear group $\mgGLn$]
The general linear group of order~$n$, written~$\mgGLn$, is the group of $n\times n$ invertible matrices.
\end{definition}

We have used this in the past.

\begin{definition}[General orthogonal group $\mgOn$]
The general orthogonal group of order $n$, written $\mgOn$, is the group of $n\times n$ square matrices that satisfy the following:
\begin{equation}
\mat{M} \mat{M}^T = \mat{M}^T \mat{M} = I.
\end{equation}
\end{definition}

\begin{definition}[General euclidean group $\mgEn$]
The general euclidean group of order $n$, written $\mgEn$, is the group of $n+1\times n+1$ square matrices of the form
\begin{equation}\label{eq:En-def}
\Ematrix{\matmor{R}}{\vectmor{t}}
\end{equation}
where $\matmor{R}\in \mgOn$ and $\vectmor{t} \in \reals^n$.
\end{definition}

\todotext{Did we define subgroup, submonoid, etc?}

We also have 3 ``special'' versions: $\mgSLn$, $\mgSOn$, $\mgSEn$.
These are subgroups of those that have determinant equal to 1.


\begin{table*}
\caption{Matrix groups}
\label{tab:matrix-groups}
\begin{tabular}{cllc}
\mgGLn & General linear group & invertible matrices \\
\mgOn & Orthogonal group & preserves length of vectors \\
\mgEn & Euclidean groups & preserves distances and angles & \\
\\
\mgSOn & Special orthogonal group& \\
\mgSLn & Special linear group &  \\
\mgSEn & Special Euclidean group \\
\end{tabular}
\end{table*}


These matrix groups are also transformations of $\reals^n$.

\subsection{Special Euclidean group}

The group $\mgSEtwo$ and $\mgSEthree$ are particular important in robotics because they represent
the roto-translations of the plane and 3D space, respectively.

From~\cref{eq:En-def} we know we can represent one by a pair~$\tupp{\matmor{R}, \vectmor{t}}$, with $\matmor{R}\in\mgSOn$ and~$\vectmor{t} \in \reals^n$.

If we look at how matrices compose, we get

\begin{equation}
\Ematrix{\matmor{R}_1}{\vectmor{t}_1} \Ematrix{\matmor{R}_2}{\vectmor{t}_2} = \Ematrix{\matmor{R}_1 \matmor{R}_2}{\matmor{R}_1\vectmor{t}_2 + \vectmor{t}_1}.
\end{equation}

The formulas for composition is

\begin{equation}
  \tupp{\matmor{R}_1, \vectmor{t}_1} \mthen_{\mgSEn} \tupp{\matmor{R}_2, \vectmor{t}_2}  = \tupp{\matmor{R}_1 \matmor{R}_2, \matmor{R}_1 \vectmor{t}_2 + \vectmor{t}_1}
\end{equation}

% %
% \begin{equation}
%   \tupp{\matmor{R}_1, \vectmor{t}_1} \mthen_{\mgSEn} \tupp{\matmor{R}_2, \vectmor{t}_2}  = \tupp{\matmor{R}_2 \matmor{R}_1, \matmor{R}_2 \vectmor{t}_1 + \vectmor{t}_2}
% \end{equation}
%
% Note that because we are using $\mthen$ (then), we do not have the nice symmetry
% that we had if we were using $\after$ (after).
% %
% %

The group $\mgSEn$ induces a transformation on the points of  $\reals^n$.
We are going to call this an \emph{action}.

The action is the following function:
\begin{equation}
\definemap{\mgact}{\mgSEn \times \reals^n}{\reals^n}{\tupp{ \tupp{\matmor{R}, \vectmor{t}}, \vectob{p}}}{\matmor{R}\vectob{p} + \vectmor{t}}
\end{equation}
Given a rototranslation and a point, the function returns the rototranslated point.
%
% We can also see this in matrix form as follows. We need to substitute for a point $\vectob{p} \in \reals^n$ a point
% $ \Epoint{\vectob{p}} \in \reals^{n+1}$.
% %
% \begin{equation}
% \Ematrix{\matmor{R}}{\vectmor{t}}
% \Epoint{\vectob{p}}
%  =
%  \Epoint{\matmor{R}\vectob{p} + \vectmor{t}}
% \end{equation}

If we apply two rototranslations, first $\monela=\tupp{\matmor{R}_{\monela}, \vectmor{t}_{\monela}}$ and then $\monelb=\tupp{\matmor{R}_{\monelb}, \vectmor{t}_{\monelb}}$, we find
% 
\begin{equation}
  \begin{aligned}
& \mgact(\tupp{\matmor{R}_{\monelb}, \vectmor{t}_{\monela}}, \mgact(\tupp{\matmor{R}_{\monela}, \vectmor{t}_{\monela}}, \vectob{p})) = \\
&  \mgact(\tupp{\matmor{R}_{\monelb}, \vectmor{t}_{\monelb}}, \matmor{R}_{\monela}\vectob{p} + \vectmor{t}_{\monela}) = \\
 & \matmor{R}_{\monelb}\matmor{R}_{\monela}\vectob{p} + \matmor{R}_{\monelb}\vectmor{t}_{\monela} + \vectmor{t}_{\monelb}.
  \end{aligned}
\end{equation}
% 
It is easy to see that it is equal to compose the two transformations in the inverse order 
% 
\begin{equation}
  \tupp{\matmor{R}_{\monelb}, \vectmor{t}_{\monelb}} \mthen_{\mgSEn} \tupp{\matmor{R}_{\monela}, \vectmor{t}_{\monela}}  = \tupp{\matmor{R}_{\monelb} \matmor{R}_{\monela}, \matmor{R}_{\monelb} \vectmor{t}_{\monela} + \vectmor{t}_{\monelb}}
\end{equation}
% 
and then apply it to the object
\begin{equation}
\mgact(\tupp{\matmor{R}_{\monelb} \matmor{R}_{\monela}, \matmor{R}_{\monelb} \vectmor{t}_{\monela} + \vectmor{t}_{\monelb}}, \vectob{p})
= \matmor{R}_{\monelb}\matmor{R}_{\monela}\vectob{p} + \matmor{R}_{\monelb}\vectmor{t}_{\monela} + \vectmor{t}_{\monelb} .
\end{equation}

Thus we have proved this property
% 
\begin{equation}
\mgact(\monelb, \mgact(\monela, \vectob{p})) = \mgact(\monelb \then \monela, \vectob{p})) .
\end{equation}
% 
The notion of semigroup action generalizes this property.
