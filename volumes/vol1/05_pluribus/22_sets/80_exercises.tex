

\section{Code exercises}

We go on assuming that you have already completed the tutorial in \cref{sec:exercise-tutorial}.

\subsection{Properties}

\begin{gradedexercise}[\exname{TestFiniteSetProperties}]
  Given two finite sets, check if they are a subset of the other.
  Implement the following interface:
  \methodsource{FiniteSetProperties}{is_subset}{}
\end{gradedexercise}

Check your results with
\begin{console}
  > make check-TestFiniteSetProperties
\end{console}

\subsection{Union and intersection}


\begin{gradedexercise}[\exname{TestFiniteSetOperations}]
  Given two finite sets, compute the union and intersection.
  \classsource{FiniteSetOperations}{}
\end{gradedexercise}


Check your results with
\begin{console}
  > make check-TestFiniteSetOperations
\end{console}

\subsection{Product and disjoint union}

\AC{Checks not implemented for the next exercises.}
\begin{gradedexercise}
  Given two finite sets, compute the product and disjoint union.

  \classsource{FiniteSetOperations2}{}
\end{gradedexercise}
\begin{gradedexercise}
  Given two finite sets, compute the disjoint union.

\end{gradedexercise}





\section{Finite Maps}

\subsection*{Interface}

A finite relation \classname{FiniteMap} ...

\classsource{FiniteMap}{}

\subsection*{Representation}

The format is shown in \cref{fig:map1}.
\margindatafilefig{map1}{map1.map.yaml}{fig:map1}


\begin{gradedexercise}[Representation]
  Create a function to load the data.


%
  \classsource{FiniteMapRepresentation}{}

\end{gradedexercise}

\subsection{Operations}


\begin{gradedexercise}[Composition]
  Implement the composition of finite maps.


%
  \methodsource{FiniteMapOperations}{compose}{}

\end{gradedexercise}


\begin{gradedexercise}[Obtain relation]
  Given a FiniteMap, obtain the relation.


%
  \methodsource{FiniteMapOperations}{as_relation}{}

\end{gradedexercise}
