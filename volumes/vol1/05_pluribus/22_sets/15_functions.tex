% !TEX root = chapter-standalone.tex


\section{Functions}
\label{sec:functions}


\linkvideo{spring2021-morphisms:functions-nomenclature}

\bigskip


One typical way of thinking about functions is in terms of ``input'' and ``output''. Given a function $\mapa$ from a set $\setA$ to a set $\setB$ (for which we write $\mapa: \setA \to \setB$) we sometimes speak of ``plugging in'' an element $\ela \in \setA$ into the function $\mapa$, and then it will ``output'' an element $\mapa(\ela) \in \setB$. 

One reason for this kind of thinking is that sometimes functions describe things that are like a computational process or a machine: for instance, we might give a software program an input, it might then perform as series of computations, and then output an answer, and all of this might described by a function. Another reason for thinking of functions in terms of input and output is because humans often use functions -- \emph{as mathematical entities} -- in a deterministic ``machine'' kind of way. Starting with some element $\ela \in \setA$, we can sometimes use the function $\mapa$ to \emph{calculate} or otherwise \emph{determine} what the ``output'' $\mapa(\ela) \in \setB$ is. Functions are \emph{deterministic}, mathematically speaking, in the sense that for any input $\ela$, there is exactly one output $\mapa(\ela)$. 

We will take the point of view that a function $\mapa: \setA \to \setB$ defines a certain kind of \emph{relation} between the elements of $\setA$ and $\setB$. Given an $\ela \in \setA$, the function $\mapa$ tells us that this $\ela$ is related to a certain $\elb \in B$, which we happen to call $\mapa(\ela)$. This point of view is fully compatible with thinking of functions as ``mathematically deterministic''. However, it is strictly more general than interpreting functions as describing processes which are in any sense ``physically deterministic'' or where ``the input precedes the output'' in any temporal sense. 

As an illustration, consider a large phone book (of personal mobile numbers), which is just a table of names and phone numbers. Let $\setA$ be the set of phone numbers in the book, and $\setB$ the set of names. There is a function $\mapa: \setA \to \setB$ which, given any phone number in $\setA$, will output the name of the person to whom that number is registered. Normally, every number is assigned to a single name, so a name as an ``output'' of the function is completely determined (mathematically speaking) by the number one ``inputs''. However, there is no ``physical determinism'' here: there is no non-mathematical process by which the name of the person was ``computed'' or ``causally determined'' by the phone number. Rather, the function we described arises simply from a table of information. 

Note also that if we were to try to define a function $\setB \to \setA$ which takes a name as an input and outputs a phone number, then we would (in a very large phone book at least) run into a problem: there may be multiple persons with the same name, but different telephone numbers. In other words, without introducing some further mechanism for choosing, we cannot define the intended function $\setB \to \setA$ because for a given name there are multiple numbers to which it might correspond. This is a situation where we lack ``mathematical determinism'', so to speak.\footnote{One thing we \emph{could} do in this situation is, for any given name, consider the \emph{set} of phone numbers to which that name is associated. This would define a function $\setB \to \powerset \setA$}

The pseudo-philosophical introduction above is intended to sensitize the reader to aspects of ``input/output'' thinking, and to motivate the formal definition of ``function'' below. The idea is that we can define a function $\mapa: \setA \to \setB$ by the pairs $\tup{\ela, \elb} \in \setA \times \setB$ which are the elemnets of what is typically called the ``graph'' of the function. (In other words, those pairs of the form ``$\tup{\ela, \mapa(\ela)}$'').

\begin{definition}
Let $\setA$ and $\setB$ be sets. A function $\mapa: \setA \to \setB$ is a subset
\begin{equation}
\mapa \subseteq \setA \times \setB
\end{equation}
with the property 
\begin{equation}\label{eq:function-determinism}
\forall \ \ela \in \setA  \ \  \exists ! \ \elb \in \setB : \tup{\ela,\elb} \in \mapa.
\end{equation}
\end{definition}

The property \cref{eq:function-determinism} describes the ``mathematical determinism'' that functions are supposed to have: for any $\ela \in \setA$ there exits \emph{exactly one} element $\elb \in \setB$ such that $\tup{\ela, \elb} \in \mapa$. 

\clearpage
