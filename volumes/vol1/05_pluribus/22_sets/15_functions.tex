% !TEX root = chapter-standalone.tex


\section{Functions}
\label{sec:functions}


\linkvideo{spring2021-morphisms:functions-nomenclature}

\bigskip


One typical way of thinking about functions is in terms of ``input'' and ``output''. Given a function $\mapa$ from a set $\setA$ to a set $\setB$ (for which we write $\mapa: \setA \to \setB$) we sometimes speak of ``plugging in'' an element $\ela \in \setA$ into the function $\mapa$, and then it will ``output'' an element $\mapa(\ela) \in \setB$. 

One reason for this kind of thinking is that sometimes functions describe things that are like a computational process or a machine: for instance, we might give a software program an input, it might then perform as series of computations, and then output an answer, and all of this might described by a function. Another reason for thinking of functions in terms of input and output is because humans often use functions -- \emph{as mathematical entities} -- in a deterministic ``machine'' kind of way. Starting with some element $\ela \in \setA$, we can sometimes use the function $\mapa$ to \emph{calculate} or otherwise \emph{determine} what the ``output'' $\mapa(\ela) \in \setB$ is. Functions are \emph{deterministic}, mathematically speaking, in the sense that for any input $\ela$, there is exactly one output $\mapa(\ela)$. 

We will take the point of view that a function $\mapa: \setA \to \setB$ defines a certain kind of \emph{relation} between the elements of $\setA$ and $\setB$. Given an $\ela \in \setA$, the function $\mapa$ tells us that this $\ela$ is related to a certain $\elb \in B$, which we happen to call $\mapa(\ela)$. This point of view is fully compatible with thinking of functions as ``mathematically deterministic''. However, it is strictly more general than interpreting functions as describing processes which are in any sense ``physically deterministic'' or where ``the input precedes the output'' in any temporal sense. 

As an illustration, consider a large phone book (of personal mobile numbers), which is just a table of names and phone numbers. Let $\setA$ be the set of phone numbers in the book, and $\setB$ the set of names. There is a function $\mapa: \setA \to \setB$ which, given any phone number in $\setA$, will output the name of the person to whom that number is registered. Normally, every number is assigned to a single name, so a name as an ``output'' of the function is completely determined (mathematically speaking) by the number one ``inputs''. However, there is no ``physical determinism'' here: there is no non-mathematical process by which the name of the person was ``computed'' or ``causally determined'' by the phone number. Rather, the function we described arises simply from a table of information. 

Note also that if we were to try to define a function $\setB \to \setA$ which takes a name as an input and outputs a phone number, then we would (in a very large phone book at least) run into a problem: there may be multiple persons with the same name, but different telephone numbers. In other words, without introducing some further mechanism for choosing, we cannot define the intended function $\setB \to \setA$ because for a given name there are multiple numbers to which it might correspond. This is a situation where we lack ``mathematical determinism'', so to speak.\footnote{One thing we \emph{could} do in this situation is, for any given name, consider the \emph{set} of phone numbers to which that name is associated. This would define a function $\setB \to \powerset \setA$}

The pseudo-philosophical introduction above is intended to sensitize the reader to aspects of ``input/output'' thinking, and to motivate our definition of functions below. The idea is that we can formally define a function $\mapa: \setA \to \setB$ by the ordered pairs $\tup{\ela, \elb} \in \setA \times \setB$ which are the elements of what might be called the ``graph'' of the function. In other words, those ordered pairs of the form ``$\tup{\ela, \mapa(\ela)}$''.

\begin{definition}\label{def:function}
Let $\setA$ and $\setB$ be sets. A \emph{function} $\mapa: \setA \to \setB$ is a subset
\begin{equation}
\mapa \subseteq \setA \times \setB
\end{equation}
with the property 
\begin{equation}\label{eq:function-determinism}
\forall \ \ela \in \setA  \ \  \exists ! \ \elb \in \setB : \tup{\ela,\elb} \in \mapa.
\end{equation}
We say that $\setA$ is the \emph{source} and $\setB$ is the \emph{target} of $\mapa$. 
\end{definition}

The property \cref{eq:function-determinism} describes the ``mathematical determinism'' that functions are supposed to have: for any $\ela \in \setA$ there exists \emph{exactly one} element $\elb \in \setB$ such that $\tup{\ela, \elb} \in \mapa$. 

Another important aspect of \cref{eq:function-determinism} is that it says \emph{for every} $\ela \in \setA$ there exists a $\elb \in \setB$ that is related to $\ela$ by $\mapa$. In other words, we do not allow functions to be ``partially defined''. For example, the formula ``$\mapa(x) = \frac{1}{x}$'' could be used to define a function $\reals \backslash \{ 0 \} \to \reals$ but it would \emph{not} be valid for defining a function $\reals \to \reals$. The ``domain'' of a function is ``baked in'' as its source set, and the source and target of a function are \emph{part of the definition} of the function. For example, the formula $\mapa(x) = x^2$ defines \emph{different} functions, depending on whether we think of it as defining a function $\reals \to \reals$, $\reals \to \reals_{\geq 0}$, or $[0,1] \to [0,1]$, etc. 

Although we take \cref{def:function} as our \emph{formal} definition of functions, we will continue to use the standard kinds of notation for functions, for example usually writing $\elb = \mapa(\ela)$ and not $\tup{\ela, \elb} \in \mapa$. The formal definition above is useful to keep in the back of our minds though. For instance, when thinking about situations involving the empty set. Do there exist functions $\emptyset \to \setB$ for any set $\setB$? What about $\setA \to \emptyset$? 

Consulting \cref{def:function}, we can figure out, using \cref{eq:function-determinism}, that there are \emph{no} functions of the type $\setA \to \emptyset$, and for any set $\setB$, except for the case $\setB = \emptyset$, there \emph{is} a function $\emptyset \to \setB$ (and in fact only one such function). For the case $\setA \to \emptyset$, there is no such function because the part ``$\exists ! \ \elb \in \setB$'' of \cref{eq:function-determinism} cannot be satisfied, since here $\setB = \emptyset$. For the case $\emptyset \to \setB$, with $\setB \neq \emptyset$, we see that because of the condition ``$\forall \ \ela \in \setA ...$'', \cref{eq:function-determinism} is trivially satisfied since $\setA = \emptyset$ here. In this case, $\mapa \subseteq \setA \times \setB$ corresponds to $\emptyset \subseteq \emptyset \times \setB = \emptyset$. 

It is worth remarking that functions can be specified in a variety of ways. Sometime they are given using formulae, such as $\mapa(x)=x^2$ (and specifying the source and target set). Sometimes the are defined using a table, like in the in the phone book example above. Other times a function might be characterized by equations or properties. For example, the exponential function $\exp : \reals \to \reals$ is known to be  characterized by the fact that it satisfies the differential equation $\mapa'(x) = f(x)$ and the initial condition $f(0) =1$. Still other times one might be able to prove the existence (and perhaps also uniqueness) of some function satisfying some given properties, but one might not have any means to ``evaluate'' or ``calculate'' that function! 

Even if we don't know a lot of the specifics of some functions, there is a lot we can still say about how functions between sets can behave \emph{in general}. In the following we review a number of basic observations and properties. 

Let $\mapa: \setA \to \setB$ be a function. Recall that $\mapa$ is said to be \emph{injective} if the condition
\begin{equation}\label{eq:injectivity-cond}
\forall \ \ela_1, \ela_2 \in \setA : \mapa(\ela_1) = \mapa(\ela_2) \ \Rightarrow \  \ela_1 = \ela_2
\end{equation}
is satisfied. And $\mapa$ is called \emph{surjective} if the condition
\begin{equation}\label{eq:surjectivity-cond}
\forall \ \elb \in \setB \ \exists \ \ela \in \setA : \mapa(\ela) = \elb
\end{equation}
holds. A function which is both injective and surjective is called \emph{bijective}. 

The \emph{image} of $\mapa: \setA \to \setB$ is the set
\begin{equation}
\mapa(\setA) := \{ \elb \in \setB \mid \exists \ela \in \setA : \mapa(\ela) = \elb \}. 
\end{equation}
More generally, given a subset $\subA \subseteq \setA$, its image under $\mapa$ is
\begin{equation}
\mapa(\subA) := \{ \elb \in \setB \mid \exists \ela \in \subA : \mapa(\ela) = \elb \}. 
\end{equation}
Given a subset $\subB \subseteq \setB$, its \emph{preimage} under $\mapa$ is
\begin{equation}
\mapa^{-1}(\subB) :=\{ \ela \in \setA \mid  \mapa(\ela) \in \subB \}. 
\end{equation}
An alternative way of phrasing injectivity of $\mapa$ is to say that for every singleton subset $\{ \elb \} \subseteq \setB$, its preimage under $\mapa$ is either a singleton set or the empty set. Surjectivity of $\mapa$ is equivalent to saying that $\mapa(\setA) = \setB$. 

Importantly, functions can be \emph{composed} when the target set of one functions is the same as the source set of another. Given functions $\mapa: \setA \to \setB$ and $\mapb : \setB \to \setC$, we will denote their composition by
\begin{equation}\label{eq:composition-notation-functions}
\mapa \then \mapb : \setA \to \setB,
\end{equation}
which is different from the more traditional notation ``$\mapb \circ \mapa$''. We speak \cref{eq:composition-notation-functions} as ``$\mapa$ then $\mapb$'', which aligns with the fact that, to evaluate the composition $\mapa \then \mapb$ at an element $\ela \in \setA$, we first apply $\mapa$ to compute $\mapa(\ela)$ and \emph{then} we apply $\mapb$ to compute
$$\mapb(\mapa(\ela))=(\mapa \then \mapb)(\ela).$$ 


\

\


\todo{define surjective, injective, bijective, section, inverse}

\todo{remark on axiom of choice}

\todo{talk about composition and identity functions}

\todo{talk about elements of sets via functions}

\todo{subsets via functions}

\todo{the set of functions from a given source to a given target}

\todo{families, index sets, generalized cartesian product}




\todo{make closing remark or footnote that this isn't the only way to define functions}

\clearpage
