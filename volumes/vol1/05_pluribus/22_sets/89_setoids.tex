\subsection{Setoids}

This is the \Setoid interface:
%
\classsource{Setoid}{}
%
A \Setoid is an object that can tell use whether other objects belong to it.
The \funcname{contains} function is akin to the indicator function.


\begin{codeexercise}
  Create the function \funcname{union_setoids}, which, given two setoids, returns another setoid that is the union of the two given setoids.
%
  \methodsource{SetoidOperations}{union_setoids}{}
%
  Hint: The indicator of the union of sets is the OR of the two indicators.
\end{codeexercise}

\begin{codeexercise}
  Create the function \funcname{intersection_setoids}, which, given two setoids, returns another setoid that is the intersection of the two given setoids.
%
  \methodsource{SetoidOperations}{intersection_setoids}{}
\end{codeexercise}


\subsection{EnumerableSets}

%
An enumerable set is one that can enumerate its elements.
The method \funcname{elements} returns an \emph{iterator} that enumerates the elements.

\begin{codeexercise}
  Create an Enumerable set given a function that provides the $i$-th element.

  The
%
  \methodsource{EnumerableSetsOperations}{make_set_sequence}{}
\end{codeexercise}

\begin{codeexercise}
  Create a function that creates the union of two enumerable sets.

  The
%
  \methodsource{EnumerableSetsOperations}{union_esets}{}
\end{codeexercise}

\begin{codeexercise}
  Explain why it is not possible to create a function that, given two enumerable sets,
  creates the \emph{intersection} of the two sets.
\end{codeexercise}
