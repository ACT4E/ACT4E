\section{Product}

\subsection{Ordered pairs}\label{subsec:ordered-pairs}

As a final topic in this section, we discuss the notion of ``ordered pairs'', and the definitions of the cartesian product and disjoint union of sets.

When we first started speaking about sets, we described them as ``unordered lists'' of things.
Sometimes, though, we want to keep track of an ordering!
Let's start with just two things, call them~$\ela$ and~$\elb$.
The \emph{ordered pair}~$\tup{\ela,\elb}$ is a list where the order \emph{does} matter.
In other words, the ordered pair~$\tup{\elb,\ela}$ is \emph{not} equal to the ordered pair~$\tup{\ela,\elb}$.
Ordered pairs can be defined rigorously using sets, by setting
\begin{equation}
    \label{eq:ordered-pair}
    \tup{\ela, \elb} \coloneqq \{ \{\ela \} , \{ \ela, \elb \} \},
\end{equation}
using the the nesting of the elements of this set to encode the order.
We will not usually need this technical definition however;
the important point is to clarify that this notion describes ordered lists, specifically denoted with certain symbols (the use of commas and brackets).
Generalizing to lists of more than two things, given any finite number~$n \in \natnumbers$ of things~$\ela_1$, $\ela_1$, \dots,~$\ela_n$, we similarly define the ordered list~$\tup{\ela_1, \ela_2, \ldots , \ela_n}$, which we call a \emph{tuple}.

\subsection{Cartesian Product}

% \todotextjira{288}{Below there was cited ``formal definition of ordered pairs'' but it doesn't exist. Add?}
Given sets~$\setA$ and~$\setB$, their \emph{cartesian product}, denoted~$\setA \cartprod \setB$, is the set
\begin{equation*}
    \{ \tup{\ela, \elb} \mid \ela \in \setA \text{ and } \elb \in \setB \}.
    \footnote{In order to accurately follow the schema given in \cref{eq:axiom-specialization} and using the formal definition of ordered pairs it would in fact be correct to define the cartesian product as
        \begin{equation}
            \setA \times \setB = \{ \elc \in \powerset \powerset (\setA \setunion \setB) \mid \exists \ela \in \setA, \exists \elb \in \setB : \elc = \tup{\ela, \elb} \}.
        \end{equation}
    }
\end{equation*}
For example, if~$\setA = \{ \sapple, \sbanana, \scarrot\}$ and~$\setB = \{ \stea, \swater \}$, then
\begin{equation*}
    \setA \cartprod \setB = \{ \tup{\sapple, \stea}, \tup{\sapple, \swater}, \tup{\sbanana, \stea}, \tup{\sbanana, \swater},  \tup{\scarrot, \stea}, \tup{\scarrot, \swater}\}.
\end{equation*}
We remark that in the special case where~$\setA = \emptyset$ or~$\setB = \emptyset$, then~$\setA \cartprod \setB = \emptyset$.

