% !TEX root = chapter-standalone.tex

\chaptersecond{Sets and functions}{placeholder}{}{
}

Set and functions are fundamental notions in mathematics.
In this chapter we given an informal treatment of those set-theoretic notions which are important for the purposes of this book, while avoiding further-going aspects.
The material in this section should be mostly familiar to anyone with some training in engineering, computer science, a natural science, mathematics, etc.
We suggest nonetheless reading through: we set some conventions, and some contents may be new or rendered in a new perspective.
The rigorous use of set theory and category theory do involve taking care of some technical details, notably in dealing with set-theory paradoxes and issues related to ``size''.
However, for our purposes in this book, one can safely ignore these details!
Nevertheless, since terms and technicalities related to this topic inevitable arise when learning (and reading) category theory, we have included a short explanation of some terms in \cref{sec:technical-terms} below.
For those interested in deeper discussion, we refer to the excellent exposition \url{https://arxiv.org/abs/0810.1279}.

\todojira{263}{@J: Include set-theory references in the bibliography and link to them}

\label{ch:sets}
% !TEX root = chapter-standalone.tex

\section{Sets}

Intuitively speaking, sets describe ``collections of things'' -- whether it be a collection of people, of objects, of abstract symbols, etc.
The ``things'' making up a set are called the \emph{elements} of the set.

\subsection{Naming the elements of a set}

There are several typical ways of indicating a set.
One is ``list-like'': we list the elements of the given set, separated by commas, and surround this list with curled brackets.
For example, the set of persons who are co-authors of this book can be written~$\{ \text{Andrea}, \text{Gioele}, \text{Jonathan} \}$, and the set consisting of the symbols ``$\sbanana$'', ``$\sapple$'', ``$\scarrot$'' and ``$\stea$'' is indicated by $\makeset{ \sbanana, \sapple, \scarrot, \stea}$.
\todotextjira{499}{@J: We call the representation ``list-like'' but actually order and repetition matter for lists.}

In a ``list-like" representation, the order that we list elements in does not matter:
$\makeset{ \sapple, \scarrot, \sbanana, \stea}$ and~$\makeset{ \stea, \sbanana, \sapple, \scarrot}$ and~$\makeset{ \sapple, \scarrot, \stea, \sbanana}$ are, for instance, difference ways of indicating one and the same set.
Also, we do not allow repetitions of elements of sets:
$\makeset{ \stea, \sbanana, \sbanana, \sbanana, \sapple, \scarrot }$ is not a valid way of indicating a set.
Given any two elements of a set, they can be only either equal (identical) or unequal (distinct).
\footnote{There is a notion of ``multiset'', where repetitions are allowed (the ``same'' element can appear multiple times), but we are not considering that notion here.}

\begin{marginfigure}
    \centering
    \includesag{sets_as_clouds}
    \caption{We represent sets as ``clouds'' or ``bags'' of nonrepeating elements.}
    \label{fig:set_as_clouds}
\end{marginfigure}

Because in general the elements of set are not ordered in any way, one often visualizes sets as ``clouds'' or ``bags'' of elements (\cref{fig:set_as_clouds}).



%(One of the ways to formalize set theory rigorously is via the axiomatic route, for which there are various different well-known approaches.)

\subsection{Membership and equality}

A fundamental notion is the relation of ``being an element of'', indicated by the symbol ``$\in$''.
For example, the statement that the symbol ``$\stea$'' is an element of the set~$\makeset{ \sapple, \scarrot, \sbanana, \stea }$ is written ``$\stea \in \makeset{ \sapple, \scarrot, \sbanana, \stea }$''.
Similarly, ``$\text{Gioele} \in \{ \text{Andrea}, \text{Gioele}, \text{Jonathan} \}$'' is a true statement.

Given  two sets, we can talk about whether they are \emph{equal} or not.
Two sets~$\setA$ and~$\setB$ are equal if and only if ``they have the same elements''.
In other words,~$\setA$ and~$\setB$ are equal if and only if the statements
``$\ela \in \setA$'' and ``$\ela \in \setB$'' are logically equivalent:
if these statements are always either both true or both false for any instantiation of the variable~$\ela$.
For example, $\makeset{ \sapple, \scarrot, \sbanana, \stea }$ and~$\makeset{ \stea, \sbanana, \sapple, \scarrot}$ and~$\makeset{ \sapple, \scarrot, \stea, \sbanana }$ are all equal as sets, because their elements are the same.


\subsection{Subsets}

Now consider the set~$\makeset{ \sapple, \scarrot, \sbanana }$, and compare it with~$\makeset{ \sapple, \scarrot, \sbanana, \stea }$.
Each of the elements of the first set is also an element of the second set;
in such a case we say that the first set is \emph{included} in the second set.
In symbols,
%
\begin{equation}
    \label{eq:subset}
    \{ \sapple, \scarrot, \sbanana \} \subseteq \{ \sapple, \scarrot, \sbanana, \stea \}.
\end{equation}
%
Generally, given sets~$\setA$ and~$\setB$, the statement~$\setA \subseteq \setB$ is logically equivalent to the statement
%
\begin{equation*}
    \prfperiod{
        \ela \in \setA
    }{
        \ela \in \setB
    }
\end{equation*}

Returning to \cref{eq:subset}, the second set is, on the other hand, \emph{not} included in the first set, since~$\stea$ is an element of the second set, but not the first.
If we say a set is ``strictly included'' in another, then we mean ``included in and not equal'';
if the adjective ``strictly'' is not used, then inclusion means for us that equality is also possible.
\todotextjira{499}{@J: seems like a good place to introduce also $\subset$ - also compare with $<$, $\leq$.}
In other words, for us it is true that any set~$\setA$ is included in itself:~$\setA \subseteq \setA$.
Given sets~$\setA$ and~$\setB$, if it is the case that~$\setA \subseteq \setB$, then we say that~$\setA$ is a \emph{subset} of~$\setB$.
Inclusion and equality are related as follows: given sets~$\setA$ and~$\setB$,
%
\begin{equation*}
    \prfdoubleperiod{
        \setA = \setB
    }{
        {\setA \subseteq \setB}
        \quad
        {\setB \subseteq \setA}
    }
\end{equation*}
%
Equivalently, two sets are equal if they have the same elements:
\begin{equation*}
    \prfdoubleperiod{
        \quad \setA = \setB \quad
    }{
        \prfdouble{
            \ela \in \setA
        }{
            \ela \in \setB
        }
    }
\end{equation*}


\subsection{New sets from old}

Suppose now for a moment that we have some sets to work with, such as for instance the sets used as examples above.
There are a few basic operations that allow to construct new sets from the ones we have, and to reason about them.

In addition to the ``list-like'' way of specifying a set that we discussed above, many times sets are specified with the help of a logical ``statement'' or ``sentence'' which is used to give a characterization of its elements.
The idea is this: we start out with some given set~$\setB$, and then we consider a statement~$S(\ela)$ which depends on a variable~$\ela$, which we think of as running over the elements of~$\setB$.
We can then ask: for which elements~$\ela$ of~$\setB$ is the statement~$S(\ela)$ true?
These elements form a subset of~$\setB$, often denoted
%
\begin{equation}
    \label{eq:axiom-specialization}
    \{ \ela \in \setB \mid S(\ela) \}.
\end{equation}
%
For example, let~$\setB = \{ \sapple, \sbanana, \scarrot \}$ and consider the statement
%
\begin{equation*}
    S(\ela) = \text{``} \ela \text{ is a fruit''}.
\end{equation*}
%
Then we can form the set
%
\begin{equation*}
    \{ \ela \in \setB \mid \ela \text{ is a fruit} \} = \{ \sapple, \sbanana \}.
\end{equation*}

There is an interesting special case of this way of constructing subsets of a set~$\setB$:
what if, for a given statement~$S(\ela)$, \emph{none} of the elements~$\ela \in \setB$ are such that~$S(\ela)$ is true?
For example, consider the statement
%
\begin{equation*}
    S(\ela) =\text{ ``} \ela \text{ is the name of a planet''}.
\end{equation*}
%
Then, for the set~$\setB = \{ \sapple, \sbanana, \scarrot \}$, the subset
\begin{equation*}
    \{ \ela \in \setB \mid \ela \text{ is the name of a planet} \}
\end{equation*}
has \emph{no} elements: it is ``empty''.
In most formalizations of set theory, mathematicians agree to use a single unique set -- called the \emph{empty set} and denoted~$\emptyset$ -- to play the role of ``the set with no elements''.
In particular, by definition, the empty set is a subset of any set!

And now is probably as good a moment as any to note that the empty set~$\emptyset$ should not be confused with the set~$\{ \emptyset \}$.
While~$\emptyset$ denotes the unique set with~$\emph{no elements}$, the notation~$\{ \emptyset \}$ denotes a set which has precisely one element, and that element happens to be the set~$\emptyset$.
Generally, sets can have other sets as their elements: for example, we might consider the set
%
\begin{equation*}
    \{ \{ \sbanana \}, \{ \sapple, \scarrot, \sbanana, \stea \}, \emptyset \},
\end{equation*}
%
which has three elements: namely~$\{ \sbanana \}$,~$\{ \sapple, \scarrot, \sbanana, \stea \}$, and~$\emptyset$.

This leads us nicely to another fundamental construction.


\subsection{A note on technical details}

The fully rigorous use of set theory and category theory does involve taking care of some technical details, notably in dealing with set-theory paradoxes and issues related to ``size''.
    However, for our purposes in this book, one can (luckily) safely ignore these details. 
    Nevertheless, since terms and technicalities related to this topic inevitable arise when learning (and reading) category theory, we have included a short explanation of some terms in \cref{sec:technical-terms} below.
    For those interested in deeper discussion, we refer to the excellent exposition \url{https://arxiv.org/abs/0810.1279}.

    \todotextjira{467}{@J: This abstract is too long -- move half to the first section}
    \todotextjira{263}{@J: Include set-theory references in the bibliography and link to them}

\todotextjira{499}{@J: AC: splitting things in sections introduced some jumps like this that need to be filled / changed. (note that the next heading is exercises)}



% !TEX root = chapter-standalone.tex


\section{Functions}
\label{sec:functions}


One typical way of thinking about functions is in terms of ``input'' and ``output'': given a function $\mapa$ from a set $\setA$ to a set $\setB$, we can ``plug in'' an element $\lea \in \setA$ and the function $\mapa$ will ``output'' an element $\elb \in \setB$. 





\linkvideo{spring2021-morphisms:functions-nomenclature}

\clearpage

% !TEX root = chapter-standalone.tex

\section{Some technical terms}
\label{sec:technical-terms}

\publictodomessage

\todotextjira{245}{Explain briefly what the following things are: collections/classes, proper classes, ZFC, axiom of choice, ... }


\sectionexercises{Sets}


We continue, assuming that you have already completed the tutorial in \cref{sec:exercise-tutorial}.

\sectionexercises{FiniteSetProperties}
\codeboilerplate{FiniteSetProperties}{
    Given two finite sets, check if they are a subset of the other.
}

\classlisting{FiniteSetProperties}

As you can see, once you define \funcname{is_subset}, you get for free the implementation of \funcname{equal} and \funcname{is_strict_subset}.
Of course, there might be more efficient ways to implement \funcname{equal} than two calls of \funcname{is_subset}.

\sectionexercises{Mappings}

\Cref{lst:Mapping} shows our interface for \Mapping.

\classlisting{Mapping}

A \Mapping has a source and a target, which, in general, are \SY{setoids}.

A \Mapping is able to take an element from the source \SY{setoid} and return an element in the second \SY{setoid}.

Here we use Python's syntax sugar.
If an object defines a \funcname{__call__} method, then these are \emph{syntactically} equivalent:

\begin{minted}{python}
assert myobject(x) == myobject.__call__(x)
\end{minted}

Therefore, once you implement a \Mapping you can pass it around like if it was a Python function.

We also define a subclass \FiniteMap which is constrained to have both source and target be a \FiniteSet rather than a \Setoid (\cref{lst:FiniteMap}).

\begin{marginfigure}
    \begin{minipage}{4cm}
        \begin{minted}{yaml}
source:
  elements: [a, b, c]
target:
  elements: [1, 2]
values:
  - [a, 1]
  - [b, 2]
  - [c, 1]
\end{minted}
    \end{minipage}
    \caption{Format for representing maps.}
    \label{fig:mapexample}
\end{marginfigure}

\classlisting{FiniteMap}

We define a YAML representation for finite maps as in~\cref{fig:mapexample}.

The fields \fieldname{source} and \fieldname{target} are two sets.
These are expressions that you can pass to the set constructors you developed previously.

The field \fieldname{values} is an array of pairs: a pair~$\tup{x, y}$ means that~$x$ maps to~$y$.

For the data to be well-formed it is necessary that for each element in the domain, there is exactly one row for that element.
If that is not the case, throw an exception \classname{InvalidFormat}.

\todo{use \str{codeboilerplate} to be consistent?}

\begin{codeexercise}[\exname{TestFiniteMapRepresentation}]
    Create a function to load the data by implementing the interface in \cref{lst:FiniteMapRepresentation}.
\end{codeexercise}

\classlisting{FiniteMapRepresentation}

As part of the exercise, you have to implement \classname{FiniteMap} with all its necessary methods.

\sectionexercises{Set products}

We define a \classname{SetProduct} to be a \Setoid that also has the structure of the Cartesian product of sets.
This means that:
\begin{itemize}
    \item It is possible to recover the ``factors'' of the product (if $\setA = \makecartprod{\setB, \setC, \setD}$, we must be able to recover the components $\setB, \setC, \setD$ from $\setA$);
    \item It is possible to access the elements of a tuple (if $\setA = \makecartprod{\setB, \setC, \setD}$ and $x = \tupp{\setBel, \setCel, \setDel}$, we should be able to access the components of the tuple).
    \item On the other way around: it is possible to construct a tuple from the single components.
\end{itemize}
Our Python interpretation of these properties is specified in the class \classname{SetProduct}.

\classlisting{SetProduct}

The semantics is the following:
\begin{itemize}
    \item The generic type variable \TypeVar{C} represents the type of the components; the generic type variable \TypeVar{E}
          refers to the type of the product.
          Note that we are not constraining the internal representation of the product.
          (A possible choice could be \pystr{E = List[C]}).
    \item The method \funcname{components} returns the ordered list of the setoid factors; these must be setoids of type \TypeVar{C}.
    \item The method \funcname{pack} takes a number of arguments and creates an element of the product.
    \item The method \funcname{unpack} allows to recover the components of an element of the product.
\end{itemize}

There is also a specialization, which corresponds to a product of finite sets; the only change is that the components are finite sets rather than setoids.

\classlisting{FiniteSetProduct}

\codeboilerplate{FiniteMakeSetProduct}{

}
\classlisting{FiniteMakeSetProduct}{}

As in the previous exercises, you need to implement the class \classname{FiniteSetProduct} and all the methods from the classes in inherits from as well as the \funcname{__init__} method.
This procedure will be similar for subsequent exercises and won't be repeated every time.

%\clearpage

\subsection{Representation}
% The interface above means that if you are passed 2 \FiniteSet{}s, you should return a \FiniteSet.
% Otherwise, you should return a \Setoid.

\margindatafilefig{set_product}{$\makeset{a,b} \cartprod \makeset{1,2}$}{fig:set_product}

\margindatafilefig{set_product111}{$\makeset{1} \cartprod \makeset{1} \times \makeset{1}$}{fig:set_product11}%

\margindatafilefig{set_product10}{$\makeset{1} \cartprod\Emptyset$}{fig:set_product10}%

\begin{codeexercise}[\exname{TestFiniteSetRepresentationProduct}]
    Extend now the code you wrote for loading and saving sets to allow the format in \cref{fig:set_product,fig:set_product11,fig:set_product10}.
    We add another clause to the parsing algorithm:
    \begin{enumerate}
        \item If it has a field \fieldname{elements}, it is a finite set described with elements directly.
        \item If there is a field \fieldname{product}, it must be a list of sets, and the semantics is the product of those sets.
        \item Otherwise, it is an error -- for now; we will introduce many more ways to describe sets.
    \end{enumerate}
    Test your results using
    %: note this cannot be boilerplate
    \checkexercise{FiniteSetRepresentationProduct}
\end{codeexercise}

\subsection{Set union \hardexercise}
\begin{remark}
    This is a more advanced exercise; feel free to skip the first time around.
\end{remark}


When in mathematics we compute the union of two sets, we obtain a set but we forget the structure.
We cannot do this when we take a constructive approach.

For example, suppose that we have two \Setoid{}s~$\setA = \{a,b,c\}$ and~$\setB = \{a,b,3\}$.
By all accounts the elements $a,b$ are in both. However, how can we judge that~$a=b$?
Should we use the first setoid or the second? Both? What if they disagree?

We choose this semantics: when comparing two elements, we see the set of components to which they both belong.
Then we compare them using all the equalities. The elements are equal if they are equal according to all of them.

From the point of view of the implementation, we create an interface like
in \cref{lst:SetUnion}, together with \EnumerableSetUnion and \FiniteSetUnion, not shown.

\classlisting{SetUnion}

The method \funcname{components} allows us to recover the components for the union.

\begin{codeexercise}[\exname{TestMakeSetUnion} (for advanced students)]
    Implement the interface in \cref{lst:MakeSetUnion}.
\end{codeexercise}

The \str{@overload} annotations means that:
\begin{compactitem}
    \item if the method is given all \FiniteSet, it should return a \FiniteSet;
    \item if the method is give all \EnumerableSet, it should return an \EnumerableSet;
    \item otherwise, in the general case, it should return a \Setoid.
\end{compactitem}

\classlisting{MakeSetUnion}

Check your results with
\begin{console}
    make check-TestMakeSetUnion
\end{console}


\margindatafilefig{set_union}{Union}{fig:set_union}

Extend now the code you wrote for loading sets to allow the format in \cref{fig:set_product}.

This is the algorithm to implement.

Given a dictionary:
\begin{compactenum}
    \item If it has a field \fieldname{elements}, it is a finite set described with elements directly.
    \item If it has a field \fieldname{union}, then the value must be a list of set representation dictionaries.
    \item Otherwise, it is an error --- for now; we will introduce many more ways to describe sets.
\end{compactenum}

In particular, note that these are valid representations.
\begin{table*}[h]
    \begin{tabular}{cccc}
        An empty set &
        The union of 1 empty set &
        The union of 2 empty sets &
        The union of zero sets \\
%
        \datafile{set_empty}{min_lines=5} &
        \datafile{set_union_empty1}{min_lines=5} &
        \datafile{set_union_empty2}{min_lines=5}
        &
        \datafile{set_union_zero}{min_lines=5}
        \\
    \end{tabular}
\end{table*}


\subsection{Setoids}

This is the \Setoid interface:
%
\classsource{Setoid}{}
%
A \Setoid is an object that can tell use whether other objects belong to it.
The \funcname{contains} function is akin to the indicator function.


\begin{codeexercise}
  Create the function \funcname{union_setoids}, which, given two setoids, returns another setoid that is the union of the two given setoids.
%
  \methodsource{SetoidOperations}{union_setoids}{}
%
  Hint: The indicator of the union of sets is the OR of the two indicators.
\end{codeexercise}

\begin{codeexercise}
  Create the function \funcname{intersection_setoids}, which, given two setoids, returns another setoid that is the intersection of the two given setoids.
%
  \methodsource{SetoidOperations}{intersection_setoids}{}
\end{codeexercise}


\subsection{EnumerableSets}

%
An enumerable set is one that can enumerate its elements.
The method \funcname{elements} returns an \emph{iterator} that enumerates the elements.

\begin{codeexercise}
  Create an Enumerable set given a function that provides the $i$-th element.

  The
%
  \methodsource{EnumerableSetsOperations}{make_set_sequence}{}
\end{codeexercise}

\begin{codeexercise}
  Create a function that creates the union of two enumerable sets.

  The
%
  \methodsource{EnumerableSetsOperations}{union_esets}{}
\end{codeexercise}

\begin{codeexercise}
  Explain why it is not possible to create a function that, given two enumerable sets,
  creates the \emph{intersection} of the two sets.
\end{codeexercise}

