% !TEX root = chapter-standalone.tex

\chaptersecond{Sets and functions}{chapter-long}{chap:sets}{
    Set and functions are fundamental notions in mathematics.
    In this chapter we given an informal treatment of those set-theoretic notions which are important for the purposes of this book, while avoiding further-going aspects.
    The material in this section should be mostly familiar to anyone with some training in engineering, computer science, a natural science, mathematics, etc.
    We suggest nonetheless reading through: we set some conventions, and some contents may be new or rendered in a new perspective.
    The rigorous use of set theory and category theory do involve taking care of some technical details, notably in dealing with set-theory paradoxes and issues related to ``size''.
    However, for our purposes in this book, one can safely ignore these details!
    Nevertheless, since terms and technicalities related to this topic inevitable arise when learning (and reading) category theory, we have included a short explanation of some terms in \cref{sec:technical-terms} below.
    For those interested in deeper discussion, we refer to the excellent exposition \url{https://arxiv.org/abs/0810.1279}.

    \todotext{@J: This abstract is too long -- move half to the first section}
    \todojira{263}{@J: Include set-theory references in the bibliography and link to them}
}

\label{chap:sets}
\subimport{./}{10_sets}
\subimport{./}{15_functions}
\subimport{./}{20_technical-terms}
\codeexercises{
    \subimport{./}{80_exercises}
    \subimport{./}{82_exercises_finitesetproperties}
    \subimport{./}{84_exercises_finitemap}
    \subimport{./}{86_exercises_setproduct}
    \subimport{./}{88_exercises_setunion}
    \subimport{./}{89_setoids}
}
