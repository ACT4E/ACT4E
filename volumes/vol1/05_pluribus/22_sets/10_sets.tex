% !TEX root = chapter-standalone.tex

\section{Sets}

Intuitively speaking, sets describe ``collections of things'' -- whether it be a collection of people, of objects, of abstract symbols, etc.
The ``things'' making up a set are called the \emph{elements} of the set.

There are several typical ways of indicating a set.
One is ``list-like'': we list the elements of the given set, separated by commas, and surround this list with curled brackets.
For example, the set of persons who are co-authors of this book can be written~$\{ \text{Andrea}, \text{Gioele}, \text{Jonathan} \}$, and the set consisting of the symbols ``$\sbanana$'', ``$\sapple$'', ``$\scarrot$'' and ``$\stea$'' is indicated by $\makeset{ \sbanana, \sapple, \scarrot, \stea}$.
\todotext{@J: We call the representation ``list-like'' but actually order and repetition matter for lists.}

In a ``list-like" representation, the order that we list elements in does not matter:
$\makeset{ \sapple, \scarrot, \sbanana, \stea}$ and~$\makeset{ \stea, \sbanana, \sapple, \scarrot}$ and~$\makeset{ \sapple, \scarrot, \stea, \sbanana}$ are, for instance, difference ways of indicating one and the same set.
Also, we do not allow repetitions of elements of sets:
$\makeset{ \stea, \sbanana, \sbanana, \sbanana, \sapple, \scarrot }$ is not a valid way of indicating a set.
Given any two elements of a set, they can be only either equal (identical) or unequal (distinct).
\footnote{There is a notion of ``multiset'', where repetitions are allowed (the ``same'' element can appear multiple times), but we are not considering that notion here.}

\begin{marginfigure}
    \centering
    \includesag{sets_as_clouds}
    \caption{We represent sets as ``clouds'' or ``bags'' of nonrepeating elements.}
    \label{fig:set_as_clouds}
\end{marginfigure}

Because in general the elements of set are not ordered in any way, one often visualizes sets as ``clouds'' or ``bags'' of elements (\cref{fig:set_as_clouds}).

Suppose now for a moment that we have some sets to work with, such as for instance the sets described above.
There are a few basic operations that allow to construct new sets from the ones we have, and to reason about them.

%(One of the ways to formalize set theory rigorously is via the axiomatic route, for which there are various different well-known approaches.)

Firstly, a fundamental notion is the relation of ``being an element of'', indicated by the symbol ``$\in$''.
For example, the statement that the symbol ``$\stea$'' is an element of the set~$\makeset{ \sapple, \scarrot, \sbanana, \stea }$ is written ``$\stea \in \makeset{ \sapple, \scarrot, \sbanana, \stea }$''.
Similarly, ``$\text{Gioele} \in \{ \text{Andrea}, \text{Gioele}, \text{Jonathan} \}$'' is a true statement.

Second, we can talk about whether two sets are \emph{equal} or not.
Two sets~$\setA$ and~$\setB$ are equal if and only if ``they have the same elements''.
In other words,~$\setA$ and~$\setB$ are equal if and only if the statements
``$\ela \in \setA$'' and ``$\ela \in \setB$'' are logically equivalent:
if these statements are always either both true or both false for any instantiation of the variable~$\ela$.
For example, $\makeset{ \sapple, \scarrot, \sbanana, \stea }$ and~$\makeset{ \stea, \sbanana, \sapple, \scarrot}$ and~$\makeset{ \sapple, \scarrot, \stea, \sbanana }$ are all equal as sets, because their elements are the same.

