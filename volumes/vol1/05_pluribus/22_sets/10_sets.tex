% !TEX root = chapter-standalone.tex


\section{Sets}
We are working on this section, and more content will appear.

\devel{
\todotext{Enter some foundational material on sets}

\todotext{Say something about the operations of "element of" and "equals"}

\todotext{Say something about the different ways a set might be specified}

\todotext{Compare sets to other related but distinct notions: multisets, sequences, ... what else?}

The material in this section should be mostly familiar to anyone with some training in engineering, computer science, a natural science, or mathematics, etc. We suggest nonetheless reading through, as some contents may be new or may give a new perspective on known material. 

Intuitively speaking, sets describe ``collections of things'' -- whether it be a collection of people, of objects, of abstract symbols, etc.  The ``things'' making up a set are called the \emph{elements} of the set. There are several typical ways of indicating a set. One is ``list-like'':   we list the elements of the given set, separated by commas, and surround this list with curled brackets. For example, the set of persons who are co-authors of this book can be written $\{ \text{Andrea}, \text{Gioele}, \text{Jonathan} \}$, and the set consisting of the symbols ``$+$'', ``$-$'', ``$?$'' and ``$\%$'' is indicated by $\{ +, -, ?, \% \}$. Let's call the former set $\setA$, and the latter $\setB$. In a ``list-like" representation, the order that we list elements in does not matter: $\{ -, ?, +, \% \}$ and $\{ \%, +, -, ?, \}$ and $\{ -, ?, \%, + \}$ are, for instance, difference ways of indicating one and the same set, namely the set $\setB$ above. Also, we do not allow repetitions: $\{ \%, +, +, +, -, ?, \}$ is not a valid way of indicating a set. Given any two elements of a set, they can be only either equal (identical) or unequal (distinct).\footnote{There is a notion of ``multiset'', where repetitions are allowed (the ``same'' element can appear multiple times), but we are not considering that notion here.} Because in general the elements of set are not ordered in any way, one often visualizes sets as ``clouds'' or ``bags'' of elements. 

\todo{insert figure}

\

Suppose now for a moment that we have some sets to work with, such as the sets $\setA$ and $\setB$ above. There are a few basic operations that allow to construct new sets from the ones we have, and to reason about them. 

(One of the ways to formalize set theory rigorously is via the axiomatic route, for which there are various different well-known approaches.) 

Firstly, a fundamental notion is the relation of ``being an element of'', indicated by the symbol ``$\in$''. For example, the statement that the symbol ``$\%$'' is an element of the set $\setB$ is written ``$\% \in \setB$''. Similarly, ``$\text{Gioele} \in \setA$'' is a true statement, given how we defined the set $\setA$. 

Second, we can talk about whether two sets are equal or not. By definition, two sets are equal if and only if they have the same elements. For example, $\{ -, ?, +, \% \}$ and $\{ \%, +, -, ?, \}$ and $\{ -, ?, \%, + \}$ are all equal as sets, because their elements are the same. 




\

\


\todo{There is a break in content here; it will be filled/bridged}

\


We will name here some special sets that we will find convenient to use later. These names are not all standard; however they should be intuitively clear.

In settings where we'd like to have a specific set of a certain finite size to refer to, we will sometimes use this naming convention:
\begin{align*}
[0]
  & \definedas \emptyset \\
  [1] & \definedas \{ 1\} \\
  [2] & \definedas \{ 1, 2\} \\
  [3] & \definedas \{ 1, 2, 3\}
\end{align*}
...and so on. Generally,
\begin{equation}
[n]
  \definedas \{ 1, ..., n\}
\end{equation}
for any $n \in \natnumbers$.}



\clearpage
