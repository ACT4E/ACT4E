% !TEX root = chapter-standalone.tex


\section{Sets}
We are working on this section, and more content will appear.

\devel{
\todotext{Enter some foundational material on sets}

\todotext{Say something about the operations of "element of" and "equals"}

\todotext{Say something about the different ways a set might be specified}

\todotext{Compare sets to other related but distinct notions: multisets, sequences, ... what else?}

The material in this section should be mostly familiar to anyone with some training in engineering, computer science, a natural science, mathematics, etc. We suggest nonetheless reading through: we set some conventions, and some contents may be new or may be rendered in a new perspective. 

Intuitively speaking, sets describe ``collections of things'' -- whether it be a collection of people, of objects, of abstract symbols, etc.  The ``things'' making up a set are called the \emph{elements} of the set. There are several typical ways of indicating a set. One is ``list-like'':   we list the elements of the given set, separated by commas, and surround this list with curled brackets. For example, the set of persons who are co-authors of this book can be written $\{ \text{Andrea}, \text{Gioele}, \text{Jonathan} \}$, and the set consisting of the symbols ``$+$'', ``$-$'', ``$?$'' and ``$\%$'' is indicated by $\{ +, -, ?, \% \}$. Let's call the former set $\setA$, and the latter $\setB$. In a ``list-like" representation, the order that we list elements in does not matter: $\{ -, ?, +, \% \}$ and $\{ \%, +, -, ?, \}$ and $\{ -, ?, \%, + \}$ are, for instance, difference ways of indicating one and the same set, namely the set $\setB$ above. Also, we do not allow repetitions: $\{ \%, +, +, +, -, ?, \}$ is not a valid way of indicating a set. Given any two elements of a set, they can be only either equal (identical) or unequal (distinct).\footnote{There is a notion of ``multiset'', where repetitions are allowed (the ``same'' element can appear multiple times), but we are not considering that notion here.} Because in general the elements of set are not ordered in any way, one often visualizes sets as ``clouds'' or ``bags'' of elements. 

\todo{insert figure}

\

Suppose now for a moment that we have some sets to work with, such as the sets $\setA$ and $\setB$ above. There are a few basic operations that allow to construct new sets from the ones we have, and to reason about them. 

%(One of the ways to formalize set theory rigorously is via the axiomatic route, for which there are various different well-known approaches.) 

Firstly, a fundamental notion is the relation of ``being an element of'', indicated by the symbol ``$\in$''. For example, the statement that the symbol ``$\%$'' is an element of the set $\setB$ is written ``$\% \in \setB$''. Similarly, ``$\text{Gioele} \in \setA$'' is a true statement, given how we defined the set $\setA$ above. 

Second, we can talk about whether two sets are \emph{equal} or not. Two sets $\setA$ and $\setB$ are equal if and only if ``they have the same elements''. In other words, $\setA$ and $\setB$ are equal if and only if the statements
``$x \in \setA$'' and  ``$x \in \setB$'' are logically equivalent; that is, if these statements are always either both true or both false for any instantiation of the variable $x$. For example, $\{ -, ?, +, \% \}$ and $\{ \%, +, -, ?, \}$ and $\{ -, ?, \%, + \}$ are all equal as sets, because their elements are the same. 

Now consider the set $\{ -, ?, + \}$, and compare it with $\{ -, ?, +, \% \}$. Each of the elements of the first set is also an element of the second set; in such a case we say that the first set is \emph{included} in the second set. In symbols,
\begin{equation}
\{ -, ?, + \} \subseteq \{ -, ?, +, \% \}. 
\end{equation}
The second set in this example is, on the other hand, \emph{not} included in the first set, since $\%$ is an element of the second set, but not the first. If we say a set is ``strictly included'' in another, then we mean ``included in and not equal''; if the adjective ``strictly'' is not used, then inclusion means for us that equality is also possible. In other words, for us it is true that any set $\setA$ is included in itself: $\setA \subseteq \setA$. Given sets $\setA$ and $\setB$, if it is the case that $\setA \subseteq \setB$, then we say that $\setA$ is a \emph{subset} of $\setB$. Inclusion and equality are related as follows: given sets $\setA$ and $\setB$, 
\begin{equation}
\setA = \setB \quad \text{if and only if} \quad \setA \subseteq \setB \ \text{and} \ \setB \subseteq \setA.
\end{equation}

In addition to the ``list-like'' way of specifying a set that we discussed above, many times sets are specified with the help of a logical ``statement'' or ``sentence'' which is used to give a characterization of its elements. The idea is this: we start out with some given set $\setB$, and then we consider a statement $S(x)$ which depends on a variable $x$, which we think of as running over the elements of $\setB$. We can then ask: for which elements $x$ of $\setB$ is the statement $S(x)$ true? These form a subset of $\setB$, often denoted 
\begin{equation}
\{ x \in \setB \mid S(x) \}.
\end{equation}
For example, let $\setB = \{ \text{apple}, \text{banana}, \text{carrot} \}$ and consider the statement 
\begin{equation}
S(x) = \text{``} x \text{ is the name of a fruit''}.
\end{equation}
Then, 
\begin{equation}
\{ x \in \setB \mid x \text{ is the name of a fruit} \} = \{ \text{apple}, \text{banana} \}.
\end{equation}

There is an interesting special case of this way of constructing subsets of a set $\setB$: what if, for a given statement $S(x)$, \emph{none} of the elements $x \in \setB$ are such that $S(x)$ is true? For example, consider the statement $S(x) =\text{ ``} x \text{ is the name of a planet''}$. Then, for the set $\setB = \{ \text{apple}, \text{banana}, \text{carrot} \}$, the subset 
\begin{equation}
\{ x \in \setB \mid \text{ x is the name of a planet} \}
\end{equation}
has \emph{no} elements: it is ``empty''. In most formalizations of set theory, mathematicians agree to use a single unique set, called the \emph{empty set} and denoted $\emptyset$, to play the role of ``the set with no elements''. In particular, by definition, the empty set is a subset of any set!


\

\


\todo{There is a break in content here; it will be filled/bridged}

\


We will name here some special sets that we will find convenient to use later. These names are not all standard; however they should be intuitively clear.

In settings where we'd like to have a specific set of a certain finite size to refer to, we will sometimes use this naming convention:
\begin{align*}
[0]
  & \definedas \emptyset \\
  [1] & \definedas \{ 1\} \\
  [2] & \definedas \{ 1, 2\} \\
  [3] & \definedas \{ 1, 2, 3\}
\end{align*}
...and so on. Generally,
\begin{equation}
[n]
  \definedas \{ 1, ..., n\}
\end{equation}
for any $n \in \natnumbers$.}



\clearpage
