% !TEX root = chapter-standalone.tex


\section{Sets}
We are working on this section, and more content will appear.

\devel{
\todotext{Enter some foundational material on sets}

\todotext{Say something about the operations of "element of" and "equals"}

\todotext{Say something about the different ways a set might be specified}

\todotext{Compare sets to other related but distinct notions: multisets, sequences, ... what else?}

Intuitively speaking, sets describe ``collections of things'' -- whether it be a collection of people, of objects, of abstract symbols, etc. And the ``things'' making up the set are called the \emph{elements} of the set. Usually, a set is indicated using curved brackets and the elements of the set are listed inside, separated by commas. For instance, the set of persons who are co-authors of this book can be written $\{ \text{Andrea}, \text{Gioele}, \text{Jonathan} \}$, and the set consisting of the symbols ``$+$", ``$-"$, ``$?$'' and ``$\%$'' is indicated by $\{ +, -, ?, \% \}$. 

\

\todo{There is a break in content here; it will be filled/bridged}

\


We will name here some special sets that we will find convenient to use later. These names are not all standard; however they should be intuitively clear.

In settings where we'd like to have a specific set of a certain finite size to refer to, we will sometimes use this naming convention:
\begin{align*}
[0]
  & \definedas \emptyset \\
  [1] & \definedas \{ 1\} \\
  [2] & \definedas \{ 1, 2\} \\
  [3] & \definedas \{ 1, 2, 3\}
\end{align*}
...and so on. Generally,
\begin{equation}
[n]
  \definedas \{ 1, ..., n\}
\end{equation}
for any $n \in \natnumbers$.}



\clearpage
