
\section{Power set}
\label{sec:power-set}
\begin{ctdefinition}[Power set]
    \label{def:power-set}
    Given a set~$\setA$, we can form a new set whose elements are the subsets of~$\setA$.
    This new set is called the \iindex{\emph{powerset}} of~$\setA$, and we denote it by~$\powerset \setA$.
\end{ctdefinition}

For example, if~$\setA = \{ \sapple, \scarrot, \sbanana \}$, then its powerset is
\begin{equation*}
    \powerset \setA = \{ \emptyset, \{ \sapple \}, \{ \scarrot \}, \{ \sbanana\}, \{ \sapple, \scarrot \}, \{ \scarrot, \sbanana \}, \{ \sapple, \sbanana\}, \{ \sapple, \scarrot, \sbanana \} \}.
\end{equation*}
The powerset is thus a set of sets.
Two basic operations on the powerset of any set are those of \emph{union} (denoted ``$\cup$'') and \emph{intersection} (denoted ``$\cap$'').
Given subsets~$\subA \subseteq \setA$ and~$\subB \subseteq \setA$, by definition
\begin{equation*}
    \subA \setunion \subB \coloneqq \{ \ela \in \setA \mid \ela \in \subA \text{ or } \ela \in \subB \}
\end{equation*}
and
\begin{equation*}
    \subA \setunion \subB \coloneqq \{ \ela \in \subA \text{ and } \ela \in \subB \}.
\end{equation*}

We will use an assumption (or ``axiom'') that will in fact allow us to take ``arbitrary'' unions and intersections.
Namely, given any two sets~$\setA$ and~$\setB$, we assume that there always exists some other set~$\setC$ such that~$\setA \subseteq \setC$ and~$\setB \subseteq \setC$.
In other words, there exists~$\setC$ such that~$\setA \in \powerset \setC$ and~$\setB \in \powerset \setC$.
And more generally, suppose that we are given not just two sets, but a whole collection~$\mathcal{S}$ of sets, in other words, a set~$\mathcal{S}$ of sets (we are using the word ``collection'' interchangeably with the word ``set'').
We will assume there exists a set~$\setC$ such that~$\setA \in \setC$ for all~$\setA \in \mathcal{S}$.

Now we can take the union and intersection of the collection~$\mathcal{S}$ of sets:
%
\begin{equation*}
    \bigcup \mathcal{S} \coloneqq \{ \ela \in \setC \mid \exists \  \setA \in \setC \colon \ela \in \setA \},
\end{equation*}
%
and
%
\begin{equation*}
    \bigcap \mathcal{S} \coloneqq \{ \ela \in \setC \mid \forall \  \setA \in \setC \colon \ela \in \setA \}.
\end{equation*}



