
\section{Building sets}

In addition to the ``list-like'' way of specifying a set that we discussed above, many times sets are specified with the help of a logical ``statement'' or ``sentence'' which is used to give a characterization of its elements.
The idea is this: we start out with some given set~$\setB$, and then we consider a statement~$S(\ela)$ which depends on a variable~$\ela$, which we think of as running over the elements of~$\setB$.
We can then ask: for which elements~$\ela$ of~$\setB$ is the statement~$S(\ela)$ true?
These elements form a subset of~$\setB$, often denoted
%
\begin{equation}
    \label{eq:axiom-specialization}
    \{ \ela \in \setB \mid S(\ela) \}.
\end{equation}
%
For example, let~$\setB = \{ \sapple, \sbanana, \scarrot \}$ and consider the statement
%
\begin{equation*}
    S(\ela) = \text{``} \ela \text{ is a fruit''}.
\end{equation*}
%
Then we can form the set
%
\begin{equation*}
    \{ \ela \in \setB \mid \ela \text{ is a fruit} \} = \{ \sapple, \sbanana \}.
\end{equation*}

There is an interesting special case of this way of constructing subsets of a set~$\setB$:
what if, for a given statement~$S(\ela)$, \emph{none} of the elements~$\ela \in \setB$ are such that~$S(\ela)$ is true?
For example, consider the statement
%
\begin{equation*}
    S(\ela) =\text{ ``} \ela \text{ is the name of a planet''}.
\end{equation*}
%
Then, for the set~$\setB = \{ \sapple, \sbanana, \scarrot \}$, the subset
\begin{equation*}
    \{ \ela \in \setB \mid \ela \text{ is the name of a planet} \}
\end{equation*}
has \emph{no} elements: it is ``empty''.
In most formalizations of set theory, mathematicians agree to use a single unique set -- called the \emph{empty set} and denoted~$\emptyset$ -- to play the role of ``the set with no elements''.
In particular, by definition, the empty set is a subset of any set!

And now is probably as good a moment as any to note that the empty set~$\emptyset$ should not be confused with the set~$\{ \emptyset \}$.
While~$\emptyset$ denotes the unique set with~$\emph{no elements}$, the notation~$\{ \emptyset \}$ denotes a set which has precisely one element, and that element happens to be the set~$\emptyset$.
Generally, sets can have other sets as their elements: for example, we might consider the set
%
\begin{equation*}
    \{ \{ \sbanana \}, \{ \sapple, \scarrot, \sbanana, \stea \}, \emptyset \},
\end{equation*}
%
which has three elements: namely~$\{ \sbanana \}$,~$\{ \sapple, \scarrot, \sbanana, \stea \}$, and~$\emptyset$.

This leads us nicely to another fundamental construction.

\todotextjira{499}{@J: AC: splitting things in sections introduced some jumps like this that need to be filled / changed. (note that the next heading is exercises)}
