\subsection{Set union \hardexercise}
\begin{remark}
    This is a more advanced exercise; feel free to skip the first time around.
\end{remark}


When in mathematics we compute the union of two sets, we obtain a set but we forget the structure.
We cannot do this when we take a constructive approach.

For example, suppose that we have two \Setoid{}s~$\setA = \{a,b,c\}$ and~$\setB = \{a,b,3\}$.
By all accounts the elements $a,b$ are in both. However, how can we judge that~$a=b$?
Should we use the first setoid or the second? Both? What if they disagree?

We choose this semantics: when comparing two elements, we see the set of components to which they both belong.
Then we compare them using all the equalities. The elements are equal if they are equal according to all of them.

From the point of view of the implementation, we create an interface like
in \cref{lst:SetUnion}, together with \EnumerableSetUnion and \FiniteSetUnion, not shown.

\classlisting{SetUnion}

The method \funcname{components} allows us to recover the components for the union.

\begin{codeexercise}[\exname{TestMakeSetUnion} (for advanced students)]
    Implement the interface in \cref{lst:MakeSetUnion}.
\end{codeexercise}

The \str{@overload} annotations means that:
\begin{compactitem}
    \item if the method is given all \FiniteSet, it should return a \FiniteSet;
    \item if the method is give all \EnumerableSet, it should return an \EnumerableSet;
    \item otherwise, in the general case, it should return a \Setoid.
\end{compactitem}

\classlisting{MakeSetUnion}

Check your results with
\begin{console}
    make check-TestMakeSetUnion
\end{console}


\margindatafilefig{set_union}{Union}{fig:set_union}

Extend now the code you wrote for loading sets to allow the format in \cref{fig:set_product}.

This is the algorithm to implement.

Given a dictionary:
\begin{compactenum}
    \item If it has a field \fieldname{elements}, it is a finite set described with elements directly.
    \item If it has a field \fieldname{union}, then the value must be a list of set representation dictionaries.
    \item Otherwise, it is an error --- for now; we will introduce many more ways to describe sets.
\end{compactenum}

In particular, note that these are valid representations.
\begin{table*}[h]
    \begin{tabular}{cccc}
        An empty set &
        The union of 1 empty set &
        The union of 2 empty sets &
        The union of zero sets \\
%
        \datafile{set_empty}{min_lines=5} &
        \datafile{set_union_empty1}{min_lines=5} &
        \datafile{set_union_empty2}{min_lines=5}
        &
        \datafile{set_union_zero}{min_lines=5}
        \\
    \end{tabular}
\end{table*}

