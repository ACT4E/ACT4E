\section{Disjoint union}

Later we will see that the cartesian product is a special case of a very general construction in category theory.
This construction also has a ``dual'', which, in the specific context of sets, corresponds to the notion \emph{disjoint union}.
Given sets~$\setA$ and~$\setB$, their disjoint union, denoted~$\setA + \setB$ is the set
\begin{equation*}
    \{ \tup{\ela, 1} \mid \ela \in \setA \} \setunion \{ \tup{\elb, 2} \mid \elb \in \setB \}.
\end{equation*}
If again~$\setA = \{ \sapple, \sbanana, \scarrot\}$ and~$\setB = \{ \stea, \swater \}$, then
\begin{equation*}
    \setA + \setB = \{ \tup{\sapple, 1}, \tup{\sbanana, 1}, \tup{\scarrot, 1}, \tup{\stea, 2},  \tup{\swater, 2}\}.
\end{equation*}
In the special case where~$\setA = \emptyset$, then~$\emptyset + \setB = \{ \tup{\elb,2} \mid \elb \in \setB \}$, and similarly~$\setA + \emptyset =  \{ \tup{\ela,1} \mid \ela \in \setA \}$.

\devel{
    \todojira{1}{@J: There is a break in content here; it will be filled/bridged}

    We will name here some special sets that we will find convenient to use later.
    These names are not all standard; however they should be intuitively clear.

    In settings where we'd like to have a specific set of a certain finite size to refer to, we will sometimes use this naming convention:
    \begin{align*}
        [0]
            & \definedas \emptyset    \\
        [1] & \definedas \{ 1\}       \\
        [2] & \definedas \{ 1, 2\}    \\
        [3] & \definedas \{ 1, 2, 3\}
    \end{align*}
    \dots and so on.
    Generally,
    %
    \begin{equation*}
        [n]
        \definedas \{ 1, \dots, n\}
    \end{equation*}
    %
    for any $n \in \natnumbers$.

    \todotext{@J: AC: but are we actully going to use these?}
}
