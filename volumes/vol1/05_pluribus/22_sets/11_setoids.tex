\sectionexercises{Type theory}

\begin{remark}
    You can skip this section until you want to get started on the exercises.
\end{remark}

At a basic level, the notion of set seems uncontroversial.
However, there are some difficulties that led to the development of \emph{type theory},
which can be used as an alternative foundation to mathematics.
Type theory is also used as the foundation for computing; the exercises utilize the type theory nomenclature.

\todotextjira{552}{@Andrea: Find a basic reference to give there}

In type theory there are two basic kinds: \emph{terms} and \emph{types}.
A term is a formal expression built from the constructions in a language.
A term might have a type.
We write $a : T$ to say that $a$ has type $T$.
For example, one could say $42 : \aword{int}$ where $\aword{int}$ is the type of integers.

In set theory there are elements and sets.
It is assumed that elements have ``an identity'': each element is equal to itself and different from the other elements.
Furthermore, elements might \emph{belong} to sets.

In type theory the concepts of \emph{sets} and \emph{belonging} are not primitive and one needs to build them from scratch.
For example, one can define a \emph{setoid} as a set-like structure in type theory.

\begin{ctdefinition}[Setoid]
    A setoid is a tuple $S = \tup{T, \aword{contains}, \aword{equal}}$ where
    \begin{itemize}
        \item $T$ is a type;
        \item $\aword{contains}:T\to\Bool$ is a function that tells use whether an element belongs to the setoid;
        \item $\aword{equal}:T\times T\to\Bool$ is the equality;
              % \item $\aword{apart}:T\times T\to\Bool$ is the apartness function.
    \end{itemize}
\end{ctdefinition}

% Ignore for now why we need $\aword{apart}$ in addition to $\aword{equal}$; we will return to this topic much later in the exposition.

Here is an example that shows the advantages of such construction.
Imagine we want to represent rational numbers with two numbers, nominator and denominator, using arbitrary integers.
We want to define a setoid for which the element $\tup{2, -4}$ is equal to $\tup{-1, 2}$.
Assume that we already have a setoid for integers defined, so that we have a function $\aword{equal}_{\aword{int}}$

We can do it as follows:
\begin{equation}
    S_1 =
    \begin{cases}
        \begin{aligned}
            T                & =
            \aword{Tuple}[\aword{int}, \aword{int}]
            \\
            \aword{contains} & = \tup{a,b} \mapsto \aword{contains}_{\aword{int}}(a)  \wedge \aword{contains}_{\aword{int}}(b)
            \\
            \aword{equal}    & = \tup{\tup{a,b}, \tup{c,d}}
            \mapsto \aword{equal}_{\aword{int}} (a\cdot c, b\cdot d)
        \end{aligned}
    \end{cases}
\end{equation}

One can define a different setoid with the same underlying type $\aword{Tuple}[\aword{int}, \aword{int}]$ where a pair of elements is interpreted as an interval:
\begin{equation}
    S_2 =
    \begin{cases}
        \begin{aligned}
            T                & = \aword{Tuple}[\aword{int}, \aword{int}] \\
            \aword{contains} & = \tup{a,b} \mapsto
            \aword{contains}_{\aword{int}}(a)
            \booland
            \aword{contains}_{\aword{int}}(b)
            \booland
            (a\leq b) \\
            \aword{equal}    & = \tup{\tup{a,b}, \tup{c,d}}
            \mapsto
            \aword{equal}_{\aword{int}}(a,c)
            \booland
            \aword{equal}_{\aword{int}}(b,d)
        \end{aligned}
    \end{cases}
\end{equation}

The elements $\tup{1,4}$ and $\tup{-2, 8}$ are in both setoids.
However, we have that
\begin{equation}
    \aword{equal}_{S_1} (\tup{1,4}, \tup{-2, 8}) = \true
    \qqand
    \aword{equal}_{S_2} (\tup{1,4}, \tup{-2, 8}) = \false .
\end{equation}
Every time we have an equality we should think about in which setoid the equality is understood.
We would rewrite the above as
\begin{equation}
    \tup{1,4} =_{S_1} \tup{-2, 8}
    \qqand
    \tup{1,4} \neq_{S_2} \tup{-2, 8}.
\end{equation}

