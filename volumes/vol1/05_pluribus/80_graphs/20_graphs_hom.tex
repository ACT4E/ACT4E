% !TEX root = chapter-standalone.tex

\section{Graph homomorphisms}
\begin{definition}[Graph homomorphism]
    \label{def:graph_homom}
    Given graphs~$\graph_1=\tup{\vertices_1,\arcs_1,\source_1,\target_1}$ and~$\graph_2=\tup{\vertices_2,\arcs_2,\source_2,\target_2}$, a graph homomorphism~$\funa \colon \graph_1 \fto \graph_2$ is given by maps~$\funaob\colon \vertices_1 \fto \vertices_2$ and~$\funamor\colon \arcs_1 \fto \arcs_2$, such that the following diagrams commute:
    \begin{equation}
        \middlesag{60_graph_homomorphism}
    \end{equation}
    \todotext{@Gioele: Note that at this point we don't know what is a commuting diagram!
        Remove the diagram, write formulas.
    }
\end{definition}
\todographicsjira{551}{
    Use $\funaob, \funamor$ in the figures for graph morphisms (60\_graph\_homomorphism, graph\_homo\_a, graph\_homo\_b).
}

\begin{remark}
    Intuitively, all this is saying is that ``arrows are bound to their vertices'', meaning that if a vertex~$\vertexa_1$ is connected to~$\vertexa_2$ via an arrow~$\arc$, the vertices~$\funaob(\vertexa_1)$ and~$\funaob(\vertexa_2)$ have to be connected via an arrow~$\funamor(\arc)$.
\end{remark}
\todographics{@Gioele:
    Let's not use fruit here.
    Maybe something with semantics?
    Anyway certainly use different sets for vertices and edges (e.g. numbers and letters).
}

\begin{example}
    \label{exa:homomorphism_graph_positive}
    Let us consider the two graphs,~$\graph$ and~$\graph'$ depicted in \cref{fig:ex_graph_homom}.
    \begin{figure*}[h]
        \centering
        \includesag{20_ex_graph_hom}
        \caption{Example of graphs for graph homomorphism.}
        \label{fig:ex_graph_homom}
        \todographicsjira{550}{@Gioele: Let's have a border/background color also for the graphs.}
    \end{figure*}

    \begin{marginfigure}
        \begin{center}
            \includesag{graph_homo_a}
        \end{center}
        \caption{\label{fig:graph_homo_a}}
    \end{marginfigure}

    \begin{marginfigure}
        \begin{center}
            \includesag{graph_homo_b}
        \end{center}
        \caption{\label{fig:graph_homo_b}}
    \end{marginfigure}
    A possible graph homomorphism between the two is given by~$\funaob,\funamor$ graphically defined as in \cref{fig:graph_homo_a} and \cref{fig:graph_homo_b}, respectively.

    \todographics{This is a horrible visualization for a morphism.
        Draw it like we would draw a functor?
    }
\end{example}

\begin{example}[Counterexample]
    By considering the graphs in \cref{exa:homomorphism_graph_positive}, one could define~$\funaob,\funamor$ in the same way, exception made for~$\funaob(\sburger)=\sgrapes$.
    Clearly, this would violate the commuting diagrams condition.
\end{example}

\publictodomessage
\devel{
    \begin{exercise}
        \todojira{71}{write exercise in which we ask to find maps $\funamor$ given $\funaob$}
    \end{exercise}
    \begin{solution}
        \missingsolution
        \todojira{71}{write solution}
    \end{solution}
}
