% !TEX root = chapter-standalone.tex

\section{Graph homomorphisms}
\begin{definition}[Graph homomorphism]
	\label{def:graph_homom}
	Given graphs~$\graph_1=\tup{\vertices_1,\arcs_1,\source_1,\target_1}$ and~$\graph_2=\tup{\vertices_2,\arcs_2,\source_2,\target_2}$, a graph homomorphism~$\funa \colon \graph_1 \fto \graph_2$ is given by maps~$\funaob\colon \vertices_1 \fto \vertices_2$ and~$\funamor\colon \arcs_1 \fto \arcs_2$, such that:
	\begin{equation}
		(\funamor \mthen \source_2)(\arc)=(\source_1\mthen \funaob)(\arc), \quad \forall \arc\in \arcs_1,
	\end{equation}
	and
	\begin{equation}
		(\funamor \mthen \target_2)(\arc)=(\target_1\mthen \funaob)(\arc), \quad \forall \arc\in \arcs_1.
	\end{equation}
	%\begin{equation}
	%    \middlesag{60_graph_homomorphism}
	%\end{equation}
\end{definition}

\begin{remark}
	Intuitively, all this is saying is that ``arrows are bound to their vertices'', meaning that if a vertex~$\vertexa_1$ is connected to~$\vertexa_2$ via an arrow~$\arc$, the vertices~$\funaob(\vertexa_1)$ and~$\funaob(\vertexa_2)$ have to be connected via an arrow~$\funamor(\arc)$.
\end{remark}
\todographics{@Gioele:
	Let's not use fruit here.
	Maybe something with semantics?
	Anyway certainly use different sets for vertices and edges (e.g. numbers and letters).
}

\begin{example}
	\label{exa:homomorphism_graph_positive}
	Let us consider the two graphs,~$\graph$ and~$\graph'$ depicted in \cref{fig:ex_graph_homom}.
	\begin{figure*}[h]
		\centering
		\includesag{20_ex_graph_hom}
		\caption{Example of graphs for graph homomorphism.}
		\label{fig:ex_graph_homom}
	\end{figure*}

	\begin{marginfigure}
		\begin{center}
			\includesag{graph_homo_a}
		\end{center}
		\caption{\label{fig:graph_homo_a}}
	\end{marginfigure}

	\begin{marginfigure}
		\begin{center}
			\includesag{graph_homo_b}
		\end{center}
		\caption{\label{fig:graph_homo_b}}
	\end{marginfigure}
	A possible graph homomorphism between the two is given by~$\funaob,\funamor$ graphically defined as in \cref{fig:graph_homo_a} and \cref{fig:graph_homo_b}, respectively.
\end{example}

\begin{example}[Counterexample]
	By considering the graphs in \cref{exa:homomorphism_graph_positive}, one could define~$\funaob,\funamor$ in the same way, exception made for~$\funaob(\sburger)=\sgrapes$.
	Clearly, this would violate the commuting diagrams condition.
\end{example}

\publictodomessage
\devel{
	\begin{exercise}
		\todojira{71}{write exercise in which we ask to find maps $\funamor$ given $\funaob$}
	\end{exercise}
	\begin{solution}
		\missingsolution
		\todojira{71}{write solution}
	\end{solution}
}
