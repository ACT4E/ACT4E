% !TEX root = chapter-standalone.tex

% \linkvideo{spring2021-morphisms:morphisms} % Morphisms
% \linkvideo{spring2021-semi-mon-gro:summary} % Summary

\section{Semigroup morphisms}
\label{sec:semigroup-morphisms}

A \emph{morphism} is a map between \SY{semigroups} that ``preserves the structure'' of composition.

% \todotextjira{16}{@Andrea: More intro discussion}

%\linkvideo{spring2021-morphisms:morphisms:semigroup-morphisms}
\begin{marginfigure}
    \centering
    \includesag{semigroup_morphism_diagram}
    \caption{Semigroup Morphism}
    \label{fig:semigrouip-morphism-diagram}
\end{marginfigure}

\begin{ctdefinition}[Semigroup morphism]
    \label{def:semigroup-mor}
    \SYNDEF{semigroup morphism}
    A morphism~$\sgrpmorA\colon \sgrpA \fto \sgrpB$ between \SY{semigroups} \begin{equation}
        \label{eq:sgrpAB}
        \sgrpA = \tup{\sgrpAset, \mtimesof\sgrpA}
        \qqand
        \sgrpB = \tup{\sgrpBset, \mtimesof\sgrpB}
    \end{equation}
    is a function~$\sgrpmorA\colon \sgrpAset \to \sgrpBset$ such that for all~$\sgrpela, \sgrpelb \setin \sgrpAset$,
    \begin{equation}
        \label{eq:sgrp-mor-comp}
        \sgrpmorA(\sgrpela \mtimesof{\sgrpA} \sgrpelb) = \sgrpmorA (\sgrpela) \mtimesof{\sgrpB} \sgrpmorA(\sgrpelb).
    \end{equation}
\end{ctdefinition}



Note that we use~$\sgrpmorA\colon \sgrpAset \sto \sgrpBset$ when we want to highlight the function between sets, and we use~$\sgrpmorA: \sgrpA \fto \sgrpB$ when we want to highlight the relation between \SY{semigroup} structures.
We think of \cref{eq:sgrp-mor-comp} as a way of saying that the function~$\sgrpmorA\colon \sgrpAset \sto \sgrpBset$ is \emph{compatible} with the multiplication operations on~$\sgrpA$ and~$\sgrpB$, respectively.

\begin{ctdefinition}[Identity morphism]
    \label{def:identity-sgrp-mor}
    \SYNDEF{identity morphism of semigroups}
    Let~$\sgrpA$ be a \SY{semigroup}.
    The \SY{identity function}~$\mapidat\sgrpAset\colon \sgrpAset \sto \sgrpAset$ is always a \SY{morphism of semigroups}.
    We can easily check that \cref{eq:sgrp-mor-comp} is satisfied:
    \begin{equation}
        \label{eq:identity-sgrp-mor-proof}
        \funid_\sgrpA (\sgrpela \mtimesof{\sgrpA} \sgrpelb) \ = \ \sgrpela \mtimesof{\sgrpA} \sgrpelb
        \ = \
        \funid_\sgrpA (\sgrpela )\mtimesof{\sgrpA}\funid_\sgrpA(\sgrpelb).
    \end{equation}
    We call this the \emph{identity morphism} of~$\sgrpA$.
\end{ctdefinition}

\linkvideo{spring2021-morphisms:morphisms:semigroup-morphisms:semigroup-isomorphisms}

\SYNDEF{morphism of semigroups}

\begin{ctdefinition}[Semigroup isomorphism]
    \label{def:semigroup-isomorphism}
    A \SY{morphism of semigroups}~$\sgrpmorA\colon \sgrpA \mto \sgrpB$ is called a \maindef{semigroup isomorphism} if there exists a \SY{morphism of semigroups}~$\sgrpmorB\colon \sgrpB \mto \sgrpA$ such that
    \begin{equation}
        \label{eq:sgrp-iso-cond}
        \sgrpmorA \fthen \sgrpmorB = \funid_\sgrpA
        \qqand
        \sgrpmorB \fthen \sgrpmorA = \funid_\sgrpB.
    \end{equation}
\end{ctdefinition}

\begin{lemma}
    \label{lem:semigroup-morphisms-compose}
    The composition of \SY{semigroup morphisms} is a morphism:
    \begin{equation}
        \label{eq:sgrp-mor-compo}
        \vmiddle{
            \prfperiod{
                \funa \colon \sgrpA \fto \sgrpB
            }{
                \funb \colon \sgrpB \fto \sgrpC
            }{
                (\funa \fthen \funb) \colon \sgrpA \fto \sgrpC
            }
        }
    \end{equation}
\end{lemma}

\begin{exercise}
    Prove \cref{lem:semigroup-morphisms-compose}.
\end{exercise}
\begin{solution}
    We have:
    \begin{equation}
        \begin{aligned}
            (\funa \fthen \funb)(\sgrpela \mtimesof{\sgrpA} \sgrpelb)
             & =\funb(\funa(\sgrpela \mtimesof{\sgrpA} \sgrpelb)) \\
             & =\funb(\funa(\sgrpela)\mtimesof{\sgrpB}\funa(\sgrpelb)) \\
             & =\funb(\funa(\sgrpela))\mtimesof{\sgrpC} \funb(\funa(\sgrpelb)) \\
             & =(\funa \fthen \funb)(\sgrpela) \mtimesof{\sgrpC} (\funa \fthen \funb)(\sgrpelb).
        \end{aligned}
    \end{equation}
\end{solution}

\begin{quiz}[\exname{SemigroupMorphismAssociativity}]
Check if the following statement is true: For any semigroups $(S,\mthen_{S}),(T,\mthen_{T})$ and $F:S\to T$a morphism of semigroups it holds that for any $x,y,z\in S$:
$$F((x\mthen_{S}y)\mthen_{S}z)=F(x)\mthen_{T}(F(y)\mthen_{T}F(z)).$$
\begin{enumerate}[label=(\alph*)]
\item True.
\item False.
\item Depends on additional properties of $S$ and $T$.
\item I have no clue. 
\end{enumerate} 
\end{quiz}

\begin{example}[Logarithms and exponentials]
    The positive reals with multiplication~$\tup{\posReals, \cdot}$ is a \SY{semigroup}.
    The reals with addition~$\realswithplusmonoid$ is a \SY{semigroup}.

    Now consider as a bridge between the two: the logarithmic function.
    We have
    \begin{equation}
        \label{eq:log}
        \mlog \colon \posReals \sto \reals,
    \end{equation}
    and its inverse
    \begin{equation}
        \label{eq:exp}
        \mexp \colon \reals \sto \posReals.
    \end{equation}
    We already know that these are inverse of each other:
    \begin{equation}
        \label{eq:exp-log-are-inverses}
        \begin{aligned}
            \mexp \fthen \mlog & = \funid_{\reals}, \\
            \mlog \fthen \mexp & = \funid_{\posReals}.
        \end{aligned}
    \end{equation}
    We can verify that~$\log$ is also a \SY{semigroup morphism}, because of the following property of the logarithms:
    \begin{equation}
        \label{eq:log-property}
        \mlog(a \cdot b) = \mlog(a) + \mlog(b).
    \end{equation}
    Because~$\mlog$ is a \SY{bijection} and~$\mexp$ is its inverse, it already follows that~$\mexp$ is a morphism in the opposite direction.
    Alternatively we can see that is the case because of the property of the exponential function:
    \begin{equation}
        \label{eq:exp-property}
        \mexp(c + d) = \mexp(c) \cdot \mexp(d).
    \end{equation}
    \Cref{eq:log-property,eq:exp-property} are both \cref{eq:sgrp-mor-comp} in disguise.
\end{example}

\begin{example}[Transition function, continuation of \cref{exa:transition-functions}]
    Consider the map
    \begin{equation}
        f\colon \nonNegReals \to (\reals^n \sto \reals^n)
    \end{equation}
    that associates to a delta~$\delta$ its transition function~$T_\delta$.
    Re-reading~\cref{eq:transition-property}, we can see that it is a morphism between the \SY{semigroup}~$\tup{\nonNegReals,+}$ and the \SY{semigroup} of endomorphisms of~$\reals^n$.
\end{example}

\begin{example}[State dimension of discrete dynamical systems]
    Consider the \SY{semigroup} of discrete dynamical systems described in \cref{sec:dynamical-systems}, and denote it by~$\sgrpA$.
    As we have seen composing two DDS~$\dtsysa,\dtsysb$ gives a DDS~$\dtsysa \mtimesof{\sgrpA} \dtsysb$, state of which has the dimension of the state of~$\dtsysa$ plus the dimension of the state of~$\dtsysb$.
    We now consider a map~$\dimstatemap\colon \text{DDS} \to \natnumbers$, which given a DDS returns the dimension of its state space.
    Now, by considering the \SY{semigroup}~$\tup{\natnumbers,+}$, denoted by~$\sgrpB$, one knows
    \begin{equation}
        \begin{aligned}
            \dimstatemap(\dtsysa\mtimesof{\sgrpA} \dtsysb) & =\dimstatemap(\dtsysa)+\dimstatemap(\dtsysb) \\
                                                           & =\dimstatemap(\dtsysa)\mtimesof{\sgrpB} \dimstatemap(\dtsysb),
        \end{aligned}
    \end{equation}
    showing that~$\mora$ is a valid \SY{semigroup morphism} between the \SY{semigroup} of DDS and~$\tup{\natnumbers,+}$.
\end{example}
\vfill%exercises
\begin{gradedexercise}[\exname{IsoViaTables}]
    \label{ex:IsoViaTables}
    \label{ex:sem-compare-tables}
    Consider the set~$\setA = \makeset{ \alphabetasymba, \alphabetasymbb }$ and the following three composition tables, each of which defines a \SY{semigroup} structure on~\setA.
    \begin{center}
        \begin{tabular}{c|cc}
            $\mtimesof 1$   & $\alphabetasymba$ & $\alphabetasymbb$ \\
            \hline
            \alphabetasymba & \alphabetasymba   & \alphabetasymba \\
            \alphabetasymbb & \alphabetasymba   & \alphabetasymbb
        \end{tabular}
        $\quad$
        \begin{tabular}{c|cc}
            $\mtimesof 2$   & $\alphabetasymba$ & $\alphabetasymbb$ \\
            \hline
            \alphabetasymba & \alphabetasymba   & \alphabetasymbb \\
            \alphabetasymbb & \alphabetasymbb   & \alphabetasymba
        \end{tabular}
        $\quad$
        \begin{tabular}{c|cc}
            $\mtimesof 3$   & $\alphabetasymba$ & $\alphabetasymbb$ \\
            \hline
            \alphabetasymba & \alphabetasymba   & \alphabetasymbb \\
            \alphabetasymbb & \alphabetasymbb   & \alphabetasymbb
        \end{tabular}
    \end{center}
    Which of the three \SY{semigroups} defined in this way are isomorphic to each other?
    Justify your answer.
\end{gradedexercise}

\solutionof{IsoViaTables}

\begin{gradedexercise}[\exname{SemigroupUpToIso}]
    \label{ex:non-isomorphic}
    How many different non-isomorphic \SY{semigroups} are there with precisely one element?
    How many with precisely two elements?
    Can you prove your answer?
\end{gradedexercise}
\solutionof{SemigroupUpToIso}

\begin{gradedexercise}[\exname{CharacterizeSemigroupIsos}]
    \label{ex:CharacterizeSemigroupIsos}
    \label{ex:semi-morph}
    Let~$\sgrpmorA\colon \sgrpA \mto \sgrpB$ be a \SY{morphism of semigroups}.
    Prove that~$\sgrpmorA$ is an \SY{isomorphism of semigroups} if and only if the function~$\sgrpmorA \colon \sgrpAset \sto \sgrpBset$ is \SY{bijective}.
\end{gradedexercise}
\solutionof{CharacterizeSemigroupIsos}

\section{Encoding as morphism }
\linkvideo{spring2021-morphisms:morphisms:semigroup-morphisms:ascii}

\begin{example}[ASCII code]
    ASCII encoding takes any alphanumerical characters and symbols into a number between 0 and 127 (\cref{fig:asciifigure}).
    Call~$\alphanums$ the set of those 128 symbols.
    We can see ASCII encoding as a \SY{semigroup morphism} of~$\listsof\alphanums$ to the free \SY{semigroup} on the integers~$\listsof\asciichar$:
    \begin{equation}
        \label{eq:ASCII-semi}
        \asciienc\colon \listsof\alphanums \to \listsof\asciichar.
    \end{equation}
    Because we can also go back, by using the inverse function,
    \begin{equation}
        \label{eq:ASCII-semi-inv}
        \asciienc^{-1}\colon \listsof\asciichar \to \listsof\alphanums,
    \end{equation}
    ASCII encoding is also an \SY{isomorphism of semigroups}.
\end{example}

\begin{example}[ASCII code to binary]
    Currently, computers use binary to store data. (There were, in fact, \emph{trinary} computers.)
    In \cref{fig:asciifigure}, you can see represented also the binary encoding of each character.
    Therefore, we can see ASCII as a morphism between~$\listsof\alphanums$ and binary lists~$\listsof{\makeset{0,1}}$.
\end{example}

\begin{exercise}
    Show that the morphism
    \begin{equation}
        \label{eq:ascii-to-bin}
        \asciienc: \listsof\alphanums \to \listsof{\makeset{0,1}}
    \end{equation}
    is \emph{not} an isomorphism.
\end{exercise}
\begin{solution}
    We can show that we cannot find an inverse morphism
    \begin{equation}
        \label{eq:bin-to-ascii}
        \asciienc^{-1}\colon \listsof{\makeset{0,1}} \to \listsof\alphanums.
    \end{equation}
    At first sight everything seems in order: if we can find an isomorphism to~$\listsof\asciichar$, and we can express integers in binary, what could hold us back?

    What fails here is something so simple it could go unnoticed: the hypothetical function~$\mapb$ is not well-defined for all points of its domain.
    We know how to translate a binary string of length~$7,14,21,\ldots$ back to symbols; but what would be the output of~$\mapb$ on the string~$111$?

    The function~$\mapb$ is a left inverse for~$\asciienc$, in the sense that~$\asciienc\mtimes \mapb = \idmon_{\listsof\alphanums}$, but it is not a right inverse.

\end{solution}

\vfill
\begin{figure}[h]
    \includegraphics[width=\textwidth]{ascii}
    \caption{7-bit US-ASCII encoding. \label{fig:asciifigure}}
\end{figure}

\section{Morse coding}
\linkvideo{spring2021-morphisms:morphisms:semigroup-morphisms:morse}

\begin{margintable}
    \footnotesize
    \centering
    \caption{Morse encoding}

    \label{tab:morse}
    \begin{tabular}{rl}
        \mst A & \morsedot \morsedash \\
        \mst B & \morsedash \morsedot \morsedot \morsedot \\
        \mst C & \morsedash \morsedot \morsedash \morsedot \\
        \mst D & \morsedash \morsedot \morsedot \\
        \mst E & \morsedot \\
        \mst F & \morsedot \morsedot \morsedash \morsedot \\
        \mst G & \morsedash \morsedash \morsedot \\
        \mst H & \morsedot \morsedot \morsedot \morsedot \\
        \mst I & \morsedot \morsedot \\
        \mst J & \morsedot \morsedash \morsedash \morsedash \\
        \mst K & \morsedash \morsedot \morsedash \\
        \mst L & \morsedot \morsedash \morsedot \morsedot \\
        \mst M & \morsedash \morsedash \\
        \mst N & \morsedash \morsedot \\
        \mst O & \morsedash \morsedash \morsedash \\
        \mst P & \morsedot \morsedash \morsedash \morsedot \\
        \mst Q & \morsedash \morsedash \morsedot \morsedash \\
        \mst R & \morsedot \morsedash \morsedot \\
        \mst S & \morsedot \morsedot \morsedot \\
        \mst T & \morsedash \\
        \mst U & \morsedot \morsedot \morsedash \\
        \mst V & \morsedot \morsedot \morsedot \morsedash \\
        \mst W & \morsedot \morsedash \morsedash \\
        \mst X & \morsedash \morsedot \morsedot \morsedash \\
        \mst Y & \morsedash \morsedot \morsedash \morsedash \\
        \mst Z & \morsedash \morsedash \morsedot \morsedot \\
    \end{tabular}
    \begin{tabular}{rl}
        \mst 0 & \morsedash \morsedash \morsedash \morsedash \morsedash \\
        \mst 1 & \morsedot \morsedash \morsedash \morsedash \morsedash \\
        \mst 2 & \morsedot \morsedot \morsedash \morsedash \morsedash \\
        \mst 3 & \morsedot \morsedot \morsedot \morsedash \morsedash \\
        \mst 4 & \morsedot \morsedot \morsedot \morsedot \morsedash \\
        \mst 5 & \morsedot \morsedot \morsedot \morsedot \morsedot \\
        \mst 6 & \morsedash \morsedot \morsedot \morsedot \morsedot \\
        \mst 7 & \morsedash \morsedash \morsedot \morsedot \morsedot \\
        \mst 8 & \morsedash \morsedash \morsedash  \morsedot \morsedot \\
        \mst 9 & \morsedash \morsedash \morsedash \morsedash \morsedot \\
    \end{tabular}

\end{margintable}

\begin{margintable}
    \caption{5 symbols for Morse encoding}
    \small
    \begin{tabular}{cll}
        \morsedot  & beep of length $\ell$     & \Morsedot \\
        \morsedash & beep of length $3\ell$    & \Morsedash \\
        \morsedsp  & silence of length $\ell$  & \Morsedsp \\
        \morselsp  & silence of length $3\ell$ & \Morselsp \\
        \morsewsp  & silence of length $7\ell$ & \Morsewsp
    \end{tabular}
    \label{tab:morse5}
\end{margintable}

\begin{example}[Morse code]
    \label{exa:morse-code}
    Consider the Morse code: a way to encode the letters and numerals to an alphabet of dots ($\morsedot$) and dashes ($\morsedash$).
    The encoding is shown in \cref{tab:morse}.
    Here, the alphabet~$\morsesymbols$ is the letters A--Z and the numbers 0--9:
    \begin{equation}
        \label{eq:morsesymbols}
        \morsesymbols = (\texttt{A} \ \text{to}\ \texttt{Z}) \setunion (0 \ \text{to}\ 9).
    \end{equation}

    There is no difference between upper and lower case, and there are no punctuation marks.

    Transcribing a text in Morse code is not just a matter of creating the right sequence of dots and dashes.
    The standard also requires a certain timing of the events.
    These are the rules:
    \begin{enumerate}
        \item If the length of~$\morsedot$ is $\ell$, then the length of~$\morsedash$ must be $3\ell$.
        \item Between dashes and dots of the same letter, there must be a silence of $\ell$.
        \item Between different letters, there must be a silence of $3\ell$.
        \item Between different words, there must be a silence of $7\ell$.
    \end{enumerate}

    Therefore, there are 5 symbols in the Morse alphabet (\cref{tab:morse5}); each representing a \emph{signal}.

    Define now the extended alphabet~$\morseesymbols$ to be the union of~$\morsesymbols$ and the set~$\makeset{\lettersp, \wordsp}$, where~$\lettersp$ is inter-letter space, and~$\wordsp$ is inter-word space:
    \begin{equation}
        \label{eq:morseesymbols}
        \morseesymbols = (\texttt{A} \ \text{to}\ \texttt{Z}) \setunion (0 \ \text{to}\ 9) \setunion \makeset{\lettersp, \wordsp}.
    \end{equation}

    Therefore, to encode the sentence
    %
    \begin{equation}
        \text{``I am well''}
    \end{equation}
    %
    we first transform it to upper case:
    %
    \begin{equation}
        \text{``I AM WELL''}.
    \end{equation}
    %
    Then we note the inter-letter space and the inter-word spaces:
    %
    \begin{equation}
        \texttt{I} \wordsp \texttt{A} \lettersp \texttt{M} \wordsp \texttt{W} \lettersp
        \texttt{E} \lettersp \texttt{L} \lettersp \texttt{L}.
    \end{equation}
    %
    At this point we can substitute the Morse code to obtain
    %
    \begin{equation}
        \morseI \morsewsp \morseA \morselsp \morseM \morsewsp \morseW \morselsp \morseE \morselsp \morseL \morselsp \morseL.
    \end{equation}
    %
    In signal space---what somebody would hear---this becomes
    %
    \begin{equation}
        \MorseI \Morsewsp \MorseA \Morselsp \MorseM \Morsewsp \MorseW \Morselsp \MorseE \Morselsp \MorseL \Morselsp \MorseL.
    \end{equation}
    %
    With this representation it is clear that 5 symbols are redundant: if we have a 1-period beep and a 1-period silence, we can obtain the 3-period silence and beeps and the 7-period silence.

    In the end, the Morse alphabet is \emph{binary} in the sense that it all reduces to two symbols: not~$\makeset{\morsedot, \morsedash}$ but rather the alphabet~$\makeset{\Morsedot, \Morsedsp}$.
\end{example}

\begin{exercise}[Morse Morphism]
    We have seen that Morse code transforms a word in the alphabet
    \begin{equation}
        (\texttt{A} \ \text{to}\ \texttt{Z})
        \setunion (0 \ \text{to}\ 9) \setunion \makeset{\lettersp, \wordsp}
    \end{equation}
    to the binary alphabet
    %
    \begin{equation}
        \setB = \makeset{\Morsedot, \Morsedsp}.
    \end{equation}
    %
    Is this map a morphism of list semigroups?
\end{exercise}
%
\begin{solution}
    The answer is \textbf{no} because the encoding is context dependent; I don't know if a single letter is followed by a space or another letter.
    For example, take the string
    \begin{equation}
        \label{eq:max-iammax}
        \mst {I} \wordsp \mst{A} \mst{M}\wordsp \mst{MAX}.
    \end{equation}
    We can decompose it as follows
    \begin{equation}
        \label{eq:max-iammax-decomp}
        \mst{I}\wordsp \mst{A}\mtimes \mst{M} \mtimes \wordsp \mtimes \mst{M} \mtimes \mst{AX}.
    \end{equation}
    If Morse encoding was a morphism~$\sgrpmorA$ then we would be able to encode the string as follows:
    \begin{equation}
        \label{eq:max-iammax-decomp-morph}
        \morsemorph(\mst{I}\wordsp \mst{A}) \mtimes \morsemorph (\mst{M}) \mtimes \morsemorph(\wordsp) \mtimes \morsemorph(\mst{M})
        \mtimes \morsemorph(\mst{AX}).
    \end{equation}
    However, this cannot work, because in the second instance of $M$ we would need to output a letter separator, while in the first case we don't.

    Can you find a way to fix it?

    For example, consider the alphabet obtained by taking the \emph{product} of the letters and numbers with the set of spaces~$\makeset{\wordsp, \lettersp}$:
    \begin{equation}
        \pars{ (\texttt{A} \ \text{to}\ \texttt{Z}) \setunion (0 \ \text{to}\ 9) } \cartprod \makeset{\wordsp, \lettersp},
    \end{equation}
    where we annotate if each symbol is followed by a letter or by a space.

    In this representation, the string can be written as
    \begin{equation}
        \tup{I, \wordsp} \tup{A, \lettersp} \tup{M, \wordsp} \tup{M, \lettersp} \tup{A, \lettersp}
        \tup{X, \wordsp}.
    \end{equation}
    Based on this representation we can define context-independent rules that make a morphism.
\end{solution}

\devel{
    \todotextjira{17}{\bernina: @Andrea: Nice example: the map from sequence of characters to sequences of sounds is monoidal in certain languages (Korean, Japanese, almost in Italian.) and also invertible.}

    \begin{example}[Phonetic languages]
        \publictodomessage

    \end{example}
}

\section{Monoid morphisms}
\linkvideo{spring2021-morphisms:morphisms:monoid-morphisms}
We have defined \SY{semigroup morphisms}.
A \SY{monoid morphism} has the same properties, and one additional one: the constraint that it be compatible with the identity elements.

\begin{marginfigure}
    \centering
    \includesag{monoid_morphism_diagram}
    \caption{Compatibility with identity elements}
    \label{fig:monoid-morphism-diagram}
\end{marginfigure}

\begin{ctdefinition}[Monoid morphism]
    \label{def:monoid-mor}
    \SYNDEF{morphism of monoids}
    A morphism~$\sgrpmorA\colon \monoidA \fto \monoidB$ between \SY{monoids}
    \begin{equation}
        \monoidAdefinition
        \qqand
        \monoidBdefinition
    \end{equation}
    is a function~$\sgrpmorA \colon \monoidAset \sto \monoidBset$ such that for all~$\monela, \monelb\setin \monoidAset$,
    \begin{equation}
        \label{eq:mon-mor-comp}
        \sgrpmorA (\monela \mtimesof{\monoidA} \monelb) = \sgrpmorA (\monela) \mtimesof{\monoidB} \sgrpmorA(\monelb),
    \end{equation}
    and
    \begin{equation}
        \label{eq:mon-id-comp}
        \sgrpmorA (\idmon_\monoidA) = \idmon_{\monoidB}.
    \end{equation}
\end{ctdefinition}

\begin{example}
    The set~$\monoidAset = \makeset{ \minusone, 0, \plusone }$, together with multiplication of whole numbers and with~$1$ as \SY{neutral element}, forms a \SY{monoid}.
    The inclusion map $\monoidA \sto \wnumbers$ is a \SY{morphism of monoids}.
\end{example}

\begin{example}
    \label{exa:string-length}
    Consider the \SY{monoid}~$\monoidB$ in \cref{exa:string-monoid} and the \SY{monoid}~$\monoidA=\tup{\natnumbers, +, 0}$.
    Define a map
    \begin{equation}
        \flength \colon \monoidB \sto \monoidA
    \end{equation}
    that maps each string to its length.
    This is a \SY{monoid morphism}, since:
    \begin{equation}
        \flength(\monela \mtimesof\monoidB \monelb)=\flength(\monela)\mtimesof\monoidA \flength(\monelb).
    \end{equation}
    In other words, the length of the concatenation of two lists is the sum of the lengths of the two lists.
\end{example}

\begin{ctdefinition}[Identity morphism]
    \label{def:identity-mon-mor}
    Let~$\monoidA$ be a \SY{monoid}.
    \SYNDEF{identity morphism of monoids}
    Similar to the case of \SY{semigroups}, the identity function induces a \SY{morphism of monoids}~$\funid_\monoidA\colon \monoidA \fto \monoidA$.
\end{ctdefinition}

\begin{ctdefinition}[Monoid isomorphism]
    \label{def:monoid-isomorphism}
    A \SY{morphism of monoids}~$\funa\colon \monoidA \fto \monoidB$ is called a \maindef{monoid isomorphism} if there is a \SY{morphism of monoids}~$\funb\colon \monoidB \fto \monoidA$ such that
    \begin{equation}
        \funa \fthen \funb = \funid_\monoidA \qqand \funb \fthen \funa = \funid_\monoidB.
    \end{equation}
\end{ctdefinition}

\todotextjira{728}{\bernina: @JL: include examples of monoid isomorphims and discuss how this is a notion of sameness}

\clearpage

~

\vfill%exercises

\begin{gradedexercise}[\exname{MorphismMonoidIsomorphism}]
    Prove: a \SY{morphism of monoids}~$\funa\colon \monoidA \fto \monoidB$ is an \SY{isomorphism of monoids} if and only if the function~$\funa\colon \monoidAset \sto \monoidBset$ is \SY{bijective}.
\end{gradedexercise}
\solutionof{MorphismMonoidIsomorphism}

\begin{gradedexercise}[\exname{TraceAndDeterminant}]
    \label{ex:TraceAndDeterminant}
    Let~$\reals^{n \times n}$ denote the set of real~$n \times n$ matrices,~$n \setin \natnumbers$.
    These form a monoid $\tupp{\reals^{n \times n}\com \cdot, \idmat}$, with matrix multiplication as composition, and the identity matrix as neutral element.
    They also form a monoid $\tupp{\reals^{n \times n}, {{+}}, 0}$ with matrix addition as composition, and the zero matrix as neutral element.
    We also note that that~$\tup{\reals, {{\cdot}}, 1}$ and~$\tup{\reals, {{+}}, 0}$ are two different monoid structures on~$\reals$.

    Recall that the trace of a real~$n \times n$ matrix is the sum of its diagonal entries.
    This defines a function
    \begin{equation}
        \trace \colon \reals^{n \times n} \sto \reals.
    \end{equation}
    Computing the determinant also corresponds to a function
    \begin{equation}
        \deter \colon \reals^{n \times n} \sto \reals.
    \end{equation}

    \begin{enumerate}
        \item Does~$\trace$ define a morphism of monoids~$\tup{\reals^{n \times n}, {{\cdot}},  \idmat} \mto \tup{\reals, {{\cdot}},  1}$?
        \item Does~$\trace$ define a morphism of monoids~$\tup{\reals^{n \times n}, {{\cdot}},  \idmat} \mto \tup{\reals, {{+}},  0}$?
        \item Does~$\deter$ define a morphism of monoids~$\tup{\reals^{n \times n}, {{\cdot}},  \idmat} \mto \tup{\reals, {{\cdot}},  1}$?
        \item Does~$\deter$ define a morphism of monoids~$\tup{\reals^{n \times n}, {{\cdot}},  \idmat} \mto \tup{\reals, {{+}},  0}$?
        \item Does~$\trace$ define a morphism of monoids~$\tup{\reals^{n \times n}, {{+}},  0} \mto \tup{\reals, {{\cdot}},  1}$?
        \item Does~$\trace$ define a morphism of monoids~$\tup{\reals^{n \times n}, {{+}},  0} \mto \tup{\reals, {{+}},  0}$?
        \item Does~$\deter$ define a morphism of monoids~$\tup{\reals^{n \times n}, {{+}},  0} \mto \tup{\reals, {{\cdot}},  1}$?
        \item Does~$\deter$ define a morphism of monoids~$\tup{\reals^{n \times n}, {{+}},  0} \mto \tup{\reals, {{+}},  0}$?
    \end{enumerate}
    Short answers (without proof) are fine.
\end{gradedexercise}
\solutionof{TraceAndDeterminant}

\section{Group morphisms}
\linkvideo{spring2021-morphisms:morphisms:group-morphisms}

After \SY{semigroup morphisms} and \SY{monoid morphisms}, we define \SY{group morphisms}.

\todotextjira{21}{\bernina: @JL: Show the inverse operation being compatible with group structure, commuting with morphisms.
    This is the simplest example of a dagger category, to be explained later on.
}

\begin{ctdefinition}[Group morphism]
    \label{def:group-mor}
    \SYNDEF{morphism of groups}
    A morphism~$\sgrpmorA\colon \grpA \fto \grpB$ between groups
    \begin{equation}
        \grpA = \tup{\grpAset, {{\mtimesof\grpA}}, \idgrp_\grpA, \ginv_\grpA}
        \qqand
        \grpB = \tup{\grpBset, {{\mtimesof\grpB}}, \idgrp_\grpB, \ginv_\grpB}
    \end{equation}
    is a function~$\sgrpmorA \colon \grpAset \sto \grpBset$ such that for all~$\monela, \monelb\setin \grpAset$,
    \begin{equation}
        \label{eq:group-mor-comp}
        \sgrpmorA (\monela \mtimesof\grpA \monelb) = \sgrpmorA (\monela) \mtimesof\grpB \sgrpmorA(\monelb).
    \end{equation}
\end{ctdefinition}

What could be surprising is that, while a group has more structure than a \SY{monoid}, there are fewer conditions than in the definition of \SY{monoid morphism}.

Where are the equations
\begin{equation}
    \label{eq:grp-id-comp}
    \sgrpmorA (\idmon_\grpA) = \idmon_{\grpB}
\end{equation}
and
\begin{equation}
    \label{eq:grp-inv-comp}
    \sgrpmorA (\ginv_\grpA(\monela)) = \ginv_\grpB(\sgrpmorA(\monela)) \ ?
\end{equation}
%
The answer is that they are not needed, because they can be deduced from the group axioms (and so we omit them, because they don't need to be checked when we want to know if something is a group morphism or not).

\begin{exercise}
    Prove that \cref{eq:grp-id-comp} and \cref{eq:grp-inv-comp} are implied by \cref{def:group-mor}.
\end{exercise}
\begin{solution}
    We start with the first one.
    Consider~$\monela \setin \grpA$.
    We know that
    %
    \begin{equation}
        \sgrpmorA(\idmon_\grpA \mtimesof\grpA \monela)=\sgrpmorA(\monela).
    \end{equation}
    %
    On the other hand, we know:
    %
    \begin{equation}
        \sgrpmorA(\idmon_\grpA \mtimesof\grpA \monela) = \sgrpmorA(\idmon_\grpA)\mtimesof\grpB \sgrpmorA(\monela).
    \end{equation}
    %
    These two are equivalent if and only if~$\sgrpmorA(\idmon_\grpA)=\idmon_\grpB$.

    For the second statement, consider again~$\monela \setin \grpA$.
    We now that
    \begin{equation}
        \begin{aligned}
            \sgrpmorA(\ginv_{\grpA}(\monela)\mtimesof{\grpA} \monela) & =\sgrpmorA(\idmon_\grpA) \\
                                                                      & =\idmon_\grpB,
        \end{aligned}
    \end{equation}
    and
    \begin{equation}
        \begin{aligned}
            \sgrpmorA(\ginv_{\grpA}(\monela)\mtimesof{\grpA} \monela) & =\sgrpmorA(\ginv_{\grpA}(\monela))\mtimesof{\grpB} \sgrpmorA(\monela).
        \end{aligned}
    \end{equation}
    These two are equivalent if and only if~$\sgrpmorA(\ginv_{\grpA}(\monela))=\ginv_{\grpB}(\sgrpmorA(\monela))$.

\end{solution}

\begin{exercise}
    Let~$\grpAset=\makeset{\plusone,\minusone,+i,-i}$ where $i$ is the imaginary unit.
    Consider the group~$\grpA=\tup{\grpAset, \cdot, 1, \ginv_\grpA}$ and the group~$\grpB$ of all integers under addition.
    Prove that~$\sgrpmorA\colon \grpB \fto \grpA$ such that~$f(n)=i^n$ for all~$n\setin \grpB$ is a group morphism.
\end{exercise}
\begin{solution}
    We have:
    \begin{equation}
        \begin{aligned}
            \sgrpmorA(m+n) & =i^{m+n} \\
                           & =i^m \cdot i^n \\
                           & =\sgrpmorA(m)\cdot \sgrpmorA(n).
        \end{aligned}
    \end{equation}
\end{solution}

\begin{example}
    Consider the group
    \begin{equation}
        \grpA=\tup{\reals^{\matdim{2}{2}},{+},\begin{bmatrix}
                0 & 0 \\0& 0
            \end{bmatrix},-}
    \end{equation}
    of real~$2\times 2$ matrices, together with sum of matrices as a binary operation, ``zero'' matrix as the \SY{neutral element}, and the ``-'' operation as inverse.
    Furthermore, consider the group~$\tup{\reals,+,0,-}$.
    Taking the \emph{trace} of a matrix corresponds to a group morphism.
    Indeed, the operation
    %
    \begin{equation}
        \begin{aligned}
            \mattrace\colon \reals^{\matdim{2}{2}} & \fto \reals, \\
            \begin{bmatrix}
                a & b \\
                c & d
            \end{bmatrix}                        & \mapsto a+d,
        \end{aligned}
    \end{equation}
    satisfies the required condition:
    \begin{equation}
        \begin{aligned}
            \mattrace \pars{
                \begin{bmatrix}
                    a & b \\
                    c & d
                \end{bmatrix} \mtimesof\grpA
                \begin{bmatrix}
                    e & f \\
                    g & h
                \end{bmatrix}
            } & =
            \mattrace
            \pars{
                \begin{bmatrix}
                    a+e & b+f \\
                    c+g & d+h
                \end{bmatrix}
            } \\
              & =\mattrace \pars{
                \begin{bmatrix}
                    a & b \\
                    c & d
                \end{bmatrix}}\mtimesof\grpB \mattrace \pars{
                \begin{bmatrix}
                    e & f \\
                    g & h
                \end{bmatrix}
            } \\
              & =(a+d)\mtimesof\grpB (e+h).
        \end{aligned}
    \end{equation}
\end{example}

\begin{example}[Square matrices with full rank]
    \label{exa:square-full}
    Fix an integer~$n\geq1 $ and consider the set of square matrices with full rank~$\mat{A} \setin \reals^{\ntimesn}$, which is to say $\mdet(\mat{A}) \neq 0$.
    This set, equipped with the usual matrix multiplication as the binary operation ($\mat{A}\mtimes \mat{B}\definedas \mat{AB}$), the identity matrix~$\idmat$ as the \SY{neutral element}, and matrix inverse as the inverse ($\ginv(\mat{A})\definedas \mat{A}^{-1}$), forms a \SY{group}.
    Furthermore, note that for this type of matrices, we have the properties:
    \begin{enumerate}
        \item $\mdet(\mat{A} \, \mat{B}) = \mdet(\mat{A}) \cdot \mdet(\mat{B})$;
              % \item $\mdet(\mat{A}^{-1}) = \frac{1}{\mdet(\mat{A})}$;
        \item $\mdet(\mat{A}^{-1}) = \pars{ \mdet(\mat{A}) } ^ {-1}$;
        \item $(\mat{A}\,\mat{B})^{-1} = {\mat{B}^{-1} \, \mat{A}^{-1}}$.
    \end{enumerate}
    This makes~$\mdet$ a \SY{group morphism} from the \SY{group of invertible square matrices} to the real numbers with multiplication.
\end{example}

As before for \SY{semigroups} and \SY{monoids}, the identity map on the underlying set of a group is a \SY{group morphism}.

\begin{ctdefinition}[Identity morphism]
    \label{def:identity-grp-mor}
    \SYNDEF{identity morphism of groups}
    Given a \SY{group}~$\grpA = \tup{\grpAset, \gtimes, \idgrp, \ginv}$, the \SY{identity function} $\grpAset \to \grpAset$ induces a \SY{morphism of groups}~$\funid_\monoidA\colon \monoidA \fto \monoidA$.
\end{ctdefinition}

\begin{ctdefinition}[Group isomorphism]
    \label{def:group-isomorphism}
    A \SY{morphism of monoids}~$\funa\colon \grpA \fto \grpB$ is called a \maindef{group isomorphism} if there is a \SY{morphism of groups}~$\funb\colon \grpB \fto \grpA$ such that
    \begin{equation}
        \funa \fthen \funb = \funid_\grpA \qqand \funb \fthen \funa = \funid_\grpB.
    \end{equation}
\end{ctdefinition}

\todotextjira{729}{\alphubel: @JL: include examples of group isomorphisms}

