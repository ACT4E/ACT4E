% !TEX root = chapter-standalone.tex

\section{Things that don't matter}

\begin{marginfigure}
    \subfloat[Mars climate orbiter] {
        \includegraphics[width=5cm]{orbiter}
    } \\
    \subfloat[]{
        \includegraphics[width=5cm]{orbiter2}
        % https://www.gocomics.com/tomtoles/1999/10/04
    }
    \caption{}
    \label{fig:orbiter}
\end{marginfigure}

In engineering we know that \textbf{using the right conventions is essential}.

There are many famous examples of unit mismatches causing disasters or near-disasters:
\begin{itemize}
    \item The loss of the Mars Climate Orbiter in 1999 was due to the fact that NASA used the metric system,
          while contractor Lockheed Martin used (by mistake) imperial units~(\cref{fig:orbiter}).
    \item In 1983, an Air Canada’s Boeing 767 jet ran out of fuel in mid-flight because there was a miscalculation of the fuel needed for the trip.
          In the end, the pilot managed to successfully land the ``Gimli Glider''.
    \item Going back in history, Columbus wound up in the Bahamas because he miscalculated the Earth's circumference, due to several mistakes, and one of them was assuming that his sources were using the \emph{Roman mile} rather than the \emph{Arabic mile}~\cite{morison2007admiral}.
          Columbus' mathematical mistakes led to a happy incident for him, but not so great outcomes for many others.
\end{itemize}

However, in category theory, we look at the ``essence'' of things, and we consider \textbf{what is true regardless of conventions}.

\section[Choice of symbols]{The choice of symbols does not matter}

\begin{marginfigure}
    \centering
    \includegraphics[width=4cm]{arabic}
    \caption{A page from Mu\d{h}ammad ibn Mūsā al-Khwārizmī's \emph{Algebra}.
        The word \emph{algorithm} comes from the name \emph{al-Khwārizmī}.
    }
    % \< محمد بن موسی خوارزمی>}
\end{marginfigure}

For example, it doesn't matter what symbols we use to represent numerals.
%
\begin{figure}[h!]
    \includegraphics[width=6cm]{numerals1}
\end{figure}
%
It also should not matter that we use base-10 numerals---certainly mathematical truths do not depend on how many fingers humans have.

Just like this book is written in rather plain English, and could be translated to another language while preserving the meaning, in category theory we look at what is not changed by a 1:1 translation that can be reversed.

This will be covered later in a section on ``isomorphisms''; but for now we can look at this intuitively.

\section[Typographical conventions]{Typographical conventions don't matter}

Some of you might have objected to the conventions that we used in this chapter for the notation for composition of morphisms.
We have used the notation $\morab$ (``$\mora$ then~$\morb$'') while usually in the rest of mathematics we would have used $\morb \after \mora$ (``$\morb$ after $\mora$'').
However, any concept we will use is ``invariant'' to the choice of notation.
We can decide to rewrite the book using the other convention and still all the theorems would remain true, and all the falsities will remain false.
More technically, we can take any formula written in one convention and rewrite it with the other convention, and vice versa using a specific mechanical rule.
For example, the formula
%
\begin{equation}
    (\morab)
    \mthen \morc = \mora \mthen (\morb \mthen \morc)
\end{equation}
%
would be transformed in
%
\begin{equation}
    \morc \after (\morb \after \mora) = (\morc \after \morb) \after \mora.
\end{equation}
(A bit more advanced category theory can describe this transformation more precisely.)

\begin{figure*}[h]
    \hfill
    \includesag{convention_comp}
    \caption{Mechanical rule to transform one convention to another.}
\end{figure*}

\ifextraspace{\vfill\clearpage}
This invariance to mechanical invertible transformations only holds for mathematical and technical writing.
In other contexts it might fail.

For example, ``Veni Vidi Vici'' has a sound to it that the English translation does not have~(\cref{fig:vedividivici}).

\begin{figure}[h!]
    \centering
    \includesag{vedividivici}
    \caption{}
    \label{fig:vedividivici}
\end{figure}

Sometimes, the meaning is in the typography, as in Apollinaire's poem ``Po\`eme \`a Lou'' (\cref{fig:apollinaire}).

\begin{figure*}[h!]
    \centering
    \fitinpage{
        \includesag{apollinaire}
    }
    \caption{}
    \label{fig:apollinaire}
\end{figure*}

\section{Arrow directions don't matter}
The same considerations apply for the convention regarding the arrow directions.
If we have a category with morphisms such as
\begin{equation}
    \motor \colon \rotationalmotion \mto \electricpower
\end{equation}
with the semantics of ``the \motor can produce \rotationalmotion given \electricpower'', we could define a \emph{different} category, where the conventions are inverted.
In this other category, for which we use arrows of different color, we would write
\begin{equation}
    {\color{blue}\motor}
    \colon \electricpower  \mto  \rotationalmotion
\end{equation}
and the semantics would be ``the \motor consumes \electricpower to produce \rotationalmotion'' (\cref{fig:inverted}).

\begin{figure}[h!]
    \centering
    \includesag{30_arrow_convention_power_a}\\[+15pt]
    \includesag{30_arrow_convention_power_b}
    \caption{Opposite convention for arrows direction. }
    \label{fig:inverted}
\end{figure}

These two categories would have the same objects, and the same number of arrows; it's just that the arrows change direction when moving from one category to the other.
In particular, there exists a transformation which maps every black arrow to a blue arrow, reverting the direction (\cref{fig:inverted_2}).
This transformation is invertible.
Intuitively, we would not expect anything substantial to change, because we are just changing a convention.
We will see that there is a concept called \emph{opposite category} that formalizes this idea of reversing the direction of the arrows.

\begin{figure}[h!]
    \centering
    \includesag{020_inverted_2}
    \caption{}
    \label{fig:inverted_2}
\end{figure}

\section[Diagram conventions]{Diagrams conventions don't matter}

Now that we are flexed our isomorphism muscles, we can also talk about the isomorphisms of the visual language.

In engineering, ``boxes and wires'' diagrams are commonly used to talk about materials transformations and signal flows.
In those diagrams one would use boxes to describe the processes and the wires to describe the materials or information that is being transformed.
Boxes have ``inputs'' and ``outputs'', and arrows have directions representing the causality.
From left to right, what is to the left causes what is to the right.
The left-to-right directionality seems an utterly obvious choice for most of you who learned languages that are written left-to-right, top-to-bottom as in this book\footnote{
    Note that this paragraph cannot be translated literally to Japanese.
    It breaks the assumption that we made before, about the fact that we can have a 1:1 literal translation of this book without changing the meaning.
    You might think that our future hypothetical Japanese translator can make an outstanding job and translate also our figures to go right-to-left, then saying that right-to-left is natural to people that write right-to-left.
    However, that does not work, because in fact Japanese engineers also use left-to-right diagrams.
}.

\begin{figure}[h!]
    \centering
    \begin{tabular}{c}
        \includesag{20_iso_diags_1} \\
        \includesag{20_iso_diags_2}
    \end{tabular}
    \caption{Isomorphisms of resource diagrams.  }
    \label{fig:isodiagrams}
\end{figure}

\Cref{fig:isodiagrams} shows how we would have visually described the first example using the boxes-and-wires conventions.
Again, we say that this is just a different convention, because we have a procedure to transform one diagram into the other.
This is not as simple as changing the direction of the arrows as in the case of an \SY{opposite category}.
Rather, to go from points-and-arrows to boxes-and-wires (\cref{fig:isodiagrams_2}):
\begin{itemize}
    \item Arrows that describe transmuters become boxes that describe processes;
    \item The points that describe the resources become wires between the boxes.
\end{itemize}

\begin{figure}[h!]
    \centering
    \includesag{20_iso_diags_3}
    \caption{Switching conventions in resource diagrams.}
    \label{fig:isodiagrams_2}
\end{figure}

Another example is the following.
Consider the bottom representation in \cref{fig:e4combination} and recall that it implies the existence of four composed arrows (wheels and motor pairings):  $\transmuter{wheels U}\mthen  \transmuter{motor A}$,~$\transmuter{wheels U} \mthen \transmuter{motor B}$,~$\transmuter{wheels V} \mthen \transmuter{motor A}$, and~$\transmuter{wheels V} \mthen \transmuter{motor B}$, all going from \translationalmotion to \electricpower.
The four possibilities can be represented via boxes and wires, as listed in the upper part of \cref{fig:e4combination}.

\begin{figure*}[h!]
    \centering
    \begin{tabular}{cc}
        \scalebox{0.5}{\includesag{20_combinations_1_1}} & \scalebox{0.5}{\includesag{20_combinations_1_2}} \\
        \scalebox{0.5}{\includesag{20_combinations_2_1}} & \scalebox{0.5}{\includesag{20_combinations_2_2}} \\
    \end{tabular}\\[+15pt]
    \includesag{20_combinations_2}
    \caption{Multiple models for wheels and motors.}
    \label{fig:e4combination}
\end{figure*}

