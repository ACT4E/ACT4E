% !TEX root = chapter-standalone.tex

\section[Preorders as categories]{Preorders as categories}
\label{sec:posets-as-cats}

\begin{ctdefinition}[Categorification of a preorder]\label{def:poscat}
    \SYNDEF{categorification of a preorder}
    Given a preorder~$\posA = \tup{\posAset, \posleq_\posA}$, its \emph{categorification}~$\poscat{\posA}$ is a category with
    \begin{enumerate}
        \item \emph{Objects}: the elements of~$\posAset$;
        \item \emph{Morphisms}: given~$\Obja,\Objb\setin \posAset$, we define the homset $\HomSet{\poscat{\posA}}{\Obja}{\Objb}$ to be
        \begin{equation}
\HomSet{\poscat{\posA}}{\Obja}{\Objb} = 
\begin{cases}
\singleton & \text{ if }  \ \Obja \posleq_\posA \Objb \\
\Emptyset & \text{ else. }
\end{cases}
\end{equation}
        \item \emph{Composition}: should be given by functions of the type
        \begin{equation}
\HomSet{\poscat{\posA}}{\Obja}{\Objb} \cartprod \HomSet{\poscat{\posA}}{\Objb}{\Objc} \mto \HomSet{\poscat{\posA}}{\Obja}{\Objc}.
\end{equation}
When either of the factors in the source set are the empty set, then there is a unique function of the desired type. When both factors are equal to the set $\singleton$, then thanks to the \emph{\SY{transitivity}} of the preorder $\posleq_\posA$ the target set $\HomSet{\poscat{\posA}}{\Obja}{\Objc}$ must also be the set $\singleton$, and there is a unique function of the type $\singleton \cartprod \singleton \mto \singleton$.
        \item \emph{Identities}: given any~$\Obja \setin \Obof{\poscat{\posA}}$, we always have $\Obja \posleq_\posA \Obja$, by the \emph{\SY{reflexivity}} of the preorder $\posleq_\posA$. Hence $\HomSet{\poscat{\posA}}{\Obja}{\Obja} = \singleton$ always. We define the single element of $\HomSet{\poscat{\posA}}{\Obja}{\Obja} = \singleton$ to be the \SY{identity morphism} of $\Obja$. 
    \end{enumerate}
\end{ctdefinition}


\begin{remark}
    A \SYN{thin category}{\emph{thin} category} is one in which there is at most one morphism in any homset. Categorifications of preorders are examples of thin categories. Conversely, every thin category can be interpreted as defining a preorder. 
\end{remark}

\begin{remark}
    If we consider a preorder which is actually a poset, then its categorification is an example of a \SYN{skeletal category}{\emph{skeletal} category}. These are categories where, if any two objects are isomorphic, then they are necessarily equal.
\end{remark}


\begin{marginfigure}
    \centering
    \includesag{40_dpcatfig_power}
    \caption{Power set~$\powerset{\makeset{\setAel,\setBel,\setCel}}$ as a poset.
        \label{fig:posetascat}}
\end{marginfigure}

\begin{example}
    We revisit \cref{def:power-poset}, in which we had a poset~\posA on $\powerset \makeset{\setAel,\setBel,\setCel}$ with order given by inclusion (\cref{fig:posetascat}).
    Its categorification~$\poscat{\posA}$ is a category, with $\Obof{\poscat{\posA}}=\powerset\pars{\makeset{\setAel,\setBel,\setCel}}$, and morphisms given by the inclusions.
    Note that we omit self-arrows for the \SY{identity morphisms}, taking these to be tacitly implied.
    Composition is given by the \SY{transitivity} law of \SY{posets}.
    For instance, since~$\makeset{\setAel}\setsubseteq \makeset{\setAel,\setBel}$ and~$\makeset{\setAel,\setBel} \setsubseteq \makeset{\setAel,\setBel,\setCel}$, we can say that~$\makeset{\setAel}\setsubseteq \makeset{\setAel,\setBel,\setCel}$.
\end{example}