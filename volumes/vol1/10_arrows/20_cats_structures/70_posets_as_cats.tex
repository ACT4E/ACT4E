% !TEX root = chapter-standalone.tex

\section[Posets as categories]{Posets as categories}
\label{sec:posets-as-cats}

\begin{ctdefinition}[Categorification of a poset]\label{def:poscat}
    \SYNDEF{categorification of a poset}
    Given a poset~\posA, its \emph{categorification}~$\poscat{\posA}$ is a category with
    \begin{enumerate}
        \item \emph{Objects}: Objects of~$\poscat{\posA}$ are elements of~$\posAset$;
        \item \emph{Morphisms}: Given~$\Obja,\Objb\setin \Obof{\poscat{\posA}}$, there exists a morphism~$\mora\colon \Obja\mto \Objb$ if and only if~$\Obja \posAleq \Objb$.
        \item \emph{Composition}: Composition is given by \emph{\SY{transitivity}}.
        \item \emph{Identities}: Given any~$\Obja \setin \Obof{\poscat{\posA}}$, the \SY{identity morphism} is given by \emph{\SY{reflexivity}} in \SY{posets}.
    \end{enumerate}
\end{ctdefinition}

\todotext{Improve the above definition by specifying the homsets explicitely (or add a remark to this effect)}

\begin{remark}
A \SYN{thin category}{\emph{thin} category} is one in which there is at most one morphism between any two objects. Categorifications of posets are examples of thin categories. Conversely, we can interpret any thin category as defining a poset. 
\end{remark}

\begin{remark}
Categorifications of posets are also examples of \SYN{skeletal category}{\emph{skeletal} categories}. These are categories where, if any two objects are isomorphic, then they are necessarily equal. Not every skeletal category can be interpreted as defining a poset. Can you see why? 
\end{remark}

\todotext{J: @J: make definition or comment about preorders also defining categories}

\begin{marginfigure}
    \centering
    \includesag{40_dpcatfig_power}
    \caption{Power set~$\powerset{\makeset{\setAel,\setBel,\setCel}}$ as a poset. \label{fig:posetascat}}
\end{marginfigure}

\begin{example}
    We revisit \cref{def:power-poset}, in which we had a poset~\posA on $\powerset \makeset{\setAel,\setBel,\setCel}$ with order given by inclusion (\cref{fig:posetascat}).
    Its categorification~$\poscat{\posA}$ is a category, with $\Obof{\poscat{\posA}}=\powerset\pars{\makeset{\setAel,\setBel,\setCel}}$, and morphisms given by the inclusions.
    Note that we omit to draw self-arrows for the \SY{identity morphisms}.
    Composition is given by the \SY{transitivity} law of \SY{posets}.
    For instance, since~$\makeset{\setAel}\setsubseteq \makeset{\setAel,\setBel}$ and~$\makeset{\setAel,\setBel} \setsubseteq \makeset{\setAel,\setBel,\setCel}$, we can say that~$\makeset{\setAel}\setsubseteq \makeset{\setAel,\setBel,\setCel}$.
\end{example}