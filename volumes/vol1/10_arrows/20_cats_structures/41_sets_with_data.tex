% !TEX root = chapter-standalone.tex

\section[Sets with data]{Sets with data}
\label{sec:sets-with-data}

There are various simple constructions where we can build categories whose objects are not just sets, but sets together with some extra data. Morphisms are then functions which are compatible with the extra data. Below we give a few examples and we encourage the reader to imagine further variations. 

\begin{ctdefinition}[Pointed sets]
    \label{def:pSet}
    The category $\pSet$ of pointed sets is:
    \begin{enumerate}
        \item \emph{Objects}: pairs $\tup{\setA, \ela}$ where $\setA$ is a set and $\ela \setin \setA$ is an element of $\setA$.
        \item \emph{Morphisms}: a morphism $\mora \colon \tup{\setA, \ela} \mto_{\pSet} \tup{\setB, \elb}$ is a function $\mora \colon \setA \mto_{\Set} \setB$ such that $\mora(\ela) = \elb$. 
        \item \emph{Composition}: the usual composition of functions.
        \item \emph{Identity morphisms}: identity functions.
    \end{enumerate}
\end{ctdefinition}

\begin{exercise}
Prove that \cref{def:pSet} really is a category. 
\end{exercise}


\begin{ctdefinition}[Endofunctions]
    \label{def:cat-endofunctions}
    The category $\EndSet$ of endofunctions is:
    \begin{enumerate}
        \item \emph{Objects}: pairs $\tup{\setA, \stylemorph{\varphi}}$ where $\setA$ is a set and $\stylemorph{\varphi} \colon \setA \to \setA$ is a function.
        \item \emph{Morphisms}: a morphism $\mora \colon \tup{\setA, \stylemorph{\varphi}} \mto_\EndSet \tup{\setB, \stylemorph{\psi}}$ is a function $\mora \colon \setA \to_\Set \setB$ with the property that $\mora \mthen \stylemorph{\psi}= \stylemorph{\varphi} \mthen \mora$. 
        \item \emph{Composition}: the usual composition of functions.
        \item \emph{Identity morphisms}: identity functions.
    \end{enumerate}
\end{ctdefinition}

\begin{exercise}
Prove that \cref{def:cat-endofunctions} is indeed a category. 
\end{exercise}

\begin{ctdefinition}[Equivalence relations]
    \label{def:cat-equivalence-relations}
    The category $\EquivRel$ of equivalence relations is:
    \begin{enumerate}
        \item \emph{Objects}: pairs $\tup{\setA, \sim_{\setA}}$ where $\setA$ is a set and $\sim_{\setA} \colon \setA \mto_{\Rel} \setA$ is an equivalence relation.
        \item \emph{Morphisms}: a morphism $\mora \colon \tup{\setA, \sim_{\setA}} \mto_{\EquivRel} \tup{\setB, \sim_{\setB}}$ is a function $\mora \colon \setA \mto_{\Set} \setB$ such that
        \begin{equation}\label{eq:morphism-of-equiv-rel}
\ela \sim_\setA \elb \ \Rightarrow \ \mora(\ela) \sim_\setB \mora(\elb). 
\end{equation}

        \item \emph{Composition}: the usual composition of functions.
        \item \emph{Identity morphisms}: identity functions.
    \end{enumerate}
\end{ctdefinition}

\begin{exercise}
We can visualize an equivalence relation on a set $\setA$ as a partition of $\setA$. Can you visualize the condition \cref{eq:morphism-of-equiv-rel} in terms of sets and partitions of them? 
\end{exercise}

\begin{remark}
The above example with equivalence relations is very similar to the category of posets and monotone maps; we are simply considering equivalence relations instead of relations which are partial orders. The category of posets, and as well as most of our other examples of categories of algebraic structures (semigroups, monoids, groups, etc.), can all be thought of as categories built from ``sets with extra data''. 
\end{remark}



