% !TEX root = chapter-standalone.tex

\section{Categories of sets and functions}

We have already introduced in \cref{exa:the-category-of-sets} the category of all sets and functions (it is a very large category).

\begin{ctdefinition}[Category of sets]
    \index{\textbf{Set}|textbf}
    \label{def:Set}
    The category \Set of sets is:
    \begin{enumerate}
        \item \emph{Objects}: all sets.
        \item \emph{Morphisms}: given sets~$\Obja$ and~$\Objb$, the hom-set~$\HomSet{\Set}{\Obja}{\Objb}$ is the set of all functions from~$\Obja$ to~$\Objb$.
        \item \emph{Composition operation}: the usual composition of functions.
        \item \emph{Identity morphisms}: given a set~$\Obja$, its identity morphism~$\catidat\Obja$ is the identity function~$\Obja \mto \Obja, \ \catidat\Obja(\ela) = \ela$.
    \end{enumerate}
\end{ctdefinition}

A close relative of this category is the category~$\FinSet$, where we only consider \emph{finite} sets as objects, but otherwise everything is the same as in the category~\Set.

Other categories of sets and functions can be obtained by restricting what type of functions we consider.
For example, there is a category~$\Injset$ where the objects are all sets and where morphisms are injective functions.
Similarly, there is a category~$\Surj$ of surjective functions.

\begin{exercise}
    Spell out a definition of the category~$\Injset$ of injective functions, and check that it is indeed a category.
    In particular:
    \begin{enumerate}
        \item Specify what the composition operations are and check if the composition of two composable injective functions is again injective;
        \item Specify what the identity morphisms are and check that they are indeed injective functions;
        \item Argue why the associativity condition is satisfied.
    \end{enumerate}
\end{exercise}

\begin{solution}
    We define the category~$\Injset$ to be such that its objects are all sets, its morphisms are injective functions, composition is the usual composition of functions, and identity morphisms are the usual identity functions.
    \begin{enumerate}
        \item We show that the composition of two injective functions is injective.
              Given functions~$\mora \colon \Obja \mto \Objb$ and~$\morb \colon \Objb \mto \Objc$ injective, suppose that~$\morb(\mora (\elna{1})) = \morb(\mora (\elna{2}))$ for some~$\elna{1}, \elna{2} \setin \Obja$.
              By the injectivity of~$\morb$ it follows that~$\mora (\elna{1}) = \mora (\elna{2})$, and then using the injectivity of~$\mora$, we can conclude that~$\elna{1} = \elna{2}$.
        \item Identity functions are clearly injective.
        \item Associativity of composition holds because it holds for all functions, so in particular also for injective ones.
    \end{enumerate}
\end{solution}

