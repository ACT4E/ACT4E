% !TEX root = chapter-standalone.tex

\section{The category \Set}

A helpful concept here is to think of our specific sets and functions as living in a very (very) large category which contains all possible sets as its objects and all possible functions as its morphisms.
This category is known as the category of sets, and it is an important protagonist in category theory.
We will denote it by~\Set.
It is a short exercise to check that the following does indeed define a category.

\begin{ctdefinition}[Category of sets]
    \label{def:Set}
    The category of sets \iindex{\Set} is defined by:
    \begin{enumerate}
        \item \emph{Objects}: all sets.
        \item \emph{Morphisms}: given sets~$\Obja$ and~$\Objb$, the homset~$\HomSet{\Set}{\Obja}{\Objb}$ is the set of all functions from~$\Obja$ to~$\Objb$.
        \item \emph{Identity morphism}: given a set~$\Obja$, its identity morphism~$\catid_\Obja$ is the identity function~$\Obja \to \Obja, \ \catid_\Obja(\ela) = \ela$.
        \item \emph{Composition operation}: the composition operation is the usual composition of functions.
    \end{enumerate}
\end{ctdefinition}
