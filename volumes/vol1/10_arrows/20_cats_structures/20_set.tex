% !TEX root = chapter-standalone.tex

\section{Categories of sets and functions}

We have already introduced in \cref{exa:cat-of-sets} the category of all sets and functions (it is a very large category). 

\begin{ctdefinition}[Category of sets]
    \label{def:Set}
    The category of sets, \iindex{\Set}, is defined by:
    \begin{enumerate}
        \item \emph{Objects}: all sets.
        \item \emph{Morphisms}: given sets~$\Obja$ and~$\Objb$, the homset~$\HomSet{\Set}{\Obja}{\Objb}$ is the set of all functions from~$\Obja$ to~$\Objb$.
        \item \emph{Identity morphism}: given a set~$\Obja$, its identity morphism~$\catid_\Obja$ is the identity function~$\Obja \mto \Obja, \ \catid_\Obja(\ela) = \ela$.
        \item \emph{Composition operation}: the usual composition of functions.
    \end{enumerate}
\end{ctdefinition}

A close relative of this category is the category $\FinSet$, where as objects we only consider \emph{finite} sets, but otherwise everything is the same as in the category $\Set$. 

Other categories of sets and functions can be obtained by restricting what type of functions we consider. For example, there is a category $\Cat{Inj}$ where the objects are all sets and where morphisms are injective functions. Similarly, there is a category $\Cat{Surj}$ of surjective functions. 

\begin{exercise}
Spell out a definition of the category $\Cat{Inj}$ of injective functions, and check that it is indeed a category. In particular:
\begin{enumerate}
\item specify what the composition operations are and check if the composition of two composable injective functions is again injective;
\item specify what the identity morphisms are and check that they are indeed injective functions;
\item argue why the associativity condition is satisfied. 
\end{enumerate}
\end{exercise}

\todotext{Write the solution of the above exercise}