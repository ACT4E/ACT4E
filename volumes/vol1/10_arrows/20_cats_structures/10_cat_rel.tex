% !TEX root = chapter-standalone.tex

\section[The category \Rel]{The category \Rel}
\label{sec:cat-of-relations}

\todojira{616}{\bernina: @JL: include / move more material on the category of relations here}

\linkvideo{spring2021-relations:relations:cat-rel} % Category of relations

Recall that a (binary) relation between a set $\setA$ and a $\setB$ is a subset~$\relA \setsubseteq \setA \cartprod \setB$.
We have already seen that relations can be composed, so it is natural now to think of a relation~$\relA \setsubseteq \setA \cartprod \setB$ as a \emph{morphism} from~$\setA$ to~$\setB$.

\begin{ctdefinition}[Category \Rel]
    \index{\Rel}
    \label{def:Rel}
    The category of relations \Rel is given by:
    \begin{enumerate}
        \item \emph{Objects}: The objects of this category are all sets.
        \item \emph{Morphisms}: Given sets~$\Obja, \Objb$, the hom-set~$\HomSet{\Rel}{\Obja}{\Objb}$ consists of all relations~$\relA\setsubseteq \Obja\cartprod \Objb$.
        \item \emph{Composition}: Given relations~$\relA \colon \Obja\mto \Objb$,~$\relB \colon \Objb\mto \Objc$, their composition is given by
              \begin{equation}
                  \label{eq:RelCompRule}
                  \relA \mthen \relB \definedas \makeset{\tup{\ela,\elc} \setin \Obja \cartprod \Objc \mid  \exists \elb \setin \Objb \colon \left(\inrel{\ela}{\relA}{\elb} \right) \booland \left(\inrel{\elb}{\relB}{\elc}\right)}.
              \end{equation}
        \item \emph{Identity morphisms}: Given a set~$\Obja$, its identity morphism is
              \begin{equation}
                  \catidat\Obja
                  \definedas
                  \makeset{ \tupp{\ela,\elb} \setin \Obja \cartprod \Obja \mid  \ela = \elb }.
              \end{equation}
    \end{enumerate}
\end{ctdefinition}

\devel{

    \begin{remark}
        Relations with the same source and target can be \emph{compared} via inclusion.
        Given~$\relA, \relB \colon \setA\mto \setB$  we can ask whether~$\relA\setsubseteq \relB$ or~$\relB\setsubseteq \relA$ (or neither).
    \end{remark}

    \todotext{\bernina: @AC: This probably goes to locally posetal structure}
}
