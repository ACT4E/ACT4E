% !TEX root = chapter-standalone.tex

\section{Biproducts}

\publictodomessage

\todotextjira{93}{\bernina: @JL: Write intro text.
    Fill in with example of direct sum of \SY{vector spaces}, and other examples\dots }

\begin{example}
    We can define a category $\VectR$, composed of: % TODO: readd index
    \begin{itemize}
        \item \emph{Objects}: real \SY{vector spaces};
        \item \emph{Morphisms}: real linear maps;
        \item \emph{Identity morphisms}: identity maps;
        \item \emph{Composition}: composition of real linear maps.
    \end{itemize}
    It is a good exercise to prove that $\VectR$ really forms a category.
    In the following, we want to look at the coproduct in $\VectR$.
    It is given by the \emph{direct sum} of \SY{vector spaces}.
    Recall that given \SY{vector spaces}~$V$ and~$W$, their direct sum is the set
    \begin{equation}
        V\oplus W\definedas \makeset{\tup{v,w}\mid v\setin V,w\setin W},
    \end{equation}
    equipped with a notion of addition and scalar multiplication derived component-wise from~$V$ and~$W$.
    For addition, this means that given~$\tup{v_1,w_1}, \tup{v_2,w_2}\setin V\oplus W$, their sum in~$V \oplus W$ is
    \begin{equation}
        \tup{v_1,w_1}+\tup{v_2,w_2}\definedas \tup{v_1+v_2,w_1+w_2}.
    \end{equation}
    The injection morphisms for the \SY{coproduct} are given by:
    \begin{equation}
        \begin{aligned}
            \inj_V \colon V & \sto V\oplus W \\
            v               & \mapsto \tup{v,0_W}, \\
            \inj_W \colon W & \sto V\oplus W \\
            w               & \mapsto \tup{0_V,w},
        \end{aligned}
    \end{equation}
    where~$0_V$ and~$0_W$ represent the zero vectors in~$V$ and~$W$.
    We now look at the universal property in this case, by considering any \SY{vector space} $U\setin \Obof{\Vect{}}$, and linear maps~$S\colon V\to U$,~$T\colon W\to U$.
    The universal property says that we need a unique linear map~$S+T\colon V\oplus W \to U$ such that~$S=\inj_V\mthen h$ and~$T=\inj_W\mthen g$.
    By taking any~$\tup{v,w}\setin V\oplus W$, we can write:
    \begin{equation}
        \begin{aligned}
            h(\tup{v,w}) & =h(\tup{v,0_W}+\tup{0_V,w}) \\
                         & =h(\inj_V(v)+\inj_W(w)) \\
                         & =h(\inj_V(v))+h(\inj_W(v)) \qquad \qquad (h \text{ is linear}) \\
                         & =(\inj_V\mthen h)(v)+(\inj_W\mthen h)(w) \\
                         & \overset{!
            }{=}
            Sv+Tw.
        \end{aligned}
    \end{equation}
    We can hence write the map $S+T$ as
    \begin{equation}
        \begin{aligned}
            S+T\colon V\oplus W & \to U \\
            \tup{v,w}           & \mapsto Sv+Tw.
        \end{aligned}
    \end{equation}
\end{example}
