% !TEX root = chapter-standalone.tex

\section{Coproducts}
\label{sec:coproductset}
\linkvideo{spring2021-coproducts:coproducts:coprod-intro-ex} % Introductory examples of coproduct

%\linkvideo{spring2021-coproducts:coproducts} % Coproducts

There exists a ``dual'' notion to ``product'' that is called ``coproduct''.
Just like the notion of \SY{categorical product} generalized the definition of the \SY{cartesian product} of two sets, the \SY{categorical coproduct} generalizes the definition of the \emph{disjoint union} of two sets.
%We'll start by illustrating the idea behind coproducts using an example. Suppose that we are considering two vending machines: from the first one can buy food and from the second one a drink. We can think of this situation in terms of resources, by saying that having \transmuted{money} is enough to get \transmuted{food} via the first machine, or to get a \transmuted{drink} via the second machine (\cref{fig:vending_1}).
%
%\begin{figure}[h!]
%    \centering
%    \includesag{60_vending_1}
%    \caption{Alternative ways to use \transmuted{money}. \label{fig:vending_1}}
%\end{figure}
%
%From this we would like to conclude that we can use \transmuted{money} to buy \textbf{either} \transmuted{food} \textbf{or} a \transmuted{drink} (\cref{fig:vending_2}).
%
%\begin{figure}[h!]
%    \centering
%    \includesag{60_vending_2}
%    \caption{We can use \transmuted{money} to buy either \transmuted{food} or a \transmuted{drink}.\label{fig:vending_2}}
%\end{figure}
\showslides{
    \begin{forslides}

        % \begin{definition}[Union of sets]
        %     \label{def:union_sets}
        %     The \emph{union} of sets~\setA and~\setB is
        %     \begin{equation}
        %         \setA \setunion \setB=\makeset{\ela \colon \ela \setin \setA \text{ or }\ela \setin \setB}.
        %     \end{equation}
        % \end{definition}
        \begin{equation}
            \label{eq:divides_1}
            \setA,\setB\setin \natnumbers
        \end{equation}
        \begin{equation}
            \label{eq:divides_2}
            \setA
        \end{equation}
        \begin{equation}
            \label{eq:divides_3}
            \setB
        \end{equation}
        \begin{equation}
            \label{eq:divides_4}
            6\to 12
        \end{equation}
        \begin{equation}
            \label{eq:coprod_1}
            \coprodMapmor{f}{g}\colon A+B\to P
        \end{equation}
        \begin{equation}
            \label{eq:coprod_2}
            \begin{aligned}
                \coprodMapmor{f}{g} \colon  \setA \setdisunion \setB & \sto \setC \\
                \stylesets{x}                                        & \mapsto
                \begin{cases}
                    \mapa(\stylesets{x}), & \text{if } \stylesets{x}=\inj_\setA(\setAel),\quad \setAel \setin \setA, \\
                    \mapb(\stylesets{x}), & \text{if } \stylesets{x}=\inj_\setB(\setBel),\quad \setBel \setin \setB.
                \end{cases}
            \end{aligned}
        \end{equation}
        \begin{equation}
            \label{eq:coprod_3}
            \begin{aligned}
                \inj_\setA \colon \setA & \sto \setA \setdisunion\setB \\
                \inj_\setB \colon \setB & \sto \setA \setdisunion \setB
            \end{aligned}
        \end{equation}
    \end{forslides}
}

% \begin{comment}
% Suppose that we are considering a hybrid car that contains two engines: an electric engine and an internal combustion engine.
% Both can produce \transmuted{motion}, but each from a different source of energy.
% The electric engine uses \transmuted{electric energy}; the internal combustion engine uses \transmuted{gasoline}.
% The situation is as in \cref{fig:e14}.
%
% \begin{figure}[h!]
%     \centering
%     \includesag{30_dpcatfig_e14}
%     \caption{Alternative ways to generate $\mathsf{motion}$.}
%     \label{fig:e14}
% \end{figure}
%
% From this we would like to conclude that we can obtain \textsf{motion} from \textbf{either} \textsf{gasoline} \textbf{or} \textsf{electric energy} (\cref{fig:e16b}).
%
% \begin{figure}[h!]
%     \centering
%     \includesag{30_dpcatfig_e15}
%     \caption{We can generate $\mathsf{motion}$ from either $\mathsf{gasoline}$ or $\mathsf{electric} \ \mathsf{energy}$.}
%     \label{fig:e15}
% \end{figure}
%
% To define the idea of ``\textbf{either} \transmuted{food} \textbf{or} \transmuted{drink}'' we can refer to the idea of \SY{disjoint union} of sets (\cref{def:disjoint-union}).
% \end{comment}

Given sets~\setA and~\setB, their disjoint union~$\setA \setdisunion \setB$ is a set that contains a distinct copy of~\setA and~\setB each.
If an element is contained in both~\setA and~\setB, then there will be two distinct copies of it in the disjoint union~$\setA \setdisunion \setB$ (\cref{sec:disjoint-union}).

%\begin{definition}[Disjoint union of sets]
%    \label{def:disjoint-union}
%    The \emph{disjoint union} (or sum) of sets~$\setA$ and~$\setB$ is
%    \begin{equation}
%        \setA \setdisunion \setB=\makeset{\disunionA\ela \mid \ela\setin \setA}\setunion\makeset{\disunionB\elb \mid \elb\setin \setB}.
%    \end{equation}
%\end{definition}

In the case of the \SY{cartesian product} of two sets we had projection maps, as in \cref{sec:combination-products}.
For the \SY{disjoint union} of sets, we have instead \emph{inclusion maps}.
Thus, we have a diagram of this form:

\begin{figure}[h!]
    \centering
    \includesag{050_coprod_disunion_diagram}
    \caption{}
    \label{fig:coprod_disunion_diagram}
\end{figure}

\begin{marginfigure}
    \centering
    \includesag{050_example_coprod_max_cont}
    \caption{Taking the minimum}
    \label{fig:exa_coprod_max_cont}
\end{marginfigure}

\begin{example}
    \label{exa:max-as-coprod}
    This example is ``dual'' to \cref{exa:min-as-prod-cont}.
    The category in question in the same one: objects are elements of~\reals and morphisms are inequalities.
    The coproduct is ``taking the maximum''; its universal property is illustrated in \cref{fig:exa_prod_min_cont}.
    It says that if~$t \setin \reals$ is such that~$t \geq \elna{1}$ and $t \geq \elna{2}$, then~$t \geq \max \makeset{ \elna{1}, \elna{2} }$.

\end{example}

\begin{marginfigure}
    \centering
    \includesag{050_example_coprod_lcm_cont}
    \caption{Taking the least common multiple}
    \label{fig:exa_coprod_lcm_cont}
\end{marginfigure}
\begin{example}
    \label{exa:lcm-as-coprod-cont}
    This example is ``dual'' to \cref{exa:gcd-as-prod-cont}.
    The category we're working in has natural numbers as its objects, and morphisms are given by the relation ``divides''.
    The coproduct is ``taking the least common multiple''; its universal property is visualized in \cref{fig:exa_prod_gcd}.
    % For a concrete example, let $m = 12$ and $n = 18$, so $\gcd \makeset{12, 18 } = 6$. If we take $t = 3$, which divides both $12$ and $18$, we see that, indeed, $3$ also divides $6 = \gcd \makeset{12, 18 }$. And if we take $t = 2$, which \emph{also} divides both $12$ and $18$, we see that it is \emph{also} true that $2$ also divides $6 = \gcd \makeset{12, 18 }$.
\end{example}

\begin{marginfigure}
    \centering
    \includesag{050_example_coprod_union_cont}
    \caption{Taking the union}
    \label{fig:exa_coprod_union_cont}
\end{marginfigure}

\begin{example}
    \label{exa:union-as-coprod-cont}
    This example is ``dual'' to \cref{exa:intersection-as-prod-cont}.
    Given a set~\setA and arbitrary subsets~$\subAn{1}, \subAn{2} \setsubseteq \setA$, we drew an arrow~$\subAn{1} \to \subAn{2}$ iff~$\subAn{1} \setsubseteq \subAn{2}$.
    The category in question here has, as its objects, the subsets of \setA, and its morphisms are inclusions between them.
    The coproduct is ``taking the union''; its universal property is visualized in \cref{fig:exa_prod_intersection}.
    As a concrete example, consider again~$\setA = \makeset{ \sbretzel, \sfondue, \schoco, \sburger }$,~$\subAn{1} = \makeset{ \sbretzel, \sfondue, \schoco }$, and~$\subAn{2} = \makeset{\sfondue, \schoco, \sburger }$.
    So~$\subAn{1} \setunion \subAn{2} = \setA$.
    If we choose~$\setC = \makeset{ \sfondue }$, we see that~$\setC \setsubseteq \subAn{1}$ and~$\setC \setsubseteq \subAn{2}$, and that also~$\setC \setsubseteq \subAn{1} \setunion \subAn{2}$ (as it must, according to the universal property).
    The situation is similar if we choose~$\setC = \makeset{ \sbretzel }$ or~$T = \Emptyset$.
\end{example}

\todomistake{J: The above example has mistakes}

\begin{marginfigure}
    \centering
    \includesag{050_example_coprod_disjunction_cont}
    \caption{Taking the disjunction.}
    \label{fig:exa_coprod_disjunction_cont}
\end{marginfigure}

\begin{marginfigure}
    \centering
    \includesag{050_example_coprod_disjunction_bool_cont}
    \caption{Taking the disjunction.}
    \label{fig:exa_coprod_disjunction_bool_cont}
\end{marginfigure}

\begin{example}
    \label{exa:disjunction-as-coprod-cont}
    This example is ``dual'' to \cref{exa:intersection-as-prod-cont}.
    Again consider the set~$\setA = \makeset{ \true, \false }$ of logical symbols and for any~$\ela, \elb  \setin \setA$, we drew an arrow~$\ela \to \elb$ iff
    \begin{equation}
        \prfperiod{\ela}{\elb}
    \end{equation}
    The category we are working with has~\setA as its set of objects, and its morphisms are logical implications.
    The coproduct is ``taking the disjunction'' (the logical operation ``or''); the universal property is shown in \cref{fig:exa_coprod_disjunction_cont}.
\end{example}

\devel{
    \begin{marginfigure}
        \centering
        \includesag{050_example_coprod_join_cont}
        \caption{Taking the join}
        \label{fig:exa_coprod_join_cont}
    \end{marginfigure}
    \begin{example}
        \label{exa:join-as-coprod-cont}
        \todotextjira{401}{\alphubel: @JL: Finish coproduct part}
        This example is ``dual'' to \cref{exa:meet-as-prod-cont}.
        We considered, and we drew an arrow~$\ela \to \elb$ iff~$\ela \leq \elb$.
        The category at play here is the one corresponding to the \SY{poset} underlying \XXX .
        The \SY{categorical product} of two elements is their \SY{join} (least upper bound); the universal property is illustrated in \cref{fig:exa_prod_meet}.
        \todotextjira{401}{\alphubel: @JL: finish \cref{exa:join-as-coprod-cont}}
    \end{example}
}

As you can see from the above list of examples, the notion of coproduct involves diagrams of the following type:
%
\equationsag{050_coprod_generic}{fig:coprod_generic}

\section{Categorical coproduct}
\linkvideo{spring2021-coproducts:coproducts:cat-prod} % Categorical coproduct

As mentioned above, the \SY{disjoint union} is a particular instance -- in the category \Set  -- of the notion of ``coproduct''.
We will now give the definition of a coproduct in an arbitrary category.
Note that it is very similar to the definition that we gave, in the previous section, for the product -- but with a few twists.
Analogous remarks to those we gave following the definition of the product apply here.

\begin{ctdefinition}[Categorical Coproduct]
    \label{def:categorical-coproduct}
    Let~\CatC be a category and let~$\Obja, \Objb \setin \ObC$ be objects.
    The \maindef{categorical coproduct} of~$\Obja$ and~$\Objb$ is: \\
    \constit
    \begin{enumerate}
        \item an object~$\styleobj{Z} \setin \ObC$ (``the coproduct of $\Obja$ and $\Objb$'')
        \item \emph{injection morphisms}~$\injA \colon \Obja \mto \styleobj{Z} $ and~$\injB \colon \Objb \mto \styleobj{Z} $
    \end{enumerate}
    \condit
    \begin{enumerate}
        \item For any~$\styleobj{T} \setin \ObC$ and any morphisms~$\mora \colon  \Obja \mto \styleobj{T}, \morb \colon \Objb \mto \styleobj{T}$, there exists a \emph{unique} morphism~$\catcoprodpsi_{\mora, \morb} \colon \styleobj{Z} \mto \styleobj{T}$ such that~$\mora = \injA\mthen \catcoprodpsi_{\mora, \morb}$ and~$\morb = \injB \mthen \catcoprodpsi_{\mora,\morb}$.
    \end{enumerate}
\end{ctdefinition}

\begin{remark}
    Diagrammatically, the condition above states that diagrams as in \cref{fig:coprod_general_1} commute.
    \begin{marginfigure}
        \centering
        \includesag{60_defcoproduct}
        \caption{}
        \label{fig:coprod_general_1}
        \label{fig:def-coproduct-diagram}
    \end{marginfigure}
    Similarly, as was the case with the \SY{categorical product}, ``the coproduct'' of~$\Obja$ and~$\Objb$ is unique only ``up to isomorphism''.
    Nevertheless, we will usually simply write~$\coprodMapob{\Obja}{\Objb}$ for ``the'' coproduct (in place of~$\Objc$ above), and we will usually write~$\coprodMapmor{\mora}{\morb}$ in place of~$\catcoprodpsi_{\mora, \morb}$.
    The diagram in \cref{fig:coprod_general_1} then looks as in \cref{fig:def-coproduct-diagram-generic}.
\end{remark}

\begin{marginfigure}
    \centering
    \includesag{60_defcoproduct_generic}
    \caption{}
    \label{fig:def-coproduct-diagram-generic}
\end{marginfigure}
%
%Note that~$\coprodMap{X}{Y}$ is different from~$\coprodMap{Y}{X}$, but the two are isomorphic~(\cref{fig:e16}).
%
%\begin{figure}[h!]
%    \centering
%    \includesag{30_dpcatfig_e16}
%    \caption{$\coprodMap{X}{Y}$ and~$\coprodMap{Y}{X}$ are isomorphic. \label{fig:e16}}
%\end{figure}

%For the case of vending machines, the inclusion maps are as in \cref{fig:inclusionvending}.
%
%\begin{figure}[h!]
%    \centering
%    \includesag{60_inclusion_vending}
%    \caption{Inclusion maps for the vending machine example. \label{fig:inclusionvending}}
%\end{figure}

\linkvideo{spring2021-coproducts:coproducts:batt-coproduct} % Extended coproduct example

% \begin{figure}[h!]
%   \begin{center}
%   %\includesag{60_coprod_batt_1}
%   \includesag{60_coprod_batt_bis}
%   \end{center}
%   \caption{Battery technologies, companies, prices, and a catalogue.}
%   \label{fig:coprod_batteries_1}
% \end{figure}

\begin{example}
    Consider two battery producers, each producing specific battery technologies.
    The first company produces a set
    \begin{equation}
        \label{eq:exa-batt-setA}
        \setA=\makeset{\technology{LiPo}, \technology{LCO},\technology{NiH2}}
    \end{equation}
    of technologies, and the second one a set
    \begin{equation}
        \label{eq:exa-batt-setB}
        \setB=\makeset{\technology{LFP},\technology{LMO},\technology{LiPo}}.
    \end{equation}
    Each technology has, for a specific desired battery mass, a specific price, belonging to a set of prices
    \begin{equation}
        \label{eq:exa-batt-prices}
        \setC=\makeset{\unit[50]{} \text{\standardcurrency},\unit[60]{} \text{\standardcurrency},\unit[70]{} \text{\standardcurrency},\unit[80]{} \text{\standardcurrency}}.
    \end{equation}
    We specify the price mappings for different technologies by specifying the functions~$\mapa \colon \setA\to \setC$ and~$\mapb\colon \setB \sto \setC$.
    A battery vendor wants to sell batteries from both producers and wants to create a battery catalogue, which needs to take into account which technology comes from which producer, to be able to distribute the earnings from the sales fairly.
    To this end, the \SY{disjoint union} of the sets of technologies is considered:
    \begin{equation}
        \label{eq:exa-batt-disunion}
        \setA \setdisunion\setB=
        \makeset{
            \disunionA{\technology{LiPo}},
            \disunionA{\technology{LCO}},
            \disunionA{\technology{NiH2}},
            \disunionB{\technology{LFP}},
            \disunionB{\technology{LMO}},
            \disunionB{\technology{LiPo}}
        }.
    \end{equation}
    It is possible to map each technology in~$\setA,\setB$ to its own representative in~$\setA\setdisunion\setB$ via the so-called injection maps:
    \begin{equation}
        \label{eq:exa-batt-inclusion-1}
        \begin{aligned}
            \inj_\setA\colon \setA & \sto \setA\setdisunion\setB \\
            \setAel                & \mapsto \disunionA{\setAel}
        \end{aligned}
    \end{equation}
    \begin{equation}
        \label{eq:exa-batt-inclusion-2}
        \begin{aligned}
            \inj_\setB\colon \setB & \sto \setA\setdisunion\setB \\
            \setBel                & \mapsto \disunionB{\setBel}.
        \end{aligned}
    \end{equation}
    This situation is graphically represented in \cref{fig:coprod_batteries_1}, and mimics the coproduct diagram presented in \cref{def:categorical-coproduct}.

    \begin{figure*}[tbh]
        \centering
        % \maxsizebox{0.9\pagewidth}{!}{
        \includesag{60_coprod_batt_bis}
        % }
        \caption{Battery technologies, companies, prices, and a catalogue.}
        \label{fig:coprod_batteries_1}
        \todographics{\alphubel: @Gioele: use proper macro rather than $\iota$ in figure.
            Also, this diagram should be such that the "test object" (prices) are at the top, so that it matches the diagram in the definition of coproduct.
        }

    \end{figure*}

    \begin{figure*}[tbh]
        \centering
        % \maxsizebox{0.9\pagewidth}{!}{
        \includesag{60_coprod_batt_2_bis}
        % }
        \caption{Example: why the union is not the coproduct in \Set.}
        \label{fig:coprod_batteries_2}
        \todographics{\alphubel: @Gioele: use proper macro rather than $\iota$ in figure}
    \end{figure*}

    Here, the universal property says that there is a \textbf{unique} function
    \begin{equation}
        \coprodMapmor{\mapa}{\mapb}\colon \setA \setdisunion \setB \sto \setC
    \end{equation} such that
    \begin{equation}
        \label{eq:exa-batt-factoring-maps}
        \inj_\setA\mthen (\mapa \funcsum \mapb)=\mapa
        \qqand
        \inj_\setB\mthen (\mapa \funcsum \mapb)=\mapb.
    \end{equation}
    If we take a~$\styleelements{x}\setin \setA \setdisunion \setB$, it is either ``from~\setA or from~\setB'':
    \begin{equation}
        \label{eq:exa-batt-explain-sum-map}
        \text{either } \exists \setAel\setin \setA\colon \styleelements{x}=\inj_\setA(\setAel) \text{ or }\exists \setBel\setin \setB\colon \styleelements{x}=\inj_\setB(\setBel).
    \end{equation}
    From this, we can deduce that the desired map~$\coprodMapmor{\mapa}{\mapb}$ is:
    \begin{equation}
        \label{eq:exa-batt-def-sum-map}
        \begin{aligned}
            \coprodMapmor{\mapa}{\mapb} \colon  \setA \setdisunion \setB & \to \setC \\
            \styleelements{x}                                            & \mapsto
            \begin{cases}
                \mapa(\styleelements{x}), & \text{if } \styleelements{x}=\inj_\setA(\setAel),\quad \setAel \setin \setA, \\
                \mapb(\styleelements{x}), & \text{if } \styleelements{x}=\inj_\setB(\setBel),\quad \setBel \setin \setB.
            \end{cases}
        \end{aligned}
    \end{equation}
    \todotext{above is a bit weird way to say it}
    This is a specific example of \Set/\FinSet, in which the coproduct is a generalization of the concept of disjoint union.
    Now, we could spontaneously ask ourselves: why does the union not ``suffice'' for the coproduct definition in \Set?
    To see this, consider the same situation as before, but now having the catalogue of technologies given by~$\setA\setunion \setB$ (\cref{fig:coprod_batteries_2}).
    The interpretation of maps~$\mapa,\mapb$ does not change, and injections work as depicted.
    Note, however, that when asked for a map from the technology~$\technology{LiPo}\setin \setA\setunion \setB$, we have no notion of the company which produces it, and we are therefore unsure whether to assign it to~$\mapa(\technology{LiPo})=\unit[50]{CHF}$ or~$\mapb(\technology{LiPo})=\unit[60]{CHF}$.
    Indeed, the unique map~$\coprodMapmor{\mapa}{\mapb}$ required by the universal property of the coproduct cannot exist, since in case~$\setA\setintersection \setB\neq \Emptyset$, any element~$\styleelements{x}\setin \setA\setintersection \setB$ should be simultaneously sent to~$\mapa(\styleelements{x})$ and~$\mapb(\styleelements{x})$.
\end{example}

% \begin{figure}[h!]
%   \begin{center}
%   %\includesag{60_coprod_batt_2}
%   \includesag{60_coprod_batt_2_bis}
%   \end{center}
%   \caption{Example of why the union is not the coproduct in \Set.}
%   \label{fig:coprod_batteries_2}
% \end{figure}

\begin{example}
    Given~$\setA,\setB\setin \Obof\Rel$ (so~\setA and~\setB are sets) their coproduct is the disjoint union~$\setA \setdisunion \setB$.
    The \SY{disjoint union} of sets comes equipped with inclusion functions~$\inj_\setA\colon \setA\mto \setA \setdisunion\setB$ and~$\inj_\setB\colon \setB\mto \setA \setdisunion\setB$.
    If we turn these functions into relations
    \begin{align}\label{eq:coprod-rel-inclusions}
        {\relA}_{\inj_\setA} & \setsubseteq \setA\cartprod (\setA \setdisunion\setB), \\
        {\relA}_{\inj_\setB} & \setsubseteq \setB\cartprod (\setA \setdisunion\setB),
    \end{align}
    then these are the injection morphisms for the coproduct in \Rel.
    As an aside, we note that in \Rel products and coproducts are \emph{both} given by the \SY{disjoint union} of sets.
    We will see later why this is might be expected.
\end{example}

\begin{example}
    Let~$m,n\setin \natnumbers$, and draw an arrow~$m\to n$ if~$m$ divides~$n$.
    For instance, 6 divides 12 and hence there is an arrow~$6\to 12$.
    The coproduct between any two~$m,n\setin \natnumbers$ in this category is given by the least common multiple.
\end{example}

\begin{example}
    Consider the ordered set~$\realswithleq$, where given~$\elna{1},\elna{2}\setin \reals$ we can draw an arrow~$\elna{1}\to \elna{2}$ if~$\elna{1}\leq \elna{2}$.
    By following the coproduct's commutative diagram, we know that the coproduct of~$\elna{1}$ and~$\elna{2}$ is a~$z\setin \reals$ such that
    \begin{itemize}
        \item $\elna{1}\leq \elb$;
        \item $\elna{2}\leq \elb$;
        \item For all~$\ela\setin \reals$ with~$\elna{1}\leq \ela$ and~$\elna{2}\leq \ela$, we have~$\elb\leq \ela$.
    \end{itemize}
    In other words, the coproduct of~$\elna{1},\elna{2}\setin \reals$ is given by~$\max\makeset{\elna{1},\elna{2}}$, and is also called \emph{join}.
\end{example}

\begin{example}
    \label{ex:subset_coprod}
    Let~$\stylesets{S}$ be a set, and~$\setA,\setB\setsubseteq \stylesets{S}$ subsets.
    We can draw an arrow~$\setA \sto \setB$ if~$\setA\setsubseteq \setB$.
    By following the coproduct's commutative diagram, it is easy to see that the coproduct of~\setA and~\setB is given by~$\setA\setunion\setB$: the ``smallest'' set containing both~\setA and~\setB.
\end{example}

\begin{example}[Adapted from~\cite{spivak2014category}]
    \label{def:ex_graph}
    Given two graphs~$\graph=\tupp{\vertices,\arcs,\source,\target}$ and~$\graph'=\tupp{\vertices',\arcs',\source',\target'}$, their coproduct is a graph
    \begin{equation}
        \graph\Cplus\graph'=\tupp{\vertices \setdisunion\vertices',\arcs \setdisunion\arcs',\source\funcsum\source',\target\funcsum\target'}.
    \end{equation}
    In~$\graph\Cplus\graph'$, an arrow connects~$\vertexa_1$ to $\vertexa_2$ if:
    \begin{itemize}
        \item $\vertexa_1,\vertexa_2\setin \vertices$ or~$\vertexa_1,\vertexa_2\setin \vertices'$ (if both vertices belong to the same graph), and
        \item an arrow between from~$\vertexa_1$ to~$\vertexa_2$ exists in~$\graph$ or~$\graph'$.
    \end{itemize}
    % Given~$\source \colon \arcs \to \vertices$ and~$\source'\colon \arcs'\to \vertices'$, we have:
    % \todomistake{\alphubel: @Gioele: wrong definition}
    % \begin{equation}
    %     \begin{aligned}
    %         \source \funcsum \source'\colon \arcs\setdisunion \arcs' & \to \vertices \setdisunion \vertices' \\
    %         x                                                        & \mapsto
    %         \begin{cases}
    %             \source(\arc)  & \text{if } \arc\setin \arcs   \\
    %             \source'(\arc) & \text{if } \arc\setin \arcs',
    %         \end{cases}
    %     \end{aligned}
    % \end{equation}
    % and
    % \begin{equation}
    %     \begin{aligned}
    %         \target \funcsum \target '\colon \arcs\setdisunion \arcs ' & \to \vertices\setdisunion \vertices' \\
    %         \arc                                                       & \mapsto
    %         \begin{cases}
    %             \target(\arc)  & \text{if } \arc\setin \arcs   \\
    %             \target'(\arc) & \text{if } \arc\setin \arcs'.
    %         \end{cases}
    %     \end{aligned}
    % \end{equation}
    This is nicely graphically representable.
    Consider two graphs as in~\cref{fig:ex_graphs_1}.

    \begin{figure}[h!]
        \centering
        \begin{tabular}{cc}
            \includesag{60_graph_1_1} & \includesag{60_graph_1_2}
        \end{tabular}
        \caption{Example of graphs for which we want to consider the coproduct. }
        \label{fig:ex_graphs_1}
    \end{figure}

    Their coproduct is a graph including the ``disjoint union'' of both original graphs, without connecting them (\cref{fig:graphs_2}).

    \begin{figure}[h!]
        \centering
        \includesag{60_graph_2_1}
        \caption{Example of coproduct of graphs. }
        \label{fig:graphs_2}
    \end{figure}

\end{example}

\begin{lemma}[Coproduct in \Category]
    \label{lem:coproduct-in-cat}
    The \SY{disjoint union} of categories is the coproduct in \Category.
\end{lemma}

\begin{publictodo}
    Proof missing.
\end{publictodo}

\todotextjira{338}{\alphubel: @Gioele: Do the proof of \cref{lem:coproduct-in-cat}}
