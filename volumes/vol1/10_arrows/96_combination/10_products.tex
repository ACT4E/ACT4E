% !TEX root = chapter-standalone.tex

\section{Products}
\label{sec:combination-products}

\linkvideo{spring2021-products:products:cart-prod} % Cartesian product

% \linkvideo{spring2021-products:products} % Products

We start off by recalling a familiar way of combining two sets~\setA and~\setB.
First, we recall the notion of \emph{\SY{cartesian product of sets}} (introduced in \cref{sec:cartesian-product}) via an example.

\showslides{
    \begin{forslides}
        \begin{equation}
            \label{eq:prod-slides-1}
            \setA = \makeset{1, 2, 3, 4}
        \end{equation}
        %
        \begin{equation}
            \label{eq:prod-slides-2}
            \setB = \makeset{ \ast, \dagger }
        \end{equation}
        %
        \begin{equation}
            \label{eq:prod-slides-3}
            \Snacks=\makeset{\sapple,\sbanana,\scarrot}
        \end{equation}
        %
        \begin{equation}
            \label{eq:prod-slides-4}
            \Drinks=\makeset{\swater,\stea}
        \end{equation}
        %
        \begin{equation}
            \label{eq:prod-slides-5}
            \Participants
        \end{equation}
        %
        \begin{equation}
            \label{eq:prod-slides-6}
            \eats \colon \Participants \sto \Snacks
        \end{equation}
        %
        \begin{equation}
            \label{eq:prod-slides-7}
            \drinks \colon  \Participants \sto \Drinks
        \end{equation}
        %
        \begin{equation}
            \label{eq:prod-slides-8}
            \meal \colon \Participants\sto \Snacks \cartprod \Drinks
        \end{equation}
        %
        \begin{equation}
            \label{eq:prod-slides-9}
            \Snacks \cartprod \Drinks
        \end{equation}
        %
        \begin{equation}
            \label{eq:prod-slides-10}
            \eats
        \end{equation}
        %
        \begin{equation}
            \label{eq:prod-slides-11}
            \drinks
        \end{equation}
        %
        \begin{equation}
            \label{eq:prod-slides-12}
            \meal
        \end{equation}
        %
        \begin{equation}
            \label{eq:prod-slides-13}
            \setA
        \end{equation}
        %
        \begin{equation}
            \label{eq:prod-slides-14}
            \subA_1, \subA_2 \setsubseteq \setA
        \end{equation}
        %
        \begin{equation}
            \label{eq:prod-slides-15}
            \subA_1 \to \subA_2 \text{ iff } \subA_1 \setsubseteq \subA_2
        \end{equation}
        %
        \begin{equation}
            \label{eq:prod-slides-16}
            \setA = \makeset{ 1, 2, 3, 4 }
        \end{equation}
        %
        \begin{equation}
            \label{eq:prod-slides-17}
            \subA_1 = \makeset{ 1, 2, 3 }
        \end{equation}
        %
        \begin{equation}
            \label{eq:prod-slides-18}
            \subA_2 = \makeset{2, 3, 4 }
        \end{equation}
        %
        \begin{equation}
            \label{eq:prod-slides-19}
            \subA_1 \setintersection \subA_2 = \makeset{2, 3 }
        \end{equation}
        %
        \begin{equation}
            \label{eq:prod-slides-20}
            \makeset{ 1, 2, 3 } \setsupseteq \makeset{2, 3 } \setsubseteq \makeset{2, 3, 4 }
        \end{equation}
        %
        \begin{equation}
            \label{eq:coprod-slides-1}
            \setA=\makeset{\star, \diamond}
        \end{equation}
        %
        \begin{equation}
            \label{eq:coprod-slides-2}
            \setB=\makeset{\ast, \dagger}
        \end{equation}
        %
    \end{forslides}
}

\begin{example}[Cartesian product]
    Consider the sets~$\setA = \makeset{\sbanana, \sapple, \sgrapes, \scarrot}$ and~$\setB = \makeset{ \sbretzel, \sfondue }$.
    Recall that the \SY{cartesian product} is given by
    %
    \begin{equation}
        \label{eq:cart-prod-example}
        \begin{aligned}
            \setA \cartprod \setB = \opensetbracket & \tup{\sbanana, \sbretzel}, \tup{\sapple, \sbretzel}, \tup{\sgrapes, \sbretzel}, \tup{\scarrot, \sbretzel}, \\
                                                    & \tup{\sbanana,\sfondue}, \tup{\sapple, \sfondue}, \tup{\sgrapes, \sfondue}, \tup{\scarrot, \sfondue} \closesetbracket.
        \end{aligned}
    \end{equation}
    % We can, however, also represent~$\setA \cartprod \setB$ in a way which highlights its structure more (\cref{fig:example_cartesian}).
    % \begin{figure}[h!]
    %     \centering
    %     \includesag{050_example_cartesian}
    %     \caption{}
    %     \label{fig:example_cartesian}
    %     \todographics{@Gioele: reduce vertical space}
    % \end{figure}

    In particular, the \SY{cartesian product} comes naturally equipped with two projection maps~$\projA$ and~$\projB$ which map an element of~$\setA \cartprod \setB$ to its first and second coordinate, respectively:
    %
    \begin{equation}\label{eq:cart-prod-projection-map-defs}
        \projA(\tup{\ela,\elb}) =  \ela  \qqand \projB(\tup{\ela,\elb}) = \elb.
    \end{equation}
    %
    We will often depict the situation like this:
    %
    \equationsag{050_example_cartesian_2}{{fig:diagram_cartesian_2}}
\end{example}

It turns out that this situation is part of a pattern that unites various different constructions across mathematics.
This pattern, of which the \SY{cartesian product} is a special case, is formalized in the notion of ``categorical product''.
Before introducing the rigorous definition of the \SY{categorical product}, let us list a number of examples that are part of the pattern.

\begin{marginfigure}
    \centering
    \includesag{050_example_prod_min}
    \caption{Taking the minimum}
    \label{fig:exa_prod_min}
\end{marginfigure}

\begin{example}
    \label{exa:min-as-prod}
    For any~$\elna{1},\elna{2}\setin \reals$ draw an arrow~$\elna{1}\to \elna{2}$ iff~$\elna{1}\Rleq \elna{2}$.
    Then ``taking the minimum'' is an example of the \SY{categorical product}; see \cref{fig:exa_prod_min}.

    For instance, choosing~$\elna{1} = 3$,~$\elna{2} = 7$, we have~$3 \geq \min \makeset{3, 7} \leq 7$.
\end{example}

\begin{marginfigure}
    \centering
    \includesag{050_example_prod_gcd}
    \caption{Taking the greatest common divisor}
    \label{fig:exa_prod_gcd}
\end{marginfigure}

\begin{example}
    \label{exa:gcd-as-prod}
    For any~$m, n \setin \natnumbers$ let us draw an arrow~$m \to n$ iff~$m$ divides $n$, which is written $m | n$.
    Then ``taking the greatest common divisor'' is an example of the \SY{categorical product}; see \cref{fig:exa_prod_gcd}.

    For instance, choosing~$m = 6$,~$n = 9$, we have~$\gcd \makeset{6, 9 } = 3$, and~$3 | 6$ and~$3 | 9$.
\end{example}

\begin{marginfigure}
    \centering
    \includesag{050_example_prod_intersection}
    \caption{Taking the intersection.}
    \label{fig:exa_prod_intersection}
\end{marginfigure}

\begin{example}
    \label{exa:intersection-as-prod}
    Let~\setA be a set.
    For any subsets~$\subAn{1}, \subAn{2} \setsubseteq \setA$, let us draw an arrow~$\subAn{1} \to \subAn{2}$ iff~$\subAn{1} \setsubseteq \subAn{2}$.
    Then ``taking the intersection'' is an example of the \SY{categorical product}; see \cref{fig:exa_prod_intersection}.

    For instance, let~$\setA = \makeset{ 1, 2, 3, 4 }$,~$\subAn{1} = \makeset{ 1, 2, 3 }$, and~$\subAn{2} = \makeset{2, 3, 4 }$.
    Then~$\subA_1 \setintersection \subA_2 = \makeset{2, 3 }$, and
    \begin{equation}
        \makeset{ 1, 2, 3 } \setsupseteq \makeset{2, 3 } \setsubseteq \makeset{2, 3, 4 }.
    \end{equation}
\end{example}

\begin{marginfigure}
    \centering
    \includesag{050_example_prod_conjunction}
    \caption{Taking the conjunction}
    \label{fig:exa_prod_conjunction}
\end{marginfigure}

\begin{example}
    \label{exa:conjunction-as-prod}
    Let~$\setA = \makeset{ \true, \false }$ be the set of logical symbols consisting of true and false.
    For any propositions~$\ela, \elb  \setin \setA$, let us draw an arrow~$\ela \sto \elb$ iff~$\ela \Imp \elb$.
    Then ``taking the conjunction'' (the operation ``and'', denoted~$\booland$) of two statements is an example of the \SY{categorical product}; see \cref{fig:exa_prod_conjunction}.

    For instance, let~$\ela= \true$,~$\elb = \false$.
    Then~$\true \booland \false = \false$ and~$\true \Leftarrow \false \Imp \false$ holds.
\end{example}

\begin{marginfigure}
    \centering
    \includesag{050_example_prod_meet}
    \caption{Taking the meet}
    \label{fig:exa_prod_meet}
\end{marginfigure}

\begin{example}
    \label{exa:meet-as-prod}
    Let~\posA be a \SY{lattice}.
    For any element~$\ela, \elb \setin \posAset$, let us draw an arrow~$\ela \to \elb$ iff~$\ela \posleq \elb$.
    Then the ``meet'' (the operation~$\meet$) of two elements is an example of the \SY{categorical product}; see \cref{fig:exa_prod_meet}.
    Indeed, the ``meet'' is an object~$\elc \setin \posAset$ with arrows~$\elc\to \ela$ and~$\elc \to \elb$ such that for any~$\eld\setin \posAset$ with maps~$\eld\to \ela$ and~$\eld\to \elb$, we have a \emph{unique} morphism~$\eld\to \elc$ which makes the triangles commute.
    Note that in a \SY{lattice}, arrows are given by~$\posleq$-relations, which, if they exist, are \emph{unique}.
    With this in mind, the above description agrees with the one of a ``meet'':
    An object~$\elc\setin \posAset$ with $\elc\posleq \ela$,~$\elc\posleq \elb$ such that for any~$\eld\setin \posAset$, if~$\elc\posleq \ela$ and~$\elc\posleq \elb$, then~$\eld\posleq \elc$.
\end{example}

\begin{marginfigure}
    \centering
    \includesag{050_prod_generic}
    \caption{}
    \label{fig:prod_generic}
\end{marginfigure}

As you can see from the above list of examples, the notion of \SY{categorical product} involves diagrams of the type in \cref{fig:prod_generic}.

There is more to the story, however.
Loosely speaking, the categorical ``product of $\Obja$ and $\Objb$'' is characterized by how it interacts, in a certain way, with all other diagrams which have a similar form.
Let us explain using an ``applied'' example involving the \SY{cartesian product} of sets.

Suppose you are at an engineering conference in Switzerland, and there will be a hike as a group outing.
The organizers have prepared snacks to go.
Each participant can choose a food item from~$\Snacks=\makeset{\sapple,\sbanana,\scarrot}$ and a drink from~$\Drinks=\makeset{\swater,\stea}$.
Let~$\Participants$ denote the set of participants.
The choice of snacks could be organized as depicted in \cref{fig:snacks_1}, in which each participant chooses a food, and chooses a drink.
This can be described via functions~$\eats \colon \Participants \sto \Snacks$ and~$\drinks \colon  \Participants \sto \Drinks$.

\begin{figure}[h!]
    \centering
    \includesag{50_snacks_1}
    \caption{Each participant chooses a food and a drink.}
    \label{fig:snacks_1}
\end{figure}

Alternatively, snacks could be pre-packaged in such a way as to allow all possible combinations of food and drink choices.
This corresponds to~$\Snacks\cartprod \Drinks$.
Then the choice participants make of which lunch package they'd like is described by a single function~$\meal \colon \Participants\sto \Snacks\cartprod \Drinks$ (see \cref{fig:snacks_2}).

\begin{figure}[h!]
    \centering
    \includesag{50_snacks_2}
    \caption{Each participant chooses a combination of food and a drink. \label{fig:snacks_2}}
\end{figure}

Intuitively, the two situations (two choices separately, or one choice of a pre-packaged snack) are ``the same'' in a certain sense.
We can make this ``sameness'' precise.
Specifically, if we start with the functions~$\eats$ and~$\drinks$, we can use them to build the following function:
\begin{equation}
    \label{eq:snacks-building-univ-map}
    \begin{aligned}
        \catprodphi_{\eats,\drinks}\colon \Participants & \to \Snacks\cartprod \Drinks \\
        p                                               & \mapsto \tup{\eats(p),\drinks(p)}.
    \end{aligned}
\end{equation}
Furthermore, given~$\catprodphi_{\eats,\drinks}$, we can recover~$\eats$ and~$\drinks$:
\begin{equation}
    \label{eq:snacks-proj-univ-map}
    \eats =\catprodphi_{\eats,\drinks}\mthen \proj_1 \qqand \drinks=\catprodphi_{\eats,\drinks}\mthen \proj_2.
\end{equation}
These two equations say that the diagram in \cref{fig:snacks_3} is commutative.
The whole situation can be summarized thus: given a set~$\Participants$ and functions
\begin{equation}
    \eats\colon \Participants \mto \Snacks
\end{equation}
and
\begin{equation}
    \drinks\colon \Participants \mto \Drinks
\end{equation}
as in \cref{fig:snacks_2}, there is a unique function
\begin{equation}
    \label{eq:snacks-unique-univ-map}
    \catprodphi_{\eats,\drinks} \colon \Participants \to \Snacks \cartprod \Drinks
\end{equation}
such that the diagram in \cref{fig:snacks_3} commutes.

\begin{figure*}[h!]
    \centering
    \includesag{50_snacks_3}
    \caption{Choosing food and drink separately is essentially the same as choosing a combination of the two. }
    \label{fig:snacks_3}
    \todographics{@Gioele: drinks and eats are too close to the arrows}
\end{figure*}

It turns out that this state of affairs \emph{characterizes} the \SY{cartesian product} $\Snacks\cartprod \Drinks$.
We think of the diagram
%
\begin{equation}
    \label{eq:snacks-prod-diagram}
    \Snacks \mfrom \Snacks \cartprod \Drinks \mto \Drinks
\end{equation}
as ``interacting'' with the diagram
\begin{equation}
    \label{eq:snacks-prod-test-diagram}
    \Snacks \mfrom  \Participants \mto \Drinks
\end{equation}
via the fact that such a map $\catprodphi_{\eats,\drinks}$ exists which links the two by making the diagram \cref{fig:snacks_3} commute.

This describes the general pattern for the definition of the \SY{categorical product}.
Namely, the \SY{categorical product} is a diagram of the kind
\begin{equation}
    \label{eq:prod-inline-generic-diagram}
    \Obja \mfrom \text{``product of } \Obja \text{ and } \Objb \text{''}  \mto \Objb
\end{equation}
which interacts with any other diagram of the form
\begin{equation}
    \label{eq:prod-inline-test-diagram}
    \Obja \mfrom \styleobj{T} \mto \Objb
\end{equation}
in a way which is analogous to the situation in \cref{fig:snacks_3} (here~$\styleobj{T}$ stands for any set that plays the role of the set~$\Participants$) .

\section{Categorical product}
%\linkvideo{spring2021-products:products:cat-prod} % Categorical product
\linkvideo{spring2021-products:products:cat-prod:def-prof} % Definition of categorical product

Let us now finally state the general definition of \SY{categorical product}.
It is probably helpful to read the definition together with the clarifying remarks that follow it.

\begin{ctdefinition}[Categorical Product]
    \label{def:categorical-product}
    Let~\CatC be a category and let~$\Obja, \Objb \setin \ObC$ be objects.
    The \maindef{categorical product} of~$\Obja$ and~$\Objb$ is: \\
    \constit
    \begin{enumerate}
        \item an object~$\Objc \setin \ObC$ (this is ``the product of $\Obja$  and $\Objb$'');
        \item \emph{projection morphisms}~$\projA \colon \Objc \mto \Obja$ and~$\projB \colon \Objc \mto \Objb$,
    \end{enumerate}
    \condit
    \begin{enumerate}
        \item For any~$\styleobj{T} \setin \ObC$ and any morphisms~$\mora \colon \styleobj{T} \mto \Obja, \morb \colon \styleobj{T} \mto \Objb$, there exists a \emph{unique} morphism~$\catprodphi_{\mora,\morb} \colon \styleobj{T} \mto \Objc$ such that $\mora = (\catprodphi_{\mora,
                      \morb})\mthen \projA$ and $\morb=(\catprodphi_{\mora, \morb})\mthen \projB$.
    \end{enumerate}
\end{ctdefinition}

\begin{marginfigure}
    \centering
    \includesag{50_defproduct}
    \caption{}
    \label{fig:prod-def-diagram}
\end{marginfigure}

\begin{remark}
    \label{re:prod-comm-diag}
    Diagrammatically, the condition above states that the diagrams of the form in~\cref{fig:prod-def-diagram} commute.
\end{remark}

\begin{marginfigure}
    \centering
    \includesag{50_defproduct_generic}
    \caption{}
    \label{fig:prod-def-diagram-generic}
\end{marginfigure}

\begin{remark}
    \label{prod unique up to iso}
    In the above definition, technically both~$\Objc$ \emph{and} the projection morphisms constitute the data of ``the product of~$\Obja$ and~$\Objb$''.
    However, for simplicity, we usually refer only to~$\styleobj{Z}$ as ``the product''.
    Furthermore, we will usually use the notation~$\prodMapob\Obja\Objb$ to denote the product of~$\Obja$ and~$\Objb$, in place of~$\styleobj{Z}$.
    Similarly, we will usually write~$\prodMapmor\mora\morb$ in place of~$\catprodphi_{\mora, \morb}$.
    Then the diagram in~\cref{fig:prod-def-diagram} looks as in~\cref{fig:prod-def-diagram-generic}.

    The reason we do not do this directly in the definition itself is the following.
    In general, for fixed~$\Obja$ and~$\Objb$, there may be several objects~$\Objc$ (together with projection morphisms) that satisfy the definition of being ``the product of~$\Obja$ and~$\Objb$''.
    Thus, there is, technically, no such thing as ``\emph{the}'' (unique) product of~$\Obja$ and~$\Objb$.
    However, we can prove that any two candidates which satisfy the definition of being ``the product of~$\Obja$ and~$\Objb$'' will necessarily be isomorphic in a canonical manner.
    Thus, for simplicity, we will sometimes be slightly sloppy and speak of ``the product of~$\Obja$ and~$\Objb$'' as if it were unique.
    In many categories there is also a choice for ``the product of~$\Obja$ and~$\Objb$'' that we are used to.
    For example, in the category \Set, given sets~$\Obja$ and~$\Objb$, the familiar choice for ``the product of~$\Obja$ and~$\Objb$'' is the \SY{cartesian product}~$\Obja \cartprod \Objb$''.
    However, other representatives of the product of~$\Obja$ and~$\Objb$ are possible: \cref{ex:univ-prop-prod} illustrates this.
\end{remark}

\begin{remark}
    The condition in the definition of the \SY{categorical product} is known as the ``universal property of the product''.
    We will attempt to explain this naming.
    The stated condition involves the product~$\Objc$ of~$\Obja$ and~$\Objb$ \emph{interacting} with every possible choice of object~$\styleobj{T}$ and every possible choice of morphisms~$\mora \colon \styleobj{T} \mto \Obja$ and~$\morb\colon \styleobj{T} \mto \Objb$.
    We think of the ambient category~\CatC as ``the universe'' (or the ``context''), and this condition states how the product ~$\Objc$ must interact ``with the whole universe''.
    We choose the letter ``$\styleobj{T}$'' because we think of this as a ``test object'' (similar, for instance,  to how in electrodynamics a ``test charge'' is used to probe an electromagnetic field).
\end{remark}

\linkvideo{spring2021-products:products:cat-prod:example-prod} % Examples of products

\begin{example}[Product code]
    \label{ex:univ-prop-prod}
    Suppose that as a manufacturer, you want to label your products with
    \begin{itemize}
        \item a production date (8-digit code), and
        \item a model number (4-digit code).
    \end{itemize}
    Instead of two separate labels, you can also make one
    \begin{equation}
        \label{eq:exa-prod-univ-prop-code}
        202101155900,
    \end{equation}
    where the first 8 digits represent a date, and the last 4 digits are a model number.
    Call this single label the \emph{product code}.
    Let~\setC denote the set of all product codes, and consider the maps~$\proj_1\colon \setC \sto \setA$, and~$\proj_2\colon \setC \sto \setB$ which, respectively, map a 12-digit product code to its first 8 digits and its last 4 digits.
    One may check that~\setC, together with the map~$\proj_1$ and~$\proj_2$, will satisfy the definition of ``the product of~\setA and~\setB'':
    %
    \equationsag{050_digits_1}{eq:50_digits_1}
    %
    However,~\setC is not precisely the \SY{cartesian product} of~\setA and~\setB (which we will call~$\setA\cartprod \setB$).
    The elements of~\setC are 12-digit codes, while elements of~$\setA\cartprod \setB$ are pairs~$\tup{\ela,\elb}$ where~$\ela$ is an 8-digit code and~$\elb$ is a 4-digit code.
    Since both~\setC and~$\setA\cartprod \setB$ satisfy the definition of \SY{categorical product}, they must, by \cref{prod unique up to iso}, be isomorphic:
    %
    \equationsag{050_digits_3}{eq:50_digits_3}
    %
    To see concretely what this isomorphism between them looks like, note that there is a unique map $\catprodphi_{\text{first 8},\text{last 4}}$ making the following diagram commute:
    %
    \equationsag{050_digits_2}{eq:50_digits_2}
    %
    Concretely,
    \begin{equation}
        \label{eq:exa-univ-iso}
        \catprodphi_{\text{first 8},\text{last 4}}\colon \setC \sto \setA\cartprod \setB
    \end{equation}
    maps for instance
    \begin{equation}
        \label{eq:exa-univ-iso-concrete}
        \begin{aligned}
            202101155900 & \mapsto \tup{20210115,5900}.
        \end{aligned}
    \end{equation}
    One can readily show that~$\catprodphi_{\text{first 8},\text{last 4}}$ is an isomorphism.
\end{example}

Now let us revisit the examples that were given earlier, before we stated the definition of \SY{categorical product}.
The idea is that all of these are instantiations of the \SY{categorical product}, but in each case, we are working the in context of a different category.
In each case, arrows denote morphisms in the category in question, specific to the example.

\begin{marginfigure}
    \centering
    \includesag{050_example_prod_min_cont}
    \caption{Taking the minimum}
    \label{fig:exa_prod_min_cont}
\end{marginfigure}

\begin{example}
    \label{exa:min-as-prod-cont}
    This is a continuation of \cref{exa:min-as-prod}.
    For any~$\elna{1},\elna{2}\setin \reals$, we drew an arrow~$\elna{1}\to \elna{2}$ iff~$\elna{1}\Rleq \elna{2}$.
    The category in question here is the category whose objects are elements of~\reals, and whose morphisms are inequalities.
    The product is ``taking the minimum'';
    its universal property is illustrated in \cref{fig:exa_prod_min_cont}.
    It says that if~$\elb \setin \reals$ is such that~$\elb \leq \elna{1}$ and~$\elb \leq \elna{2}$, then~$\elb \leq \min \makeset{ \elna{1}, \elna{2} }$.

    To make things concrete, choose~$\elna{1} = 10$,~$\elna{2} = 18$, and experiment with different choices of~$\elb$.
    Verify that everything checks out.
\end{example}

\begin{marginfigure}
    \centering
    \includesag{050_example_prod_gcd_cont}
    \caption{Taking the greatest common divisor}
    \label{fig:exa_prod_gcd_cont}
\end{marginfigure}

\begin{example}
    \label{exa:gcd-as-prod-cont}
    This is a continuation of \cref{exa:gcd-as-prod}.
    For any~$m, n \setin \natnumbers$, we drew an arrow~$m \to n$ iff~$m$ divides $n$, written $m | n$.
    The category in questions has natural numbers as its objects, and morphisms are given by the relation ``divides''.
    Then product is ``taking the greatest common divisor''; its universal property is visualized in \cref{fig:exa_prod_gcd}.

    For a concrete example, let~$m = 12$ and~$n = 18$, so~$\gcd \makeset{12, 18 } = 6$.
    If we take~$t = 3$, which divides both~$12$ and~$18$, we see that, indeed,~$3$ also divides~$6 = \gcd \makeset{12, 18 }$.
    And if we take~$t = 2$, which \emph{also} divides both~$12$ and~$18$, we see that it is \emph{also} true that~$2$ also divides~$6 = \gcd \makeset{12, 18 }$.
\end{example}

\begin{marginfigure}
    \centering
    \includesag{050_example_prod_intersection_cont}
    \caption{Taking the intersection}
    \label{fig:exa_prod_intersection_cont}
\end{marginfigure}

\begin{example}
    \label{exa:intersection-as-prod-cont}
    This is a continuation of \cref{exa:intersection-as-prod}.
    Given a set~\setA and arbitrary subsets~$\subAn{1}, \subAn{2} \setsubseteq \setA$, we drew an arrow~$\subAn{1} \to \subAn{2}$ iff~$\subAn{1} \setsubseteq \subAn{2}$.
    The category in question here has as its objects the subsets of~\setA, and its morphisms are inclusions between them.
    The product is ``taking the intersection''; its universal property is visualized in \cref{fig:exa_prod_intersection}.

    As a concrete example, consider again~$\setA = \makeset{ 1, 2, 3, 4 }$,~$\subAn{1} = \makeset{ 1, 2, 3 }$, and~$\subAn{2} = \makeset{2, 3, 4 }$.
    So~$\subA_1 \setintersection \subA_2 = \makeset{2, 3 }$.
    If we choose~$\setC = \makeset{ 2 }$, we see that~$\setC \setsubseteq \subAn{1}$ and~$\setC \setsubseteq \subAn{2}$, and that also~$\setC \setsubseteq \subAn{1} \setintersection \subAn{2}$ (as it must, according to the universal property).
    The situation is similar if we choose~$\setC = \makeset{ 1}$ or~$\setC = \Emptyset$.
\end{example}

The following Lemma describes a general fact that was illustrated in \cref{exa:intersection-as-prod-cont}.
\begin{lemma}
    \label{lem:cat-product-powerset}
    Let~\setA be any set.
    Its \SY{powerset}~$\powerset \setA$, with the relation of inclusion, is a poset.
    View this \SY{poset} as a category (this means there is a single morphism~$\subAn{1} \mto \subAn{2}$ if and only if~$\subAn{1} \setsubseteq \subAn{2}$).
    For any two objects~$\subAn{1}, \subAn{2} \setin \powerset \setA$, their \SY{categorical product} exists and is given by~$\subAn{1} \setintersection \subAn{2} \setin \powerset \setA$.
\end{lemma}

\begin{marginfigure}
    \centering
    \includesag{050_example_prod_conjunction_cont}
    \caption{Taking the conjunction}
    \label{fig:exa_prod_conjunction_cont}
\end{marginfigure}

\begin{example}
    \label{exa:conjunction-as-prod-cont}
    This is a continuation of \cref{exa:conjunction-as-prod}.
    We considered~$\setA = \makeset{ \true, \false }$ as a set of logical propositions and for any~$\ela, \elb  \setin \setA$, we drew an arrow~$\ela \to \elb$ iff $\ela$ implies $\elb$.
    So, the category at play here has~\setA as its set of objects, and its morphisms are logical implications.
    The product is ``taking the conjunction'' (the logical operation ``and''); the universal property is visible in \cref{fig:exa_prod_conjunction}.

\end{example}

\begin{marginfigure}
    \centering
    \includesag{050_example_prod_meet_cont}
    \caption{Taking the meet}
    \label{fig:exa_prod_meet_cont}
\end{marginfigure}

\begin{example}
    \label{exa:meet-as-prod-cont}
    This is a continuation of \cref{exa:meet-as-prod}.
    We considered a \SY{lattice}~\posA, and we drew an arrow~$\ela \to \elb$ iff~$\ela \leq \elb$, for~$\ela,\elb\setin \posAset$.
    So the category at play here is the one corresponding to the poset~\posA.
    The \SY{categorical product} of two elements is their \SY{meet} (greatest lower bound); the universal property is illustrated in \cref{fig:exa_prod_meet}.
\end{example}

\begin{example}
    Suppose that we are designing a vehicle, and we are thinking about choices of engine.
    Both electric engines and internal combustion engines can produce \transmuted{motion}, but each from a different source of energy.
    The electric engine uses \transmuted{electric energy}; the internal combustion engine uses \transmuted{gasoline}.
    The situation is depicted in \cref{fig:e14}, using the interpretation of the arrows that we have introduced for engineering design components.
    Namely, the arrow from motion to gasoline represents the internal combustion engine, and its direction is to be read as follows: given the desired functionality~$\transmuted{motion}$, $\transmuted{internal \ combustion \ engine}$ provides a way of getting it using~$\transmuted{gasoline}$.
    The other arrow in the figure represents the component \transmuted{electric \ engine}, and is interpreted similarly:
    %
    \equationsag{30_dpcatfig_e14}{fig:e14}
    %
    We could also consider building a hybrid vehicle, where we can obtain \transmuted{motion} from \textbf{either} \transmuted{gasoline} \textbf{or} \transmuted{electric energy}:
    %
    \equationsag{30_dpcatfig_e15}{fig:e15}
    \todographics{@Gioele: need red points.
    }
\end{example}

\vfill
\begin{gradedexercise}[\exname{CatProductPowerset}]
    \label{ex:CatProductPowerset}
    Prove \cref{lem:cat-product-powerset} by checking that \cref{def:categorical-product} is satisfied.
\end{gradedexercise}

\solutionof{CatProductPowerset}

\begin{example}
    Consider two \SY{posets}~$\posA,\posB$ as categories.
    The category~$\poscat{(\posA\Ptimes \posB)}$ arising from the product poset~$\posA\Ptimes \posB$ (\cref{def:poset-product}) is the product category~$\poscat{\posA} \Ctimes \poscat{\posB}$ of the two posetal categories:
    \begin{equation}
        \poscat{(\posA\Ptimes \posB)} = (\poscat{\posA}) \Ctimes (\poscat{\posB}).
    \end{equation}
\end{example}

\begin{gradedexercise}[\exname{PosetProductCategorical}]
    Prove that the product poset~$\posA \Ptimes \posB$ ``is'' the \SY{categorical product} of~\posA and~\posB within the category~$\Pos$ of \SY{posets} and \SY{monotone maps}.
\end{gradedexercise}

\solutionof{PosetProductCategorical}
