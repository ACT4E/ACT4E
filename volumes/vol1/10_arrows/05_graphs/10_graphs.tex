% !TEX root = chapter-standalone.tex

\section{Graphs}

\todotextjira{67}{@Andrea: Write introduction to say why this section is here}

To begin, we recall some formal definitions related to (directed) graphs.
\footnote{
    In an alternative definition, a directed graph is defined as a pair~$\graph=\tup{\vertices, \arcs}$, made of a set of vertices~$\vertices$ and a set of edges
    \begin{equation*}
        \arcs \setsubseteq \{ \tup{\vertexa, \vertexb} \mid \tup{\vertexa, \vertexb} \setin \vertices \cartprod \vertices \text{ and } \vertexa \neq \vertexb\}.
    \end{equation*}
    The constructions in this chapter can be adapted for this definition too.
}

\begin{definition}[Directed Multigraph]
    \label{def:Graph}
    A \emph{\iindex{directed multigraph}}~$\graph=\tup{\vertices,\arcs, \source,\target}$ consists of a set of vertices~$\vertices$, a set of edges~$\arcs$, and two functions~$\source,\target \colon \arcs \to \vertices$, called the \emph{source} and \emph{target} functions, respectively.
    Given~$\arc\setin \arcs$ with~$\source(\arc)=\vertexa$ and~$\target(\arc)=\vertexb$, we say that~$\arc$ is an \emph{arrow} from~$\vertexa$ to~$\vertexb$.
\end{definition}

%\begin{remark}[Undirected graph]
%    We call \emph{undirected} a graph for which each edge is \emph{bidirectional}.
%    In other words, whenever~$\arc\setin \arcs$ with~$\source(\arc)=\vertexa$ and~$\target(\arc)=\vertexb$, then there is an arc~$\arc'\setin \arcs$ with~$\source(\arc')=\vertexb$ and~$\target(\arc')=\vertexa$.
%\end{remark}

\begin{remark}
    Both directed graphs and undirected graphs play a prominent role in many kinds of mathematics.
    In this text, we work primarily with directed multigraphs and so, from now on, we will drop the ``directed'' and the ``multi'': unless indicated otherwise, the word ``graph'' will mean ``directed multigraph''.
\end{remark}
~
Graphs are widely used in many engineering disciplines, to represent, formulate, and solve complex problems.
For instance, one can represent the mobility network in a city as a directed graph (\cref{fig:graph_road_network}).

\begin{marginfigure}
    \includesag{graph_ex_network}
    \caption{\label{fig:graph_road_network}}
\end{marginfigure}
The vertices represent locations, and the edges represent different roads connecting the locations.

%Another example is the one of \emph{resource graphs} (\cref{fig:resource_graphs}), which describe the architecture of a computing unit.
%\begin{marginfigure}
%    \includesag{graph_ex_proc}
%    \caption{\label{fig:resource_graphs}}
%\end{marginfigure}
%Here, vertices represent processors (e.g., CPUs) and arcs represent communication links between them.

\begin{definition}[Paths]
    \label{def:path}
    Let~$\graph$ be a graph.
    A \emph{path} in~$\graph$ is a possibly empty sequence of edges such that the target of one edge is the source of the next.
    The \emph{length} of a path is the number of edges in the sequence.
\end{definition}
We also formally allow for sequences made up of ``zero-many'' arrows (such paths therefore have length zero).
We call such paths \emph{trivial} or \emph{empty}.
If paths describe a journey, then trivial paths correspond to ``not going anywhere''.
The notions of source and target for edges extend, in an obvious manner, to paths.
For trivial paths, the source and target always coincide.
