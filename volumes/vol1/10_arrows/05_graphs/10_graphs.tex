% !TEX root = chapter-standalone.tex

\section{Graphs}

A graph is a data structure with ``points'' and ``arrows'', usually called \emph{no\-des}/\emph{ver\-tices} and \emph{arcs}/\emph{edges}.

\begin{marginfigure}
    \includegraphics[width=\linewidth]{camod}
    \caption{Intermodal mobility network of a city.}
    \label{fig:graph_road_network}
\end{marginfigure}

Graphs are widely used in many engineering disciplines, to represent, formulate, and solve complex problems.
For instance, we can represent the intermodal mobility network in a city as a directed graph (\cref{fig:graph_road_network}).
The word ``intermodal'' means that one can jump from a mobility option to another.
For instance, in the figure, you can spot the graphs for an autonomous vehicle mobility service, a micromobility service, the subway service, and roads on which you can walk.
The vertices represent locations, and the edges represent different travel routes connecting the locations.

\subsection{Defining graphs}

The usual definition of directed graph in engineering, which we will \emph{not} use, is as follows:

\begin{definition}[Directed Graph]
    \label{def:directed-graph}
    A \maindef{directed graph} is a pair~$\graph=\tup{\vertices, \arcs}$, where~$\vertices$ is
    a set of vertices and $\arcs \setsubseteq \vertices \cartprod \vertices$ is a set of edges.
\end{definition}
In this definition, an edge is a pair of vertices $\tup{\ela,\elb}$ where $\ela$ is the source and $\elb$ is the target.
One limitation of this notion of graph is that we can only have \emph{one} edge between two vertices in either direction.

The following definition is more expressive, though a bit more abstract.

\begin{ctdefinition}[Directed Multigraph]
    \label{def:Graph}
    A \maindef{directed multigraph}~$\graph=\tupp{\vertices,\arcs, \source,\target}$ consists of a set of vertices~$\vertices$, a set of edges~$\arcs$, and two functions~$\source,\target \colon \arcs \sto \vertices$, called the \emph{source} and \emph{target} functions, respectively.
    Given~$\arc\setin \arcs$ with~$\source(\arc)=\vertexa$ and~$\target(\arc)=\vertexb$, we say that~$\arc$ is an edge (or arrow) from~$\vertexa$ to~$\vertexb$.
\end{ctdefinition}

%\begin{remark}[Undirected graph]
%    We call \emph{undirected} a graph for which each edge is \emph{bidirectional}.
%    In other words, whenever~$\arc\setin \arcs$ with~$\source(\arc)=\vertexa$ and~$\target(\arc)=\vertexb$, then there is an arc~$\arc'\setin \arcs$ with~$\source(\arc')=\vertexb$ and~$\target(\arc')=\vertexa$.
%\end{remark}

% \begin{remark}
Both directed graphs and undirected graphs play a prominent role in many kinds of mathematics.
In this text, we work primarily with directed multigraphs and so, from now on, we drop the ``directed'' and the ``multi'': unless indicated otherwise, the word ``graph'' will mean ``directed multigraph''.

% \end{remark}

%Another example is the one of \emph{resource graphs} (\cref{fig:resource_graphs}), which describe the architecture of a computing unit.
%\begin{marginfigure}
%    \includesag{graph_ex_proc}
%    \caption{\label{fig:resource_graphs}}
%\end{marginfigure}
%Here, vertices represent processors (e.g., CPUs) and arcs represent communication links between them.

\subsection{Paths}

\begin{ctdefinition}[Paths]
    \label{def:path}
    \SYNDEF{path in a graph}
    A \emph{path} in a graph $\graph=\tupp{\vertices,\arcs, \source,\target}$ is:

    \constit

    \begin{itemize}
        \item a list of edges $\maketypedlist{e_1, \dots, e_n}{\arcs}$, with $n \in \natnumbers$.
              \begin{itemize}
                  \item If $n \neq 0$, the source of a path $\maketypedlist{e_1, \dots, e_n}{\arcs}$ is defined as $\source(e_1)$ and its target is $\target(e_n)$.
                  \item If $n = 0$, we speak of a ``trivial path'' or an ``empty path'' and we must additional specify an element $\ela \setin \vertices$ which is designated as both the source and target of the path.
                        If paths describe a journey, then trivial paths correspond to ``not going anywhere''.
              \end{itemize}
    \end{itemize}

    \condit

    \begin{itemize}
        \item if $n \geq 2$, we require that, for any two subsequent edges $e_i$ and $e_{i+1}$ in $\maketypedlist{e_1, \dots, e_n}{\arcs}$,
              \begin{equation}
                  \target(e_i) = \source(e_{i+1}).
              \end{equation}
    \end{itemize}
    The length of $\maketypedlist{e_1, \dots, e_n}{\arcs}$ is called the \emph{length} of the path.
\end{ctdefinition}
