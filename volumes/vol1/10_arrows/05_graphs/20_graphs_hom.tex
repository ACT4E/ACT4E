% !TEX root = chapter-standalone.tex

\section{Graph homomorphisms}
\begin{definition}[Graph homomorphism]
    \label{def:graph_homom}
    Given graphs~$\graph_1=\tup{\vertices_1,\arcs_1,\source_1,\target_1}$ and~$\graph_2=\tup{\vertices_2,\arcs_2,\source_2,\target_2}$, a graph homomorphism~$\funa \colon \graph_1 \fto \graph_2$ is given by maps~$\funaob\colon \vertices_1 \fto \vertices_2$ and~$\funamor\colon \arcs_1 \fto \arcs_2$, such that:
    \begin{equation*}
        (\funamor \mthen \source_2)(\arc)=(\source_1\mthen \funaob)(\arc), \quad \forall \arc\in \arcs_1,
    \end{equation*}
    and
    \begin{equation*}
        (\funamor \mthen \target_2)(\arc)=(\target_1\mthen \funaob)(\arc), \quad \forall \arc\in \arcs_1.
    \end{equation*}
    %\begin{equation}
    %    \middlesag{60_graph_homomorphism}
    %\end{equation}
\end{definition}

\begin{remark}
    Intuitively, all this is saying is that ``arrows are bound to their vertices'', meaning that if a vertex~$\vertexa_1$ is connected to~$\vertexa_2$ via an arrow~$\arc$, the vertices~$\funaob(\vertexa_1)$ and~$\funaob(\vertexa_2)$ have to be connected via an arrow~$\funamor(\arc)$.
\end{remark}

\begin{example}
    \label{exa:homomorphism_graph_positive}
    Let us consider the two graphs,~$\graph_1$ and~$\graph_2$ depicted in \cref{fig:ex_graph_homom}.
    \begin{figure}[h]
        \centering
        \includesag{20_ex_graph_hom}
        \caption{Example of graphs for graph homomorphism.}
        \label{fig:ex_graph_homom}
    \end{figure}

    \begin{marginfigure}
        \begin{center}
            \includesag{graph_homo_a}
        \end{center}
        \caption{\label{fig:graph_homo_a}}
    \end{marginfigure}

    \begin{marginfigure}
        \begin{center}
            \includesag{graph_homo_b}
        \end{center}
        \caption{\label{fig:graph_homo_b}}
    \end{marginfigure}
    A possible graph homomorphism between the two is given by~$\funaob,\funamor$ graphically defined as in \cref{fig:graph_homo_a} and \cref{fig:graph_homo_b}, respectively.
\end{example}

\begin{example}[Counterexample]
    By considering the graphs in \cref{exa:homomorphism_graph_positive}, one could define~$\funaob,\funamor$ in the same way, exception made for~$\funaob(e)=x$.
    Clearly, this would violate the commuting diagrams condition.
\end{example}

\begin{exercise}
    Consider the two graphs depicted in \cref{fig:ex_graph_hom}.
    Furthermore, consider the map~$\funaob$ depicted in \cref{fig:exercise_graph_hom_a}.
    \begin{figure}[h]
        \centering
        \includesag{graph_hom_ex}
        \caption{
            \label{fig:ex_graph_hom}}
    \end{figure}
    %
    \begin{marginfigure}
        \begin{center}
            \includesag{graph_homo_ex_a}
        \end{center}
        \caption{\label{fig:exercise_graph_hom_a}}
    \end{marginfigure}

    Find a map~$\funamor$ such that~$\funaob,\funamor$ describe a graph homomorphism between~$\graph$ and~$\graph'$.
\end{exercise}

\begin{solution}
    We know:
    \begin{compactitem}
        \item $\source(1)=a$ and~$\target(1)=b$: one has~$\funaob(a)=x$ and~$\funaob(b)=y$, meaning that~$\funamor(1)=\text{I}$.
        \item $\source(2)=b$ and~$\target(2)=c$: one has~$\funaob(b)=y$ and~$\funaob(c)=z$, meaning that~$\funamor(2)=\text{II}$.
        \item $\source(3)=c$ and~$\target(3)=d$: one has~$\funaob(c)=z$ and~$\funaob(d)=x$, meaning that~$\funamor(3)=\text{III}$.
        \item $\source(4)=d$ and~$\target(4)=e$: one has~$\funaob(d)=x$ and~$\funaob(e)=z$, meaning that~$\funamor(4)=\text{IV}$.
        \item $\source(5)=e$ and~$\target(5)=a$: one has~$\funaob(e)=z$ and~$\funaob(a)=x$, meaning that~$\funamor(5)=\text{IV}$.
    \end{compactitem}
    Therefore, the map~$\funamor$ is as reported in \cref{fig:exercise_graph_hom_b}.

    \begin{marginfigure}
        \begin{center}
            \includesag{graph_homo_ex_b}
        \end{center}
        \caption{\label{fig:exercise_graph_hom_b}}
    \end{marginfigure}
\end{solution}
