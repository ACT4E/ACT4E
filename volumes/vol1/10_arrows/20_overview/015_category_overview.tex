% !TEX root = standalone.tex


\section{Compositionality}

The word \emph{intelligence}, from latin \emph{intellego} comes from Proto-Indo-European  \emph{*h${}_1$entér} (``between'') and \emph{*leǵ} (``to gather''), and we can translate it as \emph{the ability to gather things together} to obtain some goal; this, for us, is the essence of intelligence. These \emph{things} to gather could be abstract, such as pieces of evidence to achieve a conclusion, or physical, like ingredients to prepare a tasty meal, or the parts to create machines. Lately, \emph{homo sapiens} has cultivated and inclination to create machines that could help them think, and we need to tell them how to do this intelligence trick, not just solve single problems, but see how solutions to smaller problems can be composed to solve bigger problems.

The first encounter children have with composition is with toy blocks like Lego. It is a coincidence that there is a \emph{lego} in \emph{intellego}; the lego in Lego is a contraction from Danish \emph{leg godt}, which means \emph{to play well}.

Legos are compositional in this sense: when you put together two blocks, you can treat the ensemble as one block for the purpose of composing it with other blocks. For example, you can take 3 blocks of dimension $1/3 \times 2 \times 4$ and compose them together to obtain one block of dimension $1 \times 2 \times 4$. This composed block works the same as one primitive  $1 \times 2 \times 4 $ block.

\hfill
\begin{tikzpicture}
  \brick{4}{2}{red}{0}{0}{0}{0.33} %<-- 7th argument for block height
  \brick{4}{2}{red}{0}{0}{0.33}{0.33}
  \brick{4}{2}{red}{0}{0}{0.66}{0.33}
\end{tikzpicture}
\hfill
\begin{tikzpicture}
  \brick{4}{2}{red}{0}{0}{0}{1}
\end{tikzpicture}
\hfill


\todographics{I found some tikz code to draw lego, but we would need to adapt it to change also the brick height}

Here is one formal property of Lego: given two blocks of the same shape, you can always make a block of twice the shape.

More generally, to compose blocks in this way, you would only care about the lateral dimensions.
You can compose an $a \times b \times c$ block with an  $a \times b \times d$ block to obtain a  $a \times b \times( c + d)$ block.
Note that, for now, $a$ and $b$ are treated as labels, not as numbers.

We like to put this in formula as follows. We put the ingredients to the top, and the results at the bottom.
%
\begin{equation}
  \frac{\text{block }\ a \times b \times c \qquad\text{block}\ a \times b \times d }{%
    \text{block }\ a \times b \times (c+d)%
  }
\end{equation}


Do you know the game ``spot the 5 differences''? In this book we are going to play the opposite game, which
is ``spot how different things are the same at some level of abstraction''.

It might have been a while
since you played with Lego, but you are certainly familiar with plugs, sockets, and extension cords.
If you have an extension cord of length $c$ and another of length $d$, you can plug them together to get an extension cord of length $c+d$.

\todographics{Graphics of the connectors}


The rule for extension cords is similar to the rule for Lego blocks.

\begin{equation}
  \frac{\text{extension cord of length }\ c \qquad \text{extension cord of length }\ d }{%
    \text{extension cord of length }\ (c+d)%
  }
\end{equation}

Extension cords are a bit more general: Lego blocks are constrainted to have a height that is a multiple of one brick,  but extension cords can have any continuous value as length.

On the other hand, Lego blocks also have this other property of the horizontal section $a \times b$. We only gave rules for the connection of





\todographics{lego figure here: the 3 blocks composed using rotations}

If you have legos of different colors, it becomes a bit more complicated. The rule now becomes


There are also other ways you could compose the 3 blocks. You could slightly translate them, to create stairs.

\todographics{lego figure here: the 3 blocks composed in different ways, by some offset}

You can also rotate them before composing.

\todographics{lego figure here: the 3 blocks composed using rotations}

What are the rules of this composition? Let's define a block as a solidly connected set of parallelipipedes (malformed cubes with different dimensions of the 3 sides). There are 3 kinds of faces: the faces pointing up, which have the pins, the faces pointing down, which have the holes for the pins, and the lateral faces.

Connecting two blocks means that there should be at least one pin face of the lower block touching one hole face of the upper block; and, that there are no intersections of the solid blocks. Then, the ``interface'' of the blocks


You can also put decorations; but if you do, you remove the possibility of connecting to all of them.

\todographics{lego block with some decoration on top}


\section{Guidebook}

\todo[inline]{Broader overview of the abstractions that we will develop in this book,
  including posets, lattices, etc. up to operad.}


