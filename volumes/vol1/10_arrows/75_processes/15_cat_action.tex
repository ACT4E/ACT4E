% !TEX root = chapter-standalone.tex

\section{Action of a category}
\label{sec:action-of-a-category}

\linkvideo{spring2021-actions:semi-cat-actions} % Semi-category action

\todotextjira{59}{@Andrea: Enrich text, using descriptions from lecture}
\todojira{457}{Introduce LTI story}

Now it is time to generalize from actions of a semigroup to action of a semicategory.

Let~$\CatC$ a general process category; either \Moore or \More.
% as in \cref{tab:processes-types}.

We want a process in~$\HomSet\CatC\Obja\Objb$ to induce a map between signals.

Note that each process type has a different signal types.

For example, a Moore machine in~$\HomSet\Moore\Obja\Objb$ will map the set~$\natnumbers \to \Obja$ to the set~$\natnumbers \to \Objb$.
A More machine in~$\HomSet\More\Obja\Objb$  will map the set~$\seqsof \Obja$ to the set~$\seqsof \Objb$.

To look at this generically, we need to consider a map~$\catacOb$ that maps the base set~$\Obja$ to the actual set~$\catacOb(\Obja)$.
For example:
%
\begin{equation}
    \defmapcommaset{
        \catacOb_{\Moore}
    }{
        \Set
    }{
        \Set
    }{
        \Obja
    }{
        (\natnumbers \to \Obja)
    }
\end{equation}
%
\begin{equation}
    \defmapperiodset{
        \catacOb_{\More}
    }{
        \Set
    }{
        \Set
    }{
        \Obja
    }{
        \seqsof \Obja
    }
\end{equation}
%
Then we need another map~$\catacMor$ that given a process in~$\HomSet\CatC\Obja\Objb$ produces a map from~$\catacOb(\Obja)$ to~$\catacOb(\Objb)$.
%
\begin{equation}
    \label{eq:gamma-1}
    \catacMor\colon \HomSet\CatC\Obja\Objb \to  (\catacOb(\Obja) \sto \catacOb(\Objb)).
\end{equation}
%
Interpreting the arrows as morphisms in sets, we can say:
%
\begin{equation}
    \label{eq:gamma-2}
    \catacMor\colon \HomSet\CatC\Obja\Objb \to  \HomSet\Set{\catacOb(\Obja)}{\catacOb(\Objb) }.
\end{equation}
%

\begin{ctdefinition}[Semicategory action]
    \label{def:semicategory-action}
    A \emph{semicategory action} of a semicategory~$\CatC$ is defined by
    \begin{itemize}
        \item a map~$\catacOb$ that associates, to each object~$\Obja \setin \ObC$, a set~$\catacOb(\Obja)$:
              \begin{equation}
                  \catacOb \colon \ObC \to \Obof\Set;
              \end{equation}
        \item a map~$\catacMor$ that associates, to each morphism in $\CatC$, a function:
              \begin{equation}
                  \catacMor\colon \HomSet\CatC\Obja\Objb \to  \HomSet\Set{\catacOb(\Obja)}{\catacOb(\Objb) };
              \end{equation}
    \end{itemize}
    Moreover, this condition must hold:
    \begin{equation}
        \catacMor(\mora\then\morb) = \catacMor(\mora) \mthen \catacMor(\morb).
    \end{equation}
    The condition is depicted in commuting diagrams in \cref{fig:semicat_ac_comm}
\end{ctdefinition}

\begin{figure}
    \begin{center}
        \includesag{semicat_ac_comm}
    \end{center}
    \caption{\label{fig:semicat_ac_comm}}
\end{figure}

\todojira{585}{Add explanation that action of composition is composition of actions}

\showslides{

    \begin{forslides}

        \begin{equation}
            \label{eq:cat-mora}
            \catacMor(\mora)
        \end{equation}
        \begin{equation}
            \label{eq:cat-morb}
            \catacMor(\morb)
        \end{equation}
        \begin{equation}
            \label{eq:cat-morab}
            \catacMor(\mora\mthen\morb)
        \end{equation}
        \begin{equation}
            \label{eq:catC}
            \CatC
        \end{equation}
        \begin{equation}
            \label{eq:sets}
            \Set
        \end{equation}
        %
        \begin{equation}
            \label{eq:signal}
            \styleobj{s}
        \end{equation}
        %
        \begin{equation}
            \label{eq:signal-apply}
            \act(\mora, \styleobj{s})
        \end{equation}
        %
        \begin{equation}
            \label{eq:morab}
            \mora \mthen \morb
        \end{equation}
        %
        \begin{equation}
            \label{eq:mora}
            \mora
        \end{equation}
        %
        \begin{equation}
            \label{eq:morb}
            \morb
        \end{equation}
        %
        \begin{equation}
            \label{eq:mora-prdyn}
            \prdyn_{\mora}
        \end{equation}
        %
        \begin{equation}
            \label{eq:mora-prst}
            \prst_{\mora}
        \end{equation}
        %
        \begin{equation}
            \label{eq:mora-prin}
            \prin_{\mora}
        \end{equation}
        %
        \begin{equation}
            \label{eq:mora-prout}
            \prout_{\mora}
        \end{equation}
        %
        \begin{equation}
            \label{eq:morb-prst}
            \prst_{\morb}
        \end{equation}
        %
        \begin{equation}
            \label{eq:morb-prin}
            \prin_{\morb}
        \end{equation}
        %
        \begin{equation}
            \label{eq:morb-prout}
            \prout_{\morb}
        \end{equation}
        %
        \begin{equation}
            \label{eq:morab-prst}
            \prst_{\mora\mthen\morb}
        \end{equation}
        %
        \begin{equation}
            \label{eq:morab-prin}
            \prin_{\mora\mthen\morb}
        \end{equation}
        %
        \begin{equation}
            \label{eq:morab-prout}
            \prout_{\mora\mthen\morb}
        \end{equation}
        %
        \begin{equation}
            \label{eq:gen-prst}
            \prst
        \end{equation}
        %
        \begin{equation}
            \label{eq:gen-prin}
            \prin
        \end{equation}
        \begin{equation}
            \label{eq:gen-prout}
            \prout
        \end{equation}
        \begin{equation}
            \label{eq:moore-start}
            \prstart \setin \prst
        \end{equation}
        %
        \begin{equation}
            \label{eq:prgen-u}
            u\colon \natnumbers \to \prgen
        \end{equation}
        %
        \begin{equation}
            \label{eq:prgen-y}
            y\colon \natnumbers \to \prgen
        \end{equation}
        %
        \begin{equation}
            \label{eq:prgen-u-1}
            u\colon \natnumbers \to \prin
        \end{equation}
        %
        \begin{equation}
            \label{eq:prgen-y-1}
            y\colon \natnumbers \to \prout
        \end{equation}
        %
        \begin{equation}
            \label{eq:prgen-u-2}
            u\colon \prin
        \end{equation}
        %
        \begin{equation}
            \label{eq:Z}
            \stylesets{Z}
        \end{equation}
        %
        \begin{equation}
            \label{eq:Za}
            \catacOb{\stylesets{Z}}
        \end{equation}
        %
        \begin{equation}
            \label{eq:Ua}
            \catacOb{\prin}
        \end{equation}
        %
        \begin{equation}
            \label{eq:Ya}
            \catacOb{\prout}
        \end{equation}
        %
        \begin{equation}
            \label{eq:prgen-y-2}
            y\colon \seqsof  \prout
        \end{equation}
        %
        \begin{equation}
            \label{eq:prgen}
            \prgen
        \end{equation}
        %
        \begin{equation}
            \label{eq:now1}
            \act(\morb, \act(\mora, \styleobj{s}))
        \end{equation}
        %
        \begin{equation}
            \label{eq:now2}
            \act(\mora\mthen\morb, \styleobj{s})
        \end{equation}
        %
        \begin{equation}
            \label{eq:ev-sign}
            u\colon \seqsof{(\natnumbers \cartprod \prin)}
        \end{equation}
        %
        \begin{equation}
            \label{eq:ev-sign2}
            y\colon \seqsof{(\natnumbers \cartprod \prout)}
        \end{equation}
        \begin{equation}
            \label{eq:phi-moore}
            {\catacOb_{\Moore}}\colon{\Obja}\longmapsto{(\natnumbers \to \Obja)}
        \end{equation}
        \begin{equation}
            \label{eq:phi-more}
            {\catacOb_{\More}}\colon{\Obja}\longmapsto{\seqsof \Obja}
        \end{equation}
    \end{forslides}
}
