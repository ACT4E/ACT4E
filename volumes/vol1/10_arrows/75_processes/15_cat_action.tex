% !TEX root = chapter-standalone.tex

\section{Semicategory actions}
\label{sec:action-of-a-category}

\linkvideo{spring2021-actions:semi-cat-actions} % Semi-category action

In this section we generalize from actions of a \SY{semigroup} to actions of a \SY{semicategory}, using the example of \SY{Moore machines} acting on signal sequences.

To motivate our story, let us first consider \SY{Moore machines} of the form
\begin{equation}\label{eq:generic-moore-machine}
    \tup{\prinL,\prstL,\prinL,\prdyn,\prreadout,\prstart}
\end{equation}
where the input and output sets are equal.
In other words, we are considering the set~$\HomSet{\Moore}{\prinL}{\prinL}$, which is a \SY{semigroup} under morphism composition.

In \cref{sec:Moore-acting-on-sequences} we defined a standard action which associates to every morphism~$\mora \setin \HomSet{\Moore}{\prinL}{\prinL}$ a function
\begin{equation}\label{eq:defining-act-moore-functions}
    \act_{\mora} \colon \streamsof{\prinL} \to \streamsof{\prinL}.
\end{equation}
This action defines a function
\begin{equation}\label{eq:defining-act-moor-functor-on-arrows}
    \funmorspace{\act} \colon \HomSet{\Moore}{\prinL}{\prinL} \to \End (\streamsof{\prinL})
\end{equation}
where
\begin{equation}
    \funmorspace{\act}(\mora) = \act_{\mora}.
\end{equation}
And this function~$\funmorspace{\act}$ is in fact a \SY{semigroup morphism}, and so a \SY{semigroup action}, since we proved in \cref{prop:moore-action-is-a-morphism} that
\begin{equation}
    \act_{\morab} = \act_\mora \mthen \act_\morb.
\end{equation}

(Compare with \cref{def:semigroup-cov-action} of \SY{semigroup action}.)

Now consider the general situation of \SY{Moore machines} acting on signals.
Given any \SY{Moore machine} of the general form
\begin{equation}
    \mora = \tup{\prinL,\prstL,\proutL,\prdyn,\prreadout,\prstart}
\end{equation}
(the input and output spaces are no longer necessarily equal) we again have an associated function on signals
\begin{equation}
    \act_{\mora} \colon \streamsof{\prinL} \to \streamsof {\proutL}.
\end{equation}
We can assemble this data as a family of functions (all which we call~$\funmorspace{\act}$)
\begin{equation}
    \label{eq:moore-action-becoming-a-functor}
    \funmorspace{\act} \colon \HomSet{\Moore}{\prinL}{\proutL} \to \HomSet{\Set}{\streamsof\prinL}{\streamsof\proutL},
\end{equation}
where~$\prinL$ and~$\proutL$ range over all objects of~$\Moore$.
Or, if you will,
\begin{equation}
    \funmorspace{\act} \colon \Mor_\Moore \to \Mor_\Set.
\end{equation}
From \cref{prop:moore-action-is-a-morphism} we have that this function is compatible with the composition operations in~$\Moore$ and~\Set:
\begin{equation}
    \label{eq:moore-action-functor-and-composition}
    \funmorspace{\act}(\morab) = \funmorspace{\act}(\mora) \mthen \funmorspace{\act}(\morb).
\end{equation}

Note that the sets~$\streamsof\prinL$ and~$\streamsof\proutL$ involved in the right-hand side of \cref{eq:moore-action-becoming-a-functor} depend on the objects~$\prinL$ and~$\proutL$ on the left-hand side.
We will encode this also with a function
\begin{equation}
    \defmapcomma{\funobspace{\act}}
    {\Obof{\Moore}}
    {\to}
    {\Obof{\Set}}
    {\prinL}
    {\streamsof\prinL}
\end{equation}
in which case \cref{eq:moore-action-becoming-a-functor} becomes
\begin{equation}
    \label{eq:moore-action-as-a-functor}
    \funmorspace{\act} \colon \HomSet{\Moore}{\prinL}{\proutL} \sto \HomSet{\Set}{ \funobspace{\act}\prinL}{\funobspace{\act}\proutL}.
\end{equation}
%
In summary, we have reformulated \SY{Moore machine} actions as consisting of a pair of functions
\begin{equation}
    \funmorspace{\act} \colon \Mor_\Moore \sto \Mor_\Set \qquad \text{ and } \qquad \funobspace{\act} \colon \Obof{\Moore} \sto \Obof{\Set}
\end{equation}
which work together as in \cref{eq:moore-action-as-a-functor} and such that~$\funmorspace{\act}$ is compatible with composition as in \cref{eq:moore-action-functor-and-composition}.

Now we formalize this situation as a general definition.

\begin{ctdefinition}[Semicategory action]
    \label{def:semicategory-action}
    A \maindef{semicategory action} of a \SY{semicategory}~\CatC is

    \constit
    \begin{enumerate}
        \item A map~$\funobspace{\act} \colon \Obof{\CatC} \sto \Obof{\Set}$;
        \item For every two objects $\Obja, \Objb \setin \Obof{\CatC}$, a map
              \begin{equation}
                  \funmorspace{\act} \colon \HomSet{\CatC}{\Obja}{\Objb} \to \HomSet{\Set}{ \funobspace{\act}(\Obja)}{\funobspace{\act}(\Objb)}.
              \end{equation}
    \end{enumerate}

    \condit

    \begin{itemize}
        \item For all composable morphisms $\mora$ and $\morb$,
              \begin{equation}
                  \funmorspace{\act}(\morab) = \funmorspace{\act}(\mora) \mthen \funmorspace{\act}(\morb).
              \end{equation}
    \end{itemize}
\end{ctdefinition}

The compatibility of the action with composition is illustrated in \cref{fig:semicat_ac_comm}.

\begin{marginfigure}
    \centering
    \includesag{semicat_ac_comm}
    \caption{}
    \label{fig:semicat_ac_comm}
\end{marginfigure}

For reference, let us also fix the following definition.

\begin{definition}[Standard action of Moore machines]
    \label{def:moore-standard-action-on-sequences}
    The \maindef{standard action of Moore machines} on sequences is given by
    \begin{equation}
        \defmapcomma{\funobspace{\act}}
        {\Obof{\Moore}}
        {\to}
        {\Obof{\Set}}
        {\prinL}
        {\streamsof\prinL}
    \end{equation}
    on the level of objects, and on the level of morphisms, the functions
    \begin{equation}
        \funmorspace{\act} \colon \HomSet{\Moore}{\prinL}{\proutL} \to \HomSet{\Set}{\streamsof\prinL }{\streamsof\proutL}
    \end{equation}
    are defined via the recursion equations
    \begin{equation}
        \begin{cases}
            \prsteln{k+1} = \prdyn_\mora(\prineln{k} \tupconcat \prsteln{k}) \\
            \prouteln{k} = \prreadout_\mora(\prsteln{k})
        \end{cases}
    \end{equation}
    as in \cref{sec:Moore-acting-on-sequences}.
\end{definition}

