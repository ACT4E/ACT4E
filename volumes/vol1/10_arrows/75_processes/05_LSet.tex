% !TEX root = chapter-standalone.tex

\section{The category \SetL}
\label{sec:cartcatset}

We will define a category~\SetL whose objects are ``sets of tuples''.

%\subsection{Lists}
%
%Before defining L-sets, let's first recall our notion of lists.
%Given three things, for example (call them $\ela$, $\elb$, and $\elc$) we can list them in some order:
%\begin{equation}
%    [\elb, \elc, \ela].
%\end{equation}
%The square brackets are our notation for lists, and recall that the ordering matters for lists (for instance, $[\elc, \ela, \elb]$ is a different list of those same three things).
%
%Any list has a length; for example $[\elb, \elc, \ela]$ has length 3.
%For us, by definition, all lists have finite length.
%We also define such a thing as an \emph{empty list}, which we denote by $[ \ ]$.
%It has length zero.
%
%The things in a list -- for example, $\elb$, $\elc$, and $\ela$ -- are the \emph{entries} of the list.
%They each come with an index corresponding to their placement in the list (so we can speak of the 1st entry, 2nd entry, and so on).

Given sets~$\setAn{1}, \setAn{2}, \ldots, \setAn{n}$, where $n \setin \natnumbers$, we define
\begin{equation}
    \label{eq:sets-of-tuples}
    \cObj{\setAn{1}, \setAn{2}, \ldots, \setAn{n}} \definedas \makeset{ \tup{\elna{1}, \elna{2}, \ldots, \elna{n}} \mid \elna{1} \setin \setAn{1}, \elna{2} \setin \setAn{2}, \ldots, \elna{n} \setin \setAn{n} }.
\end{equation}
So for example,
\begin{equation}
    \label{eq:setL-example}
    \cObj{\wnumbers, \natnumbers, \reals} = \makeset{ \tup{\ela, \elb, \elc} \mid \ela \setin \wnumbers, \elb \setin \natnumbers, \elc \setin \reals }.
\end{equation}

% A special case is when any of the sets $\setAn{1}, \dots, \setAn{n}$ are the empty set $\Emptyset$; then we have
% \begin{equation}
%     \cObj{\setAn{1}, \setAn{2}, \ldots, \setAn{n}} = \Emptyset.
% \end{equation}
Note the case $n = 0$, where
\begin{equation}
    \cObj{} = \makeset{ \tup{} }
\end{equation}
is a singleton set whose element is the empty tuple.

% We make the convention that~$\cObj{\setAn{1}, \setAn{2}, \ldots, \setAn{n}}$ denotes the empty set if any of the sets~$\setAn{1}, \setAn{2}, \ldots , \setAn{n}$ are equal to the empty set.
% In particular, we have for example~$\Emptyset = \cObj{\Emptyset}$.

All the objects of~$\SetL$ are in particular just sets, so we can package them into a category whose morphisms are simply functions between such sets.

\begin{ctdefinition}[\SetL]\label{def:SetL}
    \SYNDEF{category of tuple-sets and functions}
    The category \SetL is the \SY{subcategory} of \Set in which the objects are the sets that can be represented as~$\cObj{\setAn{1}, \setAn{2}, \ldots, \setAn{n}}$, for any~$n \setin \natnumbers$ and any sets~$\setAn{1}, \setAn{2}, \ldots, \setAn{n}$.
\end{ctdefinition}
\todomistake{\alphubel: It is very nice to use the word \SY{subcategory} here, but we haven't defined \SY{subcategories} yet.
    How to change the structure?}
\subsection{From \Set to~\SetL}

Observe that we can turn any set into an object of \SetL: given a set~\setA, there is the associated set
\begin{equation}
    \label{eq:setintosetl}
    \cObj{\setA} = \makeset{ \tup{\ela} \mid \ela \setin \setA }
\end{equation}
whose elements are precisely all the tuples of length one whose entry is an element of~\setA.
This defines a function
\begin{equation}
    \label{eq:setintosetlfunction}
    \begin{aligned}
        \Obof{\Set} & \to \Obof{\SetL} \\
        \setA       & \mapsto \cObj{\setA}
    \end{aligned}
\end{equation}
%\Ob_{\Set} \to \Ob_{\SetL}, \setA \mapsto \cObj{\setA}
which one might call ``bracket''.

Similarly, given a function~$\mapa \colon \setA \sto \setB$ between sets, there is an associated function~$\cMor{ \mapa }\colon \cObj{\setA} \mto \cObj{\setB}$ given by
\begin{equation}
    \label{eq:functionintosetlmor}
    \cMor{ \mapa } ( \tup{\ela }) \definedas \tup{\mapa(\ela)}.
\end{equation}
This defines a function
\begin{equation}
    \label{eq:functionintosetlmorfun}
    \Hom_{\Set}(\setA,\setB) \to \Hom_{\SetL}(\cObj{\setA},\cObj{\setB})
\end{equation}
for any sets~\setA and~\setB; one might also call it ``bracket''.

\subsection{Concatenation of tuples}

Given tuples~$\tup{\ela, \elb, \elc}$ and~$\tup{\eld, \ele}$ we can stick them together to make the longer tuple~$\tup{\ela, \elb, \elc, \eld, \ele}$.
We call this operation \emph{concatenation}, and in the context of \SetL we think of it as a kind of multiplication.
Our notation for concatenation of tuples is
\begin{equation}\label{eq:concatenation-of-tuples-SetL}
    \tup{\ela, \elb, \elc} \tupconcat \tup{\eld, \ele} = \tup{\ela, \elb, \elc, \eld, \ele}.
\end{equation}
\todotext{\alphubel: We already defined the concatenation of tuples...}
For the empty tuple~$\tup{ }$ we set
\begin{equation}\label{eq:empty-tuple-concat-1}
    \tup{} \tupconcat \tup{\ela, \elb, \elc} = \tup{\ela, \elb, \elc}
\end{equation}
and
\begin{equation}\label{eq:empty-tuple-concat-2}
    \tup{\ela, \elb, \elc} \tupconcat \tup{} = \tup{\ela, \elb, \elc}.
\end{equation}

\subsection{Multiplication in \SetL}
\label{sec:multiplication-for-SetL}
We define a multiplication for objects of \SetL.
The symbol~$\cprod$ will denote this multiplication in infix notation.

Given~$\cObj{\setAn{1}, \ldots, \setAn{m}}$ and~$\cObj{\setBn{1}, \ldots, \setBn{n}}$ we define the operation
\begin{equation}
    \label{eq:mult-setl-sign}
    \cprod \colon \Obof{\SetL} \cartprod \Obof{\SetL} \to \Obof{\SetL}
\end{equation}
by
\begin{equation}
    \label{eq:strict-monoidal-product-sets}
    \makecprod{
        \cObj{\setAn{1}, \ldots, \setAn{m}},
        \cObj{\setBn{1}, \ldots, \setBn{n}}
    } \definedas \cObj{\setAn{1}, \ldots, \setAn{m}, \setBn{1}, \ldots, \setBn{n}}
\end{equation}
for any $n, m \setin \natnumbers$.

The elements of~$\cObj{\setAn{1}, \ldots, \setAn{m}} \cprod \cObj{\setBn{1}, \ldots, \setBn{n}}$ are all possible concatenations of elements of~$\cObj{\setAn{1}, \ldots, \setAn{m}}$ with elements of~$\cObj{\setBn{1}, \ldots, \setBn{n}}$.

Equation \cref{eq:strict-monoidal-product-sets} holds in particular also in the cases when~$m = 0$ or~$n =0$:
\begin{equation}
    \makecprod{
        \emptycobj,
        \cObj{\setBn{1}, \ldots, \setBn{n}}
    } =
    \cObj{
        \setBn{1}, \ldots, \setBn{n}
    }
\end{equation}
and
\begin{equation}
    \makecprod{
        \cObj{\setAn{1}, \ldots, \setAn{m}},
        \emptycobj
    } =
    \cObj{
        \setAn{1}, \ldots, \setAn{m}
    }.
\end{equation}

% Also, writing $\Emptyset = \cObj{\Emptyset}$ we find that
% \begin{equation}
%     \makecprod{
%         \cObj{\setAn{1}, \ldots, \setAn{m}},
%         \Emptyset
%     } = \Emptyset
% \end{equation}
% and
% \begin{equation}
%     \makecprod{\Emptyset, \cObj{\setBn{1}, \ldots, \setBn{n}}} = \Emptyset.
% \end{equation}

The multiplication of objects in~\SetL is \SY{associative} (which was our goal with making this construction).
Indeed, both
\begin{equation}
    \makecprod{
        ( \makecprod{\cObj{\setAn{1}, \ldots, \setAn{l}}, \cObj{\setBn{1}, \ldots, \setBn{m}}})
        ,
        \cObj{\setCn{1}, \ldots, \setCn{n}}
    }
\end{equation}
and
\begin{equation}
    \makecprod{
        \cObj{\setAn{1}, \ldots, \setAn{l}},
        (
        \makecprod{
            \cObj{\setBn{1}, \ldots, \setBn{m}},
            \cObj{\setCn{1}, \ldots, \setCn{n}}
        }
        )
    }
\end{equation}
are equal to
\begin{equation}
    \cObj{\setAn{1}, \ldots, \setAn{l}, \setBn{1}, \ldots, \setBn{m}, \setCn{1}, \ldots, \setCn{n}}.
\end{equation}

\showslides{
    \begin{forslides}
        \begin{equation}
            \label{eq:moore-assoc-a-bis}
            \mora \mthen (\morb \mthen \morc)
        \end{equation}
        %
        \begin{equation}
            \label{eq:moore-assoc-b-bis}
            (\mora \mthen \morb) \mthen \morc
        \end{equation}
        %
        \begin{equation}
            \label{eq:lset}
            \SetL
        \end{equation}
        %
        \begin{equation}
            \label{eq:definition-setl}
            \setAn{1}, \setAn{2}, \ldots, \setAn{n}
        \end{equation}
        %
        \begin{equation}
            \label{eq:definition-setlbis}
            n\setin \natnumbers
        \end{equation}
        \begin{equation}
            \label{eq:mult-setl}
            \cObj{\setAn{1}, \setAn{2}, \ldots, \setAn{m}}
        \end{equation}
        \begin{equation}
            \label{eq:mult-setlb}
            \cObj{\setBn{1}, \setBn{2}, \ldots, \setBn{n}}
        \end{equation}
    \end{forslides}
}