% !TEX root = chapter-standalone.tex

\section{The category~$\cCat{\Set}$}
\label{sec:cartcatset}

We will define a category~$\cCat{\Set}$ whose objects are ``sets of tuples''.

%\subsection{Lists}
%
%Before defining L-sets, let's first recall our notion of lists.
%Given three things, for example (call them $\ela$, $\elb$, and $\elc$) we can list them in some order:
%\begin{equation}
%    [\elb, \elc, \ela].
%\end{equation}
%The square brackets are our notation for lists, and recall that the ordering matters for lists (for instance, $[\elc, \ela, \elb]$ is a different list of those same three things).
%
%Any list has a length; for example $[\elb, \elc, \ela]$ has length 3.
%For us, by definition, all lists have finite length.
%We also define such a thing as an \emph{empty list}, which we denote by $[ \ ]$.
%It has length zero.
%
%The things in a list -- for example, $\elb$, $\elc$, and $\ela$ -- are the \emph{entries} of the list.
%They each come with an index corresponding to their placement in the list (so we can speak of the 1st entry, 2nd entry, and so on).

Given sets~$\setAn{1}, \setAn{2}, \dots , \setAn{n}$, if they are all non-empty, then we define
\begin{equation}
    \label{eq:sets-of-tuples}
    \cObj{\setAn{1}, \setAn{2}, \dots, \setAn{n}} \definedas \{ \tup{\elna{1}, \elna{2}, \dots, \elna{n}} \mid \elna{1} \in \setAn{1}, \elna{2} \in \setAn{2}, \dots, \elna{n} \in \setAn{n} \}.
\end{equation}
So for example,
\begin{equation*}
    \cObj{\wnumbers, \natnumbers, \reals} = \{ \tup{\ela, \elb, \elc} \mid \ela \in \wnumbers, \elb \in \natnumbers, \elc \in \reals \}.
\end{equation*}

Essentially, the objects of~$\cCat{\Set}$ will be sets of the form \cref{eq:sets-of-tuples}.
We also include the empty set~$\Emptyset$ as an object of~$\cCat{\Set}$, as well as a special one-element set~$\stylesets{[ \ ]} \definedas \{ \tup{} \}$ whose single element is the empty tuple.

Furthermore, we make the convention that~$\cObj{\setAn{1}, \setAn{2}, \dots, \setAn{n}}$ denotes the empty set if any of the sets~$\setAn{1}, \setAn{2}, \dots , \setAn{n}$ are equal to the empty set.

The things in a list -- for example,~$\elb$,~$\elc$, and~$\ela$ -- are the \emph{entries} of the list.
They each come with an index corresponding to their placement in the list (so we can speak of the 1st entry, 2nd entry, and so on).

All of the objects of~$\cCat{\Set}$ are in particular just sets, so we can package them into a category whose morphisms are simply functions between sets.

\begin{ctdefinition}[\cCat{\Set}]
    The category \cCat{\Set} is:

    \begin{itemize}
        \item Objects: sets of the kind $\cObj{\setA_1, \setA_2, \dots, \setA_n}$, with $n \in \natnumbers$;
        \item Morphisms: ordinary functions between sets;
        \item Composition: the usual composition of functions;
        \item Identities: identity functions.
    \end{itemize}
\end{ctdefinition}

\subsection{From \Set to \cCat{\Set}}

Observe that we can turn any set into an object of \cCat{\Set}: given a set $\setA$, there is the associated set
\begin{equation*}
    \cObj{\setA} = \{ \tup{\ela} \mid \ela \in \setA \}
\end{equation*}
whose elements are precisely all the tuples of length one whose entry is an element of~$\setA$.
This defines a function
\begin{equation*}
    \defmapcomma{\mapb}{\Ob_{\Set}}
    {\to}
    {\Ob_{\cCat{\Set}}}
    {\setA }
    {\cObj{\setA}}
\end{equation*}
%\Ob_{\Set} \to \Ob_{\cCat{\Set}}, \setA \mapsto \cObj{\setA}
which one might call ``bracket''.

Similarly, given a function $\mora \colon \setA \to \setB$ between sets, there is an associated function $\cMor{ \mora }: \cObj{\setA} \mto \cObj{\setB}$ given by
\begin{equation}
    \cMor{ \mora } ( \tup{ \ela }) := \tup{\mora(a)}.
\end{equation}
This defines a function
\begin{equation}
    \Hom_{\Set}(\setA,\setB) \to \Hom_{\Cat{LSet}}(\stylesets{[\setA]},\stylesets{[\setB]})
\end{equation}
for any sets $\setA$ and $\setB$; one might also call it ``bracket''.


\subsection{Concatenation of tuples}

Given tuples $\tup{\ela, \elb, \elc}$ and $\tup{\eld, \ele}$ we can stick them together to make the longer tuple $\tup{\ela, \elb, \elc, \eld, \ele}$.
We call this operation concatenation, and in the context of \cCat{\Set} we think of it as a kind of multiplication.
Our notation for concatenation of tuples is
\begin{equation}
    \tup{\ela, \elb, \elc} \tupconcat \tup{\eld, \ele} = \tup{\ela, \elb, \elc, \eld, \ele}.
\end{equation}
For the empty tuple $\tup{  }$ we set
\begin{equation*}
    \tup{} \tupconcat \tup{\ela, \elb, \elc}  = \tup{\ela, \elb, \elc}
\end{equation*}
and
\begin{equation*}
    \tup{\ela, \elb, \elc}  \tupconcat \tup{ } = \tup{\ela, \elb, \elc}.
\end{equation*}


\subsection{Multiplication in \cCat{\Set}}

We define a multiplication for objects of \cCat{\Set}. The symbol $\cprod$ will denote this multiplication in infix notation.

First of all, we set
\begin{equation}
    \cObj{\setAn{1}, \dots, \setAn{m}} \cprod \Emptyset = \Emptyset.
\end{equation}
and
\begin{equation}
    \Emptyset \cprod \cObj{\setBn{1}, \dots, \setBn{n}} = \Emptyset
\end{equation}

More generally, given $\cObj{\setAn{1}, \dots, \setAn{m}}$ and $\cObj{\setBn{1},  \dots, \setBn{n}}$ we define
\begin{equation}
\label{eq:strict-monoidal-product-sets}
   \cObj{\setAn{1}, \dots, \setAn{m}} \cprod \cObj{\setBn{1}, \dots, \setBn{n}} \definedas \cObj{\setAn{1}, \dots, \setAn{m}, \setBn{1},  \dots, \setBn{n}}.
\end{equation}
In other words, when both factors are non-empty, the elements of $\cObj{\setAn{1}, \dots, \setAn{m}} \cprod \cObj{\setBn{1}, \dots, \setBn{n}}$ are all possible concatenations of elements of $\cObj{\setAn{1}, \dots, \setAn{m}}$ with elements of $\cObj{\setBn{1}, \dots, \setBn{n}}$.

Equation \cref{eq:strict-monoidal-product-sets} holds in particular also in the cases when $m = 0$ or $n =0$:
\begin{equation}
    \cObj{ \ } \cprod \cObj{\setBn{1}, \dots, \setBn{n}} = \cObj{\setBn{1},  \dots, \setBn{n}}
\end{equation}
and
\begin{equation}
    \cObj{\setAn{1}, \dots, \setAn{m}} \cprod \cObj{ \ } = \cObj{\setAn{1}, \dots, \setAn{m}}.
\end{equation}


\begin{remark}
The multiplication of objects in $\cCat{\Set}$ is associative: both
\begin{equation}
( \cObj{\setAn{1}, \dots, \setAn{l}} \cprod \cObj{\setBn{1}, \dots, \setBn{m}}) \cprod \cObj{\setCn{1}, ..., \setCn{n}}
\end{equation}
and
\begin{equation}
\cObj{\setAn{1}, \dots, \setAn{l}} \cprod (\cObj{\setBn{1}, \dots, \setBn{m}} \cprod \cObj{\setCn{1}, ..., \setCn{n}})
\end{equation}
are equal to
\begin{equation}
\cObj{\setAn{1}, \dots, \setAn{l}, \setBn{1}, \dots, \setBn{m}, \setCn{1}, ..., \setCn{n}}.
\end{equation}
\end{remark}
