% !TEX root = chapter-standalone.tex

\section{The category~$\cCat{\Set}$}
\label{sec:cartcatset}

We will define a category~$\cCat{\Set}$ whose objects are ``sets of tuples''.

%\subsection{Lists}
%
%Before defining L-sets, let's first recall our notion of lists.
%Given three things, for example (call them $\ela$, $\elb$, and $\elc$) we can list them in some order:
%\begin{equation}
%    [\elb, \elc, \ela].
%\end{equation}
%The square brackets are our notation for lists, and recall that the ordering matters for lists (for instance, $[\elc, \ela, \elb]$ is a different list of those same three things).
%
%Any list has a length; for example $[\elb, \elc, \ela]$ has length 3.
%For us, by definition, all lists have finite length.
%We also define such a thing as an \emph{empty list}, which we denote by $[ \ ]$.
%It has length zero.
%
%The things in a list -- for example, $\elb$, $\elc$, and $\ela$ -- are the \emph{entries} of the list.
%They each come with an index corresponding to their placement in the list (so we can speak of the 1st entry, 2nd entry, and so on).

Given sets~$\setAn{1}, \setAn{2}, \dots , \setAn{n}$, if they are all non-empty, then we define
\begin{equation}
    \label{eq:sets-of-tuples}
    \cObj{\setAn{1}, \setAn{2}, \dots, \setAn{n}} \definedas \{ \tup{\elna{1}, \elna{2}, \dots, \elna{n}} \mid \elna{1} \in \setAn{1}, \elna{2} \in \setAn{2}, \dots, \elna{n} \in \setAn{n} \}.
\end{equation}
So for example,
\begin{equation*}
    \cObj{\wnumbers, \natnumbers, \reals} = \{ \tup{\ela, \elb, \elc} \mid \ela \in \wnumbers, \elb \in \natnumbers, \elc \in \reals \}.
\end{equation*}

Essentially, the objects of~$\cCat{\Set}$ will be sets of the form \cref{eq:sets-of-tuples}.
We also include the empty set~$\Emptyset$ as an object of~$\cCat{\Set}$, as well as a special one-element set~$\stylesets{[ \ ]} \definedas \{ \tup{} \}$ whose single element is the empty tuple.

Furthermore, we make the convention that~$\cObj{\setAn{1}, \setAn{2}, \dots, \setAn{n}}$ denotes the empty set if any of the sets~$\setAn{1}, \setAn{2}, \dots , \setAn{n}$ are equal to the empty set.

The things in a list -- for example,~$\elb$,~$\elc$, and~$\ela$ -- are the \emph{entries} of the list.
They each come with an index corresponding to their placement in the list (so we can speak of the 1st entry, 2nd entry, and so on).

All of the objects of~$\cCat{\Set}$ are in particular just sets, so we can package them into a category whose morphisms are simply functions between sets.

\begin{ctdefinition}[\cCat{\Set}]
    The category \cCat{\Set} is:

    \begin{itemize}
        \item Objects: L-sets;
        \item Morphisms: ordinary functions between L-sets;
        \item Composition: composition of functions;
        \item Identities: identity functions.
    \end{itemize}
\end{ctdefinition}

\subsection{From \Set to \cCat{\Set}}

Observe that we can turn any set into an object of \cCat{\Set}: given a set $\setA$, there is the associated set
\begin{equation*}
    \cObj{\setA} = \{ \tup{\ela} \mid \ela \in \setA \}
\end{equation*}
whose elements are precisely all the tuples of length one whose entry is an element of~$\setA$.
This defines a function
\begin{equation*}
    \defmapcomma{\mapb}{\Ob_{\Set}}
    {\to}
    {\Ob_{\cCat{\Set}}}
    {\setA }
    {\cObj{\setA}}
\end{equation*}
%\Ob_{\Set} \to \Ob_{\cCat{\Set}}, \setA \mapsto \cObj{\setA}
which one might call ``bracket''.

Similarly, given a function $\mora \colon \setA \to \setB$ between sets, there is an associated function $\cMor{ \mora }: \cObj{\setA} \mto \cObj{\setB}$ given by
\begin{equation}
    \cMor{ \mora } ( \tup{ \ela }) := \tup{\mora(a)}.
\end{equation}
This defines a function
\begin{equation}
    \Hom_{\Set}(\setA,\setB) \to \Hom_{\Cat{LSet}}(\stylesets{[\setA]},\stylesets{[\setB]})
\end{equation}
for any sets $\setA$ and $\setB$; one might also call it ``bracket''.

\subsection{Multiplication in \cCat{\Set}}

Given tuples~$\tup{\ela, \elb, \elc}$ and~$\tup{\eld, \ele}$ we can stick them together to make the longer tuple~$\tup{\ela, \elb, \elc, \eld, \ele}$.
We call this operation concatenation, and in the context of \cCat{\Set} we will think of it as a kind of multiplication of tuples.
Our notation for concatenation is
\begin{equation*}
    \tup{\ela, \elb, \elc} \tupconcat \tup{\eld, \ele} = \tup{\ela, \elb, \elc, \eld, \ele}.
\end{equation*}
For the empty tuple $\tup{  }$ we also have
\begin{equation*}
    \tup{} \tupconcat \tup{\ela, \elb, \elc}  = \tup{\ela, \elb, \elc}
\end{equation*}
and
\begin{equation*}
    \tup{\ela, \elb, \elc}  \tupconcat \tup{ } = \tup{\ela, \elb, \elc}.
\end{equation*}

Now we define a multiplication of objects of \cCat{\Set}.
Given~$\cObj{\setAn{1}, \dots, \setAn{m}}$ and~$\cObj{\setBn{1},  \dots, \setBn{n}}$ which are both neither the empty set nor~$\cObj{}$, we define
\begin{equation*}
    \cObj{\setAn{1}, \dots, \setAn{m}} \cprod \cObj{\setBn{1}, \dots, \setBn{n}} \definedas \cObj{\setAn{1}, \dots, \setAn{m}, \setBn{1},  \dots, \setBn{n}}.
\end{equation*}
In other words, the elements of~$\cObj{\setAn{1}, \dots, \setAn{m}} \cprod \cObj{\setBn{1}, \dots, \setBn{n}}$ are all possible concatenations of elements of $\cObj{\setAn{1}, \dots, \setAn{m}}$ with elements of $\cObj{\setBn{1}, \dots, \setBn{n}}$.

If either~$\cObj{\setAn{1}, \dots, \setAn{m}}$ or $\cObj{\setBn{1}, \dots, \setBn{n}}$ is the empty set, we define $\cObj{\setAn{1}, \dots, \setAn{m}} \cprod \cObj{\setBn{1}, \dots, \setBn{n}}$ to also be the empty set:
\begin{equation*}
    \Emptyset \cprod \cObj{\setBn{1}, \dots, \setBn{n}} = \Emptyset
\end{equation*}
and
\begin{equation*}
    \cObj{\setAn{1}, \dots, \setAn{m}} \cprod \Emptyset = \Emptyset.
\end{equation*}

And for~$\cObj{}$ we have the following rules:
\begin{equation*}
    \stylesets{[ \ ]} \cprod \cObj{\setBn{1}, \dots, \setBn{n}} = \cObj{\setBn{1},  \dots, \setBn{n}}
\end{equation*}
and
\begin{equation*}
    \cObj{\setAn{1}, \dots, \setAn{m}} \cprod \stylesets{[ \ ]} = \cObj{\setAn{1}, \dots, \setAn{m}}.
\end{equation*}
(Note that these rules for multiplying with $\cObj{}$ and the ones for multiplying with~$\Emptyset$ are compatible with each other in the cases where they overlap).

\begin{remark}
    The multiplication of objects in~$\cCat{\Set}$ is associative.
\end{remark}
\todotext{expand the above remark to include a formula that illustrates the statement}
