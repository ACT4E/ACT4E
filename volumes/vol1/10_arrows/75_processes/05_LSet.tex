% !TEX root = chapter-standalone.tex

\section{The category $\Cat{LSet}$}

In the following we will define a category whose objects are, loosely speaking, ``sets of typed lists''.
We will call these L-sets.

\subsection{Lists}

Before defining L-sets, let's first recall our notion of lists.
Given three things, for example (call them $\ela$, $\elb$, and $\elc$) we can list them in some order:
\begin{equation}
    [\elb, \elc, \ela].
\end{equation}
The square brackets are our notation for lists, and recall that the ordering matters for lists (for instance, $[\elc, \ela, \elb]$ is a different list of those same three things).

Any list has a length; for example $[\elb, \elc, \ela]$ has length 3.
For us, by definition, all lists have finite length.
We also define such a thing as an \emph{empty list}, which we denote by $[ \ ]$.
It has length zero.

The things in a list -- for example, $\elb$, $\elc$, and $\ela$ -- are the \emph{entries} of the list.
They each come with an index corresponding to their placement in the list (so we can speak of the 1st entry, 2nd entry, and so on).

\subsection{L-sets}

Suppose now that we are given some sets $\setA$, $\setB$ and $\setC$.
We will denote by $\stylesets{[\setA, \setB, \setC]}$ the set of all lists of the form $[\ela, \elb, \elc]$, where $\ela \in \setA$, $\elb \in \setB$, and $\elc \in \setC$.
In other words
\begin{equation}
    \stylesets{[\setA, \setB, \setC]} = \{ [\ela, \elb, \elc] \mid \ela \in \setA, \elb \in \setB, \elc \in \setC \}.
\end{equation}
Beside some details that we will address presently, this essentially captures what we mean L-sets to be: sets of lists where the entries of the lists are each specified to be an element of a certain set.

Now the details.
In our definition of L-sets, we will also include the empty set $\Emptyset$ as an L-set, and also a special one-element set $\stylesets{[ \ ]} := \{ [ \ ] \}$ whose single element is the empty list.
We'll call the latter the unit L-set.

Given sets $\setA_1, \setA_2, \dots , \setA_n$, if they are all non-empty, then we have this L-set
\begin{equation}
    \stylesets{[\setA_1, \setA_2, \dots, \setA_n]} := \{ [ \ela_1, \ela_2, \dots, \ela_n] \mid \ela_1 \in \setA_1, \ela_2 \in \setA_2, \dots, \ela_n \in \setA_n \},
\end{equation}
of typed lists.
If however one or more of the sets $\setA_1, \setA_2, \dots , \setA_n$ are equal to the empty set, then \begin{equation}
    \stylesets{[\setA_1, \setA_2, \dots, \setA_n]} := \Emptyset.
\end{equation}

\subsection{The category $\Cat{LSet}$}

\cCat{\Set}

$\cObj{\Obja} \cprod \cObj{\Objb}$

$\tup{x_1, ..., x_n} \tupconcat \tup{y_1, ..., y_n}$

\

\

Since LSets are in particular just sets, we can package them into a category whose morphisms are the simply functions between sets.

\begin{ctdefinition}[\Cat{LSet}]
    The category \Cat{LSet} is:

    \begin{itemize}
        \item Objects: L-sets;
        \item Morphisms: ordinary functions between L-sets;
        \item Composition: composition of functions;
        \item Identities: identity functions.
    \end{itemize}
\end{ctdefinition}

\subsection{From sets to L-sets}

Observe that we can turn any set into an L-set: given a set $\setA$, there is the associated L-set
\begin{equation}
    \stylesets{[\setA]} = \{ [\ela] \mid \ela \in \setA \}
\end{equation}
whose elements are precisely all the lists of length one whose entry is an element of $\setA$.
This defines a function
\begin{equation}
    \Ob_{\Set} \to \Ob_{\Cat{LSet}}, \setA \mapsto \stylesets{[\setA]}
\end{equation}
which one might call ``bracket'' or ``listify''.

Similarly, given a function $\mora \colon \setA \to \setB$ between sets, there is an associated function $\stylemorph{[ \mora ]}: \stylesets{[\setA]} \mto \stylesets{[\setB]}$ given by
\begin{equation}
    \stylemorph{[ \mora ]} ([ \ela ]) := [\mora(a)].
\end{equation}
This defines a function
\begin{equation}
    \Hom_{\Set}(\setA,\setB) \to \Hom_{\Cat{LSet}}(\stylesets{[\setA]},\stylesets{[\setB]})
\end{equation}
for any sets $\setA$ and $\setB$; one might also call it ``bracket'' or ``listify''.

\subsection{Multiplication of L-sets}

Given lists $[\ela, \elb, \elc]$ and $[\eld, \ele]$ we can stick them together to make a longer list: $[\ela, \elb, \elc, \eld, \ele]$.
We call this operation concatenation, and in the context of L-sets we will think of it as a kind of multiplication of lists.
Our notation for concatenation is
\begin{equation}
    [\ela, \elb, \elc] * [\eld, \ele] = [\ela, \elb, \elc, \eld, \ele].
\end{equation}
For the empty list $[ \ ]$ we also have
\begin{equation}
    [ \ ] * [\ela, \elb, \elc]  = [\ela, \elb, \elc]
\end{equation}
and
\begin{equation}
    [\ela, \elb, \elc]  * [ \ ] = [\ela, \elb, \elc].
\end{equation}

Now we define a multiplication of L-sets.
Given L-sets $\stylesets{[\setA_1, \dots, \setA_m]}$ and $\stylesets{[\setB_1,  \dots, \setB_n]}$ which are both neither the empty L-set nor the unit L-set, we define
\begin{equation}
    \stylesets{[\setA_1, \dots, \setA_m]} * \stylesets{[\setB_1, \dots, \setB_n]} = \stylesets{[\setA_1, \dots, \setA_m, \setB_1,  \dots, \setB_n]}.
\end{equation}
In other words, the elements of $\stylesets{[\setA_1, \dots, \setA_m]} * \stylesets{[\setB_1, \dots, \setB_n]}$ are all possible concatenations of elements of $\stylesets{[\setA_1, \dots, \setA_m]}$ with elements of $\stylesets{[\setB_1, \dots, \setB_m]}$.

If either $\stylesets{[\setA_1, \dots, \setA_m]}$ or $\stylesets{[\setB_1, \dots, \setB_n]}$ is the empty set, we define $\stylesets{[\setA_1, \dots, \setA_m]} * \stylesets{[\setB_1, \dots, \setB_n]}$ to also be the empty set:
\begin{equation}
    \Emptyset * \stylesets{[\setB_1, \dots, \setB_n]} = \Emptyset
\end{equation}
and
\begin{equation}
    \stylesets{[\setA_1, \dots, \setA_m]} * \Emptyset = \Emptyset.
\end{equation}

And for the unit L-set we have the following rules:
\begin{equation}
    \stylesets{[ \ ]} * \stylesets{[\setB_1, \dots, \setB_n]} = \stylesets{[\setB_1,  \dots, \setB_n]}
\end{equation}
and
\begin{equation}
    \stylesets{[\setA_1, \dots, \setA_m]} * \stylesets{[ \ ]} = \stylesets{[\setA_1, \dots, \setA_m]}.
\end{equation}
(Note that these rules for multiplying with the unit L-set and the ones for multiplying with the empty set are compatible with each other in the cases where they overlap).

