% !TEX root = chapter-standalone.tex

\section{Standard action of Moore machines}
\label{sec:Moore-acting-on-sequences}

Given a \SY{Moore machine}~$\mora = \tupp{\prinL_\mora,\prstL_\mora,\proutL_\mora,\prdyn_\mora,\prreadout_\mora,\prstart_\mora}$
we saw that we can use it to transform an infinite sequence of inputs~$\prineln{0}, \prineln{1}, \prineln{2}, \ldots$ into an infinite sequence of outputs~$\prouteln{0}, \prouteln{1}, \prouteln{2}, \ldots$ using the following recipe
\begin{equation}
    \label{eq:Moore-action-recursion-again}
    \begin{cases}
        \prsteln{k+1} = \prdyn_\mora(\prineln{k} \tupconcat \prsteln{k}) \\
        \prouteln{k} = \prreadout_\mora(\prsteln{k}).
    \end{cases}
\end{equation}

Rephrased mathematically, this means that \cref{eq:Moore-action-recursion-again}, together with~$\mora$, defines a function
\begin{equation}\label{eq:a-moore-acting-on-streams}
    \act_{\mora} \colon \streamsof{\prinL_\mora}  \to \streamsof{\proutL_\mora},
\end{equation}
which takes any sequence of elements of~$\prinL_\mora$ and maps it to a corresponding sequence of elements of~$\proutL_\mora$.
We call this the \emph{standard action} of~$\mora$ on sequences.

\begin{remark}
    \label{re:moore-action-two-steps}
    One way to think about the function~$\act_{\mora}$ is to imagine it being calculated in two steps.
    \begin{enumerate}
        \item Given a sequence~$\prinel = \prineln{0}, \prineln{1}, \prineln{2}, \ldots$, there is a unique solution~$\prsolel = \prsoleln{0}, \prsoleln{1}, \prsoleln{2}, \ldots$ of the recursion~$\prsteln{k+1} = \prdyn_\mora(\prineln{k}, \prsteln{k})$ with~$\prsoleln{0} = \prstart_\mora$.
              (Clearly, this solution can be computed iteratively.)
        \item Given the solution~$\prsolel$, the sequence~$\proutel = \act_{\mora}(\prinel)$ is simply~$\proutel = \prsolel \mthen \prreadout_\mora$.
    \end{enumerate}
\end{remark}

\begin{example}
    \label{exa:moore-standard-action}
    Consider a \SY{Moore machine}~$\mora$ with~$\prinL = \prstL = \proutL = \cObj{\natnumbers}$ and let
    \begin{equation}\label{eq:exa-moore-machine-integers-1}
        \defmapset{
            \prdyn_\mora
        }{
            \prinL \cprod \prstL
        }{
            \prstL
        }{
            \tup{m} \tupconcat \tup{n}
        }{
            \tup{m + n}
        }
    \end{equation}
    and
    \begin{equation}\label{eq:exa-moore-machine-integers-2}
        \defmapset{
            \prreadout_\mora
        }{
            \prstL
        }{
            \proutL
        }{
            \tup{n}
        }{
            \tup{n+1}
        }
    \end{equation}
    and $\prstart = \tup{0}$.

    Given a sequence of inputs of the form~$\prinel = \tup{1}, \tup{2}, \tup{3}, \tup{4}, \tup{5}, \ldots$, what are the first five entries of the corresponding sequence of outputs?

    To compute this, let us first calculate the first five entries of the sequence of states~$\prsolel = \prsoleln{0}, \prsoleln{1}, \prsoleln{2}, \ldots$ that solves \cref{eq:Moore-action-recursion-again}.
    We have
    \begin{align}\label{eq:moore-action-exa-calc-1}
        \prsoleln{0} & = \prstart = \tup{0}, \\
        \prsoleln{1} & = \tup{1 + 0} = \tup{1}, \\
        \prsoleln{2} & = \tup{2 + 1} = \tup{3}, \\
        \prsoleln{3} & = \tup{3 + 3} = \tup{6}, \\
        \prsoleln{4} & = \tup{4 + 6} = \tup{10}.
    \end{align}
    Now, applying~$\prreadout_\mora$ to the entries of this sequence of state, we obtain the first five entries of the output sequence:
    \begin{align}\label{eq:moore-action-exa-calc-2}
        \prouteln{0} & = \tup{0 + 1} = \tup{1}, \\
        \prouteln{1} & = \tup{1 + 1} = \tup{2}, \\
        \prouteln{2} & = \tup{3 + 1} = \tup{4}, \\
        \prouteln{3} & = \tup{6 + 1} = \tup{7}, \\
        \prouteln{4} & = \tup{10 + 1} = \tup{11}.
    \end{align}
\end{example}

\begin{gradedexercise}[\exname{ComputingMooreActions}]
    \label{ex:ComputingMooreActions}

    Consider the \SY{Moore machine}~$\mora$ with~$\prinL = \prstL = \proutL = \cObj{\wnumbers}$, $\prstart = \tup{0}$, and
    \begin{equation}
        \defmapset{
            \prdyn_\mora
        }{
            \prinL \cprod \prstL
        }{
            \prstL
        }{
            \tup{m} \tupconcat \tup{n}
        }{
            \tup{n - m},
        }
    \end{equation}
    \begin{equation}
        \defmapset{
            \prreadout_\mora
        }{
            \prstL
        }{
            \proutL
        }{
            \tup{n}
        }{
            \tup{n + 2}.
        }
    \end{equation}

    Given a stream~$\prinel \setin \streamsof \prinL$ of inputs of the form~$\prinel = \tup{1}, \tup{2}, \tup{3}, \tup{4}, \tup{5}, \ldots$, compute the first five entries of the corresponding stream of outputs for the following two actions:
    \begin{enumerate}
        \item The standard action defined via the recursion
              \begin{equation}
                  \begin{cases}
                      \prsteln{k+1} = \prdyn_\mora(\prineln{k} \tupconcat \prsteln{k}) \\
                      \prouteln{k} = \prreadout_\mora(\prsteln{k})
                  \end{cases}
              \end{equation}
              with $\prsteln{0} = \prstart$.
        \item The action defined via the recursion
              \begin{equation}
                  \begin{cases}
                      \prsteln{k+1} = \prdyn_\mora(\prineln{k} \tupconcat \prsteln{k})                                                           \\
                      \tilde{\prstel}_{\styleelements{k+1}} = \prdyn_\mora(\prinel_{\styleelements{k}} \tupconcat \prstel_{\styleelements{k+1}}) \\
                      \prouteln{k}   = \prreadout_\mora(\tilde{\prstel}_{\styleelements{k}})
                  \end{cases}
              \end{equation}
              with $\prsteln{0} = \tilde{\prsteln{0}} = \prstart$.
    \end{enumerate}
\end{gradedexercise}

\solutionof{ComputingMooreActions}

\subsection{Compositionality}

\begin{proposition}
    \label{prop:moore-action-is-a-morphism}
    The standard action of \SY{Moore machines} on signal sequences is compatible with \SY{Moore machine} composition in the sense that
    \begin{equation}
        \label{eq:moore-action-compositional}
        \act_{\morab} = \act_\mora \mthen \act_\morb
    \end{equation}
    for any composable \SY{Moore machines}~$\mora$ and~$\morb$.
\end{proposition}

\begin{proof}
    Suppose we have Moore machines
    \begin{equation}
        \mora = \tupp{\prinL_{\mora},\prstL_{\mora},\proutL_{\mora},\prdyn_{\mora},\prreadout_{\mora},\prstart_{\mora}}
    \end{equation}
    %
    and
    %
    \begin{equation}
        \morb = \tupp{\prinL_{\morb},\prstL_{\morb},\proutL_{\morb},\prdyn_{\morb},\prreadout_{\morb},\prstart_{\morb}},
    \end{equation}
    with~$\proutL_\mora = \prinL_\morb$.

    We will first calculate the right-hand side of \cref{eq:moore-action-compositional}, following the procedure of \cref{re:moore-action-two-steps}.

    Given a sequence~$\prinel \setin \streamsof{\prinL_{\mora}}$, to calculate~$\act_\mora(\prinel)$ we first map~$\prinel$ to the unique solution~$\styleelements{r}_\prinel \setin \streamsof{\prstL_{\mora}}$ of the recursion
    \begin{equation}
        \prsteln{k+1} = \prdyn_\mora(\prineln{k}, \prsteln{k}) \qquad \qquad k \setin \natnumbers,
    \end{equation}
    such that~$\styleelements{r}_\prinel(0) = \prstart_{\mora}$.
    Then~$\act_\mora(\prinel) = \styleelements{r}_\prinel \mthen \prreadout_\mora$, an element of~$\streamsof{\proutL_{\mora}} = \streamsof{\prinL_{\morb}}$.

    Next, to calculate~$(\act_\mora \mthen \act_\morb)(\prinel)$, we first map~$\act_\mora(\prinel)$ to the unique solution~$\prsolel_{\act_\mora(\prinel)} \setin \streamsof{\prstL_{\morb}}$ of the recursion
        \begin{equation}
            \prsteln{k+1} = \prdyn_\morb(\act_\mora(\prineln{k}, \prsteln{k}) \qquad \qquad k \setin \natnumbers
        \end{equation}
        such that~$\prsolel_{\act_\mora(\prinel)}(0) = \prstart_{\morb}$, and then we have
        \begin{equation}
            \label{eq:Moore-act-composition-1}
            (\act_\mora \mthen \act_\morb)(\prinel) = \prsolel_{\act_\mora(\prinel)} \mthen \prreadout_g.
        \end{equation}

        We calculate the left-hand side of \cref{eq:moore-action-compositional}.
        Given~$\prinel \setin \streamsof{\prinL_{\mora}}$, we map it to the unique solution~$\styleelements{q}_\prinel \setin \streamsof{\prstL_{\morab}}  = \streamsof{({\prstL_{\mora} \cprod \prstL_{\morb}})}$ of the recursion
    \begin{equation}
        \label{eq:moore-composite-recursion-equation}
        \prsteln{k+1} = \prdyn_{\morab}(\prineln{k}, \prsteln{k}) \qquad \qquad k \setin \natnumbers
    \end{equation}
    with~$\styleelements{q}_\prinel(0) = \prstart_{\mora} \tupconcat \prstart_{\morb}$.
    Using the definition of~$\prdyn_{\morab}$ and the fact that we can write any~$\prsteln{k} \setin \prstL_{\mora} \cprod \prstL_{\morb}$ as
    \begin{equation}
        \prsteln{k} = \prsteln{k}^\mora \tupconcat \prsteln{k}^\morb \qquad \qquad \text{ with } \prsteln{k}^\mora \setin \prstL_{\mora}, \prsteln{k}^\morb \setin \prstL_{\morb},
    \end{equation}
    we can rewrite \cref{eq:moore-composite-recursion-equation} as
    \begin{equation}
        \prsteln{k+1}^\mora \tupconcat \prsteln{k+1}^\morb = \prdyn_{\mora}(\prineln{k}, \prsteln{k}^\mora) \tupconcat \prdyn_{\morb}(\prreadout_\mora(\prsteln{k}^\mora), \prsteln{k}^\morb) \qquad \qquad k \setin \natnumbers.
    \end{equation}
    Now observe that the sequence~$\styleelements{r}_{\prinel}(k) \tupconcat \prsol_{\act_\mora(\prinel)}(k)$ solves this recursion:
    by definition~$\styleelements{r}_\prinel$ solves the recursion
    \begin{equation}
        \prsteln{k+1} = \prdyn_\mora(\prineln{k}, \prsteln{k}) \qquad \qquad k \setin \natnumbers,
    \end{equation}
    and, recalling that~$\act_\mora(\prinel)_k = \prreadout_\mora(\styleelements{r}_\prinel(k))$, we see that~$\prsolel_{\act_\mora(\prinel)}$ solves the recursion
    \begin{equation}
        \prsteln{k+1} = \prdyn_\morb(\prreadout_\mora(\styleelements{r}_\prinel(k)), \prsteln{k}) \qquad \qquad k \setin \natnumbers.
    \end{equation}
    Since~$\styleelements{r}_\prinel(0) \tupconcat \prsolel_{\act_\mora(\prinel)}(0) = \prstart_{\mora} \tupconcat \prstart_{\morb}$, this implies that the unique solution~$\styleelements{q}_\prinel$ above is precisely~$\styleelements{q}_\prinel(k) = \styleelements{r}_\prinel(k) \tupconcat \prsolel_{\act_\mora(\prinel)}(k)$.

    Finally, we have
    \begin{equation}
        \act_{\morab}(\prinel) = \styleelements{q}_\prinel \mthen \prreadout_{\morab}.
    \end{equation}
    Evaluating at any~$k \setin \natnumbers$ we find
    \begin{align}
        (\styleelements{q}_\prinel \mthen \prreadout_{\morab})(k)
         & = \prreadout_{\morab}(\styleelements{q}_\prinel(k)) \\
         & = \prreadout_{\morab}(\styleelements{r}_\prinel(k) \tupconcat \prsolel_{\act_\mora(\prinel)}(k)) \\
         & = \prreadout_{\morb}(\prsolel_{\act_\mora(\prinel)}(k)) \\
         & = (\prsolel_{\act_\mora(\prinel)} \mthen \prreadout_{\morb})(k).
    \end{align}
    Comparing with \cref{eq:Moore-act-composition-1}, we conclude that
    \begin{equation}
        \act_{\morab}(\prinel) = (\act_\mora \mthen \act_\morb)(\prinel).
    \end{equation}
\end{proof}

\todotext{\bernina: J: Idea: maybe we can model ODE solvers such as the Euler method, \etc, as \SY{Moore machines} and their actions?
    Could be a cool class of examples.
}

%\todotext{J: below is older material, not sure if worth keeping?}
%
%
%Let's now look at how machines like the above act on sequences.
%
%For now we only have defined semi-group, \SY{monoid}, and group actions, and have not talked yet about (semi)category actions.
%Let's consider the set of machines systems with~$\prin = \prout = \prgen$; this is the homset~$\HomSet\Moore\prgen\prgen$.
%
%
%\
%
%\begin{equation}
%    \begin{cases}
%        x_{k+1} = \prdyn(u_k, x_k) \\
%        y_{k}   = \prreadout(x_{k}).
%    \end{cases}
%\end{equation}
%
%\
%
%
%Given a finite input sequence~$u\colon \natnumbers \to \prgen$ of length~$n$, the output is an instantaneous transformation of the state:
%%
%\begin{equation}
%    \label{eq:actions-on-sequences-y}
%    \begin{aligned}
%        y_0   & = \prreadout(\tupset(x_0)), \\
%        y_1   & = \prreadout(\tupset(x_1)), \\
%        y_2   & = \prreadout(\tupset(x_2)), \\
%        \ldots & = \ldots \\
%        y_{k} & = \prreadout(\tupset(x_{k-1})).
%    \end{aligned}
%\end{equation}
%%
%The state is computed recursively as follows:
%\begin{equation}
%    \label{eq:actions-on-sequences-x}
%    \begin{aligned}
%        x_0   & = \prdyn(u_0, \prstart), \\
%        x_1   & = \prdyn(u_1, x_0), \\
%        x_2   & = \prdyn(u_2, x_1), \\
%        \ldots & = \ldots \\
%        x_{k} & = \prdyn(u_{k}, x_{k-1}).
%    \end{aligned}
%\end{equation}
%%
%Therefore, given a machine~$\mora \colon \prgen \mtoin{\Moore} \prgen$ we have defined a map from $\natnumbers \to \prgen$ to itself.
%Let's call it $\act$.
%It is defined as a map of the form
%%
%\begin{equation}
%    \label{eq:actions-on-sequences-1}
%    \act_{\mora}\colon  (\natnumbers \to \prgen)  \sto  (\natnumbers \to \prgen),
%\end{equation}
%%
%or, more formally,
%%
%\begin{equation}
%    \label{eq:actions-on-sequences-2}
%    \act: \HomSet\Moore\prgen\prgen \sto \Endof {\natnumbers \to \prgen}.
%\end{equation}
%%
%Note that both~$\HomSet\Moore\prgen\prgen$ and $\Endof {\natnumbers \to \prgen}$ are semigroups.
%Could it be that~$\act$ is a \SY{semigroup} morphism?
%And, consequently, is~$\act$ a covariant \SY{semigroup} action or a contravariant \SY{semigroup} action?
%
%Let's check the condition for it being a morphism (\cref{eq:sgrp-mor-comp}):
%%
%\begin{align}
%    \act ( \mora \mthenof{\Moore} \morb) & \mathrel{\stackrel{?
%    }{=} } \act ( \mora) \mthenof{\Endof  {\natnumbers \to \prgen}} \act(\morb) \label{eq:actions-on-sequences-left} \\
%    \act ( \morb \mthenof{\Moore} \mora) & \mathrel{\stackrel{?}{=}}  \act ( \mora) \mthenof{\Endof  {\natnumbers \to \prgen}} \act(\morb) \label{eq:actions-on-sequences-right}
%\end{align}
%
%We can check this graphically.
%First, we have
%%
%\begin{equation}
%    \prfperiod{\prftree{\includesag{moore_right_s_f}}{\includesag{moore_right_s_f_conn}}}{\prftree{\includesag{moore_right_b}}{\includesag{moore_right_b_space}}}{\includesag{moore_right_comp_1}}
%\end{equation}
%%
%Second, we have
%%
%\begin{equation}
%    \prfperiod{\prftree{\includesag{moore_right_s}}{\includesag{moore_right_s_space}}}{\prftree{\includesag{moore_right_1}}{\includesag{moore_right_2}}}{\includesag{moore_right_5}}
%\end{equation}
%%
%Therefore, we have a \emph{covariant} \SY{semigroup} action.
%
%\begin{publictodo}
%    The rest of the section is missing.
%\end{publictodo}
