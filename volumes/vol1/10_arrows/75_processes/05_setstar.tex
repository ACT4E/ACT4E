\section{The \SetStar category}
\label{sec:SetStar}

We illustrate a tecnical construction that is going to be helpful in reducing bookkeeping in the next sections.

We are going to define a category similar to \Set, but that has as objects sequences of sets.

Consider the monoid of sequences of sets:
%
\begin{equation}
    \label{eq:sets-monoid}
    \prst_{\mora} \setconcat \prst_{\morb} \setconcat \prst_{\morc} \in \seqsof\Set,
\end{equation}
%

We consider each element of~$\seqsof\Set$ as the name for a set.

The elements of this set are lists of elements.

If~$\ela$ is in~$\setA$ and~$\elb$ is in~$\setB$, the list~$\tupcat \ela \elb$ is in~$\setA \setconcat \setB$:
%
\begin{equation}
    \label{eq:sets-monoid-axiom1}
    \prftree{\ela \in \setA}{\elb \in \setB}{ \tupcat \ela \elb \in (\setA \setconcat \setB) }
\end{equation}
%
Conversely, if~$\ela \in \setA$ and~$\tupcat \ela \elb \in (\setA \setconcat \setB)$, we can conclude that~$\elb \in \setB$:
%
\begin{equation}
    \label{eq:sets-monoid-axiom2}
    \prftree[r]{,}{\ela \in \setA}{ \tupcat \ela \elb \in (\setA \setconcat \setB) }{  \elb \in  \setB }
\end{equation}
% 
and, symmetrically, 
% 
\begin{equation}
    \label{eq:sets-monoid-axiom3}
    \prftree[r]{.}{ \tupcat \ela \elb \in (\setA \setconcat \setB) }{\elb \in \setA}{  \ela \in  \setA }
\end{equation}
% 
Because monoid composition~$\setconcat$ is associative, we have
%
\begin{equation}
    \label{eq:sets-monoid-comp}
    (\prst_{\mora} \setconcat \prst_{\morb} )\setconcat \prst_{\morc} =
    \prst_{\mora} \setconcat \prst_{\morb} \setconcat \prst_{\morc} =
    \prst_{\mora} \setconcat (\prst_{\morb} \setconcat \prst_{\morc}),
\end{equation}
%
Therefore, this works as a cross product that is associative.

The objects of \SetStar are lists of sets.
We can also define morphisms as lists of functions.
We construct them as follows:
%
\begin{equation}
    \label{eq:sets-monoid-maps2}
    \prftree{ \mora \colon \setA \to \setB}{ \morb \colon \setC \to \setD}{ \tupcat \mora \morb \colon (\setA \setconcat \setC)\to (\setB \setconcat \setD)}.
\end{equation}
%
\begin{definition}[The category \SetStar]
    The category \SetStar is defined as follows: 
    \begin{enumerate}
        \item \emph{Objects:} lists of sets.
        \item \emph{Morphisms:} A morphism from $\setA$ to $\setB$ is a list of $n$ functions $\mora_i: \setA_i \mto \setB_i$ such that $\setA = \setA_1 \setconcat \dots \setconcat \setA_n$ and  $\setB = \setB_1 \setconcat \dots \setconcat \setB_n$.
        \item \emph{Composition of morphisms:} Composition is given by function composition.
        \item \emph{Identities:} The identity on an object $\setA_1 \setconcat \dots \setconcat \setA_n$ is given by $\tupca{\catid_{\setA_1}\elconcat \cdots \elconcat \catid_{\setA_n}}$.
    \end{enumerate}
\end{definition}
One of the objects of \SetStar is the empty list of sets, the monoid unit~$\idmon$.
This set contains only one element: the empty tuple.
It is not the empty set.

The empty set can be represented as any list sets of which at least a set is empty, for example 
$\emptyset \setconcat \natnumbers$.

The monoid unit~$\idmon$ is not the only singleton. 
In fact, one may create many singletons by collating zero or more singletons,
for example $
    \singleton \setconcat \singleton 
$.




% \begin{equation}
%     \label{eq:sets-monoid-maps}
%     \prftree{ \mora \colon \setA \to \setC}{ \morb \colon \setB \to \setC}{ \tupcat \mora \morb \colon (\setA\setconcat \setB)\to \setC }.
% \end{equation}
%

\showslides{
    \begin{forslides}
        
        \begin{equation}
            \tupca \ela   = \ela
        \end{equation}
        ~
        \begin{equation}
            \label{eq:sets-monoid-el-cat}
            \tupcat \ela \elb \elconcat \elc = \tupcatt \ela \elb \elc
        \end{equation}
    \end{forslides}
}
