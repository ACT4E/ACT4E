% !TEX root = chapter-standalone.tex

\section{The category $\SetL$}
\label{sec:SetL}

We illustrate a technical construction that is going to be helpful in reducing bookkeeping in the next sections.

We are going to define a category similar to \Set, but that has as objects lists of sets.


\todojira{672}{Modify macros to have squared brackets}
\begin{definition}[The category~$\SetL$]
    \label{def:SetL}
    The category~$\SetL$ is:
    \begin{enumerate}
        \item \emph{Objects:} lists $\Tupcatt {\setA_1}  \dots {\setA_n}$, $n \setin \natnumbers$, of sets.
              This includes the empty list $\Tupca {}$.
        \item \emph{Morphisms:}
              A morphism from~$\setLA =\Tupcatt {\setA_1}  \dots {\setA_n}$ to~$\setLB = \Tupcatt {\setB_1} \dots {\setB_m}$ is a function:
              \begin{equation*}
                  \mapa\colon {\setA_1} \cartprod (\setA_2 \cartprod (\ldots \cartprod {\setA_n})) \sto {\setB_1} \cartprod (\setB_2 \cartprod (\ldots \cartprod {\setB_m}))
              \end{equation*}
        \item \emph{Composition:}
              Composition is the usual composition of functions.
        \item \emph{Identities:}
              The identity morphism on an object~$\Tupcatt {\setA_1} \dots {\setA_n}$ is given by the identity map~$\catid_{{\setA_1} \cartprod (\setA_2 \cartprod (\ldots \cartprod {\setA_n}))}$.
    \end{enumerate}
\end{definition}



\devel{
\subsection{A functor from $\SetL$ to \Set}
\label{sec:funsetstarset}

We now define a functor~$\tupset\colon \SetL \fto \Set$:
\begin{itemize}
    \item On objects, one has
          \begin{equation*}
              \defmapperiod{\funob{\tupset}}{\SetL}{\fto}{\Set}{\Tupcatt {\setA_1}  \dots {\setA_n}}{{\setA_1}\cartprod (\ldots \cartprod {\setA_n})}
          \end{equation*}
    \item On morphisms, one has
          \begin{equation*}
              \defmapcomma{\funmor{\tupset}}{\HomSet{\SetL}{\Tupcatt{\setA_1}\dots{\setA_n}}{\Tupcatt{\setB_1} \dots {\setB_m}}}
              {\fto}{\HomSet{\Set}{\setA}{\setB}}{\mapa}{\mapa}
          \end{equation*}
          where~$\setA={\setA_1} \cartprod (\ldots \cartprod {\setA_n})$ and~$\setB={\setB_1} \cartprod (\ldots \cartprod {\setB_m})$.
\end{itemize}
We can now check that this is indeed a functor.
Unitality follows from the definition of~$\funmor{\tupset}$ and the definition of~$\SetL$.
Furthermore, given
\begin{equation*}
    \begin{aligned}
         & \mora \setin \HomSet{\SetL}{\Tupcatt{\setA_1}\dots{\setA_n}}{\Tupcatt{\setB_1} \dots {\setB_m}} \\
         & \morb \setin \HomSet{\SetL}{\Tupcatt{\setB_1}\dots{\setB_m}}{\Tupcatt{\setC_1}\dots {\setC_o}},
    \end{aligned}
\end{equation*}
one has
\begin{equation*}
    \begin{aligned}
        \funmor{\tupset}(\mora \mthenof{\SetL} \morb) & =\mora\mthenof{\SetL} \morb \\
                                                      & =\funmor{\tupset}(\mora)\mthenof{\Set} \funmor{\tupset}(\morb).
    \end{aligned}
\end{equation*}

\showslides{
    \begin{forslides}

        \begin{equation}
            \tupca \ela   = \ela
        \end{equation}
        ~
        \begin{equation}
            \label{eq:sets-monoid-el-cat}
            \tupcat \ela \elb \elconcat \elc = \tupcatt \ela \elb \elc
        \end{equation}
    \end{forslides}
}

Associated to the functor~$\tupset$, one has a morphism which we call \emph{coherence morphism}.
This is a morphism~$\cohm\colon \tupset(\setLA\listconcat \setLB)\mto \tupset(\setLA)\cartprod \tupset(\setLB)$.

\todo{Explain without explaining monoidal functor...}
}