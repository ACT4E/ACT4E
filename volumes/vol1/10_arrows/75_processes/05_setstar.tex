\section{The \SetStar category}
\label{sec:SetStar}

We illustrate a technical construction that is going to be helpful in reducing bookkeeping in the next sections.

We are going to define a category similar to \Set, but that has as objects sequences of sets.

Consider the monoid of sequences of sets:
%
\begin{equation}
    \label{eq:sets-monoid}
    \Tupcatt \setA  \setB  \setC \in \seqsof\Set.
\end{equation}
%
We consider each element of $\seqsof\Set$ as the name for a set.
The elements of this set are lists of elements.

If~$\ela$ is in~$\setA$ and~$\elb$ is in~$\setB$, the list~$\tupcat \ela \elb$ is in~$\Tupcat \setA \setB$:
%
\begin{equation}
    \label{eq:sets-monoid-axiom1}
    \prftree[r]{.}{
        \ela \in \setA
    }{
        \elb \in \setB
    }{ 
        \tupcat \ela \elb \in \Tupcat \setA \setB 
    }
\end{equation}
%
Conversely, if $\ela \in \setA$ and $\tupcat \ela \elb \in (\setA \setconcat \setB)$, we can conclude that $\elb \in \setB$:
%
\begin{equation}
    \label{eq:sets-monoid-axiom2}
    \prftree[r]{,}{
        \ela \in \setA
    }{ 
        \tupcat \ela \elb \in \Tupcat \setA  \setB
    }{
        \phantom{\elb \in \setB}
    }{  
        \rule{39mm}{0pt} \elb \in  \setB 
    }
\end{equation}
% 
and, symmetrically, 
% 
\begin{equation}
    \label{eq:sets-monoid-axiom3}
    \prftree[r]{,}{
        \phantom{\ela \in \setA}
    }{
        \tupcat \ela \elb \in \Tupcat \setA \setB  
    }{
        \elb \in \setB
    }{  
        \ela \in  \setA \rule{39mm}{0pt}
    }
\end{equation}
%
or, more compactly,
% 
\begin{equation}
    \label{eq:sets-monoid-axiom3bis}
    \prftree[r]{.}{
        \tupcat \ela \elb \in \Tupcat \setA \setB  
    }{
        \prftree[double]{
            \ela \in \setA
        }{
            \elb \in \setB
        }
    }
\end{equation}
%
Because monoid composition $\setconcat$ is associative, we have
%
\begin{equation}
    \label{eq:sets-monoid-comp}
    \Tupcat {\Tupcat \setA  \setB}  \setC  =
    \Tupcatt \setA    \setB \setC  =
    \Tupcat   \setA  {\Tupcat \setB  \setC}.
\end{equation}
%
Therefore, this works as a cross product that is associative.

The objects of \SetStar are lists of sets.
We can also define morphisms as lists of functions.
We construct them as follows:
%
\begin{equation}
    \label{eq:sets-monoid-maps2}
    \prftree[r]{.}{ 
        \mora \colon \setA \to \setB
    }{ 
        \morb \colon \setC \to \setD
    }{ 
        \tupmorcat \mora \morb \colon \Tupcat \setA   \setC \to \Tupcat \setB  \setD
    }
\end{equation}
%

One of the objects of \SetStar is the empty list of sets, the monoid unit $\idmon$.
This set contains only one element: the empty tuple.
It is not the empty set.

The empty set can be represented as any list sets of which at least a set is empty.


% \begin{equation}
%     \label{eq:sets-monoid-maps}
%     \prftree{ \mora \colon \setA \to \setC}{ \morb \colon \setB \to \setC}{ \tupcat \mora \morb \colon (\setA\setconcat \setB)\to \setC }.
% \end{equation}
%

\showslides{
    \begin{forslides}
        
        \begin{equation}
            \tupca \ela   = \ela
        \end{equation}
        ~
        \begin{equation}
            \label{eq:sets-monoid-el-cat}
            \tupcat \ela \elb \elconcat \elc = \tupcatt \ela \elb \elc
        \end{equation}
    \end{forslides}
}
%
%
