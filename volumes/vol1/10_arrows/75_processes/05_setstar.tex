% !TEX root = chapter-standalone.tex

\section{The category $\SetL$}
\label{sec:SetL}

In the following we present a technical construction which will allow us to define a semicategory of Moore machines after all, and will be of general interest later.

We define a category similar to \Set, but its objects are \emph{lists} of sets.

\begin{definition}[The category~$\SetL$]
    \label{def:SetL}
    The category~$\SetL$ is:
    \begin{enumerate}
        \item \emph{Objects:} lists~$\Tupcatt {\setA_1}  \dots {\setA_n}$,~$n \setin \natnumbers$, of sets.
              This includes the empty list~$\Tupca {}$.
        \item \emph{Morphisms:}
              A morphism from~$\setLA =\Tupcatt {\setA_1}  \dots {\setA_n}$ to~$\setLB = \Tupcatt {\setB_1} \dots {\setB_m}$ is a function:
              \begin{equation*}
                  \mapa\colon {\setA_1} \cartprod (\setA_2 \cartprod (\ldots \cartprod {\setA_n})) \sto {\setB_1} \cartprod (\setB_2 \cartprod (\ldots \cartprod {\setB_m})).
              \end{equation*}
        \item \emph{Composition:}
              Composition is the usual composition of functions.
        \item \emph{Identities:}
              The identity morphism on an object~$\Tupcatt {\setA_1} \dots {\setA_n}$ is given by the identity map~$\catid_{{\setA_1} \cartprod (\setA_2 \cartprod (\ldots \cartprod {\setA_n}))}$.
    \end{enumerate}
\end{definition}

As an auxiliary tool, let us define a function $\tupset$ that, given a list of sets $\Tupcatt {\setA_1}  \dots {\setA_n}$, turns it into a set via the following recipe:
\begin{equation*}
    \tupset (\Tupcatt {\setA_1}  \dots {\setA_n}) \definedas {\setA_1} \cartprod (\setA_2 \cartprod (\ldots \cartprod {\setA_n})).
\end{equation*}
For the special case of the empty list, we choose a one-element set~$\{*\}$ and define
\begin{equation*}
    \tupset (\makelist{}) \definedas \{*\}.
\end{equation*}
Incorporating~$\tupset$ in our notation, a morphism~$\setLA \mto \setLB$ in~$\SetL$ is, by definition, a function~$\tupset(\setLA) \mto \tupset(\setLB)$.

We also equip the category~$\SetL$ with an operation of \emph{concatenation} of lists.
Given lists~$\setLA =\Tupcatt {\setA_1}  \dots {\setA_n}$ and~$\setLB = \Tupcatt {\setB_1} \dots {\setB_m}$, their concatenation is
\begin{equation*}
    \setLA  \listconcat  \setLB \definedas \Tupcattt {\setA_1}{\dots}{\setA_n}{\setB_1}{\dots}{\setB_m}.
\end{equation*}
Empty lists act neutrally for concatenation.

The operation~$\tupset$ of turning lists into sets works coherently with concatenation, in that there is a `coherence' isomorphism
\begin{equation}
    \label{def:coherence-iso-lists}
    \cohm_{\setLA, \setLB} \colon \tupset( \setLA) \cartprod \tupset(\setLB) \to \tupset(\setLA  \listconcat  \setLB);
\end{equation}
namely,~$\cohm_{\setLA, \setLB}$ denotes for us the function
\begin{equation*}
    ({\setA_1} \cartprod (\setA_2 \cartprod (\ldots \cartprod {\setA_n}))) \cartprod ({\setB_1} \cartprod (\setB_2 \cartprod (\ldots \cartprod {\setB_m}))) \to  {\setA_1} \cartprod (\setA_2 \cartprod (\ldots \setB_{m-1} \cartprod {\setB_m}))
\end{equation*}
that reorders the bracketing of elements so that they are all nested `to the right'.

For example, if~$\setLA = \makelist{\setA_1, \setA_2, \setA_3}$ and~$\setLB = \makelist{\setB_1, \setB_2}$, then
\begin{equation*}
    \defmapperiodset{ \cohm_{\setLA, \setLB}
    }{
        (\setA_1 \cartprod ( \setA_2 \cartprod  \setA_3)) \cartprod (\setB_1 \cartprod  \setB_2 )
    }{
        \setA_1 \cartprod ( \setA_2 \cartprod ( \setA_3 \cartprod (\setB_1 \cartprod  \setB_2 )))
    }{
        \tup{\tup{\ela_1, \tup{\ela_2, \ela_3}}, \tup{\elb_1, \elb_2}}
    }{
        \tup{\ela_1, \tup{\ela_2, \tup{\ela_3, \tup{\elb_1, \elb_2}}}}
    }
\end{equation*}
(Can you convince yourself that this function is indeed invertible?
)

When the empty list is involved, we also have a special rule for \cref{def:coherence-iso-lists}.
If~$\setLA = \makelist{}$, we set
\begin{equation*}
    \defmapperiod{\cohm_{\setLA, \setLB}}{\{*\} \cartprod \tupset(\setLB)}{\to}{\tupset(\setLA  \listconcat  \setLB) = \tupset(\setLB)}{\tup{*, \elb}}{\elb}
\end{equation*}

An analogous isomorphism is used if instead~$\setLB = \makelist{}$, and the two cases coincide if~$\setLA = \setLB = \makelist{}$.

The following will be needed later.

\begin{lemma}\label{lem:coh-isos-compatible}
    For any~$\setLA$,~$\setLB$,~$\setLC$ in~$\SetL$, the following diagram
    \begin{center}
        \middlesag{coh-isos-compat-lemma}
    \end{center}
    commutes, where~$\alpha$ is the canonical isomorphism that re-brackets tuples in the obvious way.
\end{lemma}

\devel{
    \begin{proof}
        We look at the two sides of the commuting diagram.
        On one hand one has:
        \begin{equation*}
            \begin{aligned}
                ((\cohm_{\setLA, \setLB} \times \catid_{ \tupset(\setLC) }) \mthen \cohm_{\setLA \listconcat \setLB, \setLC})(\tup{\tup{\ela,\elb},\elc}) & =
                \cohm_{\setLA \listconcat \setLB, \setLC}((\tup{\tup{\ela,\elb},\elc})) \\
                                                                                                                                                          & =\tup{\ela, \tup{\elb,\elc}}.
            \end{aligned}
        \end{equation*}
        On the other hand, one has
        \begin{equation*}
            \begin{aligned}
                 & (\alpha\mthen (\catid_{ \tupset(\setLA) } \times  \cohm_{\setLB, \setLC}) \mthen \cohm_{\setLA, \setLB  \listconcat\setLC})(\tup{\tup{\ela,\elb},\elc}) \\
                 & =
                ((\catid_{\tupset(\setLA)} \times  \cohm_{\setLB, \setLC})\mthen \cohm_{\setLA, \setLB  \listconcat \setLC})(\tup{\ela,\tup{\elb,\elc}}) \\
                 & =\cohm_{\setLA, \setLB  \listconcat\setLC}(\tup{\ela, \tup{\elb,\elc}}) \\
                 & =\tup{\ela, \tup{\elb,\elc}}.
            \end{aligned}
        \end{equation*}
    \end{proof}

}

\todotext{prove the lemma}

\devel{
    \subsection{A functor from $\SetL$ to \Set}
    \label{sec:funsetstarset}

    We now define a functor~$\tupset \colon \SetL \fto \Set$:
    \begin{itemize}
        \item On objects, one has
              \begin{equation*}
                  \defmapperiod{\funob{\tupset}}{\SetL}{\fto}{\Set}{\Tupcatt {\setA_1}  \dots {\setA_n}}{{\setA_1}\cartprod (\ldots \cartprod {\setA_n})}
              \end{equation*}
        \item On morphisms, one has
              \begin{equation*}
                  \defmapcomma{\funmor{\tupset}}{\HomSet{\SetL}{\Tupcatt{\setA_1}\dots{\setA_n}}{\Tupcatt{\setB_1} \dots {\setB_m}}}
                  {\fto}{\HomSet{\Set}{\setA}{\setB}}{\mapa}{\mapa}
              \end{equation*}
              where~$\setA={\setA_1} \cartprod (\ldots \cartprod {\setA_n})$ and~$\setB={\setB_1} \cartprod (\ldots \cartprod {\setB_m})$.
    \end{itemize}
    We can now check that this is indeed a functor.
    Unitality follows from the definition of~$\funmor{\tupset}$ and the definition of~$\SetL$.
    Furthermore, given
    \begin{equation*}
        \begin{aligned}
             & \mora \setin \HomSet{\SetL}{\Tupcatt{\setA_1}\dots{\setA_n}}{\Tupcatt{\setB_1} \dots {\setB_m}} \\
             & \morb \setin \HomSet{\SetL}{\Tupcatt{\setB_1}\dots{\setB_m}}{\Tupcatt{\setC_1}\dots {\setC_o}},
        \end{aligned}
    \end{equation*}
    one has
    \begin{equation*}
        \begin{aligned}
            \funmor{\tupset}(\mora \mthenof{\SetL} \morb) & =\mora\mthenof{\SetL} \morb \\
                                                          & =\funmor{\tupset}(\mora)\mthenof{\Set} \funmor{\tupset}(\morb).
        \end{aligned}
    \end{equation*}

    \showslides{
        \begin{forslides}

            \begin{equation}
                \tupca \ela   = \ela
            \end{equation}
            ~
            \begin{equation}
                \label{eq:sets-monoid-el-cat}
                \tupcat \ela \elb \elconcat \elc = \tupcatt \ela \elb \elc
            \end{equation}
        \end{forslides}
    }

}


\section{The category $\Cat{LSet}$}

In the following we will define a category whose objects are, loosely speaking, ``sets of typed lists''. We will call these L-sets. 

\subsection{Lists}

Before defining L-sets, let's first recall our notion of lists. Given three things, for example (call them $\ela$, $\elb$, and $\elc$) we can list them in some order: 
\begin{equation}
[\elb, \elc, \ela].
\end{equation}
The square brackets are our notation for lists, and recall that the ordering matters for lists (for instance, $[\elc, \ela, \elb]$ is a different list of those same three things). 

Any list has a length; for example $[\elb, \elc, \ela]$ has length 3. For us, by definition, all lists have finite length. We also define such a thing as an \emph{empty list}, which we denote by $[ \ ]$. It has length zero. 

The things in a list -- for example, $\elb$, $\elc$, and $\ela$ -- are the \emph{entries} of the list. They each come with an index corresponding to their placement in the list (so we can speak of the 1st entry, 2nd entry, and so on).  


\subsection{L-sets}

Suppose now that we are given some sets $\setA$, $\setB$ and $\setC$.  We will denote by $\stylesets{[\setA, \setB, \setC]}$ the set of all lists of the form $[\ela, \elb, \elc]$, where $\ela \in \setA$, $\elb \in \setB$, and $\elc \in \setC$. In other words
\begin{equation}
\stylesets{[\setA, \setB, \setC]} = \{ [\ela, \elb, \elc] \mid \ela \in \setA, \elb \in \setB, \elc \in \setC \}.
\end{equation}
Beside some details that we will address presently, this essentially captures what we mean L-sets to be: sets of lists where the entries of the lists are each specified to be an element of a certain set. 

Now the details. In our definition of L-sets, we will also include the empty set $\Emptyset$ as an L-set, and also a special one-element set $\stylesets{[ \ ]} := \{ [ \ ] \}$ whose single element is the empty list. We'll call the latter the unit L-set. 

Given sets $\setA_1, \setA_2, \dots , \setA_n$, if they are all non-empty, then we have this L-set
\begin{equation}
\stylesets{[\setA_1, \setA_2, \dots, \setA_n]} := \{ [ \ela_1, \ela_2, \dots, \ela_n] \mid \ela_1 \in \setA_1, \ela_2 \in \setA_2, \dots, \ela_n \in \setA_n \},
\end{equation}
of typed lists. 
If however one or more of the sets $\setA_1, \setA_2, \dots , \setA_n$ are equal to the empty set, then \begin{equation}
\stylesets{[\setA_1, \setA_2, \dots, \setA_n]} := \Emptyset. 
\end{equation}


\subsection{The category $\Cat{LSet}$}

\todotext{write this out. Objects are L-sets, Morphisms are functions}

\subsection{Multiplication of L-sets}

Given lists $[\ela, \elb, \elc]$ and $[\eld, \ele]$ we can stick them together to make a longer list: $[\ela, \elb, \elc, \eld, \ele]$. We call this operation concatenation, and in the context of L-sets we will think of it as a kind of multiplication of lists. Our notation for concatenation is
\begin{equation}
[\ela, \elb, \elc] * [\eld, \ele] = [\ela, \elb, \elc, \eld, \ele].
\end{equation}
For the empty list $[ \ ]$ we also have
\begin{equation}
[ \ ] * [\ela, \elb, \elc]  = [\ela, \elb, \elc] 
\end{equation}
and 
\begin{equation}
[\ela, \elb, \elc]  * [ \ ] = [\ela, \elb, \elc]. 
\end{equation}



Now we define a multiplication of L-sets. Given L-sets $\stylesets{[\setA_1, \dots, \setA_m]}$ and $\stylesets{[\setB_1,  \dots, \setB_n]}$ which are both neither the empty L-set nor the unit L-set, we define
\begin{equation}
\stylesets{[\setA_1, \dots, \setA_m]} * \stylesets{[\setB_1, \dots, \setB_n]} = \stylesets{[\setA_1, \dots, \setA_m, \setB_1,  \dots, \setB_n]}.
\end{equation}
In other words, the elements of $\stylesets{[\setA_1, \dots, \setA_m]} * \stylesets{[\setB_1, \dots, \setB_n]}$ are all possible concatenations of elements of $\stylesets{[\setA_1, \dots, \setA_m]}$ with elements of $\stylesets{[\setB_1, \dots, \setB_m]}$. 

If either $\stylesets{[\setA_1, \dots, \setA_m]}$ or $\stylesets{[\setB_1, \dots, \setB_n]}$ is the empty set, we define $\stylesets{[\setA_1, \dots, \setA_m]} * \stylesets{[\setB_1, \dots, \setB_n]}$ to also be the empty set:
\begin{equation}
\Emptyset * \stylesets{[\setB_1, \dots, \setB_n]} = \Emptyset
\end{equation}
and 
\begin{equation}
\stylesets{[\setA_1, \dots, \setA_m]} * \Emptyset = \Emptyset.
\end{equation}


And for the unit L-set we have the following rules: 
\begin{equation}
\stylesets{[ \ ]} * \stylesets{[\setB_1, \dots, \setB_n]} = \stylesets{[\setB_1,  \dots, \setB_n]}
\end{equation}
and 
\begin{equation}
\stylesets{[\setA_1, \dots, \setA_m]} * \stylesets{[ \ ]} = \stylesets{[\setA_1, \dots, \setA_m]}.
\end{equation}
(Note that these rules for multiplying with the unit L-set and the ones for multiplying with the empty set are compatible with each other in the cases where they overlap). 










