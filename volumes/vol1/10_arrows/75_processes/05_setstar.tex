% !TEX root = chapter-standalone.tex

\section{The category $\SetL$}
\label{sec:SetL}

We illustrate a technical construction that is going to be helpful in reducing bookkeeping in the next sections.

We are going to define a category similar to \Set, but that has as objects lists of sets.

\devel{
    \todojira{645}{re-write this part considering the updated definition.}
    Consider the monoid of sequences of sets:
    %
    \begin{equation}
        \label{eq:sets-monoid}
        \Tupcatt \setA  \setB  \setC \setin \seqsof\Set.
    \end{equation}
    %
    We consider each element of~$\seqsof\Set$ as the name for a set.

    The elements of this set are \emph{tuples} of elements.

    If~$\ela$ is in~$\setA$ and~$\elb$ is in~$\setB$, the tuple~$\tupcat \ela \elb$ is in~$\Tupcat \setA \setB$:
    %
    \begin{equation}
        \label{eq:sets-monoid-axiom1}
        \prfperiod{
            \ela \setin \setA
        }{
            \elb \setin \setB
        }{
            \tupcat \ela \elb \setin \Tupcat \setA \setB
        }
    \end{equation}
    %
    Conversely, if~$\ela \setin \setA$ and~$\tupcat \ela \elb \setin \Tupcat \setA \setB$, we can conclude that~$\elb \setin \setB$:
    %
    \begin{equation}
        \label{eq:sets-monoid-axiom2}
        \prfcomma{
            \ela \setin \setA
        }{
            \tupcat \ela \elb \setin \Tupcat \setA  \setB
        }{
            \phantom{\elb \setin \setB}
        }{
            \rule{39mm}{0pt} \elb \setin  \setB
        }
    \end{equation}
    %
    and, symmetrically,
    %
    \begin{equation}
        \label{eq:sets-monoid-axiom3}
        \prfcomma{
            \phantom{\ela \setin \setA}
        }{
            \tupcat \ela \elb \setin \Tupcat \setA \setB
        }{
            \elb \setin \setB
        }{
            \ela \setin  \setA \rule{39mm}{0pt}
        }
    \end{equation}
    %
    or, more compactly,
    %
    \begin{equation}
        \label{eq:sets-monoid-axiom3bis}
        \prfperiod{
            \tupcat \ela \elb \setin \Tupcat \setA \setB
        }{
            \prfdouble{
                \ela \setin \setA
            }{
                \elb \setin \setB
            }
        }
    \end{equation}
    %
    % Because monoid composition~$\setconcat$ is associative, we have
    % %
    % \begin{equation}
    %     \label{eq:sets-monoid-comp}
    %     \Tupcat {\Tupcat \setA  \setB}  \setC  =
    %     \Tupcatt \setA    \setB \setC  =
    %     \Tupcat   \setA  {\Tupcat \setB  \setC}.
    % \end{equation}
    % %
    % Therefore, this works as a cross product that is associative.
}
The objects of $\SetL$ are lists of sets.
%We can also define morphisms as lists of functions.
%We construct them as follows:
%
%\begin{equation}
%    \label{eq:sets-monoid-maps2}
%    \prfperiod{
%        \mora \colon \setA \to \setB
%    }{
%        \morb \colon \setC \to \setD
%    }{
%        \tupmorcat \mora \morb \colon \Tupcat \setA   \setC \to \Tupcat \setB  \setD
%    }
%\end{equation}
%
%\begin{definition}[The category \SetL]
%    \label{def:SetL}
%    The category \SetL is defined as follows:
%    \begin{enumerate}
%        \item \emph{Objects:} lists of sets.
%        \item \emph{Morphisms:}
%              A morphism from~$\setA$ to~$\setB$ is a list of~$n$ functions~$\mora_i\colon \setA_i \mto \setB_i$ such that~$\setA = \Tupcatt {\setA_1}  \dots {\setA_n}$ and~$\setB = \Tupcatt {\setB_1} \dots {\setB_m}$.
%        \item \emph{Composition of morphisms:}
%              Composition is given by function composition.
%        \item \emph{Identities:}
%              The identity on an object~$\Tupcatt {\setA_1} \dots {\setA_n}$ is given by~$\tupmorcatt {\catid_{\setA_1}} \dots  {\catid_{\setA_n}} $.
%    \end{enumerate}
%\end{definition}

\todojira{672}{Modify macros to have squared brackets}
\begin{definition}[The category~$\SetL$]
    \label{def:SetL}
    The category~$\SetL$ is:
    \begin{enumerate}
        \item \emph{Objects:} lists $\Tupcatt {\setA_1}  \dots {\setA_n}$, $n \setin \natnumbers$, of sets.
              This includes the empty list $\Tupca {}$.
        \item \emph{Morphisms:}
              A morphism from~$\setLA =\Tupcatt {\setA_1}  \dots {\setA_n}$ to~$\setLB = \Tupcatt {\setB_1} \dots {\setB_m}$ is a function:
              \begin{equation*}
                  \mapa\colon {\setA_1} \cartprod (\setA_2 \cartprod (\ldots \cartprod {\setA_n})) \sto {\setB_1} \cartprod (\setB_2 \cartprod (\ldots \cartprod {\setB_m}))
              \end{equation*}
        \item \emph{Composition:}
              Composition is the usual composition of functions.
        \item \emph{Identities:}
              The identity morphism on an object~$\Tupcatt {\setA_1} \dots {\setA_n}$ is given by the identity map~$\catid_{{\setA_1} \cartprod (\setA_2 \cartprod (\ldots \cartprod {\setA_n}))}$.
    \end{enumerate}
\end{definition}

\todotextjira{396}{@Andrea: later, mention that $\SetL$ is a subcategory of \Set.}

One of the objects of $\SetL$ is the empty list of sets, the monoid unit~$\Tupca{ }$.
This set contains only one element: the empty tuple~$\tupca{ }.
$
It is not the empty set.

The empty set can be represented as any list sets of which at least a set is empty, for example~$\Tupcat \emptyset \natnumbers$.

The monoid unit~$\Tupca{ }$ is not the only singleton.
In fact, one may create many singletons by collating zero or more singletons, for example~$\Tupcat \singleton \singleton$.

% \begin{equation}
%     \label{eq:sets-monoid-maps}
%     \prftree{ \mora \colon \setA \to \setC}{ \morb \colon \setB \to \setC}{ \tupcat \mora \morb \colon (\setA\setconcat \setB)\to \setC }.
% \end{equation}
%

\section{A functor from $\SetL$ to \Set}
\label{sec:funsetstarset}
\todojira{619}{J: Explain strictification argument}
We now define a functor~$\tupset\colon \SetL \fto \Set$:
\begin{itemize}
    \item On objects, one has
          \begin{equation*}
              \defmapperiod{\funob{\tupset}}{\SetL}{\fto}{\Set}{\Tupcatt {\setA_1}  \dots {\setA_n}}{{\setA_1}\cartprod (\ldots \cartprod {\setA_n})}
          \end{equation*}
    \item On morphisms, one has
          \begin{equation*}
              \defmapcomma{\funmor{\tupset}}{\HomSet{\SetL}{\Tupcatt{\setA_1}\dots{\setA_n}}{\Tupcatt{\setB_1} \dots {\setB_m}}}
              {\fto}{\HomSet{\Set}{\setA}{\setB}}{\mapa}{\mapa}
          \end{equation*}
          where~$\setA={\setA_1} \cartprod (\ldots \cartprod {\setA_n})$ and~$\setB={\setB_1} \cartprod (\ldots \cartprod {\setB_m})$.
\end{itemize}
We can now check that this is indeed a functor.
Unitality follows from the definition of~$\funmor{\tupset}$ and the definition of~$\SetL$.
Furthermore, given
\begin{equation*}
    \begin{aligned}
         & \mora \setin \HomSet{\SetL}{\Tupcatt{\setA_1}\dots{\setA_n}}{\Tupcatt{\setB_1} \dots {\setB_m}} \\
         & \morb \setin \HomSet{\SetL}{\Tupcatt{\setB_1}\dots{\setB_m}}{\Tupcatt{\setC_1}\dots {\setC_o}},
    \end{aligned}
\end{equation*}
one has
\begin{equation*}
    \begin{aligned}
        \funmor{\tupset}(\mora \mthenof{\SetL} \morb) & =\mora\mthenof{\SetL} \morb \\
                                                      & =\funmor{\tupset}(\mora)\mthenof{\Set} \funmor{\tupset}(\morb).
    \end{aligned}
\end{equation*}

\showslides{
    \begin{forslides}

        \begin{equation}
            \tupca \ela   = \ela
        \end{equation}
        ~
        \begin{equation}
            \label{eq:sets-monoid-el-cat}
            \tupcat \ela \elb \elconcat \elc = \tupcatt \ela \elb \elc
        \end{equation}
    \end{forslides}
}

Associated to the functor~$\tupset$, one has a morphism which we call \emph{coherence morphism}.
This is a morphism~$\cohm\colon \tupset(\setLA\listconcat \setLB)\mto \tupset(\setLA)\cartprod \tupset(\setLB)$.

\todo{Explain without explaining monoidal functor...}