% !TEX root = chapter-standalone.tex

\section{Moore machines, first attempt}
\label{sec:moore-machines}


%\linkvideo{spring2021-actions:semi-actions:processes} % Signals and processes

    \linkvideo{spring2021-actions:semi-actions:processes:moore} % Moore machines
    We now look at processes, especially dynamical systems, and see them as (semi)\-categories that act on sequences.

    We have already seen linear discrete time systems.

    We can generalize them by allowing non-linear functions.
    We call these \textbf{Moore} machines, and describe them as a pair of functions
    %
    \begin{equation}
        \label{eq:moore-1}
        \begin{cases}
            \prdyn \colon \prin \cartprod \prst \sto \prst, \\
            \prreadout \colon \prst \sto \prout,
        \end{cases}
    \end{equation}
    %
    where~$\prin$ represents inputs,~$\prst$ states,~$\prout$ outputs,~$\prdyn$ the dynamics, and~$\prreadout$ the readout.
    As introduced in \cref{sec:actions}, we can apply currying to~$\prdyn$, to obtain a map from inputs to endomorphisms on the states:
    %
    \begin{equation}
        \label{eq:moore-1-endo}
        \begin{cases}
            \prdyn \colon \prin \sto \Endof \prst, \\
            \prreadout \colon \prst \sto \prout.
        \end{cases}
    \end{equation}
    %

    We also need to have an~$\prstart \setin \prst$ to act as the initial state.

    Suppose we have two morphisms
    %
    \begin{equation}
        \label{eq:moore-mora}
        \mora = \tupp{\prin_{\mora},\prst_{\mora},\prout_{\mora},\prdyn_{\mora},\prreadout_{\mora},\prstart_{\mora}}
    \end{equation}
    %
    and
    %
    \begin{equation}
        \label{eq:moore-morb}
        \morb = \tupp{\prin_{\morb},\prst_{\morb},\prout_{\morb},\prdyn_{\morb},\prreadout_{\morb},\prstart_{\morb}},
    \end{equation}
    such that~$\prout_{\mora} = \prin_{\morb}$.

    The composition of these two systems should have a joint state that is the product of the states.

    \begin{marginfigure}
        \centering
        \includesag{10_moore_comp_seq}
        \caption{Composition of Moore machines (first version).}
        \label{fig:comp_moore_1}
    \end{marginfigure}

    Here is one way to do it.
    We specify the spaces:
    %
    \begin{equation}
        \label{eq:moore-comp-naive-1}
        \begin{aligned}
            \prin_{\mora\mthen\morb}    & = \prin_{\mora}, \\
            \prst_{\mora\mthen\morb}    & = \prst_{\mora} \cartprod \prst_{\morb}, \\
            \prstart_{\mora\mthen\morb} & = \tupp{\prstart_{\mora}, \prstart_{\morb}}, \\
            \prout_{\mora\mthen\morb}   & = \prout_{\morb}.
        \end{aligned}
    \end{equation}
    %
    Furthermore, we specify the dynamics
    %
    \begin{equation}
        \label{eq:moore-comp-naive-2}
        \defmapset{
            \prdyn_{\mora\mthen\morb}
        }{
            \prin_{\mora} \cartprod (\prst_{\mora} \cartprod \prst_{\morb})
        }{
            (\prst_{\mora} \cartprod \prst_{\morb})
        }{
            \tupp{u, \tupp{x_{\mora}, x_{\morb}}}
        }{
            \tupp{ \prdyn_{\mora} (u, x_{\mora}), \prdyn_{\morb}(\prreadout_{\mora}(x_{\mora}), x_{\morb})}
        }
    \end{equation}
    %
    and the ``readout'':
    %
    \begin{equation}
        \label{eq:moore-comp-naive-3}
        \defmapperiodset{
            \prreadout_{\mora\mthen\morb}
        }{
            (\prst_{\mora} \cartprod \prst_{\morb})
        }{
            \prout_{\morb}
        }{
            \tupp{x_{\mora}, x_{\morb}}
        }{
            \prreadout_{\morb}(x_{\morb})
        }
    \end{equation}
    %
    We represent the composition graphically as in \cref{fig:comp_moore_1}.

    However, if we define these using the Cartesian product~$\prst_{\mora} \cartprod \prst_{\morb}$, we cannot compose the systems in an associative way.
    When we have three systems, composing in the two ways would bring to~$(\prst_{\mora} \cartprod \prst_{\morb}) \cartprod \prst_{\morc}$ and~$\prst_{\mora} \cartprod (\prst_{\morb} \cartprod \prst_{\morc})$, which are \emph{isomorphic} sets but not equal:
    %
    \begin{equation}
        \label{eq:assoc-fails1}
        (\prst_{\mora} \cartprod \prst_{\morb}) \cartprod \prst_{\morc} \neq \prst_{\mora} \cartprod (\prst_{\morb} \cartprod \prst_{\morc})
    \end{equation}
    %
    Elements of these sets are of the form~$\tup{\tup{a,b},c}$ and~$\tup{a, \tup{b,c}}$.
    %
    You can clearly spot the isomorphism:
    %
    \begin{equation}
        \label{eq:assoc-fails2}
        (\prst_{\mora} \cartprod \prst_{\morb}) \cartprod \prst_{\morc} \simeq \prst_{\mora} \cartprod (\prst_{\morb} \cartprod \prst_{\morc}).
    \end{equation}
    %




       