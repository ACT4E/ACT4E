% !TEX root = chapter-standalone.tex

\section{Moore machines}

\linkvideo{spring2021-actions:semi-actions:processes} % Signals and processes

\linkvideo{spring2021-actions:semi-actions:processes:moore} % Moore machines
We now look at processes, especially dynamical systems, and see them as categories that act on sequences.

We have already seen linear discrete time systems. We can generalize them by allowing non-linear functions, so to have
the definition as follows. We call these \textbf{Moore} machines, and describe them as:
\begin{equation}\label{eq:moore-1}
    \begin{cases}
    \prdyn \colon \prin \cartprod \prst \sto \prst \\
    \prreadout \colon \prst \sto \prout,
    \end{cases}
\end{equation}
where~$\prin$ represents inputs,~$\prst$ states,~$\prout$ outputs,~$\prdyn$ the dynamics, and~$\prreadout$ the readout.
As introduced in \cref{sec:actions}, we can apply currying to~$\prdyn$, to obtain a map from inputs to endomorphisms on the states:
\begin{equation}\label{eq:moore-1-endo}
  \begin{cases}
  \prdyn \colon \prin \sto \Endof \prst \\
  \prreadout \colon \prst \sto \prout
  \end{cases}
\end{equation}

\begin{exercise}
fake one
\end{exercise}
\begin{solution}
\end{solution}

\showslides{
\begin{forslides}
\includesag{10_moore_1}\\
\includesag{10_moore_comp_seq}
  \begin{equation}
    \label{eq:cond_moore_comp}
    \prout_{\mora} \subseteq \prin_{\morb}
\end{equation}
  \includesag{10_moore_comp_seq_bis}
\end{forslides}}

We also need to have an~$\prstart \in \prst$ to act as the initial state.


Suppose we have two morphisms
\begin{equation}\label{eq:moore-mora}
  \mora = \tupp{\prin_{\mora},\prst_{\mora},\prout_{\mora},\prdyn_{\mora},\prreadout_{\mora},\prstart_{\mora}}
\end{equation}
and
\begin{equation}\label{eq:moore-morb}
\morb = \tupp{\prin_{\morb},\prst_{\morb},\prout_{\morb},\prdyn_{\morb},\prreadout_{\morb},\prstart_{\morb}},
\end{equation}
such that~$\prout_{\mora} \subseteq \prin_{\morb}$. The composition of these two systems should have a joint state that is the product of the states.

\begin{figure}[h]
  \begin{center}
    \includesag{10_moore_comp_seq}
    \caption{Composition of Moore machines (first version).}
    \label{fig:comp_moore_1}
\end{center}
\end{figure}


Here is one way to do it. We specify the spaces:

\begin{equation}\label{eq:moore-comp-naive-1}
  \begin{aligned}
  \prin_{\mora\mthen\morb} &= \prin_{\mora}   \\
  \prst_{\mora\mthen\morb} &= \prst_{\mora} \cartprod \prst_{\morb} \\
  \prstart_{\mora\mthen\morb} &= \tupp{\prstart_{\mora}, \prstart_{\morb}} \\
  \prout_{\mora\mthen\morb} &= \prout_{\morb}
  \end{aligned}
\end{equation}

Furthermore, we specify the dynamics

\begin{equation}\label{eq:moore-comp-naive-2}
  \definemap{
    \prdyn_{\mora\mthen\morb}
    }{
      \prin_{\mora} \times (\prst_{\mora} \times \prst_{\morb})
    }{
      (\prst_{\mora} \times \prst_{\morb})
    }{
      \tupp{u, \tupp{x_{\mora}, x_{\morb}}}
    }{
    \tupp{ \prdyn_{\mora} (u, x_{\mora}), \prdyn_{\morb}(\prreadout_{\mora}(x_{\mora}), x_{\morb})}
    }
\end{equation}

and the ``readout'':

\begin{equation}\label{eq:moore-comp-naive-3}
  \definemap{
    \prreadout_{\mora\mthen\morb}
    }{
      (\prst_{\mora} \times \prst_{\morb})
    }{
      \prout_{\morb}
    }{
      \tupp{x_{\mora}, x_{\morb}}
    }{
      \prreadout_{\morb}(x_{\morb})
    }
\end{equation}

We represent the composition graphically as in \cref{fig:comp_moore_1}.

However, if we define these using the Cartesian product~$\prst_{\mora} \cartprod \prst_{\morb}$, we cannot compose the systems in an associative way.
When we have three systems, composing in the two ways would bring to~$(\prst_{\mora} \cartprod \prst_{\morb}) \cartprod \prst_{\morc}$ and~$\prst_{\mora} \cartprod (\prst_{\morb} \cartprod \prst_{\morc})$, which are \emph{isomorphic} sets but not equal.
Elements of these sets are of the form~$\tup{\tup{a,b},c}$ and~$\tup{a, \tup{b,c}}$. You can clearly spot the isomorphism:
\begin{equation}\label{eq:assoc-fails1}
(\prst_{\mora} \cartprod \prst_{\morb}) \cartprod \prst_{\morc} \neq \prst_{\mora} \cartprod (\prst_{\morb} \cartprod \prst_{\morc})
\end{equation}
\begin{equation}\label{eq:assoc-fails2}
  (\prst_{\mora} \cartprod \prst_{\morb}) \cartprod \prst_{\morc} \simeq \prst_{\mora} \cartprod (\prst_{\morb} \cartprod \prst_{\morc})
  \end{equation}
We can avoid lengthy book-keeping by using a slightly different construction.

Let's consider the monoid of sequences of sets:

\begin{equation}\label{eq:sets-monoid}
\prst_{\mora} \setconcat \prst_{\morb} \setconcat \prst_{\morc} \in \seqsof\Set,
\end{equation}

for which we have, for sets

\begin{equation}\label{eq:sets-monoid-comp}
  (\prst_{\mora} \setconcat \prst_{\morb} )\setconcat \prst_{\morc} =
  \prst_{\mora} \setconcat \prst_{\morb} \setconcat \prst_{\morc} =
  \prst_{\mora} \setconcat (\prst_{\morb} \setconcat \prst_{\morc}),
\end{equation}

and for elements

\begin{equation}
  \tupca a   = a
\end{equation}
~
\begin{equation}\label{eq:sets-monoid-el-cat}
   \tupcat a b \elconcat c = \tupcatt a b c
\end{equation}

\begin{equation}\label{eq:sets-monoid-axiom1}
  \prftree{a \colon A}{b \colon B}{ \tupcat a b \colon (A \setconcat B) }
\end{equation}

\begin{equation}\label{eq:sets-monoid-axiom2}
  \prftree{a \colon A}{ \tupcat a b \colon (A \setconcat B) }{  b \colon  B }
\end{equation}
and
\begin{equation}\label{eq:sets-monoid-maps}
  \prftree{ f \colon A \to C}{ g \colon B \to C}{ \tupcat f g \colon (A\setconcat B)\to C }.
\end{equation}

\begin{figure}[h]
  \begin{center}
    \includesag{10_moore_comp_seq_bis}
    \caption{Composition of Moore machines (second version).}
    \label{fig:comp_moore_2}
\end{center}
\end{figure}

Here is a different way to do composition (\cref{fig:comp_moore_2}). First, we write spaces as:

\begin{equation}\label{eq:moore-comp-good-1}
  \begin{aligned}
  \prin_{\mora\mthen\morb} &= \prin_{\mora}   \\
  \prst_{\mora\mthen\morb} &= \prst_{\mora} \setconcat \prst_{\morb} \\
  \prstart_{\mora\mthen\morb} &= \tupcat{\prstart_{\mora}}{\prstart_{\morb}} \\
  \prout_{\mora\mthen\morb} &= \prout_{\morb} \\
  \end{aligned}
\end{equation}

We then write the dynamics

\begin{equation}\label{eq:moore-comp-good-2}
  \definemap{
    \prdyn_{\mora\mthen\morb}
    }{
      \prin_{\mora} \times (\prst_{\mora} \setconcat \prst_{\morb})
    }{
      (\prst_{\mora} \setconcat \prst_{\morb})
    }{
      \tupp{u, \tupcat{x_{\mora}}{x_{\morb}}}
    }{
      \tupcat{ \prdyn_{\mora} (u, x_{\mora})}{\prdyn_{\morb}(\prreadout_{\mora}(x_{\mora}), x_{\morb})}
    }
\end{equation}

\begin{equation}\label{eq:moore-comp-good-joint}
  \prst_{\mora} \setconcat \prst_{\morb}
\end{equation}

and the readout

\begin{equation}\label{eq:moore-comp-good-3}
  \definemap{
    \prreadout_{\mora\mthen\morb}
    }{
      (\prst_{\mora} \setconcat \prst_{\morb})
    }{
      \prout_{\morb}
    }{
      \tupcat{x_{\mora}}{x_{\morb}}
    }{
      \prreadout_{\morb}(x_{\morb})
    }
\end{equation}


With this definition we can define the semi-category $\Moore$ of Moore machines.

\begin{definition}[\Moore]
  \label{def:Moore}
  The \emph{semi-category of Moore machines} \Moore is given by:
  \begin{enumerate}
    \item \emph{Objects:} sets.
    \item \emph{Morphisms:} A morphism is a tuple
    \begin{equation}
    \mora = \tupp{\prin_{\mora},\prst_{\mora},\prout_{\mora},\prdyn_{\mora},\prreadout_{\mora},\prstart_{\mora}},
    \end{equation}
    where:
    \begin{itemize}
      \item $\prin$,~$\prst$,~$\prout$ are sets;
      \item $ \prdyn \colon \prin \to \Endof \prst$;
      \item $ \prreadout \colon \prst \sto \prout$.
  \end{itemize}
    \item \emph{Composition of morphisms:} Composition is given by:
    \begin{equation}
      \begin{aligned}
      \prin_{\mora\mthen\morb} &= \prst_{\mora}   \\
      \prst_{\mora\mthen\morb} &= \prst_{\mora} \setconcat \prst_{\morb} \\
      \prstart_{\mora\mthen\morb} &= \tupcat{\prstart_{\mora}}{\prstart_{\morb}} \\
      \prout_{\mora\mthen\morb} &= \prout_{\morb},
      \end{aligned}
    \end{equation}
    with
    \begin{equation}
      \label{eq:moore_dyn_comp}
      \definemap{
        \prdyn_{\mora\mthen\morb}
        }{
          \prin_{\mora} \times (\prst_{\mora} \setconcat \prst_{\morb})
        }{
          (\prst_{\mora} \setconcat \prst_{\morb})
        }{
          \tup{u, \tupcat{x_{\mora}}{x_{\morb}}}
        }{
          \tupcat{ \prdyn_{\mora} (u, x_{\mora})}{\prdyn_{\morb}(\prreadout_{\mora}(x_{\mora}), x_{\morb})},
        }
    \end{equation}
    and
    \begin{equation}
      \definemap{
        \prreadout_{\mora\mthen\morb}
        }{
          (\prst_{\mora} \setconcat \prst_{\morb})
        }{
          \prout_{\morb}
        }{
          \tupcat{x_{\mora}}{x_{\morb}}
        }{
          \prreadout_{\morb}(x_{\morb})
        }
    \end{equation}
  \end{enumerate}
\end{definition}

\begin{exercise}
  Show that indeed \Moore is a semi-category.
\end{exercise}
\begin{solution}
  Given valid, composable machines~$\mora$,~$\morb$,~$\morc$, we can specify the spaces and initial conditions:
  \begin{equation*}
    \label{eq:moore_assoc_spaces}
\begin{aligned}
\prin_{(\mora \mthen \morb)\mthen \morc}&=\prin_{\mora \mthen (\morb \mthen \morc)}=\prin_{\mora},\\
  \prst_{(\mora \mthen \morb)\mthen \morc}&=\prst_{\mora \mthen (\morb \mthen \morc)}=\prst_{\mora} \setconcat \prst_{\morb} \setconcat \prst_{\morc}\\
\prstart_{(\mora \mthen \morb)\mthen \morc}&=\prstart_{\mora \mthen (\morb \mthen \morc)}=\tupcatt{\prstart_\mora}{\prstart_\morb}{\prstart_\morc}\\
  \prout_{(\mora \mthen \morb)\mthen \morc}&=\prout_{\mora \mthen (\morb \mthen \morc)}=\prout_\morc
\end{aligned}
\end{equation*}
Starting from \cref{eq:moore_dyn_comp} one can now check associativity of the dynamics:
  \begin{equation*}
    \label{eq:assoc_moore_1}
\begin{aligned}
  \prdyn_{(\mora\mthen\morb)\mthen \morc}\colon \prin_{\mora} \times ((\prst_{\mora} \setconcat \prst_{\morb})\setconcat \prst_{\morc})&\to (\prst_{\mora} \setconcat \prst_{\morb})\setconcat \prst_{\morc}\\
  \tup{u, \tupcatt{x_{\mora}}{x_{\morb}}{x_\morc}}&\mapsto \tupcat{\prdyn_{\mora \mthen \morb}(u,\tupcat{x_\mora}{x_\morb})}{\prdyn_{\morc}(\prreadout_{\mora \mthen \morb}(\tupcat{x_\mora}{x_\morb}),x_\morc)}\\
  &=\tupcatt{\prdyn_\mora(u,x_\mora)}{\prdyn_\morb(\prreadout(x_\mora),x_\morb)}{\prdyn_\morc(\prreadout_\morb(x_\morb),x_\morc)}.
\end{aligned}
\end{equation*}
  On the other hand, one has:
    \begin{equation*}
    \label{eq:assoc_moore_2}
    \begin{aligned}
      \prdyn_{\mora\mthen(\morb\mthen \morc)}\colon \prin_{\mora} \times (\prst_{\mora} \setconcat (\prst_{\morb}\setconcat \prst_{\morc}))&\to \prst_{\mora} \setconcat (\prst_{\morb}\setconcat \prst_{\morc})\\
  \tup{u, \tupcatt{x_{\mora}}{x_{\morb}}{x_\morc}}&\mapsto \tupcat{\prdyn_{\mora}(u,x_\mora)}{\prdyn_{\morb\mthen \morc}(\prreadout_{\mora}(x_\mora),\tupcat{x_\morb}{x_\morc})}\\
  &=\tupcatt{\prdyn_\mora(u,x_\mora)}{\prdyn_\morb(\prreadout(x_\mora),x_\morb)}{\prdyn_\morc(\prreadout_\morb(x_\morb),x_\morc)}.
\end{aligned}
\end{equation*}
  Finally, one can check associativity of the readout:
  \begin{equation*}
    \label{eq:assoc_moore_3}
    \begin{aligned}
      \prreadout_{(\mora \mthen \morb)\mthen \morc}\colon (\prst_\mora \setconcat \prst_\morb)\setconcat \prst_\morc &\to \prout_\morc\\
      \tupcatt{x_\mora}{x_\morb}{x_\morc} &\mapsto \prreadout_\morc(x_\morc),
\end{aligned}
\end{equation*}
  and
    \begin{equation*}
    \label{eq:assoc_moore_4}
    \begin{aligned}
      \prreadout_{\mora \mthen (\morb\mthen \morc)}\colon \prst_\mora \setconcat (\prst_\morb\setconcat \prst_\morc) &\to \prout_\morc\\
      \tupcatt{x_\mora}{x_\morb}{x_\morc} &\mapsto \prreadout_\morc(x_\morc).
\end{aligned}
\end{equation*}
\end{solution}
\section{Action on sequences}

Let's now look at how machines like the above acts on sequences.

For now we only have defined semi-group, monoid, and group actions, and have not talked yet about (semi)category actions.
Let's consider the set of machines systems with~$\prin = \prout = \prgen$, which is the homset~$\HomSet\Moore\prgen\prgen$.

Given a finite input sequence~$u\colon \natnumbers \to \prgen$ of length~$n$, the output is an instantaneous transformation of the state:

\begin{equation}\label{eq:actions-on-sequences-y}
  \begin{aligned}
  y_0 &= \prreadout(x_0) \\
  y_1 &= \prreadout(x_1) \\
  y_2 &= \prreadout(x_2) \\
  ... &= ...  \\
  y_{k} &= \prreadout(x_{k-1})
\end{aligned}
\end{equation}

The state is computed recursively as follows:
\begin{equation}\label{eq:actions-on-sequences-x}
  \begin{aligned}
  x_0 &= \prdyn(u_0, \prstart) \\
  x_1 &= \prdyn(u_1, x_0) \\
  x_2 &= \prdyn(u_2, x_1) \\
  ... &= ... \\
  x_{k} &= \prdyn(u_{k}, x_{k-1})
  \end{aligned}
\end{equation}

Therefore, given a machine~$\mora \colon \prgen \to_{\Moore} \prgen$ we have defined a map from $\natnumbers \to \prgen$ to itself. Let's call it $\act$. It is defined as a map of the form
%
\begin{equation}\label{eq:actions-on-sequences-1}
\act_{\mora}\colon  (\natnumbers \to \prgen)  \sto  (\natnumbers \to \prgen),
\end{equation}

%
or, more formally,
%
  \begin{equation}\label{eq:actions-on-sequences-2}
    \act: \HomSet\Moore\prgen\prgen \sto \Endof {\natnumbers \to \prgen}.
\end{equation}
%
Note that both~$\HomSet\Moore\prgen\prgen$ and $\Endof {\natnumbers \to \prgen}$ are semigroups.
Could it be that~$\act$ is a semigroup morphism? (and consequently, is~$\act$ a covariant semigroup action)?

Let's check the condition for it being a morphism (\cref{eq:sgrp-mor-comp}):
%
\begin{equation}\label{eq:actions-on-sequences-left}
  \act ( \mora \mthen_{\Moore} \morb) \stackrel{?}{=}  \act ( \mora) \mthen_{\Endof  {\natnumbers \to \prgen}} \act(\morb)
\end{equation}
%
\begin{equation}\label{eq:actions-on-sequences-right}
  \act ( \morb \mthen_{\Moore} \mora) \stackrel{?}{=}  \act ( \mora) \mthen_{\Endof  {\natnumbers \to \prgen}} \act(\morb)
\end{equation}

\todo{check graphically}

\section{Other machines}


\begin{table*}[b]
  \caption{Some types of signals and processes}
  \label{tab:processes-types}
  \begin{tabular}{rccc}
  & \multicolumn{2}{c}{\textbf{Signals}} & \textbf{Processes} \\
  &\rule{0pt}{10pt} one-sided & two-sided &   \\
  Moore machines ($\Moore$) &
  $\natnumbers \sto \prgen$
  &
  $\wnumbers \sto \prgen$
  &
  \begin{minipage}{4cm}\raggedright
  \begin{equation*}
      \begin{cases}
      \prdyn \colon \prin \sto \Endof \prst \\
      \prreadout \colon \prst \sto \prout
      \end{cases}
  \end{equation*}
  \end{minipage}
  \\

  More machines ($\More$)& $\seqsof \prgen$ &  $\streamsof \prgen$ &
  \begin{minipage}{4cm}\raggedright
  \begin{equation*}
  \begin{cases}
  \prdyn \colon \seqsof \prin \sto \Endof \prst \\
  \prreadout \colon \prst \sto \seqsof \prout
  \end{cases}
  \end{equation*}

  \end{minipage}
  \\
  event-based (\tmpEB) & $\seqsof{(\natnumbers \cartprod \prgen)}$&
   $\streamsof{(\natnumbers \cartprod \prgen)}$&
  \begin{minipage}{4cm}\raggedright
  \begin{equation*}
  \begin{cases}
  \prdyn \colon \seqsof{(\natnumbers \cartprod \prin)} \sto \Endof \prst \\
  \prreadout \colon \prst \sto \seqsof{ (\natnumbers \cartprod \prout)}
  \end{cases}
  \end{equation*}
  \end{minipage}\\
  continuous  (\tmpDS) & $\nonNegReals \sto \prgen$ & $\reals \sto \prgen$ &
  \begin{minipage}{4cm}\raggedright
  \begin{equation*}
  \begin{cases}
    \prdyn\colon \prin \to \vectorfield(\prst) \\
  \prreadout\colon \prst \sto   \prout
  \end{cases}
  \end{equation*}
  \end{minipage}
  \\
  \end{tabular}
  \end{table*}



But there are many different types of machines.

\linkvideo{spring2021-actions:semi-actions:processes:more} % More machines
A Moore machine outputs 1 element at each time step; what if the machine was able to output more than one or zero output?

The signature of this would be this:
\begin{equation}\label{eq:more-signature}
  \begin{cases}
  \prdyn \colon \seqsof\prin \sto \Endof \prst \\
  \prreadout \colon \prst \sto \seqsof \prout
  \end{cases}
\end{equation}
where the output is not just~$\prout$ but~$\seqsof \prout$: the machine can produce zero or more outputs. We call these \emph{More} machines.

\begin{exercise}
Define the semi-category~$\More$ of More machines.
\end{exercise}

We have these act on~$\seqsof \prgen$; the timing information is not there.

Another type of machine are the event-based machines, with signature
\begin{equation} \label{eq:event-based}
  \begin{cases}
    \prdyn \colon \seqsof{(\natnumbers \cartprod \prin)} \sto \Endof \prst \\
    \prreadout \colon \prst \sto \seqsof{ (\natnumbers \cartprod \prout)}
    \end{cases}
\end{equation}

The natural numbers are the deltas between events. (This could be generalized to $\nonNegReals$.)

Continuous-time dynamical systems would be described as
%
\begin{equation} \label{eq:continuous}
\begin{cases}
  \prdyn\colon \prin \to \vectorfield(\prst) \\
\prreadout\colon \prst \sto   \prout
\end{cases}
\end{equation}
%
One should put more conditions on the objects--typically, that they are manifolds, and more constraints about the dynamics for the trajectories to exist.


\section{Action of a category}

Now it is time to generalize from actions of a semigroup to action of a semicategory.

Let~$\CatC$ a general process category, as in \cref{tab:processes-types}.

We want a process in~$\HomSet\CatC\Obja\Objb$ to induce a map between signals.

Note that each process type has a different signal types.

For example, a Moore machine in~$\HomSet\Moore\Obja\Objb$ will map the set~$\natnumbers \to \Obja$ to the set~$\natnumbers \to \Objb$.
A More machine in~$\HomSet\More\Obja\Objb$  will map the set~$\seqsof \Obja$ to the set~$\seqsof \Objb$.
An event based machine will map the set~$\seqsof{\natnumbers \cartprod \Obja}$ to the set~$\seqsof{\natnumbers \cartprod \Objb}$.

To look at this generically, we need to consider a map~$\catacOb$ that maps the base set~$\Obja$ to the actual set~$\catacOb(\Obja)$. For example:
\begin{equation}
  \definemap{\catacOb_{\Moore}}{\Set}{\Set}{\Obja}{(\natnumbers \to \Obja)}
\end{equation}
\begin{equation}
\definemap{\catacOb_{\More}}{\Set}{\Set}{\Obja}{\seqsof \Obja}
\end{equation}
\begin{equation}
\definemap{\catacOb_{\tmpEB}}{\Set}{\Set}{\Obja}{\seqsof{(\natnumbers \cartprod \Obja)}}
\end{equation}
\begin{equation}
\definemap{\catacOb_{\tmpDS}}{\Set}{\Set}{\Obja}{(\nonNegReals \to \Obja)}
\end{equation}

\begin{equation}\label{eq:phi-moore}
{\catacOb_{\Moore}}\colon{\Obja}\longmapsto{(\natnumbers \to \Obja)}
\end{equation}
\begin{equation}\label{eq:phi-more}
{\catacOb_{\More}}\colon{\Obja}\longmapsto{\seqsof \Obja}
\end{equation}
\begin{equation}\label{eq:phi-EB}
{\catacOb_{\tmpEB}}\colon{\Obja}\longmapsto{\seqsof{(\natnumbers \cartprod \Obja)}}
\end{equation}
\begin{equation}\label{eq:phi-DS}
{\catacOb_{\tmpDS}}\colon{\Obja}\longmapsto{(\nonNegReals \to \Obja)}
\end{equation}
Note for the last one we need to define manifolds.

Then we need another map~$\catacMor$ that given a process in~$\HomSet\CatC\Obja\Objb$
produces a map from~$\catacOb(\Obja)$ to~$\catacOb(\Objb)$.
%
\begin{equation}\label{eq:gamma-1}
  \catacMor\colon \HomSet\CatC\Obja\Objb \to  (\catacOb(\Obja) \sto \catacOb(\Objb))
\end{equation}
%
Interpreting the arrows as morphisms in sets, we can say:
\begin{equation}\label{eq:gamma-2}
  \catacMor\colon \HomSet\CatC\Obja\Objb \to  \HomSet\Set{\catacOb(\Obja)}{\catacOb(\Objb) }
\end{equation}

\linkvideo{spring2021-actions:semi-cat-actions} % Semi-category action
\begin{ctdefinition}[Semi-category action]\label{def:semicategory-action}
  A \emph{semi-category action} of a semi-category~$\CatC$ is defined by
  \begin{itemize}
    \item a map~$\catacOb$ that associates from each object~$\Obja \in \ObC$, a set~$\catacOb(\Obja)$:
    \begin{equation}
      \catacOb \colon \ObC \to \Ob_\Set
    \end{equation}
    \item a map~$\catacMor$ that associates to each morphism a function:
    \begin{equation}
      \catacMor\colon \HomSet\CatC\Obja\Objb \to  \HomSet\Set{\catacOb(\Obja)}{\catacOb(\Objb) }
    \end{equation}
  \end{itemize}
  Moreover, this condition must hold:
  \begin{equation}
    \catacMor(\mora\then\morb) = \catacMor(\mora) \mthen \catacMor(\morb).
  \end{equation}
\end{ctdefinition}

\devel{

\showslides{forslides}

\begin{equation}\label{eq:cat-mora}
  \catacMor(\mora)
\end{equation}
\begin{equation}\label{eq:cat-morb}
  \catacMor(\morb)
\end{equation}
\begin{equation}\label{eq:cat-morab}
  \catacMor(\mora\mthen\morb)
\end{equation}
\begin{equation}\label{eq:catC}
  \CatC
\end{equation}
\begin{equation}\label{eq:sets}
  \Set
\end{equation}

  \begin{equation}\label{eq:signal}
  \styleobj{s}
  \end{equation}
  \begin{equation}\label{eq:signal-apply}
    \act(\mora, \styleobj{s})
    \end{equation}
    \begin{equation}\label{eq:morab}
      \mora \mthen \morb
      \end{equation}
      \begin{equation}\label{eq:mora}
        \mora
        \end{equation}
        \begin{equation}\label{eq:morb}
          \morb
          \end{equation}
          \begin{equation}\label{eq:mora-prdyn}
            \prdyn_{\mora}
            \end{equation}
\begin{equation}\label{eq:mora-prst}
\prst_{\mora}
\end{equation}
\begin{equation}\label{eq:mora-prin}
\prin_{\mora}
\end{equation}
\begin{equation}\label{eq:mora-prout}
\prout_{\mora}
\end{equation}
\begin{equation}\label{eq:morb-prst}
  \prst_{\morb}
  \end{equation}
  \begin{equation}\label{eq:morb-prin}
  \prin_{\morb}
  \end{equation}
  \begin{equation}\label{eq:morb-prout}
  \prout_{\morb}
  \end{equation}
  \begin{equation}\label{eq:morab-prst}
    \prst_{\mora\mthen\morb}
    \end{equation}
    \begin{equation}\label{eq:morab-prin}
    \prin_{\mora\mthen\morb}
    \end{equation}
    \begin{equation}\label{eq:morab-prout}
    \prout_{\mora\mthen\morb}
    \end{equation}
  \begin{equation}\label{eq:gen-prst}
    \prst
    \end{equation}
    \begin{equation}\label{eq:gen-prin}
    \prin
    \end{equation}
    \begin{equation}\label{eq:gen-prout}
    \prout
    \end{equation}
      \begin{equation}\label{eq:moore-start}
    \prstart \in \prst
  \end{equation}

  \begin{equation}\label{eq:prgen-u}
    u\colon \natnumbers \to \prgen
  \end{equation}

  \begin{equation}\label{eq:prgen-y}
    y\colon \natnumbers \to \prgen
  \end{equation}


  \begin{equation}\label{eq:prgen-u-1}
    u\colon \natnumbers \to \prin
  \end{equation}

  \begin{equation}\label{eq:prgen-y-1}
    y\colon \natnumbers \to \prout
  \end{equation}

  \begin{equation}\label{eq:prgen-u-2}
    u\colon \prin
  \end{equation}



  \begin{equation}\label{eq:Z}
    \stylesets{Z}
  \end{equation}


  \begin{equation}\label{eq:Za}
    \catacOb{\stylesets{Z}}
  \end{equation}
  \begin{equation}\label{eq:Ua}
    \catacOb{\prin}
  \end{equation}
  \begin{equation}\label{eq:Ya}
    \catacOb{\prout}
  \end{equation}



  \begin{equation}\label{eq:prgen-y-2}
    y\colon \seqsof  \prout
  \end{equation}

  \begin{equation}\label{eq:prgen}
    \prgen
  \end{equation}

  \begin{equation}\label{eq:now1}
    \act(\morb, \act(\mora, \styleobj{s}))
  \end{equation}


  \begin{equation}\label{eq:now2}
    \act(\mora\mthen\morb, \styleobj{s})
  \end{equation}

  \begin{equation}\label{eq:ev-sign}
    u\colon \seqsof{(\natnumbers \times \prin)}
\end{equation}

  \begin{equation}\label{eq:ev-sign2}
    y\colon \seqsof{(\natnumbers \times \prout)}
\end{equation}

\end{forslides}}
