% !TEX root = chapter-standalone.tex
\section{Matrix groups}


\begin{margintable}
\caption{Matrix groups}
\label{tab:matrix-groups}
\begin{tabular}{cccc}
\mgGLn & general linear group & invertible matrices \\
\mgOn & orthogonal group & preserves depths \\
\mgEn & Euclidean groups & preserve distance & \\
\mgSOn & special orthogonal group& \\
\mgSLn & special linear group &  \\
\mgSEn & Special Euclidean group \\
\end{tabular}
\end{margintable}

There are many \emph{matrix groups} (\cref{tab:matrix-groups}) that represent linear transformations of a vector space that have some special properties.

\begin{definition}[General linear group $\mgGLn$]
The general linear group of order $n$, written $\mgGLn$, is the group of $n\times n$ invertible matrices.
\end{definition}

We have used this in the past.

\begin{definition}[General orthogonal group $\mgOn$]
The general orthogonal group of order $n$, written $\mgOn$, is the group of $n\times n$ square matrices that satisfy the following:
\begin{equation}
\mat{M} \mat{M}^T = \mat{M}^T \mat{M} = I.
\end{equation}
\end{definition}

\begin{definition}[General euclidean group $\mgEn$]
The general euclidean group of order $n$, written $\mgEn$, is the group of $n+1\times n+1$ square matrices of the form
\begin{equation}\label{eq:En-def}
\Ematrix{\matmor{R}}{\vectmor{t}}
\end{equation}
where $\matmor{R}\in \mgOn$ and $\vectmor{t} \in \reals^n$.
\end{definition}

\todotext{Did we define subgroup, submonoid, etc?}

We also have 3 ``special'' versions: $\mgSLn$, $\mgSOn$, $\mgSEn$.
These are subgroups of those that have determinant equal to 1.

These matrix groups are also transformations of $\reals^n$.

\subsection{Special Euclidean group}

The group $\mgSEtwo$ and $\mgSEthree$ are particular important in robotics because they represent
the roto-translations of the plane and 3D space, respectively.

From~\cref{eq:En-def} we know we can represent one by a pair~$\tup{\matmor{R}, \vectmor{t}}$, with $\matmor{R}\in\mgSOn$ and
$\vectmor{t} \in \reals^n$.

The formulas for composition is
%
\begin{equation}
  \tup{\matmor{R}_1, \vectmor{t}_1} \mthen_{\mgSEn} \tup{\matmor{R}_2, \vectmor{t}_2}  = \tup{\matmor{R}_2 \matmor{R}_1, \matmor{R}_2 \vectmor{t}_1 + \vectmor{t}_2}
\end{equation}
%
% Note that because we are using $\mthen$ (then), we do not have the nice symmetry
% that we had if we were using $\after$ (after).
% %
% \begin{equation}
% \Ematrix{\matmor{R}_1}{\vectmor{t}_1} \Ematrix{\matmor{R}_2}{\vectmor{t}_2} = \Ematrix{\matmor{R}_1 \matmor{R}_2}{\matmor{R}_1\vectmor{t}_2 + \vectmor{t}_1},
% \end{equation}
% %

The group $\mgSEn$ induces a transformation on the points of  $\reals^n$.
We are going to call this an \emph{action}.

The action is the following function:
\begin{equation}
\definemap{\act}{\mgSEn \times \reals^n}{\reals^n}{\tup{ \tup{\matmor{R}, \vectmor{t}}, \vectob{p}}}{\matmor{R}\vectob{p} + \vectmor{t}}
\end{equation}
Given a rototranslation and a point, the function returns the rototranslated point.
%
% We can also see this in matrix form as follows. We need to substitute for a point $\vectob{p} \in \reals^n$ a point
% $ \Epoint{\vectob{p}} \in \reals^{n+1}$.
% %
% \begin{equation}
% \Ematrix{\matmor{R}}{\vectmor{t}}
% \Epoint{\vectob{p}}
%  =
%  \Epoint{\matmor{R}\vectob{p} + \vectmor{t}}
% \end{equation}

If we apply two rototranslations in sequence, we have
\begin{equation}
  \begin{aligned}
& \act(\tup{\matmor{R}_2, \vectmor{t}_2}, \act(\tup{\matmor{R}_1, \vectmor{t}_1}, \vectob{p})) \\
= & \act(\tup{\matmor{R}_2, \vectmor{t}_2}, \matmor{R}_1\vectob{p} + \vectmor{t}_1) \\
= & \matmor{R}_2\matmor{R}_1\vectob{p} + \matmor{R}_2\vectmor{t}_1 + \vectmor{t}_2.
  \end{aligned}
\end{equation}
It is easy to see that it is equal to compose the two transformations and then apply it to the object
\begin{equation}
\act({\matmor{R}_2 \matmor{R}_1, \matmor{R}_2 \vectmor{t}_1 + \vectmor{t}_2}, \vectob{p})
= \matmor{R}_2\matmor{R}_1\vectob{p} + \matmor{R}_2\vectmor{t}_1 + \vectmor{t}_2 .
\end{equation}
Thus we have proved this property
\begin{equation}
\act(T_1, \act(T_2, \vectob{p})) = \act(T_1 \then T_2, \vectob{p})) .
\end{equation}
The notion of semigroup action generalizes this property.

\section{Actions}

\begin{ctdefinition}[Semigroup action, preliminary version]\label{def:semigroup-action-prelim}
  A \emph{semigroup action} of a semigroup $\sgrpA$ onto a set $\setA$ is a map
  \begin{equation}\label{eq:act1}
    \act \colon \sgrpAset \cartprod \setA \to \setA
  \end{equation}
  such that
  \begin{equation}\label{eq:act1cond}
    \act(\monela, \act(\monelb, a)) = \act(\monela \mthen_\sgrpA \monelb, a)).
  \end{equation}
\end{ctdefinition}

This definition is the standard algebraic definition.
We will compress it a bit using the language of category theory.

\begin{marginfigure}
Haskell Curry
\end{marginfigure}

We need to introduce Dr.~Haskell Curry. His first name,  Haskell, named the language.
 His last name, Curry, named the operation of \emph{currying} that we are going to need.

Curry noticed that for describing the domain of a function, we do not need to have the cartesian product.
If we have a function
\begin{equation}
f \colon \setA \cartprod \setB \to \setC
\end{equation}
we can rewrite it as a function of higher type
\begin{equation}
f \colon \setA \to ( \setB \to \setC).
\end{equation}
If you feel like a cool computer scientist, you can also drop the parenthesis and write
\begin{equation}
f \colon \setA \to \setB \to \setC,
\end{equation}
because the expression is not ambiguous, as~$\to$ associates from the left.



This specifies $f$ as a function that, given a~$\setA$, provides a function that, given a $\setB$,
provides a $\setC$; this is the same as a function that needs the two arguments~$\setA$ and $\setB$ before giving the~$\setC$.  To describe the isomorphism we can write it as 
\begin{equation}
  \setA \to ( \setB \to \setC) \simeq  \setA \cartprod \setB \to \setC,
\end{equation}
or, more precisely, highlighting that these are morphism arrows in \Set, 
\begin{equation}
  \HomSet\Set\setA{\HomSet\Set\setB\setC} \simeq \HomSet\Set{\setA \cartprod \setB}\setC.
\end{equation}
Keep this in mind, it will show up later.

Now armed with currying, we can take a second look at~\cref{eq:act1} and realize that we can rewrite 
\begin{equation}
\act \colon \sgrpAset \cartprod \setA \to \setA
\end{equation}
as 
\begin{equation}
  \act \colon \sgrpAset \to (\setA \to \setA).
\end{equation}
We gave a name to functions of type~$\setA \to \setA$: these are the endomorphisms of $\setA$, written as~$\Endof\setA$. Thus, we rewrite this as 
\begin{equation}\label{eq:act-to-endo}
  \act \colon \sgrpAset \to \Endof\setA.
\end{equation}
Now we take a second look at \cref{eq:act1cond}
\begin{equation} 
  \act(\monela, \act(\monelb, a)) =_{\setA} \act(\monela \mthen_\sgrpA \monelb, a)).
\end{equation}
If we rewrite it as an equality of functions, we obtain
\begin{equation} \label{eq:act-to-endo-properties}
  \act(\monela) \then_{\Endof\setA}  \act(\monelb) =_{\Endof\setA} \act(\monela \mthen_\sgrpA \monelb). 
\end{equation}

Looking at \cref{eq:act-to-endo,eq:act-to-endo-properties} we recognize that together 
they indicated that~$\act$ is a semigroup morphism~(\cref{def:semigroup-morphism}). This brings us to a more compact description of what is a semigroup action. 

\begin{ctdefinition}[Semigroup action]\label{def:semigroup-action}
  A \emph{semigroup action} of a semigroup~$\sgrpA$ onto a set~$\setA$ is a semigroup morphism 
  \begin{equation}\label{eq:act-semigroup}
    \act \colon \sgrpA \fto \Endof\setA.
  \end{equation} 
\end{ctdefinition}

As it turns out, what could look like a new notion, is actually a special case of a general notion we already encountered. 

For completeness, we also define monoid action.

\begin{ctdefinition}[Monoid action]\label{def:monoid-action}
  A \emph{monoid action} of a monoid~$\monoidA$ onto a set~$\setA$ is a monoid morphism 
  \begin{equation}\label{eq:act-monoid}
    \act \colon \monoidA \fto \Endof\setA.
  \end{equation} 
\end{ctdefinition}

The neutral element of the monoid~$\Endof\setA$ is the identity function~$\id_{\setA}$.
Thus, a monoid action must map the neutral element of~$\monoidA$ to~$\id_{\setA}$.

For defining a group action, we must introduce a slight variation.
The endomorphisms~$\Endof\setA$ are not a group, because they also contain non-invertible maps.

We need to reference the \emph{automorphisms}~$\Autof\setA$, which is a group.

\begin{ctdefinition}[Group action]\label{def:group-action}
  A \emph{group action} of a group~$\grpA$ onto a set~$\setA$ is a group morphism 
  \begin{equation}\label{eq:act-group}
    \act \colon \grpA \fto \Autof\setA.
  \end{equation} 
\end{ctdefinition}

\todotext{Usual exercises}





\section{\statusdraft{Modeling processes}}
%
%\newcommand{\prgen}{\stylesets{A}}
%\newcommand{\prin}{\stylesets{\prin}}
%\newcommand{\prout}{\stylesets{\prout}}
%\newcommand{\prst}{\stylesets{\prst}}
%\newcommand{\prdyn}{\stylemorphisms{f}}
%\newcommand{\prreadout}{\stylemorphisms{h}}

\begin{table*}[p]
\caption{Some types of signals and processes}
\begin{tabular}{rccc}
& \multicolumn{2}{c}{\textbf{Signals}} & \textbf{Processes} \\
&\rule{0pt}{10pt} one-sided & two-sided &   \\
discrete time &
$\natnumbers \sto \prgen$
&
$\wnumbers \sto \prgen$
&
\begin{minipage}{4cm}\raggedright
\begin{equation*}\label{eq:a}
    \begin{cases}
    \prdyn \colon \prin \sto \Endof \prst \\
    \prreadout \colon \prst \sto \prout
    \end{cases}
\end{equation*}
\end{minipage}
\\

asynchronous& $\seqsof \prgen$ &  $\streamsof \prgen$ &
\begin{minipage}{4cm}\raggedright
\begin{equation*}
\begin{cases}
\prdyn \colon \prin \sto \Endof \prst \\
\prreadout \colon \prst \sto \seqsof \prout
\end{cases}
\end{equation*}

\end{minipage}
\\
event-based& $\seqsof{(\nonNegReals \cartprod \prgen)}$&
 $\streamsof{(\nonNegReals \cartprod \prgen)}$&
\begin{minipage}{4cm}\raggedright
\begin{equation*}
\begin{cases}
\prdyn \colon (\nonNegReals \cartprod \prin) \sto \Endof \prst \\
\prreadout \colon \prst \sto \seqsof{ (\nonNegReals \cartprod \prout)}
\end{cases}
\end{equation*}
\end{minipage}\\
continuous& $\nonNegReals \sto \prgen$ & $\reals \sto \prgen$ &
\begin{minipage}{4cm}\raggedright
\begin{equation*}
\begin{cases}
  \prdyn\colon \prin \cartprod \prst \sto \tangbundle \prst \\
\prreadout\colon \prst \sto   \prout
\end{cases}
\end{equation*}
\end{minipage}
\\
\end{tabular}
\end{table*}



