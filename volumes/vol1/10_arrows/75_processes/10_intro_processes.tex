% !TEX root = chapter-standalone.tex

\section{Moore machines, first version}
\label{sec:moore-machines}

%\linkvideo{spring2021-actions:semi-actions:processes} % Signals and processes

\linkvideo{spring2021-actions:semi-actions:processes:moore} % Moore machines

In the following we consider \textbf{Moore machines}\footnote{They are name after Edward F. Moore.}, which are a type of basic model for describing certain dynamical systems. It's characteristic features are: 
\begin{itemize}
\item a \textbf{state space} which describes all possible states that the system can possible be in;
\item an \textbf{input space} and an \textbf{output space};
\item a \textbf{dynamics} which describes how, given an input, the system's state changes according to that input;
\item a \textbf{read-out} which relates the current state of the system to the output space.
\end{itemize}
 %
 The idea then is that we are interested in the ways that such a machine can transform a sequence of inputs into a sequence of outputs. 
 
 \subsection{A first mathematical model}
 
Our first version for formalizing the idea of a Moore machine looks as follows. We model the input space, state space, and output space as sets  $\prin$, $\prst$, and $\prout$ respectively, and we model the dynamics and read-out as functions
\begin{equation}
    \label{eq:moore-1}
    \begin{cases}
        \prdyn \colon \prin \cartprod \prst \sto \prst, \\
        \prreadout \colon \prst \sto \prout.
    \end{cases}
\end{equation}


%    As introduced in \cref{sec:actions}, we can apply currying to~$\prdyn$, to obtain a map from inputs to endomorphisms on the states:
%    %
%    \begin{equation}
%        \label{eq:moore-1-endo}
%        \begin{cases}
%            \prdyn \colon \prin \sto \Endof \prst, \\
%            \prreadout \colon \prst \sto \prout.
%        \end{cases}
%    \end{equation}
%    %

We will also choose an element~$\prstart \setin \prst$ of the state space as an initial state; we will use this when specifying how a Moore machine acts on sequences of inputs.

Thus, in total, a Moore machine is specified by a tuple of the following type:
\begin{equation}
    \label{eq:moore-tuple}
    \tupp{\prin,\prst,\prout,\prdyn,\prreadout,\prstart}.
\end{equation}

\

\todotext{provide a simple example here to illustrate}

\

\subsection{Moore machines acting on sequences}

Given a Moore machine $\tup{\prin, \prout, \prst, \prdyn, \prreadout, \prstart}$, the following is a standard way to think of it as acting on a sequence of inputs, transforming it into a sequence of outputs. 

Given a sequence of inputs $u_0, u_1, u_2, \dots $, we use the following recipe
\begin{equation}
    \label{eq:Moore-action-recursion}
    \begin{cases}
        x_{k+1} = \prdyn(u_k, x_k) \\
        y_{k}   = \prreadout(x_{k}).
    \end{cases}
\end{equation}
   ($k \setin \natnumbers$) to produce a sequence $y_0, y_1, y_2, \dots $ of outputs. For the very first step, when $k = 0$, we need the initial state $x_0 = \prstart$ in order to compute $x_1 = \prdyn(u_0, x_0)$ and $y_0 = \prreadout(x_0)$. 



\

\todotext{continute the simple example above to illustrate}

\


For a fixed Moore machine, we not that the recipe \cref{eq:Moore-action-recursion} defines a function which maps any given sequence $(u_k)_{k \setin \natnumbers}$ of inputs to a corresponding sequence of outputs $(u_k)_{k \setin \natnumbers}$. In other words, a function of the type $\prin^\natnumbers \to \prout^\natnumbers$. 
%    \begin{equation}
%    \label{eq:behavior-Moore-machine}
%        \prin^\natnumbers \to \prout^\natnumbers,  \quad (u_k)_{k \setin \natnumbers} \mapsto (y_k)_{k \setin \natnumbers}.
%    \end{equation}
We will think of this function as describing possible external behaviors of the Moore machine, because it encodes what is externally observable in terms of how the Moore machine is used to relate inputs to outputs.

\todotext{Is the remark below needed or relevant?}

\begin{remark}
As an aside, we note that in concrete applications it might be the case that we do not actually know a precise model of a Moore machine in terms of the data \cref{eq:moore-tuple}, but we can still observe what it does when we feed it inputs. And in some cases this external behavior is in fact all we really care about.
\end{remark}


\subsection{Composing Moore machines}

Let us consider composing Moore machines serially by letting the output of one machine be the input of the next. We'd like the result to again be a Moore machine. 

\begin{marginfigure}
    \centering
    \includesag{10_moore_comp_seq}
    \caption{Composition of Moore machines (first version).}
    \label{fig:comp_moore_1}
\end{marginfigure}

In other words, given Moore machines
%
\begin{equation}
    \label{eq:moore-mora}
    \mora = \tupp{\prin_{\mora},\prst_{\mora},\prout_{\mora},\prdyn_{\mora},\prreadout_{\mora},\prstart_{\mora}}
\end{equation}
%
and
%
\begin{equation}
    \label{eq:moore-morb}
    \morb = \tupp{\prin_{\morb},\prst_{\morb},\prout_{\morb},\prdyn_{\morb},\prreadout_{\morb},\prstart_{\morb}},
\end{equation}
with~$\prout_{\mora} = \prin_{\morb}$, we'd like to define their composition as a Moore machine
\begin{equation}
    \mora \mthen \morb = \tupp{\prin_{\mora \mthen \morb},\prst_{\mora \mthen \morb},\prout_{\mora \mthen \morb},\prdyn_{\mora \mthen \morb},\prreadout_{\mora \mthen \morb},\prstart_{\mora \mthen \morb}}.
\end{equation}
The situation is illustrated in \cref{fig:comp_moore_1}.


Here is one way to do it: we set
%
\begin{equation}
    \label{eq:moore-comp-naive-1}
    \begin{aligned}
        \prin_{\mora\mthen\morb}    & \definedas \prin_{\mora}, \\
        \prst_{\mora\mthen\morb}    & \definedas \prst_{\mora} \cartprod \prst_{\morb}, \\
        \prout_{\mora\mthen\morb}   & \definedas \prout_{\morb} \\
        \prstart_{\mora\mthen\morb} & \definedas \tupp{\prstart_{\mora}, \prstart_{\morb}},
    \end{aligned}
\end{equation}
%
we define the composite dynamics to be 
%
\begin{equation}
    \label{eq:moore-comp-naive-2}
    \defmapset{
        \prdyn_{\mora\mthen\morb}
    }{
        \prin_{\mora} \cartprod (\prst_{\mora} \cartprod \prst_{\morb})
    }{
        (\prst_{\mora} \cartprod \prst_{\morb})
    }{
        \tupp{u, \tupp{x_{\mora}, x_{\morb}}}
    }{
        \tupp{ \prdyn_{\mora} (u, x_{\mora}), \prdyn_{\morb}(\prreadout_{\mora}(x_{\mora}), x_{\morb}),}
    }
\end{equation}
%
and we define the composite readout to be
%
\begin{equation}
    \label{eq:moore-comp-naive-3}
    \defmapperiodset{
        \prreadout_{\mora\mthen\morb}
    }{
        (\prst_{\mora} \cartprod \prst_{\morb})
    }{
        \prout_{\morb}
    }{
        \tupp{x_{\mora}, x_{\morb}}
    }{
        \prreadout_{\morb}(x_{\morb})
    }
\end{equation}
%

This formalization works very nicely -- it models composition using the output of one machine as the input of the next, and result of composition is again a Moore machine. 

However, there is one aspect which we wish were different: this composition operation is not associative. It \emph{almost} is, but not quite. We explain why below. 

The reason we wish it \emph{were} associative is that Moore machines would then form a semicategory and we could integrate them nicely into the mathematics we have been developing thusfar. 

The culprit for associativity failing is the Cartesian product that we use to define the state space $\prst_{\mora} \cartprod \prst_{\morb}$ of a composite Moore machine $\mora \mthen \morb$. Consider three composable systems~$\mora$,~$\morb$, and~$\morc$. If we compose them in the two ways~$(\mora \mthen \morb) \mthen \morc$ and~$\mora \mthen (\morb \mthen \morc)$, then their respective state spaces are~$(\prst_{\mora} \cartprod \prst_{\morb}) \cartprod \prst_{\morc}$ and~$\prst_{\mora} \cartprod (\prst_{\morb} \cartprod \prst_{\morc})$.
These sets are \emph{isomorphic}, but they are not equal on the nose, and hence also the Moore machines $\mora \mthen \morb$ and $\mora \mthen (\morb \mthen \morc)$ are very close to being equal, but are not quite. 

To see why
\begin{equation}
    \label{eq:assoc-fails1}
    (\prst_{\mora} \cartprod \prst_{\morb}) \cartprod \prst_{\morc} \neq \prst_{\mora} \cartprod (\prst_{\morb} \cartprod \prst_{\morc}).
\end{equation}
recall that the elements of $(\prst_{\mora} \cartprod \prst_{\morb}) \cartprod \prst_{\morc}$ are nested tuples of the form 
$$\tupp{ \tupp{x_{\mora}, x_{\morb}}, x_{\morc} }$$ while the elements of $\prst_{\mora} \cartprod (\prst_{\morb} \cartprod \prst_{\morc})$ are nested tuples of the form $$\tupp{x_{\mora}, \tupp{x_{\morb}, x_{\morc}} }.$$

An isomorphism between the two sets is given by the following function 
\begin{equation}
    \label{eq:assoc-fails2}
    \begin{aligned}
        (\prst_{\mora} \cartprod \prst_{\morb}) \cartprod \prst_{\morc} & \to \prst_{\mora} \cartprod (\prst_{\morb} \cartprod \prst_{\morc}) \\
        \tupp{ \tupp{x_{\mora}, x_{\morb}}, x_{\morc} }                 & \mapsto \tupp{x_{\mora}, \tupp{x_{\morb}, x_{\morc}} }
    \end{aligned}
\end{equation}
which simply re-brackets and tuples. 

In the next section we will introduce a technical construction for defining a product similarly to the cartesian product, but which is associative on the nose, and not just ``up to an isomorphism''. This construction will allow us to make a new, modified formalization of Moore machines which form a bona fide semicategory, and it will prove useful in other respects further down the road. 

