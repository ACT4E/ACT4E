% !TEX root = chapter-standalone.tex


\section{\statusdraft{Generators and relations}}

\todotext{Presentation of monoids}

\todotext{Important because it is ancestor example of commuting diagrams}


In \cref{plant-trafo-semigroup} we considered a set of states $X = \{ \text{sprout}, \text{young}, \text{mature}, \text{old}, \text{dead} \}$, a function $T: X \sto X$, and the semigroup
\begin{equation}
\sgrpA = \{ T^n \mid n \in \natnumbers \}.
\end{equation}
Note that $\sgrpA$ has a special form: all of its elements can be expressed in terms one of its elements, $T$, and the multiplication operation (which, in this case, is function composition). To describe this state of affairs we say that $\sgrpA$ is \emph{generated} by the element $T$.


Recall that $T$ was defined by
\begin{align*}
T(\text{sprout}) &=  \text{young} \\
T(\text{young}) &=  \text{mature} \\
T(\text{mature}) &=  \text{old} \\
T( \text{old}) &= \text{dead} \\
T (\text{dead}) &= \text{dead}.
\end{align*}
Observe that the function $T^4$ will map all elements of $X$ to the element ``\text{dead}''. For example, if we start with the element ``\text{sprout}'', the result of applying $T$ four times is
$$
\text{sprout} \overset{T}{\sto} \text{young} \overset{T}{\sto} \text{mature} \overset{T}{\sto} \text{old} \overset{T}{\sto} \text{dead}.
$$

Note also that for \emph{any} $n \geq 4$, the function $T^n$ will map all elements of $X$ to the element ``\text{dead}''.
If we consider $T^6$, for example, then, for any $x \in X$,
$$
T^6(x) = T^2(T^4(x)) = T^2(\text{dead}) = T(T(\text{dead})) = T(\text{dead}) = \text{dead}.
$$
It follows that all $T^n$, for $n \geq 4$, are actually \emph{all the same map}: the one that sends every state to the dead state. Thus $\sgrpA = \{ T^n \mid n \in \natnumbers \}$ actually only has at most four elements! Namely $T$, $T^2$, $T^3$, and $T^4$. (Are any of these four maps actually equal, too?)

When two elements which a priori could be distinct from each other (such as $T^6$ and $T^4$ above) turn out to be equal, we call this a \emph{relation} between the elements of $\sgrpA$.

\begin{definition}
Let $\tup{\sgrpA, \mtimes}$ be a semigroup, and let $\setA \subseteq \sgrpA$ be a subset. We say that $\sgrpA$ is \emph{generated} by $\setA$ if every element of $\sgrpA$ can be expressed as a finite multiplication of elements of $\setA$.
\end{definition}

\begin{example}
Consider \cref{string-semigroup}, where elements of the semigroup $\sgrpA$ were non-empty strings built using the elements of the ``alphabet'' set $\setA = \{ L, R \}$. In this case, $\sgrpA$ is generated by $\setA$.
\end{example}


\begin{example}
Consider the natural numbers (without zero) as a semigroup, where addition is the semigroup multiplication (see \label{natnum-semigroup}). This semigroup is generated by the subset $\{1 \}$.
\end{example}

\begin{definition}
A \emph{relation} on a semigroup $\tup{\sgrpA, \mtimes}$ is an equation between two multiplications of elements of $\sgrpA$.
\end{definition}

\begin{example}
Consider the semigroup $\tup{\natnumbers, +}$, and consider $l, k \in \natnumbers$. The equation $l + k = k + l$ is an example of  a relation. 
\end{example}


\


