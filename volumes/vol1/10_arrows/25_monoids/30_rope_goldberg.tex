% !TEX root = chapter-standalone.tex


\section{Rope Goldberg Machines}

The inventions of Professor Butts, transcribed by Rube Goldberg, are prodigious machines that provide practical ways to perform everyday tasks (\cref{fig:Rube3}). They also highlight the power of compositionality.

\begin{figure}[h]
  \includegraphics[width=0.7\textwidth]{Rube3}
  \caption{\emph{Life’s Little Jokes \#59,380}, Rube Goldberg}
  \label{fig:Rube3}
\end{figure}



We are going to assemble similar machines. We start by introducing some of the essential elements:

\begin{itemize}
  \item Rope
  \item Elastic
  \item Brick
  \item Glass
  \item Spring
  \item Rubber
\end{itemize}

\hfill
\includegraphics[width=2cm, height=2cm]{placeholder}
\hfill
\includegraphics[width=2cm, height=2cm]{placeholder}
\hfill
\includegraphics[width=2cm, height=2cm]{placeholder}
\hfill

\hfill
\includegraphics[width=2cm, height=2cm]{placeholder}
\hfill
\includegraphics[width=2cm, height=2cm]{placeholder}
\hfill
\includegraphics[width=2cm, height=2cm]{placeholder}
\hfill


\todographics{ Would be nice to have one icon each and have it as 6 figures one after another instead of a vertical list.}

\todotext{Describe here the properties of these components (elasticity, etc.)}


These are the classes that define the data.

\classsource{Component}
\classsource{Rope}
\classsource{Brick}
\classsource{Elastic}

\begin{gradedexercise}[\exname{TestRopeGoldberg}]
  In this exercise you will compute the behavior of a chain of those components.

  The interface you need to implement is:

  \classsource{RupeGoldbergSolver}


  \begin{enumerate}
    \item The first function asks what would happen if you were to hang this chain on one end, and let it hang (\cref{fig:hang}).
    Would it break? If not, what would be the extended length?
    \item The second question asks what would happen if you were to let the chain rest horizontally (\cref{fig:push}). Then fix one end to an \emph{immovable wall} and push to the other end. What would happen? If the push destroys some of the objects, consider
    them destroyed and of length 0.
    \item  The third question asks what would happen if you were to pull with a certain force  (\cref{fig:pull}). Either you will break the chain, or there will be an equilibrium. What would be the length?
  \end{enumerate}

\end{gradedexercise}

\begin{figure}
  \hfill
  \subfloat[\label{fig:hang}Hanging the chain]{
    \includegraphics[width=3cm, height=4cm]{placeholder}
  }
  \hfill
  \subfloat[\label{fig:push}Horizontal compression]{
  \includegraphics[width=3cm, height=4cm]{placeholder}
  }
  \hfill
  \subfloat[\label{fig:pull}Horizontal extension]{
  \includegraphics[width=3cm, height=4cm]{placeholder}
  }
  \hfill
  \caption{Visualization of the three scenarios.
  \todographics{Make graphics. }}
\end{figure}


We will walk you through the theoretical part, but we will leave the implementation to you.

We should think \emph{compositionally}. We should first see what happens in the specific cases,
and see if the answer can be generalized in a way that the solution is compositional.

\todotext{Go through the specific cases}


\todotext{Give breakthrough observation that all the behavior can be described by one relation between force and length. Hence what we are talking about is really composition in a certain monoid.}

\todotext{The solution is to create for each component this relation, and then compose the relations.}
