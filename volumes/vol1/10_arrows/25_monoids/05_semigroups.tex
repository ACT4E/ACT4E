% !TEX root = standalone.tex

\section{\statusdraft{Semigroups}}\label{sec:semigroups}


\begin{definition}[Semigroup]
  \label{def:semigroup}
  A \emph{\iindex{semigroup}} is a set~$\sgrpA$, together with a binary operation~$\mtimes \colon \sgrpA \cartprod \sgrpA \sto \monoidA$, called \emph{multiplication}, which satisfies:
  \begin{enumerate}
    \item Associative law:~$(x\mtimes y)\mtimes z=x\mtimes (y\mtimes z)$ for all $x , y, z \in \sgrpA$. 
  \end{enumerate}
\end{definition}

\begin{example}\label{string-semigroup}
Consider a finite set $A$, which we think of as an alphabet. For instance, consider $A = \{ L, R \}$. 
Let $\sgrpA$ be the set of non-empty strings of elements of $A$. For example, 
$$ LLRLRRRL $$ 
is a non-empty string of elements of $A$. 

We may define a multiplication operation on $\sgrpA$ simply by concatenating strings. Given the strings
$$ LLRLRRRL  \quad \text{ and } \quad RRLR, $$ 
their concatenation is the string
$$ LLRLRRRLRRLR.$$
It is readily seen that concatenation satisfies the associative law, so $\sgrpA$, together with this multiplication, forms a semigroup. 
\end{example}


\begin{example}\label{max-semigroup}
Let $\sgrpA$ denote the natural numbers $\natnumbers$. The function $\Max : \sgrpA \times \sgrpA \sto \sgrpA$ defines a multiplication operation making $\sgrpA$ a semigroup. 
\end{example}

\begin{exercise}
Verify the statement made in \cref{max-semigroup}; that is, check that the associative law holds. 
\end{exercise}

\begin{example}\label{trafo-semigroup}
Let $X$ denote a set... 
\end{example}

\begin{definition}[Semigroup homomorphism]
  \label{def:semigroup-mor}
Let $\sgrpA$ and $\sgrpB$ be semigroups. A homomorphism of semigroups from $\sgrpA$ to $\sgrpB$ is a function $\sgrpmorA: \sgrpA \sto \sgrpB$ such that
\begin{equation}\label{sgrp-mor-comp}
\sgrpmorA(\sgrpelAa \mtimes_{\sgrpA} \sgrpelAb) = \sgrpmorA (\sgrpelAa) \mtimes_{\sgrpB} \sgrpmorA(\sgrpelAb) \quad \quad \forall \sgrpelAa, \sgrpelAb \in \sgrpA
\end{equation}
We think of the equation \cref{sgrp-mor-comp} as a way of saying that the function of sets $\sgrpmorA: \sgrpA \sto \sgrpB$  is \emph{compatible} with the multiplication operations on $\sgrpA$ and $\sgrpB$, respectively. 
\end{definition}



\begin{remark}[Identity homomorphism]
  \label{def:semigroup-mor}
Let $\sgrpA$ be a semigroup. The identity function $\id: \sgrpA \sto \sgrpA$ is always a morphism of semigroups. Indeed, the condition
\begin{equation}\label{sgrp-mor-comp}
\id (\sgrpelAa \mtimes_{\sgrpA} \sgrpelAb) = \id (\sgrpelAa) \mtimes_{\sgrpA} \id(\sgrpelAb) \quad \quad \forall \sgrpelAa, \sgrpelAb \in \sgrpA
\end{equation}
is satisfied. 
\end{remark}

\begin{exercise}
How many different non-isomorphic semigroups are there with precisely one element? How many with precisely two elements?
\end{exercise}

\begin{exercise}
Let $\sgrpmorA: \sgrpA \mto \sgrpB$ be a morphism of semigroups. Prove that $\sgrpmorA$ is an isomorphism of semigroups if and only if the function $\sgrpmorA: \sgrpA \sto \sgrpB$ is bijective. 
\end{exercise}




