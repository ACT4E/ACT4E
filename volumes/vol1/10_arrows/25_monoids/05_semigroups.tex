% !TEX root = chapter-standalone.tex

\section{\statusdraft{Semigroups}}\label{sec:semigroups}


\begin{definition}[Semigroup]
  \label{def:semigroup}
  A \emph{\iindex{semigroup}} is a set~$\sgrpA$, together with a binary operation~$\mtimes \colon \sgrpA \cartprod \sgrpA \sto \monoidA$, called \emph{multiplication}, which satisfies:
  \begin{enumerate}
    \item Associative law:~$(x\mtimes y)\mtimes z=x\mtimes (y\mtimes z)$ for all $x , y, z \in \sgrpA$.
  \end{enumerate}
\end{definition}

\begin{remark}
Given a fixed set $\sgrpA$, there will in general be many different choices of multiplication operation which make $\sgrpA$ into a semigroup. So, technically, a semigroup is a pair $\tup{\sgrpA, \mtimes}$ consisting of a set $\sgrpA$ and a choice of multiplication $\mtimes$. The set $\sgrpA$ is the \emph{underlying set} of the semigroup. Often we will be slightly imprecise and refer to a semigroup simply by the name of its underlying set; this is practical when it is clear from context which multiplication operation we are considering, or when it is not necessary to refer to the multiplication explicitly.
\end{remark}

\begin{example}\label{natnum-semigroup}
  Consider~$\tup{\natnumbers,+}$. This is a semigroup, since, for all~$l,m, n \in \natnumbers$, we have
  \begin{equation*}
  (l+m)
    +n=l+(m+n).
  \end{equation*}
\end{example}

\begin{example}\label{string-semigroup}
Consider a finite set $A$, which we think of as an alphabet. For instance, consider $A = \{ L, R \}$.
Let $\sgrpA$ be the set of non-empty strings of elements of $A$. For example,
$$ LLRLRRRL $$
is a non-empty string of elements of $A$.

We may define a multiplication operation on $\sgrpA$ simply by concatenating strings. Given the strings
$$ LLRLRRRL  \quad \text{ and } \quad RRLR, $$
their concatenation is the string
$$ LLRLRRRLRRLR.$$
It is readily seen that concatenation satisfies the associative law, so $\sgrpA$, together with this multiplication, forms a semigroup.
\end{example}





When the underlying set of a semigroup $\tup{\sgrpA, \mtimes}$ is a finite set, one way to specify the multiplication $\mtimes$ is simply by writing out what it does with each pair of elements of $\sgrpA$. Since $\mtimes$ is a function of two variable, this can be conveniently displayed as a table, called a \emph{multiplication table}.

\todotext{JL: write out (and include graphic) of a simply multiplication table}

\begin{exercise}
Consider the set $\sgrpA$ of finite non-empty strings of symbols from the alphabet $A$, as in \cref{string-semigroup}. Can you think of other candidates for multiplication operations on $\sgrpA$, besides the straightforward concatenation of strings considered above? Do your candidates define semigroup multiplications -- that is, do they obey the associative law?

For example, one might consider the operation where, given an ordered pair of strings, one first doubles the last symbol of the first string, and then concatenates. Is this operation associative?
\end{exercise}
\begin{solution}
  \todotext{to write}
\end{solution}


\begin{example}\label{max-semigroup}
The function $\max : \natnumbers \times \natnumbers \sto \natnumbers$ defines a multiplication operation which equips $\natnumbers$ with the structure of a semigroup.
\end{example}

\begin{exercise}
Verify the statement made in \cref{max-semigroup}; that is, check that the associative law holds.

Does $\min : \natnumbers \times \natnumbers \sto \natnumbers$ also define a semigroup structure on $\natnumbers$?
\end{exercise}
\begin{solution}
  \todotext{to write}
\end{solution}


\begin{example}\label{plant-trafo-semigroup}
Consider the set $X = \{ \text{sprout}, \text{young}, \text{mature}, \text{old}, \text{dead} \}$ which describes five possible states of a plant. Let $T : X \sto X$ be the function that describes ``development'':
\begin{align*}
T(\text{sprout}) &=  \text{young} \\
T(\text{young}) &=  \text{mature} \\
T(\text{mature}) &=  \text{old} \\
T( \text{old}) &= \text{dead} \\
T (\text{dead}) &= \text{dead}
\end{align*}

In other words, we think of $T$ as the change of state of the plant during a given time interval (say, three months). Composing the function $T$ with itself corresponds to considering multiples of the given time interval. For example, the function
$$
T \then T \then T : X \sto X
$$
models the change over the course of nine months. In general, for the n-fold composition of $T$ with itself we write $T^n$. The set $\sgrpA = \{ T^n \mid n \in \natnumbers \}$ is a semigroup, with the multiplication given by the composition operation.

\end{example}

\begin{definition}[Semigroup homomorphism]
  \label{def:semigroup-mor}
Let $\sgrpA$ and $\sgrpB$ be semigroups. A homomorphism of semigroups from $\sgrpA$ to $\sgrpB$ is a function $\sgrpmorA: \sgrpA \sto \sgrpB$ such that
\begin{equation}\label{sgrp-mor-comp}
\sgrpmorA(\sgrpelAa \mtimes_{\sgrpA} \sgrpelAb) = \sgrpmorA (\sgrpelAa) \mtimes_{\sgrpB} \sgrpmorA(\sgrpelAb) \quad \quad \forall \sgrpelAa, \sgrpelAb \in \sgrpA
\end{equation}
We think of the equation \cref{sgrp-mor-comp} as a way of saying that the function of sets $\sgrpmorA: \sgrpA \sto \sgrpB$  is \emph{compatible} with the multiplication operations on $\sgrpA$ and $\sgrpB$, respectively.
\end{definition}


\begin{definition}[Identity homomorphism]
  \label{def:identity-sgrp-mor}
Let $\sgrpA$ be a semigroup. The identity function $\id_\sgrpA: \sgrpA \sto \sgrpA$ is always a morphism of semigroups. Indeed, the condition
\begin{equation}
\id (\sgrpelAa \mtimes_{\sgrpA} \sgrpelAb) = \id (\sgrpelAa) \mtimes_{\sgrpA} \id(\sgrpelAb) \quad \quad \forall \sgrpelAa, \sgrpelAb \in \sgrpA
\end{equation}
is satisfied. We call this the \emph{identity homomorphism} of $\sgrpA$.
\end{definition}

\begin{definition}[Semigroup isomorphism]
  \label{def:semigroup-iso}
Let $\sgrpA$ and $\sgrpB$ be semigroups. A homomorphism of semigroups $\sgrpmorA: \sgrpA \mto \sgrpB$ is called a \emph{semigroup isomorphism} if there exists a homomorphism of semigroups $\sgrpmorB: \sgrpB \mto \sgrpA$ such that
\begin{equation}\label{sgrp-iso-cond}
\sgrpmorA \then \sgrpmorB = \id_\sgrpA \quad \text{ and } \quad  \sgrpmorB \then \sgrpmorA = \id_\sgrpB.
\end{equation}
\end{definition}

\begin{exercise}
How many different non-isomorphic semigroups are there with precisely one element? How many with precisely two elements?
\end{exercise}
\begin{solution}
  \todotext{to write}
\end{solution}

\begin{exercise}
Let $\sgrpmorA: \sgrpA \mto \sgrpB$ be a morphism of semigroups. Prove that $\sgrpmorA$ is an isomorphism of semigroups if and only if the function $\sgrpmorA: \sgrpA \sto \sgrpB$ is bijective.
\end{exercise}
\begin{solution}
  \todotext{to write}
\end{solution}


Given a semigroup $\tup{S, \mtimes}$, for each $n \in \natnumbers$, we can define an induced n-ary multiplication operation
$$
S^n \sto S, \ \tup{s_1, s_2, ..., s_n} \mapsto s_1 \mtimes s_2 \dots \mtimes s_n.
$$
Thanks to the associative law, this is well-defined -- that is, we do not need to set parenthases. We'll say that an element $s \in S$ is an \emph{n-fold multiplication} if it is in the image of this n-ary multiplication operation. At times we may not wish to specify the arity of the multiplication, in which case we just speak of a \emph{multiplication}. 
