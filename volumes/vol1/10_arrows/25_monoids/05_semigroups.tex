% !TEX root = standalone.tex

\section{\statusdraft{Semigroups}}\label{sec:semigroups}


\begin{definition}[Semigroup]
  \label{def:semigroup}
  A \emph{\iindex{semigroup}} is a set~$\semigroupA$, together with a binary operation~$\mtimes \colon \semigroupA \cartprod \semigroupA \sto \monoidA$, called \emph{multiplication}, which satisfies:
  \begin{enumerate}
    \item Associative law:~$(x\mtimes y)\mtimes z=x\mtimes (y\mtimes z)$ for all $x , y, z \in \semigroupA$. 
  \end{enumerate}
\end{definition}

\begin{example}\label{string-semigroup}
Consider a finite set $A$, which we think of as an alphabet. For instance, consider $A = \{ L, R \}$. 
Let $\semigroupA$ be the set of non-empty strings of elements of $A$. For example, 
$$ LLRLRRRL $$ 
is a non-empty string of elements of $A$. 

We may define a multiplication operation on $\semigroupA$ simply by concatenating strings. Given the strings
$$ LLRLRRRL  \quad \text{ and } \quad RRLR, $$ 
their concatenation is the string
$$ LLRLRRRLRRLR.$$
It is readily seen that concatenation satisfies the associative law, so $\semigroupA$, together with this multiplication, forms a semigroup. 
\end{example}


\begin{example}\label{string-semigroup}
Let $\semigroupA$ denote the natural numbers $\natnumbers$. Consider the function $\Max : \semigroupA \times \semigroupA \sto \semigroupA$
\end{example}