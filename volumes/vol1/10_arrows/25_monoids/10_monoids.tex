% !TEX root = chapter-standalone.tex


\section{\statusdraft{Monoids}}
\label{sec:parallelism-monoids}


Algebraic structures are often defined in \emph{layers}. For example, in the definition of semigroup, we start with a set $\sgrpA$ as a basic building block, and we add a layer of structure to it, namely a multiplication operation $\mtimes \colon \sgrpA \cartprod \sgrpA \sto \monoidA$. The multiplication operation for semigroups was not only a new \emph{structure} that we added, but we also required that this structure obey a \emph{condition}, namely that it satisfy the associative law. One might also say that the multiplication operation was a new \emph{constituent} or a new \emph{datum}, and that satisfying the associative law is a \emph{property}. Mathematicians often use such words in an intuitive, non-rigorous way as a tool for structuring their thinking. We will do the same. For clarity, we will aim to stick with the words \emph{constituents} and \emph{conditions}. Roughly speaking, we think of constituents as building blocks, and we think of conditions as rules for how those blocks fit together and behave.

Using the constituent vs. condition distinction we will, in particular, present some definitions in the following succinct, list-like fashion:

\begin{definition}[Monoid]
  \label{def:monoid}
  A \emph{\iindex{monoid}} is:
  \begin{quote}
    \constit
    \begin{enumerate}
      \item a set~$\monoidA$;
      \item a binary operation~$\mtimes \colon \monoidA \cartprod \monoidA \sto \monoidA$, called \emph{multiplication};
      \item a specified element $\idmon\in \monoidA$, called \emph{neutral element}.
    \end{enumerate}
    \condit
    \begin{enumerate}
      \item Associative law:~$(x\mtimes y)\mtimes z=x\mtimes (y\mtimes z)$;
      \item Neutrality Laws:~$\idmon\mtimes x=x=x\mtimes \idmon$.
    \end{enumerate}
  \end{quote}
\end{definition}


\begin{remark}
  The way that we presented the definition of a monoid is certainly not unique. For example, we could have done the following.

  \begin{quote}
    A \emph{\iindex{monoid}} is:
    \begin{quote}
      \constit
      \begin{enumerate}
        \item a semigroup~$\tup{\monoidA, \mtimes}$;
        \item a specified element $\idmon\in \monoidA$, called \emph{neutral element}.
      \end{enumerate}
      \condit
      \begin{enumerate}
        \item Neutrality Laws:~$\idmon\mtimes x=x=x\mtimes \idmon$.
      \end{enumerate}
    \end{quote}
  \end{quote}
  In this version, two constituents and one condition from \cref{def:monoid} are ``compressed'' into the information that we are using here a semigroup as a constituent. This kind of ``compression'' has its pros and cons; depending on the context will use it to varing degrees.

  There is a similar dilemma when considering the software interfaces to describe these structure.
  In terms of software engineering, the two strategies are \emph{composition} (a monoid has a semigroup as a constituent)
  and \emph{inheritance} (a monoid \emph{is} a semigroup with additional data). 

\end{remark}




\begin{example}
  Consider~$\tup{\reals,+,0}$. This is a monoid, since, for all~$x,y, z \in \reals$, we have
  \begin{equation*}
  (x+y)
    +z=x+(y+z),
  \end{equation*}
  and
  \begin{equation*}
    x+0=x=0+x.
  \end{equation*}
\end{example}

\begin{example}
  The set~$\wnumbers$, together with the operation of multiplication of whole numbers, forms a monoid. The neutral element is the number $1$.
\end{example}

\begin{lemma}
  \label{neut-el-unique}
  Let $\tup{\sgrpA, \mtimes}$ be a semigroup. If there exists elements $1 \in M$ and $1' \in M$ such that $\tup{\sgrpA, \mtimes, 1}$ and $\tup{\sgrpA, \mtimes, 1'}$ are each monoids, then $1 = 1'$ must hold. In other words, the neutral element of a monoid is uniquely determined by the underlying semigroup structure.
\end{lemma}

\begin{exercise}
  Prove \cref{neut-el-unique}.
\end{exercise}

\begin{definition}[Monoid homomorphism]
  \label{def:monoid-mor}
  Let $\tup{\monoidA, \mtimes_\monoidA, 1_\monoidA}$ and $\tup{\monoidB, \mtimes_\monoidB, 1_\monoidB}$ be monoids. A homomorphism of monoids from $\monoidA$ to $\monoidB$ is a function $\mapa : \monoidA \sto \monoidB$ such that
  \begin{equation}
    \label{mon-mor-comp}
    \mapa (\monelAa \mtimes_{\monoidA} \monelAb) = \mapa (\monelAa) \mtimes_{\monoidB}  \mapa(\monelAb) \quad \quad \forall \monelAa, \monelAb \in \monoidA
  \end{equation}
  and
  \begin{equation}
    \label{mon-id-comp}
    \mapa (1_\monoidA) = 1_{\monoidB}
  \end{equation}
\end{definition}

\begin{example}
  The set~$\monoidA = \{ -1, 0, 1 \}$, together with multiplication of whole numbers and with $1$ as neutral element, forms a monoid. The inclusion map $ \monoidA \sto \wnumbers$ is a homomorphism of monoids.
\end{example}

\todo{Nice example: the map from sequence of characters to sequences
of sounds is monoidal in certain languages (Korean, Japanese, almost in Italian.) and
also invertible.}


\begin{definition}[Identity homomorphism]
  \label{def:identity-mon-mor}
  Let $\monoidA$ be a monoid. Similar to the case of semigroups, the identity function $\id_\monoidA: \monoidA \sto \monoidA$ is also a homomorphisms of monoids.
\end{definition}



\begin{definition}[Monoid isomorphism]
  \label{def:monoid-iso}
  A homomorphism of semigroups $\mora: \monoidA \mto \monoidB$ is called a \emph{monoid isomorphism} if there exists a homomorphism of monoids $\morb: \monoidB \sto \monoidA$ such that
  \begin{equation}
    \label{sgrp-iso-cond}
    \mora \then \morb = \id_\monoidA \quad \text{ and } \quad  \morb \then \mora = \id_\monoidB.
  \end{equation}
\end{definition}


\begin{exercise}
  Prove: a morphism of monoids $\mora: \monoidA \mto \monoidB$ is an isomorphism of monoids if and only if the function $\mora: \sgrpA \sto \sgrpB$ is bijective.
\end{exercise}

\

\

\



\

\

\


\todotext{JL: I would change the example below; it feels slightly misleading to use ``0''. This construction works for any closed ray open toward + $\infty$}

\begin{example}
  Consider~$\tup{\nonNegReals,\max,0}$. This is a monoid, since, for all~$x,y\in \nonNegReals$, we have:
  \begin{equation*}
    \max(\max(x,y),z)=\max(x,\max(y,z)),
  \end{equation*}
  and
  \begin{equation*}
    \max(x,0)=x=\max(0,x).
  \end{equation*}
\end{example}



\

\

\



\

\

\


\todotext{JL: We should edit / change the example below (for coherence with the string example for semigroups, and to change the definition of sequence)}
\begin{example}[Sequences]
  A sequence is a function whose domain is a subset of~$\natnumbers$ \JL{this is a strange and non-standard definition of sequence; I would avoid}, and are called \emph{finite} if the domain of the function is finite. Often finite sequences are referred to as \emph{lists}. Given a set~$S$, we denote the set of all lists on~$S$ by $S^\ast$. This can be made into a monoid, by considering \emph{concatenation} as the operation, and the empty list as the neutral element. Specifically, a list is an element $s\in S^\ast$ consists of a $n\in \natnumbers$ and a function~$f\colon [n]\to S$, where~$[n]=\{i\colon \natnumbers\mid i<n\}\subseteq \natnumbers$. The empty list, denoted~$()$, is the unique list of length 0. Given~$n>0$, we can write the list which assigns~$0,\ldots,n-1$ to~$s_0,s_1,\ldots,s_{n-1}$ as $(s_0,s_1,\ldots,s_{n-1})$. Given a list~$x\in S^\ast$ of length~$m$ and a list~$y\in S^\ast$ of length~$n$, we can define their \emph{concatenation}~$x*y$ as list of length~$m+n$ with:
  \begin{equation*}
    i\mapsto
    \begin{cases}
      x_i&\text{if }i<m\\
      y_{i-m}&\text{if }i\geq m.
    \end{cases}
  \end{equation*}
  Clearly, this definition of concatenation satisfies associativity and unitality, making this construction a monoid. This is often referred to as the \emph{free monoid on~$S$}.
\end{example}


\todostructure{JL: The example below needs to be moved}
\begin{example}
  Given any category~\CatC, and any object~$\Obja\in \CatC$, the set of \emph{endomorphisms}~$\Hom_{\CatC}(\Obja,\Obja)$ is a monoid. The category depicted in \cref{fig:monoid_endomorphisms} has three objects~$\Obja,\Objb,\Objc$ and several morphisms.~$\Obja$ has four endomorphisms,~$\Objb$ two, and~$\Objc$ three (including identity morphisms). Let's now take the binary operation~$\mtimes$ to be the composition~$\then$ in~\CatC, and the neutral element to be the identity~$\catid_\Obja$. The associativity and unitality laws of the category~\CatC coincide with the ones of the monoid's definition, and are satisfied. Therefore, we can identity a monoid as a one-object category.
\end{example}

\begin{figure}[h!]
  \begin{center}
    \includesag{043_monoid_endomorphisms}
    \caption{}
    \label{fig:monoid_endomorphisms}
  \end{center}
\end{figure}


%\section{Dynamical systems and monoids}

\AC{in the end I would make this only a simple example of monoid - no introduction of group etc.}
\JL{inserting this here as an un-baked idea for a subsection. maybe it could be the first subsection of this chapter; that way idendity laws and associative laws can be introduced before talking about categories}
\gray{
  What are the simplest kinds of mathematical models of a dynamical system that we can think of?

  One possible answer is something like this: we can describe a dynamical system as a set $S$ of possible states, together with a description of how states change over time. For the latter, consider time to be labeled by distinct ``points in time''. Then, we can just think in terms of time-steps, \eg  seconds, or we can think of points in time where \eg  an action is triggered and the system passes to a new state.

  One thing we want to describe is how the state of our system changes over time, and in particular from one moment in time to the next. For any time step, we will not assume that we know what specific state the system is in, but rather we will describe, at once, all possible evolutions during that time step, \ie  we consider all possible initial conditions at once. Given two consecutive moments in time, we might describe the possible changes in the system by a function $T : S \rightarrow S$, which maps each state $s \in S$ to a next state $T(s) \in S$. This is a deterministic change of state: given $s$, the function $T$ determines the next state $T(s)$. The function $T$ is like a rule. Let's call $T$ an ``evolution operator'', because it describes how the system states might evolve over a time step.

  We might want to consider various possible evolution operators. We could consider functions $T_a$, $T_b$, $T_c$, \etc. We can also compose these functions: given $T_a$ and $T_b$, we might have, over the course of two time steps, the change described by $T_a \circ T_b$. For simplicity, let's suppose we work with three evolution operations $T_a$, $T_b$, and $T_c$.

  \

  -> introduce semigroups (implicitly or explicitly)

  \


  -> introduce monoids
}

\AC{
  For me the basic example of monoid with dynamical systems is taking the transition function.

  Let $E^s_t: X \to X$ be the evolution function from $s$ to $t$. Then states evolve like this: $x_t = E^s_t (x_s)$.

  If you assume that the system is time invariant, then the evolution only depends on the difference $\delta = s-t$.
  You have now a communative monoid of transition functions $T_\delta$ where $T_0 = \text{identity}$.

  (No need to do semigroups.)

}
