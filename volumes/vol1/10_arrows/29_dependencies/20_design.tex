% !TEX root = standalone.tex

\section{\statusdraft{Design}}

\todotext{This was moved after the definition of a category. Change accordingly.}


\todo[inline]{Gives more examples before getting to the definition. Give more examples
with different conventions for the arrow direction.}

In engineering design, one creates \emph{systems} out of \emph{components}. Each component has a reason to be in there. We will show how category theory can help in formalizing the chains of causality that underlie a certain design.

We will need to reason at the level of abstraction where we consider the ``function'', or ``functionality'', which each component provides, and the ``requirements'' that are needed to provide the function.

We will start with a simple example of the functioning principle of an electric car.

In an electric car, there is a battery, a store of the electric energy resource. We can see the production of motion as the chain of two transformations:

\begin{itemize}
  \item The \motor transmutes the \electricpower into \rotationalmotion.
  \item The \rotationalmotion is converted into \translationalmotion by the \wheels and their friction with the road.
\end{itemize}

We see that there are two types of things in this example:
\begin{enumerate}
  \item The ``transmuters'': the \motor and \wheels.
  \item The ``transmuted'': the \electricpower, the \rotationalmotion, the \translationalmotion.
\end{enumerate}

For a first qualitative description of the scenario, we might choose to just keep track of what is transmuted into what. We can draw a diagram in which each resource is a point (\cref{fig:e1}).

\begin{figure}[h!]
  \centering
  \includesag{30_dpcatfig_e1}
  \caption{Resources in the electric car example.}
  \label{fig:e1}
\end{figure}



Now, we can draw arrows between two points if there is a transmuter between them.

We choose the direction of the arrow such that
\begin{equation}
  \label{eq:arrow1}
  \transmuted{X} \stackrel{\transmuter{transmuter}}{\mto} \transmuted{Y}
\end{equation}
means that ``using \textsf{transmuter}, having $\transmuted{Y}$ is sufficient to produce $\transmuted{X}$''.

\begin{remark}[Are we going the wrong direction?]
  The chosen direction
  for the arrows is completely the opposite of what you would expect if you thought about
  ``input and outputs''. There is a good reason to use this convention, though it will
  be apparent only a few chapters later. In the meantime, it is a good exercise
  to liberate your mind about the preconception of what an arrow means; in category theory
  there will be categories where the arrows represent much more abstract concepts than input/output.
\end{remark}

Another way to write \cref{eq:arrow1} would be as follows:
\begin{equation}
  \transmuter{transmuter} \colon \transmuted{X} \mto \transmuted{Y}.
\end{equation}
This is now to you something syntactically familiar; when we study the categories of sets and functions between sets we will see that in that context the familiar meaning is also the correct meaning.

With these conventions, we can describe the two transmuters as these arrows:
%
\begin{align}
  \motor&:  \rotationalmotion \mto \electricpower, \\
  \wheels&: \translationalmotion \mto \rotationalmotion.
\end{align}
%
We can put these arrows in the diagrams, and obtain
the following (\cref{fig:e2}).
% In this simple setting, we can identify the key systems and subsystems: car, engine, axle, wheels, and road.

% \AC{we don't refer to these subsystems later}


% We can identify the functionality/resources of interest as:~\textsf{electric} \textsf{power}, \textsf{rotational} \textsf{motion}, and~\textsf{translational} \textsf{motion}. Note that each of these quantities plays a dual role. For example, the \textsf{rotational motion} is something which is produced by the motor, so it is a functionality for the motor, while it is a resource for axle/wheels, because they need it to provide~$\mathsf{translational}\ \mathsf{motion}$.


% Furthermore, we can draw an arrow between two resources if we can obtain one from the other. In the example, we have described how \textsf{electric power} becomes \textsf{rotational motion}, described by the \textsf{engine} arrow, and how \textsf{rotational motion} becomes \textsf{translational motion}, described by the \textsf{wheel} arrow (\cref{fig:e2}).

\begin{figure}[h!]
  \centering
  \includesag{30_dpcatfig_e2}
  \caption{Transmuters are arrows between resources.}
   \label{fig:e2}
\end{figure}

In this representation, the arrows are the components of the system.
We will learn how to compose these arrows according to the rules of category theory.
The basic rule will be \emph{composition}.
If we use the semantics that an arrow from resource~$X$ to resource~$Y$ means ``having~$Y$ is
enough to obtain~$X$'', then, since~$Y$ is enough for~$Y$ per definition, we can add a self-loop for each
resource. We will call the self-loops \emph{identities} (\cref{fig:e3}).

\begin{figure}[h!]
  \centering
  \includesag{30_dpcatfig_e3}
  \caption{System components and identities. }
  \label{fig:e3}
\end{figure}

Furthermore, we might consider the idea of composition of arrows.
Suppose that we know that
\begin{equation*}
  \transmuted{X} \stackrel{\transmuter{a}}{\mto} \transmuted{Y}
  \qquad \text{and} \qquad
  \transmuted{Y} \stackrel{\transmuter{b}}{\mto} \transmuted{Z},
\end{equation*}
that is, using a $b$ we can get a $\transmuted{Y}$ from a $\transmuted{Z}$, and using an $\transmuter{a}$ we can get a $\transmuted{X}$ from a $\transmuted{Y}$,
then we conclude that using and $\transmuter{a}$ and a $\transmuter{b}$ we can get an $\transmuted{X}$ from a $\transmuted{Z}$.

In our example, if the arrows \wheels  and \motor exist, then also the arrow ``\wheels then \motor'' exists~(\cref{fig:e4}).

\begin{figure}[h!]
  \centering
  \includesag{30_dpcatfig_e4}
  \caption{Composition of system components. }
  \label{fig:e4}
\end{figure}

So far, we have drawn only one arrow between two points, but we can draw as many as we want.
If we want to distinguish between different brands of motors, we would just draw
one arrow for each model. For example,~\cref{fig:e4bis} shows two models of
motors (\transmuter{motor A}, and \transmuter{motor B}) and two models of wheels
(\transmuter{wheels U} and \transmuter{wheels V}).


\begin{figure}[h!]
  \centering
  \includesag{20_wheel_motor_options}
  \caption{Multiple models for wheels and motors.}
   \label{fig:e4bis}
\end{figure}

The figure implies now the existence of \emph{four} composed
arrows: ``\transmuter{wheels U} then \transmuter{motor A}'',
``\transmuter{wheels U} then \transmuter{motor B}'',
``\transmuter{wheels V} then \transmuter{motor A}'', and
``\transmuter{wheels V} then \transmuter{motor B}'', all going from \translationalmotion to \electricpower;


A ``category`` is an abstract mathematical structure that captures the properties
of these systems of points and arrows and the logic of composition.

The basic terminology is that the points are called \textbf{objects},
and the arrows are called \textbf{morphisms}.

In our example, the \motor and the \wheels are the morphisms, and \electricpower, \rotationalmotion,
\translationalmotion are the objects.

Many things can be defined as categories and we will see many examples in this book.

We are now just biding our time before introducing the formal definition of category.
At first sight it will be intimidating: there are four parts to the definition, two axioms to define.
Moreover, it is quite a bit technical and it takes half a page to write.

