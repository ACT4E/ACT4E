% !TEX root = chapter-standalone.tex


\section{Endorelations}
\label{sec:endorelations}

\linkvideo{spring2021-relations:relations:endorel} % Endorelations
\todotext{@Gioele:
    So ugly. Let’s use $\ela\relA\elb$ instead of $\tup{\ela,\ela}\in\relA$ .
}
\begin{table*}[bp]
    \caption{Summary of endorelation properties.}
    \begin{tabular}{cccc}
        Reflexive & Total & Transitive & Symmetric\\[+10pt]
        \prftree{\true}{\tup{\ela,\ela}\in \relA} & \prftree{\true}{\tup{\ela,\elb}\in \relA\vee \tup{\elb,\ela}\in \relA}&
        \prftree{\tup{\ela,\elb}\in \relA}{\tup{\ela,\elb}\in \relA}{\tup{\ela,\elc}\in \relA}& \prftree[double line]{\tup{\ela,\ela'}\in \relA}{\tup{\ela',\ela}\in \relA}\\[+10pt]
        \multicolumn{2}{c}{Asymmetric} & Irreflexive & Antisymmetric\\[+10pt]
        \multicolumn{2}{c}{\prftree{\tup{\ela,\elb}\in \relA}{\tup{\elb,\ela}\in \relA}{\false}}& \prftree{\tup{\ela,\ela}\in \relA}{\false}& \prftree{\tup{\ela,\elb}\in \relA}{\tup{\elb,\ela}\in \relA}{\ela=\elb}
    \end{tabular}
    \label{tab:endo_properties}
\end{table*}


\begin{definition}[Endorelation]
    \label{def:endorelation}
    An \emph{\iindex{endorelation}} on a set~$\setA$ is a relation~$\relA\subseteq \setA\times \setA$.
\end{definition}

\begin{example}
    ``Equality'' is an endorelation of the form
    \begin{equation*}
        \{\tup{\ela,\elb}\in \setA\times \setA \mid \ela=\elb\}.
    \end{equation*}
\end{example}

\begin{example}
    Take~$\setA=\natnumbers$. The relation ``less than or equal'' is an endorelation of the form
    \begin{equation*}
        \{\tup{\ela,\elb}\in \natnumbers\times \natnumbers\mid \ela \leq \elb\}.
    \end{equation*}
\end{example}

\begin{example}
    The relation depicted in \cref{fig:power_internodal} is an endorelation between the set of high voltage nodes.
\end{example}

\begin{definition}[Properties of endorelations]
    \label{def:properties-endorelations}
    An endorelation~$\relA\subseteq \setA\times \setA$ is:
    \begin{compactitem}
        \item \emph{Symmetric} if for~$\ela, \ela'\in \setA$:
        %~$\forall \ela, \ela'\in \setA$:~$ \tup{\ela,\ela'}\in \relA \Leftrightarrow \tup{\ela',\ela}\in \relA$;
        \begin{equation*}
            \prftree[double line]{\tup{\ela,\ela'}\in \relA}{\tup{\ela',\ela}\in \relA}
        \end{equation*}
        \item \emph{Reflexive} if~$\forall \ela\in \setA$:~$\tup{\ela,\ela}\in \relA$;
        \item \emph{Transitive} if for~$\ela, \ela', \ela'' \in \setA$:
        %~$\forall \ela, \ela', \ela'' \in \setA$: $\left( \tup{\ela,\ela'}\in \relA \wedge \tup{\ela',\ela''}\in \relA \right)  \Rightarrow \tup{\ela, \ela''}\in \relA$.
        \begin{equation*}
            \prftree{\tup{\ela,\ela'}\in \relA}{\tup{\ela',\ela''}\in \relA}{\tup{\ela, \ela''}\in \relA}
        \end{equation*}
    \end{compactitem}
\end{definition}

\begin{example}
    The relation ``less than or equal'' on~$\natnumbers$ is not symmetric. It is reflexive since~$n\leq n \ \forall n\in \natnumbers$, and it is transitive since~$l\leq m$ and~$m\leq n$ implies~$l\leq m$.
\end{example}

\begin{example}
    The relation depicted in \cref{fig:power_internodal} is reflexive (each node is connected to itself).
\end{example}
\begin{marginfigure}
    \centering
    \includesag{030_ex_sym_rel}
    \caption{Example of symmetric endorelation.}
    \label{fig:ex_sym_rel}
\end{marginfigure}
\begin{example}
    The endorelation depicted in \cref{fig:ex_sym_rel} is a symmetric relation on~$X=\{x_1,x_2\}$.
\end{example}

\devel{\includepdf[scale=0.8,pages={8},nup=1x3,frame,pagecommand={}]{ACT4E-06-posets.pdf} }% equivalence}

\linkvideo{spring2021-relations:relations:equivalence-rel} % Equivalence relations
\begin{definition}[Equivalence relation]
    \label{def:equivalence-relation}
    An endorelation~$\relA\subseteq \setA\times \setA$ is an \emph{\iindex{equivalence relation}} if it is symmetric, reflexive, and transitive. We write~$\ela\sim \ela'$ if~$\tup{\ela,\ela'}\in \relA$.
\end{definition}

\begin{example}
    The relation ``equals'' on~$\natnumbers$ is an equivalence relation.
    The relation ``less than or equal'' on~$\natnumbers$ is not.
\end{example}

\begin{example}
    The relation ``has the same birthday as'' on the set of all people is an equivalence relation.
    It is symmetric, because if Anna has the same birthday as Bob, then Bob has the same birthday as Anna.
    It is reflexive because everyone has the same birthday as itself.
    It is transitive because if Anna has the same birthday as Bob, and Bob has the same birthday as Clara, then Anna has the same birthday as Clara.
\end{example}

\begin{example}
    Let~$\mapa\colon \setA\to \setB$ be a function between sets.
    The following defines an equivalence relation:
    \begin{equation*}
        \prftree[double line]{\ela\sim \ela'}{\mapa(\ela)=\mapa(\ela')}
        %\ela\sim \ela'\Leftrightarrow \mapa(\ela)=\mapa(\ela').
    \end{equation*}
\end{example}

\begin{definition}[Partition]
    \label{def:partition}
    A \emph{\iindex{partition}} of a set~$\setA$ is a collection~$\{\setA_i\}_{i\in I}$ of subsets~$\setA_i\subseteq \setA$ such that
    \begin{compactenum}
        \item $\setA_i\cap \setA_j=\emptyset \quad \forall i\neq j$;
        \item $\bigcup_{i\in I}\setA_i=\setA$.
    \end{compactenum}
\end{definition}

\begin{remark}
    Equivalence relations are a way to group together elements of a set which we think of as ``the same'' in some respect.
    There is a one-to-one correspondence between equivalence relations on a set~$\setA$ and partitions on~$\setA$.
\end{remark}


\begin{marginfigure}
    \centering
    \includesag{030_information_networks}
    \caption{
        \label{fig:info_network}
    }
\end{marginfigure}

\begin{example}
    An example of partitions can be shown through information networks.
    An exemplary network is depicted in \cref{fig:info_network}.
    Here, nodes represent data centers, and the arrows represent information flows.
    We say that data centers~$x$ and~$y$ are equivalent ($x\sim y$) if and only if there is a path from~$x$ to~$y$ and a path from~$y$ to~$x$.
    In \cref{fig:info_network}, we have that~$a\sim b$,~$e\sim d$, and also every center is equivalent with itself.
\end{example}


\begin{gradedexercise}[\exname{CountingEquivalenceRelations}]
    \label{ex:CountingEquivalenceRelations}
    Let $\setA = \{ 1, 2, 3, 4 \}$.
    How many different equivalence relations are there on $\setA$?
    Explain how you found your answer.
\end{gradedexercise}

\solutionof{CountingEquivalenceRelations}

