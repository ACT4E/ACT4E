% !TEX root = chapter-standalone.tex


\section{Relations}
\label{sec:connection-relations}
\linkvideo{spring2021-relations:relations:rel-def} % Definition of relation
A basic mathematical notion which underlies the above discussion is that of a \textbf{binary relation}.

\begin{definition}[Binary relation]
    \label{def:binary-relation}
    A \emph{\iindex{binary relation}} from a set~$\setA$ to a set~$\setB$ is a subset of the Cartesian product~$\setA\cartprod \setB$.
\end{definition}

\begin{remark}
    We will often drop the word ``binary'' and simply use the name ``relation''.
\end{remark}

\begin{example}
    \label{exa:simple-rel}
    Let~$\setA = \{ \ela_1, \ela_2, \ela_3 \}$ and~$\setB = \{ \elb_1, \elb_2, \elb_3, \elb_4 \}$.
    An example of a relation is the subset
    \begin{equation}
        \relA = \{ \tup{\ela_1, \elb_1}, \tup{\ela_2, \elb_3}, \tup{\ela_1, \elb_4} \} \subseteq \setA \times \setB.
    \end{equation}
\end{example}


\showslides{
    \begin{forslides}
        \includesag{30_rel_identity}
        
        $$\setA = \{ \ela_1, \ela_2, \ela_3 \} \label{eq:slides-1}$$
        
        
        
        $$\setB = \{ \elb_1, \elb_2, \elb_3, \elb_4 \} \label{eq:slides-2}$$
        
        $$\setC = \{ \elc_1, \elc_2  \} \label{eq:slides-4}$$
        
        $$\relA \subseteq \setA \times \setB \label{eq:slides-5} $$
        
        $$\relB \subseteq \setB \times \setC \label{eq:slides-6} $$
        
        
        \begin{equation}
            \relA = \{ \tup{\ela_1, \elb_1}, \tup{\ela_2, \elb_3}, \tup{\ela_1, \elb_4} \} \subseteq \setA \times \setB \label{eq:slides-7}
        \end{equation}
        
        \begin{equation}
            \relB = \{ \tup{\ela_2, \elc_1}, \tup{\ela_2, \elc_2} \} \subseteq \setA \times \setB \label{eq:slides-8}
        \end{equation}
        
        
        \begin{equation}
            \relA \then \relB = \{ \tup{\ela_2, \elc_1}, \tup{\ela_2, \elc_2} \} \subseteq \setA \times \setB \label{eq:slides-9}
        \end{equation}
        
        
        \begin{center}
            $R =$ ``$\leq$'', on $\natnumbers $ \label{eq:slides-10}
        \end{center}
        
        \begin{center}
            $R =$ ``$\leq$'', on $\reals^2 $ \label{eq:slides-11}
        \end{center}
        
        \begin{center}
            $R =$ ``is married to'', on the set of british royals \label{eq:slides-11a}
        \end{center}
        
        
        $$ \setA = \setB = \reals^2 \label{eq:slides-12}$$
        
        
        \begin{center}
            $\relA =$ ``$\leq$''  \label{eq:slides-13}
        \end{center}
        
        $$ \tup{\ela_1, \ela_2} \leq \tup{\elb_1, \elb_2} \quad \Leftrightarrow \quad \ela_1 \leq \elb_1 \text{ and } \ela_2 \leq \elb_2 \label{eq:slides-14}$$
        
        \begin{equation}
            \ela \relA \elb \definedas \tup{\ela, \elb} \in \relA  \label{eq:slides-15}
        \end{equation}
        
        $$ \ela_1 \relA \elb_1 \label{eq:slides-16}$$
        
        $$ \ela_2 \relA \elb_3 \label{eq:slides-17}$$
        
        $$ \ela_2 \relA \elb_4 \label{eq:slides-18}$$
        
        \
        
        A relation $\relA \subseteq \setA \times \setB$ is ``the same thing'' as a function $\phi : \setA \times \setB \sto \{\true, \false\}$.
        
        \
        
        
        A relation $\relA \subseteq \setA \times \setB$ is ``the same thing'' as a function $h : \setA  \sto \powerset (\setB)$.
        
        \
        
        $$
        f : X \to Y
        $$
        
        
        $$
        R_f \definedas \{ \tup{x, y} \in X \times Y \mid y = f(x) \}
        $$
    
    \end{forslides}
}


\begin{marginfigure}
    \centering
    \includesag{30_rel_1}
    \caption{}
    \label{fig:example_rel}
\end{marginfigure}

If~$\setA$ and~$\setB$ are finite sets, we can depict a relation~$\relA \subseteq \setA \cartprod \setB$ graphically as in \cref{fig:example_rel}.
For each element~$\tup{\ela,\elb} \in \setA \cartprod \setB$, we draw an arrow from~$\ela$ to~$\elb$ if and only if~$\tup{\ela,\elb} \in \relA \subseteq \setA \cartprod \setB$.
\begin{remark}[Notation for relations]
    From now on, instead of explicitly writing~$\tup{\ela,\elb}\in \relA\subseteq \setA\cartprod \setB$, we adopt the notation~$\inrel{\ela}{\relA}{\elb}$.
\end{remark}


\begin{marginfigure}
    \centering
    \includesag{30_rel_graph}
    \caption{Relations visualized in ``coordinate systems''.}
    \label{fig:example_rel_coord}
\end{marginfigure}

We can also depict this relation graphically as a subset of~$\setA \cartprod \setB$ in a ``coordinate system way'', as in \cref{fig:example_rel_coord}.

The shaded grey area is the subset~$\relA$ defining the relation.


The visualization in \cref{fig:example_rel} hints at the fact that we can think of a relation~$\relA \subseteq \setA \cartprod \setB$ as a \emph{morphism} from~$\setA$ to~$\setB$.

\linkvideo{spring2021-relations:relations:comp-rel} % Composing relations
\linkvideo{spring2021-relations:relations:cat-rel} % Category of relations

\begin{ctdefinition}[Category \Rel]
    \label{def:Rel}
    The category of relations \iindex{\Rel}  is given by:
    \begin{compactenum}
        \item \emph{Objects}: The objects of this category are all sets.
        \item \emph{Morphisms}: Given sets~$\Obja, \Objb$, the homset~$\HomSet{\Rel}{\Obja}{\Objb}$ consists of all
        relations~$\relA\subseteq \Obja\times \Objb$.
        \item \emph{Identity morphisms}: Given a set~$\Obja$, its identity morphism is
        \begin{equation}
            \catid_\Obja \definedas \{ \tup{\ela,\ela'} \in \Obja \cartprod \Obja \mid  \ela = \ela' \}.
        \end{equation}
        \item \emph{Composition}: Given relations~$\relA \colon \Obja\mto \Objb$,~$\relB \colon \Objb\mto \Objc$, their composition is given by
        \begin{equation}
            \label{eq:RelCompRule}
            \relA \mthen \relB \definedas \{\tup{\ela,\elc} \in \Obja \cartprod \Objc \mid  \exists \elb \in \Objb \colon \left(\inrel{\ela}{\relA}{\elb} \right) \booland \left(\inrel{\elb}{\relB}{\elc}\right)\}.
        \end{equation}
    \end{compactenum}
\end{ctdefinition}

To illustrate the composition rule in \cref{eq:RelCompRule} for relations, let us consider a simple example, involving sets~$\setA$,~$\setB$, and~$\setC$, and relations~$\relA \colon \setA \to \setB$ and~$\relB \colon \setB \to \setC$, as depicted graphically below in \cref{fig:example_rel_composable}.
%
\begin{figure}[h!]
    \centering
    \includesag{30_rel_2}
    \caption{Relations compatible for composition.}
    \label{fig:example_rel_composable}
\end{figure}
%
Now, according to the rule in \cref{eq:RelCompRule}, the composition~$\relA \mthen \relB \subseteq \setA \cartprod \setC$ will be such that~$\inrel{\ela}{\relA \mthen \relB}{\elc}$ if and only if there exists some~$\elb \in \setB$ such that~$\inrel{\ela}{\relA}{\elb}$ and~$\inrel{\elb}{\relB}{\elc}$, which, graphically, means that for~$\tup{\ela,\elc}$ to be an element of the relation~$\relA \mthen \relB$,~$\ela$ and~$\elb$ need to be connected by at least one sequence of two arrows such that the target of the first arrow is the source of the second.
For example, in \cref{fig:example_rel_composable}, there is an arrow from~$\ela_2$ to~$\elb_3$, and from there on to~$\elc_1$, and therefore, in the composition~$\relA \mthen \relB$ depicted in \cref{fig:example_rel_composed}, there is an arrow from~$\ela_2$ to~$\elc_1$.
%
\begin{figure}[h!]
    \centering
    \includesag{30_rel_3}
    \caption{Composition of relations.}
    \label{fig:example_rel_composed}
\end{figure}

\begin{remark}
    Relations with the same source and target can be \emph{compared} via inclusion.
    Given~$\relA \subseteq \setA\times \setB$ and~$\relA'\subseteq \setA\times \setB$, we can ask whether~$\relA\subseteq \relA'$ or~$\relA'\subseteq \relA$.
\end{remark}

\devel {
    \begin{definition}[Transitive closure]
        \label{def:transitive-closure}
        \todotext{Write definition of transitive closure}
    \end{definition}
}
