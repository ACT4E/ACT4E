% !TEX root = standalone.tex


\section{Relations}
\label{sec:connection-relations}

A basic mathematical notion which underlies the above discussion is that of a \textbf{binary relation}.

\todographics{Need to change a lot of symbols. Use $\setA, \setB$ for sets, $\relA, \relB$ for the relation}
\begin{definition}[Binary relation]
  \label{def:binary-relation}
  A \emph{\iindex{binary relation}} from a set~$\setA$ to a set~$\setB$ is a subset of the Cartesian product $\setA\cartprod \setB$.
\end{definition}

\begin{remark}
  We will often drop the word ``binary'' and simply use the name ``relation''.
\end{remark}


If~$\setA$ and~$\setB$ are finite sets, we can depict a relation~$\relA \subseteq \setA \cartprod \setB$ graphically as in \cref{fig:example_rel}. For each element~$\tup{\ela,\elb} \in \setA \cartprod \setB$, we draw an arrow from~$\ela$ to~$\elb$ if and only if~$\tup{\ela,\elb} \in \relA \subseteq \setA \cartprod \setB$.

\begin{figure}[h!]
  \centering
  \includesag{30_rel_1}
  \caption{}
  \label{fig:example_rel}
\end{figure}

We can also depict this relation graphically as a subset of~$\setA \cartprod \setB$ in a ``coordinate system way'', as in \cref{fig:example_rel_coord}. The shaded grey area is the subset~$\relA$ defining the relation.

\begin{figure}[h!]
  \begin{center}
    \includesag{30_rel_graph}
  \end{center}
  \caption{Relations visualized in ``coordinate systems''.}
  \label{fig:example_rel_coord}
\end{figure}

\begin{exercise}
  Let~$\setA = 7setB = \{1, 2, 3, 4 \}$ and consider the relation~$\relA \subseteq \setA \cartprod \setB$ defined by
  \begin{equation}
    \relA = \{ \tup{\ela,\elb} \in \setA \cartprod \setB \mid \ela \leq \elb \}.
  \end{equation}

  Visualize the relation~$\relA$ via the method in \cref{fig:example_rel} and \cref{fig:example_rel_coord} each.
\end{exercise}

The visualization in \cref{fig:example_rel} hints at the fact that we can think of a relation~$\relA \subseteq \setA \cartprod \setB$ as a \emph{morphism} from~$\setA$ to~$\setB$.

\begin{ctdefinition}[Category \Rel]
  \label{def:Rel}
  The category of relations \iindex{\Rel}  is given by:
  \begin{compactenum}
    \item \emph{Objects}: The objects of this category are all sets.
    \item \emph{Morphisms}: Given sets~$\Obja, \Objb$, the homset~$\HomSet{\Rel}{\Obja}{\Objb}$ consists of all
    relations~$\relA\subseteq \Obja\times \Objb$.
    \item \emph{Identity morphisms}: Given a set~$\Obja$, its identity morphism is
    \begin{equation}
      \catid_\Obja \definedas \{ \tup{\ela,\ela} \mid  \ela \in \Obja \}.
    \end{equation}
    \item \emph{Composition}: Given relations~$\relA \colon \Obja\mto \Objb$,~$\relB \colon \Objb\mto \Objc$, their composition is given by
    \begin{equation}
      \label{eq:RelCompRule}
      \relA \then \relB \definedas \{\tup{\ela,\elc} \mid  \exists \elb \in \Objb \colon \ \left(\tup{\ela,\elb} \in \relA\right) \wedge \left(\tup{\elb,\elc} \in \relB\right)\}.
    \end{equation}
  \end{compactenum}
\end{ctdefinition}

To illustrate the composition rule in \cref{eq:RelCompRule} for relations, let's consider a simple example, involving sets~$\setA$,~$\setB$, and~$\setC$, and relations~$\relA \colon \setA \to \setB$ and $\relB \colon \setB \to \setC$, as depicted graphically below in \cref{fig:example_rel_composable}.
%
\begin{figure}[h!]
  \centering
  \includesag{30_rel_2}
  \caption{Relations compatible for composition.}
  \label{fig:example_rel_composable}
\end{figure}
%
Now, according to the rule in \cref{eq:RelCompRule}, the composition~$\relA \then \relB \subseteq \setA \times \setC$ will be such that~$\tup{\ela,\elc} \in \relA \then \relB$ if and only if there exists some~$\elb \in \setB$ such that~$\tup{\ela,\elb} \in \relA$ and~$\tup{\elb,\elc} \in \relB$, which, graphically, means that for~$\tup{\ela,\elc}$ to be an element of the relation~$\relA \then \relB$,~$\ela$ and~$\elb$ need to be connected by at least one sequence of two arrows such that the target of the first arrow is the source of the second.
For example, in \cref{fig:example_rel_composable}, there is an arrow from~$\ela_2$ to~$\elb_3$, and from there on to~$\elc_1$, and therefore, in the composition~$\relA \then \relB$ depicted in \cref{fig:example_rel_composed}, there is an arrow from~$\ela_2$ to~$\elc_1$.
%
\begin{figure}[h!]
  \centering
  \includesag{30_rel_3}
%\includegraphics[width=0.5\linewidth]{pics/dist_net_10.png}
  \caption{Composition of relations.}
  \label{fig:example_rel_composed}
\end{figure}

\begin{remark}
  Relations with the same source and target can be \emph{compared} via inclusion.
  Given~$\relA \subseteq \setA\times \setB$ and~$\relA'\subseteq \setA\times \setB$, we can ask whether~$\relA\subseteq \relA'$ or~$\relA'\subseteq \relA$.
\end{remark}
