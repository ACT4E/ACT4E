% !TEX root = chapter-standalone.tex

\section{Relations}
\label{sec:connection-relations}
\linkvideo{spring2021-relations:relations:rel-def} % Definition of relation
A basic mathematical notion which underlies the above discussion is that of a \textbf{binary relation}.

\begin{ctdefinition}[Binary relation]
    \label{def:binary-relation}
    A \emph{\iindex{binary relation}} from a set~$\setA$ to a set~$\setB$ is a subset of the Cartesian product~$\setA\cartprod \setB$.
\end{ctdefinition}

\begin{remark}
    We will often drop the word ``binary'' and simply use the name ``relation''.
\end{remark}

\begin{marginfigure}
    \centering
    \includesag{30_rel_1}
    \caption{}
    \label{fig:example_rel}
\end{marginfigure}

\begin{example}
    \label{exa:simple-rel}
    Let~$\setA = \{ \sbretzel, \sfondue, \schoco \}$ and~$\setB = \{ \stea, \smilk, \swater, \swine \}$.
    An example of a relation is the subset
    \begin{equation}
        \relA = \{ \tup{\sbretzel, \stea}, \tup{\sfondue, \swater}, \tup{\sbretzel, \swine} \} \subseteq \setA \cartprod \setB.
    \end{equation}
\end{example}

\showslides{
    \begin{forslides}
        \includesag{30_rel_identity}

        $$\setA = \{ \sbretzel, \sfondue, \schoco \} \label{eq:slides-1}$$

        $$\setB = \{ \stea, \smilk, \swater, \swine \} \label{eq:slides-2}$$

        $$\setC = \{ \elc_1, \elc_2  \} \label{eq:slides-4}$$

        $$\relA \subseteq \setA \cartprod \setB \label{eq:slides-5} $$

        $$\relB \subseteq \setB \cartprod \setC \label{eq:slides-6} $$

        \begin{equation}
            \relA = \{ \tup{\sbretzel, \stea}, \tup{\sfondue, \swater}, \tup{\sbretzel, \swine} \} \subseteq \setA \cartprod \setB \label{eq:slides-7}
        \end{equation}
        %
        \begin{equation}
            \relB = \{ \tup{\sfondue, \elc_1}, \tup{\sfondue, \elc_2} \} \subseteq \setA \cartprod \setB \label{eq:slides-8}
        \end{equation}
        %
        \begin{equation}
            \relA \then \relB = \{ \tup{\sfondue, \elc_1}, \tup{\sfondue, \elc_2} \} \subseteq \setA \cartprod \setB \label{eq:slides-9}
        \end{equation}
        %
        \begin{center}
            $R =$ ``$\leq$'', on $\natnumbers $ \label{eq:slides-10}
        \end{center}

        \begin{center}
            $R =$ ``$\leq$'', on $\reals^2 $ \label{eq:slides-11}
        \end{center}

        \begin{center}
            $R =$ ``is married to'', on the set of british royals \label{eq:slides-11a}
        \end{center}

        $$ \setA = \setB = \reals^2 \label{eq:slides-12}$$

        \begin{center}
            $\relA =$ ``$\leq$''  \label{eq:slides-13}
        \end{center}

        $$ \tup{\sbretzel, \sfondue} \leq \tup{\stea, \smilk} \quad \Leftrightarrow \quad \sbretzel \leq \stea \text{ and } \sfondue \leq \smilk \label{eq:slides-14}$$
        \begin{equation}
            \ela \relA \elb \definedas \tup{\ela, \elb} \in \relA  \label{eq:slides-15}
        \end{equation}
        %
        $$ \sbretzel \relA \stea \label{eq:slides-16}$$

        $$ \sfondue \relA \swater \label{eq:slides-17}$$

        $$ \sfondue \relA \swine \label{eq:slides-18}$$

        \

        A relation $\relA \subseteq \setA \cartprod \setB$ is ``the same thing'' as a function $\phi : \setA \cartprod \setB \sto \{\true, \false\}$.

        \

        A relation $\relA \subseteq \setA \cartprod \setB$ is ``the same thing'' as a function $h : \setA  \sto \powerset (\setB)$.

        \

        $$
            f : X \to Y
        $$

        $$
            R_f \definedas \{ \tup{x, y} \in X \cartprod Y \mid y = f(x) \}
        $$

    \end{forslides}
}

If~$\setA$ and~$\setB$ are finite sets, we can depict a relation~$\relA \subseteq \setA \cartprod \setB$ graphically as in \cref{fig:example_rel}.
For each element~$\tup{\ela,\elb} \in \setA \cartprod \setB$, we draw an arrow from~$\ela$ to~$\elb$ if and only if~$\tup{\ela,\elb} \in \relA$.

\begin{marginfigure}
    \centering
    \includesag{30_rel_graph}
    \caption{Relations visualized in ``coordinate systems''.}
    \label{fig:example_rel_coord}
\end{marginfigure}

We can also depict this relation graphically as a subset of~$\setA \cartprod \setB$ in a ``coordinate system way'', as in \cref{fig:example_rel_coord}.

The shaded grey area is the subset~$\relA$ defining the relation.

\begin{remark}[Notation for relations]
    From now on, instead of explicitly writing~$\tup{\ela,\elb}\in \relA$, we adopt the notation~$\inrel \ela \relA \elb$:
    \begin{equation}
        \inrel \ela \relA \elb \definedas \tup{\ela,\elb}\in \relA.
    \end{equation}
\end{remark}

\vfill
\begin{gradedexercise}[\exname{VisualizeLeqRelation}]
    \label{ex:visualize-leq-relation}
    Let~$\setA = \setB = \{1, 2, 3, 4 \}$ and consider the relation~$\relA \subseteq \setA \cartprod \setB$ defined by
    \begin{equation}
        \relA = \{ \tup{\ela,\elb} \in \setA \cartprod \setB \mid \ela \leq \elb \}.
    \end{equation}
    %
    Visualize the relation~$\relA$ via the method in \cref{fig:example_rel} and \cref{fig:example_rel_coord} each.
\end{gradedexercise}

\solutionof{VisualizeLeqRelation}

\section{Composing relations}

\linkvideo{spring2021-relations:relations:comp-rel} % Composing relations

The visualization in \cref{fig:example_rel} hints at the fact that we can compose relations
if they have compatible source and target.

To illustrate the composition rule in \cref{eq:RelCompRule} for relations, let us consider a simple example, involving sets~$\setA$,~$\setB$, and~$\setC$, and relations~$\relA \colon \setA \to \setB$ and~$\relB \colon \setB \to \setC$, as depicted graphically below in \cref{fig:example_rel_composable}.
%
\begin{figure}[h!]
    \centering
    \subfloat[Relations compatible for composition. \label{fig:example_rel_composable}]{
        \includesag{30_rel_2}}\\
    \subfloat[Composition of relations. \label{fig:example_rel_composed}]{
        \includesag{30_rel_3}}
    \caption{Illustrations for relations composition.}

\end{figure}
%
Now, according to the rule in \cref{eq:RelCompRule}, the composition~$\relA \mthen \relB \colon \setA \mto \setC$ will be such that~$\inrel \ela {(\relA \mthen \relB)} \elc$ if and only if there exists some~$\elb \in \setB$ such that~$\inrel \ela \relA \elb $ and~$\inrel \elb \relB{\elc}$, which, graphically, means that for~$\tup{\ela,\elc}$ to be an element of the relation~$\relA \mthen \relB$,~$\ela$ and~$\elb$ need to be connected by at least one sequence of two arrows such that the target of the first arrow is the source of the second.
For example, in \cref{fig:example_rel_composable}, there is an arrow from~$\sfondue$ to~$\swater$, and from there on to~$\sapple$, and therefore, in the composition~$\relA \mthen \relB$ depicted in~\cref{fig:example_rel_composed}, there is an arrow from~$\sfondue$ to~$\sapple$.

\section{The category of relations \Rel}

\linkvideo{spring2021-relations:relations:cat-rel} % Category of relations

Now that we know how relations compose, it is natural to think of  a relation~$\relA \subseteq \setA \cartprod \setB$ as a \emph{morphism} from~$\setA$ to~$\setB$.

\begin{ctdefinition}[Category \Rel]
    \label{def:Rel}
    The category of relations \iindex{\Rel}  is given by:
    \begin{enumerate}
        \item \emph{Objects}: The objects of this category are all sets.
        \item \emph{Morphisms}: Given sets~$\Obja, \Objb$, the homset~$\HomSet{\Rel}{\Obja}{\Objb}$ consists of all relations~$\relA\subseteq \Obja\cartprod \Objb$.
        \item \emph{Identity morphisms}: Given a set~$\Obja$, its identity morphism is
              \begin{equation}
                  \catid_\Obja \definedas \{ \tupp{\ela,\elb} \in \Obja \cartprod \Obja \mid  \ela = \elb \}.
              \end{equation}
        \item \emph{Composition}: Given relations~$\relA \colon \Obja\mto \Objb$,~$\relB \colon \Objb\mto \Objc$, their composition is given by
              \begin{equation}
                  \label{eq:RelCompRule}
                  \relA \mthen \relB \definedas \{\tup{\ela,\elc} \in \Obja \cartprod \Objc \mid  \exists \elb \in \Objb \colon \left(\inrel{\ela}{\relA}{\elb} \right) \booland \left(\inrel{\elb}{\relB}{\elc}\right)\}.
              \end{equation}
    \end{enumerate}
\end{ctdefinition}

\begin{remark}
    Relations with the same source and target can be \emph{compared} via inclusion.
    Given~$\relA, \relB \colon \setA\mto \setB$  we can ask whether~$\relA\subseteq \relB$ or~$\relB\subseteq \relA$ (or neither).
\end{remark}
