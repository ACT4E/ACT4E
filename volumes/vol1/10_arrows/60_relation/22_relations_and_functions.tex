% !TEX root = chapter-standalone.tex


\section{Relations and functions}

Every function between sets can be thought as a relation. For example, consider the sets $\setA = \{ \ela_1, \ela_2, \ela_3 \}$ and $\setB = \{ \elb_1, \elb_2, \elb_3, \elb_4 \}$, and the function $\mapa : \setA \sto \setB$ defined by 
\begin{equation}
\mapa (\ela_1) = \elb_1, \ \mapa (\ela_2) = \elb_3, \ \mapa(\ela_3) = \elb_4.
\end{equation}
This can be depicted graphically as in [MAKE AND REF FIGURE] and can be written as the following relation
\begin{equation}
\{ \tup{\ela_1, \elb_1}, \tup{\ela_2, \elb_3}, \tup{\ela_3, \elb_4} \} \subseteq \setA \times \setB.
\end{equation}
This relation [REF] that we associated to the function $\mapa$ may be thought of as its \emph{graph}. That it, is the of tuples in $\setA \times \setB$ which are each a pairing of an element of the source set $\setA$ with the element which is its image under $\mapa]$. See [MAKE AND REF FIGURE]. 

Not every relation comes from a function in this way, however. Part of the definition of the notion of a ``function'' is that it maps \emph{every} element of its source set to some element of its target. And a function must map each element of it source to \emph{exactly one} element of its target (``multivalued functions'' are not functions in the strict sense). So, for example, we see that the relation in [REF FIGURE] does not arise from a function. Not every element of the source set is related to some element of the target ($\ela_3$ is not related to anything). Nor is the relation ``single-valued'': the element $\ela_2$ is related to two distinct elements of the target set.

We can however think about relations \emph{in terms of} functions. Here are three ways: 
\begin{enumerate}
\item We can think of a relation $\relA \subseteq \setA \times \setB$ as a function $\setA \times \setB \sto \{ \false, \true \}$, where we think of ``$\false$'' as ``false'' and ``$\true$'' as ``true''. 

Given $\relA$, we can define a function $\phi_{\relA} : \setA \times \setB \sto \{ \false, \true \}$ from it by setting 
\begin{equation}
\phi_{\relA}(\tup{\ela, \elb}) = 
\left 
\{\begin{array}{lr}
        \true & \text{if } \tup{\ela, \elb} \in \relA \\
        \false & \text{if } \tup{\ela, \elb} \notin \relA.
        \end{array}
\right. 
\end{equation}
Conversely, given a function $\phi: \setA \times \setB \sto \{ \false, \true \}$ we can define a relation $\relA_{\phi} \subseteq \setA \times \setB$ from it by setting 
\begin{equation}
\relA_{\phi} = \{ \tup{\ela, \elb} \in \setA \times \setB \mid \phi_{\relA}(\tup{\ela, \elb}) = \true \}.
\end{equation}
These two constructions are inverse to one-another. 

\item We can think of a relation $\relA \subseteq \setA \times \setB$ as a function $\setA  \sto \powerset (\setB)$. 

Given $\relA$, we can define a function $\hat \phi_{\relA} : \setA \sto \powerset (\setB)$ via 
\begin{equation}
\hat \phi_{\relA} (\ela) = \{ \elb \in \setB \mid \tup{\ela, \elb} \in \relA \}. 
\end{equation}
Conversely, given a function $\hat \phi : \setA \sto \powerset (\setB)$, we can define 
\begin{equation}
\relA_{\hat \phi} = \{ \tup{\ela, \elb} \in \setA \times \setB \mid \elb \in \hat \phi_{\relA}(\ela)   \}.
\end{equation}
These two constructions are inverse to one another, too. 

\item We can think of a relation $\relA \subseteq \setA \times \setB$ as a function $\setB  \sto \powerset (\setA)$. 

Given $\relA$, we can define a function $\check \phi_{\relA} : \setB \sto \powerset (\setA)$ via 
\begin{equation}
\check \phi_{\relA} (\elb) = \{ \ela \in \setA \mid \tup{\ela, \elb} \in \relA \}. 
\end{equation}
Conversely, given a function $\check \phi : \setB \sto \powerset (\setA)$, we can define 
\begin{equation}
\relA_{\check \phi} = \{ \tup{\ela, \elb} \in \setA \times \setB \mid \ela \in \check \phi_{\relA}(\elb)   \}.
\end{equation}
These two constructions are \emph{also} inverse to one another. 

\end{enumerate}

\begin{gradedexercise}
For the relation [REF], write out the three functions that describe it, respectively, in the three ways outlined in [REF]. 
\end{gradedexercise}
