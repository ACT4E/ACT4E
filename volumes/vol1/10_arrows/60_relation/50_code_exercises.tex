\section{Code exercises}

\subsection*{Interface}

A finite relation \classname{FiniteRelation} ...

\classlisting{FiniteRelation}{}

\subsection*{Representation}{}

The format is shown in \cref{fig:rel1}.

\datafilefig{rel11code}{rel1.finrel.yaml}{fig:rel1}

\begin{gradedexercise}[Loading finite relations]
  Create a function to load the data.


%
  \classlisting{FiniteRelationRepresentation}{}

\end{gradedexercise}


%  \input{10_arrows/10_composition/10_topic1}

\begin{gradedexercise}[Relation properties]

  Check that an endorelation has the properties mentioned in \cref{sec:relations}.



\end{gradedexercise}

\classlisting{FiniteRelationProperties}{}


\begin{gradedexercise}[Opposite]
  Compute the opposite relation.

  \methodsource{FiniteEndorelationOperations}{transitive_closure}{}

\end{gradedexercise}

\begin{gradedexercise}[Endorelation properties]

  Check that an endorelation has the properties mentioned in \cref{sec:endorelations}.

  The interfaces to implement:


\end{gradedexercise}
\classlisting{FiniteEndorelationProperties}{}
\begin{gradedexercise}[Transitive closure]
  Compute the transitive closure (\cref{def:transitive-closure}) of a relation.

  \methodsource{FiniteEndorelationOperations}{transitive_closure}{}

\end{gradedexercise}
