\section{Code exercises}

\subsection*{Interface}

A finite relation \classname{FiniteRelation} \dots \XXX

\classlisting{FiniteRelation}{}

\subsection*{Representation}{}

The format is shown in \cref{fig:rel1}.

\datafilefig{rel11code}{rel1.finrel.yaml}{fig:rel1}

\begin{codeexercise}[Loading finite relations]
    Create a function to load the data.
    %
    \classlisting{FiniteRelationRepresentation}{}

\end{codeexercise}

%  \input{10_arrows/10_composition/10_topic1}

\begin{codeexercise}[Endorelation properties]

    Check that an endorelation has the properties mentioned in \cref{sec:endorelations}.

\end{codeexercise}

\classlisting{FiniteRelationProperties}{}

\begin{codeexercise}[Opposite]
    Compute the opposite relation.

    \methodsource{FiniteEndorelationOperations}{transitive_closure}{}

\end{codeexercise}

\begin{codeexercise}[Endorelation properties]

    Check that an endorelation has the properties mentioned in \cref{sec:endorelations}.

    The interfaces to implement:

\end{codeexercise}
\classlisting{FiniteEndorelationProperties}{}
\begin{codeexercise}[Transitive closure]
    Compute the transitive closure (\cref{def:transitive-closure}) of a relation.

    \methodsource{FiniteEndorelationOperations}{transitive_closure}{}

\end{codeexercise}
