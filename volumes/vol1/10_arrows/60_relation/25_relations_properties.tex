% !TEX root = chapter-standalone.tex


\section{Properties of relations}

We have seen that relations generalize functions -- every function defines a relation, via its graph, but not every relation comes from a function in this way. Many notions that we are familiar with for functions also generalize to relations. Here are a few. 

\begin{definition}[Properties of a relation]
  \label{def:rel_properties}
  Let~$\relA\subseteq \setA \times \setB$ be a relation.~$\relA$ is:
  \begin{compactenum}
    \item \emph{Surjective} if for all~$\elb\in \setB$ there exists an~$\ela\in \setA$ such that~$\tup{\ela,\elb}\in \relA$;
    \item \emph{Injective} if for all~$ \tup{\ela_1,\elb_1},\tup{\ela_2,\elb_2}\in \relA$ it holds:~$\elb_2=\elb_2 \Imp \ela_1=\ela_2$;
    \item \emph{Everywhere-defined} if for all~$\ela\in \setA$ there exists an ~$\elb \in \setB\colon \tup{\ela,\elb}\in \relA$;
    \item \emph{Single-valued} if~$\forall \tup{\ela,\elb_1},\tup{\ela_2,\elb_2}\in \relA$ it holds:~$\ela_1=\ela_2\Imp \elb_1=\elb_2$.
  \end{compactenum}
\end{definition}

\begin{example}
  The relation depicted in \cref{fig:example_rel} is injective but not surjective. It is not single-valued, nor everywhere-defined. \end{example}

One can notice a certain duality in the properties listed in \cref{def:rel_properties}. This is made more precise through the following definition.

\begin{definition}[Transpose of a relation]
  \label{def:relation-transpose}
  Let~$\relA\subseteq \setA\times \setB$ be a relation. The \emph{transpose} (or \emph{opposite}, \emph{reverse}) of~$\relA$ is the relation given by:
  \begin{equation*}
    \relA\reltransp \definedas \{\tup{\elb,\ela}\in \setB\times \setA \mid \tup{\ela,\elb}\in \relA \}.
  \end{equation*}
  note that~$\relA\reltransp\colon \setB\to \setA$, while~$\relA\colon \setA\to \setB$.
\end{definition}
\begin{remark}\label{re:rel-op-properties}
  Some useful properties of a relation $\relA\colon \setA\to \setB$  and its opposite $\relA\reltransp\colon \setB\to \setA$:  \begin{enumerate}
    \item $\left(\relA\reltransp\right)\reltransp = \relA $;
    \item If~$\relA$ is everywhere-defined, then~$\relA\reltransp$ is surjective;
    \item If~$\relA$ is single-valued, then~$\relA\reltransp$ is injective.
    \item $\relA$ is everywhere-defined if and only if~$\catid_\setA\subseteq \relA \then \relA\reltransp$;
    \item $\relA$ is single-valued if and only if~$\relA\reltransp\then \relA\subseteq \catid_\setB$.
  \end{enumerate}
\end{remark}

\begin{gradedexercise}[\exname{RelProperties}]\label{ex:RelProperties}
Provide a proof of each of the properties listed in \cref{re:rel-op-properties}.
\end{gradedexercise}
\solutionof{RelProperties}



\begin{remark}
  The aforementioned duality can be seen by ``reading the relations (arrows) backwards'' (\cref{fig:rel_transpose}).
\end{remark}

\begin{figure}[h!]
  \centering
  \includesag{030_rel_transpose}
  \caption{}
  \label{fig:rel_transpose}
\end{figure}
