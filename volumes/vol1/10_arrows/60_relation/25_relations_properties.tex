% !TEX root = chapter-standalone.tex

\section{Properties of relations}
\linkvideo{spring2021-relations:relations:prop-rel} % Properties of relations

% \linkvideo{spring2021-tradeoffs:relations-properties} % Recap of relations
We have seen that relations generalize functions -- every function defines a relation, via its graph, but not every relation comes from a function in this way.
Many notions that we are familiar with for functions also generalize to relations.
Here are a few.

\begin{ctdefinition}[Properties of a relation]
    \label{def:rel_properties}
    Consider a relation $\relA \colon \setA \mto \setB$.
    We say that~$\relA$ is:
    \begin{enumerate}
        \item \emph{Injective} if
              \begin{equation}
                  \prfsemi{
                      \inrel \ela \relA \elb
                  }{
                      \inrel \elc \relA \elb
                  }{
                      \ela=\elc
                  }
              \end{equation}
        \item \emph{Single-valued} if
              \begin{equation}
                  \prfsemi{
                      \inrel \ela \relA \elb
                  }{
                      \inrel \ela \relA \eld
                  }{
                      \elb = \eld
                  }
              \end{equation}
        \item \emph{Surjective} if for all~$\elb\in \setB$ there exists an~$\ela\in \setA$:~$\inrel{\ela}{\relA}{\elb}$;
        \item \emph{Everywhere-defined} if for all~$\ela\in \setA$ there exists an ~$\elb \in \setB\colon \inrel{\ela}{\relA}{\elb}$.
    \end{enumerate}
\end{ctdefinition}

\begin{example}
    The relation depicted in \cref{fig:example_rel} is injective but not surjective.
    It is not single-valued, nor everywhere-defined.
\end{example}

\section{Transpose of a relation}
\linkvideo{spring2021-relations:relations:trans-rel} % Transposition of relations

One can notice a certain duality in the properties listed in \cref{def:rel_properties}.
This is made more precise through the following definition.

\begin{ctdefinition}[Transpose of a relation]
    \label{def:relation-transpose}
    Let~$\relA\colon \setA\mto \setB$ be a relation.
    We define its \emph{transpose} (or \emph{opposite}, or \emph{reverse}) $\relA\reltransp\colon \setB \mto \setA$ as follows:
    \begin{equation*}
        \prfdoubleperiod{
            \inrel \ela \relA \elb
        }{
            \inrel \elb {\relA\reltransp} \ela
        }
    \end{equation*}
    % Note that~$\relA\reltransp\colon \setB\to \setA$, while~$\relA\colon \setA\to \setB$.
\end{ctdefinition}
\begin{remark}
    \label{re:rel-op-properties}
    Here are some useful properties of a relation~$\relA\colon \setA\to \setB$  and its opposite~$\relA\reltransp\colon \setB\to \setA$:
    \begin{enumerate}
        \item $\left(\relA\reltransp\right)\reltransp = \relA $;
        \item If~$\relA$ is everywhere-defined if and only if~$\relA\reltransp$ is surjective;
        \item If~$\relA$ is single-valued if and only if~$\relA\reltransp$ is injective.
        \item $\relA$ is everywhere-defined if and only if~$\catid_\setA\subseteq \relA \mthen \relA\reltransp$;
        \item $\relA$ is single-valued if and only if~$\relA\reltransp\mthen \relA\subseteq \catid_\setB$.
    \end{enumerate}
\end{remark}

\begin{marginfigure}
    \centering
    \includesag{030_rel_transpose}
    \caption{}
    \label{fig:rel_transpose}
\end{marginfigure}

\begin{marginfigure}
    \centering
    \includesag{030_rel_transpose_bis}
    \caption{}
    \label{fig:rel_transpose_bis}
\end{marginfigure}

\begin{remark}
    The aforementioned duality can be seen by ``reading the relations (arrows) backwards'' (\cref{fig:rel_transpose,fig:rel_transpose_bis}).
\end{remark}

\vfill
\begin{gradedexercise}[\exname{RelProperties}]
    \label{ex:RelProperties}
    Provide a proof of each of the properties listed in \cref{re:rel-op-properties}.
\end{gradedexercise}
\solutionof{RelProperties}

