% !TEX root = standalone.tex


\section{Properties of relations}

\todostructure{Re-do this part following the lecture exposition}

\begin{definition}[Properties of a relation]
  \label{def:rel_properties}
  Let~$\relA\subseteq \setA \times \setB$ be a relation.~$\relA$ is:
  \begin{compactenum}
    \item \emph{Surjective} if for all $\elb\in \setB$ there exists an~$\ela\in \setA$ such that~$\tup{\ela,\elb}\in \relA$;
    \item \emph{Injective} if for all~$ \tup{\ela_1,\elb_1},\tup{\ela_2,\elb_2}\in \relA$ it holds:~$\elb_2=\elb_2 \Imp \ela_1=\ela_2$;
    \item \emph{Defined-everywhere} if for all~$\ela\in \setA$ there exists an ~$\elb \in \setB\colon \tup{\ela,\elb}\in \relA$;
    \item \emph{Single-valued} if~$\forall \tup{\ela,\elb_1},\tup{\ela_2,\elb_2}\in \relA$ it holds:~$\ela_1=\ela_2\Imp \elb_1=\elb_2$.
  \end{compactenum}
\end{definition}

\begin{example}
  The relation depicted in \cref{fig:example_rel} is injective but not surjective: if~$\tup{\ela,\elb},\tup{\ela',\elb'}\in \relA$ and~$\elb=\elb'$, then~$\ela=\ela'$.
\end{example}

One can notice a certain duality in the properties listed in \cref{def:rel_properties}. This is made more precise through the following definition.

\begin{definition}[Transpose of a relation]
  \label{def:relation-transpose}
  Let~$\relA\subseteq \setA\times \setB$ be a relation. The \emph{transpose} (or \emph{opposite}, \emph{reverse}) of~$\relA$ is the relation given by:
  \begin{equation*}
    \relA\reltransp \definedas \{\tup{\elb,\ela}\in \setB\times \setA \mid \tup{\ela,\elb}\in \relA \}.
  \end{equation*}
  note that~$\relA\reltransp\colon \setB\to \setA$, while~$\relA\colon \setA\to \setB$.
\end{definition}
\begin{remark}
  In the following, we list some properties which refer to relations and their opposites. It is a good exercise to prove them:
  \begin{compactitem}
    \item $\left(\relA\reltransp\right)\reltransp = \relA $;
    \item If~$\relA$ is everywhere-defined, then~$\relA\reltransp$ is surjective;
    \item If~$\relA$ is single-valued, then~$\relA\reltransp$ is injective.
    \item If~$\relA$ is everywhere defined, then~$\catid_\setA\subseteq \relA\then \relA\reltransp$;
    \item If~$\relA$ is single-valued, then~$\relA\reltransp\then \relA\subseteq \catid_\setB$.
  \end{compactitem}
\end{remark}


\begin{remark}
  The aforementioned duality can be seen by ``reading the relations (arrows) backwards'' (\cref{fig:rel_transpose}).
\end{remark}

\begin{figure}[h!]
  \centering
  \includesag{030_rel_transpose}
  \caption{}
  \label{fig:rel_transpose}
\end{figure}
