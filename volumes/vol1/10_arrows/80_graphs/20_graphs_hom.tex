% !TEX root = chapter-standalone.tex

\section{Graph homomorphisms}
\begin{definition}[Graph homomorphism]
    \label{def:graph_homom}
    Given graphs~$\graph=\tup{\vertices,\arcs,\source,\target}$ and~$\graph'=\tup{\vertices',\arcs',\source',\target'}$, a graph homomorphism~$\mapa \colon \graph \fto \graph'$ is given by maps~$\mapa_0\colon \vertices \fto \vertices'$ and~$\mapa_1\colon \arcs \fto \arcs'$, such that the following diagrams commute:
    \begin{center}
        \includesag{60_graph_homomorphism}
    \end{center}
\end{definition}
\begin{remark}
    Intuitively, all this is saying is that ``arrows are bound to their vertices'', meaning that if a vertex~$\vertexa_1$ is connected to~$\vertexa_2$ via an arrow~$\arc$, the vertices~$\mapa_0(\vertexa_1)$ and~$\mapa_0(\vertexa_2)$ have to be connected via an arrow~$\mapa_1(\arc)$.
\end{remark}

\begin{example}
    \label{exa:homomorphism_graph_positive}
    Let us consider the two graphs,~$\graph$ and~$\graph'$ depicted in \cref{fig:ex_graph_homom}.
    \begin{figure}[h]
        \centering
        \includesag{20_ex_graph_hom}
        \caption{Example of graphs for graph homomorphism.}
        \label{fig:ex_graph_homom}
    \end{figure}

    A possible graph homomorphism between the two is given by~$\mapa_0,\mapa_1$ defined as
    \begin{equation*}
        \begin{aligned}
            \mapa_0\colon \vertices & \to \vertices'   \\
            \sbretzel               & \mapsto \sapple,
            \sfondue\mapsto \sgrapes,
            \schoco\mapsto \sbanana,
            \scheese\mapsto \scarrot,
            \sburger\mapsto \scarrot,
        \end{aligned}
    \end{equation*}
    and
    \begin{equation*}
        \begin{aligned}
            \mapa_1\colon \arcs & \to \arcs'       \\
            \swine              & \mapsto \sdrink,
            \smilk\mapsto \spritz,
            \swater\mapsto \spritz,
            \stea\mapsto \smate.
        \end{aligned}
    \end{equation*}
\end{example}

\begin{example}[Counterexample]
    By considering the graphs in \cref{exa:homomorphism_graph_positive}, one could define~$\mapa_0,\mapa_1$ in the same way, exception made for~$\mapa_0(\sburger)=\sgrapes$.
    Clearly, this would violate the commuting diagrams condition.
\end{example}

\publictodomessage
\devel{
    \begin{exercise}
        \todojira{71}{write exercise in which we ask to find maps $\mapa_1$ given $\mapa_0$}
    \end{exercise}
    \begin{solution}
        \todojira{71}{write solution}
    \end{solution}}
