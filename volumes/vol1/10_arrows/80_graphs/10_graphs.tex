% !TEX root = chapter-standalone.tex

\section{Graphs}

\publictodomessage
\todotextjira{67}{@Andrea: Write introduction to say why this section is here}
To begin, we recall some formal definitions related to (directed) graphs.

\todographicsjira{68}{Need nice pictures of graphs and various quantities.}
\begin{definition}[Graph]
    \label{def:Graph}
    A \emph{\iindex{graph}}~$\graph=\tup{\vertices,\arcs, \source,\target}$ consists of a set of vertices~$\vertices$, a set of arrows~$\arcs$, and two functions~$\source,\target \colon \arcs \to \vertices$, called the \emph{source} and \emph{target} functions, respectively.
    Given~$\arc\in \arcs$ with~$\source(\arc)=\vertexa$ and~$\target(\arc)=\vertexb$, we say that~$\arc$ is an \emph{arrow} from~$\vertexa$ to~$\vertexb$.
\end{definition}

\begin{remark}
    Both directed graphs and undirected graphs play a prominent role in many kinds of mathematics.
    In this text, we work primarily with directed graphs and so, from now on, we will drop the ``directed'': unless indicated otherwise, the word ``graph'' will mean ``directed graph''.
\end{remark}

\begin{definition}[Path]
    \label{def:path}
    Let~$\graph$ be a graph.
    A \emph{path} in~$\graph$ is a sequence of arrows such that the target of one arrow is the source of the next.
    The \emph{length} of a path is the number of arrows in the sequence.
    We also formally allow for sequences made up of ``zero-many'' arrows (such paths therefore have length zero).
    We call such paths \emph{trivial} or \emph{empty}.
    If paths describe a journey, then trivial paths correspond to ``not going anywhere''.
    The notions of source and target for arrows extend, in an obvious manner, to paths.
    For trivial paths, the source and target always coincide.
\end{definition}
