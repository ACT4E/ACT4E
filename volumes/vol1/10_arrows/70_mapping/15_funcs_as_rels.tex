% !TEX root = chapter-standalone.tex


\section{Functions as relations}

A question on your mind at this point might be: what is the relationship between relations and functions? One point of view is that functions are special kinds of relations.

\begin{definition}[Functions as relations]
    \label{def:functions_as_relations}
    Let~$\setA$ and~$\setB$ be sets. A \iindex{relation}~$\relA \subseteq \setA \times \setB$ is a \emph{\iindex{function}} if it satisfies the following two conditions:
    \begin{enumerate}
        \item $\forall \ela \in \setA \quad \exists \ \elb \in \setB\colon  \ \tup{\ela,\elb} \in \relA$
        \item $\forall \tup{\ela_1, \elb_1}, \tup{\ela_2, \elb_2} \in \relA  \text{ holds}\colon \ela_1 = \ela_2 \Imp \elb_1 = \elb_2$.
    \end{enumerate}
\end{definition}

What does this definition have to do with the ``usual'' way that we think about functions?

Let's start with a relation~$\relA \subseteq \setA \times \setB$ satisfying the conditions of \cref{def:functions_as_relations}. We will build from it a function~$\mapa_\relA \colon \setA \to \setB$. Choose an arbitrary $\ela \in \setA$. According to point $1.$ in \cref{def:functions_as_relations}, there exists a~$\elb \in \setB$ such that~$\tup{\ela,\elb} \in \relA$. So let's choose such a~$\elb$, and call it~$\mapa_\relA(\ela)$. This gives us recipe to get from any~$\ela$ to a~$\elb$. But maybe you are worried: given a specific~$\ela \in \setA$, what if we choose~$\elb$ differently each time we apply the recipe? Point~$2.$ guarantees that this can't happen: it says that the element~$\mapa_\relA(\ela)$ that we associate to a given~$\ela \in \setA$ is in fact uniquely determined by that~$\ela$. Put another way, the condition~$2.$ says: if~$\mapa_\relA(\ela_1) \neq \mapa_\relA(\ela_2)$, then~$\ela_1 \neq\ela_2$.

Given a function~$\mapa \colon \setA \to \setB$, we can turn it into a relation in a simple way: we consider its graph
\begin{equation*}
    \relA_\mapa \definedas \text{graph}(\mapa) = \{ \tup{\ela,\elb} \in \setA \times \setB \mid \elb= \mapa(\ela) \}.
\end{equation*}
The relation~$\relA_\mapa$ encodes the same information that~$\mapa$ encodes -- simply in a different form.

In this text, we take \cref{def:functions_as_relations} as our rigorous definition of a what a function is. Nevertheless, we'll often use functions ``in the usual way'' and we write things like~$\elb = \mapa(\ela)$.

Another question you may be wondering about is this: if we define functions as special kinds of relations, how then do we define the composition of functions? The answer is that we compose functions simply by the rule for composing relations.

\begin{lemma}
    \label{lem:comprelfun}
    Let~$\relA \subseteq \setA \times \setB$ and~$\relB \subseteq \setB \times \setC$ be relations which are functions. Then their composition~$\relA \mthen \relB \subseteq \setA \times \setC$ is again a function.
\end{lemma}

\begin{proof}
    We check that~$\relA \mthen \relB$ satisfies the two conditions stated in \cref{def:functions_as_relations}.

    \begin{enumerate}
        \item Choose an arbitrary $\ela \in \setA$. We need to show that there exists~$\elc \in \setC$ such that~$\tup{\ela,\elc} \in \relA \mthen \relB$. Since~$\relA$ is a function, there exists~$\elb \in \setB$ such that~$\tup{\ela,\elb} \in \relA$. Choose such a~$\elb \in \setB$. Then, because~$\relB$ is a function, there exists~$\elc \in \setC$ such that~$\tup{\elb,\elc} \in \relB$. By the definition of composition of relations, we see that~$\elc$ is such that~$\tup{\ela,\elc} \in \relA \mthen \relB$.
        \item Let~$\tup{\ela_1, \elc_1}$,~$\tup{\ela_2,\elc_2} \in \relA \mthen \relB$. We need to show that if~$\ela_1 = \ela_2$, then~$\elc_1 = \elc_2$. So suppose~$\ela_1 = \ela_2$. Since~$\tup{\ela_1, \elc_1}$,~$\tup{\ela_2,\elc_2} \in \relA \then \relB$, there exist~$\elb_1, \elb_2 \in \setB$ such that, respectively,
        \begin{equation*}
            \tup{\ela_1, \elb_1} \in \relA \text{ and } \tup{\elb_1, \elc_1} \in \relB,
        \end{equation*}
        \begin{equation*}
            \tup{\ela_2, \elb_2} \in \relA \text{ and } \tup{\elb_2, \elc_2} \in \relB.
        \end{equation*}
        Since~$\ela_1 = \ela_2$ and~$\relA$ is a function, we conclude that~$\elb_1 = \elb_2$ must hold. Now, since~$\relB$ is also a function, this implies that~$\elc_1 = \elc_2$, which is what was to be shown.
    \end{enumerate}
\end{proof}

\begin{exercise}
    Can we have a function (or relation) whose source is the empty set~$\emptyset$?
\end{exercise}
\begin{solution}
    Given any set~$\setB$, such a relation would be of the form~$\relA \subseteq \emptyset \times \setB \definedas \emptyset$. This implies that~$\relA=\emptyset$. We now need to check whether~$\relA=\emptyset$ corresponds to a function~$\emptyset\to \setB$:
    \begin{compactitem}
        \item For all~$\ela\in \setA=\emptyset$, there exists a~$\elb \in \setB$ such that $\tup{\ela,\elb}\in \relA$ (trivially satisfied).
        \item Clearly, given~$\tup{\ela,\elb}, \tup{\ela',\elb'}\in \relA=\emptyset$, having~$\ela=\ela'$ implies~$\elb=\elb'$.
    \end{compactitem}
    Therefore, the answer to the original question is yes.
\end{solution}

\begin{exercise}
    Can we have a function (or relation) whose target is the empty set~$\emptyset$?
\end{exercise}
\begin{solution}
    Again, given any set~$\setA$, such a relation would be of the form~$\relA \subseteq \setA \times \emptyset\definedas \emptyset$. This, again, implies~$\relA=\emptyset$. We now need to check whether~$\relA=\emptyset$ corresponds to a function~$\setA \to \emptyset$:
    \begin{itemize}
        \item For all~$\ela\in \setA$, there exists a~$\elb\in \Objb=\emptyset$ such that~$\tup{\ela,\elb}\in \relA$? Unless~$\setA=\emptyset$, this is not satisfied.
    \end{itemize}
    Therefore, given~$\setA \neq \emptyset$, there is no function (or relation)~$\setA \to \emptyset$.
\end{solution}