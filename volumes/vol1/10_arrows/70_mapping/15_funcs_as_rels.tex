% !TEX root = chapter-standalone.tex
\section{Functions as relations}
\todojira{580}{@J: Move up to initial part and make sure no duplicates are there.}
A question on your mind at this point might be: what is the relationship between relations and functions?
One point of view is that functions are special kinds of relations.

\begin{definition}[Functions as relations]
    \label{def:functions_as_relations}
    Let~$\setA$ and~$\setB$ be sets.
    A \iindex{relation}~$\relA \subseteq \setA \cartprod \setB$ is a \emph{\iindex{function}} if it satisfies the following two conditions:
    \begin{enumerate}
        \item for all~$\ela \in \setA \quad \exists \ \elb \in \setB\colon  \inrel{\ela}{\relA}{\elb}$;
        \item for all~$\inrel{\ela_1}{\relA}{\elb_1}, \inrel{\ela_2}{\relA}{\elb_2}$ holds:
              \begin{equation*}
                  \prfperiod{\ela_1 = \ela_2}{\elb_1 = \elb_2}
              \end{equation*}
    \end{enumerate}
\end{definition}

What does this definition have to do with the ``usual'' way that we think about functions?

Let's start with a relation~$\relA \subseteq \setA \cartprod \setB$ satisfying the conditions of \cref{def:functions_as_relations}.
We will build from it a function~$\mapa_\relA \colon \setA \to \setB$.
Choose an arbitrary $\ela \in \setA$.
According to point~$1.
$ in \cref{def:functions_as_relations}, there exists a~$\elb \in \setB$ such that~$\inrel{\ela}{\relA}{\elb}$.
So let's choose such a~$\elb$, and call it~$\mapa_\relA(\ela)$.
This gives us recipe to get from any~$\ela$ to a~$\elb$.
But maybe you are worried: given a specific~$\ela \in \setA$, what if we choose~$\elb$ differently each time we apply the recipe?
Point~$2.
$ guarantees that this can't happen: it says that the element~$\mapa_\relA(\ela)$ that we associate to a given~$\ela \in \setA$ is in fact uniquely determined by that~$\ela$.
Put another way, the condition~$2.
$ says: if~$\mapa_\relA(\ela_1) \neq \mapa_\relA(\ela_2)$, then~$\ela_1 \neq\ela_2$.

Given a function~$\mapa \colon \setA \to \setB$, we can turn it into a relation in a simple way: we consider its graph
\begin{equation*}
    \relA_\mapa \definedas \text{graph}(\mapa) = \{ \tup{\ela,\elb} \in \setA \cartprod \setB \mid \elb= \mapa(\ela) \}.
\end{equation*}
The relation~$\relA_\mapa$ encodes the same information that~$\mapa$ encodes -- simply in a different form.

In this text, we take \cref{def:functions_as_relations} as our rigorous definition of a what a function is.
Nevertheless, we'll often use functions ``in the usual way'' and we write things like~$\elb = \mapa(\ela)$.






