% !TEX root = standalone.tex


\section{The Category \Set}

A helpful concept here is to think of our specific sets and functions as living in a very (very) large category which contains all possible sets as its objects and all possible functions as its morphisms. This category is known as the category of sets, and it is an important protagonist in category theory. We will denote it by~\Set. It is a short exercise to check that the following does indeed define a category.

\begin{ctdefinition}[Category of sets]
  \label{def:Set}
  The category of sets \iindex{\Set} is defined by:
  \begin{compactenum}
    \item \emph{Objects}: all sets.
    \item \emph{Morphisms}: given sets~$\Obja$ and~$\Objb$, the homset~$\HomSet\Set \Obja \Objb$ is the set of all functions from~$\Obja$ to~$\Objb$.
    \item \emph{Identity morphism}: given a set~$\Obja$, its identity morphism~$\catid_\Obja$ is the identity function~$\Obja \to \Obja, \ \catid_\Obja(\obja) = \obja$.
    \item \emph{Composition operation}: the composition operation is the usual composition of functions.
  \end{compactenum}
\end{ctdefinition}

We did say above, however, that we could build a category~\Database which only involves the sets that we are using for our database, and the functions between them that we are working with. What we would need for~\Database to be a category is that if any function is in~\Database, then also its sources and target sets are, and we would need that any composition of functions in~\Database is again in~\Database. (Also, we define the identity morphism for any set in~\Database to be the identity function on that set.) If these conditions are met,~\Database is what is called a \emph{subcategory} of~\Set.
