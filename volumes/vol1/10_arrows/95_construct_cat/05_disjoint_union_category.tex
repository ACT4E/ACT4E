% !TEX root = chapter-standalone.tex

\section{Disjoint Union of Categories}

\begin{ctdefinition}[Disjoint union category]
    \label{def:disjoint-union-category}
    Given two categories~\CatC and~\CatD, their \emph{disjoint union}~$\CatC\Cplus\CatD$ is the category specified as follows:
    \begin{enumerate}
        \item \emph{Objects}: Objects are elements of~$\ObC \setdisunion \ObD$; that is, objects are tuples of the form~$\disunionI{\Obja}$, with $i=1$ or $i=2$, depending on whether~$\Obja \setin \ObC$ or~$\Obja \setin \ObD$.
        \item \emph{Morphisms}: Given objects~$\disunionI{\Obja}, \disunionJ{\Objb} \setin \Obof{\CatC\Cplus\CatD}$,
              \begin{equation}
                  \HomSet{\CatC\Cplus\CatD}{\disunionI{\Obja}}{\disunionJ{\Objb}} \definedas
                  \left\{
                  \begin{tabular}{ccc}
                      $\HomSet{\CatC}{\Obja}{\Objb}$ &  & if $i=j = 1$, \\
                      $\HomSet{\CatD}{\Obja}{\Objb}$ &  & if $i = j =2$, \\
                      $\Emptyset$                    &  & else.
                  \end{tabular}
                  \right.{}
              \end{equation}
              %
        \item \emph{Identity morphisms}: The identities  are copied from either category:
              \begin{align}
                  \catidofat{\CatC\Cplus\CatD}{\disunionA{\Obja}} & \definedas \catidofat\CatC\Obja, \\
                  \catidofat{\CatC\Cplus\CatD}{\disunionB{\Obja}} & \definedas \catidofat\CatD\Obja.
              \end{align}
        \item \emph{Composition}: The composition $\mthenof{\CatC\Cplus\CatD}$ is equal to $\mthenof\CatC$ when acting on the objects of the type $\tup{1, \Obja}$ or equal to $\mthenof\CatD$
              when acting on the objects of the type $\tup{2, \Obja}$.
              %   that two morphisms $\mora$ and $\morb$ are composable only if they both  The composition~$\mora\colon \Obja \mto \Objb\setin \HomSet{\CatC + \CatD}{\disunionI{\Obja}}{\disunionI{\Objb}}$ and~$\morb\colon \Objb\mto \Objc\setin \HomSet{\CatC + \CatD}{\disunionJ{\Objb}}{\disunionJ{\Objc}}$, we have
              %       \begin{equation}
              %           \mora \mthenof{\CatC+\CatD} \morb \definedas
              %           \left\{
              %           \begin{tabular}{ccc}
              %               $\mora\mthenof\CatC \morb$  &  & if $i=j = 1$, \\
              %               $\mora \mthenof\CatD \morb$ &  & if $i = j =2$, \\
              %               does not exist              &  & else.
              %           \end{tabular}
              %           \right.{}
              %       \end{equation}
              %       \todotext{\alphubel: A little convoluted}
    \end{enumerate}
\end{ctdefinition}

\begin{remark}
    If you think about categories in diagrammatic form, this operation corresponds to placing two categories side-by-side, without connecting them.
\end{remark}

\begin{exercise}
    Show that the disjoint union of two categories indeed forms a category.
\end{exercise}
\begin{solution}
    \todotext{\alphubel: @Gioele: the math doesn't match the definition.}
    We check the two conditions.
    First, consider a morphism~$\mora\colon \Obja \mto \Objb\setin \HomSet{\CatC\Cplus\CatD}{\tup{\Obja, i}}{\tup{\Objb, i}}$ (the index~$i$ is repeated, because following the definition of morphisms, no morphism connects objects of one category to objects of the other one).
    We have
    \begin{equation}
        \catidat{\CatC\Cplus\CatD}\mthen \mora=
        \left\{
        \begin{tabular}{ccc}
            $\catidof\CatC\mthen \mora=\mora$ &  & if $i=1$, \\
            $\catidof\CatD\mthen \mora=\mora$ &  & if $i=2$,
        \end{tabular}
        \right.
        {}
    \end{equation}
    and
    \begin{equation}
        \mora \mthen \catidat{\CatC\Cplus\CatD}=
        \left\{
        \begin{tabular}{ccc}
            $\mora \mthen \catidof\CatC=\mora$ &  & if $i=1$, \\
            $\mora \mthen \catidof\CatD=\mora$ &  & if $i=2$.
        \end{tabular}
        \right.{}
    \end{equation}
    Second, consider the morphisms~$\mora\colon \Obja \mto \Objb\setin \HomSet{\CatC\Cplus\CatD}{\tup{\Obja, i}}{\tup{\Objb, i}}$,~$\morb\colon \Objb \mto \Objc\setin \HomSet{\CatC\Cplus\CatD}{\tup{\Obja, j}}{\tup{\Objb, j}}$, and~$\morc\colon \Objc \mto \Objd\setin \HomSet{\CatC\Cplus\CatD}{\tup{\Obja, k}}{\tup{\Objb, k}}$.
    We have
    \begin{equation}
        (\mora \mthenof{\CatC\Cplus\CatD} \morb)
        \mthenof{\CatC\Cplus\CatD} \morc \definedas
        \left\{
        \begin{tabular}{ccc}
            $(\mora\mthenof\CatC \morb)\mthenof\CatC \morc= \mora\mthenof\CatC \morb\mthenof\CatC \morc$ &  & if $i=j =k= 1$, \\
            $(\mora\mthenof\CatD \morb)\mthenof\CatD \morc= \mora\mthenof\CatD \morb\mthenof\CatD \morc$ &  & if $i = j=k =2$, \\
            does not exist                                                                               &  & else.
        \end{tabular}
        \right.{}
    \end{equation}
    and
    \begin{equation}
        \mora \mthenof{\CatC\Cplus\CatD} (\morb\mthenof{\CatC\Cplus\CatD} \morc) \definedas
        \left\{
        \begin{tabular}{ccc}
            $\mora\mthenof\CatC (\morb\mthenof\CatC \morc)= \mora\mthenof\CatC \morb\mthenof\CatC \morc$ &  & if $i=j =k =1$, \\
            $\mora\mthenof\CatD (\morb\mthenof\CatD \morc)= \mora\mthenof\CatD \morb\mthenof\CatD \morc$ &  & if $i = j=k =2$, \\
            does not exist                                                                               &  & else.
        \end{tabular}
        \right.{}
    \end{equation}
\end{solution}

