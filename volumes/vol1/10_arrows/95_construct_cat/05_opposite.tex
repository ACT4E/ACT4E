% !TEX root = chapter-standalone.tex

\section{Opposite Category}

\begin{ctdefinition}[Opposite category]
    \label{def:opposite-category}
    \label{def:oppositecat}
    Given a category~\CatC, its \emph{\iindex{opposite category}}~$\CatC\op$ is specified by:
    \begin{enumerate}
        \item \emph{Objects}: The objects are the same as the original: $\Obof\CatCop = \Obof\CatC$.

        \item \emph{Morphisms}: Given objects~$\Obja,  \Objb \setin \Obof\CatCop = \Obof\CatC$,
              \begin{equation}
                  \label{eq:homset-op-cat}
                  \HomSet\CatCop{\Obja}{\Objb} \definedas \HomSet{\CatC}{\Objb}{\Obja}.
              \end{equation}
              %
              For each morphism $\mora\colon\Obja\mtoin\CatC\Objb$ there is a morphism
              $\mora\op:\Objb\mtoin\CatCop\Obja$.
            %   Given~$\mora \setin \HomSet{\CatC}{\Objb}{\Obja}$, when we are thinking of it as an element of $\HomSet{\CatC\op}{\Obja\op}{\Objb\op}$, we will sometimes write~$\mora\op$.
              %
              Graphically, in~$\CatC$:
              %
              \equationsag{opposite_mor_1}{eq:opposite_mor_1}
              %
              and in~$\CatC\op$:
              %
              \equationsag{opposite_mor_2}{eq:opposite_mor_2}
              %
        \item \emph{Identity morphisms}: The identities are the same as the original category.
        %  Given~$\Obja\op \setin \Obof{\CatC\op}$, its identity morphism is
        %       \begin{equation}
        %           \catidat{\Obja\op} \definedas \catidat\Obja\op.
        %       \end{equation}
              %
        \item \emph{Composition}: For two morphisms
              \begin{equation}
                  \mora\op\colon \Obja \mtoin\CatCop \Objb 
                  \qqand
                  \morb\op\colon \Objb \mtoin\CatCop \Objc
              \end{equation}
              their composition is 
              \begin{equation}
                \mora\op \mthenof\CatCop \morb\op \definedas (\morb \mthenof\CatC \mora)\op.
              \end{equation}
    \end{enumerate}
\end{ctdefinition}
Given~$\Obja \setin  \Obof\CatC$, we will sometimes (though not always) write $\Obja\op$ to signify when we are thinking of~$\Obja$ as an object of~$\Obof\CatC\op$.

\vfill
\begin{gradedexercise}[\exname{OppositeCat}]
    \label{ex:OppositeCat}
    Verify that \cref{def:oppositecat} defines a category.
    In other words, check that its constituents satisfy the conditions of associative and unitality.
\end{gradedexercise}

\solutionof{OppositeCat}

\linkvideo{spring2021-tradeoffs:tradeoffs:orders:preorder-as-cat} % Pre-order as a category
\begin{example}[Opposite of a poset]
    A single poset~$\posA=\tup{\posAset, \posleq}$ can be described as a category, in which each point is an object, and there is a morphism between two objects~$\posela$ and~$\poselb$ if and only if~$\posela \posAleq \poselb$.
    We have defined the opposite of a poset in \cref{sec:opposite-of-a-poset}.
    The opposite category of a category for a poset, is the category for the opposite poset.
\end{example}
\todotext{Add formula}