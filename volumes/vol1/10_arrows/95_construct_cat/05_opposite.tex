% !TEX root = chapter-standalone.tex

\section{Opposite Category}

\begin{ctdefinition}[Opposite category]
    \label{def:opposite-category}
    \label{def:oppositecat}
    Given a category~\CatC, its \maindef{opposite category}~$\CatCop$ is specified by:
    \begin{enumerate}
        \item \emph{Objects}: $\Obof\CatCop = \Obof\CatC$.

        \item \emph{Morphisms}: Given objects~$\Obja,  \Objb \setin \Obof\CatCop = \Obof\CatC$,
              \begin{equation}\label{eq:homset-op-cat}
                  \HomSet\CatCop{\Obja}{\Objb} \definedas \HomSet{\CatC}{\Objb}{\Obja}.
              \end{equation}
              %
              For each morphism $\mora\colon\Obja\mtoin\CatC\Objb$, there is a morphism
              $\mora\op:\Objb\mtoin\CatCop\Obja$.
              %   Given~$\mora \setin \HomSet{\CatC}{\Objb}{\Obja}$, when we are thinking of it as an element of $\HomSet{\CatC\op}{\Obja\op}{\Objb\op}$, we will sometimes write~$\mora\op$.
              %
              Graphically, given
              \begin{equation}\label{eq:opposite_mor_1}
                  \includesag{opposite_mor_1}
              \end{equation}
              we have
              % 
              \begin{equation}
                  \label{eq:opposite_mor_2}
                  \includesag{opposite_mor_2}
              \end{equation}
              %
        \item \emph{Composition}: Given morphisms
              \begin{equation}
                  \mora\op\colon \Obja \mtoin\CatCop \Objb
                  \qqand
                  \morb\op\colon \Objb \mtoin\CatCop \Objc
              \end{equation}
              their composition is defined as
              \begin{equation}
                  \mora\op \mthenof\CatCop \morb\op \definedas (\morb \mthenof\CatC \mora)\op.
              \end{equation}
        \item \emph{Identity morphisms}: given by the identities of the original category $\CatC$.
    \end{enumerate}
\end{ctdefinition}
Given~$\Obja \setin \Obof\CatC$, we will sometimes (though not always) write $\Obja\op$ to signify when we are thinking of~$\Obja$ as an object of~$\Obof\CatCop$.

\vfill
\begin{gradedexercise}[\exname{OppositeCat}]
    \label{ex:OppositeCat}
    Verify that \cref{def:oppositecat} defines a category.
    In other words, check that its constituents satisfy the conditions of associativity and unitality.
\end{gradedexercise}

\solutionof{OppositeCat}

\todotext{@J: write up solution}

\linkvideo{spring2021-tradeoffs:tradeoffs:orders:preorder-as-cat} % Pre-order as a category
\begin{example}[Opposite of a poset]
    A single poset~$\posAdefinition$ can be described as a category, in which each point is an object, and there is a morphism between two objects~$\posela$ and~$\poselb$ if and only if~$\posela \posAleq \poselb$.
    We have defined the opposite of a \SY{poset} in \cref{sec:opposite-of-a-poset}.
    The \SY{opposite category} of a category for a poset, is the category for the \SY{opposite poset}.
    \begin{equation}
        (\poscat \posA)\op \simeq \poscat (\posAop) .
    \end{equation}
\end{example}
