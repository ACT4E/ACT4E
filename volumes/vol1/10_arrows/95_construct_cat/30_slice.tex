% !TEX root = chapter-standalone.tex

\section{(Co)slice construction}

\todotextjira{746}{\bernina: @JL: Write more in the (co)slice construction section.}

\begin{ctdefinition}[Slice categories]
    \label{def:slice-category}

    Let \CatC be a category and fix an object $\slicetarget \in \Ob_\CatC$.
    The \maindef{slice category} $\sliceCat{C}{\slicetarget}$ \emph{of} $\CatC$ \emph{over} $\slicetarget$ is:
    \begin{enumerate}
        \item \emph{Objects:} morphisms in \CatC of the type $\Obja \mto \slicetarget$, where $\Obja$ ranges over all objects of $\CatC$.
        \item \emph{Morphisms:} given objects $\Obja \overset{\mora}{\mto} \slicetarget$ and $\Objb \overset{\morb}{\mto} \slicetarget$, a morphism
              \begin{equation}
                  \slicemora_{\sliceCat{C}{\slicetarget}} \colon (\Obja \overset{\mora}{\mto} \slicetarget)
                  \mtoin{\sliceCat{C}{\slicetarget}} (\Objb \overset{\morb}{\mto} \slicetarget)
              \end{equation}
              is specified by a morphism $\slicemora \colon \Obja \mtoin{\CatC} \Objb$ such that the diagram
              \begin{equation}\label{eq:slice-morphism}
                  \includesag{slice-morphism}
              \end{equation}
              commutes.

        \item \emph{Composition:} defined via the composition in $\CatC$.
              Concretely, given composable morphisms $\slicemora_{\sliceCat{C}{\slicetarget}}$ and $\slicemorb_{\sliceCat{C}{\slicetarget}}$ of $\sliceCat{C}{\slicetarget}$, we define
              \begin{equation}
                  \slicemora_{\sliceCat{C}{\slicetarget}} \mthenof{\sliceCat{C}{\slicetarget}} \slicemorb_{\sliceCat{C}{\slicetarget}} \definedas (\slicemora \mthenof{\CatC} \slicemorb)_{\sliceCat{C}{\slicetarget}}.
              \end{equation}
        \item \emph{Identities:} defined by the identities in $\CatC$.
    \end{enumerate}
\end{ctdefinition}

\begin{gradedexercise}[\exname{SliceCat}]
    \label{ex:SliceCat}
    Let~$\CatC$ be a category, fix~$\slicetarget \setin \Ob_{\CatC}$, and consider the slice category~$\sliceCat{\CatC}{\slicetarget}$.
    Your task is to check that the composition of two composable morphisms in~$\sliceCat{\CatC}{\slicetarget}$ is again in fact a morphism in~$\sliceCat{\CatC}{\slicetarget}$.
\end{gradedexercise}

\solutionof{SliceCat}

\begin{ctdefinition}[Coslice categories]
    \label{def:coslice-category}

    Let \CatC be a category and fix an object~$\coslicesource \setin \Ob_\CatC$.
    The \maindef{coslice category} $\cosliceCat{C}{\coslicesource}$ \emph{of} $\CatC$ \emph{under} $\coslicesource$ is:
    \begin{enumerate}
        \item \emph{Objects:} morphisms in \CatC of the type $\coslicesource \mto \Obja$, where $\Obja$ ranges over all objects of $\CatC$.
        \item \emph{Morphisms:} given objects $\coslicesource \overset{\mora}{\mto} \Obja$ and $\coslicesource  \overset{\morb}{\mto} \Objb$, a morphism
              \begin{equation}
                  \slicemora_{\cosliceCat{C}{\coslicesource}} \colon (\coslicesource \overset{\mora}{\mto} \Obja)
                  \mtoin{\cosliceCat{C}{\coslicesource}}
                  (\coslicesource \overset{\mora}{\mto} \Objb)
              \end{equation}
              is specified by a morphism $\coslicemora \colon \Obja \mtoin{\CatC} \Objb$ such that the diagram
              \begin{equation}\label{eq:coslice-morphism}
                  \includesag{coslice-morphism}
              \end{equation}
              commutes.

        \item \emph{Composition:} defined by the composition in $\CatC$.
        \item \emph{Identities:} defined by the identities in $\CatC$.
    \end{enumerate}
\end{ctdefinition}

