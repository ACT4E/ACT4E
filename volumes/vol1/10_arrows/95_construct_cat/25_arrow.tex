% !TEX root = chapter-standalone.tex

\section{Arrow construction}

\todotextjira{259}{\alphubel: @JL: \emph{Arrow construction}
    Intro to write}

\begin{ctdefinition}[Arrow category]
    \label{def:arrow-category}
    Given any category~\CatC, its \maindef{arrow category}~$\Arrow\CatC$ is the category in which:
    \begin{enumerate}
        \item \emph{Objects:} An object~$\Obja \setin \Arrow\CatC$ is a morphism~$\Obja\colon \Objan{0}\mto \Objan{1}$ of~\CatC.
        \item \emph{Morphisms:} A morphism~$\mora\colon \Obja \mto \Objb$ in~$\Arrow\CatC$ is a pair of morphisms $\mora_0,\mora_1$ that make the following diagram into a \SY{commutative square} in~\CatC
              \equationsag{arrow_cat}{eq:arrow_cat}.
        \item \emph{Composition:} Composition in~$\Arrow \CatC$ is given by placing \SY{commutative squares} side by side.
              Consider~$\mora\colon \Obja \mto \Objb$ and~$\morb\colon \Objb \mto \Objc$ in~$\Arrow\CatC$.
              Then:
              \equationsag{180_arrow_comp}{eq:180_arrow_comp}
              Therefore, we have that
              \begin{equation}\label{eq:arrow-cat-composition}
                  \tup{\mora_0,\mora_1} \mthenof{\Arrow \CatC} \tup{\morb_0,\morb_1}
                  \definedas \tup{\mora_0\mthenof\CatC\morb_0, \mora_1\mthenof\CatC\morb_1
                  }.
              \end{equation}
    \item \emph{Identities:} given an object $\Obja\colon \Objan{0}\mto \Objan{1}$ of $\Arrow\CatC$, its identity morphism is $\tup{\catid_{\Objan{0}},\catid_{\Objan{1}}}$.
    \end{enumerate}
\end{ctdefinition}

\begin{example}[Intervals]
    \label{exa:arrow-poset}
    Consider a poset~\posA.
    The arrow category~$\Arrow{\pars{\poscat{\posA}}}$ is isomorphic to the \SY{poset} (viewed as a category)~$\poscat{\pars{\posintbis\posA}}$ of nonempty \emph{intervals} in~\posA:
    \begin{equation}
        \Arrow{\pars{\poscat{\posA}}} \simeq
        \poscat{\pars{\posintbis\posA}}.
    \end{equation}
\end{example}

\todotextjira{406}{\alphubel: @JL: Explain why composition is well defined.
    We didn't talk that much
    about the use of commutative diagrams so this is a good example.
}
\todotextjira{406}{\alphubel: Add remark about \SY{functor} category isomorphism}
