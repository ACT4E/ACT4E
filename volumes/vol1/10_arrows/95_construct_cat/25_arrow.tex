% !TEX root = chapter-standalone.tex

\section{Arrow construction}

\todotextjira{259}{@J: Intro to write}

\begin{ctdefinition}[Arrow category]
    \label{def:arrow_category}
    Given any category~$\CatC$, its \emph{\iindex{arrow category}}~$\Arrow\CatC$ is the category in which:
    \begin{enumerate}
        \item \emph{Objects:} An object~$\Obja \setin \Arrow\CatC$ is a morphism~$\Obja\colon \Objan{0}\mto \Objan{1}$ of~$\CatC$;
        \item \emph{Morphisms:} A morphism~$\mora\colon \Obja \mto \Objb$ in~$\Arrow\CatC$ is a commutative square
        \equationsag{arrow_cat}{eq:arrow_cat}
        in~$\CatC$;
        \item \emph{Composition:} Composition in~$\Arrow \CatC$ is given by playing commutative squares side by side.
        Consider~$\mora\colon \Obja \mto \Objb$ and~$\morb\colon \Objb \mto \Objc$ in~$\Arrow\CatC$.
        Then:
        \equationsag{180_arrow_comp}{eq:180_arrow_comp}
    \end{enumerate}
\end{ctdefinition}

\begin{example}[Intervals]
    \label{exa:arrow-poset}
    Consider a poset~$\posA$.
    The arrow category~$\Arrow \poscat{\posA}$ is isomorphic to the poset (viewed as a category)~$\poscat{\left(\posintbis\posA\right)}$ of nonempty \emph{intervals} in~$\posA$.
\end{example}

\todotextjira{539}{@J: Give the formula for composition explicitly?
}

\todotextjira{406}{@J: Explain why composition is well defined.
We didn't talk that much
about the use of commutative diagrams so this is a good example.
}
\todotextjira{406}{Add remark about functor category isomorphism}
