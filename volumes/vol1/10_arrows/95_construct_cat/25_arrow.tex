% !TEX root = chapter-standalone.tex

\section{Arrow construction}

\todotextjira{259}{\alphubel: @JL: \emph{Arrow construction}
    Intro to write}

\begin{ctdefinition}[Arrow category]
    \label{def:arrow-category}
    Given any category~\CatC, its \maindef{arrow category}~$\Arrow\CatC$ is:
    \begin{enumerate}
        \item \emph{Objects:} morphisms of~\CatC.
        \item \emph{Morphisms:}
              Given objects $\mora \colon \Obja \mtoin{\CatC} \Objb$ and $\morb \colon \Objc \mtoin{\CatC} \Objd$, a morphism $\arrowmora \colon \mora \mtoin{\Arrow\CatC} \morb$ in $\Arrow\CatC$ is a pair of morphisms $\tup{\arrowmorasource,\arrowmoratarget}$ in $\CatC$ that make the following diagram
              \begin{equation}
                  \label{eq:arrow_cat}
                  \includesag{arrow_cat}
              \end{equation}
              a \SY{commutative square} in~\CatC.
        \item \emph{Composition:}
              Composition in~$\Arrow \CatC$ is given by placing \SY{commutative squares} side by side.
              Consider~$\tup{\arrowmorasource,\arrowmoratarget} \colon \mora \mto \morb$ and~$\tup{\arrowmorbsource,\arrowmorbtarget} \colon \morb \mto \morc$ in~$\Arrow\CatC$ giving rise to the following composite commutative diagram
              \begin{equation}\label{eq:180_arrow_comp}
                  \includesag{180_arrow_comp}.
              \end{equation}
              Since this diagram is again commutative, we define
              \begin{equation}\label{eq:arrow-cat-composition}
                  \tup{\arrowmorasource,\arrowmoratarget}\mthenof{\Arrow \CatC} \tup{\arrowmorbsource,\arrowmorbtarget}
                  \definedas \tup{\arrowmorasource\mthenof\CatC \arrowmorbsource, \arrowmoratarget \mthenof\CatC\arrowmorbtarget
                  }.
              \end{equation}
        \item \emph{Identities:} given an object $\mora \colon \Obja \mto \Objb$ of $\Arrow\CatC$, its identity morphism is $\tup{\catid_{\Obja},\catid_{\Objb}}$.
    \end{enumerate}
\end{ctdefinition}

\begin{example}[Intervals]
    \label{exa:arrow-poset}
    Consider a poset~\posA.
    The arrow category~$\Arrow{\pars{\poscat{\posA}}}$ is isomorphic to the \SY{poset} (viewed as a category)~$\poscat{\pars{\posintbis\posA}}$ of nonempty \emph{intervals} in~\posA:
    \begin{equation}
        \Arrow{\pars{\poscat{\posA}}} \simeq
        \poscat{\pars{\posintbis\posA}}.
    \end{equation}
\end{example}

\todotextjira{406}{\alphubel: @JL: Explain why composition is well defined.
    We didn't talk that much
    about the use of commutative diagrams so this is a good example.
}
\todotextjira{406}{\alphubel: Add remark about \SY{functor} category isomorphism}
