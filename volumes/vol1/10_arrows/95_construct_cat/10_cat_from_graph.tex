% !TEX root = chapter-standalone.tex

\section[Categories from graphs]{Generating categories from graphs}
\label{sec:catsfromgraphs}

The following definition provides a way of turning any graph into a category.

\begin{ctdefinition}[Free category on a graph]
    \label{def:free-category}
    Let~$\graph=\tup{\vertices,\arcs,\source,\target}$ be a graph.
    The \emph{free category on~$\graph$}, denoted~$\Free(\graph)$, has as objects the vertices~$\vertices$ of~$\graph$, and given vertices~$\vertexa\setin \vertices$ and~$\vertexb\setin \vertices$, the morphisms~$\HomSet{\Free(\graph)}{\vertexa}{\vertexb}$ are the paths from~$\vertexa$ to~$\vertexb$.
    %A path is a sequence of ``consecutive'' edges, \text{\ie } the source of a subsequent edge is equal to the target of its predecessor. We also formally allow for ``empty paths'', \text{\ie } a sequence of "zero"-many edges which starts and ends at the same vertex.
    The composition of morphisms is given by concatenation of paths, and for any object~$\vertexa \setin \vertices$, the associated \SY{identity morphism}~$\catidat\vertexa$ is the trivial path which starts and ends at~$\vertexa$.
\end{ctdefinition}

For instance, consider the graph
\equationsag{graph_cat_1}{eq:graph_cat_1}
The free category on this graph is given by
\equationsag{graph_cat_2}{eq:graph_cat_2}
and has 5 morphisms (each vertex/object has \SY{identity morphisms}, and~$\arca,\arcb$ give rise to two morphisms).

Does \cref{def:free-category} define a category?
We can check it ourselves.
For it to define a category, unitality and associativity need to be satisfied.
In order to prove both of them, we first need to formally define what we mean by ``concatenation''.
Consider a graph~$\graph=\tup{\vertices,\arcs,\source,\target}$ and denote a path (\cref{def:path}) by~$p=\tup{\arc_1,\ldots, \arc_n}$, where~$n\setin \natnumbers$,~$\source(p)\definedas\source(\arc_1)=\vertexa$ is the source of the path,~$\target(\arc_i)=\source(\arc_{i+1})$ for all~$i\setin \makeset{1,\ldots,n-1}$, and~$\target(p)=\target(\arc_n)=\vertexb$ which is the target of the path.
Note that in case the path has 0 length ($n=0$), the target of the path is simply~$\vertexa$.
Now consider two paths~$p=\tup{\arca_1,\ldots,\arca_n}$ and~$q=\tup{\arcb_1,\ldots,\arcb_m}$, with~$m,n\setin \natnumbers$.
Paths~$p$ and~$q$ can be concatenated if~$\target(p)=\source(q)$.
In this case, the concatenation reads
%
\begin{equation}
    p\mthen q\definedas \tup{\arca_1,\ldots, \arca_n,\arcb_1,\ldots,\arcb_m}.
\end{equation}
%
We now check unitality.
A trivial path is a path~$\catidat\vertexa$, consisting of a vertex and a trivial arc, starting and ending at~$\vertexa$.
Such a path can be concatenated with another path~$q=\tup{\arcb_1,\ldots,\arcb_m}$ in case~$\vertexa=\source(q)$.
In this case, we obtain the path~$\catidat\vertexa \mthen q=q$.
Similarly,~$p$ can be concatenated with~$\catidat\vertexa$ if~$\target(p)=\vertexa$, obtaining~$p\mthen \catidat\vertexa=p$, and proving unitality.
For associativity, consider~$p$ and~$q$ as above, and~$r=\tup{\arcc_1,\ldots,\arcc_l}$,~$l\setin \natnumbers$.
Assuming that concatenations~$p\mthen q$ and~$q\mthen r$ are feasible, we have
\begin{equation}
    \begin{aligned}
        (p\mthen q)
        \mthen r & =(\tup{\arc_1,\ldots, \arc_n}\mthen \tup{\arcb_1,\ldots,\arcb_m})\mthen \tup{\arcc_1,\ldots,\arcc_l} \\
                 & =\tup{\arc_1,\ldots, \arc_n,\arcb_1,\ldots,\arcb_m}\mthen \tup{\arcc_1,\ldots,\arcc_l} \\
                 & = \tup{\arc_1,\ldots, \arc_n,\arcb_1,\ldots,\arcb_m, \arcc_1,\ldots,\arcc_l} \\
                 & =\tup{\arc_1,\ldots, \arc_n} \mthen (\tup{\arcb_1,\ldots,\arcb_m}\mthen \tup{\arcc_1,\ldots,\arcc_l}) \\
                 & =p\mthen (q\mthen r),
    \end{aligned}
\end{equation}
proving associativity.

%\text{\ie } to check that the composition of paths is again a path, and that the \SY{associative law} and the law for identity morphisms hold.
\vfill
\begin{widepar}
    \begin{gradedexercise}[\exname{HowManyMorphisms}]
        \label{ex:HowManyMorphisms}
        Consider the following five graphs.
        For each graph~$\graph$, how many morphisms in total are there in the associated category~$\Free(\graph)$?
        \begin{equation}
            \middlesag{20_dpcatfig_example_graphs}
        \end{equation}
    \end{gradedexercise}
\end{widepar}
\solutionof{HowManyMorphisms}
