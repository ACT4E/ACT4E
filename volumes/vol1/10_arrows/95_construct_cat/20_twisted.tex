% !TEX root = chapter-standalone.tex

\section{Intervals as categories}

\subsection{Twisted arrow category}

\begin{ctdefinition}[Twisted arrow category]
    \label{def:twisted-arrow-category}
    \label{def:twisted-arrow}
    Given a category \CatC, we denote its \emph{twisted arrow category} by~$\TwistedArrow\CatC$.
    This is a category which is composed of:
    \begin{enumerate}
        \item \emph{Objects:} Arrows (morphisms) in \CatC.
        \item \emph{Morphisms:}
        A morphism from an arrow~$\mora\colon \Obja \mto \Objb $, to an arrow~$\morb\colon \Objc \mto \Objd$ is given by a pair of arrows~$\tup{\stylemorph{p},\stylemorph{q}}$ such that the following diagram commutes:
        \equationsag{180_twistedarrow}{eq:180_twistedarrow}
        \item \emph{Composition:} Composition in~$\TwistedArrow\CatC$ is given by placing commutative squares side by side.
        Consider~$\mora\colon \Obja \mto \Objb$ with~$\tup{\stylemorph{p},\stylemorph{q}}$, where~$\stylemorph{p}\colon \Objc\mto \Obja$,~$\stylemorph{q}\colon \Objd \mto \Objb$, and~$\morb\colon \Objc \mto \Objd$ with~$\tup{\stylemorph{r},\stylemorph{s}}$, where~$\stylemorph{r}\colon \Obje\mto \Objc$,~$\stylemorph{s}\colon \Objf \mto \Objd$.
        Then:
        \equationsag{180_twistedarrow_comp}{eq:180_twistedarrow_comp}
    \end{enumerate}
\end{ctdefinition}

\todotextjira{539}{@J: Give the formula for composition explicitly?
}

\begin{example}[Intervals]
    \label{exa:twisted-arrow-poset}
    Consider a poset~$\posA$.
    The twisted arrow category~$\TwistedArrow \poscat{\posA}$ is isomorphic to the poset (viewed as a category)~$\poscat{\left(\posint\posA\right)}$ of nonempty \emph{intervals} in~$\posA$.
    Recall that, given elements~$\posela, \poselb \setin \posA$, the interval~$\interv{\posela}{\poselb}$ is
    \begin{equation*}
        \interv{\posela}{\poselb}
        \definedas \makeset{\poselc \setin \posA \mid \posela\posAleq \poselc \posAleq \posela}.
    \end{equation*}
\end{example}

\begin{exercise}
    Prove the statement in \cref{exa:twisted-arrow-poset}.
\end{exercise}
\begin{solution}
    Consider~$\poscat{\left(\posint\posA\right)}$.
    Take two morphisms in~$\posA$:~$\mora\colon \Obja \mto \Objb$ (from~$\Obja \posleq \Objb$) and~$\morb\colon \Objc\mto \Objd$ (from~$\Objc \posleq \Objd$).
    These are two objects in~$\poscat{\left(\posint\posA\right)}$.
    Now, a morphism in~$\poscat{\left(\posint\posA\right)}$ is a pair~$\tup{\morc,\mord}$ where~$\morc\colon \Objc \mto \Obja$ (from~$\Objc\posleq \Obja$) and~$\mord\colon \Objb \mto \Objd$ (from~$\Objb\posleq \Objd$).
    Therefore, one has~$\Objc\posleq \Obja\posleq \Objb\posleq \Objd$, which corresponds to~$\interv{\Obja}{\Objb}\posleq_{\posint{\posA}}\interv{\Objc}{\Objd}$.
    Therefore, morphisms in~$\poscat{\left(\posint\posA\right)}$ between arrows (intervals) correspond to order relations between intervals.
\end{solution}

\devel{
    \begin{remark}
        Recall \cref{sec:posetsarecats} and note that the map which sends a poset (a category) to its twisted arrow category is a functor, which sends objects of the poset
        \todotextjira{116}{@J: Finish this remark}
    \end{remark}
}

\vfill

\begin{gradedexercise}[\exname{TwistedCat}]
    \label{ex:TwistedCat}
    Prove that \cref{def:twisted-arrow-category} does define a category.
\end{gradedexercise}

\solutionof{TwistedCat}
