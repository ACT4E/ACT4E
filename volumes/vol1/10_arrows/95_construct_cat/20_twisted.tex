% !TEX root = chapter-standalone.tex

\section{Twisted arrow construction}
\todotext{\alphubel: @JL/@Gioele: Don't use \str{stylemorph}, use the various $\morc,\mord,\more,\etc.$
    It's very hard to edit, otherwise.
    Don't forget to change the graphics construction.
}
\begin{ctdefinition}[Twisted arrow category]
    \label{def:twisted-arrow-category}
    \label{def:twisted-arrow}
    Given a category \CatC, its \maindef{twisted arrow category} $\TwistedArrow\CatC$ is:
    \begin{enumerate}
        \item \emph{Objects:} morphisms in \CatC.
        \item \emph{Morphisms:}
              A morphism in $\TwistedArrow\CatC$ from $\mora\colon \Obja \mtoin{\CatC} \Objb $ to $\morb \colon \Objc \mtoin{\CatC} \Objd$ is given by a pair of morphisms $\tup{\twarrowmorasource,\twarrowmoratarget}$ in $\CatC$ such that the following diagram commutes:
              \begin{equation}\label{eq:180_twistedarrow}
                  \includesag{180_twistedarrow}
              \end{equation}
        \item \emph{Composition:} Composition in~$\TwistedArrow\CatC$ is given by placing \SY{commutative squares} side by side.
              Consider~$\tup{\twarrowmorasource,\twarrowmoratarget} \colon \mora \mto \morb$ and~$\tup{\twarrowmorbsource,\twarrowmorbtarget} \colon \morb \mto \morc$ in~$\Arrow\CatC$, giving rise to the following composite diagram
              \begin{equation}\label{eq:180_twistedarrow_comp}
                  \includesag{180_twistedarrow_comp}.
              \end{equation}
              It is commutative because its component squares are, and hence it defines a morphism $\mora \mto \morc$ in $\TwistedArrow\CatC$, namely
              \begin{equation}\label{eq:twarrow-cat-composition}
                  \tup{\twarrowmorasource,\twarrowmoratarget}\mthenof{\Arrow \CatC} \tup{\twarrowmorbsource,\twarrowmorbtarget}
                  \definedas \tup{ \twarrowmorbsource \mthenof\CatC  \twarrowmorasource,    \twarrowmoratarget\mthenof\CatC \twarrowmorbtarget }.
              \end{equation}
        \item \emph{Identitites:} given an object $\mora \colon \Obja \mto \Objb$ of $\TwistedArrow\CatC$, its identity morphism is $\tup{\catid_{\Obja},\catid_{\Objb}}$.
    \end{enumerate}
\end{ctdefinition}

\begin{remark}
    The above construction might be more precisely called the ``source twisted arrow category'' of $\CatC$, because it is a modification of the arrow construction where we are ``twisting'' the arrow between source objects of morphisms of $\CatC$ by reversing its direction.
    An analogous construction exists where instead we twist the arrow construction by reversing the arrow between targets of morphisms of $\CatC$.
    This latter construction might be called the ``target twisted arrow category''.
    For brevity, we have only spelled out the source twisted variant here.
\end{remark}

\begin{example}[Twisted construction in posets and categories]
    \label{exa:twisted-arrow-poset}
    Consider a poset~\posA.
    The twisted arrow category~$\TwistedArrow \poscat{\posA}$ is isomorphic to the \SY{poset} (viewed as a category)~$\poscat{\pars{\posint\posA}}$ of nonempty \emph{intervals} in~\posA:
    \begin{equation}
        \TwistedArrow (\poscat \posA) \simeq \poscat (\posint\posA) .
    \end{equation}
    \todotext{\alphubel: Did we define what is an isomorphic category, at this point?}
\end{example}

\begin{exercise}
    Prove the statement in \cref{exa:twisted-arrow-poset}.
    Recall that, given elements~$\posela, \poselb \setin \posAset$, the interval~$\interv{\posela}{\poselb}$ is
    \begin{equation}
        \interv{\posela}{\poselb}
        \definedas \makeset{\poselc \setin \posAset \mid \posela\posAleq \poselc \posAleq \posela}.
    \end{equation}
    Start to show that the partial order is equivalent to a twisted morphism.
\end{exercise}
\begin{solution}
    Consider~$\poscat{\pars{\posint\posA}}$.
    Take two morphisms in~\posA: $\mora\colon \Obja \mto \Objb$ (from~$\Obja \posleq \Objb$) and $\morb\colon \Objc\mto \Objd$ (from~$\Objc \posleq \Objd$).
    These are two objects in~$\poscat{\pars{\posint\posA}}$.
    Now, a morphism in~$\poscat{\pars{\posint\posA}}$ is a pair~$\tup{\morc,\mord}$ where~$\morc\colon \Objc \mto \Obja$ (from~$\Objc\posleq \Obja$) and~$\mord\colon \Objb \mto \Objd$ (from~$\Objb\posleq \Objd$).
    Therefore, we have~$\Objc\posleq \Obja\posleq \Objb\posleq \Objd$, which corresponds to~$\interv{\Obja}{\Objb}\posleqof{\posint{\posA}}\interv{\Objc}{\Objd}$.
    Therefore, morphisms in~$\poscat{\pars{\posint\posA}}$ between arrows (intervals) correspond to order relations between intervals.
\end{solution}

\devel{
    \begin{remark}
        Recall \cref{sec:posetsarecats} and note that the map which sends a \SY{poset} (a category) to its twisted arrow category is a \SY{functor}, which sends objects of the poset
        \todotextjira{116}{\alphubel: @JL: Finish remark about twisted map functor}
    \end{remark}
}

\vfill

\begin{gradedexercise}[\exname{TwistedCat}]
    \label{ex:TwistedCat}
    Prove that \cref{def:twisted-arrow-category} does define a category.
\end{gradedexercise}

\solutionof{TwistedCat}
