% !TEX root = chapter-standalone.tex
\section{\statusmissing{Graph homomorphisms}}

\begin{definition}[Graph homomorphism]
\label{def:graph_homom}
Given graphs~$\graph=\tup{\vertices,\arcs,\source,\target}$ and~$\graph'=\tup{\vertices',\arcs',\source',\target'}$, a graph homomorphism~$\mapa \colon \graph \to \graph'$ is given by maps~$\mapa_0\colon \vertices \to \vertices'$ and~$\mapa_1\colon \arcs \to \arcs'$, such that the following diagrams commute:
 \begin{center}
\includesag{60_graph_homomorphism}
 \end{center}
\end{definition}
\begin{remark}
Intuitively, all this is saying is that ``arrows are bound to their vertices'', meaning that if a vertex~$\vertexa_1$ is connected to~$\vertexa_2$ via an arrow~$\arc$, the vertices~$\mapa_0(\vertexa_1)$ and~$\mapa_0(\vertexa_2)$ have to be connected via an arrow~$\mapa_1(\arc)$.
\end{remark}

\begin{example}
Let us consider the two graphs,~$\graph$ and~$\graph'$ depicted in \cref{fig:ex_graph_homom}. 

\begin{figure}[h]
\begin{center}
\begin{tikzpicture}
\node (first) at (0,0){\begin{tikzcd}[column sep=large, row sep=large]
\vertexa_1\arrow{r}{\arc_1}\arrow[d,bend right=20,"\arc_2", swap]\arrow[d,bend left=20,"\arc_3"]&\vertexa_2&\\
\vertexa_3\arrow{r}{\arc_4}&\vertexa_4\arrow{r}{\arc_5}&\vertexa_5
\end{tikzcd}};
\node[right=2cm of first]
{\begin{tikzcd}[column sep=large, row sep=large]
\vertexa_1'\arrow[r,shift right=4pt, "\arcb_1", swap]\arrow[d,"\arcb_2", swap]&\vertexa_2'\arrow[l,shift right=4pt,"\arcb_3",swap]\\
\vertexa_3'&\vertexa_4'\arrow[loop right, "\arcb_4"]
\end{tikzcd}};
\end{tikzpicture}
\end{center}
\end{figure}
\todo{finish}
\end{example}

\begin{example}[Counterexample]
\todo{give counterexample}
\end{example}

\begin{exercise}
\todo{write exercise in which we ask to find maps $\mapa_1$ given $\mapa_0$}
\end{exercise}