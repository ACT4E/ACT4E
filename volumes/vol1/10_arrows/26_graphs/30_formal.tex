% !TEX root = chapter-standalone.tex
\section{\statusdraft{Formal definition of (semi) category}}



\begin{ctdefinition}[Semicategory]
  \label{def:semicategory}
  A \emph{\iindex{semicategory}}~\CatC is:
\begin{quote}
    \constit
  \begin{compactenum}
    \item Objects: a collection\footnotemark~$\ObC$, whose elements are called \emph{objects}.
    \item Morphisms: for every pair of objects~$\Obja,\Objb\in \ObC$, there is a set~$\HomSet{\CatC}{\Obja}{\Objb}$, elements of which are called
    \emph{morphisms} from~$\Obja$ to~$\Objb$. The set is called the ``hom-set from~$\Obja$ to~$\Objb$''.
    \item Composition operations: given any morphism~${\mora} \in  \HomSet{\CatC}{\Obja}{\Objb}$ and any morphism~${\morb} \in \HomSet{\CatC}{\Objb}{\Objc}$, there exists a morphism~$\mora \mthen \morb\in \HomSet{\CatC}{\Obja}{\Objc}$ which is the \emph{composition of~$\mora$ and~$\morb$}.
  \end{compactenum}
 \condit
  \begin{compactenum}
    \item Associativity: for any morphisms~$\mora\in \HomSet\CatC\Obja\Objb$,~$\morb\in \HomSet\CatC\Objb\Objc$, and~$\morc\in \HomSet\CatC\Objc\Objd$,
    \begin{equation}
    (\mora\mthen \morb)
      \mthen \morc= \mora \mthen (\morb \mthen \morc).
    \end{equation}
  \end{compactenum}
  \end{quote}
\end{ctdefinition}


\begin{ctdefinition}[Semicategory]
  \label{def:semicategory-compact }
  A \emph{\iindex{semicategory}}~\CatC is:
\begin{quote}
    \constit
  \begin{compactenum}
    \item A collection\footnotemark~$\ObC$ whose elements are called \emph{objects}.
    \item For every pair of objects~$\Obja,\Objb$ in $\ObC$, there is a set~$\HomSet{\CatC}{\Obja}{\Objb}$, elements of which are called \emph{morphisms}. We write 
    \begin{equation}
    \mora : \Obja \to_\CatC \Objb 
    \end{equation}
    to indicate 
    \begin{equation}
      \mora \in \HomSet{\CatC}{\Obja}{\Objb}.
      \end{equation}
    \item For every three objects~$\Obja,\Objb,\Objc$ in $\ObC$ there is a composition map 
    \begin{equation}
        \mthen_{\Obja,\Objb,\Objc}: \HomSet{\CatC}{\Obja}{\Objb} \times \HomSet{\CatC}{\Objb}{\Objc}\to \HomSet{\CatC}{\Obja}{\Objc}
    \end{equation}
        operations: given any morphism~${\mora} \in  \HomSet{\CatC}{\Obja}{\Objb}$ and any morphism~${\morb} \in \HomSet{\CatC}{\Objb}{\Objc}$, there exists a morphism~$\mora \mthen \morb\in \HomSet{\CatC}{\Obja}{\Objc}$ which is the \emph{composition of~$\mora$ and~$\morb$}.
  \end{compactenum}
 \condit
  \begin{compactenum}
    \item Associativity:  it holds that 
    % for any morphisms~$\mora\in \HomSet\CatC\Obja\Objb$,~$\morb\in \HomSet\CatC\Objb\Objc$, and~$\morc\in \HomSet\CatC\Objc\Objd$,
    \begin{equation}
      \prftree{\mora: \Obja \to \Objb}{\morb: \Objb \to \Objc}{\morc: \Objc \to \Objd}
      {
    (\mora\mthen \morb)
      \mthen \morc= \mora \mthen (\morb \mthen \morc)
      } 
    \end{equation}
  \end{compactenum}
  \end{quote}
\end{ctdefinition}


\todotext{Remove the part about the motor, now moved later.}

The following is the formal definition.

%\let\oldmora\mora
%\renewcommand{\mora}{{\colTransmuter \oldmora}}
%\let\oldmorb\morb
%\renewcommand{\morb}{{\colTransmuter \oldmorb}}
%\let\oldmorc\morc
%\renewcommand{\morc}{{\colTransmuter \oldmorc}}
%\let\oldObja\Obja
%\renewcommand{\Obja}{{\colTransmuted \oldObja}}
%\let\oldObjb\Objb
%\renewcommand{\Objb}{{\colTransmuted \oldObjb}}
%\let\oldObjc\Objc
%\renewcommand{\Objc}{{\colTransmuted \oldObjc}}
%\let\oldObjd\Objd
%\renewcommand{\Objd}{{\colTransmuted \oldObjd}}
%
%\let\oldOb\Ob
%\renewcommand{\Ob}{{\colTransmuted \oldOb}}
%
%\let\oldHom\Hom
%\renewcommand{\Hom}{{\colTransmuter \oldHom}}


\begin{ctdefinition}[Category]
  \label{def:categorymain}
  A \emph{\iindex{category}}~\CatC is:
\begin{quote}
    \constit
  \begin{compactenum}
    \item Objects: a collection\footnotemark~$\ObC$, whose elements are called \emph{objects}.
    \item Morphisms: for every pair of objects~$\Obja,\Objb\in \ObC$, there is a set~$\HomSet{\CatC}{\Obja}{\Objb}$, elements of which are called
    \emph{morphisms} from~$\Obja$ to~$\Objb$. The set is called the ``hom-set from~$\Obja$ to~$\Objb$''.
    \item Identity morphisms: for each object~$\Obja$, there is
    an element~$\catid_{\Obja} \in \HomSet{\CatC}{\Obja}{\Obja}$ which is called \emph{the identity
    morphism of~$\Obja$}.
    \item Composition operations: given any morphism~${\mora} \in  \HomSet{\CatC}{\Obja}{\Objb}$ and any morphism~${\morb} \in \HomSet{\CatC}{\Objb}{\Objc}$, there exists a morphism~$\mora \mthen \morb\in \HomSet{\CatC}{\Obja}{\Objc}$ which is the \emph{composition of~$\mora$ and~$\morb$}.
  \end{compactenum}
 \condit
  \begin{compactenum}
    \item Unitality: for any \iindex{morphism}~$\mora\in \HomSet\CatC\Obja\Objb$,
    \begin{equation}
      \catid_\Obja \mthen \mora= \mora = \mora \mthen \catid_\Objb.
    \end{equation}
    \item Associativity: for any morphisms~$\mora\in \HomSet\CatC\Obja\Objb$,~$\morb\in \HomSet\CatC\Objb\Objc$, and~$\morc\in \HomSet\CatC\Objc\Objd$,
    \begin{equation}
    (\mora\mthen \morb)
      \mthen \morc= \mora \mthen (\morb \mthen \morc).
    \end{equation}
  \end{compactenum}
  \end{quote}
\end{ctdefinition}
\footnotetext{A ``collection'' is something which may be thought of as a set, but may be ``too large" to technically be a set in the formal sense. This distinction is necessary in order to avoid such issues as Russel's paradox.}

\begin{gradedexercise}[Category of semigroups]
There is a category where the objects are semigroups and the morphisms are semigroup homomorphisms. Spell out explicitely what this category is: check in detail each of the points of \cref{def:categorymain}.
\end{gradedexercise}


\begin{remark}[Are we sure we are not going in the wrong direction?]
  We denote composition of morphisms in a somewhat unusual way--sometimes preferred by category-theorists and computer scientists--namely in \emph{diagrammatic order}.

  That is, given~$\mora \colon \Obja\to \Objb$ and~$\morb \colon \Objb \to \Objc$, we denote their composite by~$(\mora\mthen \morb)\colon \Obja\to \Objc$, pronounced ``$\mora$ then~$\morb$''. This is in contrast to the more typical notation for composition, namely~$\morb\after \mora$, or simply~$\morb \mora$, which reads as ``$\morb$ after~$\mora$''. The notation~$\mora \mthen \morb$ is sometimes called \emph{infix notation}.

  We promise, at some point it will be clear what are the advantages of seemingly doing everything
  in the wrong direction.
\end{remark}

Note that we may save some ink when drawing diagrams of morphisms:
\begin{compactitem}
  \item We do not need to draw the identity arrows from one object to itself, because, by \cref{def:categorymain}, they always exist.
%However, we will see how there might be multiple such loops.
  \item  Given arrows~$\Obja \mto \Objb$ and~$\Objb \mto \Objc$, we do not need to draw their composition because, by \cref{def:categorymain}, this composition is guaranteed to exist.
\end{compactitem}

With these conventions, we can just draw the arrows \motor and \wheels in the diagram, and the rest of the diagram is implied (\cref{fig:e5}).

\begin{figure}[h!]
  \centering
  \includesag{30_dpcatfig_e5}
  \caption{
    Electric car example.
    The \textcolor{gray}{gray} arrows are implied by the properties
    of a category.
  }
  \label{fig:e5}
\end{figure}

In particular, the electric car example corresponds to the category~\CatC specified by
\begin{compactitem}
  \item \emph{Objects:} $\ObC=\{ \electricpower,\rotationalmotion,\translationalmotion\}$.
  \item \emph{Morphisms}: The system components are the morphisms. For instance, we have \motor, \wheels, and the morphism $\wheels \mthen \motor$, implied by the properties of the category.
\end{compactitem}

We can slightly expand this example by noting the reverse transformations. In an electric car
it is possible to regenerate power; that is, we can obtain \rotationalmotion of the \wheels from
\translationalmotion (via the morphism \move), and then convert the \rotationalmotion into \electricpower (via the morphism \dynamo)~(\cref{fig:e6},~\cref{fig:e6-together}).

\begin{figure}[h!]
  \centering
  \includesag{30_dpcatfig_e6}
  \caption{Electric power can be produced from motion.}
  \label{fig:e6}
\end{figure}

\begin{figure}[h!]
  \centering
  \includesag{30_dpcatfig_e7}
  \caption{Electric car example: forward and backward transformations.\label{fig:e6-together}}
\end{figure}
Given the semantics of the arrows in a category, all compositions of arrows exist, even if they are not drawn
explicitly. For example, we can consider the composition
\begin{equation*}
    \wheels \mthen \motor \mthen \dynamo \mthen \move,
\end{equation*}
which converts \translationalmotion into \rotationalmotion, into \electricpower, then back to
\rotationalmotion and \translationalmotion. Note that this is an arrow that has the same head and tail as the identity arrow on \translationalmotion~(\cref{fig:e8}). However, these two arrows are not necessarily the same. In this example we are representing physical systems, so we would in fact not expect them to be the same, since there will be some losses during the many conversions.

\begin{figure}[h!]
  \centering
  \includesag{30_dpcatfig_e8}
  \caption{There can be multiple morphisms from an object to itself.}
  \label{fig:e8}
\end{figure}

The directionality of the arrows is also important. While the convention of
which resource is the tail and which the head is just a typographic convention,
it might be the case that we know how to convert one resource into another, but
not vice versa. \cref{fig:e10} shows an example of a diagram that describes a process which is definitely
not invertible.

\begin{figure}[h!]
  \centering
  \includesag{30_dpcatfig_e10}
  \caption{An example of a process which is not invertible. }
  \label{fig:e10}
\end{figure}



\begin{example}
  Given any category~\CatC, and any object~$\Obja\in \CatC$, the set of \emph{endomorphisms}~$\Hom_{\CatC}(\Obja,\Obja)$ is a monoid. The category depicted in \cref{fig:monoid_endomorphisms} has three objects~$\Obja,\Objb,\Objc$ and several morphisms.~$\Obja$ has four endomorphisms,~$\Objb$ two, and~$\Objc$ three (including identity morphisms). Let's now take the binary operation~$\mtimes$ to be the composition~$\then$ in~\CatC, and the neutral element to be the identity~$\catid_\Obja$. The associativity and unitality laws of the category~\CatC coincide with the ones of the monoid's definition, and are satisfied. Therefore, we can identify a monoid as a one-object category.
\end{example}


\begin{figure}[h!]
  \begin{center}
    \includesag{043_monoid_endomorphisms}
    \caption{}
    \label{fig:monoid_endomorphisms}
  \end{center}
\end{figure}


\devel{
  \begin{forslides}
\begin{ctdefinition}[Semicategory]
  \label{def:categorymain_2}
  A \emph{\iindex{semicategory}}~\CatC is:
\begin{quote}
    \constit
  \begin{compactenum}
    \item Objects: a collection\footnotemark~$\ObC$, whose elements are called \emph{objects}.
    \item Morphisms: for every pair of objects~$\Obja,\Objb\in \ObC$, there is a set~$\HomSet{\CatC}{\Obja}{\Objb}$, elements of which are called
    \emph{morphisms} from~$\Obja$ to~$\Objb$. The set is called the ``hom-set from~$\Obja$ to~$\Objb$''.
    \item Composition operations: given any morphism~${\mora} \in  \HomSet{\CatC}{\Obja}{\Objb}$ and any morphism~${\morb} \in \HomSet{\CatC}{\Objb}{\Objc}$, there exists a morphism~$\mora \mthen \morb\in \HomSet{\CatC}{\Obja}{\Objc}$ which is the \emph{composition of~$\mora$ and~$\morb$}.
  \end{compactenum}
 \condit
  \begin{compactenum}
    \item Associativity: for any morphisms~$\mora\in \HomSet\CatC\Obja\Objb$,~$\morb\in \HomSet\CatC\Objb\Objc$, and~$\morc\in \HomSet\CatC\Objc\Objd$,
    \begin{equation}
    (\mora\mthen \morb)
      \mthen \morc= \mora \mthen (\morb \mthen \morc).
    \end{equation}
  \end{compactenum}
  \end{quote}
\end{ctdefinition}

\begin{ctdefinition}[Category]
  \label{def:categorymain_3}
  A \emph{\iindex{category}}~\CatC is:
\begin{quote}
    \constit
  \begin{compactenum}
    \item Objects: a collection\footnotemark~$\ObC$, whose elements are called \emph{objects}.
    \item Morphisms: for every pair of objects~$\Obja,\Objb\in \ObC$, there is a set~$\HomSet{\CatC}{\Obja}{\Objb}$, elements of which are called
    \emph{morphisms} from~$\Obja$ to~$\Objb$. The set is called the ``hom-set from~$\Obja$ to~$\Objb$''.
    \item Identity morphisms: for each object~$\Obja$, there is
    an element~$\catid_{\Obja} \in \HomSet{\CatC}{\Obja}{\Obja}$ which is called \emph{the identity
    morphism of~$\Obja$}.
    \item Composition operations: given any morphism~${\mora} \in  \HomSet{\CatC}{\Obja}{\Objb}$ and any morphism~${\morb} \in \HomSet{\CatC}{\Objb}{\Objc}$, there exists a morphism~$\mora \mthen \morb\in \HomSet{\CatC}{\Obja}{\Objc}$ which is the \emph{composition of~$\mora$ and~$\morb$}.
  \end{compactenum}
 \condit
  \begin{compactenum}
    \item Unitality: for any \iindex{morphism}~$\mora\in \HomSet\CatC\Obja\Objb$,
    \begin{equation}
      \catid_\Obja \mthen \mora= \mora = \mora \mthen \catid_\Objb.
    \end{equation}
    \item Associativity: for any morphisms~$\mora\in \HomSet\CatC\Obja\Objb$,~$\morb\in \HomSet\CatC\Objb\Objc$, and~$\morc\in \HomSet\CatC\Objc\Objd$,
    \begin{equation}
    (\mora\mthen \morb)
      \mthen \morc= \mora \mthen (\morb \mthen \morc).
    \end{equation}
  \end{compactenum}
  \end{quote}
\end{ctdefinition}




$$ \mora \quad \morb \quad \morc $$

$$X = \{ \sprout, \yng, \mature, \old, \dead \}$$


$$T \colon X \sto X$$


  \begin{align*}
    T(\text{sprout}) &=  \yng \\
    T(\yng) &=  \mature \\
    T(\mature) &=  \old \\
    T( \old) &= \dead \\
    T (\dead) &= \dead
  \end{align*}


  $$\sgrpA = \{ T^n \mid n \in \natnumbers \}$$

$$ T^n = T^4 \quad \forall \ n \geq 4 $$

  $$\sgrpA = \{T, T^2, T^3, T^4 \}$$

  $$  \setA = \{ \alphabeta, \alphabetb \}$$

  $$\sgrpA = \{ \text{set of non-empty strings of elements of }\setA\}$$

  $$Y = \{ \text{alive}, \dead \}$$

  $$ T' : Y \sto Y $$

  $$\sgrpA' = \{T' \}$$


   \begin{align*}
    T'(\text{alive}) &=  \dead \\
    T' (\dead) &= \dead
  \end{align*}

  $$ R: X \sto Y $$

    \begin{align*}
    R(\text{sprout}) &= \text{alive} \\
    R(\yng) &=  \text{alive} \\
    R(\mature) &= \text{alive}\\
    R( \old) &= \text{alive} \\
    R (\dead) &= \dead
  \end{align*}

   $$\monoidA = \{ \id, T, T^2, T^3, T^4 \}$$

   $$\monoidA' = \{ \id, T' \}$$

  \begin{equation*}
    \worda \text{ and } \wordb
  \end{equation*}

  \begin{equation*}  \label{eq:string-semigroup-wordab-2}
    \worda \mtimes  \wordb =  \worda \wordb
  \end{equation*}


  $$ \tup{ \monoidA, \mtimes, \idmon} $$


  \begin{equation*} \label{eq:string-semigroup-wordab-concat-2}
     \worda \wordb
  \end{equation*}


  \end{forslides}}


