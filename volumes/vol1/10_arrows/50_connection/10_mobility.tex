% !TEX root = chapter-standalone.tex


\section{Mobility}


For a specific mode of transportation, say a car, we can define a graph
\begin{equation*}
    \graph_\mathrm{c}=\tup{\vertices_\mathrm{c},\arcs_\mathrm{c},\source_\mathrm{c},\target_\mathrm{c}},
\end{equation*}
where~$\vertices_\mathrm{c}$ represents geographical locations which the car can reach and~$\arcs_\mathrm{c}$ represents the paths it can take, such as roads.
Similarly, we consider a graph~$\graph_\mathrm{s}=\tup{\vertices_\mathrm{s},\arcs_\mathrm{s},\source_\mathrm{s},\target_\mathrm{s}}$, representing the subway system of a city, with stations~$\vertices_\mathrm{s}$ and subway lines going through paths~$\arcs_\mathrm{s}$, and a graph $\graph_\mathrm{b}=\tup{\vertices_\mathrm{b},\arcs_\mathrm{b},\source_\mathrm{b},\target_\mathrm{b}}$, representing onboarding and offboarding at airports.
In the following, we want to express intermodality: the phenomenon that someone might travel to a certain intermediate location in a car and then take the subway to reach their final destination.

By considering the graph~$\graph=(\vertices,\arcs,\source,\target)$ with~$\vertices=\vertices_\mathrm{c}\cup \vertices_\mathrm{s}\cup \vertices_\mathrm{b}$ and~$\arcs=\arcs_\mathrm{c}\cup \arcs_\mathrm{s}\cup \arcs_\mathrm{b}$, we obtain the desired intermodality graph.
Graph~$\graph$ can be seen as a new category, with objects~$\vertices$ and morphisms~$\arcs$.
\begin{example}
    \label{exa:car-category}
    Consider the \Car category, describing your road trip through Italy and Switzerland, with
    \begin{equation*}
        \vertices_\mathrm{c}=\{\FCOc, \transmuted{Florence},\transmuted{Bologna},\MPXc,\transmuted{Gotthard},\ZRHc\},
    \end{equation*}
    and arrows as in~\cref{fig:carcat}.
The nodes represent typical touristic road-trip checkpoints in Italy and Switzerland and the arrows represent highways connecting them.
    
    \begin{figure*}[h!]
        \includesag{30_carcategory}
        \caption{The \Car category.}
        \label{fig:carcat}
    \end{figure*}
    
    Furthermore, consider the \Flight category with $\vertices_\mathrm{f}=\{\FCOf, \LIN, \MPXf, \ZRHf\}$ and arrows as in~\cref{fig:flight}.
    The nodes represent airports in Italy and Switzerland and the arrows represent connections, offered by specific flight companies.
    
    \begin{figure}[h!]
        \centering
        \includesag{30_flight}
        \caption{The \Flight category.}
        \label{fig:flight}
    \end{figure}
    
    We then consider the \Board category, with nodes
    \begin{equation*}
        \vertices_\mathrm{b}=\{\FCOf,\FCOc,\MPXf,\MPXc,\ZRHf,\ZRHc\}
    \end{equation*}
    and arrows as in~\cref{fig:boarding}.
Nodes represent airports and airport parkings, and arrows represent the onboarding and offboarding paths one has to walk to get from the parkings to the airport and vice-versa.
    
    \begin{figure}[h!]
        \centering
        \includesag{30_boarding}
        \caption{The \Board category. }
        \label{fig:boarding}
    \end{figure}
    
    
    The combination of the three, which we call the \emph{intermodal graph}, can be represented as a graph, in which we use dashed arrows for intermodal morphisms, arising from composition of morphisms involving multiple modes (\cref{fig:intermodal}).
    Imagine that you are in the parking lot of \ZRH airport and you want to reach \transmuted{Florence}.
    From there, you can onboard to a \transmuter{Swiss} flight to~$\FCOf$, will then offboard reaching the parking lot~$\FCOc$, and drive on highway \transmuter{A1} reaching \transmuted{Florence}.
    This is intermodality.
    
    \begin{figure*}[h!]
        \centering
        \includesag{30_intermodal}
        \caption{Intermodal graph.
The dashed arrows represent intermodal morphisms, and we depict just one of them for simplicity.
        }
        \label{fig:intermodal}
    \end{figure*}
\end{example}

The intermodal network category~\Intermodal is the free category on the graph illustrated in \cref{fig:intermodal}.
