% !TEX root = chapter-standalone.tex

\section{Trekking in the Swiss Mountains}
\label{sec:trekking}

In the section we'll discuss a more ``continuum-flavored'' (as opposed to ``discrete-flavored'') example of how one might describe ``connectedness'' using a category.

Suppose we are planning a hiking tour in the Swiss Alps.
In particular, we wish to consider various routes for hikes.
We have a map of the relevant region which uses coordinates~$\tup{x,y,z}$.
We assume the~$z$-th coordinate is given by an ``elevation function'',~$z = h(x,y)$, and that~$h$ is~$C^1$ (a continuously differentiable function).
This means that our map of the landscape forms a~$C^1$-manifold; let's call it~$L$.

\todotextjira{43}{@J: Let's not assume that people know what is a manifold.
    This was also a questions.}

We will now define a category where the morphisms are built from~$C^1$ paths through the landscape, and such that these paths can be composed, essentially, by concatenation.
We take paths which are~$C^1$ so that we can speak of the slope (steapness) of a path in any given point, as given by its derivative.
\todographicsjira{44}{@Gioele: Add a figure here.
    To the left a picture of a Swiss mountain, to the right a stylization of the same mountain as a 3D object with some paths traced.
    Maybe this is better to do without TikZ.}

To set things up, we need to have a way to compose~$C^1$ paths such that their composition is again~$C^1$.
For this, the derivative (velocity) at the end of one path must match the starting velocity of the subsequent path.

\todotext{@Gioele.
    This is not using the convention for defining categories.}

\begin{definition}[\Berg]
    \label{def:Berg}
    Let \Berg be the category defined as follows:
    \begin{compactitem}
        \item Objects are tuples~$\tup{p,v}$, where
        \begin{compactitem}
            \item $p \in L$,
            \item $v \in \reals^3$ (we think of this as a tangent vector to~$L$ at~$p$).
        \end{compactitem}

        \item A morphism~$\tup{p_1, v_1} \to \tup{p_2, v_2}$ is~$\tup{\gamma, T}$,
        where
        \begin{compactitem}
            \item $T \in \nonNegReals$,
            \item $\gamma \colon [0, T] \sto L$ is a~$C^1$ function with~$\gamma(0)=p_1$ and~$\gamma(T)=p_2$, as well as~$\dot \gamma(0) = v_1$ and~$\dot \gamma(T) = v_2$  (we take one-sided derivatives at the boundaries).
        \end{compactitem}
        \item For any object~$\tup{p, v}$, we define its identity morphism~$\catid_{\tup{p,v}} = \tup{\gamma, 0}$ formally: its path~$\gamma$ is defined on the closed interval~$[0,0]$, (with $T=0$ and $\gamma(0) = p$).
        We declare this path to be~$C^1$ by convention, and declare its derivative at~$0$ to be~$v$.

        \item Given morphisms~$\tup{\gamma_1, T_1}\colon \tup{p_1, v_1} \to \tup{p_2, v_2}$ and~$\tup{\gamma_2, T_2}\colon \tup{p_2, v_2} \to \tup{p_3, v_3}$, their composition is~$\tup{\gamma, T}$ with~$T = T_1 + T_2$ and
        \begin{equation}
            \gamma(t) = \begin{cases}
                \gamma_1(t)       & 0 \leq t \leq T_1          \\
                \gamma_2(t - T_1) & T_1 \leq t \leq T_1 + T_2. 
            \end{cases}
        \end{equation}
    \end{compactitem}
\end{definition}

\todographicsjira{45}{@Gioele: Make a technical sketch of the manifold showing what are the velocities,
    how do paths look like, etc.}

Since we are only amateurs, we don't feel comfortable with hiking on paths that are too steep in some places.
We want to only consider paths that have a certain maximum inclination.
Mathematically speaking, for any path -- as described by a morphism~$\tup{\gamma, T}$ in the category \Berg -- we can compute its vertical inclination (vertical slope) and renormalize it to give a number in the interval~$(-1, 1)$, say. (Here~$-1$ represents vertical descent, and~$1$ represents vertical ascent.) Taking absolute values of inclinations -- call the resulting quantity ``steepness'' -- we can compute the maximum steepness that a path~$\gamma$ obtains over its domain~$[0,T]$.
This gives, for every homset~$\HomSet{\Berg}{\tup{p_1, v_1}}{\tup{p_2, v_2}}$, a function
\begin{equation*}
    \mathsf{MaxSteepness}\colon \HomSet{\Berg}{\tup{p_1, v_1}}{\tup{p_2, v_2}} \longrightarrow [0, 1).
\end{equation*}
Now, suppose we decide that we don't want to traverse paths which have a maximal steepness greater than~$1/2$.
Paths which satisfy this condition we call \emph{feasible}.
Let's consider only the feasible paths in \Berg.
If we keep the same objects as \Berg, but only consider feasible path, will the resulting structure still form a category?
Should we restrict the set of objects for this to be true?
We'll let you ponder here; this type of question leads to the notion of a \emph{subcategory}, which we'll introduce soon in a subsequent chapter.

%\gray{
%-------------------------------------------------------------------------
%
%Supposed we are tasked with managing a scientific mission for a Mars rover. In particular, we need need to plan the route that the rover will take in order to travel from its landing position ``$a$'' to a target destination ``$b$''. We have a map of the relevant region on Mars, complete with elevation data, but, of course, only to a certain degree of accuracy.
%
%To model the landscape, we divide it into a grid of~$1 \times 1$ meter squares, and the center of each is labeled with an~$\tup{x,y}$ coordinate. In total, our model has~$1000 \times 1000$ squares, and we let~$L$ denote the set of coordinates of the centers of the squares. We think of these coordinate labels as the objects of a category~\CatC which are the possible (approximate) locations that the rover might be. If the rover is at a given location~$l = \tup{x,y}$, then in our model there are eight possible directions that the rover can move:
%
%\begin{figure}[h!]
%\centering
% \includegraphics[width=0.5\linewidth]{pics/path_planning_1.png}
% \caption{}
%\label{fig:rover_moves}
%\end{figure}
%
%
%If we draw such arrows for each location label $c \in \CatC$, then we obtain a (rather full-looking) directed graph such as indicated in Figure \ref{fig:move_graph}.
%
%\begin{figure}[h!]
%\centering
% \includegraphics[width=0.5\linewidth]{pics/path_planning_2.png}
% \caption{}
%\label{fig:move_graph}
%\end{figure}
%
%To model the possible paths the rover might potential travel, we take the free category on this graph. That is, we let the morphisms in our category \CatC be all possible paths obtained by concatenating directed arrows from the above graph Figure \ref{fig:move_graph}. In particular, given the locations~$a$ and~$b$, we have
%\begin{equation}
%\CatC(a, b) = \{ \text{paths from } a \text{ to } b \}.
%\end{equation}
%
%\begin{figure}[h!]
%\centering
% \includegraphics[width=0.5\linewidth]{pics/path_planning_3.png}
% \caption{}
%\label{fig:rover_paths}
%\end{figure}
%
%In Figure \ref{fig:rover_paths}, two possible paths are drawn in green. Note that by allowing \emph{all} possible paths we are also allowing ones of infinite length (\text{\eg } where the rover moves around indefinitely long). Since we want our rover to reach its destination in a finite amount of time, we will subsequently take \CatC to be the category where the morphisms are only the paths of \emph{finite length}.
%
%Next, we include the elevation information in our model, in order to start to optimize the planning of which path we wish the rover to take. We encode elevation data as a function~$h \colon L \to \reals$ which assigns a real number to each location label. If~$l$ and~$l'$ are two locations, the absolute elevation difference between~$l$ and $~l'$ is~$\vert h(l') - h(l) \vert$. For our rover, we only want to consider paths such that the absolute elevation difference between any two adjacent locations along that path is less than a given threshold (if the path is too steep, the rover might tip over!). This is one kind of constraint which determines certain paths in our category~\CatC to be infeasible.
%
%Among the feasible paths, we wish to optimize the path taken by the rover so that it uses the least amount of energy, say.  For any path~$p$ of finite length, we model what the energy cost~$E(p)$ of that path for the rover would be. (We skip the precise details of how we might model this; surely the energy cost of a path will be related to the length of the path, for example.) This defines a function on each of the homsets of our category \CatC,
%\begin{equation}
%E \colon \CatC(l, l') \longrightarrow \reals,
%\end{equation}
%which, to each path from~$l$ to~$l'$, assigns the corresponding energy cost of that path. Thus our specific optimization problem is to find those feasible paths which minimize the function~$E\colon \CatC(a,b) \to \reals$ defined on paths from the starting position~$a$ to the target position~$b$.
%
%--------------------------------
%
%
%
%\
%
%
%Consider a geographical region whose locations are expressed through coordinates~$(x,y)\in \reals^2$, \text{\eg } as given by a map of that region. Furthermore, consider a function~$\mathsf{alt}: \reals^2 \to \reals_{\geq 0}$ which, for a known location, returns its altitude.
%
%We can think about this situation using a category, call it~$\mathbf{trek}$, where objects are geographical locations~$\tup{x,y}\in \reals^2$ and morphisms are continuous paths between them. The identity morphism for each location consists of the trivial path (\ie , not moving), and composition is given by concatenation of paths.
%\JL{We need to be more precise about what ``continuous path'' means here! The typical mathematical definition of paths from topology is as a function of a (``time'') parameter, and leads to a well-known situation where concatenation is not an associative operation on the nose... and/or there is also issue that there are crazy kinds of continuous paths, such as space-filling curves... perhaps this example can be modified a bit to capture the basic idea, but avoid the math issues...}
%\GZ{Agree, here we just want the connectivity and the filtering of paths which have too large inclinations}
%
%Suppose that a human can only traverse trails which have a maximum inclination of $\alpha>0$ when going uphill and $\beta>0$ when going downhill.
%We can now think of the aforementioned human, wanting to go from a location~$\tup{x,y}$ to a location~$\tup{v,w}$. Finding a path consists of finding at least a morphism in~$\Hom_\mathbf{trek}(\tup{x,y},\tup{v,w})$ satisfying the condition on the maximum inclinations~$\alpha$ and~$\beta$.
%
%
%Using the terminology from \cref{sec:catsfromgraphs}, we can see that~$\mathbf{trek}$ is the free category on a graph with vertices given by geographical locations~$\tup{x,y}\in \reals^2$ and arrows given by paths between them. In particular, a valid path~$p\colon \tup{x,y}\mapsto \tup{v,w}$ for the human to be able to reach a destination, has not to exceed the maximum inclination~$\alpha$ when climbin and the maximum inclination~$\beta$ when descending.
%}
%-------------------------------
