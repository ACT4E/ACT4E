
\section{Extension cords}

We talked about ropes

\begin{equation*}
\prftree{\rope{ropecola}{ropecolb}{2}{0}{0}{1}{$a$}}
{\rope{ropecola}{ropecolb}{2}{0}{0}{1}{$b$}}
{\rope{ropecola}{ropecolb}{4}{0}{0}{1}{$a+b$}}
\end{equation*}
%
%
%\begin{figure}
%  \begin{center}
%    \begin{tabular}{c@{\hskip 1cm}c@{\hskip 1cm}c@{\hskip 1cm}c@{\hskip 1cm}}
%      \figplug{A} & \figplug{B} & \figplug{C} & \figplug{D} \\
%      \figplug{E} & \figplug{F} & \figplug{G} & \figplug{H} \\
%      \figplug{I} & \figplug{L} & \figplug{M} & \figplug{N}
%    \end{tabular}
%  \end{center}
%
%\end{figure}


\begin{marginfigure}
\figplug{A}

\figplug{B}

\figplug{C}

\figplug{D}

\figplug{E}

\figplug{F}

\figplug{G}

\figplug{H}

\figplug{I}

\figplug{L}

\figplug{M}

\figplug{N}
\caption{Plug/socket types used in the world}
  \label{fig:plugs}
\end{marginfigure}

Let~$\cordG{c}$ be an extension cord of length~$c$.

If you have an extension cord of length~$c$ and another of length~$d$, you can plug them together to get an extension cord of length~$c+d$.

\begin{equation}
  \label{eq:cords}
  \prftree{\cordG{c}}{\cordG{d}}{\cordG{c+d}}
\end{equation}

The rule for extension cords is similar to the rule for Lego blocks.
%
%\begin{equation}
%  \frac{\text{extension cord of length }\ c \qquad \text{extension cord of length }\ d }{%
%    \text{extension cord of length }\ (c+d)%
%  }
%\end{equation}


Extension cords are a bit more general: Lego blocks are constrainted to have a height that is a multiple of~$1/3$ of brick,  but extension cords can have any continuous value as length.

On the other hand, Lego blocks also have this other property of the horizontal section~$a \times b$.
We only gave rules for the connection of blocks of the same horizontal section.
What would be the equivalent for extension cords?


As you read this book and start to plan to visit Switzerland, at some point you need to buy some adapters.
Switzerland uses the connector of type~N (\cref{fig:TypeN}).
If you come from Ireland, your appliances use type~G (\cref{fig:TypeG}).

\begin{equation}
  \label{eq:irish-swiss}
  \prftree{\cord{\TypeIrish}{c}{\TypeIrish}}{\cord{\TypeIrish}{0}{\TypeSwiss}}{\cord{\TypeIrish}{c}{\TypeSwiss}}
\end{equation}
