% !TEX root = chapter-standalone.tex

\section{Interfaces}\label{sec:interfaces}

One way to understand \SY{semicategories} is to see them as a generalization of semigroups.
In \SY{semigroups}, \SY{monoids}, and \SY{groups} we could take any two elements and compose them: the elements always had a ``compatible'' interface.

To motivate the need for interfaces, consider the ropes of~\cref{chap:putting-things-together}, which had this composition rule:
%
\begin{equation}
    \prfperiod{
        \rope{ropecola}{ropecolb}{2}{0}{0}{1}{$a$}
    }{
        \rope{ropecola}{ropecolb}{2}{0}{0}{1}{$b$}
    }{
        \rope{ropecola}{ropecolb}{4}{0}{0}{1}{$a+b$}
    }
\end{equation}
%
Two chapters later, we can recognize that we were describing the \SY{monoid}~$\tupp{\nonNegReals,+, 0}$.
Being a \SY{monoid}, all pieces of rope are compatible and can be composed.

A first step towards discussing interfaces is to think of things that have a direction.
For example, consider extension cords.
Let~$\cordG{c}$ be an extension cord of length~$c$.
If you have an extension cord of length~$c$ and another of length~$d$, you can plug them together to get an extension cord of length~$c+d$:
%
\begin{equation}
    \label{eq:cords}
    \prfperiod{
        \cordG{c}
    }{
        \cordG{d}
    }{
        \cordG{c+d}
    }
\end{equation}
%
In this form, this is still the same \SY{monoid}.

But suppose now that, reading this book, you fall in love with Switzerland and want to visit.
As you start to plan your trip, at some point you need to think about electrical adapters.
Switzerland uses the connector of type~N (\cref{fig:plugs}).
If you come from Ireland, your appliances use type~G.
\todographicsjira{383}{\bernina: @Gioele: Make a figure that shows a physical cable and the math representation, and another with an adapter and the math representation.}
Now when we think of extension cords, we might allow either end to have a plug type.
%
These would be Irish and Swiss extension cords of length $\ell$:
%
\begin{equation}
    \label{eq:irish-swiss}
    {\cord{\TypeIrish}{\ell}{\TypeIrish} \qquad \cord{\TypeSwiss}{\ell}{\TypeSwiss}}.
\end{equation}
%
You might want a cord that has a Swiss male end and an Irish female end:
%
\begin{equation}
    \label{eq:want}
    {\cord{\TypeIrish}{\ell}{\TypeSwiss}}.
\end{equation}
%
Unfortunately these devices don't exist.
What you can buy are adapters, which we can think of extension cord of length zeros:
\begin{equation}
    {\cord{\TypeIrish}{0}{\TypeSwiss}}.
\end{equation}
%
If you have an adapter, then you can attach an extension cord to it to obtain~\cref{eq:want}:
%
\begin{equation}
    \label{eq:irish-swiss-result}
    \prfperiod{
        \cord{\TypeIrish}{\ell}{\TypeIrish}
    }{
        \cord{\TypeIrish}{0}{\TypeSwiss}
    }{
        \cord{\TypeIrish}{\ell}{\TypeSwiss}
    }
\end{equation}
%
The general formula to compose cords with generic types~$\Obja, \Objb, \Objc$ is
%
\begin{equation}
    \label{eq:cords-generic}
    \prfperiod{
        \cord{\Obja}{a}{\Objb}
    }{
        \cord{\Objb}{b}{\Objc}
    }{
        \cord{\Obja}{a+b}{\Objc}
    }
\end{equation}
%
This kind of composition of things that have an input and an output interface, like cords, can be modeled by the notions of \SY{semicategory} and category.

\todotext{\bernina: Would be nice to continue the example by considering voltages (220 v 110), voltages converters, appliances that can use one or both.}

%
%
%\begin{figure}
%  \begin{center}
%    \begin{tabular}{c@{\hskip 1cm}c@{\hskip 1cm}c@{\hskip 1cm}c@{\hskip 1cm}}
%      \figplug{A} & \figplug{B} & \figplug{C} & \figplug{D} \\
%      \figplug{E} & \figplug{F} & \figplug{G} & \figplug{H} \\
%      \figplug{I} & \figplug{L} & \figplug{M} & \figplug{N}
%    \end{tabular}
%  \end{center}
%
%\end{figure}

\begin{marginfigure}
    \centering
    \figplug{A}

    \figplug{B}

    \figplug{C}

    \figplug{D}

    \figplug{E}

    \figplug{F}

    \figplug{G}

    \figplug{H}

    \figplug{I}

    \figplug{L}

    \figplug{M}

    \figplug{N}
    \caption{Plug/socket types used in the world}
    \label{fig:plugs}
\end{marginfigure}

% \begin{equation}
%   \label{eq:irish-swiss}
%   \prftree{\cord{\TypeIrish}{c}{\TypeIrish}}{\cord{\TypeIrish}{0}{\TypeSwiss}}{\cord{\TypeIrish}{c}{\TypeSwiss}}
% \end{equation}
