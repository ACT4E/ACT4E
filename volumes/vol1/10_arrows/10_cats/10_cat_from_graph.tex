% !TEX root = chapter-standalone.tex

\section[Categories from graphs]{Generating categories from graphs}
\label{sec:catsfromgraphs}

We can turn any graph into a (semi)category.

\begin{ctdefinition}[Semicategory generated by a graph]
    \label{def:free-semicategory}
    \SYNDEF{free semicategory on a graph}

    Let~$\graph=\tup{\vertices,\arcs,\source,\target}$ be a graph.
    The \emph{free semicategory on} $\graph$, denoted $\pathscat{\graph}$, has as objects the vertices~$\vertices$ of~$\graph$, and given vertices~$\vertexa\setin \vertices$ and~$\vertexb\setin \vertices$, morphisms~$\HomSet{\CatSymbol{SC}(\graph)}{\vertexa}{\vertexb}$ are the non-trivial paths from~$\vertexa$ to~$\vertexb$.
    The composition of morphisms is given by concatenation of paths.
\end{ctdefinition}

\begin{ctdefinition}[Category generated by a graph]
    \label{def:free-category}
    Given~a graph $\graph=\tup{\vertices,\arcs,\source,\target}$, the \emph{free category on} $\graph$, denoted~$\pathcat{\graph}$, is defined analogously to $\pathscat{\graph}$ but with the modification that $\HomSet{\CatSymbol{C}(\graph)}{\vertexa}{\vertexb}$ is equal to \emph{all} paths from~$\vertexa$ to~$\vertexb$.
    Identity morphisms are the trivial paths.
\end{ctdefinition}

For instance, consider the graph
%
\equationsag{graph_cat_1}{eq:graph_cat_1}
%
The free category on this graph is given by
%
\equationsag{graph_cat_2}{eq:graph_cat_2}
%
and has 8 morphisms (each vertex/object has \SY{identity morphisms},~$\arca,\arcb,\arcc$ give rise to three morphisms, and the composition of $\arca,\arcc$ gives rise to a morphism, omitted from the drawing).

Does \cref{def:free-category} define a category?
We can check it ourselves.
For it to define a category, unitality and associativity need to be satisfied.
Given our definition of path, this is easy.
The concatenation of paths is just list concatenation (which we already proved to be associative).
Furthermore, a trivial path can be expressed via an empty list, which acts as an identity when composed to any other path.
%Consider a graph~$\graph=\tup{\vertices,\arcs,\source,\target}$ and denote a path (\cref{def:path}) by~$p=\tup{\arcan{1},\ldots, \arcan{n}}$, where~$n\setin \natnumbers$,~$\source(p)\definedas\source(\arcan{1})=\vertexa$ is the source of the path,~$\target(\arcan{i})=\source(\arcan{i+1})$ for all~$i\setin \makeset{1,\ldots,n-1}$, and~$\target(p)=\target(\arcan{n})=\vertexb$ which is the target of the path.
%Note that in case the path has 0 length ($n=0$), the target of the path is simply~$\vertexa$.
%Now consider two paths~$p=\tup{\arcan{1},\ldots,\arcan{n}}$ and~$q=\tup{\arcbn{1},\ldots,\arcbn{m}}$, with~$m,n\setin \natnumbers$.
%Paths~$p$ and~$q$ can be concatenated if~$\target(p)=\source(q)$.
%In this case, the concatenation reads
%
%\begin{equation}
%    p\mthen q\definedas \tup{\arcan{1},\ldots, \arcan{n},\arcbn{1},\ldots,\arcbn{m}}.
%\end{equation}
%
%\todomistake{\alphubel: Wrong - we defined a path as a list of edges.
%    A trivial path is only an empty list.
%    Concatenation is just list concatenation.
%    We already proved it's \SY{associative}.
%    Very convoluted.
%}
%We now check unitality.
%A trivial path is a path~$\catidat\vertexa$, consisting of a vertex and a trivial arc, starting and ending at~$\vertexa$.
%Such a path can be concatenated with another path~$q=\tup{\arcbn{1},\ldots,\arcbn{m}}$ in case~$\vertexa=\source(q)$.
%In this case, we obtain the path~$\catidat\vertexa \mthen q=q$.
%Similarly,~$p$ can be concatenated with~$\catidat\vertexa$ if~$\target(p)=\vertexa$, obtaining~$p\mthen \catidat\vertexa=p$, and proving unitality.
%For associativity, consider~$p$ and~$q$ as above, and~$r=\tup{\arccn{1},\ldots,\arccn{l}}$,~$l\setin \natnumbers$.
%Assuming that concatenations~$p\mthen q$ and~$q\mthen r$ are feasible, we have
%\begin{equation}
%    \begin{aligned}
%        (p\mthen q)
%        \mthen r & =(\tup{\arcan{1},\ldots, \arcan{n}}\mthen \tup{\arcbn{1},\ldots,\arcbn{m}})\mthen \tup{\arccn{1},\ldots,\arccn{l}} \\
%                 & =\tup{\arcan{1},\ldots, \arcan{n},\arcbn{1},\ldots,\arcbn{m}}\mthen \tup{\arccn{1},\ldots,\arccn{l}} \\
%                 & = \tup{\arcan{1},\ldots, \arcan{n},\arcbn{1},\ldots,\arcbn{m}, \arccn{1},\ldots,\arccn{l}} \\
%                 & =\tup{\arcan{1},\ldots, \arcan{n}} \mthen (\tup{\arcbn{1},\ldots,\arcbn{m}}\mthen \tup{\arccn{1},\ldots,\arccn{l}}) \\
%                 & =p\mthen (q\mthen r),
%    \end{aligned}
%\end{equation}
%proving associativity.

%\text{\ie } to check that the composition of paths is again a path, and that the \SY{associative law} and the law for identity morphisms hold.
\vfill\pagebreak

\begin{widepar}
    \begin{gradedexercise}[\exname{HowManyMorphisms}]
        \label{ex:HowManyMorphisms}
        Consider the following five graphs.
        For each graph~$\graph$, how many morphisms in total are there in the associated category~$\pathscat{\graph}$?
        %
        \equationsag{20_dpcatfig_example_graphs}{eq:20_dpcatfig_example_graphs}
    \end{gradedexercise}
\end{widepar}
\solutionof{HowManyMorphisms}
