% !TEX root = chapter-standalone.tex

\section[Semicategories]{Semicategories}
\label{sec:formal-def-semicat}

\todotextjira{29}{@J: Add johnny's lecture}

\linkvideo{spring2021-semicats-cats:intro_semi_cat} % Introduction for Semicategories and Categories

We will now begin to introduce the concept of a semicategory by listing key aspects of the electrical chords example in the previous section which each correspond to a part of the formal definition of semicategory.

\subsection{Objects, morphisms, composition}

Firstly, in the chords example, he have a set of types of electrical interfaces.
In a semicategory, the things that play the role of interfaces are called \emph{objects}.
(In our book, we often denote generic objects by letters~$\Obja$,~$\Objb$,~$\Objc$, etc.)

Secondly, in the electrical cords example, the chords themselves connect two interfaces and have a directionality: one end has a socket, the other end has a plug.
In a semicategory, the things that play the role of chords are called \emph{morphisms}.
(This is a word we've already gotten to know, and the connection here is not accidental.) The object that denotes the type of the ``socket end'' of the morphism is called the \emph{domain} or \emph{source} of the morphism.
The object that denotes the ``plug end'' of a morphism is called its \emph{codomain} or \emph{target}.
So morphisms are directed: the go from their source to their target.

To visualize morphisms, we often draw arrows.
For example, if~$\Obja$ and~$\Objb$ are objects in some semicategory, then we draw a morphism from~$\Obja$ to~$\Objb$ (let's call it~$\mora$) like this:
\begin{equation}
    \mora\colon \Obja \mto \Objb.
\end{equation}
So here~$\Obja$ is the source of~$\mora$, and~$\Objb$ is its target.

Thirdly, a key feature of the chords example is that we can compose two chords and the result is again a chord.
However, for this to work, the plug-end of the first chord being composed must match the socket-end of the other.
Similarly, in any semicategory we specify a way to compose two morphisms, provided that the target object of the first morphism matches the source object of the second morphism.
(When this is the case, the morphisms are said to be \emph{composable}.)

In other words, if~$\mora\colon \Obja \mto \Objb$ and~$\morb\colon \Objb \mto \Objc$ are morphisms in some semicategory (note: they are composable), then the semicategory has an operation for composing them, and the result is another morphism.
Our notation for composition of~$\mora$ and~$\morb$ is~$\mora \then \morb$, read ``$\mora$ then~$\morb$''.
(Once again, in contrast to the more traditional notatation~$\morb \circ \mora$.) Thus we have
\begin{equation}
    \prfcomma{
        \mora \colon \Obja \mto \Objb
    }{
        \morb\colon \Objb \mto \Objc
    }{
        (\mora \mthen \morb )\colon \Obja \mto \Objc
    }
\end{equation}
which is analogous to \cref{eq:cords-generic}  in the chords example.

\subsection{Associativity}

\todojira{614}{@J: the section here needs some attention/editing/rewriting/addtiions}

Additionally, we require an \emph{associativity law} to hold (this is analogous to the fact that if we look at three electrical chords connected together, we cannot tell if the first two were connected together first and then the result was connected to the third, or if the connecting together happened the other way around).

\subsection{The definition}

Categories will be more of a protagonist in this book than semicategories; however we introduce semicategories first, because they are more rudimentary.
The step from semicategories to categories will then be very similar to the step from semigroups to monoids.

Have a read through the formal definition of a semicategory, then we we'll illustrate it further with examples.
\linkvideo{spring2021-semicats-cats:semicats} % Semicategories
%\begin{comment}
%    \begin{ctdefinition}[Semicategory]
%        \label{def:semicategory}
%        A \emph{\iindex{semicategory}}~\CatC is:
%        \begin{body}
%            \constit
%            \begin{enumerate}
%                \item Objects: a collection\footnotemark~$\ObC$, whose elements are called \emph{objects}.
%                \item Morphisms: for every pair of objects~$\Obja,\Objb\in \ObC$, there is a set~$\HomSet{\CatC}{\Obja}{\Objb}$, elements of which are called
%                \emph{morphisms} from~$\Obja$ to~$\Objb$. The set is called the ``hom-set from~$\Obja$ to~$\Objb$''.
%                \item Composition operations: given any morphism~${\mora} \in  \HomSet{\CatC}{\Obja}{\Objb}$ and any morphism~${\morb} \in \HomSet{\CatC}{\Objb}{\Objc}$, there exists a morphism~$\mora \mthen \morb\in \HomSet{\CatC}{\Obja}{\Objc}$ which is the \emph{composition of~$\mora$ and~$\morb$}.
%            \end{enumerate}
%            \condit
%            \begin{enumerate}
%                \item Associativity: for any morphisms~$\mora\in \HomSet\CatC\Obja\Objb$,~$\morb\in \HomSet\CatC\Objb\Objc$, and~$\morc\in \HomSet\CatC\Objc\Objd$,
%                \begin{equation}
%                (\mora\mthen \morb)
%                    \mthen \morc= \mora \mthen (\morb \mthen \morc).
%                \end{equation}
%            \end{enumerate}
%        \end{body}
%    \end{ctdefinition}
%\end{comment}
\begin{ctdefinition}[Semicategory]
    \label{def:semicategory-compact}
    A \emph{\iindex{semicategory}}~\CatC is specified by:
    \begin{body}
        \constit
        \begin{enumerate}
            \item Objects: A collection\footnotemark~$\ObC$ whose elements are called \emph{objects}.
            \item Morphisms: For every pair of objects~$\Obja,\Objb$ in $\ObC$, there is a set called ``hom-set'' and indicated as~$\HomSet{\CatC}{\Obja}{\Objb}$, elements of which are called \emph{morphisms}.
                  We write
                  \begin{equation}
                      \mora \colon \Obja \mtoin\CatC \Objb
                  \end{equation}
                  to indicate
                  \begin{equation}
                      \mora \in \HomSet{\CatC}{\Obja}{\Objb}.
                  \end{equation}
            \item Composition operations: For every three objects~$\Obja,\Objb,\Objc$ in~$\ObC$ there is a composition map
                  \begin{equation}
                      \mthenof{\Obja,\Objb,\Objc}: \HomSet{\CatC}{\Obja}{\Objb} \times \HomSet{\CatC}{\Objb}{\Objc}\to \HomSet{\CatC}{\Obja}{\Objc}.
                  \end{equation}
                  We usually just write $\mthen$ instead of $\mthenof{\Obja,\Objb,\Objc}$:
                  \begin{equation}
                      \prfperiod{
                          \mora \colon \Obja \mto \Objb
                      }{
                          \morb\colon \Objb \mto \Objc
                      }{
                          (\mora \mthen \morb )\colon \Obja \mto \Objc
                      }
                  \end{equation}
                  The morphism~$\mora \mthen \morb$ is called the \emph{composition} of~$\mora$ and~$\morb$.
                  %Given any morphism~${\mora} \in  \HomSet{\CatC}{\Obja}{\Objb}$ and any morphism~${\morb} \in \HomSet{\CatC}{\Objb}{\Objc}$, there exists a morphism~$\mora \mthen \morb\in \HomSet{\CatC}{\Obja}{\Objc}$ which is the \emph{composition of~$\mora$ and~$\morb$}.
        \end{enumerate}
        \condit
        \begin{enumerate}
            \item Associativity: it holds that
                  % for any morphisms~$\mora\in \HomSet\CatC\Obja\Objb$,~$\morb\in \HomSet\CatC\Objb\Objc$, and~$\morc\in \HomSet\CatC\Objc\Objd$,
                  \begin{equation}
                      \prfperiod{
                          \mora\colon \Obja \mto \Objb
                      }{
                          \morb\colon \Objb \mto \Objc
                      }{
                          \morc\colon \Objc \mto \Objd
                      }{
                          (\mora\mthen \morb)
                          \mthen \morc= \mora \mthen (\morb \mthen \morc)
                      }
                  \end{equation}
        \end{enumerate}
    \end{body}
\end{ctdefinition}

\begin{remark}
    We denote composition of morphisms using the symbol ``$\mthen$'' (pronounced ``then''), as already introduced for functions in \cref{sec:functions}.
    %    Thus, given~$\mora \colon \Obja\mto \Objb$ and~$\morb \colon \Objb \mto \Objc$, we denote their composite by~$(\mora\mthen \morb)\colon \Obja\mto \Objc$, pronounced ``$\mora$ then~$\morb$''.
    This is in contrast to the more common notation for composition, namely~$\morb\after \mora$, or simply~$\morb \mora$, which reads as ``$\morb$ after~$\mora$''.
    As usual,~$\mora^2$ denotes~$\mora \mthen \mora$, and~$\mora^3$ denotes~$\mora \mthen \mora \mthen \mora$, and so on.
\end{remark}

\todojira{615}{@J: the paragraph below is redundant, so I think I will delete.}

Given a morphism~$\mora \colon \Obja \mtoin\CatC \Objb$ in a semicategory~$\CatC$, the object~$\Obja$ is called its \emph{source} (or \emph{domain}) and the object~$\Objb$ is called its \emph{target} (or \emph{codomain}).
Note that according to \cref{def:semicategory-compact}, we can only compose two morphisms if the target of one of the morphisms matches the source of the other.
When this is the case, we say that the morphisms are \emph{composable}.

%\begin{marginfigure}
%    \begin{center}
%        \includesag{points_and_arrow}
%    \end{center}
%    \caption{}
%    \label{fig:points-and-arrow-diagram}
%\end{marginfigure}

\begin{marginfigure}
    \begin{center}
        \includesag{points_and_two_arrows}
    \end{center}
    \caption{}
    \label{fig:points-and-two-arrows-diagram}
\end{marginfigure}

\begin{marginfigure}
    \begin{center}
        \includesag{points_and_composed_arrows}
    \end{center}
    \caption{}
    \label{fig:points-and-composed-arrows-diagram}
\end{marginfigure}

We will often visualize objects and morphisms in a (semi)category using diagrams, involving symbols for the objects (either their names or a generic dot) and arrows to indicate morphisms.
For instance, if~$\Obja$,~$\Objb$,~$\Objc$ are objects in a semicategory~$\CatC$, and~$\mora\colon \Obja \mto \Objb$ and~$\morb\colon \Objb \mto \Objc$ are morphisms, we draw this as in \cref{fig:points-and-two-arrows-diagram}, for example.
The composition~$\mora \mthen \morb$ of~$\mora$ with~$\morb$ corresponds to yet another arrow, as in \cref{fig:points-and-composed-arrows-diagram}

\begin{example}
    \label{exa:semicat-plant-states}
    This example is an extension of \cref{exa:plant-trafo-semigroup}.
    We will describe a semicategory~$\CatC$ with two objects,~$\ObC = \{ \setA, \setB \}$.
    Each of the objects is a set which describes possible states of a plant.
    Let
    \begin{equation}
        \setA = \{ \sprout, \yng, \mature, \old, \dead \},
    \end{equation}
    and
    \begin{equation}
        \setB = \{ \alive, \dead \}.
    \end{equation}
    Also, let~$\mapa \colon \setA \to \setA$ be the function with
    \begin{align*}
        \mapa(\sprout) & =  \yng, \\
        \mapa(\yng)    & =  \mature, \\
        \mapa(\mature) & =  \old, \\
        \mapa( \old)   & = \dead, \\
        \mapa(\dead)   & = \dead;
    \end{align*}
    let~$\morb \colon \setB \to \setB$ be the function with
    \begin{align*}
        \mapb(\alive) & =  \dead, \\
        \mapb(\dead)  & =  \dead;
    \end{align*}
    and let~$\morc \colon \setA \to \setB$ be the function with
    \begin{align*}
        \mapc(\sprout) & = \alive, \\
        \mapc(\yng)    & =  \alive, \\
        \mapc(\mature) & = \alive , \\
        \mapc( \old)   & = \alive, \\
        \mapc (\dead)  & = \dead.
    \end{align*}
    For our sets of morphisms, we let
    \begin{align*}
        \HomSet{\CatC}{\setA}{\setA} & = \{ \mora, \mora^2, \mora^3, \mora^4 \}, \\
        \HomSet{\CatC}{\setB}{\setB} & =  \{ \morb, \morb^2 \}, \\
        \HomSet{\CatC}{\setA}{\setB} & =  \{ \morc \}, \\
        \HomSet{\CatC}{\setB}{\setA} & = \emptyset;
    \end{align*}
    and for the composition operations, we define these to be the usual composition of functions, which we know obeys the associative law.
    Thus we have a semicategory.
    \todojira{269}{Finish writing the above example involving plant states by adding a nice figure to illustrate it}
\end{example}

\todotextjira{394}{@J: Add remark about relationship between semicategories and semigroups}

\todotextjira{378}{@Gioele: add the example of discrete-time linear time-invariant systems as an example of a semicategory}

%Now let's view electrical chords as forming a semicategory. For simplicity (and for the ``C''), let's call our semicategory-to-be by the name $\CatC$, same as in the definition above. To define $\CatC$, let the collection $\ObC$ of objects to be the collection of all electrical interface types: one object for each type. Next, given any two types $\Obja$ and $\Objb$ (\text{e.g.}, $\Obja = \TypeIrish$ and $\Objb = \TypeSwiss$), let's define the set of morphisms $\HomSet{\CatC}{\Obja}{\Objb}$ to be the set of all electrical chords that have the type $\Obja$ as their incoming interface and type $\Objb$ as their outgoing interface.

%\let\oldmora\mora
%\renewcommand{\mora}{{\colTransmuter \oldmora}}
%\let\oldmorb\morb
%\renewcommand{\morb}{{\colTransmuter \oldmorb}}
%\let\oldmorc\morc
%\renewcommand{\morc}{{\colTransmuter \oldmorc}}
%\let\oldObja\Obja
%\renewcommand{\Obja}{{\colTransmuted \oldObja}}
%\let\oldObjb\Objb
%\renewcommand{\Objb}{{\colTransmuted \oldObjb}}
%\let\oldObjc\Objc
%\renewcommand{\Objc}{{\colTransmuted \oldObjc}}
%\let\oldObjd\Objd
%\renewcommand{\Objd}{{\colTransmuted \oldObjd}}
%
%\let\oldOb\Ob
%\renewcommand{\Ob}{{\colTransmuted \oldOb}}
%
%\let\oldHom\Hom
%\renewcommand{\Hom}{{\colTransmuter \oldHom}}

\section{Categories}
\linkvideo{spring2021-semicats-cats:categories} % Categories

Now we come to one of the central protagonists of this book: the concept of a category.
Categories are like semicategories, but with one more ingredient added: identity morphisms.
A good analogy is that categories are to semicategories as monoids are to semigroups.
A monoid is a semigroup that additionally has an identity element, and similarly a category is a semicategory that additionally has identity morphisms.
So what are identity morphisms?

One might say loosely that identity morphisms are morphisms that ``do nothing'': they do not have any effect when we compose with them.
This is analogous to how the the identity element of a monoid ``does nothing'' when we multiply it with other elements of the monoid.

\subsection{Sets and functions}

Let's look at identity morphisms in a familiar setting: the composition of functions between sets.

\subsection{Identity morphisms}

\begin{ctdefinition}[Identity morphisms]
    \label{def:identity-morphism}
    An \emph{identity morphism}, or just \emph{identity}, for an object $\Obja$
    is a morphism
    \begin{equation}
        \catid_{\Obja}\colon \Obja \mto \Obja
    \end{equation}
    that acts neutrally with respect to composition with any morphism in the category with which it is composable:
    \begin{equation}
        \prfperiod{
            \mora \colon \Obja \mto \Objb
        }{
            \catid_\Obja \mthen \mora= \mora = \mora \mthen \catid_\Objb
        }
        % \prftree{\catid_{\Obja}\colon \Obja \to \Obja}{\mora \colon \Obja \to \Objb}{\catid_{\Objb}\colon \Objb \to \Objb}{\catid_\Obja \mthen \mora= \mora = \mora \mthen \catid_\Objb}
    \end{equation}
\end{ctdefinition}

Note that if an identity morphism $\catid_{\Obja}$ for an object $\Obja$ exists, then it is unique.
To see this, observe that $\HomSet{\CatC}{\Obja}{\Obja}$ is a semigroup, and $\catid_{\Obja}$ is a neutral element for this semigroup, making $\HomSet{\CatC}{\Obja}{\Obja}$ a monoid.
We have seen earlier that neutral elements for semigroups must be unique.

\subsection{Categories}

First off, the definition:

\begin{ctdefinition}[Category]
    \label{def:categorymain}
    A \emph{\iindex{category}}~\CatC is a semicategory in which there is an identity for every object.
\end{ctdefinition}
%  an additional constituent and an additional rule:
%     \begin{body}
%         \constit
%         \begin{enumerate}
%             \item Identity morphisms: for each object~$\Obja$, there is a morphism
%             \begin{equation}
%                 \catid_{\Obja}\colon \Obja \mto \Obja
%             \end{equation}
%             called \emph{the identity morphism of~$\Obja$}.
%         \end{enumerate}
%         \condit
%         \begin{enumerate}
%             \item Unitality: It holds that composing a morphism with a compatible identity leaves the morphism unchanged:
%             \begin{equation}
%                 \prfperiod{
%                     \mora \colon \Obja \mto \Objb
%                     }{
%                         \catid_\Obja \mthen \mora= \mora = \mora \mthen \catid_\Objb
%                         }
%                 % \prftree{\catid_{\Obja}\colon \Obja \to \Obja}{\mora \colon \Obja \to \Objb}{\catid_{\Objb}\colon \Objb \to \Objb}{\catid_\Obja \mthen \mora= \mora = \mora \mthen \catid_\Objb}
%             \end{equation}
%         \end{enumerate}
%     \end{body}
% \end{ctdefinition}

% \begin{ctdefinition}[Category]
%   \label{def:categorymain}
%   A \emph{\iindex{category}}~\CatC is:
% \begin{quote}
%     \constit
%   \begin{enumerate}
%     \item Objects: a collection\footnotemark~$\ObC$, whose elements are called \emph{objects}.
%     \item Morphisms: For every pair of objects~$\Obja,\Objb$ in $\ObC$, there is a set~$\HomSet{\CatC}{\Obja}{\Objb}$, elements of which are called \emph{morphisms}. We write
%     \begin{equation}
%     \mora \colon \Obja \to_\CatC \Objb
%     \end{equation}
%     to indicate
%     \begin{equation}
%       \mora \in \HomSet{\CatC}{\Obja}{\Objb}.
%       \end{equation}
%     \item Identity morphisms: for each object~$\Obja$, there is
%     an element~$\catid_{\Obja} \in \HomSet{\CatC}{\Obja}{\Obja}$ which is called \emph{the identity
%     morphism of~$\Obja$}.
%     \item Composition operations: For every three objects~$\Obja,\Objb,\Objc$ in $\ObC$ there is a composition map
%     \begin{equation}
%         \mthenof{\Obja,\Objb,\Objc}: \HomSet{\CatC}{\Obja}{\Objb} \times \HomSet{\CatC}{\Objb}{\Objc}\to \HomSet{\CatC}{\Obja}{\Objc}
%     \end{equation}
%     It holds that
%     \begin{equation}
%         \prftree{\mora \colon \Obja \to \Objb}{\morb\colon \Objb \to \Objc}{\mora \mthen \morb \colon \Obja \to \Objc}
%     \end{equation}
%     The morphism~$\mora \mthen \morb$ is called the \emph{composition} of~$\mora$ and~$\morb$.
%     %Composition operations: given any morphism~${\mora} \in  \HomSet{\CatC}{\Obja}{\Objb}$ and any morphism~${\morb} \in \HomSet{\CatC}{\Objb}{\Objc}$, there exists a morphism~$\mora \mthen \morb\in \HomSet{\CatC}{\Obja}{\Objc}$ which is the \emph{composition of~$\mora$ and~$\morb$}.
%   \end{enumerate}
%  \condit
%   \begin{enumerate}
%     \item Unitality: It holds that:
%     \begin{equation}
%         \prftree{\catid_{\Obja}\colon \Obja \to \Obja}{\mora \colon \Obja \to \Objb}{\catid_{\Objb}\colon \Objb \to \Objb}{\catid_\Obja \mthen \mora= \mora = \mora \mthen \catid_\Objb}
%     \end{equation}
%     \item Associativity:  it holds that
%     % for any morphisms~$\mora\in \HomSet\CatC\Obja\Objb$,~$\morb\in \HomSet\CatC\Objb\Objc$, and~$\morc\in \HomSet\CatC\Objc\Objd$,
%     \begin{equation}
%       \prftree{\mora\colon \Obja \to \Objb}{\morb\colon \Objb \to \Objc}{\morc\colon \Objc \to \Objd}
%       {
%     (\mora\mthen \morb)
%       \mthen \morc= \mora \mthen (\morb \mthen \morc)
%       }
%     \end{equation}
%   \end{enumerate}
%   \end{quote}
% \end{ctdefinition}

\footnotetext{A ``collection'' is something which may be thought of as a set, but may be ``too large" to technically be a set in the formal sense.
    This distinction is necessary in order to avoid such issues as Russel's paradox.
}
\todotextjira{476}{@J: Footnote content was actually mentioned in the initial parts.}
\vfill
\begin{gradedexercise}[\exname{CategorySemigroups}]
    There is a category where the objects are semigroups and the morphisms are semigroup homomorphisms.
    Spell out explicitely what this category is: check in detail each of the points of \cref{def:categorymain}.
\end{gradedexercise}

\solutionof{CategorySemigroups}

\showslides{
    \begin{forslides}

        \begin{equation}
            \mora \quad \morb \quad \morc
        \end{equation}

        \begin{equation}
            X = \{ \sprout, \yng, \mature, \old, \dead \}
        \end{equation}

        \begin{equation}
            T \colon X \sto X
        \end{equation}

        \begin{align*}
            T(\text{sprout}) & =  \yng \\
            T(\yng)          & =  \mature \\
            T(\mature)       & =  \old \\
            T( \old)         & = \dead \\
            T (\dead)        & = \dead
        \end{align*}

        \begin{equation}
            \sgrpA = \{ T^n \mid n \in \natnumbers \}
        \end{equation}

        \begin{equation}
            T^n = T^4 \quad \forall \ n \geq 4
        \end{equation}

        \begin{equation}
            \sgrpA = \{T, T^2, T^3, T^4 \}
        \end{equation}

        \begin{equation}
            \setA = \{ \alphabeta, \alphabetb \}
        \end{equation}

        \begin{equation}
            \sgrpA = \{ \text{set of non-empty strings of elements of }\setA\}
        \end{equation}

        \begin{equation}
            Y = \{ \text{alive}, \dead \}
        \end{equation}

        \begin{equation}
            T' : Y \sto Y
        \end{equation}

        \begin{equation}
            \sgrpA' = \{T' \}
        \end{equation}

        \begin{align*}
            T'(\text{alive}) & =  \dead \\
            T' (\dead)       & = \dead
        \end{align*}

        \begin{equation}
            R: X \sto Y
        \end{equation}

        \begin{align*}
            R(\text{sprout}) & = \text{alive} \\
            R(\yng)          & =  \text{alive} \\
            R(\mature)       & = \text{alive} \\
            R( \old)         & = \text{alive} \\
            R (\dead)        & = \dead
        \end{align*}

        \begin{equation}
            \monoidA = \{ \id, T, T^2, T^3, T^4 \}
        \end{equation}

        \begin{equation}
            \monoidA' = \{ \id, T' \}
        \end{equation}

        \begin{equation*}
            \worda \text{ and } \wordb
        \end{equation*}

        \begin{equation*}
            \label{eq:string-semigroup-wordab-2}
            \worda \mtimes  \wordb =  \worda \wordb
        \end{equation*}

        \begin{equation}
            \tup{ \monoidA, \mtimes, \idmon}
        \end{equation}

        \begin{equation*}
            \label{eq:string-semigroup-wordab-concat-2}
            \worda \wordb
        \end{equation*}

        \begin{equation}
            \label{eq:00}
            \Objd
        \end{equation}

        \begin{equation}
            \label{eq:01}
            \Obja
        \end{equation}

        \begin{equation}
            \label{eq:02}
            \Objb
        \end{equation}

        \begin{equation}
            \label{eq:03}
            \Objc
        \end{equation}

        \begin{equation}
            \label{eq:04}
            \mora
        \end{equation}

        \begin{equation}
            \label{eq:05}
            \morb
        \end{equation}

        \begin{equation}
            \label{eq:06}
            \morc
        \end{equation}

        \begin{equation}
            \label{eq:07}
            \mord
        \end{equation}

        \begin{equation}
            \label{eq:08}
            \mora \mthen \morb
        \end{equation}

        \begin{equation}
            \label{eq:09}
            \morb \mthen \morc
        \end{equation}

        \begin{equation}
            \label{eq:10}
            \morc \mthen \mord
        \end{equation}

        \begin{equation}
            \label{eq:11}
            (\mora \mthen \morb) \mthen \morc
        \end{equation}

        \begin{equation}
            \label{eq:12}
            \mora \mthen  (\morb \mthen \morc )
        \end{equation}

        \begin{equation}
            \label{eq:13}
            \mora \mthen  ((\morc \mthen \morb ) \mthen (( \mord \mthen \morb ) \mthen \mora ))
        \end{equation}

        \begin{equation}
            \label{eq:14}
            \mora \mthen  \morc \mthen \morb  \mthen \mord \mthen \morb  \mthen \mora
        \end{equation}

        \begin{ctdefinition}
            \label{def:category-var}
            A \emph{\iindex{category}}~\CatC is specified by:
            \begin{body}
                \constit
                \begin{enumerate}
                    \item Objects: A set $\ObC$ whose elements are called \emph{objects}.
                    \item Morphisms: For every pair of objects~$\Obja,\Objb$ in $\ObC$, there is a set~$\HomSet{\CatC}{\Obja}{\Objb}$, elements of which are called \emph{morphisms} from $\Obja$ to $\Objb$.
                          A morphism $\mora \in \HomSet{\CatC}{\Obja}{\Objb}$ is often indicated by writing
                          \begin{equation*}
                              \mora \colon \Obja \mto \Objb.
                          \end{equation*}
                    \item Identity morphisms: for each object~$\Obja$, there is a morphism~$\catid_{\Obja}\colon \Obja \mto \Obja$  called \emph{the identity morphism of~$\Obja$}.
                    \item Composition operations: For every three objects~$\Obja,\Objb,\Objc$ in $\ObC$ there is a composition function
                          \begin{equation*}
                              \mthenof{\Obja,\Objb,\Objc}: \HomSet{\CatC}{\Obja}{\Objb} \times \HomSet{\CatC}{\Objb}{\Objc}\to \HomSet{\CatC}{\Obja}{\Objc}
                          \end{equation*}
                          We denote the \emph{composition} of composable morphisms~$\mora$ and~$\morb$ by~$\mora \mthen \morb$.
                          (Traditionally, the typical notation would be $\morb \circ \mora$).

                \end{enumerate}
                \condit
                \begin{enumerate}
                    \item Associativity: for all composable morphisms $\mora, \morb, \morc$,
                          \begin{equation}
                              (\mora\mthen \morb)
                              \mthen \morc= \mora \mthen (\morb \mthen \morc).
                          \end{equation}
                    \item Unitality: for every morphism $\mora \colon \Obja \mto \Objb$,
                          \begin{equation}
                              \catid_{\Obja} \then \mora = \mora \qqand \mora \then \catid_{\Objb} = \mora.
                          \end{equation}
                \end{enumerate}
            \end{body}
        \end{ctdefinition}

        \begin{equation}
            \label{eq:15}
            \CatC
        \end{equation}

        \begin{equation}
            \label{eq:16}
            \ObC
        \end{equation}

        \begin{equation}
            \label{eq:17}
            \HomSet{\CatC}{\Obja}{\Objb}
        \end{equation}

        \begin{equation}
            \label{eq:18}
            \catid_{\Obja}\colon \Obja \mto \Obja
        \end{equation}

        \begin{equation}
            \label{eq:19}
            \mthenof{\Obja,\Objb,\Objc}: \HomSet{\CatC}{\Obja}{\Objb} \times \HomSet{\CatC}{\Objb}{\Objc}\to \HomSet{\CatC}{\Obja}{\Objc}
        \end{equation}

        \begin{equation}
            \label{eq:20}
            (\mora\mthen \morb)
            \mthen \morc= \mora \mthen (\morb \mthen \morc)
        \end{equation}

        \begin{equation}
            \label{eq:21}
            \catid_{\Obja} \then \mora = \mora
        \end{equation}

        \begin{equation}
            \label{eq:22}
            \mora \then \catid_{\Objb} = \mora
        \end{equation}

        \begin{equation}
            \label{eq:23}
            \CatC = \Set
        \end{equation}

        \begin{equation}
            \label{eq:24}
            \catid_{\Obja}
        \end{equation}

        \begin{equation}
            \label{eq:25}
            \catid_{\Objb}
        \end{equation}

        \begin{equation}
            \label{eq:26}
            \catid_{\Objc}
        \end{equation}

        \begin{equation}
            \label{eq:27}
            \styleobj{\posReals}
        \end{equation}

        \begin{equation}
            \label{eq:28}
            \stylemorph{f(x) = x^2}
        \end{equation}

        \begin{equation}
            \label{eq:29}
            \stylemorph{g(y) = \frac{1}{y}}
        \end{equation}

        \begin{equation}
            \label{eq:30}
            \stylemorph{h(x) = \frac{1}{x^2}}
        \end{equation}

        \begin{equation}
            \label{eq:31}
            \mora \then \morb = \catid_\Obja
        \end{equation}

        \begin{equation}
            \label{eq:32}
            \morb \then \mora = \catid_\Objb
        \end{equation}

        \begin{equation}
            \label{eq:33}
            \styleobj{U}
        \end{equation}

        \begin{equation}
            \label{eq:34}
            \styleobj{P}
        \end{equation}

        \begin{equation}
            \label{eq:35}
            \styleobj{X \times Y}
        \end{equation}

        \begin{equation}
            \label{eq:36}
            \styleobj{X \setintersection Y}
        \end{equation}

        \begin{equation}
            \label{eq:37}
            \styleobj{X \setunion Y}
        \end{equation}

        \begin{equation}
            \label{eq:38}
            l
        \end{equation}

        \begin{equation}
            \label{eq:39}
            m
        \end{equation}

        \begin{equation}
            \label{eq:40}
            n
        \end{equation}

        \begin{equation}
            \label{eq:41}
            l, m, n \in \natnumbers
        \end{equation}

        \begin{equation}
            \label{eq:42}
            \mtimescat
        \end{equation}

        \begin{equation}
            \label{eq:43}
            \morb \circ \mora
        \end{equation}

        \begin{equation}
            \label{eq:44}
            0 + n = n = n + 0   \quad \quad \forall \ n \in \natnumbers
        \end{equation}

        \begin{equation}
            \label{eq:45}
            (l + m) + n = l + (m + n) \quad \quad  \forall \ l, m, n \in \natnumbers
        \end{equation}

        \begin{equation}
            \label{eq:46}
            \mora \then \morb = \morc \then \mord
        \end{equation}

        \begin{equation}
            \label{eq:47}
            \styleobj{A}
        \end{equation}

        \begin{equation}
            \label{eq:48}
            \styleobj{B}
        \end{equation}

        \begin{equation}
            \label{eq:49}
            \styleobj{C}
        \end{equation}

        \begin{equation}
            \label{eq:50}
            \styleobj{D}
        \end{equation}

        \begin{equation}
            \label{eq:51}
            \stylemorph{j}
        \end{equation}

        \begin{equation}
            \label{eq:52}
            \stylemorph{k}
        \end{equation}

    \end{forslides}
}

% \linkvideo{spring2021-semicats-cats:summary} % Summary
