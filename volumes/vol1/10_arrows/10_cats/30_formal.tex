% !TEX root = chapter-standalone.tex


\section{Formal definition of (semi)category}\label{sec:formal-def-semicat}

\todojira{29}{Add johnny's lecture}

\linkvideo{spring2021-semicats-cats:intro_semi_cat} % Introduction for Semicategories and Categories

We are now ready to introduce the mathematical definitions of semicategories and categories. A semicategory is something that captures, in a general form, the pattern \cref{eq:cords-generic} of composing electrical chords with different interfaces. Namely, a semicategory consists of 
\begin{itemize}
\item a collection of \emph{objects} (analogous to set of types of electrical interfaces)
\item sets of \emph{morphisms} (which are analogous to electric chords between types of interfaces)
\item a way to \emph{compose} morphisms (analogous to composing electrical chords). 
\end{itemize}
Additionally, we require an \emph{associativity law} to hold (this is analogous to the fact that if we look at three electrical chords connected together, we cannot tell if the first two were connected together first and then the result was connected to the third, or if the connecting together happened the other way around). 

Categories will be more of a protagonist in this book than semicategories; however we introduce semicategories first, because they are more rudimentary. The step from semicategories to categories will then be very similar to the step from semigroups to monoids. 

Have a read through the formal definition of a semicategory, then we we'll work on understanding it with the help of some examples.
\linkvideo{spring2021-semicats-cats:semicats} % Semicategories
%\begin{comment}
%    \begin{ctdefinition}[Semicategory]
%        \label{def:semicategory}
%        A \emph{\iindex{semicategory}}~\CatC is:
%        \begin{body}
%            \constit
%            \begin{compactenum}
%                \item Objects: a collection\footnotemark~$\ObC$, whose elements are called \emph{objects}.
%                \item Morphisms: for every pair of objects~$\Obja,\Objb\in \ObC$, there is a set~$\HomSet{\CatC}{\Obja}{\Objb}$, elements of which are called
%                \emph{morphisms} from~$\Obja$ to~$\Objb$. The set is called the ``hom-set from~$\Obja$ to~$\Objb$''.
%                \item Composition operations: given any morphism~${\mora} \in  \HomSet{\CatC}{\Obja}{\Objb}$ and any morphism~${\morb} \in \HomSet{\CatC}{\Objb}{\Objc}$, there exists a morphism~$\mora \mthen \morb\in \HomSet{\CatC}{\Obja}{\Objc}$ which is the \emph{composition of~$\mora$ and~$\morb$}.
%            \end{compactenum}
%            \condit
%            \begin{compactenum}
%                \item Associativity: for any morphisms~$\mora\in \HomSet\CatC\Obja\Objb$,~$\morb\in \HomSet\CatC\Objb\Objc$, and~$\morc\in \HomSet\CatC\Objc\Objd$,
%                \begin{equation}
%                (\mora\mthen \morb)
%                    \mthen \morc= \mora \mthen (\morb \mthen \morc).
%                \end{equation}
%            \end{compactenum}
%        \end{body}
%    \end{ctdefinition}
%\end{comment}
\begin{ctdefinition}[Semicategory]
    \label{def:semicategory-compact}
    A \emph{\iindex{semicategory}}~\CatC is specified by:
    \begin{body}
        \constit
        \begin{compactenum}
            \item Objects: A collection\footnotemark~$\ObC$ whose elements are called \emph{objects}.
            \item Morphisms: For every pair of objects~$\Obja,\Objb$ in $\ObC$, there is a set~$\HomSet{\CatC}{\Obja}{\Objb}$, elements of which are called \emph{morphisms}. We write
            \begin{equation}
                \mora \colon \Obja \mto_\CatC \Objb
            \end{equation}
            to indicate
            \begin{equation}
                \mora \in \HomSet{\CatC}{\Obja}{\Objb}.
            \end{equation}
            \item Composition operations: For every three objects~$\Obja,\Objb,\Objc$ in $\ObC$ there is a composition map
            \begin{equation}
                \mthen_{\Obja,\Objb,\Objc}: \HomSet{\CatC}{\Obja}{\Objb} \times \HomSet{\CatC}{\Objb}{\Objc}\to \HomSet{\CatC}{\Obja}{\Objc}.
            \end{equation}
            We usually just write $\mthen$ instead of $\mthen_{\Obja,\Objb,\Objc}$:
            \begin{equation}
                \prftree{\mora \colon \Obja \mto \Objb}{\morb\colon \Objb \mto \Objc}{\mora \mthen \morb \colon \Obja \mto \Objc}
            \end{equation}
            The morphism~$\mora \mthen \morb$ is called the \emph{composition} of~$\mora$ and~$\morb$.
            %Given any morphism~${\mora} \in  \HomSet{\CatC}{\Obja}{\Objb}$ and any morphism~${\morb} \in \HomSet{\CatC}{\Objb}{\Objc}$, there exists a morphism~$\mora \mthen \morb\in \HomSet{\CatC}{\Obja}{\Objc}$ which is the \emph{composition of~$\mora$ and~$\morb$}.
        \end{compactenum}
        \condit
        \begin{compactenum}
            \item Associativity: it holds that
            % for any morphisms~$\mora\in \HomSet\CatC\Obja\Objb$,~$\morb\in \HomSet\CatC\Objb\Objc$, and~$\morc\in \HomSet\CatC\Objc\Objd$,
            \begin{equation}
                \prftree{\mora\colon \Obja \mto \Objb}{\morb\colon \Objb \mto \Objc}{\morc\colon \Objc \mto \Objd}
                {
                    (\mora\mthen \morb)
                    \mthen \morc= \mora \mthen (\morb \mthen \morc)
                }
            \end{equation}
        \end{compactenum}
    \end{body}
\end{ctdefinition}

\begin{remark}
   Note that we denote composition of morphisms using the symbol ``$\mthen$'' (pronounced ``then''), as already introduced for functions in \cref{sec:functions}.
%    Thus, given~$\mora \colon \Obja\mto \Objb$ and~$\morb \colon \Objb \mto \Objc$, we denote their composite by~$(\mora\mthen \morb)\colon \Obja\mto \Objc$, pronounced ``$\mora$ then~$\morb$''.
    This is in contrast to the more common notation for composition, namely~$\morb\after \mora$, or simply~$\morb \mora$, which reads as ``$\morb$ after~$\mora$''. As usual, $\mora^2$ denotes $\mora \mthen \mora$, and $\mora^3$ denotes $\mora \mthen \mora \mthen \mora$, and so on. 
\end{remark}

\begin{remark}
It is common to visualize objects and morphisms in a (semi)category using dots and arrows. For instance, if $\Obja$ and $\Objb$ are objects in a (semi)category $\CatC$, and $\mora: \Obja \mto \Objb$ is a morphism, we draw this as in \cref{fig:points-and-arrow-diagram}.  
\begin{marginfigure}
\label{fig:points-and-arrow-diagram}
\begin{center}
\begin{tikzcd}
\Obja \arrow[r,"\mora"] & \Objb
\end{tikzcd}
\end{center}

\end{marginfigure}

Similarly, the following diagram depicts a morphism $\mora: \Obja \mto \Objb$, a morphism $\morb: \Objb \mto \Objc$, and their composition $\mora \mthen: \Obja \mto \Objc$. 

\todojira{379}{insert figures showing objects and an arrow between them}

\end{remark}

\begin{example}\label{exa:semicat-plant-states}
We will describe a semicategory $\CatC$ with two objects, $\ObC = \{ \setA, \setB \}$. Each of the objects is a set which describes possible states of a plant (this example is an extension of \cref{exa:plant-trafo-semigroup}). 
Let 
\begin{equation}
\setA = \{ \sprout, \yng, \mature, \old, \dead \}
\end{equation}
and
\begin{equation}
\setB = \{ \alive, \dead \}.
\end{equation}
Also, let $\mapa : \setA \to \setA$ be the function with
\begin{align*}
        \mapa(\sprout) &=  \yng, \\
        \mapa(\yng) &=  \mature, \\
        \mapa(\mature) &=  \old, \\
        \mapa( \old) &= \dead, \\
        \mapa (\dead) &= \dead;
    \end{align*}
let $\morb : \setB \to \setB$ be the function with 
\begin{align*}
        \mapb(\alive) &=  \dead, \\
        \mapb(\dead) &=  \dead;
\end{align*}
and let $\morc : \setA \to \setB$ be the function with
\begin{align*}
        \mapc(\sprout) &= \alive, \\
        \mapc(\yng) &=  \alive, \\
        \mapc(\mature) &= \alive , \\
        \mapc( \old) &= \alive, \\
        \mapc (\dead) &= \dead.
    \end{align*}
For our sets of morphisms, we let 
\begin{align*}
        \HomSet{\CatC}{\setA}{\setA} &= \{ \mora, \mora^2, \mora^3, \mora^4 \}, \\
        \HomSet{\CatC}{\setB}{\setB} &=  \{ \morb, \morb^2 \}, \\
        \HomSet{\CatC}{\setA}{\setB} &=  \{ \morc \}, \\
        \HomSet{\CatC}{\setB}{\setA} &= \emptyset;
\end{align*}
and for the composition operations, we define these to be the usual composition of functions, which we know obeys the associative law. Thus we have a semicategory. 
\todojira{269}{Finish writing the above example involving plant states by adding a nice figure to illustrate it}
\end{example}


\todotext{Add remark about relationship between semicategories and semigroups}  \XXX


\todojira{378}{add the example of discrete-time linear time-invariant systems as an example of a semicategory}

%Now let's view electrical chords as forming a semicategory. For simplicity (and for the ``C''), let's call our semicategory-to-be by the name $\CatC$, same as in the definition above. To define $\CatC$, let the collection $\ObC$ of objects to be the collection of all electrical interface types: one object for each type. Next, given any two types $\Obja$ and $\Objb$ (\text{e.g.}, $\Obja = \TypeIrish$ and $\Objb = \TypeSwiss$), let's define the set of morphisms $\HomSet{\CatC}{\Obja}{\Objb}$ to be the set of all electrical chords that have the type $\Obja$ as their incoming interface and type $\Objb$ as their outgoing interface. 

%\let\oldmora\mora
%\renewcommand{\mora}{{\colTransmuter \oldmora}}
%\let\oldmorb\morb
%\renewcommand{\morb}{{\colTransmuter \oldmorb}}
%\let\oldmorc\morc
%\renewcommand{\morc}{{\colTransmuter \oldmorc}}
%\let\oldObja\Obja
%\renewcommand{\Obja}{{\colTransmuted \oldObja}}
%\let\oldObjb\Objb
%\renewcommand{\Objb}{{\colTransmuted \oldObjb}}
%\let\oldObjc\Objc
%\renewcommand{\Objc}{{\colTransmuted \oldObjc}}
%\let\oldObjd\Objd
%\renewcommand{\Objd}{{\colTransmuted \oldObjd}}
%
%\let\oldOb\Ob
%\renewcommand{\Ob}{{\colTransmuted \oldOb}}
%
%\let\oldHom\Hom
%\renewcommand{\Hom}{{\colTransmuter \oldHom}}
\linkvideo{spring2021-semicats-cats:categories} % Categories
\begin{ctdefinition}[Category]
    \label{def:categorymain}
    A \emph{\iindex{category}}~\CatC is a semicategory with an additional constituent and an additional rule:
    \begin{body}
        \constit
        \begin{compactenum}
            \item Identity morphisms: for each object~$\Obja$, there is
            a morphism~$\catid_{\Obja}\colon \Obja \mto \Obja$  called \emph{the identity
            morphism of~$\Obja$}.
        \end{compactenum}
        \condit
        \begin{compactenum}
            \item Unitality: It holds
            \begin{equation}
                \prftree{\mora \colon \Obja \mto \Objb}{\catid_\Obja \mthen \mora= \mora = \mora \mthen \catid_\Objb}
                % \prftree{\catid_{\Obja}\colon \Obja \to \Obja}{\mora \colon \Obja \to \Objb}{\catid_{\Objb}\colon \Objb \to \Objb}{\catid_\Obja \mthen \mora= \mora = \mora \mthen \catid_\Objb}
            \end{equation}
        \end{compactenum}
    \end{body}
\end{ctdefinition}


% \begin{ctdefinition}[Category]
%   \label{def:categorymain}
%   A \emph{\iindex{category}}~\CatC is:
% \begin{quote}
%     \constit
%   \begin{compactenum}
%     \item Objects: a collection\footnotemark~$\ObC$, whose elements are called \emph{objects}.
%     \item Morphisms: For every pair of objects~$\Obja,\Objb$ in $\ObC$, there is a set~$\HomSet{\CatC}{\Obja}{\Objb}$, elements of which are called \emph{morphisms}. We write
%     \begin{equation}
%     \mora \colon \Obja \to_\CatC \Objb
%     \end{equation}
%     to indicate
%     \begin{equation}
%       \mora \in \HomSet{\CatC}{\Obja}{\Objb}.
%       \end{equation}
%     \item Identity morphisms: for each object~$\Obja$, there is
%     an element~$\catid_{\Obja} \in \HomSet{\CatC}{\Obja}{\Obja}$ which is called \emph{the identity
%     morphism of~$\Obja$}.
%     \item Composition operations: For every three objects~$\Obja,\Objb,\Objc$ in $\ObC$ there is a composition map
%     \begin{equation}
%         \mthen_{\Obja,\Objb,\Objc}: \HomSet{\CatC}{\Obja}{\Objb} \times \HomSet{\CatC}{\Objb}{\Objc}\to \HomSet{\CatC}{\Obja}{\Objc}
%     \end{equation}
%     It holds that
%     \begin{equation}
%         \prftree{\mora \colon \Obja \to \Objb}{\morb\colon \Objb \to \Objc}{\mora \mthen \morb \colon \Obja \to \Objc}
%     \end{equation}
%     The morphism~$\mora \mthen \morb$ is called the \emph{composition} of~$\mora$ and~$\morb$.
%     %Composition operations: given any morphism~${\mora} \in  \HomSet{\CatC}{\Obja}{\Objb}$ and any morphism~${\morb} \in \HomSet{\CatC}{\Objb}{\Objc}$, there exists a morphism~$\mora \mthen \morb\in \HomSet{\CatC}{\Obja}{\Objc}$ which is the \emph{composition of~$\mora$ and~$\morb$}.
%   \end{compactenum}
%  \condit
%   \begin{compactenum}
%     \item Unitality: It holds that:
%     \begin{equation}
%         \prftree{\catid_{\Obja}\colon \Obja \to \Obja}{\mora \colon \Obja \to \Objb}{\catid_{\Objb}\colon \Objb \to \Objb}{\catid_\Obja \mthen \mora= \mora = \mora \mthen \catid_\Objb}
%     \end{equation}
%     \item Associativity:  it holds that
%     % for any morphisms~$\mora\in \HomSet\CatC\Obja\Objb$,~$\morb\in \HomSet\CatC\Objb\Objc$, and~$\morc\in \HomSet\CatC\Objc\Objd$,
%     \begin{equation}
%       \prftree{\mora\colon \Obja \to \Objb}{\morb\colon \Objb \to \Objc}{\morc\colon \Objc \to \Objd}
%       {
%     (\mora\mthen \morb)
%       \mthen \morc= \mora \mthen (\morb \mthen \morc)
%       }
%     \end{equation}
%   \end{compactenum}
%   \end{quote}
% \end{ctdefinition}

\footnotetext{A ``collection'' is something which may be thought of as a set, but may be ``too large" to technically be a set in the formal sense. This distinction is necessary in order to avoid such issues as Russel's paradox.}



\begin{gradedexercise}[\exname{CategorySemigroups}]
    There is a category where the objects are semigroups and the morphisms are semigroup homomorphisms. Spell out explicitely what this category is: check in detail each of the points of \cref{def:categorymain}.
\end{gradedexercise}

\solutionof{CategorySemigroups}

\showslides{
    \begin{forslides}


        $$ \mora \quad \morb \quad \morc $$

        $$X = \{ \sprout, \yng, \mature, \old, \dead \}$$


        $$T \colon X \sto X$$


        \begin{align*}
            T(\text{sprout}) &=  \yng \\
            T(\yng) &=  \mature \\
            T(\mature) &=  \old \\
            T( \old) &= \dead \\
            T (\dead) &= \dead
        \end{align*}


        $$\sgrpA = \{ T^n \mid n \in \natnumbers \}$$

        $$ T^n = T^4 \quad \forall \ n \geq 4 $$

        $$\sgrpA = \{T, T^2, T^3, T^4 \}$$

        $$  \setA = \{ \alphabeta, \alphabetb \}$$

        $$\sgrpA = \{ \text{set of non-empty strings of elements of }\setA\}$$

        $$Y = \{ \text{alive}, \dead \}$$

        $$ T' : Y \sto Y $$

        $$\sgrpA' = \{T' \}$$


        \begin{align*}
            T'(\text{alive}) &=  \dead \\
            T' (\dead) &= \dead
        \end{align*}

        $$ R: X \sto Y $$

        \begin{align*}
            R(\text{sprout}) &= \text{alive} \\
            R(\yng) &=  \text{alive} \\
            R(\mature) &= \text{alive}\\
            R( \old) &= \text{alive} \\
            R (\dead) &= \dead
        \end{align*}

        $$\monoidA = \{ \id, T, T^2, T^3, T^4 \}$$

        $$\monoidA' = \{ \id, T' \}$$

        \begin{equation*}
            \worda \text{ and } \wordb
        \end{equation*}

        \begin{equation*}
            \label{eq:string-semigroup-wordab-2}
            \worda \mtimes  \wordb =  \worda \wordb
        \end{equation*}


        $$ \tup{ \monoidA, \mtimes, \idmon} $$


        \begin{equation*}
            \label{eq:string-semigroup-wordab-concat-2}
            \worda \wordb
        \end{equation*}

        ---

        \begin{equation}
            \label{eq:00}
            \Objd
        \end{equation}

        \begin{equation}
            \label{eq:01}
            \Obja
        \end{equation}

        \begin{equation}
            \label{eq:02}
            \Objb
        \end{equation}

        \begin{equation}
            \label{eq:03}
            \Objc
        \end{equation}

        \begin{equation}
            \label{eq:04}
            \mora
        \end{equation}

        \begin{equation}
            \label{eq:05}
            \morb
        \end{equation}

        \begin{equation}
            \label{eq:06}
            \morc
        \end{equation}

        \begin{equation}
            \label{eq:07}
            \mord
        \end{equation}

        \begin{equation}
            \label{eq:08}
            \mora \mthen \morb
        \end{equation}

        \begin{equation}
            \label{eq:09}
            \morb \mthen \morc
        \end{equation}

        \begin{equation}
            \label{eq:10}
            \morc \mthen \mord
        \end{equation}

        \begin{equation}
            \label{eq:11}
            (\mora \mthen \morb) \mthen \morc
        \end{equation}

        \begin{equation}
            \label{eq:12}
            \mora \mthen  (\morb \mthen \morc )
        \end{equation}

        \begin{equation}
            \label{eq:13}
            \mora \mthen  ((\morc \mthen \morb ) \mthen (( \mord \mthen \morb ) \mthen \mora ))
        \end{equation}

        \begin{equation}
            \label{eq:14}
            \mora \mthen  \morc \mthen \morb  \mthen \mord \mthen \morb  \mthen \mora
        \end{equation}


        \begin{ctdefinition}
            \label{def:category-var}
            A \emph{\iindex{category}}~\CatC is specified by:
            \begin{body}
                \constit
                \begin{compactenum}
                    \item Objects: A set $\ObC$ whose elements are called \emph{objects}.
                    \item Morphisms: For every pair of objects~$\Obja,\Objb$ in $\ObC$, there is a set~$\HomSet{\CatC}{\Obja}{\Objb}$, elements of which are called \emph{morphisms} from $\Obja$ to $\Objb$. A morphism $\mora \in \HomSet{\CatC}{\Obja}{\Objb}$ is often indicated by writing
                    \begin{equation*}
                        \mora \colon \Obja \mto \Objb.
                    \end{equation*}
                    \item Identity morphisms: for each object~$\Obja$, there is
                    a morphism~$\catid_{\Obja}\colon \Obja \mto \Obja$  called \emph{the identity
                    morphism of~$\Obja$}.
                    \item Composition operations: For every three objects~$\Obja,\Objb,\Objc$ in $\ObC$ there is a composition function
                    \begin{equation*}
                        \mthen_{\Obja,\Objb,\Objc}: \HomSet{\CatC}{\Obja}{\Objb} \times \HomSet{\CatC}{\Objb}{\Objc}\to \HomSet{\CatC}{\Obja}{\Objc}
                    \end{equation*}
                    We denote the \emph{composition} of composable morphisms~$\mora$ and~$\morb$ by~$\mora \mthen \morb$. (Traditionally, the typical notation would be $\morb \circ \mora$).

                \end{compactenum}
                \condit
                \begin{compactenum}
                    \item Associativity: for all composable morphisms $\mora, \morb, \morc$,
                    $$ (\mora\mthen \morb)
                    \mthen \morc= \mora \mthen (\morb \mthen \morc).$$
                    \item Unitality: for every morphism $\mora \colon \Obja \mto \Objb$,
                    $$ \catid_{\Obja} \then \mora = \mora \quad \quad \text{and} \quad \quad \mora \then \catid_{\Objb} = \mora. $$
                \end{compactenum}
            \end{body}
        \end{ctdefinition}

        \begin{equation}
            \label{eq:15}
            \CatC
        \end{equation}

        \begin{equation}
            \label{eq:16}
            \ObC
        \end{equation}

        \begin{equation}
            \label{eq:17}
            \HomSet{\CatC}{\Obja}{\Objb}
        \end{equation}

        \begin{equation}
            \label{eq:18}
            \catid_{\Obja}\colon \Obja \mto \Obja
        \end{equation}

        \begin{equation}
            \label{eq:19}
            \mthen_{\Obja,\Objb,\Objc}: \HomSet{\CatC}{\Obja}{\Objb} \times \HomSet{\CatC}{\Objb}{\Objc}\to \HomSet{\CatC}{\Obja}{\Objc}
        \end{equation}

        \begin{equation}
            \label{eq:20}
            (\mora\mthen \morb)
            \mthen \morc= \mora \mthen (\morb \mthen \morc)
        \end{equation}

        \begin{equation}
            \label{eq:21}
            \catid_{\Obja} \then \mora = \mora
        \end{equation}

        \begin{equation}
            \label{eq:22}
            \mora \then \catid_{\Objb} = \mora
        \end{equation}

        \begin{equation}
            \label{eq:23}
            \CatC = \Set
        \end{equation}


        \begin{equation}
            \label{eq:24}
            \catid_{\Obja}
        \end{equation}

        \begin{equation}
            \label{eq:25}
            \catid_{\Objb}
        \end{equation}

        \begin{equation}
            \label{eq:26}
            \catid_{\Objc}
        \end{equation}

        \begin{equation}
            \label{eq:27}
            \styleobj{\posReals}
        \end{equation}

        \begin{equation}
            \label{eq:28}
            \stylemorph{f(x) = x^2}
        \end{equation}

        \begin{equation}
            \label{eq:29}
            \stylemorph{g(y) = \frac{1}{y}}
        \end{equation}

        \begin{equation}
            \label{eq:30}
            \stylemorph{h(x) = \frac{1}{x^2}}
        \end{equation}

        \begin{equation}
            \label{eq:31}
            \mora \then \morb = \catid_\Obja
        \end{equation}

        \begin{equation}
            \label{eq:32}
            \morb \then \mora = \catid_\Objb
        \end{equation}

        \begin{equation}
            \label{eq:33}
            \styleobj{U}
        \end{equation}

        \begin{equation}
            \label{eq:34}
            \styleobj{P}
        \end{equation}

        \begin{equation}
            \label{eq:35}
            \styleobj{X \times Y}
        \end{equation}

        \begin{equation}
            \label{eq:36}
            \styleobj{X \cap Y}
        \end{equation}

        \begin{equation}
            \label{eq:37}
            \styleobj{X \cup Y}
        \end{equation}

        \begin{equation}
            \label{eq:38}
            l
        \end{equation}

        \begin{equation}
            \label{eq:39}
            m
        \end{equation}

        \begin{equation}
            \label{eq:40}
            n
        \end{equation}

        \begin{equation}
            \label{eq:41}
            l, m, n \in \natnumbers
        \end{equation}

        \begin{equation}
            \label{eq:42}
            \mtimescat
        \end{equation}

        \begin{equation}
            \label{eq:43}
            \morb \circ \mora
        \end{equation}

        \begin{equation}
            \label{eq:44}
            0 + n = n = n + 0   \quad \quad \forall \ n \in \natnumbers
        \end{equation}

        \begin{equation}
            \label{eq:45}
            (l + m) + n = l + (m + n) \quad \quad  \forall \ l, m, n \in \natnumbers
        \end{equation}

        \begin{equation}
            \label{eq:46}
            \mora \then \morb = \morc \then \mord
        \end{equation}

        \begin{equation}
            \label{eq:47}
            \styleobj{A}
        \end{equation}

        \begin{equation}
            \label{eq:48}
            \styleobj{B}
        \end{equation}

        \begin{equation}
            \label{eq:49}
            \styleobj{C}
        \end{equation}

        \begin{equation}
            \label{eq:50}
            \styleobj{D}
        \end{equation}

        \begin{equation}
            \label{eq:51}
            \stylemorph{j}
        \end{equation}

        \begin{equation}
            \label{eq:52}
            \stylemorph{k}
        \end{equation}

    \end{forslides}
}


\linkvideo{spring2021-semicats-cats:summary} % Summary
