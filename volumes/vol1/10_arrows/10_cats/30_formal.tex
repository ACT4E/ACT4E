% !TEX root = chapter-standalone.tex

\section[Semicategories]{Semicategories}
\label{sec:formal-def-semicat}

\linkvideo{spring2021-semicats-cats:intro_semi_cat} % Introduction for Semicategories and Categories

We begin to introduce the concept of a \SY{semicategory} by listing key aspects of the electrical cords example in the previous section which each correspond to a part of the formal definition of \SY{semicategory}.

\subsection{Objects, morphisms, composition}

Firstly, in the cords example, we have a set of types of electrical interfaces.
In a \SY{semicategory}, the things that play the role of interfaces are called \emph{objects}.
(In this text, we often denote generic objects by letters~$\Obja$,~$\Objb$,~$\Objc$, \etc)

Secondly, in the electrical cords example, the cords themselves connect two interfaces and have a directionality: one end has a socket, the other end has a plug.
In a \SY{semicategory}, the things that play the role of cords are called \emph{morphisms}.
(This is a word we've already gotten to know, and the connection here is not accidental.) The object that denotes the type of the ``socket end'' of the morphism is called the \emph{domain} or \emph{source} of the morphism.
The object that denotes the ``plug end'' of a morphism is called its \emph{codomain} or \emph{target}.
So morphisms are directed: they go from their source to their target.

To visualize morphisms, we often draw arrows.
For example, if~$\Obja$ and~$\Objb$ are objects in some \SY{semicategory}, then we draw a morphism from~$\Obja$ to~$\Objb$ (call it~$\mora$) like this:
\begin{equation}
    \mora\colon \Obja \mto \Objb.
\end{equation}
So here~$\Obja$ is the source of~$\mora$, and~$\Objb$ is its target.

Thirdly, a key feature of the cords example is that we can compose two cords and the result is again a chord.
However, for this to work, the plug-end of the first chord must match the socket-end of the other.
Similarly, in a \SY{semicategory} we specify a way to compose two morphisms, provided that the target object of the first morphism matches the source object of the second morphism.
(When this is the case, the morphisms are said to be \emph{composable}.)

In other words, if~$\mora\colon \Obja \mto \Objb$ and~$\morb\colon \Objb \mto \Objc$ are morphisms in some \SY{semicategory} (note: they are composable), then the \SY{semicategory} has an operation for composing them, and the result is another morphism.
Our notation for composition of~$\mora$ and~$\morb$ is~$\morab$, read ``$\mora$ then~$\morb$''.
(Once again: this is in contrast to the more traditional notation~$\morb \after \mora$.)
Thus, we have
\begin{equation}
    \prfcomma{
        \mora \colon \Obja \mto \Objb
    }{
        \morb\colon \Objb \mto \Objc
    }{
        (\morab )\colon \Obja \mto \Objc
    }
\end{equation}
which is analogous to \cref{eq:cords-generic} in the cords example.

\subsection{Associativity}

So far we have described the building blocks, or constituents, of a \SY{semicategory}: objects, morphisms, and composition operations.
We also want these to obey a certain condition called the \emph{associativity law}.
This condition says that if we are given a string of three composable morphisms, then it doesn't matter in which order we choose to compose them:

\begin{equation}
    \prfperiod{
        \mora \colon \Obja \mto \Objb
    }{
        \morb\colon \Objb \mto \Objc
    }{
        \morc \colon \Objc \mto \Objd
    }{
        (\morab ) \mthen \morc = \mora \mthen ( \morb  \mthen \morc )
    }
\end{equation}

This is analogous to the fact, in the electrical cords example, that if we look at three cords connected together, we cannot tell if the first two were connected together first and then the result was connected to the third, or if the connecting happened the other way around.

\subsection{The definition of a semicategory}
\linkvideo{spring2021-semicats-cats:semicats} % Semicategories

Here is the full formal definition of a \SY{semicategory}.
Have a read through, then we illustrate it further with examples.

\todomistake{\alphubel: @JL: the footnote mark after ``collection'' goes nowhere.
    Need to define collection.
    Or just call this a ``small semicategory''.
}
\SYNDEF{semicategory}
\SYNDEF{hom-set}
\begin{ctdefinition}[Semicategory]
    \label{def:semicategory-compact}
    A \maindef{semicategory}~\CatC is specified by:
    \begin{body}
        \constit
        \begin{enumerate}
            \item \label{semicategory-objects} Objects: A collection\footnotemark~$\ObC$ whose elements are called \emph{objects}.
            \item \label{def:hom-set}
                  Morphisms: For every pair of objects~$\Obja,\Objb$ in $\ObC$, there is a set called a ``hom-set'' and indicated as~$\HomSet{\CatC}{\Obja}{\Objb}$, elements of which are called \emph{morphisms} and denoted $\mora \colon \Obja \mto \Objb$.

                  For such an $\mora$, we call $\Obja$ its \emph{source} and $\Objb$ its \emph{target}.
            \item \label{semicategory-composition}
                  Composition operations: For every three objects~$\Obja,\Objb,\Objc$ in~$\ObC$ there is a composition map
                  \begin{equation}\label{eq:semicategory-composition}
                      \mthenof{\Obja,\Objb,\Objc}\colon \HomSet{\CatC}{\Obja}{\Objb} \cartprod \HomSet{\CatC}{\Objb}{\Objc}\mto \HomSet{\CatC}{\Obja}{\Objc}.
                  \end{equation}
                  We usually just write~$\mthen$ instead of~$\mthenof{\Obja,\Objb,\Objc}$:
                  \begin{equation}
                      \prfperiod{
                          \mora \colon \Obja \mto \Objb
                      }{
                          \morb\colon \Objb \mto \Objc
                      }{
                          (\morab )\colon \Obja \mto \Objc
                      }
                  \end{equation}
                  The morphism~$\morab$ is called the \emph{composition} of~$\mora$ and~$\morb$.
                  %Given any morphism~${\mora} \setin  \HomSet{\CatC}{\Obja}{\Objb}$ and any morphism~${\morb} \setin \HomSet{\CatC}{\Objb}{\Objc}$, there exists a morphism~$\mora \mthen \morb\setin \HomSet{\CatC}{\Obja}{\Objc}$ which is the \emph{composition of~$\mora$ and~$\morb$}.
        \end{enumerate}
        \condit
        \begin{enumerate}
            \item \label{semicategory-associativity}
                  Associativity: it holds that
                  % for any morphisms~$\mora\setin \HomSet\CatC\Obja\Objb$,~$\morb\setin \HomSet\CatC\Objb\Objc$, and~$\morc\setin \HomSet\CatC\Objc\Objd$,
                  \begin{equation}
                      \prfperiod{
                          \mora\colon \Obja \mto \Objb
                      }{
                          \morb\colon \Objb \mto \Objc
                      }{
                          \morc\colon \Objc \mto \Objd
                      }{
                          (\morab)
                          \mthen \morc= \mora \mthen (\morb \mthen \morc)
                      }
                  \end{equation}
        \end{enumerate}
    \end{body}
\end{ctdefinition}

\begin{remark}
    We denote composition of morphisms using the symbol ``$\mthen$'' (pronounced ``then''), as already introduced for functions in \cref{sec:functions}.
    %    Thus, given~$\mora \colon \Obja\mto \Objb$ and~$\morb \colon \Objb \mto \Objc$, we denote their composite by~$(\mora\mthen \morb)\colon \Obja\mto \Objc$, pronounced ``$\mora$ then~$\morb$''.
    This is in contrast to the more common notation for composition, namely~$\morb\after \mora$, or simply~$\morb \mora$, which reads as ``$\morb$ after~$\mora$''.
    As usual,~$\mora^2$ denotes~$\mora \mthen \mora$, $\mora^3$ denotes~$\mora \mthen \mora \mthen \mora$, and so on.
\end{remark}

\begin{remark}
    \begin{enumerate}
        \item When we want to emphasize which \SY{semicategory} we are working with, we will sometimes write
              \begin{equation}
                  \mora \colon \Obja \mtoin\CatC \Objb
              \end{equation}
              to indicate
              \begin{equation}
                  \mora \setin \HomSet{\CatC}{\Obja}{\Objb}.
              \end{equation}
        \item Sometimes we will use the notation $\Mor_\CatC$ to denote the collection of \emph{all} morphisms in a semicategory $\CatC$, not just a certain hom-set.
    \end{enumerate}
\end{remark}

%\begin{marginfigure}
%    \begin{center}
%        \includesag{points_and_arrow}
%    \end{center}
%    \caption{}
%    \label{fig:points-and-arrow-diagram}
%\end{marginfigure}

\begin{marginfigure}
    \centering
    \includesag{points_and_two_arrows}
    \caption{}
    \label{fig:points-and-two-arrows-diagram}
\end{marginfigure}

\begin{marginfigure}
    \centering
    \includesag{points_and_composed_arrows}
    \caption{}
    \label{fig:points-and-composed-arrows-diagram}
\end{marginfigure}

\begin{remark}
    We will often visualize objects and morphisms using diagrams where symbols or dots indicate objects and arrows indicate morphisms.
    For instance, if~$\Obja$,~$\Objb$,~$\Objc$ are objects in a semicategory~\CatC, and~$\mora\colon \Obja \mto \Objb$ and~$\morb\colon \Objb \mto \Objc$ are morphisms, we draw this as in \cref{fig:points-and-two-arrows-diagram}, for example.
    The composition~$\morab$ of~$\mora$ with~$\morb$ corresponds to yet another arrow, as in \cref{fig:points-and-composed-arrows-diagram}.
\end{remark}

\begin{example}
    \label{exa:the-semicategory-of-sets}
    There is a semicategory whose objects are all sets, morphisms are functions between sets, and the composition operations are the usual composition of functions.

    Let us also check that composition of functions does indeed satisfy the associativity law.
    Suppose $\mora \colon \Obja \mto \Objb$,  $\morb\colon \Objb \mto \Objc$, and  $\morc \colon \Objc \mto \Objd$ are three composable functions, and let $\ela \setin \Obja$ be an arbitrary element.
    On the one hand
    \begin{equation}\label{eq:function-associativity-1}
        ((\morab) \mthen \morc)(\ela) =  \morc((\morab)(\ela)) = \morc (\morb (\mora (\ela))),
    \end{equation}
    while on the other hand
    \begin{equation}\label{eq:function-associativity-2}
        (\mora \mthen (\morb \mthen \morc))(\ela)
        =  (\morb \mthen \morc)(\mora(\ela)) = \morc (\morb (\mora (\ela))).
    \end{equation}
    So
    \begin{equation}\label{eq:function-associativity-3}
        (\morab)
        \mthen \morc = \mora \mthen (\morb \mthen \morc)
    \end{equation}
    holds.
\end{example}

\begin{example}
    \label{exa:semicat-plant-states}
    This example is an extension of \cref{exa:plant-trafo-semigroup}.
    We will describe a \SY{semicategory}~\CatC with two objects,~$\ObC = \makeset{ \setA, \setB }$.
    Each of the objects is a set which describes possible states of a plant.
    Let
    \begin{equation}
        \setA = \makeset{ \sprout, \yng, \mature, \old, \dead },
    \end{equation}
    and
    \begin{equation}
        \setB = \makeset{ \alive, \dead }.
    \end{equation}
    Also, let~$\mapa \colon \setA \sto \setA$ be the function with
    \begin{equation}
        \begin{aligned}
            \mapa(\sprout) & =  \yng, \\
            \mapa(\yng)    & =  \mature, \\
            \mapa(\mature) & =  \old, \\
            \mapa( \old)   & = \dead, \\
            \mapa(\dead)   & = \dead;
        \end{aligned}
    \end{equation}
    let~$\morb \colon \setB \sto \setB$ be the function with
    \begin{equation}
        \begin{aligned}
            \mapb(\alive) & =  \dead, \\
            \mapb(\dead)  & =  \dead;
        \end{aligned}
    \end{equation}
    and let~$\morc \colon \setA \sto \setB$ be the function with
    \begin{equation}
        \begin{aligned}
            \mapc(\sprout) & = \alive, \\
            \mapc(\yng)    & =  \alive, \\
            \mapc(\mature) & = \alive, \\
            \mapc( \old)   & = \alive, \\
            \mapc (\dead)  & = \dead.
        \end{aligned}
    \end{equation}
    For our sets of morphisms, we let
    \begin{equation}
        \begin{aligned}
            \HomSet{\CatC}{\setA}{\setA} & = \makeset{\mora, \mora^2, \mora^3, \mora^4 }, \\
            \HomSet{\CatC}{\setB}{\setB} & =  \makeset{ \morb, \morb^2 }, \\
            \HomSet{\CatC}{\setA}{\setB} & =  \makeset{ \morc, \mora \mthen \morc, \mora^2 \mthen \morc, \mora^3 \mthen \morc, \mora^4 \mthen \morc}, \\
            \HomSet{\CatC}{\setB}{\setA} & = \Emptyset;
        \end{aligned}
    \end{equation}
    and for the composition operations, we define these to be the usual composition of functions, which we know obeys the \SY{associative law}.
    Thus, we have a \SY{semicategory}.
    \todojira{269}{\bernina: Finish writing the above example involving plant states by adding a nice figure to illustrate it}
\end{example}

\todotext{\bernina: @JL: Seems like it would be good to add more examples here}

\todotextjira{394}{\bernina: @JL: Add remark about relationship between \SY{semicategories} and semigroups}

%Now let's view electrical chords as forming a semicategory. For simplicity (and for the ``C''), let's call our semicategory-to-be by the name $\CatC$, same as in the definition above. To define $\CatC$, let the collection $\ObC$ of objects to be the collection of all electrical interface types: one object for each type. Next, given any two types $\Obja$ and $\Objb$ (\text{e.g.}, $\Obja = \TypeIrish$ and $\Objb = \TypeSwiss$), let's define the set of morphisms $\HomSet{\CatC}{\Obja}{\Objb}$ to be the set of all electrical chords that have the type $\Obja$ as their incoming interface and type $\Objb$ as their outgoing interface.

%\let\oldmora\mora
%\renewcommand{\mora}{{\colTransmuter \oldmora}}
%\let\oldmorb\morb
%\renewcommand{\morb}{{\colTransmuter \oldmorb}}
%\let\oldmorc\morc
%\renewcommand{\morc}{{\colTransmuter \oldmorc}}
%\let\oldObja\Obja
%\renewcommand{\Obja}{{\colTransmuted \oldObja}}
%\let\oldObjb\Objb
%\renewcommand{\Objb}{{\colTransmuted \oldObjb}}
%\let\oldObjc\Objc
%\renewcommand{\Objc}{{\colTransmuted \oldObjc}}
%\let\oldObjd\Objd
%\renewcommand{\Objd}{{\colTransmuted \oldObjd}}
%
%\let\oldOb\Ob
%\renewcommand{\Ob}{{\colTransmuted \oldOb}}
%
%\let\oldHom\Hom
%\renewcommand{\Hom}{{\colTransmuter \oldHom}}

\section{Categories}
\linkvideo{spring2021-semicats-cats:categories} % Categories

Now we come to one of the central protagonists of this book: the concept of a category.
Categories are like \SY{semicategories}, but with one more ingredient added: \SY{identity morphisms}.
A good analogy is that categories are to \SY{semicategories} as \SY{monoids} are to \SY{semigroups}.
A \SY{monoid} is a \SY{semigroup} that additionally has an identity element, and similarly a category is a \SY{semicategory} that additionally has \SY{identity morphisms}.

\subsection{Identity morphisms}

One might say that  \SY{identity morphisms} are morphisms that ``do nothing'': they do not have any effect when we compose with them.
This is analogous to how the identity element of a \SY{monoid} ``does nothing'' when we multiply it with other elements of the \SY{monoid}.

\SYNDEF{identity morphism}
\begin{ctdefinition}[Identity morphisms]
    \label{def:identity-morphism}
    Let~\CatC be a \SY{semicategory}.
    An \maindef{identity morphism}, or just \emph{identity}, for an object~$\Obja$ of~\CatC is a morphism
    \begin{equation}
        \catidat\Obja \colon \Obja \mto \Obja
    \end{equation}
    in~\CatC that acts neutrally with respect to composition with any morphism in the category with which it is composable:
    \begin{equation}\label{eq:identities-unitality-1}
        \prfcomma{
            \mora \colon \Objf \mto \Obja
        }{
            \mora \mthen \catidat\Obja = \mora
        }
    \end{equation}
    and
    \begin{equation}\label{eq:identities-unitality-2}
        \prfperiod{
            \morb \colon \Obja \mto \Objb
        }{
            \catidat\Obja \mthen \morb = \morb
        }
    \end{equation}
\end{ctdefinition}

\begin{remark}
    If an \SY{identity morphism}~$\catidat\Obja$ for an object~$\Obja$ exists, then it is unique.
    To see this, observe that~$\HomSet{\CatC}{\Obja}{\Obja}$ is a \SY{semigroup}, and~$\catidat\Obja$ is a \SY{neutral element} for this \SY{semigroup}, making~$\HomSet{\CatC}{\Obja}{\Obja}$ a \SY{monoid}.
    We have seen earlier that neutral elements for \SY{semigroups} are necessarily unique.
\end{remark}

\subsection{Categories}

\begin{ctdefinition}[Category]
    \SYNDEF{category}
    \label{def:categorymain}
    A \maindef{category}~\CatC is a \SY{semicategory} in which there is an \SY{identity morphism} for every object.
\end{ctdefinition}
%  an additional constituent and an additional rule:
%     \begin{body}
%         \constit
%         \begin{enumerate}
%             \item Identity morphisms: for each object~$\Obja$, there is a morphism
%             \begin{equation}
%                 \catid_{\Obja}\colon \Obja \mto \Obja
%             \end{equation}
%             called \emph{the identity morphism of~$\Obja$}.
%         \end{enumerate}
%         \condit
%         \begin{enumerate}
%             \item Unitality: It holds that composing a morphism with a compatible identity leaves the morphism unchanged:
%             \begin{equation}
%                 \prfperiod{
%                     \mora \colon \Obja \mto \Objb
%                     }{
%                         \catid_\Obja \mthen \mora= \mora = \mora \mthen \catid_\Objb
%                         }
%                 % \prftree{\catid_{\Obja}\colon \Obja \to \Obja}{\mora \colon \Obja \to \Objb}{\catid_{\Objb}\colon \Objb \to \Objb}{\catid_\Obja \mthen \mora= \mora = \mora \mthen \catid_\Objb}
%             \end{equation}
%         \end{enumerate}
%     \end{body}
% \end{ctdefinition}

% \begin{ctdefinition}[Category]
%   \label{def:categorymain}
%   A \maindef{category}~\CatC is:
% \begin{quote}
%     \constit
%   \begin{enumerate}
%     \item Objects: a collection\footnotemark~$\ObC$, whose elements are called \emph{objects}.
%     \item Morphisms: For every pair of objects~$\Obja,\Objb$ in $\ObC$, there is a set~$\HomSet{\CatC}{\Obja}{\Objb}$, elements of which are called \emph{morphisms}. We write
%     \begin{equation}
%     \mora \colon \Obja \to_\CatC \Objb
%     \end{equation}
%     to indicate
%     \begin{equation}
%       \mora \setin \HomSet{\CatC}{\Obja}{\Objb}.
%       \end{equation}
%     \item Identity morphisms: for each object~$\Obja$, there is
%     an element~$\catid_{\Obja} \setin \HomSet{\CatC}{\Obja}{\Obja}$ which is called \emph{the identity
%     morphism of~$\Obja$}.
%     \item Composition operations: For every three objects~$\Obja,\Objb,\Objc$ in $\ObC$ there is a composition map
%     \begin{equation}
%         \mthenof{\Obja,\Objb,\Objc}: \HomSet{\CatC}{\Obja}{\Objb} \times \HomSet{\CatC}{\Objb}{\Objc}\to \HomSet{\CatC}{\Obja}{\Objc}
%     \end{equation}
%     It holds that
%     \begin{equation}
%         \prftree{\mora \colon \Obja \to \Objb}{\morb\colon \Objb \to \Objc}{\mora \mthen \morb \colon \Obja \to \Objc}
%     \end{equation}
%     The morphism~$\mora \mthen \morb$ is called the \emph{composition} of~$\mora$ and~$\morb$.
%     %Composition operations: given any morphism~${\mora} \setin  \HomSet{\CatC}{\Obja}{\Objb}$ and any morphism~${\morb} \setin \HomSet{\CatC}{\Objb}{\Objc}$, there exists a morphism~$\mora \mthen \morb\setin \HomSet{\CatC}{\Obja}{\Objc}$ which is the \emph{composition of~$\mora$ and~$\morb$}.
%   \end{enumerate}
%  \condit
%   \begin{enumerate}
%     \item Unitality: It holds that:
%     \begin{equation}
%         \prftree{\catid_{\Obja}\colon \Obja \to \Obja}{\mora \colon \Obja \to \Objb}{\catid_{\Objb}\colon \Objb \to \Objb}{\catid_\Obja \mthen \mora= \mora = \mora \mthen \catid_\Objb}
%     \end{equation}
%     \item Associativity:  it holds that
%     % for any morphisms~$\mora\setin \HomSet\CatC\Obja\Objb$,~$\morb\setin \HomSet\CatC\Objb\Objc$, and~$\morc\setin \HomSet\CatC\Objc\Objd$,
%     \begin{equation}
%       \prftree{\mora\colon \Obja \to \Objb}{\morb\colon \Objb \to \Objc}{\morc\colon \Objc \to \Objd}
%       {
%     (\mora\mthen \morb)
%       \mthen \morc= \mora \mthen (\morb \mthen \morc)
%       }
%     \end{equation}
%   \end{enumerate}
%   \end{quote}
% \end{ctdefinition}

\begin{example}
    \label{exa:the-category-of-sets}
    The semicategory of sets and functions described above in \cref{exa:the-semicategory-of-sets} is in fact a category.
    Given a set $\Obja$, the \SY{identity morphism} for this set is the function
    \begin{equation}
        \defmapperiod{
            \catidat\Obja
        }{
            \Obja
        }{
            \to
        }{
            \Obja
        }{
            \ela
        }{
            \ela
        }
    \end{equation}

    Let us check that the conditions \cref{eq:identities-unitality-1} and \cref{eq:identities-unitality-2} are satisfied.
    Given a function $\mora \colon \Objf \mto \Obja$, the function composition $\mora \mthen \catidat\Obja$ is the same function as just $\mora$ on its own:
    \begin{equation}
        (\mora \mthen \catidat\Obja)(\ela)
        = \catidat\Obja(\mora (\ela)) = \mora (\ela).
    \end{equation}
    Given a function $\morb \colon \Obja \mto \Objb$, we can show similarly that $\catidat\Obja \mthen \morb = \morb$.
\end{example}

\begin{ctdefinition}[Category of sets]
    \SYNDEF{category of sets and functions}
    \label{def:Set}
    The category \Set of sets is defined by:
    \begin{enumerate}
        \item \emph{Objects}: all sets.
        \item \emph{Morphisms}: given sets~$\Obja$ and~$\Objb$, the hom-set~$\HomSet{\Set}{\Obja}{\Objb}$ is the set of all functions from~$\Obja$ to~$\Objb$.
        \item \emph{Composition}: the usual composition of functions.
        \item \emph{Identity morphisms}: given a set~$\Obja$, its \SY{identity morphism}~$\catidat\Obja$ is the identity function~$\Obja \mto \Obja, \ \catidat\Obja(\ela) = \ela$.
    \end{enumerate}
\end{ctdefinition}

\todojira{667}{\alphubel: @JL: Fill this section.
    Add an example of a category that just involved three sets? }

\subsection{Isomorphisms}

What are identity morphisms good for? One thing we can do with them is define, for any category, the important notion of isomorphism.
This concept describes a way of saying when two objects are ``the same'', even if they are not equal.

\begin{definition}\label{def:isomorphism-in-any-cat}
    Let $\CatC$ be a category.
    A morphism $\mora \colon \Obja \mto \Objb$ in $\CatC$ is an \emph{isomorphism} if there exists a morphism $\morb \colon \Objb \mto \Obja$ such that
    \begin{equation}
        \mora \mthen \morb = \catid_\Obja
    \end{equation}
    and
    \begin{equation}
        \morb \mthen \mora = \catid_\Objb.
    \end{equation}
\end{definition}

\begin{remark}
    Note that the above definition coincides, for the category $\Set$ of sets and functions, with \cref{def:function-isomorphism}.
    We saw in \cref{ex:bijective-functions-are-isomorphisms} that an isomorphism in the category \Set is the same thing as a bijective function.
\end{remark}

% \linkvideo{spring2021-semicats-cats:summary} % Summary
