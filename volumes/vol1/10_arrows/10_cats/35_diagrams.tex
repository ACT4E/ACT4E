% !TEX root = chapter-standalone.tex


\section{Diagrams}


\todojira{367}{@J: Write section introducing the notion of diagrams and commutative diagrams}

In category theory it is common to use diagrams to represent and reason about mathematical situations.
In this section we will explain the precise meaning that is usually given to the word ``diagram'' when working with (semi)categories.
Later in the book we will also get to know another kind of diagram, called \emph{string diagrams}, that are another, and slightly different, visual tool for reasoning diagrammatically.

As we have seen in \cref{sec:formal-def-semicat}, a category consists of objects and morphisms (or ``arrows'') between objects, along with identity morphisms and a way to compose arrows.
Suppose we are considering a specific category $\CatC$, and suppose $\mora: \Obja \mto \Objb$ is a morphism in $\CatC$.
This is depicted visually as an arrow between dots:

\todotext{Insert figure}


