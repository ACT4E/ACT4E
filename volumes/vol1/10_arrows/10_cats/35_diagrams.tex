% !TEX root = chapter-standalone.tex

\section{Diagrams}

\todojira{367}{@J: Write section introducing the notion of diagrams and commutative diagrams}

When working with (semi)categories, it is typical to work with diagrams that look like directed graphs, with nodes representing objects and directed arrows. Diagrams are usually used as a tool to speak and think about specific situations, where one is focusing on certain objects and morphisms. In other words, the diagrams we draw don't have a symbol for every single object and morphism in the (semi)category, but rather we just draw the ones that we want to refer to. For example, we might draw a diagram as in \cref{fig:points-and-two-arrows-diagram-again} because we are considering the morphisms $\mora$, $\morb$, and $\morc$. This diagram encodes their sources and targets, and hints at the ways that they may be composed. 

\begin{marginfigure}
    \begin{center}
        \includesag{points_and_three_arrows}
    \end{center}
    \caption{}
    \label{fig:points-and-three-arrows-diagram}
\end{marginfigure}

\subsection{Commutative diagrams}

Often, we will be interested in knowing whether two given morphisms in a homset, say $\HomSet{\CatC}{\Obja}{\Objc}$, are \emph{equal} or not. Consider for example the situation in \cref{fig:commutative-diagram-exa}

\begin{marginfigure}
    \centering
        \includesag{commutative_diagram_exa}
    \caption{}
    \label{fig:commutative-diagram-exa}
\end{marginfigure}

\todo{J: writing in progress here... }



\subsection{Semicategories vs directed graphs}

\todo{J: writing in progress here... }

One might be lead to ask: what is actually the difference between a semicategory and a directed graph? 

The basic building block of a directed graph are nodes and directed edges, and the basic building blocks of semicategories are objects and morphisms. So far, these are the same ingredients, apart from name differences. A further essential ingredient though in the definition of a semicategory is the composition operation: for any two morphisms where the target of one is the source of the other, we can compose them to obtain a further morphism (c.f. \cref{fig:points-and-composed-arrows-diagram-again}). 

\begin{marginfigure}
    \centering
        \includesag{points_and_composed_arrows}
    \caption{}
    \label{fig:points-and-composed-arrows-diagram-again}
\end{marginfigure}

One might thus say: a semicategory corresponds to a special kind of directed graph, where for any two adjacent directed edges there must exist a third edge corresponding to the ``composite'' of those edges. This is technically a true statement, however it misses the following two aspects, which are typical ways of thinking in category theory.  







\

\

\subsection{Graphical calculi}

\todo{J: writing in progress here... }


Later in the book we will also get to know another kind of diagram, called \emph{string diagrams}, that are another, and slightly different, visual tool for reasoning diagrammatically.


