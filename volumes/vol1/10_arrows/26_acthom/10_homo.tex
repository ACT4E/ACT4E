% !TEX root = chapter-standalone.tex


\section{Homomorphisms}


\begin{definition}[Semigroup homomorphism]
  \label{def:semigroup-mor}
  Let~$\sgrpA$ and~$\sgrpB$ be semigroups. A homomorphism of semigroups from~$\sgrpA$ to~$\sgrpB$ is a function~$\sgrpmorA\colon \sgrpA \sto \sgrpB$ such that
  \begin{equation}
    \label{eq:sgrp-mor-comp}
    \sgrpmorA(\sgrpelAa \mtimes_{\sgrpA} \sgrpelAb) = \sgrpmorA (\sgrpelAa) \mtimes_{\sgrpB} \sgrpmorA(\sgrpelAb) \quad \quad \forall \sgrpelAa, \sgrpelAb \in \sgrpA
  \end{equation}
  We think of the equation \cref{eq:sgrp-mor-comp} as a way of saying that the function of sets~$\sgrpmorA\colon \sgrpA \sto \sgrpB$  is \emph{compatible} with the multiplication operations on~$\sgrpA$ and$\sgrpB$, respectively.
\end{definition}

\begin{example}[Phonetic languages]
  \todotext{Nice example: the map from sequence of characters to sequences
  of sounds is monoidal in certain languages (Korean, Japanese, almost in Italian.) and
  also invertible.}
\end{example}

\begin{example}[Logarithms]
  \todotext{Example of log from $\tup{\reals, \times}$ to  $\tup{\nonNegReals, +}$. }
\end{example}


\begin{example}[State dimension of discrete dynamical systems]
  \todotext{Example of the map $f: \text{DDS} \to \natnumbers$ that gives the size of the state.}
\end{example}


\begin{example}[Transition function, continuation of \cref{exa:transition-functions}]
  Consider the map $f: \nonNegReals \to (\reals^n \to \reals^n)$ that associated to a delta $\delta$
  its transition function $T_\delta$.
  Re-reading~\cref{eq:transition-property}, we can see that it is a homomorphism between the semigroup $\tup{\nonNegReals,+}$ and the semigroup of endormorphisms of $\reals^n$.
\end{example}


\begin{definition}[Identity homomorphism]
  \label{def:identity-sgrp-mor}
  Let~$\sgrpA$ be a semigroup.
  The identity function~$\id_\sgrpA\colon \sgrpA \sto \sgrpA$ is always a morphism of semigroups.
  Indeed, the condition
  \begin{equation}
    \id (\sgrpelAa \mtimes_{\sgrpA} \sgrpelAb) = \id (\sgrpelAa) \mtimes_{\sgrpA} \id(\sgrpelAb) \quad \forall \sgrpelAa, \sgrpelAb \in \sgrpA
  \end{equation}
  is satisfied. We call this the \emph{identity homomorphism} of~$\sgrpA$.
\end{definition}

\begin{definition}[Semigroup isomorphism]
  \label{def:semigroup-iso}
  Let~$\sgrpA$ and~$\sgrpB$ be semigroups.
  A homomorphism of semigroups~$\sgrpmorA\colon \sgrpA \mto \sgrpB$ is called a \emph{semigroup isomorphism} if there exists a homomorphism of semigroups~$\sgrpmorB\colon \sgrpB \mto \sgrpA$ such that
  \begin{equation}
    \label{eq:sgrp-iso-cond}
    \sgrpmorA \then \sgrpmorB = \id_\sgrpA \text{ and }  \sgrpmorB \then \sgrpmorA = \id_\sgrpB.
  \end{equation}
\end{definition}

\begin{exercise}
  \label{ex:non-isomorphic}
  How many different non-isomorphic semigroups are there with precisely one element?
  How many with precisely two elements?
\end{exercise}
\begin{solution}
  \todotext{Write solution of \cref{ex:non-isomorphic}.}
\end{solution}

\begin{exercise}
  \label{ex:semi-morph}
  Let~$\sgrpmorA\colon \sgrpA \mto \sgrpB$ be a morphism of semigroups.
  Prove that~$\sgrpmorA$ is an isomorphism of semigroups if and only if the function~$\sgrpmorA\colon \sgrpA \sto \sgrpB$ is bijective.
\end{exercise}
\begin{solution}
  \todotext{Write solution of \cref{ex:semi-morph}.}
\end{solution}


\section{\statusmissing{Monoid homomorphism}}

\begin{definition}[Monoid homomorphism]
  \label{def:monoid-mor}
  Let~$\tup{\monoidA, \mtimes_\monoidA, 1_\monoidA}$ and $\tup{\monoidB, \mtimes_\monoidB, 1_\monoidB}$ be monoids.
  A homomorphism of monoids from~$\monoidA$ to~$\monoidB$ is a function~$\mapa \colon \monoidA \sto \monoidB$ such that
  \begin{equation}
    \label{mon-mor-comp}
    \mapa (\monelAa \mtimes_{\monoidA} \monelAb) = \mapa (\monelAa) \mtimes_{\monoidB}  \mapa(\monelAb) \quad \forall \monelAa, \monelAb \in \monoidA
  \end{equation}
  and
  \begin{equation}
    \label{mon-id-comp}
    \mapa (1_\monoidA) = 1_{\monoidB}
  \end{equation}
\end{definition}

\begin{example}
  The set~$\monoidA = \{ -1, 0, 1 \}$, together with multiplication of whole numbers and with~$1$ as neutral element, forms a monoid. The inclusion map~$\monoidA \sto \wnumbers$ is a homomorphism of monoids.
\end{example}

\todo{Nice example: the map from sequence of characters to sequences
of sounds is monoidal in certain languages (Korean, Japanese, almost in Italian.) and
also invertible.}


\begin{definition}[Identity homomorphism]
  \label{def:identity-mon-mor}
  Let~$\monoidA$ be a monoid. Similar to the case of semigroups, the identity function~$\id_\monoidA\colon \monoidA \sto \monoidA$ is also a homomorphisms of monoids.
\end{definition}



\begin{definition}[Monoid isomorphism]
  \label{def:monoid-iso}
  A homomorphism of semigroups~$\mora\colon \monoidA \mto \monoidB$ is called a \emph{monoid isomorphism} if there exists a homomorphism of monoids~$\morb\colon \monoidB \sto \monoidA$ such that
  \begin{equation}
    \mora \then \morb = \id_\monoidA \text{ and } \morb \then \mora = \id_\monoidB.
  \end{equation}
\end{definition}


\begin{exercise}
  Prove: a morphism of monoids~$\mora\colon \monoidA \mto \monoidB$ is an isomorphism of monoids if and only if the function~$\mora\colon \sgrpA \sto \sgrpB$ is bijective.
\end{exercise}
\begin{solution}
  \todotext{Write solution}
\end{solution}


\section{Group homomorphism}


\todotext{Show the inverse operation being compatible with group structure, commuting with homomorphisms. This is the simplest
example of a dagger category, to be explained later on.}


