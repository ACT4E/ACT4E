% !TEX root = chapter-standalone.tex

\section{A different poset of intervals}
\begin{definition}
    [Poset of intervals $\posintbis\posA$]
    \label{def:second_interval_poset}
    We define a different poset of intervals on the poset $\posA$ by setting the order:
    \begin{equation}
        \prfdoubleperiod{
            \interv \posela \poselb
            \posleqof{\posintbis{\posA}}
            \interv \poselc \poseld
        }{
            (\posela \posleqof\posA \poselc)
            \booland
            (\poselb \posleqof\posA \poseld)
        }
        %\tupp{\posAel_1,\posAel_2}\posleqof{\posintbis{P}}\tupp{\posBel_1,\posBel_2} \Leftrightarrow (\posAel_1\posleqof\posA \posBel_1) \wedge (\posAel_2\posleqof\posA \posBel_2).
    \end{equation}
\end{definition}
This is similar to taking the product of $\posA$ with itself;
however we are only considering intervals, so we obtain a subset of $\posA \cartprod \posA$:
\begin{equation}
    \posintbis\posA \subseteq   \posA \cartprod \posA.
\end{equation}

This partially ordered set will be instrumental when we define uncertainty in design problems.

\begin{exercise}
    Check that the relation defined in \cref{def:second_interval_poset} is indeed a poset.
\end{exercise}
\begin{solution}
    We check the three conditions.
    \begin{itemize}
        \item First, we know that~$\tupp{\posAel_1,\posBel_1}\posleqof{\posintbis{\posA}}\tupp{\posAel_1,\posBel_1}$, since~$\posAel_1\posAleq \posAel_1$ and~$\posBel_1\posAleq \posBel_1$.
        \item Second,~$\tupp{\posAel_1,\posBel_1}\posleqof{\posintbis{\posA}}\tupp{\posAel_2,\posBel_2}$ and~$\tupp{\posAel_2,\posBel_2}\posleqof{\posintbis{\posA}}\tupp{\posAel_3,\posBel_3}$ imply
              \begin{equation}
                  \tupp{\posAel_1,\posBel_1}\posleqof{\posintbis{\posA}}\tupp{\posAel_3,\posBel_3}.
              \end{equation}
        \item Third, if~$\tupp{\posAel_1,\posBel_1}\posleqof{\posintbis{\posA}}\tupp{\posAel_2,\posBel_2}$ and~$\tupp{\posAel_2,\posBel_2}\posleqof{\posintbis{\posA}}\tupp{\posAel_1,\posBel_1}$, then~$\posAel_1=\posAel_2$ and~$\posBel_1=\posBel_2$.
    \end{itemize}
\end{solution}

\todotextjira{472}{@Gioele: Missing a lot of stuff here... not even a figure about these intervals?}
