% !TEX root = chapter-standalone.tex

\section{Disjoint union of posets}
\linkvideo{spring2021-tradeoffs:tradeoffs:orders:composing-posets:disj-poset} % Disjoint union of posets
Similarly to what we have done for sets in \cref{sec:coproductset}, we can think of alternatives in the poset case through their disjoint union.

\begin{definition}[Disjoint union of posets]
    Given posets~$\posA=\tupp{\posAset, \posAleq}$ and~$\posB=\tupp{\posBset, \posBleq}$, we can define their \emph{disjoint union}~$\posA+\posB=\tupp{\posAset \setdisunion \posBset, \posleqof{\posA + \posB}}$, where~$\posAset \setdisunion \posBset$ is the disjoint union of the sets~$\posAset$ and~$\posBset$ (\cref{def:disjoint-union}), and the order~$\posleqof{\posA + \posB}$ is given by:
    % \begin{equation}
    %     \posAel \posleqof{\posA + \posB} \posBel \quad\equiv\quad
    %     \begin{cases}
    %         \posAel \posAleq \posBel, & \text{if }\posAel,\posBel \in \posA, \\
    %         \posAel \posBleq \posBel, & \posAel,\posBel \in \posB,
    %          \false,  & \text{otherwise}.
    %     \end{cases}
    % \end{equation}
    % with
    \begin{equation}
        \begin{aligned}
            \posleqof{\posA+\posB}\colon
            (\posAset\setdisunion\posBset)\cartprod (\posAset
            \setdisunion\posBset)                           & \to \Bool, \\
            \tupp{\disunionA{\posela},\disunionA{\poselb} } &
            \mapsto (\posela\posAleq \poselb), \\
            \tupp{\disunionB{\cdot},\disunionA{\cdot} }     &
            \mapsto \false, \\
            \tupp{\disunionA{\cdot},\disunionB{\cdot}}      &
            \mapsto \false, \\
            \tupp{\disunionB{\posela},\disunionB{\poselb}}  &
            \mapsto (\posela\posBleq \poselb).
        \end{aligned}
    \end{equation}
\end{definition}

\begin{example}
    Consider the posets~$\posA,\posB$, over the sets~$\posAset=\tup{\sbretzel, \sfondue}$ with~$\sbretzel \posAleq \sfondue$, and~$\posBset=\tup{\stea,\smilk}$, with~$\smilk \posBleq \stea$.
    Their disjoint union can be represented as in \cref{fig:poset-coproduct}.

    \begin{figure}[h!]
        \centering
        \includesag{40_disjoint_union}
        \caption{Disjoint union of posets.}
        \label{fig:poset-coproduct}
    \end{figure}
\end{example}
\vfill
\begin{gradedexercise}
    Prove the following properties:
    \begin{enumerate}
        \item The width of the sum is the sum of the widths:
        \begin{equation}
        \posetwidth{\posA + \posB} = \posetwidth{\posA} + \posetwidth{\posB}
        \end{equation}
        \item The height of the sum is the max of the heights:
        \begin{equation}
        \posetheight{\posA + \posB} = \max(\posetwidth{\posA}, \posetwidth{\posB})
        \end{equation}
    \end{enumerate}
\end{gradedexercise}
\solutionof{MeasurePosetSum}