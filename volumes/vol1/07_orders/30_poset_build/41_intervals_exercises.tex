\sectionexercises{Twisted poset construction}
\codeboilerplate{FinitePosetConstructionTwisted}{

    Write code to compute the poset of intervals of a poset according to the twisted interval
    construction.
}

\begin{widepar}
    \aligninner{%
        \begin{minipage}{15cm}
            \classlisting{FinitePosetConstructionTwisted}

            \classlisting{PosetOfIntervals}
            \classlisting{FinitePosetOfIntervals}

        \end{minipage}
    }
\end{widepar}

The code requires that you produce a \classname{PosetOfIntervals}.
This is a special poset for which we can create elements by naming the two boundary points, and from an element, we can extract the two boundary points.

