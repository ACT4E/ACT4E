% !TEX root = chapter-standalone.tex

\section{Disjoint union of posets}
\linkvideo{spring2021-tradeoffs:tradeoffs:orders:composing-posets:disj-poset} % Disjoint union of posets

Following the pattern, we can define the disjoint union of \SY{poset} as a \SY{poset} with the underlying set being the \SY{disjoint union} of the underlying sets (\cref{sec:coproducts}).
\todotext{\bernina: after adding disjoint union of relations, then just use $\posAleq+\posBleq$}
\begin{ctdefinition}[Disjoint union of posets]
    \label{def:disjoint-union-of-posets}
    \SYNDEF{disjoint union of posets}
    Given \SY{posets}~$\posAdefinition$ and~$\posBdefinition$, their \emph{disjoint union}~$\posA\Pplus\posB=\tupp{\posAset \setdisunion \posBset, {{\posleqof{\posA \Pplus \posB}}}}$, is the set $\posAset \setdisunion \posBset$ equipped with the order~$\posleqof{\posA \Pplus \posB}$ given by
    %
    \begin{equation} \label{eq:poset-disunion}
        \begin{aligned}
            \posleqof{\posA\Pplus\posB}\colon
            (\posAset\setdisunion\posBset)\cartprod (\posAset
            \setdisunion\posBset)                           & \sto \boolset, \\
            \tupp{\disunionA{\posela},\disunionA{\poselb} } &
            \mapsto (\posela\posAleq \poselb), \\
            \tupp{\disunionB{\cdot},\disunionA{\cdot} }     &
            \mapsto \false, \\
            \tupp{\disunionA{\cdot},\disunionB{\cdot}}      &
            \mapsto \false, \\
            \tupp{\disunionB{\posela},\disunionB{\poselb}}  &
            \mapsto (\posela\posBleq \poselb).
        \end{aligned}
    \end{equation}
\end{ctdefinition}

The expression \cref{eq:poset-disunion} can be intimidating at first, but all it is saying is that the order relation of the \SY{disjoint union} is obtained by stitching together the two order relations.
No element of $\posAset$ is related to an element of $\posBset$, and vice versa.

\begin{example}
    Consider the \SY{posets}~$\posA,\posB$, over the sets~$\posAset=\tup{\sbretzel, \sfondue}$ with~$\sbretzel \posAleq \sfondue$, and~$\posBset=\tup{\stea,\smilk}$, with~$\smilk \posBleq \stea$.
    Their \SY{disjoint union} can be represented as in \cref{fig:poset-coproduct}.
    %
    \begin{figure}[h!]
        \centering
        \includesag{40_disjoint_union}
        \caption{Disjoint union of \SY{posets}.}
        \label{fig:poset-coproduct}
    \end{figure}
\end{example}
\vfill
\begin{gradedexercise}[\exname{MeasurePosetSum}]
    \label{ex:MeasurePosetSum}
    Prove the following properties:
    \begin{enumerate}
        \item The width of the sum is the sum of the widths:
              \begin{equation}\label{eq:MeasurePosetSum-1}
                  \posetwidth(\posA \Pplus \posB) = \posetwidth(\posA) + \posetwidth(\posB).
              \end{equation}
        \item The height of the sum is the maximum of the heights:
              \begin{equation}\label{eq:MeasurePosetSum-2}
                  \posetheight(\posA \Pplus \posB) = \max(\posetheight(\posA), \posetheight(\posB)).
              \end{equation}
    \end{enumerate}
\end{gradedexercise}
\solutionof{MeasurePosetSum}
