% !TEX root = chapter-standalone.tex

\section{Arrow poset of intervals}

We can order intervals in a second way, which we call ``arrow construction''.

\begin{marginfigure}
    \centering
    %\includesag{second_poset_intervals}
    \includesag{arr_pos_fig}
    \caption{
        $\interv \posela \poselb
            \posleqof{\posintbis{\posA}}
            \interv \poselc \poseld$
    }
\end{marginfigure}

\begin{definition}[``Arrow'' poset of intervals $\posintbis\posA$]
    \label{def:second_interval_poset}
    \SYNDEF{arrow poset of intervals}
    We define an ``Arrow'' \SY{poset} of intervals on the \SY{poset} \posA by setting the order:
    \begin{equation}
        \prfdoubleperiod{
            \interv \posela \poselb
            \posleqof{\posintbis{\posA}}
            \interv \poselc \poseld
        }{
            (\posela \posleqof\posA \poselc)
            \booland
            (\poselb \posleqof\posA \poseld)
        }
    \end{equation}
\end{definition}
This is similar to taking the product of~\posA with itself; however, we are only considering intervals, so we obtain a subposet of~$\posA \Ptimes \posA$.
% \begin{equation}
%     \posintbis\posA \setsubseteq   \posA \Ptimes \posA.
% \end{equation}

% This partially ordered set will be instrumental when we define uncertainty in \SY{design problems}.

\begin{exercise}
    Check that the relation defined in \cref{def:second_interval_poset} is indeed a poset.
\end{exercise}
\begin{solution}
    We check the three conditions.
    \begin{itemize}
        \item First, we know that~$\interv{\posAnel{1}}{\posBnel{1}}\posleqof{\posintbis{\posA}}\interv{\posAnel{1}}{\posBnel{1}}$, since~$\posAnel{1}\posAleq \posAnel{1}$ and~$\posBnel{1}\posAleq \posBnel{1}$.
        \item Second,~$\interv{\posAnel{1}}{\posBnel{1}}\posleqof{\posintbis{\posA}}\interv{\posAnel{2}}{\posBnel{2}}$ and~$\interv{\posAnel{2}}{\posBnel{2}}\posleqof{\posintbis{\posA}}\interv{\posAnel{3}}{\posBnel{3}}$ imply
              \begin{equation}
                  \interv{\posAnel{1}}{\posBnel{1}}\posleqof{\posintbis{\posA}}\interv{\posAnel{3}}{\posBnel{3}}.
              \end{equation}
        \item Third, if~$\interv{\posAnel{1}}{\posBnel{1}}\posleqof{\posintbis{\posA}}\interv{\posAnel{2}}{\posBnel{2}}$ and~$\interv{\posAnel{2}}{\posBnel{2}}\posleqof{\posintbis{\posA}}\interv{\posAnel{1}}{\posBnel{1}}$, then~$\posAnel{1}=\posAnel{2}$ and~$\posBnel{1}=\posBnel{2}$.
    \end{itemize}
\end{solution}
