% !TEX root = chapter-standalone.tex

\section{Poset of intervals}
\linkvideo{spring2021-tradeoffs:tradeoffs:orders:interval-poset} % Poset of intervals
\begin{definition}[Interval]
    \label{def:interval}
    An \emph{interval} of a poset~$\posA$ is a subset of~$\posA$ bounded above and below by two elements.
    We write
    \begin{equation}
        \interv \posela \poselb
        \definedas
        \makeset{
            \poselc \in \posAset :
            \posela \posAleq \poselc \posAleq \poselb
        }.
    \end{equation}
\end{definition}

\begin{definition}
    [Poset of intervals $\posint\posA$]
    \label{def:poset_intervals}
    Given a poset~$\posA$, one can define a \emph{poset of intervals}~$\posint\posA$
    by ordering the intervals by reverse inclusion:
    %
    \begin{equation}
        \prfdoubleperiod{
            \interv \posela \poselb
            \posleqof{\posint{\posA}}
            \interv \poselc \poseld
        }{
            \interv \posela \poselb
            \supseteq
            \interv \poselc \poseld
        }
    \end{equation}
    Equivalently we only need check the bounds:
    \begin{equation}
        \prfdoubleperiod{
            \interv \posela \poselb
            \posleqof{\posint{\posA}}
            \interv \poselc \poseld
        }{
            (\posela \posAleq \poselc ) \booland (\poseld \posAleq \poselb)
        }
    \end{equation}
    Note that
    \begin{align}
        % \top_{\posint \posA} & = \emptyset, \\
        \bot_{\posint \posA} & = \posAset.
    \end{align}
\end{definition}
\begin{exercise}
    Check that the relation defined in \cref{def:poset_intervals} is indeed a poset.
\end{exercise}
\begin{solution}
    We check the three conditions.
    \begin{itemize}
        \item First, we know that~$\tup{\posAel_1,\posBel_1}\posleqof{\posint{\posA}}\tup{\posAel_1,\posBel_1}$, since $\posAel_1\posAleq \posAel_1$ and $\posBel_1\posAleq \posBel_1$.
        \item Second,~$\tup{\posAel_1,\posBel_1}\posleqof{\posint{\posA}}\tup{\posAel_2,\posBel_2}$ and $\tup{\posAel_2,\posBel_2}\posleqof{\posint{\posA}}\tup{\posAel_3,\posBel_3}$ imply
              \begin{equation}
                  \tup{\posAel_1,\posBel_1}\posleqof{\posint{\posA}}\tup{\posAel_3,\posBel_3}.
              \end{equation}
        \item Third, if~$\tup{\posAel_1,\posBel_1}\posleqof{\posint{\posA}}\tup{\posAel_2,\posBel_2}$ and $\tup{\posAel_2,\posBel_2}\posleqof{\posint{\posA}}\tup{\posAel_1,\posBel_1}$, then~$\posAel_1=\posAel_2$ and~$\posBel_1=\posBel_2$.
    \end{itemize}
\end{solution}
\vfill
\clearpage
\subsection{Set-based filtering}
\linkvideo{spring2021-tradeoffs:tradeoffs:orders:set-based-filtering} % Set-based filtering
\todojira{458}{Expand.
    Need cute pictures of dynamics.
}
We now look at an example of \textbf{set-based filtering}, where filtering refers to online inference.
Suppose that we want to track the value of a quantity~$x\in [0,100]$, without having \emph{a priori} information about~$x$.
We are equipped with sensors, which periodically measure the quantity~$x$ with some variable precision.
At time~$t\in \nonNegReals $ they produce an \emph{observation}~$y_t\colon x_t\in [l_t,u_t]$.
Also, note that the quantity fluctuates randomly, and we bound its ``velocity'' to be~$\dot{x}_t\in [-1,1]$ (except at boundaries).
At the beginning, our information state~$\bar{i}_0$ could be that~$x\in [0,100]$.
At time 0, we get an observation~$y_0$, that says~$x\in [21,24]$.
The new information state can be obtained by ``fusing'' the two inputs we have received about~$x$.
This corresponds to the intersection
\begin{equation*}
    x\in \left( \interv{0}{100} \setintersection \interv{21}{24}\right)\equiv x\in \interv{21}{24}.
\end{equation*}
Let's now say we get an observation~$y_1$ which says~$x\in \interv{19}{22}$.
We now need to take into account the evolution/dynamics of the quantity we are tracking.
From the interval~$\interv{21}{24}$ we know that the variable could have evolved in~$\interv{20}{25}$ (dynamics are bounded with a unit increase/decrease).
Therefore, the new information state is given by
\begin{equation*}
    x\in \left( \interv{20}{25} \setintersection \interv{21}{24}\right)\equiv x\in \interv{21}{24}.
\end{equation*}
One of the structures which could sustain this kind of inference, is the of \emph{posets of intervals} (\cref{def:poset_intervals}).

The Hasse diagram representing a situation related to this diagram could be as reported in \cref{fig:hasse_filtering}.
\begin{figure}[h!]
    \centering
    \includesag{080_hasse_filtering}
    \caption{}
    \label{fig:hasse_filtering}
\end{figure}
