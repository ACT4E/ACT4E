% !TEX root = chapter-standalone.tex

\section{Opposite of a poset}\label{sec:opposite-of-a-poset}

\begin{ctdefinition}[Opposite of a poset]
    \label{def:poset-opposite}
    The \maindef{opposite of a poset}~$\posAdefinition$ is the \SY{poset} denoted~$\posAop=\tupp{\posAset, \posAleqop}$.
    It has the same elements as~\posA, but is equipped with the reverse ordering, in the sense that, for all~$\posela,\poselb \setin \posAset$,
    \begin{equation}\label{eq:poset-opposite}
        \prfdoubleperiod{
            \posela \posAleq \poselb
        }{
            \poselb \posAleqop \posela
        }
    \end{equation}
\end{ctdefinition}
For a given~$\posela \setin \posAset$, we will sometimes write~$\posela^*$ do denote its corresponding copy in~$\posAop$, in order to emphasize that~$\posela$ and~$\posela^*$ belong to distinct \SY{posets}.
However, often we will not be so pedantic with our notation.

\begin{figure}[tbh]
    \centering
    \includesag{40_dpcatfig_opposite}
    \caption{
        Opposite of a poset.
    }
    \label{fig:poset-opposite}
\end{figure}

\begin{example}[Credit and debt]
    Let us define the set
    %
    \begin{equation}\label{eq:credit-debit}
        \posAset=\makeset{0.00,0.01,0.02,\dots}\setsubseteq\reals
    \end{equation}
    %
    of all \CHFneutral \ monetary quantities approximated to the cent.
    From this set we can define two \SY{posets},~$\posA^{+} = \tup{\posAset, {{\Rleq}}}$ and~$\posA^{-} = \tup{\posAset,{{ \Rgeq}}}$, that are the opposite of each other.
    If the context is that, given two quantities \unit[1]{\CHFneutral} and \unit[2]{\CHFneutral}, we prefer \unit[1]{\CHFneutral} to \unit[2]{\CHFneutral} (for example because it is a cost to pay to acquire a component), then we are working in~$\posA^{+}$, otherwise we are working in~$\posA^{-}$ (for example because it represents the price at which we are selling our product).
    Traditionally, in double-entry ledger systems, the numbers were not written with negative signs, but rather in color: red and black.
    From this convention we get the idioms ``being in the black'' and ``being in the red''.
    \todotext{\bernina: well,  would be nice to have one poset red and the other black to go along with the story}
\end{example}
