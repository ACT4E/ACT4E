% !TEX root = chapter-standalone.tex

\section{Opposite of a poset}\label{sec:opposite-of-a-poset}

\begin{definition}[Opposite of a poset]
	\label{def:poset-opposite}
	The \emph{opposite} of a poset~$\posA=\tupp{\posAset, \posAleq}$ is the poset denoted~$\posA\op=\tupp{\posAset, \posAleqop}$.
	It has the same elements as~$\posA$, but is equipped with the reverse ordering (\cref{fig:poset-opposite}).
	For a given~$\posela \in \posA$, we will sometimes write~$\posela^*$ do denote its corresponding copy in~$\posA\op$, in order to emphasize that~$\posela$ and~$\posela^*$ belong to distinct posets.
	However, often we will not be so pedantic with our notation.
	Reversing the order means that, for all $\posela,\poselb \in \posA$,
	\begin{equation}
		\prfperiod{
			\posela \posAleq \poselb
		}{
			\poselb^* \posAleqop \posela^*
		}
	\end{equation}
\end{definition}

\begin{figure}[tbh]
	\centering
	\includesag{40_dpcatfig_opposite}
	\caption{
		Opposite of a poset.
	}
	\label{fig:poset-opposite}
\end{figure}

\begin{example}[Credit and debt]
	Let us define the set
	\begin{equation*}
		\posAset=\reals=\{0.00,0.01,0.02,\dots\}
	\end{equation*}
	of all \CHFneutral \ monetary quantities approximated to the cent.
	From this set we can define two posets, $\posA^{+} = \tup{\posAset, \leq}$ and~$\posA^{-} = \tup{\posAset, \geq}$, that are the opposite of each other.
	If the context is that, given two quantities~\unit[1]{\CHFneutral} and \unit[2]{\CHFneutral}, we prefer \unit[1]{\CHFneutral} to \unit[2]{\CHFneutral} (for example because it is a cost to pay to acquire a component), then we are working in~$\posA^{+}$, otherwise we are working in~$\posA^{-}$ (for example because it represents the price at which we are selling our product).
	Traditionally, in double-entry ledger systems, the numbers were not written with negative signs, but rather in color: red and black.
	From this convention we get the idioms ``being in the black'' and ``being in the red''.
\end{example}
