% !TEX root = chapter-standalone.tex

\section{Product of posets}
%\linkvideo{spring2021-tradeoffs:tradeoffs:orders:composing-posets} % Composing posets
\linkvideo{spring2021-tradeoffs:tradeoffs:orders:composing-posets:product-poset} % Product of posets
We can think of the product of posets.

\begin{definition}[Product of posets]
    \label{def:productposet}
    Given two posets~$\posA=\tupp{\posAset, \posAleq}$ and~$\posB=\tupp{\posBset, \posBleq}$, the \emph{product poset} is
    \begin{equation}
        \posA\Ctimes \posB=\tup{\posAset \cartprod \posBset, \ordleq_{\posA\Ctimes \posB}},
    \end{equation}
    where~$\posAset \cartprod \posBset$ is the Cartesian product of the sets~$\posAset$ and $\posBset$, and the order~$\posleqof{\posA \Ctimes \posB}$ is given by:
    \begin{equation*}
        \prfdoubleperiod{\tup{\posAnel{1}, \posBnel{1}}
            \posleqof{\posA\Ctimes \posB}
            \tup{\posAnel{2}, \posBnel{2}}}{(\posAnel{1} \posAleq \posAnel{2}) \booland
            (\posBnel{1} \posBleq \posBnel{2})}
    \end{equation*}
\end{definition}
Recalling the battery choice example, we have the two posets representing time and money.
Given that we want to minimize both time and costs, by considering the money poset containing elements ``\poscheap'', ``\posmidrange'', and ``\posexpensive'', and the weight poset containing elements ``\poslight'', and ``\posheavy'', one can represent the product as in~\cref{fig:productpizza}.

\begin{figure*}[h!]
    \centering
    \aligninner{\includesag{70_hasse_pizza_product}}
    \caption{Product poset of time and weight for battery choices.}
    \label{fig:productpizza}
\end{figure*}

\begin{example}
    Consider now two posets and their product, given in~\cref{fig:composing_posets_1}.
    \begin{figure*}[h!]
        \aligninner{
            \includesag{40_exposet_1_1}
        }
        \caption{Product of two posets.}
        \label{fig:composing_posets_1}
    \end{figure*}
\end{example}
\vfill
\clearpage

\subsection{Measuring the product}
If we know the height/width of the posets~$\posA$ and~$\posB$, it is not possible to know exactly the height/width of~$\posA\Ctimes\posB$.
However, we can find some bounds.
\begin{widepar}
    \begin{lemma}
        \label{lem:measuring-product}
        If~$\posA,\posB$ are non-empty posets, then
        \begin{align}
            \posetwidth(\posA) \cdot \posetwidth(\posB)
             & \leq \posetwidth(\posA\Ctimes \posB)
            \leq \min \makeset{
                \vert \posAset \vert \cdot \posetwidth(\posB),\vert \posBset \vert \cdot \posetwidth(\posA)
            }, \label{eq:bound-width} \\
            \posetheight(\posA)+\posetheight(\posB)-1
             & \leq
            \posetheight(\posA\Ctimes \posB).
            \label{eq:bound-height}
        \end{align}
        Moreover, these bounds are tight.
    \end{lemma}
\end{widepar}

\begin{proof}
    The bound \cref{eq:bound-width} can be found in ~\cite{bezrukovantichains}.
    As for \cref{eq:bound-height}, we have the following proof.
    First, one can construct the longest chain in~$\posA$:
    \begin{equation*}
        \setA=\makeset{\posAnel{1},\ldots, \posAnel{\posetheight(\posA)}}.
    \end{equation*}
    Furthermore, one can construct the longest chain in~$\posB$:
    \begin{equation*}
        \setB=\makeset{\posBel_1,\ldots, \posBnel{\posetheight(\posB)}}.
    \end{equation*}
    Out of them, one can construct the chain
    \begin{equation*}
        \setC=\makeset{ \tup{\posAnel{1},\posBnel{1}},\tup{\posAnel{2},\posBnel{1}},\ldots, \tup{\posAnel{\posetheight(\posA)}, \posBnel{1}}, \tup{\posAnel{\posetheight(\posA)}, \posBnel{2}},\ldots},
    \end{equation*}
    which has height~$\posetheight(\posA)+\posetheight(\posB)-1$.
    So we know that at least~$\posetheight(\posA\Ctimes \posB)\geq \posetheight(\posA)+\posetheight(\posB)-1$.
    Now, consider a chain~$\makeset{\tup{\posAnel{1},\posBnel{1}},\ldots, \tup{\posAnel{n},\posBnel{n}}}$ in~$\posA\Ctimes \posB$.
    In general, this means that at least a coordinate of~$\tup{\posAnel{i},\posBnel{i}}$ must increase in~$\tup{\posAnel{i+1},\posBnel{i+1}}$.
    The first coordinate can only increase~$\posetheight(\posA)-1$ times, and the second one~$\posetheight(\posB)-1$ times.
    Summing up, the total number of elements in the chain is \emph{at most}~$\posetheight(\posA)+\posetheight(\posB)-1$.
    Note that this result holds only assuming that~$\posA$ and~$\posB$ are not empty (for that case,~$\posetheight(\posA\Ctimes \posB)=0$).
\end{proof}
