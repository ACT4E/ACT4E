% !TEX root = chapter-standalone.tex

\section{Product of posets}
%\linkvideo{spring2021-tradeoffs:tradeoffs:orders:composing-posets} % Composing posets
\linkvideo{spring2021-tradeoffs:tradeoffs:orders:composing-posets:product-poset} % Product of posets
Just like the product of sets, we can construct the product of posets.
That is a poset with the underlying set being the product of the underlying sets.

\begin{definition}[Product of posets]
    \label{def:productposet}
    Given posets~$\posA=\tupp{\posAset, \posAleq}$ and~$\posB=\tupp{\posBset, \posBleq}$, the \emph{product poset}
    \begin{equation}\label{eq:productposet-1}
        \posA\Ctimes \posB=\tup{\posAset \cartprod \posBset, \ordleq_{\posA\Ctimes \posB}},
    \end{equation}
    is the set~$\posAset \cartprod \posBset$ equipped with the order~$\posleqof{\posA \Ctimes \posB}$ given by
    \begin{equation}\label{eq:productposet-2}
        \prfdoubleperiod{
            \tup{\posAnel{1}, \posBnel{1}}
            \posleqof{\posA\Ctimes \posB}
            \tup{\posAnel{2}, \posBnel{2}}
        }{
            (\posAnel{1} \posAleq \posAnel{2})
            \booland
            (\posBnel{1} \posBleq \posBnel{2})
        }
    \end{equation}
\end{definition}

Recalling the battery choice example, we have the two posets representing time and money.
Given that we want to minimize both time and costs, by considering the money poset containing elements ``\poscheap'', ``\posmidrange'', and ``\posexpensive'', and the weight poset containing elements ``\poslight'', and ``\posheavy'', we can represent the product as in~\cref{fig:productpizza}.

\begin{figure*}[h!]
    \centering
    \aligninner{\includesag{70_hasse_pizza_product}}
    \caption{Product poset of time and weight for battery choices.}
    \label{fig:productpizza}
\end{figure*}

\begin{example}
    Consider now two posets and their product, given in~\cref{fig:composing_posets_1}.
    \begin{figure*}[h!]
        \aligninner{
            \includesag{40_exposet_1_1}
        }
        \caption{Product of two posets.}
        \label{fig:composing_posets_1}
    \end{figure*}
\end{example}
\vfill
\clearpage

\subsection{Measuring the product}
The following lemma gives expressions for the width and height of the product of two posets.

\begin{widepar}
    \begin{lemma}
        \label{lem:measuring-product}
        If~$\posA,\posB$ are non-empty finite posets, then we know the height of their product:
        \begin{equation}\label{eq:bound-height}
            \posetheight(\posA\Ctimes \posB) = \posetheight(\posA)+\posetheight(\posB)-1
        \end{equation}
        We can derive this bound for the width of the product:
        \begin{equation}\label{eq:bound-width}
            \posetwidth(\posA) \cdot \posetwidth(\posB)
            \leq \posetwidth(\posA\Ctimes \posB)
            \leq \min \makeset{
                \cardof \posAset  \cdot \posetwidth(\posB),\cardof \posBset   \cdot \posetwidth(\posA)
            }.
        \end{equation}
        This bound is tight, in the sense that there exist posets that reach this bound.
    \end{lemma}
\end{widepar}

\begin{proof}
    The bound \cref{eq:bound-width} can be found in ~\cite{bezrukovantichains}.
    As for \cref{eq:bound-height}, we have the following proof.
    First, we can construct the longest \SY{chain}  in~\posA:
    \begin{equation}
        \setA=\makeset{\posAnel{1},\ldots, \posAnel{\posetheight(\posA)}}.
    \end{equation}
    Furthermore, we can construct the longest \SY{chain}  in~\posB:
    \begin{equation}
        \setB=\makeset{\posBel_1,\ldots, \posBnel{\posetheight(\posB)}}.
    \end{equation}
    Out of them, we can construct the \SY{chain}
    \begin{equation}
        \setC=\makeset{ \tup{\posAnel{1},\posBnel{1}},\tup{\posAnel{2},\posBnel{1}},\ldots, \tup{\posAnel{\posetheight(\posA)}, \posBnel{1}}, \tup{\posAnel{\posetheight(\posA)}, \posBnel{2}},\ldots},
    \end{equation}
    which has height~$\posetheight(\posA)+\posetheight(\posB)-1$.
    So we know a lower bound for the height:
    \begin{equation}
        \posetheight(\posA\Ctimes \posB)\geq \posetheight(\posA)+\posetheight(\posB)-1.
    \end{equation}
    Now, consider a \SY{chain} ~$\makeset{\tup{\posAnel{1},\posBnel{1}},\ldots, \tup{\posAnel{n},\posBnel{n}}}$ in~$\posA\Ctimes \posB$.
    In general, this means that at least a coordinate of~$\tup{\posAnel{i},\posBnel{i}}$ must increase in~$\tup{\posAnel{i+1},\posBnel{i+1}}$.
    The first coordinate can only increase~$\posetheight(\posA)-1$ times, and the second one~$\posetheight(\posB)-1$ times.
    Summing up, the total number of elements in the \SY{chain}  is \emph{at most} $\posetheight(\posA)+\posetheight(\posB)-1$:
    \begin{equation}
        \posetheight(\posA\Ctimes \posB)\leq \posetheight(\posA)+\posetheight(\posB)-1.
    \end{equation}
    Because upper and lower bounds are the same, we have an exact expression for the height.
    Note that this result holds only assuming that~\posA and~\posB are not empty (for that case,~$\posetheight(\posA\Ctimes \posB)=0$).
\end{proof}
