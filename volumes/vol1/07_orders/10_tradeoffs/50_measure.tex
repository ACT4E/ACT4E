% !TEX root = chapter-standalone.tex

\section{Measuring posets}
We can define two measurements for a poset: the height and the width.
These measurements allow to quantify the performance of several algorithms we will see in the latter parts of the book.
\begin{definition}[Width of a poset]
    \label{def:poset-width}
    The \maindef{width of a poset}, denoted~$\posetwidth(\posA)$, is the maximum cardinality of an \SY{antichain} in~\posA.
\end{definition}

\begin{definition}[Height of a poset]
    \label{def:poset-height}
    The \maindef{height of a poset}, denoted~$\posetheight(\posA)$, is the maximum cardinality of a \SY{chain} in~\posA.
\end{definition}

Note that an empty \SY{poset} has exactly one \SY{chain} and one \SY{antichain}: the empty set.
Therefore, the height and width are zero.

\begin{example}
    Consider the poset~\posA in \cref{fig:poset-height-width}.
    The longest \SY{antichains} of~\posA are~$\makeset{\sapple, \sburger, \scheese}$, $\makeset{\sapple, \schoco, \scheese}$, $\makeset{\sbretzel, \schoco, \scheese}$, $\makeset{\sapple, \schoco, \sgrapes}$, $\makeset{\sbretzel, \schoco, \sgrapes}$, and~$\makeset{\sbretzel, \sburger, \scheese}$.
    Therefore,
    \begin{equation}\label{eq:poset-height-width-1}
        \posetwidth(\posA)=3.
    \end{equation}
    The longest \SY{chain} in the \SY{poset} is given by~$\makeset{\scarrot,\schoco,\sburger,\sfondue}$, and therefore
    \begin{equation}\label{eq:poset-height-width-2}
        \posetheight(\posA)=4.
    \end{equation}
\end{example}

\begin{figure*}[h]
    \aligninner{
        \subfloat[
            \label{fig:poset-height-width}
        ]{
            \includesag{height_width_posets}
        }
        \hspace{3em}
        \subfloat[
            \label{fig:poset-height-width2}
            The longest \SY{chain} ]{
            \includesag{height_width_posets2}
        }
        \hspace{3em}
        \subfloat[
            \label{fig:poset-height-width3}
            One of the largest \SY{antichains} ]{
            \includesag{height_width_posets3}
        }
    }
    \caption{Example for height and width of a poset.
    }
\end{figure*}

\vfill
\begin{gradedexercise}[\exname{MeasurePowerPoset}]
    \label{ex:MeasurePowerPoset}

    Let~\setA be a finite set with~$n$ elements.
    Obtain an expression (without proof) for
    \begin{enumerate}
        \item $\posetwidth(\powerset \setA)$;
        \item $\posetheight(\powerset \setA)$.
    \end{enumerate}
\end{gradedexercise}

\solutionof{MeasurePowerPoset}
