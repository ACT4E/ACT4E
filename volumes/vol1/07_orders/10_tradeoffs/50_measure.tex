% !TEX root = chapter-standalone.tex
\section{Measuring posets}
We can define two measurements for a poset: the height and the width.
These measurements allow to quantify the performance of several algorithms we will see in the latter parts of the book.
\begin{definition}[Width of a poset]
    \label{def:poset-width}
    The \emph{width} of a poset, denoted~$\posetwidth(\posA)$, is the maximum cardinality of an antichain in~$\posA$.
\end{definition}

\begin{definition}[Height of a poset]
    \label{def:poset-height}
    The \emph{height} of a poset, denoted~$\posetheight(\posA)$, is the maximum cardinality of a chain in~$\posA$.
\end{definition}

An empty poset does not have any chain or antichain; in that case we say that the height and width are zero.

\begin{example}
    Consider the poset~$\posA$ reported in \cref{fig:poset-height-width}.
    The longest antichains of~$\posA$ are~$\makeset{\sapple, \schoco, \scheese}$,~$\makeset{\sbretzel, \schoco, \scheese}$,~$\makeset{\sapple, \schoco, \sgrapes}$,~$\makeset{\sbretzel, \schoco, \sgrapes}$, and~$\makeset{\sbretzel, \sburger, \scheese}$.
    Therefore,~$\posetwidth(\posA)=3$.
    The longest chain in the poset is given by~$\makeset{\scarrot,\schoco,\sburger,\sfondue}$, and therefore~$\posetheight(\posA)=4$.
\end{example}

\begin{figure*}[h]
    \aligninner{
        \subfloat[
            \label{fig:poset-height-width}
        ]{
            \includesag{height_width_posets}
        }
        \hspace{3em}
        \subfloat[
            \label{fig:poset-height-width2}
            The longest chain
        ]{
            \includesag{height_width_posets2}
        }
        \hspace{3em}
        \subfloat[
            \label{fig:poset-height-width3}
            One of the largest antichains
        ]{
            \includesag{height_width_posets3}
        }
    }
    \caption{Example for height and width of a poset.
    }
\end{figure*}

\vfill
\begin{gradedexercise}[\exname{MeasurePowerPoset}]
\label{ex:MeasurePowerPoset}

    Let~$\setA$ be a finite set with $n$ elements.
    Obtain an expression (without proof) for
    \begin{enumerate}
        \item $\posetwidth(\powerset  \setA)$;
        \item $\posetheight(\powerset \setA)$.
    \end{enumerate}
\end{gradedexercise}

\solutionof{MeasurePowerPoset}