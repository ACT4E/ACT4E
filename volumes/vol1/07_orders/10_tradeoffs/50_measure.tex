\section{Measuring posets}
\todotextjira{494}{@AC: Missing intro}
\begin{definition}[Width of a poset]
    \label{def:poset-width}
    The \emph{width} of a poset, denoted~$\posetwidth(\posA)$, is the maximum cardinality of an antichain in~$\posA$.
\end{definition}

\begin{definition}[Height of a poset]
    \label{def:poset-height}
    The \emph{height} of a poset, denoted~$\posetheight(\posA)$, is the maximum cardinality of a chain in~$\posA$.
\end{definition}

\begin{marginfigure}
    \centering
    \includesag{height_width_posets}
    \caption{Example for height and width of a poset.}
    \label{fig:poset-height-width}
\end{marginfigure}

\begin{example}
    Consider the poset~$\posA$ reported in \cref{fig:poset-height-width}.
    The longest antichains of~$\posA$ are~$\makeset{\sapple, \schoco, \scheese}$,~$\makeset{\sbretzel, \schoco, \scheese}$,~$\makeset{\sapple, \schoco, \sgrapes}$,~$\makeset{\sbretzel, \schoco, \sgrapes}$, and~$\makeset{\sbretzel, \sburger, \scheese}$.
    Therefore,~$\posetwidth(\posA)=3$.
    The longest chain in the poset is given by~$\makeset{\scarrot,\schoco,\sburger,\sfondue}$, and therefore~$\posetheight(\posA)=4$.
\end{example}

