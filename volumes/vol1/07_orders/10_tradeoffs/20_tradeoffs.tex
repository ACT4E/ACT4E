% !TEX root = chapter-standalone.tex

\section{Ordered sets}

\label{sec:tradeoffs-ordered-sets}
\linkvideo{spring2021-tradeoffs:tradeoffs:orders:pre-pos-tot} % Pre-orders, partial orders, and total orders

%\linkvideo{spring2021-tradeoffs:tradeoffs:orders} % Orders

\showslides{
    \begin{forslides}
        \begin{equation}
            \label{eq:tradeoff_map}
            \mapa \colon \stylesets{X} \sto \stylesets{Y}
        \end{equation}
        \begin{equation}
            \label{eq:tradeoff_mor}
            \mora \colon \Obja \mto \Objb
        \end{equation}
        \begin{equation}
            \label{eq:tradeoff_typw}
            \mora \colon (\Obja \mto \Objb)
        \end{equation}
        \begin{equation}
            \label{eq:tradeoff_cap}
            \begin{aligned}
                \setA & \setintersection \setB \\
                \setA & \setunion \setB \\
                \setA & \setsubseteq \setB
            \end{aligned}
        \end{equation}
        \begin{equation}
            \label{eq:tradeoff_wedge}
            \begin{aligned}
                a & \booland b \\
                a & \boolor b \\
            \end{aligned}
        \end{equation}
        \begin{equation}
            \label{eq:tradeoff_leq}
            \setA \leq \setB
        \end{equation}
        \begin{equation}
            \label{eq:abc}
            \prflineextra=0.6em
            \prftree
            {\prftree
                {a}
                {b}}
            {c}
        \end{equation}
        \begin{equation}
            \label{eq:abc_2}
            \begin{aligned}
                           & a \\
                \Imp \quad & b \\
                \Imp \quad & c
            \end{aligned}
        \end{equation}
        \begin{equation}
            \label{eq:abc_3}
            \prftree{a}{b}{c}
        \end{equation}
        \begin{equation}
            \label{eq:abc_4}
            (a\booland b)\Imp c
        \end{equation}
        \begin{equation}
            \label{eq:abc_5}
            \prflineextra=0.6em
            \prftree[d]
            {\prftree[d]
                {a}
                {b}}
            {c}
        \end{equation}
        \begin{equation}
            \label{eq:abc_6}
            a\Leftrightarrow b \Leftrightarrow c
        \end{equation}
        \begin{equation}
            \label{eq:abc_7}
            \prflineextra=0.6em
            \prftree{a}{b}
        \end{equation}
        \begin{equation}
            \label{eq:abc_8}
            \prflineextra=0.6em
            \prftree{b}{c}
        \end{equation}
        \begin{equation}
            \label{eq:abc_9}
            \prflineextra=0.6em
            \prftree{a}{c}
        \end{equation}
        \begin{equation}
            \label{eq:mapf}
            \mapa \colon
        \end{equation}
        \begin{equation}
            \label{eq:mapg}
            \mapb \colon
        \end{equation}
        \begin{equation}
            \label{eq:mapcomp}
            \mapa \mthen \mapb \colon
        \end{equation}
        \begin{equation}
            \label{eq:top1}
            \prflineextra=0.6em
            \prftree{\true}{a}
        \end{equation}
        \begin{equation}
            \label{eq:bot1}
            \prflineextra=0.6em
            \prftree{a}{\false}
        \end{equation}
        \begin{equation}
            \label{eq:rel_refl}
            \prflineextra=0.6em
            \prftree{\true}{a\relA a}
        \end{equation}
        \begin{equation}
            \label{eq:rel_tot}
            \prflineextra=0.6em
            \prftree{\true}{a\relA b \quad \vee \quad b\relA a}
        \end{equation}
        \begin{equation}
            \label{eq:rel_tran}
            \prflineextra=0.6em
            \prftree{a\relA b}{b\relA c}{a\relA c}
        \end{equation}
        \begin{equation}
            \label{eq:rel_sym}
            \prflineextra=0.6em
            \prftree[d]{a \relA b}{b\relA a}
        \end{equation}
        \begin{equation}
            \label{eq:rel_irr}
            \prflineextra=0.6em
            \prftree{a \relA a}{\false}
        \end{equation}
        \begin{equation}
            \label{eq:rel_asy}
            \prflineextra=0.6em
            \prftree{a \relA b}{b\relA a}{\false}
        \end{equation}
        \begin{equation}
            \label{eq:rel_antisy}
            \prflineextra=0.6em
            \prftree{a \relA b}{b\relA a}{a=b}
        \end{equation}
        \begin{equation}
            \label{eq:ex}
            a\relA b \Leftrightarrow (a-b) \text{ mod }3=0
        \end{equation}
        \begin{equation}
            \label{eq:fun_cost}
            (\F{\text{keeps warm }}\Ptimes \F{\text{ music fidelity}})\cartprod(\R{\text{ price }} \Ptimes \R{\text{ frequency of charging }}\Ptimes \R{\text{ wires hassle}})
        \end{equation}
        \begin{equation}
            \label{eq:id_ex}
            \prftree{\true}{\catidat\Obja\colon \Obja \mto \Obja}
        \end{equation}
        \begin{equation}
            \label{eq:comp_cat}
            \prftree{\mora \colon \Obja \mto \Objb}{\morb\colon \Objb \mto \Objc}{\morab \colon \Obja \mto \Objc}
        \end{equation}
        \begin{equation}
            \label{eq:mor_pre}
            \prftree[d]{\ast\colon \posAel \mto \posBel}{\posAel \posleq \posBel}
        \end{equation}
        \begin{equation}
            \label{eq:hom_pre}
            \HomSet{}{\posAel}{\posBel}=\begin{cases}
                \singleton & \text{iff} \posAel \posleq \posBel \\
                \Emptyset  & \text{otherwise}
            \end{cases}
        \end{equation}
        \begin{equation}
            \label{eq:comp_pre}
            \singletonel \mthen \singletonel = \singletonel
        \end{equation}
        \begin{equation}
            \label{eq:ex_pre}
            \begin{aligned}
                 & \posAel,\posBel,\posCel \\
                 & \posAel \posleq \posBel \\
                 & \posCel \posleq \posBel
            \end{aligned}
        \end{equation}
        \begin{equation}
            \label{eq:pre_p}
            \posAel
        \end{equation}
        \begin{equation}
            \label{eq:pre_q}
            \posBel
        \end{equation}
        \begin{equation}
            \label{eq:pre_r}
            \posCel
        \end{equation}
        \begin{equation}
            \label{eq:eq_rel}
            \prftree[d]{\posAel\sim \posBel}{(\posAel \posleq \posBel)\wedge (\posBel \posleq \posAel)}
        \end{equation}
        \begin{equation}
            \label{eq:pos_int}
            \prflinepadbefore=0.6em
            \prflinepadafter=0.6em
            \prftree[d]{\tup{l_1,u_1}\posleqof{\styleelements{\text{Int}}\posA} \tup{l_2,u_2}}{(l_1\posAleq l_2)\wedge (u_2\posAleq u_1)}
        \end{equation}
        \begin{equation}
            \label{eq:leq}
            \posleq
        \end{equation}
        \begin{equation}
            \label{eq:pos_prod}
            \prflinepadbefore=0.6em
            \prflinepadafter=0.6em
            \prftree[d]{\tup{\posAel_1,\posBel_1}\posleqof{\posA\Ptimes \posB} \tup{\posAel_2,\posBel_2}}{(\posAel_1\posAleq \posAel_2)\wedge(\posBel_1\posBleq \posBel_2)}
        \end{equation}
        \begin{equation}
            \label{eq:pos_P}
            \posA
        \end{equation}
        \begin{equation}
            \label{eq:pos_Q}
            \posB
        \end{equation}
        \begin{equation}
            \label{eq:pos_PQ}
            \posA\Ptimes \posB
        \end{equation}
        \begin{equation}
            \label{eq:set_disj}
            \posA\Pplus\posB=\makeset{ \tup{1,\posAel} \mid \posAel\setin \posAset }\setunion \makeset{\tup{2,\posBel}\mid \posBel \setin \posBset}
        \end{equation}
        \begin{equation}
            \label{eq:latt_1}
            \posAel \wedge \posBel
        \end{equation}
        \begin{equation}
            \label{eq:latt_2}
            \posAel \vee \posBel
        \end{equation}
        \begin{equation}
            \label{eq:latt_3}
            (\posAel \wedge \posBel)\posleq \posAel \posleq (\posAel \vee \posBel)
        \end{equation}
        \begin{equation}
            \label{eq:latt_4}
            \posAel
        \end{equation}
        \begin{equation}
            \label{eq:latt_5}
            \posBel
        \end{equation}
        \begin{equation}
            \label{eq:prop_1}
            \prftree[d]{\posAel \posleq \posBel}{\posAel \Imp \posBel}
        \end{equation}
        \begin{equation}
            \label{eq:prop_2}
            \prftree[d]{\posAel \Leftrightarrow \posBel}{\posAel = \posBel}
        \end{equation}
        \begin{equation}
            \label{eq:prop_3}
            \prftree{\true}{\posAel \Imp \posAel}
        \end{equation}
        \begin{equation}
            \label{eq:prop_4}
            \prftree{(\posAel \Imp \posBel)\wedge (\posBel \Imp \posCel)}{\posAel \Imp \posCel}
        \end{equation}
        \includesag{060_propositions_lat}
        \begin{equation}
            \label{eq:pow_1}
            \setA \posleq \setB \definedas \setA \setsubseteq \setB
        \end{equation}
        \begin{equation}
            \label{eq:pow_2}
            \setA \vee \setB \definedas \setA \setunion \setB
        \end{equation}
        \begin{equation}
            \label{eq:pow_3}
            \setA \wedge \setB \definedas \setA \setintersection \setB
        \end{equation}
        \begin{equation}
            \label{eq:pow_4}
            \postop \definedas \stylesets{S}
        \end{equation}
        \begin{equation}
            \label{eq:pow_5}
            \posbot \definedas \stylesets{\Emptyset}
        \end{equation}
        \begin{equation}
            \label{eq:pow_6}
            \powerset{\stylesets{S}}
        \end{equation}
        \begin{equation}
            \label{eq:pow_7}
            \stylesets{S}
        \end{equation}
        \begin{equation}
            \label{eq:pre_ex1}
            \makeset{\tup{a,a}, \tup{a,b}, \tup{b,a}, \tup{b,b}}
        \end{equation}
        \begin{equation}
            \label{eq:pre_ex2}
            \makeset{\tup{a,b}, \tup{b,c}, \tup{c,a}}
        \end{equation}
        \begin{equation}
            \label{eq:pre_ex3}
            \makeset{\tup{a,a}, \tup{b,b}, \tup{c,c}, \tup{d,d}}
        \end{equation}
        % \begin{definition}
        %
        %     A symmetric matrix~$\mat{M}\setin \reals^{\ntimesn}$ is \emph{positive semi-definite} if~$x\mattransp \mat{M}x\geq 0$ for all non-zero~$x\setin \reals^n$.
        %     We call the set of all such matrices~$\mathcal{P}^n$.
        % \end{definition}
        \begin{equation}
            \label{eq:posdef_1}
            \mat{A}\posleq \mat{B} \Leftrightarrow (\mat{A}-\mat{B})\setin \mathcal{P}^n, \quad \mat{A},\mat{B}\setin \mathcal{P}^n
        \end{equation}
    \end{forslides}
}

%
% While we discussed how categories can describe the way in which one resource can be turned into another, this kind of modelling did not allow for quantitative statements.
% For example, it is good to know that we can obtain motion from electric power, but, how fast can we go with a certain amount of power?

% To achieve a quantitative theory, we need to specify various degrees of resources and functionality.
% One way of doing this is using the idea of orders.

% Such orderings arise naturally in engineering as criteria for judging whether one design is better or worse than another.
% As an example, suppose you need to buy a battery.
% In this simple example, you can think of having two resources: mass and money.
% A lighter battery might be more expensive, and a heavier one could be more affordable.
% How to choose among batteries, if you do not prefer one resource over the other?
% How to model this?
So far, the discussion has been purely qualitative.
In this section, we introduce pre-orders, partial orders, and total orders.
Davey and Priestley~\cite{davey02} and Roman~\cite{roman08} are possible reference texts.

% We introduce these concepts by adding levels of specificity.

\vspace{2cm}

\subsection{Pre-order}
\begin{marginfigure}
    \centering
    \includesag{pre_order_graph}
    \caption{A \SY{pre-order} represented as a graph.}
    \label{fig:pre_order_graph}
\end{marginfigure}

A \SY{pre-order} is a set together with a relation that is both \SY{reflexive} (\cref{def:endo_reflexive_irreflexive}) and \SY{transitive}~(\cref{def:transitive-relation}).
\begin{ctdefinition}[Pre-ordered set]
    \label{def:preorder}
    A \maindef{pre-ordered set} is a tuple~$\posAdefinition$, where~$\posAset$ is a set, called the \emph{carrier set} or \emph{underlying set}, together with a relation~$\posAleq$ that is
    \SY{reflexive} and \SY{transitive}.
\end{ctdefinition}
An example of a \SY{pre-ordered set} represented as a graph is reported in \cref{fig:pre_order_graph}.
In the graph representation of a \SY{pre-order}~\posA, we draw an arrow between~$\posela$ and~$\poselb$ if~$\posela \posAleq \poselb$.

\vfill

\subsection{Partial order}

\begin{marginfigure}
    \centering
    \includesag{pos_order_graph}
    \caption{A partial order represented as a graph.}
    \label{fig:pos_order_graph}
\end{marginfigure}
By adding \emph{\SY{antisymmetry}} (\cref{def:antisymmetry}), we obtain a partially ordered set.

\begin{ctdefinition}[Partially ordered set]
    \label{def:poset}
    A pre-ordered set~$\posAdefinition$ is a \maindef{partially-ordered set} (poset) if the relation~$\posAleq$ is antisymmetric.
\end{ctdefinition}

An example of a partially ordered set represented as a graph is reported in \cref{fig:pos_order_graph}.
% (we will formally define graphs in \cref{def:Graph}).
By comparing this with \cref{fig:pre_order_graph}, we notice that the double-headed arrow is not allowed anymore (indeed, its existence would imply that source and target of the arrow are the same element in the poset).

\vfill

\subsection{Total order}
\begin{marginfigure}
    \centering
    \includesag{tot_order_graph}
    \caption{A total order.}
    \label{fig:tot_order_graph}
\end{marginfigure}
By imposing \emph{\SY{totality}} (\cref{{def:total-relation}}), we obtain a total order.

\begin{ctdefinition}[Totally ordered set]
    \label{def:total-order}
    A partially ordered set~$\posA = \tupp{\posAset,\posAleq}$ is a \maindef{totally ordered set} if the relation~$\posAleq$ is total.
    % In other words, if:
    % \begin{equation}
    %     \prfperiod{\true}{(\posAel \posAleq \posBel) \vee (\posBel \posAleq \posAel)}
    % \end{equation}
\end{ctdefinition}

An example of a totally ordered set represented as a graph is reported in \cref{fig:tot_order_graph}.

\begin{example}[Reals]
    The real numbers \reals form a totally ordered \SY{poset} $\realswithleq$ with order relation given by the usual ordering.
\end{example}
\vfill
\clearpage

\subsection{Hasse diagrams}
\linkvideo{spring2021-tradeoffs:tradeoffs:orders:hasse} % Hasse diagrams
We can represent partial orders in various ways.
We now take a proxy partially ordered set and represent it using different conventions.
Consider~$\posAdefinition$, where~$\posAset=\makeset{\posela,\poselb,\poselc}$ and~$\posela\posAleq \poselb$,~$\poselb\posAleq \poselc$.
First, we could represent this using the same visualization we had for relations (\cref{fig:poset_as_relation}).

However, this is quite heavy, and does not exploit the fact that partial orders are endorelations.
Therefore, we could think to only draw the carrier set once, and to drop the order relations arising from \SY{reflexivity} (\cref{fig:poset_as_graph}).

However, the arrow from~$\posela$ to~$\poselb$ is implicit in partial orders, because of \SY{transitivity}.

\label{sec:hasse-diagram}
A \maindef{Hasse diagram} is an economical (in terms of arrows) way to visualize a poset.
In a Hasse diagram elements are points, and if~$\posAel \posAleq \posBel$ then~$\posAel$ is drawn lower than~$\posBel$ and with an edge connected to it, if no other point is in between (\cref{fig:poset_as_hasse}).
Hasse diagrams are directed graphs.

\begin{figure*}[h!]
    % \aligninner{
    \hspace{4em}
    \subfloat[\posA as a relation\label{fig:poset_as_relation}]{
        \includesag{poset_representation_1}
    }
    \hspace{4em}
    \subfloat[\posA as a graph \label{fig:poset_as_graph}]{
        \rule{1cm}{0pt} % TODO: FIX PARBOX
        \includesag{poset_representation_2}
        \rule{1cm}{0pt}
    }
    \hspace{4em}
    \subfloat[\posA as a Hasse diagram\label{fig:poset_as_hasse}]{
        \rule{1cm}{0pt}  % TODO: FIX PARBOX
        \includesag{poset_representation_3}
        \rule{1cm}{0pt}
    }
    % }
    \caption{Three different representations for a poset}
    \label{fig:poset_representation}
\end{figure*}

\clearpage
\begin{marginfigure}
    \centering
    \includesag{40_discrete}
    \caption{}
    \label{fig:40_discrete}
\end{marginfigure}

\begin{example}[Discrete partially ordered sets]

    \label{ex:discreteposet}
    Every set~$\posAset$ can be considered as a \emph{discrete poset}~$\posA = \tup{\posAset,{{=}}}$ using equality as the partial order.
    (Notice that equality is symmetric, \SY{transitive}, and antisymmetric.)
    When visualized as a \SY{Hasse diagram}, discrete \SY{posets} are a collection of points (\cref{fig:40_discrete}).
\end{example}
\vspace{1cm}
\begin{marginfigure}[4mm]
    \centering
    \includesag{40_dpcatfig_boolean}
    \caption{}
\end{marginfigure}

\begin{definition}[Boolean poset \Bool]
    \label{def:bool-poset}
    \SYNDEF{boolean poset}
    The set of booleans $\boolset = \makeset{\false, \true}$ can be made into a \SY{poset} by choosing the order $\false \posleqof\Bool \true$.
    This is equivalent to using ``$\Imp$'' as a relation.
    We obtain the poset
    \begin{equation}
        \label{eq:bool-poset}
        \Bool \definedas \tup{\boolset, {{\Imp}}}.
    \end{equation}
    % \index{\Bool} is a \SY{poset} with carrier set~$\makeset{\true,\false}$ and the order relation given by~
\end{definition}

\vspace{1cm}
\begin{marginfigure}[4mm]
    \centering
    \includesag{70_hasse_pizza}
    \caption{The cost of a battery can be represented as a poset.
    }
    \label{fig:hassebattery}
\end{marginfigure}
%
\begin{example}[Qualitative information]
    In the example of the battery choice, both mass and money can be thought of as partially ordered sets.
    Imagine that you have batteries which are ``\poscheap'', ``\posmidrange'', and ``\posexpensive''.
    Clearly, if the partially ordered set represents cost, we can say that~$\poscheap \posleq\posmidrange \posleq\posexpensive$.
    While this is a quantitative judgement (indeed, if I care about cost, I will prefer a cheap battery over a midrange one), it is not a numeric one (cheap could represent a number, but also a range of numbers or just a price category).
    This can be represented as in~\cref{fig:hassebattery}.
\end{example}

\vspace{1cm}
\begin{marginfigure}[4mm]
    \centering
    \includesag{40_dpcatfig_hasse}
    \caption{Example of Hasse diagram of~\posA.}
    \label{fig:hasse}
\end{marginfigure}
\begin{example}
    Consider a poset~\posA representing a person's food preference over the set~$\posAset=\makeset{\sbretzel,\sfondue,\schoco,\sburger,\sapple}$ with~$\sbretzel\posAleq \schoco$,~$\sburger\posAleq \schoco$,~$\schoco\posAleq \sfondue$, and~$\sburger\posAleq \sapple$.
    This can be represented with a Hasse diagram as in~\cref{fig:hasse}.
\end{example}

\vfill\clearpage

\subsection{Example: positive definite matrices as ellipsoids}

\begin{marginfigure}
    \centering
    \subfloat[
        Example of ellipses representing positive definite matrices.
        \label{fig:posdef_draw}
    ]{
        \centering
        \maxsizebox{0.9\marginparwidth}{!}{%
            \includesag{20_ellipses_mat}
        }
    }

    \subfloat[
        Example of order between positive semi-definite matrices.
        \label{fig:posdef_hasse}
    ]{%
        \parbox{\marginparwidth}{
            \centering
            \includesag{20_mat_order}
        }%
    }
    \caption{}
    \label{fig:posdef}
\end{marginfigure}

\begin{definition}[Positive definite matrix]
    \label{def:positive-definite-matrices}
    A symmetric matrix~$\mat{M}\setin \reals^{\ntimesn}$ is \maindef{positive definite} if~$\vec{x}\mattransp \mat{M}\vec{x}> 0$ for all non-zero~$\vec{x}\setin \reals^n$.
    We call the set of all such matrices~$\posdefmatset{n}$.
\end{definition}
Positive definite matrices have real, positive eigenvalues, which can be interpreted as axes lengths of ellipsoids.
Any matrix~$\mat{A}\setin \posdefmatset{n}$ describes an ellipsoid, which can be written as a quadratic equation:
\begin{equation}
    \label{eq:posdef-ellipsoid}
    \vec{x}\mattransp \mat{A}\vec{x}=1,\quad \vec{x}\setin \reals^n.
\end{equation}
We can define a partial order on~$n$ as
\begin{equation}
    \label{eq:posdef-order}
    \prfdoubleperiod{
        \mat{A}\posleqof{\posdefmatset{n}} \mat{B}
    }{
        \vec{x}\mattransp \mat{A}\vec{x}\leq \vec{x}\mattransp \mat{B}\vec{x} \quad \forall \vec{x}\setin \reals^n
    }
\end{equation}
Note that the condition~$\mat{A}=\mat{B}$ is needed to ensure \SY{reflexivity}.
The order can be interpreted as an inclusion of ellipsoids.
Take for instance the matrices
\begin{equation}
    \label{eq:posdef-matrices}
    \mat{A}=
    \begin{bmatrix}
        1 & 0 \\ 0& 1
    \end{bmatrix}, \quad \mat{B}=
    \begin{bmatrix}
        3/4 & -1/8 \\ -1/8& 3/4
    \end{bmatrix},\quad \mat{C}=
    \begin{bmatrix}
        1/2 & 0 \\0& 2
    \end{bmatrix}.
\end{equation}
The order~\posA on the set~$\makeset{\mat{A},\mat{B},\mat{C}}$ is reported in \cref{fig:posdef_hasse}, and it is easily explained via \cref{fig:posdef_draw}.
The ellipse representing~$\mat{A}$ (in red) is included by the one representing matrix~$\mat{B}$ (in blue), but not by the one representing matrix~$\mat{C}$ (in green).
Furthermore, the one representing~$\mat{B}$ includes the one representing~$\mat{C}$.

\vfill
\begin{gradedexercise}[\exname{PolynomialDivisibility}]
    \label{ex:PolynomialDivisibility}
    Let~\setA be the set of all polynomials with coefficients in \reals.
    Recall that a polynomial~$p$ \emph{divides} a polynomial~$q$ if there exists a polynomial~$m$ such that~$p \cdot m = q$.
    If~$p$ divides~$q$ we denote this by~$\divides p q$.
    Divisibility defines an \SY{endorelation} on~\setA by saying~$p$ is related to~$q$ iff~$\divides p q$.
    Does this define a \SY{pre-order} structure on~\setA?
    Does this define a \SY{poset} structure on~\setA?
    Justify your answer.
\end{gradedexercise}
\solutionof{PolynomialDivisibility}

\section{Counting orders}

\linkvideo{spring2021-tradeoffs:tradeoffs:orders:counting-orders} % Counting orders
\todotext{\bernina: Put this section in its own file.}

Let's count the number of \SY{posets}.

If there is only one element, there is only one way to order it (\cref{fig:singleton}).

With a 2-elements set, there are 2 \SY{posets} (panel \emph{b}), ``up to isomorphism'',
that is, if we do not care about the labels of points.

On 3-elements sets, we have 5 \SY{posets} (panel \emph{c}).

On 4-elements sets, we have 16 \SY{posets} (panel \emph{d}).

\vfill

\begin{figure*}[h]
    % \hfill
    \subfloat[\label{fig:singleton}
        The singleton poset.
    ]{
        \rule{2cm}{0pt}\middlesag{40_dpcatfig_singleton}\rule{2cm}{0pt}
    }
    % \hfill
    \subfloat[All \SY{posets} on 2-elements sets, up to isomorphism.
        \label{fig:twoelementspos}
    ]{
        \centering
        \setlength{\tabcolsep}{20pt}
        \rule{1cm}{0pt}
        \begin{tabular}{cc}
            \middlesag{70_pos_2_1} & \middlesag{70_pos_2_2}
        \end{tabular}
        \rule{1cm}{0pt}
    }

    % \medskip
    \subfloat[All \SY{posets} on 3-elements sets, up to isomorphisms.
        \label{fig:threeelementspos}]{
        \setlength{\tabcolsep}{20pt}
        \begin{tabular}{ccccc}
            \middlesag{70_pos_3_1} &
            \middlesag{70_pos_3_2} &
            \middlesag{70_pos_3_3} &
            \middlesag{70_pos_3_4} &
            \middlesag{70_pos_3_5}
        \end{tabular}
    }

    % \medskip
    \subfloat[
        All \SY{posets} on a 4-element set, up to isomorphism.
        \label{fig:fourelementspos}
    ]{
        \centering
        \setlength{\tabcolsep}{6pt}
        \maxsizebox{\marginparwidth+\textwidth}{!}{
            \begin{tabular}{cccccc}
                \middlesag{70_pos_1}
                 & \middlesag{70_pos_2}
                 & \middlesag{70_pos_3}
                 & \middlesag{70_pos_4}
                 & \middlesag{70_pos_5} \\[+30pt]
                \middlesag{70_pos_6}
                 & \middlesag{70_pos_7}
                 & \middlesag{70_pos_8}
                 & \middlesag{70_pos_14}
                 & \middlesag{70_pos_15} \\[+30pt]
                \middlesag{70_pos_11}
                 & \middlesag{70_pos_12}
                 & \middlesag{70_pos_13}
                 & \middlesag{70_pos_9}
                 & \middlesag{70_pos_10}
                 & \middlesag{70_pos_16}
            \end{tabular}
        }
    }
    \caption{}
    \label{fig:posets-up-to-four}
\end{figure*}
