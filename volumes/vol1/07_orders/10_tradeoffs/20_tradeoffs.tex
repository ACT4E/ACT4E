% !TEX root = chapter-standalone.tex

\section{Ordered sets}

\label{sec:tradeoffs-ordered-sets}
\linkvideo{spring2021-tradeoffs:tradeoffs:orders:pre-pos-tot} % Pre-orders, partial orders, and total orders

%\linkvideo{spring2021-tradeoffs:tradeoffs:orders} % Orders

\showslides{
	\begin{forslides}
		\begin{equation*}
			\label{eq:tradeoff_map}
			\mapa \colon \stylesets{X} \sto \stylesets{Y}
		\end{equation*}
		\begin{equation*}
			\label{eq:tradeoff_mor}
			\mora \colon \Obja \mto \Objb
		\end{equation*}
		\begin{equation*}
			\label{eq:tradeoff_typw}
			\mora \colon (\Obja \mto \Objb)
		\end{equation*}
		\begin{equation*}
			\label{eq:tradeoff_cap}
			\begin{aligned}
				\setA & \setintersection \setB \\
				\setA & \setunion \setB        \\
				\setA & \subseteq \setB
			\end{aligned}
		\end{equation*}
		\begin{equation*}
			\label{eq:tradeoff_wedge}
			\begin{aligned}
				a & \wedge b \\
				a & \vee b   \\
			\end{aligned}
		\end{equation*}
		\begin{equation*}
			\label{eq:tradeoff_leq}
			\setA \leq \setB
		\end{equation*}
		\begin{equation*}
			\label{eq:abc}
			\prflineextra=0.6em
			\prftree
			{\prftree
				{a}
				{b}}
			{c}
		\end{equation*}
		\begin{equation*}
			\label{eq:abc_2}
			\begin{aligned}
				           & a \\
				\Imp \quad & b \\
				\Imp \quad & c
			\end{aligned}
		\end{equation*}
		\begin{equation*}
			\label{eq:abc_3}
			\prftree{a}{b}{c}
		\end{equation*}
		\begin{equation*}
			\label{eq:abc_4}
			(a\wedge b)\Imp c
		\end{equation*}
		\begin{equation*}
			\label{eq:abc_5}
			\prflineextra=0.6em
			\prftree[d]
			{\prftree[d]
				{a}
				{b}}
			{c}
		\end{equation*}
		\begin{equation*}
			\label{eq:abc_6}
			a\Leftrightarrow b \Leftrightarrow c
		\end{equation*}
		\begin{equation*}
			\label{eq:abc_7}
			\prflineextra=0.6em
			\prftree{a}{b}
		\end{equation*}
		\begin{equation*}
			\label{eq:abc_8}
			\prflineextra=0.6em
			\prftree{b}{c}
		\end{equation*}
		\begin{equation*}
			\label{eq:abc_9}
			\prflineextra=0.6em
			\prftree{a}{c}
		\end{equation*}
		\begin{equation*}
			\label{eq:mapf}
			\mapa \colon
		\end{equation*}
		\begin{equation*}
			\label{eq:mapg}
			\mapb \colon
		\end{equation*}
		\begin{equation*}
			\label{eq:mapcomp}
			\mapa \then \mapb \colon
		\end{equation*}
		\begin{equation*}
			\label{eq:top1}
			\prflineextra=0.6em
			\prftree{\true}{a}
		\end{equation*}
		\begin{equation*}
			\label{eq:bot1}
			\prflineextra=0.6em
			\prftree{a}{\false}
		\end{equation*}
		\begin{equation*}
			\label{eq:rel_refl}
			\prflineextra=0.6em
			\prftree{\true}{a\relA a}
		\end{equation*}
		\begin{equation*}
			\label{eq:rel_tot}
			\prflineextra=0.6em
			\prftree{\true}{a\relA b \quad \vee \quad b\relA a}
		\end{equation*}
		\begin{equation*}
			\label{eq:rel_tran}
			\prflineextra=0.6em
			\prftree{a\relA b}{b\relA c}{a\relA c}
		\end{equation*}
		\begin{equation*}
			\label{eq:rel_sym}
			\prflineextra=0.6em
			\prftree[d]{a \relA b}{b\relA a}
		\end{equation*}
		\begin{equation*}
			\label{eq:rel_irr}
			\prflineextra=0.6em
			\prftree{a \relA a}{\false}
		\end{equation*}
		\begin{equation*}
			\label{eq:rel_asy}
			\prflineextra=0.6em
			\prftree{a \relA b}{b\relA a}{\false}
		\end{equation*}
		\begin{equation*}
			\label{eq:rel_antisy}
			\prflineextra=0.6em
			\prftree{a \relA b}{b\relA a}{a=b}
		\end{equation*}
		\begin{equation*}
			\label{eq:ex}
			a\relA b \Leftrightarrow (a-b) \text{ mod }3=0
		\end{equation*}
		\begin{equation*}
			\label{eq:fun_cost}
			(\F{\text{keeps warm }}\cartprod \F{\text{ music fidelity}})\cartprod(\R{\text{ price }} \cartprod \R{\text{ frequency of charging }}\cartprod \R{\text{ wires hassle}})
		\end{equation*}
		\begin{equation*}
			\label{eq:id_ex}
			\prftree{\true}{\catid_\Obja\colon \Obja \mto \Obja}
		\end{equation*}
		\begin{equation*}
			\label{eq:comp_cat}
			\prftree{\mora \colon \Obja \mto \Objb}{\morb\colon \Objb \mto \Objc}{\mora\mthen \morb \colon \Obja \mto \Objc}
		\end{equation*}
		\begin{equation*}
			\label{eq:mor_pre}
			\prftree[d]{\ast\colon \posAel \mto \posBel}{\posAel \posleq \posBel}
		\end{equation*}
		\begin{equation*}
			\label{eq:hom_pre}
			\HomSet{}{\posAel}{\posBel}=\begin{cases}
				\{\ast\}  & \text{iff } \posAel \posleq \posBel \\
				\emptyset & \text{otherwise}
			\end{cases}
		\end{equation*}
		\begin{equation*}
			\label{eq:comp_pre}
			\ast \mthen \ast = \ast
		\end{equation*}
		\begin{equation*}
			\label{eq:ex_pre}
			\begin{aligned}
				 & \posAel,\posBel,\posCel \\
				 & \posAel \posleq \posBel \\
				 & \posCel \posleq \posBel
			\end{aligned}
		\end{equation*}
		\begin{equation*}
			\label{eq:pre_p}
			\posAel
		\end{equation*}
		\begin{equation*}
			\label{eq:pre_q}
			\posBel
		\end{equation*}
		\begin{equation*}
			\label{eq:pre_r}
			\posCel
		\end{equation*}
		\begin{equation*}
			\label{eq:eq_rel}
			\prftree[d]{\posAel\sim \posBel}{(\posAel \posleq \posBel)\wedge (\posBel \posleq \posAel)}
		\end{equation*}
		\begin{equation*}
			\label{eq:pos_int}
			\prflinepadbefore=0.6em
			\prflinepadafter=0.6em
			\prftree[d]{\tup{l_1,u_1}\posleqof{\styleelements{\text{Int}}\posA} \tup{l_2,u_2}}{(l_1\posAleq l_2)\wedge (u_2\posAleq u_1)}
		\end{equation*}
		\begin{equation*}
			\label{eq:leq}
			\posleq
		\end{equation*}
		\begin{equation*}
			\label{eq:pos_prod}
			\prflinepadbefore=0.6em
			\prflinepadafter=0.6em
			\prftree[d]{\tup{\posAel_1,\posBel_1}\posleqof{\posA\cartprod \posB} \tup{\posAel_2,\posBel_2}}{(\posAel_1\posAleq \posAel_2)\wedge(\posBel_1\posBleq \posBel_2)}
		\end{equation*}
		\begin{equation*}
			\label{eq:pos_P}
			\posA
		\end{equation*}
		\begin{equation*}
			\label{eq:pos_Q}
			\posB
		\end{equation*}
		\begin{equation*}
			\label{eq:pos_PQ}
			\posA\cartprod \posB
		\end{equation*}
		\begin{equation*}
			\label{eq:set_disj}
			\posA+\posB=\{ \tup{1,\posAel} \mid \posAel\in \posA \}\setunion \{\tup{2,\posBel}\mid \posBel \in \posB\}
		\end{equation*}
		\begin{equation*}
			\label{eq:latt_1}
			\posAel \wedge \posBel
		\end{equation*}
		\begin{equation*}
			\label{eq:latt_2}
			\posAel \vee \posBel
		\end{equation*}
		\begin{equation*}
			\label{eq:latt_3}
			(\posAel \wedge \posBel)\posleq \posAel \posleq (\posAel \vee \posBel)
		\end{equation*}
		\begin{equation*}
			\label{eq:latt_4}
			\posAel
		\end{equation*}
		\begin{equation*}
			\label{eq:latt_5}
			\posBel
		\end{equation*}
		\begin{equation*}
			\label{eq:prop_1}
			\prftree[d]{\posAel \posleq \posBel}{\posAel \Imp \posBel}
		\end{equation*}
		\begin{equation*}
			\label{eq:prop_2}
			\prftree[d]{\posAel \Leftrightarrow \posBel}{\posAel = \posBel}
		\end{equation*}
		\begin{equation*}
			\label{eq:prop_3}
			\prftree{\true}{\posAel \Imp \posAel}
		\end{equation*}
		\begin{equation*}
			\label{eq:prop_4}
			\prftree{(\posAel \Imp \posBel)\wedge (\posBel \Imp \posCel)}{\posAel \Imp \posCel}
		\end{equation*}
		\includesag{060_propositions_lat}
		\begin{equation*}
			\label{eq:pow_1}
			\setA \posleq \setB \definedas \setA \subseteq \setB
		\end{equation*}
		\begin{equation*}
			\label{eq:pow_2}
			\setA \vee \setB \definedas \setA \setunion \setB
		\end{equation*}
		\begin{equation*}
			\label{eq:pow_3}
			\setA \wedge \setB \definedas \setA \setintersection \setB
		\end{equation*}
		\begin{equation*}
			\label{eq:pow_4}
			\postop \definedas \stylesets{S}
		\end{equation*}
		\begin{equation*}
			\label{eq:pow_5}
			\posbot \definedas \stylesets{\emptyset}
		\end{equation*}
		\begin{equation*}
			\label{eq:pow_6}
			\powerset{\stylesets{S}}
		\end{equation*}
		\begin{equation*}
			\label{eq:pow_7}
			\stylesets{S}
		\end{equation*}
		\begin{equation*}
			\label{eq:pre_ex1}
			\{\tup{a,a},\tup{a,b},\tup{b,a},\tup{b,b}\}
		\end{equation*}
		\begin{equation*}
			\label{eq:pre_ex2}
			\{\tup{a,b},\tup{b,c},\tup{c,a}\}
		\end{equation*}
		\begin{equation*}
			\label{eq:pre_ex3}
			\{\tup{a,a},\tup{b,b},\tup{c,c},\tup{d,d}\}
		\end{equation*}
		\begin{definition}
			\label{def:posdef}
			A symmetric matrix~$\mat{M}\in \reals^{n\times n}$ is \emph{positive semi-definite} if~$x^\intercal \mat{M}x\geq 0$ for all non-zero~$x\in \reals^n$.
			We call the set of all such matrices~$\mathcal{P}^n$.
		\end{definition}
		\begin{equation*}
			\label{eq:posdef_1}
			\mat{A}\posleq \mat{B} \Leftrightarrow (\mat{A}-\mat{B})\in \mathcal{P}^n, \quad \mat{A},\mat{B}\in \mathcal{P}^n
		\end{equation*}
	\end{forslides}
}

So far, the discussion has been purely qualitative.
While we discussed how categories can describe the way in which one resource can be turned into another, this kind of modelling did not allow for quantitative statements.
For example, it is good to know that we can obtain motion from electric power, but, how fast can we go with a certain amount of power?

To achieve a quantitative theory, we need to specify various degrees of resources and functionality.
One way of doing this is using the idea of orders.

Such orderings arise naturally in engineering as criteria for judging whether one design is better or worse than another.
As an example, suppose you need to buy a battery.
In this simple example, you can think of having two resources: mass and money.
A lighter battery might be more expensive, and a heavier one could be more affordable.
How to choose among batteries, if you do not prefer one resource over the other?
How to model this?
In this section, we will assume that functionality and resources are \emph{ordered sets}, and will introduce pre-orders, partial orders, and total orders.

Davey and Priestley~\cite{davey02} and Roman~\cite{roman08} are possible reference texts.

We introduce these concepts by adding levels of specificity.

\begin{ctdefinition}[Pre-ordered set]
	\label{def:preorder}
	A \emph{pre-ordered} set is a tuple~$\posA = \tup{\posAset,\posAleq}$, where~$\posAset$ is a set, called the \emph{carrier set} or \emph{underlying set}), together with a relation~$\posAleq$ that is
	reflexive (\cref{def:endo_reflexive_irreflexive}) and transitive (\cref{def:endo_transitive}).
	% \begin{itemize}
	%     \item \emph{Reflexive:}
	%           \begin{equation}
	%               \prfperiod{\true}{\posAel \posAleq \posAel}
	%           \end{equation}
	%     \item \emph{Transitive:}
	%           \begin{equation}
	%               \prfperiod{\posAel \posAleq \posBel}{\posBel \posAleq \posCel}{\posAel \posAleq \posCel}
	%           \end{equation}
	% \end{itemize}
\end{ctdefinition}
By adding \emph{antisymmetry} (\cref{def:antisymmetry}), one obtains a partially ordered set.
\begin{ctdefinition}[Partially ordered set]
	\label{def:poset}
	A pre-ordered set~$\posA = \tupp{\posAset,\posAleq}$ is a \emph{partially-ordered set (\iindex{poset})} if the relation~$\posAleq$ is \emph{antisymmetric} (\cref{def:endo_sym_asym_antisym}).
	% In other words, if:
	% \begin{equation}
	%     \prfperiod{\posAel \posAleq \posBel}{\posBel \posAleq \posAel}{\posAel = \posBel}
	% \end{equation}
\end{ctdefinition}

By imposing \emph{totality} (\cref{def:endo_total}), one obtains a total order.

\begin{ctdefinition}[Totally ordered set]
	\label{def:total_order}
	A partially ordered set~$\posA = \tupp{\posAset,\posAleq}$ is a \emph{totally ordered set} if the relation~$\posAleq$ is \emph{total}
	(\cref{def:endo_total}).
	% In other words, if:
	% \begin{equation}
	%     \prfperiod{\true}{(\posAel \posAleq \posBel) \vee (\posBel \posAleq \posAel)}
	% \end{equation}
\end{ctdefinition}
\vfill
\clearpage

\subsection{Hasse diagrams}
\linkvideo{spring2021-tradeoffs:tradeoffs:orders:hasse} % Hasse diagrams

A \emph{\iindex{Hasse diagram}} is an economical (in terms of arrows) way to visualize a poset.
In a Hasse diagram elements are points, and if~$\posAel \posAleq \posBel$ then~$\posAel$ is drawn lower than~$\posBel$ and with an ege connected to it, if no other point is in between.
Hasse diagrams are directed graphs.

\todotextjira{502}{@Gioele: Explain how this is an economical way to visualize a relation, by showing an alternative}

In the example of the battery choice, both mass and money can be thought of as partially ordered sets.
Imagine that you have batteries which are ``\poscheap'', ``\posmidrange'', and ``\posexpensive''.
Clearly, if the partially ordered set represents cost, one can say that~$\poscheap \posleq\posmidrange \posleq\posexpensive$.
While this is a quantitative judgement (indeed, if I care about cost, I will prefer a cheap battery over a midrange one), it is not a numeric one (cheap could represent a number, but also a range of numbers or just a price category).
This can be represented as in~\cref{fig:hassebattery}.

\begin{figure}[h!]
	\centering
	\includesag{70_hasse_pizza}
	\caption{The cost of a battery can be represented as a poset.}
	\label{fig:hassebattery}
\end{figure}

\todotextjira{504}{@Gioele: Bad example for Hasse diagram because there are no implicit arrows.
}
\begin{example}
	Consider a poset~$\posA$ representing your food preference over the set~$\posAset=\{\sbretzel,\sfondue,\schoco,\sburger,\sapple\}$ with~$\sbretzel\posAleq \sfondue$,~$\sbretzel\posAleq \schoco$,~$\sburger\posAleq \schoco$, and~$\sburger\posAleq \sapple$.
	This can be represented with a Hasse diagram as in~\cref{fig:hasse}.
\end{example}

\begin{figure}[h!]
	\centering
	\includesag{40_dpcatfig_hasse}
	\caption{Example of Hasse diagram of~$\posA$.}
	\label{fig:hasse}
\end{figure}

\begin{marginfigure}
	\centering
	\includesag{40_dpcatfig_boolean}
	\caption{}
	\label{fig:boolean}
\end{marginfigure}

\begin{margintable}
	\centering
	\begin{tabular}{cc|ccc}
		$a$      & $b$      & $a \posleq  b$ & $a \wedge b$ & $a \vee b$ \\ \hline
		$\true$  & $\true$  & $\true$        & $\true$      & $\true$    \\
		$\true$  & $\false$ & $\false$       & $\false$     & $\true$    \\
		$\false$ & $\true$  & $\true$        & $\false$     & $\true$    \\
		$\false$ & $\false$ & $\true$        & $\false$     & $\false$
	\end{tabular}
	\caption{Properties of the \Bool poset.
		Note that $\posleq \equiv \Imp$.
	}
	\label{tab:boolposet}
\end{margintable}

\begin{example}[Booleans]
	\label{ex:bool}
	The booleans \index{\Bool} is a poset with carrier set~$\{\true,\false\}$ and the order relation given by~$b_1 \posleqof\Bool b_2$ iff~$b_1 \Imp b_2$, that is,~$\false \posleqof\Bool \true$ (\cref{fig:boolean}).

	This relation should be familiar from~\cref{tab:boolposet}.

	In addition to the operation
	\begin{equation*}
		\Imp\colon\Bool\cartprod \Bool\to\Bool,
	\end{equation*}
	called \emph{implication}, there are also the familiar \emph{and} ($\wedge$) and \emph{or} ($\vee$) operations.
	Note that~$\wedge$ and~$\vee$ are commutative ($b\wedge c = c\wedge b$,~$b\vee c = c\vee b$ ), whereas~$\Imp$ is not.
\end{example}

\todostructure{Move this thing about boolean operations later to lattices.}

\begin{example}[Reals]
	The real numbers \reals form a poset with carrier \reals and order relation given by the usual ordering~$r_1 \Rleq r_2$.
\end{example}

\begin{marginfigure}
	\centering
	\includesag{40_discrete}
	\caption{Example of a discrete poset.}
	\label{fig:discretepos}
\end{marginfigure}

\begin{example}[Discrete partially ordered sets]
	\label{ex:discreteposet}
	Every set~$\posAset$ can be considered as a \emph{discrete poset}~$\posA = \tup{\posAset,=}$.
	Discrete posets are represented as collection of points (\cref{fig:discretepos}).
\end{example}

\newcommand{\fitinmargin}[1]{%
	\maxsizebox{\marginparwidth}{!}{#1}%
}

\newcommand{\fitinline}[1]{%
	\maxsizebox{\textwidth}{!}{#1}%
}

% \vfill\clearpage

\subsection{Example: positive definite matrices}

\begin{marginfigure}
	\centering
	\subfloat[
		Example of ellipses representing positive semi-definite matrices.
		\label{fig:posdef_draw}
	]{
		\centering
		\maxsizebox{0.9\marginparwidth}{!}{%
			\includesag{20_ellipses_mat}
		}
	}

	\subfloat[
		Example of order between positive semi-definite matrices.
		\label{fig:posdef_hasse}
	] {
		\centering
		\rule{1cm}{0pt}
		\includesag{20_mat_order}
		\rule{1cm}{0pt}
	}
	\caption{}
	\todographics{Label poset}
	\label{fig:posdef}
\end{marginfigure}

\begin{example}
	\begin{definition}[Positive semi-definite matrix]
		A symmetric matrix~$\mat{M}\in \reals^{n\times n}$ is \emph{positive semi-definite} if~$x^\intercal \mat{M}x\geq 0$ for all non-zero~$x\in \reals^n$.
		We call the set of all such matrices~$\mathcal{P}^n$.
	\end{definition}
	Positive semi-deminite matrices have real, semi-positive eigenvalues, which can be interpreted as axes lenghts of ellipsoids.
	Any matrix~$\mat{A}\in \mathcal{P}^n$ describes an ellipsoid, descriptive equation of which can be written as a quadratic form:
	\begin{equation*}
		x^\intercal \mat{A}x=1,\quad x\in \reals^n.
	\end{equation*}
	\todotext{Use macro symbol for pos. def. matrices}
	We can define a partial order on~$\mathcal{P}^n$ as
	\begin{equation*}
		\prfdoubleperiod{
			\mat{A}\posleqof{\mathcal{P}^n} \mat{B}
		}{
			(\mat{B}-\mat{A})\in \mathcal{P}^n
		}
	\end{equation*}
	The order can be interpreted as an inclusion of ellipsoids.
	Take for instance the matrices
	\begin{equation*}
		\mat{A}=
		\begin{pmatrix}
			1 & 0 \\0& 1
		\end{pmatrix}, \quad \mat{B}=
		\begin{pmatrix}
			2 & 0 \\0& 1
		\end{pmatrix},\quad \mat{C}=
		\begin{pmatrix}
			2 & 0 \\0& 0.5
		\end{pmatrix}.
	\end{equation*}
	The order on the set~$\{\mat{A},\mat{B},\mat{C}\}$ is reported in \cref{fig:posdef_hasse}, and it is easily explained via \cref{fig:posdef_draw}.
	The ellipse representing~$\mat{A}$ (in red) is included by the one representing matrix~$\mat{B}$ (in blue), but not by the one representing matrix~$\mat{C}$ (in green).
	Furthermore, the one representing~$\mat{B}$ includes the one representing~$\mat{C}$.
\end{example}

\vfill
%

\begin{gradedexercise}[\exname{PolynomialDivisibility}]
	\label{ex:PolynomialDivisibility}
	Let~$\setA$ be the set of all polynomials with coefficients in $\reals$.
	Recall that a polynomial~$p$ \emph{divides} a polynomial~$q$ if there exists a polynomial~$m$ such that~$p \cdot m = q$.
	If~$p$ divides~$q$ we denote this by~$p \vert q$.
	Divisibility defines an endorelation on~$\setA$ by saying~$p$ is related to~$q$ iff~$p \vert q$.
	Does this define a preorder structure on~$\setA$?
	Does this define a poset structure on~$\setA$?
	Justify your answer.
\end{gradedexercise}

\solutionof{PolynomialDivisibility}

\devel{
	\todo{To move somewhere else}

	\paragraph{A note on preorders}
	The theory of design problems can be easily generalized to preorders.
	This means that there could be two elements~$\posAel$ and~$\posBel$ such that~$\posAel\posAleq \posBel$ and~$\posAel \posAgeq \posBel$ but~$\posAel \neq \posBel$.

	This is actually common in practice.
	For example, if the order relation comes from human judgement, such as customer preference, all bets are off regarding the consistency of the relation.
	We will only refer to posets for two reasons:
	\begin{enumerate}
		\item The exposition is smoother.
		\item Given a pre-order, computation will always involve passing to the poset representation.
	\end{enumerate}
	This means that, given a pre-order, we can consider the poset of its isomorphism classes, by means of the following equivalence relation:
	\begin{equation}
		\posAel \simeq \posBel \quad \equiv \quad (\posAel \posAleq \posBel) \wedge (\posBel \posAleq \posAel).
	\end{equation}
}

\section{Counting orders}
\linkvideo{spring2021-tradeoffs:tradeoffs:orders:counting-orders} % Counting orders

Let's count the number of posets.

If there is only one element, there is only one way to order it (\cref{fig:singleton}).

With a 2-elements set, there are 2 posets (panel \emph{b}), ``up to isomorphism'',
that is, if we do not care about the labels of points.
.

On 3-elements sets, one has 5 posets (panel \emph{c}).

On 4-elements sets, one has 16 posets (panel \emph{d}).

\vfill

\begin{figure*}[h]
	% \hfill
	\subfloat[\label{fig:singleton}
		The singleton poset.
	]{
		\rule{2cm}{0pt}\middlesag{40_dpcatfig_singleton}\rule{2cm}{0pt}
	}
	% \hfill
	\subfloat[All posets on 2-elements sets, up to isomorphisms.
		\label{fig:twoelementspos}
	]{
		\centering
		\setlength{\tabcolsep}{20pt}
		\rule{1cm}{0pt}
		\begin{tabular}{cc}
			\middlesag{70_pos_2_1} & \middlesag{70_pos_2_2}
		\end{tabular}
		\rule{1cm}{0pt}
	}

	% \medskip
	\subfloat[All posets on 3-elements sets, up to isomorphisms.
		\label{fig:threeelementspos}]{
		\setlength{\tabcolsep}{20pt}
		\begin{tabular}{ccccc}
			\middlesag{70_pos_3_1} &
			\middlesag{70_pos_3_2} &
			\middlesag{70_pos_3_3} &
			\middlesag{70_pos_3_4} &
			\middlesag{70_pos_3_5}
		\end{tabular}
	}

	% \medskip
	\subfloat[
		All posets on a 4-element sets, up to isomorphism.
		\label{fig:fourelementspos}
	]{
		\centering
		\setlength{\tabcolsep}{6pt}
		\maxsizebox{\marginparwidth+\textwidth}{!}{
			\begin{tabular}{cccccc}
				\middlesag{70_pos_1}
				 & \middlesag{70_pos_2}
				 & \middlesag{70_pos_3}
				 & \middlesag{70_pos_4}
				 & \middlesag{70_pos_5}  \\[+30pt]
				\middlesag{70_pos_6}
				 & \middlesag{70_pos_7}
				 & \middlesag{70_pos_8}
				 & \middlesag{70_pos_9}
				 & \middlesag{70_pos_10} \\[+30pt]
				\middlesag{70_pos_11}
				 & \middlesag{70_pos_12}
				 & \middlesag{70_pos_13}
				 & \middlesag{70_pos_14}
				 & \middlesag{70_pos_15}
				 & \middlesag{70_pos_16}
			\end{tabular}
		}
	}
	\caption{}
	\label{fig:posets-up-to-four}
	\todographics{I had to use maxsizebox to make everything fit.
		However we could make it some figure more compact and not resort to it.
		Fir example, in panel d, the second, fourth, fifth posets cocan be formatted on two "floors".
		Then we can put all three-floors posets on the same line.
		Finally, the last poset could be shrunk so that it fits in the equivalent of 3 floors.
		Same think for the last poset of panel c.
	}
\end{figure*}
