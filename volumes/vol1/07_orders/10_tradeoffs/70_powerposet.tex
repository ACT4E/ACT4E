% !TEX root = chapter-standalone.tex

\section{Power poset}

We have introduced the concept of power set in \cref{sec:power-set}.
There is a natural order on subsets, given by set inclusion.
We can thus define the \emph{power poset}.

\SYNDEF{power poset}
\begin{definition}[Power poset]
    \label{def:power-poset}
    Given a set~\setA, define the \maindef{power poset}~$\powerposet\setA = \tup{\powerset\setA, {{\setsubseteq}}}$ by ordering the subsets in its power set~$\powerset\setA$ by inclusion.

    A subset~$\subA$ precedes~$\subB$ if~$\subA \setsubseteq \subB$:
    \begin{equation} \label{eq:pospower}
        \vmiddle{
            \prfdoubleperiod{
                \subA \posleqof{\powerposet\setA} \subB
            }{
                \subA \setsubseteq \subB
            }
        }
    \end{equation}
    % Composition is given by composition of inclusions, meaning that if~$\setA\setsubseteq \setB \setsubseteq \setC$, then~$\setA\setsubseteq \setC$.
\end{definition}
This is illustrated in~\cref{fig:powersetcat} for sets of 0, 1, 2, 3 elements.
\begin{exercise}
    Check formally that $\posleqof{\powerposet\setA}$ defined in \cref{eq:pospower} is a partial order.
\end{exercise}
\begin{solution}
    Consider a set~\setA.
    Clearly, given~$\subA\setin \powerset \setA$, we have~$\subA\setsubseteq \subA$.
    Furthermore, given also~$\subB\setin \powerset \setA$, we have
    \begin{equation}\label{eq:ex-pospower-1}
        \prfperiod{
            \subA\setsubseteq \subB
        }{
            \subB\setsubseteq \subA
        }{
            \subA=\subB
        }
    \end{equation}
    Finally, given also~$\subC\setin \powerset \setA$, we have
    \begin{equation}\label{eq:ex-pospower-2}
        \prfperiod{
            \subA\setsubseteq \subB
        }{
            \subB\setsubseteq \subC
        }{
            \subA\setsubseteq\subC
        }
    \end{equation}
\end{solution}
\vfill
\begin{figure*}[h]
    \centering
    \subfloat[$\powerposet\Emptyset$]{
        \includesag{40_dpcatfig_power_1}
    }
    \hfill
    \subfloat[$\powerposet\makeset{\posela}$]{
        \includesag{40_dpcatfig_power_2}
    }
    \hfill
    \subfloat[$\powerposet\makeset{\posela,\poselb}$]{
        \includesag{40_dpcatfig_power_3}
    }
    \hfill
    \subfloat[$\powerposet\makeset{\posela,\poselb,\poselc}$ \label{fig:powersetcat3}]{
        \includesag{40_dpcatfig_power}
    }
    \caption{Power set as a poset.
    }
    \label{fig:powersetcat}
\end{figure*}
\vfill
