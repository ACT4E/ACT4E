\section{Power poset}

We have introduced the concept of power set in \cref{sec:power-set}.
There is a natural order on subsets, given by set inclusion.
We can thus define the \emph{power poset}.

\begin{definition}[Power poset]
    \label{def:power-poset}
    Given a set~$\setA$,
    define the \emph{power poset} $\powerposet\setA = \tup{\powerset\setA,\subseteq}$ by ordering the subsets in its power set~$\powerset\setA$ by inclusion.

    A subset $\subA$ precedes $\subB$ if $\subA \subseteq \subB$:
    \begin{equation}\label{eq:pospower}
        \vmiddle{
            \prfdoubleperiod{
                \subA \posleqof{\powerposet\setA} \subB
            }{
                \subA \subseteq \subB
            }
        }
    \end{equation}
    % Composition is given by composition of inclusions, meaning that if~$\setA\subseteq \setB \subseteq \setC$, then~$\setA\subseteq \setC$.
\end{definition}
This is illustrated in~\cref{fig:powersetcat} for sets of 0, 1, 2, 3 elements.
\begin{exercise}
    Check formally that $\posleqof{\powerset\setA}$ defined in \cref{eq:pospower} is a partial order.
\end{exercise}
\begin{solution}
    Consider a set~$\setA$.
    Clearly, given~$\subA\in \powerset \setA$, one has~$\subA\subseteq \subA$.
    Furthermore, given also~$\subB\in \powerset \setA$, one has
    \begin{equation*}
        \prfperiod{\subA\subseteq \subB}{\subB\subseteq \subA}{\subA=\subB}
    \end{equation*}
    Finally, given also~$\subC\in \powerset \setA$, one has
    \begin{equation*}
        \prfperiod{\subA\subseteq \subB}{\subB\subseteq \subC}{\subA\subseteq\subC}
    \end{equation*}
\end{solution}
\vfill
\begin{figure*}[h]
    \centering
    \subfloat[$\powerposet\emptyset$]{
        \includesag{40_dpcatfig_power_1}
    }
    \hfill
    \subfloat[$\powerposet\{\posela\}$]{
        \includesag{40_dpcatfig_power_2}
    }
    \hfill
    \subfloat[$\powerposet\{\posela,\poselb\}$]{
        \includesag{40_dpcatfig_power_3}
    }
    \hfill
    \subfloat[$\powerposet\{\posela,\poselb,\poselc\}$]{
        \includesag{40_dpcatfig_power}
    }
    \caption{Power set as a poset.
    }
    \label{fig:powersetcat}
\end{figure*}\
\vfill
