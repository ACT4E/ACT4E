\section{Power poset}

We have introduced the concept of power set in \cref{def:power-set}.

\begin{definition}[Power poset]
    \label{ex:hasseinclusion}
    \todotext{Use \str{\posela}, \str{\poselb}, etc. also in the figure \str{40_dpcatfig_power}.}
    Given a set~$\setA$, consider its power set~$\powerset\setA$.
    We can see the power set as a poset where a subset $\subA$ precedes $\subB$ if $\subA \subseteq \subB$:
    \begin{equation}\label{eq:pospower}
        \prfdoubleperiod{
            \subA \posleqof{\powerset\setA} \subB
        }{
            \subA \subseteq \subB
        }
    \end{equation}
    This is illustrated in~\cref{fig:powersetcat} for a set of 3 elements.
    % Composition is given by composition of inclusions, meaning that if~$\setA\subseteq \setB \subseteq \setC$, then~$\setA\subseteq \setC$.
\end{definition}
\begin{exercise}
    Check formally that $\posleqof{\powerset\setA}$ defined in \cref{eq:pospower} is a partial order.
\end{exercise}
\begin{solution}
    Consider a set~$\setA$.
    Clearly, given~$\subA\in \powerset \setA$, one has~$\subA\subseteq \subA$.
    Furthermore, given also~$\subB\in \powerset \setA$, one has
    \begin{equation*}
        \prfperiod{\subA\subseteq \subB}{\subB\subseteq \subA}{\subA=\subB}
    \end{equation*}
    Finally, given also~$\subC\in \powerset \setA$, one has
    \begin{equation*}
        \prfperiod{\subA\subseteq \subB}{\subB\subseteq \subC}{\subA\subseteq\subC}
    \end{equation*}
\end{solution}

\begin{figure*}[h]
    \centering
    \subfloat[$\powerset\emptyset$]{
        \includesag{40_dpcatfig_power}
    }
    \subfloat[$\powerset\{\posela\}$]{
        \includesag{40_dpcatfig_power}
    }
    \subfloat[$\powerset\{\posela,\poselb\}$]{
        \includesag{40_dpcatfig_power}
    }
    \subfloat[$\powerset\{\posela,\poselb,\poselc\}$]{
        \includesag{40_dpcatfig_power}
    }
    \caption{Power set as a poset.
    }
    \label{fig:powersetcat}
    \todographics{Make the missing figures. Use $\posela$ etc.}
\end{figure*}

