% !TEX root = chapter-standalone.tex

\section{Trade-offs}
%\linkvideo{spring2021-tradeoffs:tradeoffs} % Trade-offs

\begin{marginfigure}
    \centering
    \includesag{070_fast_good_cheap}
    \caption{}
    \label{fig:fast_good_cheap}
\end{marginfigure}

Trade-offs characterize all engineering disciplines, and can be literally found everywhere.
A typical trade-off is the one reported in \cref{fig:fast_good_cheap}.
When designing a product, you want it to be \emph{good}, \emph{fast}, and \emph{cheap}, and typically you can just choose between two of these qualities.
Indeed, if a product is fast and good, it will not be cheap, if it is good and cheap, it will not be fast, and if it is cheap and fast, it will not be good.

To characterize engineering trade-offs, we will use the mathematical structure of partial orders.
In the next section, we will explore some examples, to better contextualize trade-offs.

\subsection{The 3 diagrams}

%\devel{\includepdf[scale=0.8,pages={10},nup=1x3,frame,pagecommand={}]{ACT4E-06-posets.pdf}} % diagram shapes}

In this section, we introduce concepts which will be important throughout the book, when talking about theories of design.
We distinguish semantically between \textF{functionalities} and \textR{requirements/costs}.
In general, you prefer \textF{functionalities} to be ``large'' (\cref{fig:fun_large}) and \textR{requirements/costs} to be ``small'' (\cref{fig:res_small}).

\begin{figure*}[h]
    \begin{center}
        \subfloat[
            \label{fig:res_small}
            Requirements/costs.
        ]{
            \centering
            \includesag{res_small}
        }
        \subfloat[
            \label{fig:fun_large}
            Functionality.
        ]{
            \centering
            \includesag{fun_large}
        }
    \end{center}
    \caption{}
\end{figure*}

We think of three achievable accuracy plots (\cref{fig:accu_res_fun}).

In \cref{fig:accu_1_a} we plot trade-offs in costs and add a ``feasibility'' curve.

\begin{figure*}[b]
    \centering
    \subfloat[
        \label{fig:accu_1_a}
        Costs.
    ]
    {\includesag{accu_1}}
    \hspace{0.5cm}
    \subfloat[
        \label{fig:accu_1_b}
        Functionalities.
    ]

    {\includesag{accu_2}}
    \hspace{0.5cm}
    \subfloat[
        \label{fig:accu_1_c}
        Functionality vs.
        costs.]
    {\includesag{accu_3}}
    \caption{}
    \label{fig:accu_res_fun}
\end{figure*}

Everything above this curve is feasible and will cost more than what is \emph{on} the curve.
In \cref{fig:accu_1_b} we plot trade-offs in functionalities and add a ``feasibility'' curve.
Everything below the curve is feasible, but is below the ``standards'' required by the curve.
Finally, in \cref{fig:accu_1_c} we plot functionality and resource together, representing the trade-offs between ``how good a product is'' and ``how much one needs to pay for it''.
Feasibly pairs are represented via the feasibility curve.
Everything above the curve will be feasible (by paying more).
A good exercise would be to open any engineering book, find the graphs talking about ``achievable'' performance and ``resources'' needed, and classify into one of the ones reported in \cref{fig:accu_res_fun}.
In the following we will have a careful look at specific examples involving trade-offs.

\vfill
\clearpage

\subsection{Trade-offs for the human body}

We need to see the difference in physique.
A good example for trade-offs for the human body is given by sports.
Indeed, when looking at different disciplines, various physical abilities are desired and trade-offs between them characterize athletes.
For instance, we can think about trade-offs between \emph{speed} and \emph{strength} for humans (\cref{fig:fast_strong}).

\begin{marginfigure}
    \centering
    \includesag{fast_strong}
    \caption{}
    \label{fig:fast_strong}
\end{marginfigure}

These are functionalities, which different athletes might want to maximize.
Consider Usain Bolt, who owns the 100 meters, 200 meters, and~$4\times 100$ meters relay world records.
Without doubts, in the human speed-strength trade-off curve he positions himself close to the highest achievable speeds.
At the same time, however, Usain Bolt is not among the strongest men in the world.
To see the other end of the curve, we need to introduce Oleksii Novikov, who won the 2020 World's Strongest Man competition.
Similarly to Bolt, he is among the best in his discipline, reaching very high strength.
Again, the speed-strength trade-off implies that Oleksii cannot be among the fastest men in the world, if he wants to be among the strongest ones.
For this specific case, we can think of resources that make functionalities possible.
In sports, resources are typically related to training (\cref{fig:strong_training}).

\begin{marginfigure}
    \centering
    \includesag{strong_training}
    \caption{}
    \label{fig:strong_training}
\end{marginfigure}

If we want to relate the invested training and the resulting strength reached by the athletes, we will notice that with a lot of training, Novikov will improve his results, approaching perfection.
On the other hand, the kind of training Bolt undergoes is not optimizing strength, and therefore his results will be less effective.

%\devel{\includepdf[scale=0.8,pages={11,12},nup=1x3,frame,pagecommand={}]{ACT4E-06-posets.pdf}} % human body}

\subsection{Masks}

Orders give us a rich way to describe products under various lenses.
Recently, we all needed to become experts of protective masks.
In this section, we will show various ways in which one can order the latter by functionality.
By first thinking about the effectiveness of the mask in protecting the wearer from a virus, we can order masks as in \cref{fig:masks_covid}.

\begin{figure}[h!]
    \centering
    \includesag{masks_covid}
    \caption{}
    \label{fig:masks_covid}
\end{figure}

In general masks are classified following their filter abilities and inward leakages.
The FFP1 class filters at least \unit[80]{\%} of airborne particles and allows less than \unit[22]{\%} inward leakage.
The FFP2 class filters at least \unit[96]{\%} of airborne particles and allows less than \unit[8]{\%} inward leakage, and the FFP3 class filters at least \unit[99]{\%} of airborne particles and allows less than \unit[2]{\%} inward leakage

Obviously, based on the protection level, the most performant in \cref{fig:masks_covid} is FFP3, and the worst is the fashion one.
However, this is not the only way in which we can classify masks.
If, for instance, we want to consider a functionality ``how much does the mask say about the wearer'', we can order the masks differently.
Arguably, the ordering could look like the one in \cref{fig:masks_expressive}.

\begin{figure}[h!]
    \centering
    \includesag{masks_expressive}
    \caption{}
    \label{fig:masks_expressive}
\end{figure}

Indeed, choosing a fashion mask might say that the wearer cares more about aesthetics than safety, and choosing a FFP3 highlights responsible behaviors, care, and research in masks models.

Similarly, one could order masks based on different performance criteria, adding the functionality ``how much does it protect others?
'' (\cref{fig:masks_others}).

\begin{figure}[h!]
    \centering
    \includesag{masks_others}
    \caption{}
    \label{fig:masks_others}
\end{figure}

On the other hand, one could think about the trade-offs between the mask performance and its cost, presenting a functionality-resource plot (\cref{fig:masks_price}).

\begin{figure}[h!]
    \centering
    \includesag{masks_price}
    \caption{}
    \label{fig:masks_price}
\end{figure}

More performant masks are typically more expensive, and the fashion mask will be probably the least performance and most expensive.

This example once again highlights the flexibility and richness of the ``orders approach''.
This will be much more evident in the next example.

%\devel{\includepdf[scale=0.8,pages={13-16},nup=1x3,frame,pagecommand={}]{ACT4E-06-posets.pdf} }% masks}

\subsection{Hats and headphones}
Another good example of ordering of multiple functionalities and costs is the one of headphones.
Consider a set of headphones and let's order them based on their abilities to ``keep warm'' and to ``reproduce music'' (\cref{fig:headpho_fun}).

\begin{figure}[h!]
    \centering
    \includesag{headpho_fun}
    \caption{}
    \label{fig:headpho_fun}
\end{figure}

Clearly, these two functionalities represent different objectives and diverse product ranges will satisfy them in different ways.
For instance, winter hats clearly cannot reproduce music, but keep very warm.
On the other hand, large headphones are the best in reproducing music, but cannot keep as warm as winter hats.
Functionalities come at a cost.
For instance, we could plot the trade-off between ``keep warm'' and price (\cref{fig:headpho_price}).

\begin{figure}[h!]
    \centering
    \includesag{headpho_price}
    \caption{}
    \label{fig:headpho_price}
\end{figure}
\vfill
\clearpage

Other interesting costs could be expressed via the frequency of charging (\cref{fig:headpho_charge}):

\begin{figure}[h!
    ]
    \centering
    \includesag{headpho_charge}
    \caption{}
    \label{fig:headpho_charge}
\end{figure}

or the hassle of dealing with wires (\cref{fig:headpho_wires}):

\begin{figure}[h!]
    \centering
    \includesag{headpho_wires}
    \caption{}
    \label{fig:headpho_wires}
\end{figure}

\subsection{The law of successful products}
It is interesting to notice that by considering all the aforementioned characteristics
\begin{equation*}
    (\textF{keeps warm}\cartprod \textF{music fidelity})
    \cartprod (\textR{price}\cartprod \textR{frequency of charging}\cartprod \textR{wires hassle}),
\end{equation*}
no product dominates another.
This is also known as the \emph{law of successful products}.
At equilibrium, in an efficient and free market, no product completely dominates another by both functionality and costs.
Otherwise, the dominated product would not sell.
Once we \emph{specify the design purpose} and the related constraints, we can (partially) order things.

%\devel{\includepdf[scale=0.8,pages={17-21},nup=1x3,frame,pagecommand={}]{ACT4E-06-posets.pdf}} % producs
