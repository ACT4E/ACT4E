% !TEX root = chapter-standalone.tex
\section{Counting orders}

\linkvideo{spring2021-tradeoffs:tradeoffs:orders:counting-orders} % Counting orders

Let's count the number of \SY{posets}.

If there is only one element, there is only one way to order it (\cref{fig:singleton}).

With a 2-elements set, there are 2 \SY{posets} (panel \emph{b}), ``up to isomorphism'',
that is, if we do not care about the labels of points.

On 3-elements sets, we have 5 \SY{posets} (panel \emph{c}).

On 4-elements sets, we have 16 \SY{posets} (panel \emph{d}).

\vfill

\begin{figure}[h]
    % \hfill
    \subfloat[\label{fig:singleton}
        The singleton poset.
    ]{
        \rule{2cm}{0pt}\middlesag{40_dpcatfig_singleton}\rule{2cm}{0pt}
    }
    % \hfill
    \subfloat[All \SY{posets} on 2-elements sets, up to isomorphism.
        \label{fig:twoelementspos}
    ]{
        \centering
        \setlength{\tabcolsep}{20pt}
        \rule{1cm}{0pt}
        \begin{tabular}{cc}
            \middlesag{70_pos_2_1} & \middlesag{70_pos_2_2}
        \end{tabular}
        \rule{1cm}{0pt}
    }

    % \medskip
    \subfloat[All \SY{posets} on 3-elements sets, up to isomorphisms.
        \label{fig:threeelementspos}]{
        \setlength{\tabcolsep}{20pt}
        \begin{tabular}{ccccc}
            \middlesag{70_pos_3_1} &
            \middlesag{70_pos_3_2} &
            \middlesag{70_pos_3_3} &
            \middlesag{70_pos_3_4} &
            \middlesag{70_pos_3_5}
        \end{tabular}
    }

    % \medskip
    \subfloat[
        All \SY{posets} on a 4-element set, up to isomorphism.
        \label{fig:fourelementspos}
    ]{
        \centering
        \setlength{\tabcolsep}{6pt}
        \maxsizebox{\marginparwidth+\textwidth}{!}{
            \begin{tabular}{cccccc}
                \middlesag{70_pos_1}
                 & \middlesag{70_pos_2}
                 & \middlesag{70_pos_3}
                 & \middlesag{70_pos_4}
                 & \middlesag{70_pos_5} \\[+30pt]
                \middlesag{70_pos_6}
                 & \middlesag{70_pos_7}
                 & \middlesag{70_pos_8}
                 & \middlesag{70_pos_14}
                 & \middlesag{70_pos_15} \\[+30pt]
                \middlesag{70_pos_11}
                 & \middlesag{70_pos_12}
                 & \middlesag{70_pos_13}
                 & \middlesag{70_pos_9}
                 & \middlesag{70_pos_10}
                 & \middlesag{70_pos_16}
            \end{tabular}
        }
    }
    \caption{}
    \label{fig:posets-up-to-four}
\end{figure}
