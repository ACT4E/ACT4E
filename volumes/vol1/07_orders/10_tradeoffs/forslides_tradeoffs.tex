\showslides{

\section{Lecture slides materials}

\begin{forslides}
    \subsection{Section: Tradeoffs}

    \begin{equation}
        \label{eq:tradeoff_map}
        \mapa \colon \stylesets{X} \sto \stylesets{Y}
    \end{equation}
    \begin{equation}
        \label{eq:tradeoff_mor}
        \mora \colon \Obja \mto \Objb
    \end{equation}
    \begin{equation}
        \label{eq:tradeoff_typw}
        \mora \colon (\Obja \mto \Objb)
    \end{equation}
    \begin{equation}
        \label{eq:tradeoff_cap}
        \begin{aligned}
            \setA & \setintersection \setB \\
            \setA & \setunion \setB \\
            \setA & \setsubseteq \setB
        \end{aligned}
    \end{equation}
    \begin{equation}
        \label{eq:tradeoff_wedge}
        \begin{aligned}
            a & \booland b \\
            a & \boolor b \\
        \end{aligned}
    \end{equation}
    \begin{equation}
        \label{eq:tradeoff_leq}
        \setA \leq \setB
    \end{equation}
    \begin{equation}
        \label{eq:abc}
        \prflineextra=0.6em
        \prftree
        {\prftree
            {a}
            {b}}
        {c}
    \end{equation}
    \begin{equation}
        \label{eq:abc_2}
        \begin{aligned}
                       & a \\
            \Imp \quad & b \\
            \Imp \quad & c
        \end{aligned}
    \end{equation}
    \begin{equation}
        \label{eq:abc_3}
        \prftree{a}{b}{c}
    \end{equation}
    \begin{equation}
        \label{eq:abc_4}
        (a\booland b)\Imp c
    \end{equation}
    \begin{equation}
        \label{eq:abc_5}
        \prflineextra=0.6em
        \prftree[d]
        {\prftree[d]
            {a}
            {b}}
        {c}
    \end{equation}
    \begin{equation}
        \label{eq:abc_6}
        a\Leftrightarrow b \Leftrightarrow c
    \end{equation}
    \begin{equation}
        \label{eq:abc_7}
        \prflineextra=0.6em
        \prftree{a}{b}
    \end{equation}
    \begin{equation}
        \label{eq:abc_8}
        \prflineextra=0.6em
        \prftree{b}{c}
    \end{equation}
    \begin{equation}
        \label{eq:abc_9}
        \prflineextra=0.6em
        \prftree{a}{c}
    \end{equation}
    \begin{equation}
        \label{eq:mapf}
        \mapa \colon
    \end{equation}
    \begin{equation}
        \label{eq:mapg}
        \mapb \colon
    \end{equation}
    \begin{equation}
        \label{eq:mapcomp}
        \mapa \mthen \mapb \colon
    \end{equation}
    \begin{equation}
        \label{eq:top1}
        \prflineextra=0.6em
        \prftree{\true}{a}
    \end{equation}
    \begin{equation}
        \label{eq:bot1}
        \prflineextra=0.6em
        \prftree{a}{\false}
    \end{equation}
    \begin{equation}
        \label{eq:rel_refl}
        \prflineextra=0.6em
        \prftree{\true}{a\relA a}
    \end{equation}
    \begin{equation}
        \label{eq:rel_tot}
        \prflineextra=0.6em
        \prftree{\true}{a\relA b \quad \vee \quad b\relA a}
    \end{equation}
    \begin{equation}
        \label{eq:rel_tran}
        \prflineextra=0.6em
        \prftree{a\relA b}{b\relA c}{a\relA c}
    \end{equation}
    \begin{equation}
        \label{eq:rel_sym}
        \prflineextra=0.6em
        \prftree[d]{a \relA b}{b\relA a}
    \end{equation}
    \begin{equation}
        \label{eq:rel_irr}
        \prflineextra=0.6em
        \prftree{a \relA a}{\false}
    \end{equation}
    \begin{equation}
        \label{eq:rel_asy}
        \prflineextra=0.6em
        \prftree{a \relA b}{b\relA a}{\false}
    \end{equation}
    \begin{equation}
        \label{eq:rel_antisy}
        \prflineextra=0.6em
        \prftree{a \relA b}{b\relA a}{a=b}
    \end{equation}
    \begin{equation}
        \label{eq:ex}
        a\relA b \Leftrightarrow (a-b) \text{ mod }3=0
    \end{equation}
    \begin{equation}
        \label{eq:fun_cost}
        (\F{\text{keeps warm }}\Ptimes \F{\text{ music fidelity}})\cartprod(\R{\text{ price }} \Ptimes \R{\text{ frequency of charging }}\Ptimes \R{\text{ wires hassle}})
    \end{equation}
    \begin{equation}
        \label{eq:id_ex}
        \prftree{\true}{\catidat\Obja\colon \Obja \mto \Obja}
    \end{equation}
    \begin{equation}
        \label{eq:comp_cat}
        \prftree{\mora \colon \Obja \mto \Objb}{\morb\colon \Objb \mto \Objc}{\morab \colon \Obja \mto \Objc}
    \end{equation}
    \begin{equation}
        \label{eq:mor_pre}
        \prftree[d]{\ast\colon \posAel \mto \posBel}{\posAel \posleq \posBel}
    \end{equation}
    \begin{equation}
        \label{eq:hom_pre}
        \HomSet{}{\posAel}{\posBel}=\begin{cases}
            \singleton & \text{iff} \posAel \posleq \posBel \\
            \Emptyset  & \text{otherwise}
        \end{cases}
    \end{equation}
    \begin{equation}
        \label{eq:comp_pre}
        \singletonel \mthen \singletonel = \singletonel
    \end{equation}
    \begin{equation}
        \label{eq:ex_pre}
        \begin{aligned}
             & \posAel,\posBel,\posCel \\
             & \posAel \posleq \posBel \\
             & \posCel \posleq \posBel
        \end{aligned}
    \end{equation}
    \begin{equation}
        \label{eq:pre_p}
        \posAel
    \end{equation}
    \begin{equation}
        \label{eq:pre_q}
        \posBel
    \end{equation}
    \begin{equation}
        \label{eq:pre_r}
        \posCel
    \end{equation}
    \begin{equation}
        \label{eq:eq_rel}
        \prftree[d]{\posAel\sim \posBel}{(\posAel \posleq \posBel)\wedge (\posBel \posleq \posAel)}
    \end{equation}
    \begin{equation}
        \label{eq:pos_int}
        \prflinepadbefore=0.6em
        \prflinepadafter=0.6em
        \prftree[d]{\tup{l_1,u_1}\posleqof{\styleelements{\text{Int}}\posA} \tup{l_2,u_2}}{(l_1\posAleq l_2)\wedge (u_2\posAleq u_1)}
    \end{equation}
    \begin{equation}
        \label{eq:leq}
        \posleq
    \end{equation}
    \begin{equation}
        \label{eq:pos_prod}
        \prflinepadbefore=0.6em
        \prflinepadafter=0.6em
        \prftree[d]{\tup{\posAel_1,\posBel_1}\posleqof{\posA\Ptimes \posB} \tup{\posAel_2,\posBel_2}}{(\posAel_1\posAleq \posAel_2)\wedge(\posBel_1\posBleq \posBel_2)}
    \end{equation}
    \begin{equation}
        \label{eq:pos_P}
        \posA
    \end{equation}
    \begin{equation}
        \label{eq:pos_Q}
        \posB
    \end{equation}
    \begin{equation}
        \label{eq:pos_PQ}
        \posA\Ptimes \posB
    \end{equation}
    \begin{equation}
        \label{eq:set_disj}
        \posA\Pplus\posB=\makeset{ \tup{1,\posAel} \mid \posAel\setin \posAset }\setunion \makeset{\tup{2,\posBel}\mid \posBel \setin \posBset}
    \end{equation}
    \begin{equation}
        \label{eq:latt_1}
        \posAel \wedge \posBel
    \end{equation}
    \begin{equation}
        \label{eq:latt_2}
        \posAel \vee \posBel
    \end{equation}
    \begin{equation}
        \label{eq:latt_3}
        (\posAel \wedge \posBel)\posleq \posAel \posleq (\posAel \vee \posBel)
    \end{equation}
    \begin{equation}
        \label{eq:latt_4}
        \posAel
    \end{equation}
    \begin{equation}
        \label{eq:latt_5}
        \posBel
    \end{equation}
    \begin{equation}
        \label{eq:prop_1}
        \prftree[d]{\posAel \posleq \posBel}{\posAel \Imp \posBel}
    \end{equation}
    \begin{equation}
        \label{eq:prop_2}
        \prftree[d]{\posAel \Leftrightarrow \posBel}{\posAel = \posBel}
    \end{equation}
    \begin{equation}
        \label{eq:prop_3}
        \prftree{\true}{\posAel \Imp \posAel}
    \end{equation}
    \begin{equation}
        \label{eq:prop_4}
        \prftree{(\posAel \Imp \posBel)\wedge (\posBel \Imp \posCel)}{\posAel \Imp \posCel}
    \end{equation}
    \includesag{060_propositions_lat}
    \begin{equation}
        \label{eq:pow_1}
        \setA \posleq \setB \definedas \setA \setsubseteq \setB
    \end{equation}
    \begin{equation}
        \label{eq:pow_2}
        \setA \vee \setB \definedas \setA \setunion \setB
    \end{equation}
    \begin{equation}
        \label{eq:pow_3}
        \setA \wedge \setB \definedas \setA \setintersection \setB
    \end{equation}
    \begin{equation}
        \label{eq:pow_4}
        \postop \definedas \stylesets{S}
    \end{equation}
    \begin{equation}
        \label{eq:pow_5}
        \posbot \definedas \stylesets{\Emptyset}
    \end{equation}
    \begin{equation}
        \label{eq:pow_6}
        \powerset{\stylesets{S}}
    \end{equation}
    \begin{equation}
        \label{eq:pow_7}
        \stylesets{S}
    \end{equation}
    \begin{equation}
        \label{eq:pre_ex1}
        \makeset{\tup{a,a}, \tup{a,b}, \tup{b,a}, \tup{b,b}}
    \end{equation}
    \begin{equation}
        \label{eq:pre_ex2}
        \makeset{\tup{a,b}, \tup{b,c}, \tup{c,a}}
    \end{equation}
    \begin{equation}
        \label{eq:pre_ex3}
        \makeset{\tup{a,a}, \tup{b,b}, \tup{c,c}, \tup{d,d}}
    \end{equation}
    % \begin{definition}
    %
    %     A symmetric matrix~$\mat{M}\setin \reals^{\ntimesn}$ is \emph{positive semi-definite} if~$x\mattransp \mat{M}x\geq 0$ for all non-zero~$x\setin \reals^n$.
    %     We call the set of all such matrices~$\mathcal{P}^n$.
    % \end{definition}
    \begin{equation}
        \label{eq:posdef_1}
        \mat{A}\posleq \mat{B} \Leftrightarrow (\mat{A}-\mat{B})\setin \mathcal{P}^n, \quad \mat{A},\mat{B}\setin \mathcal{P}^n
    \end{equation}

\end{forslides}

}