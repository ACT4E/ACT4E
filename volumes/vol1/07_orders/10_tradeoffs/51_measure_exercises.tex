% !TEX root = chapter-standalone.tex
\sectionexercises{Measuring posets}
\codeboilerplate{FinitePosetMeasurement}{
    Compute the width and height of a poset.

}

\classlisting{FinitePosetMeasurement}

\begin{hint}
    For computing the height of a finite (small) poset, it is easiest to just enumerate the \SY{chains} and check which one is the longest.
    You can enumerate the \SY{chains} by looking at the paths from the minimal to the maximal elements.
\end{hint}

Note: There is one test case for a large poset.
This will probably time out for a naive algorithm.
This timeout is not counted as failure for the purpose of the grading.

\begin{hint}
    Computing the \emph{width} of a \SY{poset} is more complicated.

    Note that enumerating all subsets and check which one is an \SY{antichain} and record the largest we have complexity~$2^n$, so it is impracticable even for small sets.

    We mention here two results that can help.

    Dilwort's theorem shows that the \emph{width} of a \SY{poset} (size of maximum \emph{antichain}) relates to a \SY{partition} in \emph{chains}, while Mirksy's theorem shows that the \emph{height} of a \SY{poset} (size of the longest \emph{chain}) relates to a \SY{partition} in \emph{antichains}.

    \begin{theorem}[Dilwort's theorem]
        \label{thm:dilwort}
        A finite \SY{poset} of width $w$ can be partitioned into $w$ \SY{chains}.
    \end{theorem}

    \begin{theorem}[Mirsky's theorem]
        \label{thm:mirsky}
        A finite \SY{poset} of height $h$ can be partitioned into $h$ \SY{antichains}.
    \end{theorem}

    Given Dilwort's theorem, one approach that works for small \SY{posets} is to enumerate the \SY{chains} and then find the smallest subset that covers the poset.

    There are many results published that give faster algorithms; see, \eg, ~\cite{chen12decomposition}.

\end{hint}

