% !TEX root = chapter-standalone.tex

\section{More general preferences}

\linkvideo{spring2021-tradeoffs:tradeoffs:orders:general-pref} % More general preference structures

There are more general ways to describe preferences, which might be useful in certain applications.

We give two such examples.

\begin{definition}[Quasitransitive relation]\label{def:quasitransitive-relation}
    A \maindef{quasitransitive relation} is a relation~$\posAleq$ over a set~$\posAset$ for which
    \begin{equation}\label{eq:quasitransitive}
        \prfperiod{
            (\posAel \posAleq \posBel)
        }{
            \neg(\posBel \posAleq \posAel)
        }{
            (\posBel \posAleq \posCel)
        }{
            \neg(\posCel \posAleq \posBel)
        }{
            (\posAel \posAleq \posCel) \booland \neg(\posCel \posAleq \posAel)
        }
    \end{equation}
\end{definition}

Semiorders are particular quasitransitive relations.
\begin{definition}[Semiorder]\label{def:semiorder}
    A \maindef{semiorder} is a relation~$\posAleq$ over a set~$\posAset$ such that:
    \begin{itemize}
        \item \emph{Asymmetry} (not antisymmetry) holds:
              \begin{equation}\label{eq:semiorder-asym}
                  \prfperiod{
                      \posAel \posAleq \posBel
                  }{
                      \neg(\posBel \posAleq \posAel)
                  }
              \end{equation}
        \item Denote two elements~$\posBel, \posCel \setin \posAset$ which are \emph{incomparable} by~$\posBel \sim \posCel$.
              We have:
              \begin{equation}\label{eq:el-incomparable}
                  \prfperiod{
                      \posAel \posAleq \posBel
                  }{
                      \posBel \sim \posCel
                  }{
                      \posCel \posAleq \posDel
                  }{
                      \posAel \posAleq \posDel
                  }
              \end{equation}
        \item \emph{Semi-transitivity}:
              \begin{equation}\label{eq:semiorder-2}
                  \prfperiod{
                      \posAel \posAleq \posBel
                  }{
                      \posBel \posAleq \posCel
                  }{
                      (\posAel \posAleq \posDel) \boolor (\posDel \posCleq \posCel)
                  }
              \end{equation}
    \end{itemize}

\end{definition}
\todotextjira{731}{\bernina: @Andrea: Not really clear explanation.
    Also what is the source? And: the symbol "~" is usually used for equivalence relations.
}
\todographics{\bernina: Add diagram like the one in the wikipedia page about semiorders.
}
\begin{example}[Interval semi-orders: expressing tolerances]
    An example of semiorder is the following.
    Say that you have to express your preference over numbers in~$\posA=\makeset{10,11,12}$.
    You are indifferent between 10 and 11, and between 11 and 12.
    However, you prefer 10 to 12.
    This order satisfies the condition for it to be a semiorder.

    More in general, suppose that, when evaluating the cost of a widget, represented as a nonnegative real number, you are indifferent to prices within 1\% of the cost.
    For two real numbers $x,y\setin\reals$, you are indifferent if
    \begin{equation}\label{eq:semiorder-tolerances}
        \prfdoubleperiod{
            x \sim y
        }{
            \frac{| x - y |}{ (x+y)/2} \leq 0.01
        }
    \end{equation}
    This is also a semiorder.
\end{example}

\todotextjira{471}{\bernina: @Andrea: properties/examples of semi-orders, quasitransitive relations}
