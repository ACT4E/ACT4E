% !TEX root = chapter-standalone.tex

\section{Chains and Antichains}
\label{sec:chains-antichains}

There are two special types of subsets of a poset: chains and antichains.
Their definitions are dual.

\begin{definition}[Chain in a poset]
    \label{def:chain}
    Given a poset~$\posA = \tup{\posAset, \posAleq}$, a \emph{chain} is a subset~$\stylesets{S} \subseteq \posAset$ such that any two elements of~$\stylesets{S}$ are comparable:
    \begin{equation}
        \prfperiod{
            \setAel,\setBel \in \subA
        }{
            (\setAel \posAleq  \setBel) \boolor (\setBel \posAleq  \setAel)
        }
    \end{equation}
    %
    % If~$\ela, \elb \in \stylesets{S}$, then~$\ela \posAleq \elb$ or~$\elb \posAleq \ela$:
    %a sequence of elements~$
    %{\posAel_i}$ in~$\posA$ where two successive elements are comparable:
    %    \begin{equation}
    %        \prftree{i \Nleq j }{\posAel_i \posAleq \posAel_j}
    %        %i \Nleq j \Rightarrow s_i \posAleq s_j.
    %    \end{equation}
\end{definition}

\begin{definition}[Antichain in a poset]
    \label{def:antichain}
    An \emph{antichain} is a subset~$\subA$ of a poset where no two distinct elements are comparable:
    %
    \begin{equation}
        \prfperiod{
            \setAel,\setBel \in \subA
        }{
            \setAel \posAleq \setBel
        }{
            \setAel = \setBel
        }
    \end{equation}
    %
\end{definition}
\begin{remark}
    Note that the empty set~$\emptyset$ is both a chain and an antichain.
\end{remark}

We denote the set of antichains of a poset~$\posA$ by~$\antichains\posA$.

\begin{example}[Chains and antichains in power set]
    Consider the poset in~\cref{fig:powersetcat}.
    \todotext{use \str{posela} etc.}
    Examples of chains are
    \begin{equation}
        \{\varnothing,\{\setAel\},\{\setAel,\setBel\},\{\setAel,\setBel,\setCel\}\}
        \qqand
        \{\varnothing,\{\setBel\},\{\setBel,\setCel\},\{\setAel,\setBel,\setCel\}\}.
    \end{equation}
    Examples of antichains are
    \begin{equation}
        \{\{\setAel\},\{\setBel\},\{\setCel\}\}
        \qqand
        \{ \{\setAel,\setBel\},\{\setAel,\setCel\}, \{\setBel,\setCel\}\}.
    \end{equation}

    \todographics{Make pictures for these two. }
\end{example}

\begin{marginfigure}
    \centering
    \includesag{70_antichain}
    \caption{Example of discrete antichains.}
    \label{fig:antichain}
    \todographics{Show the principal upper set with dashed boundaries for the two points in the antichain}
\end{marginfigure}

In the context of battery choices, consider the diagram reported in~\cref{fig:antichain}.
The black markers represent an antichain of choices
\begin{equation}
    \{
    \tupp{\unit[10]{\standardcurrency},\unit[500]{g}},
    \tupp{\unit[20]{\standardcurrency},\unit[250]{g}}
    \}.
\end{equation}
It is a set of antichains because they do not dominate each other: one is cheaper, but takes longer, and the other is more expensive, but quicker, making them uncomparable.
The red marker~$\textcolor{red}{\marker}$ is an eleemnt that cannot be part of the antichain, since it is dominated by~$\tupp{\unit[20]{\standardcurrency},\unit[250]{g}}$.

%
\begin{marginfigure}
    \centering
    \includesag{70_antichain_2}
    \caption{Example of continuous antichains.}
    \label{fig:antichain_2}
    \todographics{Pick two points and show the principal upper set with dashed boundaries for the two points in the antichain}
\end{marginfigure}

Similarly, one could think of a continuous law which relates battery cost and mass.
For instance, consider the antichain given by~$\text{mass}=500-25\cdot \text{cost}$, with maximum possible cost~$\unit[20]{\standardcurrency}$ (\cref{fig:antichain_2}).

We will now have a look at couple more examples.
\begin{example}
    Consider the poset~$\posA=\tup{\posAset,\posAleq}$ where~$(\posAel \posAleq \posBel)$ if~$\posAel$ is a divisor of~$\posBel$ and~$\posAset=\{1,5,10,11,13,15\}$.
    A chain of~$\posA$ is~$\{1,5,10,15\}$.
    An antichain of~$\posA$ is~$\{10,11,13\}$.
    \todotext{This example is worthless without a picture. (can you picture the poset in your mind?) Thinking of dropping}
\end{example}

