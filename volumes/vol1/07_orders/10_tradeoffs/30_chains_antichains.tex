% !TEX root = chapter-standalone.tex

\section{Chains and Antichains}
\label{sec:chains-antichains}

There are two special types of subsets of a poset: chains and antichains.
Their definitions are dual.

\begin{definition}[Chain in a poset]
    \label{def:chain}
    Given a poset~$\posA = \tup{\posAset, \posAleq}$, a \emph{chain} is a subset~$\subA \subseteq \posAset$ such that any two elements of~$\subA$ are comparable:
    \begin{equation*}
        \prfperiod{
            \posela,\poselb \setin \subA
        }{
            (\posela \posAleq  \poselb) \boolor (\poselb \posAleq  \posela)
        }
    \end{equation*}
    %
    % If~$\ela, \elb \setin \stylesets{S}$, then~$\ela \posAleq \elb$ or~$\elb \posAleq \ela$:
    %a sequence of elements~$
    %{\posAel_i}$ in~$\posA$ where two successive elements are comparable:
    %    \begin{equation}
    %        \prftree{i \Nleq j }{\posAel_i \posAleq \posAel_j}
    %        %i \Nleq j \Rightarrow s_i \posAleq s_j.
    %    \end{equation}
\end{definition}

\begin{definition}[Antichain in a poset]
    \label{def:antichain}
    An \emph{antichain} is a subset~$\subA$ of a poset where no two distinct elements are comparable:
    %
    \begin{equation*}
        \prfperiod{
            \posela,\poselb \setin \subA
        }{
            \posela \posAleq \poselb
        }{
            \posela = \poselb
        }
    \end{equation*}
    %
\end{definition}
\begin{remark}
    Note that the empty set~$\emptyset$ is both a chain and an antichain.
\end{remark}

We denote the set of antichains of a poset~$\posA$ by~$\antichains\posA$.

\begin{example}[Chains and antichains in power set]
    Consider the poset in~\cref{fig:powersetcat}.
    Examples of chains are
    \begin{equation*}
        \{\varnothing,\{\posela\},\{\posela,\poselb\},\{\posela,\poselb,\poselc\}\}
        \qqand
        \{\varnothing,\{\poselb\},\{\poselb,\poselc\},\{\posela,\poselb,\poselc\}\},
    \end{equation*}
    depicted in \cref{fig:power_chains_a} and \cref{fig:power_chains_b}, respectively.

    Examples of antichains are
    \begin{equation*}
        \{\{\posela\},\{\poselb\},\{\poselc\}\}
        \qqand
        \{ \{\posela,\poselb\},\{\posela,\poselc\}, \{\poselb,\poselc\}\},
    \end{equation*}
    depicted in \cref{fig:power_antichains_a} and \cref{fig:power_antichains_b}, respectively.
\end{example}

\begin{figure*}
    \fitinpage{
        \subfloat[\label{fig:power_chains_a}]{
            \includesag{40_dpcatfig_power_chains}
        }
        \subfloat[\label{fig:power_chains_b}]{
            \includesag{40_dpcatfig_power_chains_bis}
        }
        \subfloat[\label{fig:power_antichains_a}]{
            \includesag{40_dpcatfig_power_antichains}
        }
        \subfloat[\label{fig:power_antichains_b}]{
            \includesag{40_dpcatfig_power_antichains_bis}
        }
    }
\end{figure*}

\clearpage

\begin{marginfigure}
    \centering
    \includesag{70_antichain}
    \caption{Example of discrete antichains.}
    \label{fig:antichain}
\end{marginfigure}
\begin{example}
    In the context of battery choices, consider the diagram reported in~\cref{fig:antichain}.
    The black markers represent an antichain of choices
    \begin{equation*}
        \{
        \tupp{\poscheap,\posheavy},
        \tupp{\posexpensive,\poslight}
        \}.
    \end{equation*}
    It is a set of antichains because they do not dominate each other: one is cheaper, but takes longer, and the other is more expensive, but quicker, making them uncomparable.
    The red marker~$\textcolor{red}{\marker}$ is an element that cannot be part of the antichain, since it is dominated by~$\tupp{\posexpensive,\poslight}$.

    %
    \begin{marginfigure}
        \centering
        \includesag{70_antichain_2}
        \caption{Example of continuous antichains.}
        \label{fig:antichain_2}
    \end{marginfigure}

    Similarly, one could think of a continuous law which relates battery cost and mass.
    Let's assume that cheap means~$\unit[10]{\standardcurrency}$, expensive means~$\unit[20]{\standardcurrency}$, light means~$\unit[250]{g}$, and heavy means~$\unit[500]{g}$.
    For instance, consider the antichain given by~$\text{mass}=500-25\cdot \text{cost}$, with maximum possible cost~$\unit[20]{\standardcurrency}$ (\cref{fig:antichain_2}).
\end{example}

\vspace{5cm}
\todojira{612}{I achieved this gap above with a \str{vspace}.
    Is there a way to skip until after the marginfigure is finished?
}

\begin{marginfigure}
    \begin{center}
        \includesag{chain_divisor_poset}
    \end{center}
    \caption{\label{fig:chain_divisor_poset}}
\end{marginfigure}

\begin{marginfigure}
    \begin{center}
        \includesag{antichain_divisor_poset}
    \end{center}
    \caption{\label{fig:antichain_divisor_poset}}
\end{marginfigure}

% We will now have a look at couple more examples.
\begin{example}
    Consider the poset~$\posA=\tup{\posAset,\posAleq}$ where~$(\posAel \posAleq \posBel)$ if~$\posAel$ is a divisor of~$\posBel$ and~$\posAset=\{1,5,10,11,13,15\}$.
    A chain of~$\posA$ is~$\{1,5,10\}$ (\cref{fig:chain_divisor_poset}).
    An antichain of~$\posA$ is~$\{10,11,13,15\}$ (\cref{fig:antichain_divisor_poset}).
\end{example}
