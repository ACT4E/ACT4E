% !TEX root = chapter-standalone.tex

\sectionexercises{Posets representation}
% \subsection{Interface}
\Cref{lst:Poset} shows the interface for posets.
\Cref{lst:FinitePoset} shows the interface for finite posets:
a \classname{FinitePoset} is a poset with an underlying finite carrier set.
%
\begin{widepar}
    \begin{tabular}{cc}
        \begin{minipage}{0.45\textwidth}
            \classlisting{Poset}
        \end{minipage} &
        \begin{minipage}{0.45\textwidth}
            \classlisting{FinitePoset}
        \end{minipage}
    \end{tabular}
\end{widepar}
\begin{figure}[h!]
    \includesag{poset-finiteposet}
    \caption{}
    \label{fig:poset-finiteposet}
\end{figure}

\subsection{Concrete representation}
\begin{marginfigure}
    \yamldatafile{poset1.poset.yaml}{}
    \caption{}
    \label{fig:poset1}
\end{marginfigure}
\begin{marginfigure}
    \yamldatafile{poset_empty.poset.yaml}{}
    \caption{An empty poset}
    \label{fig:poset_empty}
\end{marginfigure}

The representation of a finite poset (\cref{fig:poset1}) is described compactly using the Hasse representation.
Compared to the \classname{FiniteRelation} representation, in which we describe all relations among elements, here we do not need to specify the redundant information that can be recovered using the properties of symmetry and reflexivity.

\todo{missing discussion here}

\begin{codeexercise}[\exname{TestFinitePosetRepresentation}]
    Create a function to load and save the data.

    Implement the interface in \cref{lst:FinitePosetRepresentation}.
\end{codeexercise}

\classlisting{FinitePosetRepresentation}

