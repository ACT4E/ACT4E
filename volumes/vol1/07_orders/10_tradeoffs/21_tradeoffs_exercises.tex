% !TEX root = chapter-standalone.tex

\sectionexercises{Posets representation}

\Cref{lst:Poset} shows the interface for \SY{posets}.
A \SY{poset} is an endorelation: a relation with same source and target.
(Unfortunately, using the Python type system we cannot express the fact that \funcname{holds} must be a partial order.)

\Cref{lst:FinitePoset} shows the interface for finite posets: a \classname{FinitePoset} is a \classname{Poset} whose underlying carrier set is finite.

\classlisting{Poset}

\classlisting{FinitePoset}

% %
% \todographicsjira{560}{@Andrea: How to align the two listings?}
% \begin{widepar}
%     \begin{tabular}{cc}
%         \begin{minipage}{0.45\textwidth}
%         \end{minipage} &
%         \begin{minipage}{0.45\textwidth}
%         \end{minipage}
%     \end{tabular}
% \end{widepar}

\begin{figure}[h!]
    \includesag{poset-finiteposet}
    \caption{}
    \label{fig:poset-finiteposet}
\end{figure}

\subsection{Concrete representation}
\begin{marginfigure}
    \includesag{40_dpcatfig_hasse2}
    \datafile{poset_fruit1}{}
    \caption{}
    \label{fig:poset1}
\end{marginfigure}
\begin{marginfigure}
    \datafile{poset_empty}{}
    \caption{An empty poset.}
    \label{fig:poset_empty}
\end{marginfigure}

The representation of a finite \SY{poset} (\cref{fig:poset1}) is described compactly using the Hasse representation.
Compared to the \classname{FiniteRelation} representation, in which we describe all relations among elements, here we do not need to specify the redundant information that can be recovered using the properties of antisymmetry, reflexivity, and transitivity.

\codeboilerplate{FinitePosetRepresentation}{
    Create a function to load and save the data.

}
\classlisting{FinitePosetRepresentation}

