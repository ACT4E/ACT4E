% !TEX root = chapter-standalone.tex

\section{Monoidal posets}
\label{sec:monoidal-posets}
\linkvideo{spring2021-par-feedback:mon-cat:mon-pos} % Monoidal posets

A monoidal poset is a poset that is also a monoid, and in which the monoidal product is a monotone map that is compatible with the order.

% In order to ``warm up", we first consider the definition of a monoidal structure for a poset.
% Posets are a simpler special case of categories, and the following definition is a special case of the general definition of a monoidal category.

\begin{ctdefinition}[\iindex{Monoidal poset}]
    \label{def:monoidal-poset}
    A \emph{monoidal structure} on a poset~$\Cat{P} = \tup{\posAset, \posAleq}$ is specified by:

    \constit
    \begin{enumerate}
        \item A monotone map~$\mtimescat \colon \Cat{P} \Ctimes \Cat{P} \toinPos \Cat{P}$, called the \emph{monoidal product}.
        \item An element~$\idmoncat \in \posA$, called the \emph{monoidal unit}.
    \end{enumerate}

    \condit
    \begin{enumerate}
        \item Associativity: for all~$\posela, \poselb, \poselc \in \posA$:
              \begin{equation}
                  (\posela \mtimescat \poselb)
                  \mtimescat \poselc =  \posela  \mtimescat ( \poselb \mtimescat \poselc).
              \end{equation}
        \item Left and right unitality: for all~$\posela \in \posA$:
              \begin{equation}
                  \idmoncat \mtimescat \posela = \posela
                  \qqand
                  \posela \mtimescat \idmoncat = \posela.
              \end{equation}
    \end{enumerate}

    \noindent A poset equipped with a monoidal structure is called a \emph{monoidal poset}.
\end{ctdefinition}

Note that here we are implicitly assuming~$\Cat{P} \Ctimes \Cat{P}$ as having the product order (\cref{def:productposet}).
In detail, monotonicity means that, for all~$\posela_1, \posela_2, \poselb_1, \poselb_2 \in \posA$:
\begin{equation}
    \prfperiod{\posela_1 \posAleq \poselb_1}{\posela_2 \posAleq \poselb_2}{ (\posela_1 \mtimescat  \posela_2) \posAleq  (\poselb_1 \mtimescat  \poselb_2)}
\end{equation}

\begin{ctdefinition}[\iindex{Symmetric monoidal poset}]
    \label{def:sym-monoidal-poset}
    A \emph{symmetric monoidal poset} is a monoidal poset~$\posA=\tup{\posAset, \posAleq, \mtimescat, \idmoncat}$ such that, for all~$\posela, \poselb \in \posA$,
    \begin{equation}
        \posela \mtimescat \poselb = \poselb \mtimescat \posela.
    \end{equation}
\end{ctdefinition}

\begin{example}[Reals with addition]
    \label{ex:monoidal-pos-reals}
    Consider the real numbers~\reals with the poset structure given the usual ordering.
    Consider 0 as the monoidal unit and the operation~$+\colon \reals\cartprod \reals\sto \reals$ as mononidal product.
    It is easy to see that the conditions of~\cref{def:monoidal-poset} are satisfied:
    \begin{enumerate}[(a)]
        \item Given~$p_1,p_2,q_1,q_2\in \reals$, we know:
              \begin{equation*}
                  \prfperiod{
                      p_1\Rleq  p_2
                  }{
                      q_1\Rleq  q_2
                  }{
                      (p_1+p_2) \Rleq  (q_1+q_2)
                  }
              \end{equation*}
        \item $0+p=p+0=0$,~$\forall p\in \reals$.
        \item $(p+q)+r=p+(q+r)$,~$\forall p,q,r\in \reals$.
    \end{enumerate}
\end{example}

\begin{counterexample}
    Someone proposes now to substitute the monoidal unit in \cref{ex:monoidal-pos-reals} with 1 and the monoidal product with multiplication ``$\cdot$''.
    This does not form a monoidal poset anymore.
    To see a simple counterexample, consider the fact that~$-5\Rleq 0$ and~$-4\Rleq 3$.
    However,~$(-5)\cdot (-4) \not{\Rleq} 0 \cdot 3$.
\end{counterexample}

\begin{example}[Boolean monoid]
    The booleans form a monoidal poset $\tupp{\Bool,\posleqof{\Bool},\true,\booland}$
    with the unit being~$\true$ and the product being $\booland$.
    The action of the monoidal product ``$\booland$'' can be summarized in a table:
    \begin{center}
        \begin{tabular}{c|cc}
            $\booland$ & $\false$ & $\true$ \\
            \hline
            $\false$   & $\false$ & $\false$ \\
            $\true$    & $\false$ & $\true$
        \end{tabular}
    \end{center}
    From this table, it is clear that given~$\ela_1\posleqof{\Bool}\elb_1$ and~$\ela_2\posleqof{\Bool} \elb_2$, one has~$\ela_1\booland x_2\posleqof{\Bool} \elb_1\booland \elb_2$ (if you do not believe it, try all possible combinations).
    Furthermore,~$\ela\booland \true=\ela=\true \booland \ela$.
\end{example}

\showslides{
    \begin{forslides}
        \begin{equation*}
            \label{eq:monpos_1}
            \prftree{p_1\leq p_2}{q_1\leq q_2}{p_1+p_2\leq q_1+q_2}
        \end{equation*}
        \begin{equation*}
            \label{eq:monpos_2}
            0+p=p+0=0
        \end{equation*}
        \begin{equation*}
            \label{eq:monpos_3}
            (p+q)+r=p+(q+r)
        \end{equation*}

        \includesag{scooter}
        \includesag{av}
        \includesag{av_scooter}
        \includesag{av_scooter_bis}

    \end{forslides}
}
