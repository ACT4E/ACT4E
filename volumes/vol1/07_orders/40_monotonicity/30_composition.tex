% !TEX root = chapter-standalone.tex

\section{Compositionality of monotonicity}
\SY{Monotonicity} is a compositional property: the series composition of two \SY{monotone maps} is \SY{monotone}.
\begin{lemma}\label{lem:monotonicity-compositional}
    Given \SY{posets}~$\posA,\posB,\posC$ and two \SY{monotone maps}.
    $\mapa\colon \posA \mto \posB$ and $\mapb\colon \posB \mto \posC$, the composite map~$\mapa\mthen \mapb\colon \posA \mto \posC$ is monotone as well.
\end{lemma}
\begin{proof}
    Consider~$\posAnel{1},\posAnel{2} \setin \posAset$,~$\posBnel{1},\posBnel{2}\setin \posBset$.
    We have, by the assumption that $\mora$ and $\morb$ are \SY{monotone},
    %
    \begin{equation}\label{eq:monotonicity-compositional-1}
        \prftree{\posAnel{1}\posAleq \posAnel{2}}{\mapa(\posAnel{1})\posAleq \mapa(\posAnel{2})}
    \end{equation}
    and
    %
    \begin{equation}\label{eq:monotonicity-compositional-2}
        \prfperiod{\posBnel{1}\posBleq \posBnel{2}}{\mapb(\posBnel{1})\posCleq \mapb(\posBnel{2})}
    \end{equation}
    %
    By substituting the above in the map composition formula, we have
    %
    \begin{equation}\label{eq:monotonicity-compositional-3}
        \prfcomma{\posAnel{1}\posAleq \posAnel{2} }{ (\mapa\mthen \mapb)(\posAnel{1}) \posCleq (\mapa\mthen \mapb)(\posAnel{2})}
    \end{equation}
    %
    which is the \SY{monotonicity} condition for the composite map~$(\mapa\mthen\mapb)$.
\end{proof}

\todotext{JL: the above proof is has mistakes and is quite unclear, in my opinion}
