% !TEX root = chapter-standalone.tex
\section{Compositionality of monotonicity}
Note that monotonicity is a compositional property.
\begin{lemma}
    Given posets~$\posA,\posB,\posC$ and two monotone maps~$\mapa\colon \posA \sto \posB$ and $\mapb\colon \posB \sto \posC$, the composite map~$\mapa\then \mapb\colon \posA \sto \posC$ is monotone as well.
\end{lemma}
\begin{proof}
    Consider~$\posAnel{1},\posAnel{2} \setin \posA$,~$\posBnel{1},\posBnel{2}\setin \posB$.
    We have, by definition,
    \begin{equation*}
        \prftree{\posAnel{1}\posAleq \posAnel{2}}{\mapa(\posAnel{1})\posAleq \mapa(\posAnel{2})}
    \end{equation*}
    and
    \begin{equation*}
        \prfperiod{\posBnel{1}\posBleq \posBnel{2}}{\mapb(\posBnel{1})\posCleq \mapb(\posBnel{2})}
    \end{equation*}
    By substituting the above in the map composition formula, one has
    \begin{equation*}
        \prfcomma{\posAnel{1}\posAleq \posAnel{2} }{ (\mapa\then \mapb)(\posAnel{1}) \posCleq (\mapa\then \mapb)(\posAnel{2})}
    \end{equation*}
    which is the monotonicity condition for the composite map~$(\mapa\then \mapb)$.
\end{proof}
