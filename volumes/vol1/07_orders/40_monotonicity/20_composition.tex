% !TEX root = chapter-standalone.tex

\section{Compositionality of monotonicity}
Note that monotonicity is a compositional property.
\begin{lemma}
    Given posets~$\posA,\posB,\posC$ and two monotone maps $\mapa\colon \posA \sto \posB$ and $\mapb\colon \posB \sto \posC$, the composite map $\mapa\then \mapb\colon \posA \sto \posC$ is monotone as well.
\end{lemma}
\begin{proof}
    Consider~$\posAel_1,\posAel_2 \setin \posA$,~$\posBel_1,\posBel_2\setin \posB$.
    We have, by definition,
    \begin{equation*}
        \prftree{\posAel_1\posAleq \posAel_2}{\mapa(\posAel_1)\posAleq \mapa(\posAel_2)}
    \end{equation*}
    and
    \begin{equation*}
        \prfperiod{\posBel_1\posBleq \posBel_2}{\mapb(\posBel_1)\posCleq \mapb(\posBel_2)}
    \end{equation*}
    By substituting the above in the map composition formula, one has
    \begin{equation}
        \prfcomma{\posAel_1\posAleq \posAel_2 }{ (\mapa\then \mapb)(\posAel_1) \posCleq (\mapa\then \mapb)(\posAel_2)}
    \end{equation}
    which is the monotonicity condition for the composite map~$(\mapa\then \mapb)$.
\end{proof}
