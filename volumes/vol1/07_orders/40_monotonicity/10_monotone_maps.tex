% !TEX root = chapter-standalone.tex

\section{Monotone maps}\label{sec:monotonicity-monotone-maps}
\todojira{80}{@Andrea: Add some motivation from design.
	\eg generalization of "non-decreasing" and "non-increasing" from real numbers to partial orders}

%\linkvideo{spring2021-functors:semi-and-fun:mon-functions} % Monotone functions
\linkvideo{spring2021-functors:semi-and-fun:mon-functions:fun-req-mon} % Functionalities and requirements
\linkvideo{spring2021-functors:semi-and-fun:mon-functions:mon-on-pos} % Monotone functions on posets

\begin{definition}[Monotone map]
	\label{def:monotone}
	A \emph{\iindex{monotone map}} between two posets~$\posA=\tupp{\posAset, \posAleq}$ and~$\posB=\tupp{\posBset, \posBleq}$ is a map~$\mapa$ that preserves the ordering, in the sense that
	\begin{equation}
		\prfperiod{
			\posela \posAleq \poselb
		}{
			\mapa(\posela) \posBleq \mapa(\poselb)
		}
		%\posAel_1 \posAleq \posAel_2 \quad \Imp \quad \mapa(\posAel_1) \posBleq \mapa(\posAel_2).
	\end{equation}
\end{definition}

\begin{remark}
	Given a poset~$\posA$, the map~$\catid_{\posA}\colon \posA \mto \posA$ is monotone, since if $\posela\posAleq \poselb$, then $\catid_{\posA}(\posela)=\posela \posAleq  \poselb =  \catid_{\posA}(\poselb )$.

\end{remark}

\begin{definition}[Antitone map]
	\label{def:antitone}
	An \emph{\iindex{antitone map}} between two posets~$\posA=\tupp{\posAset, \posAleq}$ and~$\posB=\tupp{\posBset, \posBleq}$ is a map~$\mapa$ that reverses the ordering, in the sense that
	\begin{equation}
		\prfperiod{
			\posela \posAleq \poselb
		}{
			\mapa(\posela) \posgeq_\posB \mapa(\poselb)
		}
		%\posAel_1 \posAleq \posAel_2 \quad \Imp \quad \mapa(\posAel_1) \posBleq \mapa(\posAel_2).
	\end{equation}
\end{definition}

\begin{example}[Unit cost, total cost]
	Assume that you want to produce some widgets, and that the manifacturing cost depends on the number of widgets.
	The function describing the total cost~$\stylemaps{t}\colon \natnumbers\to \nonNegReals$ is a map between the ordered sets~$\natnumbers$ and~$\nonNegReals$, and maps each quantity of widgets to a total manufacturing cost (\cref{fig:total_manufacturing}).
	Clearly,~$\stylemaps{t}$ is a monotone function.
	Conversely, the unit cost function~$\stylemaps{u}\colon \natnumbers\to \nonNegReals$ is antitone (\cref{fig:unit_manufacturing}).
\end{example}

\begin{figure}[h!]
	\subfloat[
		Unit cost  \vs  number of widgets.
		\label{fig:unit_manufacturing}
	]{
		\includesag{unit_manufacturing}
	}
	\subfloat[
		Total cost \vs number of widgets.
		\label{fig:total_manufacturing}
	]{
		\includesag{total_manufacturing}
	}
\end{figure}

\begin{example}[Rounding functions]
	\label{ex:rounding-functions}
	In this example we look at three rounding functions: \funceil~(\cref{fig:ceil}), \funfloor~(\cref{fig:floor}), and $\rtntte$~(\cref{fig:rtntte}).
	Both the maps
	\begin{equation*}
		\begin{aligned}
			\funceil\colon \tup{\reals,\leq} & \to \tup{\natnumbers,\leq}                    \\
			x                                & \mapsto i \in \natnumbers \colon i-1<x\leq i,
		\end{aligned}
	\end{equation*}
	and
	\begin{equation*}
		\begin{aligned}
			\funfloor\colon \tup{\reals,\leq} & \to \tup{\natnumbers,\leq}                     \\
			x                                 & \mapsto i \in \natnumbers \colon i\leq x< i+1.
		\end{aligned}
	\end{equation*}
	are monotone, since~$x\leq y$ implies both~$\funceil(x)\leq \funceil(y)$ and~$\funfloor(x)\leq\funfloor(y)$.
	\todotextjira{512}{@Andrea: define function $\rtntte$.}
	\todotextjira{83}{@Andrea: Define IEEE754 formally}
	\begin{figure*}[h!]
		\centering
		\subfloat[$\funceil$. \label{fig:ceil}]{
			\includesag{ceil}}
		\hfill
		\subfloat[$\funfloor$. \label{fig:floor}]{
			\includesag{floor}}
		\hfill
		\subfloat[$\rtntte$. \label{fig:rtntte}]{
			\includesag{rtntte}}
	\end{figure*}
\end{example}

\begin{example}[Cardinality map]
	In \cref{def:power-poset} we presented the poset arising from the power set of a set~$\setA=\{\posela,\poselb,\poselc\}$ and ordered via subset inclusion.

	The map~$\vert \cdot \vert \colon \powerposet{\setA} \to \tup{\natnumbers,\leq}$ (cardinality), is a monotone map (\cref{fig:cardinality}).
	\begin{figure*}[h!]
		\centering
		\includesag{40_dpcatfig_exmonotone}
		\caption{The cardinality map is a monotone map. }
		\label{fig:cardinality}
	\end{figure*}
\end{example}

\begin{lemma}
	Consider a discrete poset~$\posA$ and a poset~$\posB$.
	Any map~$\mapa \colon \posA\to \posB$ is monotone.
\end{lemma}
\newcommand{\samewidth}[1]{\makebox[3cm]{$#1$}}
\begin{proof}
	Since~$\posA$ is a discrete poset, one has
	\begin{equation}
		\prfdoubleperiod{
			\posAel_1\posAleq \posAel_2
		}{
			\posAel_1=\posAel_2
		}
		%\posAel_1\posAleq \posAel_2 \iff \posAel_1=\posAel_2.
	\end{equation}
	Therefore, one has
	\begin{equation}
		\prfperiod{
			\prftree{
				\prftree[r]{\quad($P$ discrete)}{
					\samewidth{\posAel_1\posAleq \posAel_2}
				}{
					\samewidth{\posAel_1=\posAel_2}
				}
			}{
				\samewidth{\mapa(\posAel_1)=\mapa(\posAel_2)}
			}
		}{
			\samewidth{\mapa(\posAel_1)\posBleq \mapa(\posAel_2)}
		}
	\end{equation}
\end{proof}
Unless indicated otherwise, in this book all maps between posets are assumed to be monotone or will turn out to be monotone.
In a similar way, one can define antitone maps.

\subsection{Other examples}

\begin{lemma}
	The upper closure operator~$\upit$ is an antitone map.
\end{lemma}
\begin{proof}
	Consider the posets~$\tupp{\posPA,\subseteq}$ and $\tupp{\setOfUppersets \posA,\supseteq}$, and two sets of sets $\stylesets{S_1}\com \stylesets{S_2}\in \linebreak[1] \posPA$.
	It is clear that given~$\stylesets{S_1}\subseteq \stylesets{S_2}$, one has
	\begin{equation*}
		\{\styleelements{y}\in \posA\mid \exists \styleelements{x}\in \stylesets{S_1}\colon \styleelements{x}\posAleq \styleelements{y}\} \subseteq \{\styleelements{y}\in \posA\mid \exists \styleelements{x}\in \stylesets{S_2}\colon \styleelements{x}\posAleq \styleelements{y}\}.
	\end{equation*}
	Therefore,~$\upit \stylesets{S_1} \subseteq \upit  \stylesets{S_2}$, satisfing the antitone map property for~$\upit$.
\end{proof}

\begin{lemma}
	\label{lem:lower_closure_monotone}
	The lower closure operator~$\downit$ is a monotone map.
\end{lemma}

\begin{exercise}
	Prove \cref{lem:lower_closure_monotone}.
\end{exercise}
\begin{solution}
	Consider the posets~$\tup{\posPA,\subseteq}$ and~$\tup{\setOfLowersets \posA,\subseteq}$, and let~$\stylesets{S_1},\stylesets{S_2}\in \posPA$.
	It is clear that given~$\stylesets{S_1}\subseteq \stylesets{S_2}$, one has
	\begin{equation}
		\{\styleelements{y}\in \posA\mid \exists \styleelements{x}\in \stylesets{S_1}\colon \styleelements{y}\posAleq \styleelements{x}\} \subseteq \{\styleelements{y}\in \posA\mid \exists \styleelements{x}\in \stylesets{S_2}\colon \styleelements{y}\posAleq \styleelements{x}\}.
	\end{equation}
	Therefore,~$\downit \stylesets{S_1}\subseteq \ \downit \stylesets{S_2}$, satisfing the monotonicity property for~$\downit$.
\end{solution}

