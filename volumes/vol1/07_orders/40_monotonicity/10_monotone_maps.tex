% !TEX root = chapter-standalone.tex

\section{Monotone maps}
\label{sec:monotonicity-monotone-maps}

%\linkvideo{spring2021-functors:semi-and-fun:mon-functions} % Monotone functions
% \linkvideo{spring2021-functors:semi-and-fun:mon-functions:fun-req-mon} % Functionalities and requirements
\linkvideo{spring2021-functors:semi-and-fun:mon-functions:mon-on-pos} % Monotone functions on posets

A \SY{monotone map} is the generalization to \SY{posets} of a ``non-decreasing'' function on real numbers.
The function $x \mapsto \max(0, 42 x)$ is non-decreasing on the real numbers because
\begin{equation}\label{eq:monotone-eq-1}
    \prfperiod{
        x \leq y
    }{
        \max(0, 42 x) \leq \max(0, 42 y)
    }
\end{equation}
Note that we use ``$\leq$'' and not ``$<$''.
``Non-decreasing'' is a weaker condition than ``increasing''.

The definition of \SY{monotone function} on a \SY{poset} is the direct generalization of this concept; the only change is that we use the partial orders at hand, rather than the total order on the reals.

\begin{definition}[Monotone map]
    \label{def:monotone}
    A \maindef{monotone map} between two \SY{posets}~$\posAdefinition$ and~$\posBdefinition$ is a function~$\mapa \colon \posAset \sto \posBset$ that is compatible with the partial-orderings on its source and target in the sense that
    \begin{equation}\label{eq:monotone-map-definition}
        \prfperiod{
            \posela \posAleq \poselb
        }{
            \mapa(\posela) \posBleq \mapa(\poselb)
        }
    \end{equation}
\end{definition}
\showslides{
    \begin{forslides}
        \begin{equation}\label{eq:monotone-map-identity}
            \prfperiod{
                \posela \posAleq \poselb
            }{
                \posela \posAleq \poselb
            }
        \end{equation}
        \begin{equation}\label{eq:monotone-map-discrete-1}
            \prfdoubleperiod{
                \posela \posAleq \poselb
            }{
                \posela = \poselb
            }
        \end{equation}
        \begin{equation}\label{eq:monotone-map-discrete-2}
            \prfperiod{
                \posela \posAleq \posela
            }{
                \mora(\posela) \posAleq \mora(\posela)
            }
        \end{equation}
        \begin{equation}\label{eq:monotone-map-constant}
            \prfperiod{
                \posela \posAleq \poselb
            }{
                \poselc \posAleq \poselc
            }
        \end{equation}
    \end{forslides}
}

\begin{example}[The identity is monotone]
    Given a poset~\posA, the identity function $\catidat{\posAset}\colon \posAset \mto\linebreak[0] \posAset$ is a monotone map, since if~$\posela\posAleq \poselb$, then~$\mapidat{\posAset}(\posela)=\posela \posAleq \poselb = \mapidat{\posAset}(\poselb )$.
\end{example}

\begin{example}[Constant functions]
    Every constant function is a monotone map.
\end{example}

\begin{figure*}[b]
    \centering
    \includesag{40_dpcatfig_exmonotone}
    \caption{The cardinality map is a \SY{monotone map}. }
    \label{fig:cardinality}
\end{figure*}

\begin{example}[Cardinality map]\label{exa:cardinality}
    Consider the power \SY{poset} (\cref{def:power-poset}) $\powerposet\setA$ of a finite set \setA.
    The cardinality map
    \begin{equation}\label{eq:cardinality-map}
        \cardmap : \powerset\setA \sto \natnumbers
    \end{equation}
    is monotone when considered as a map from the \SY{poset} $\powerposet\setA$ to the \SY{poset} $\natswithleq$.
    % \begin{equation}
    %     \defmap{
    %         \cardmap
    %     }{
    %         \powerposet{\setA}
    %     }{
    %         \sto
    %     }{
    %         \tup{\natnumbers,\leq}
    %     }{
    %         \subA
    %     }{
    %         \vert\subA\vert
    %     }
    % \end{equation}
    % is monotone.
    \Cref{fig:cardinality} shows a visualization of this map for the set~$\setA=\makeset{\posela,\poselb,\poselc}$.
    To prove this, recall that in the power \SY{poset} subsets are ordered by inclusion.
    Therefore, we need to show that
    \begin{equation}\label{eq:cardinality}
        \prfdoubleperiod{
            \subA \setsubseteq \subB
        }{
            \cardof \subA \leq \cardof \subB
        }
    \end{equation}
    \todotext{JL: the above is incorrect, it is not iff, just an implication}
    
    This is easy to see that, because all elements of $\subA$ are also in $\subB$, the cardinality of~$\subA$ cannot be more than the cardinality of~$\subB$.
    Monotonicity depends on the partial order used on the domain and the codomain.
    To indicate that a map is monotone, we write it indicating the two \SY{posets} as the domain/codomain:
    \begin{equation}\label{eq:cardinality-as-posmap}
        \cardmap \colon \tup{\powerset\setA, \setsubseteq} \mto \tup{\natnumbers, {{\Nleq}}}.
    \end{equation}
\end{example}
\showslides{
    \begin{forslides}
        \begin{equation}\label{eq:scalar-nondecreasing}
            \prftree{
                x \leq y
            }{
                \mora(x) \leq \mora(y)
            }
        \end{equation}

        \begin{equation}\label{eq:scalar-increasing}
            \prftree{
                x < y
            }{
                \mora(x) < \mora(y)
            }
        \end{equation}
    \end{forslides}
}

\begin{lemma}\label{lem:discrete-is-monotone}
    Consider a discrete poset~\posA and a poset~\posB.
    Any map~$\mapa \colon \posA\to \posB$ is monotone.
\end{lemma}
\newcommand{\samewidth}[1]{\makebox[3cm]{$#1$}}


\begin{gradedexercise}[\exname{FromDiscretePosets}]
Prove \cref{lem:discrete-is-monotone}
\end{gradedexercise}
\solutionof{FromDiscretePosets}


\clearpage
\vfill
\begin{gradedexercise}[\exname{MonotoneMapCheck}]
    \label{ex:MonotoneMapCheck?}

    Prove your answers to the following questions.
    \begin{enumerate}
        \item Is the function
              \begin{equation}\label{eq:MonotoneMapCheck-1}
                  \defmap{\mapa}{\tup{\wnumbers,{{\Nleq}}}}{\to}{\tup{\wnumbers, {{\Nleq}}}}{\ela}{\ela^2}
              \end{equation}
              monotone?
              % \item Let~$\setA = \makeset{\ela, \elb, \elc}$ and consider the \SY{posets}~$\tup{\powerset{\setA}, {{\setsubseteq}}}$ and~$\tup{\natnumbers, \leq}$.
              %       Let $\cardmap$ be
              %       \begin{equation}
              %           \defmap{\mapa}{\posA}{\to}{\natnumbers}{\subA}{\vert \subA \vert}
              %       \end{equation}
              %       be the function which calculates the cardinality of any subset of~$\setA$.
              %       Is~$\mapa$ monotone?
              %       \todotext{\alphubel: @JL: Note that this question cannot be used as exercise because it's an example.}
        \item Consider the set of natural numbers which divide the number 36, equip\-ped with the partial order such that~$\ela \leq \elb$ if and only if~$\ela$ divides~$\elb$.
              Call this poset~$\posAdefinition$, and let~$\mapa \colon \posAset \sto \boolset$ be defined by
              \begin{equation}\label{eq:MonotoneMapCheck-2}
                  \mapa(\ela) =
                  \begin{cases}
                      \true  & \text{ if } \ela \text{ is an even number, } \\
                      \false & \text{ if } \ela \text{ is an odd number.
                      }
                  \end{cases}
              \end{equation}
              Is~$\mapa$ monotone if we equip~$\makeset{\false, \true}$ with the usual partial order such that~$\false \leq \true$?
    \end{enumerate}
\end{gradedexercise}
\solutionof{MonotoneMapCheck}

