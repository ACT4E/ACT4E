% !TEX root = chapter-standalone.tex

\section{Monotone maps}
\label{sec:monotonicity-monotone-maps}

%\linkvideo{spring2021-functors:semi-and-fun:mon-functions} % Monotone functions
% \linkvideo{spring2021-functors:semi-and-fun:mon-functions:fun-req-mon} % Functionalities and requirements
\linkvideo{spring2021-functors:semi-and-fun:mon-functions:mon-on-pos} % Monotone functions on posets

A \SY{monotone map} is the generalization to \SY{posets} of a ``non-decreasing'' function on real numbers.
The function~$\ela \mapsto \max(0, 42 \ela)$ is non-decreasing on the real numbers because
\begin{equation}\label{eq:monotone-eq-1}
    \prfperiod{
        \ela \leq \elb
    }{
        \max(0, 42 \ela) \leq \max(0, 42 \elb)
    }
\end{equation}
Note that we use ``$\leq$'' and not ``$<$''.
``Non-decreasing'' is a weaker condition than ``increasing''.

The definition of \SY{monotone function} on a \SY{poset} is the direct generalization of this concept; the only change is that we use the partial orders at hand, rather than the total order on the reals.

\begin{definition}[Monotone map]
    \label{def:monotone}
    A \maindef{monotone map} between two \SY{posets}~$\posAdefinition$ and~$\posBdefinition$ is a function~$\mapa \colon \posAset \sto \posBset$ that is compatible with the partial-orderings on its source and target in the sense that
    \begin{equation}\label{eq:monotone-map-definition}
        \prfperiod{
            \posela \posAleq \poselb
        }{
            \mapa(\posela) \posBleq \mapa(\poselb)
        }
    \end{equation}
\end{definition}

\begin{example}[The identity is monotone]
    Given a poset~\posA, the identity function~$\catidat{\posAset}\colon \posAset \mto\linebreak[0] \posAset$ is a monotone map, since if~$\posela\posAleq \poselb$, then~$\mapidat{\posAset}(\posela)=\posela \posAleq \poselb = \mapidat{\posAset}(\poselb )$.
\end{example}

\begin{example}[Constant functions]
    Every constant function is a monotone map.
\end{example}

\begin{figure*}[b]
    \centering
    \includesag{40_dpcatfig_exmonotone}
    \caption{The cardinality map is a \SY{monotone map}.}
    \label{fig:cardinality}
\end{figure*}

\begin{example}[Cardinality map]\label{exa:cardinality}
    Consider the power \SY{poset} (\cref{def:power-poset}) $\powerposet\setA$ of a finite set \setA.
    The cardinality map
    \begin{equation}\label{eq:cardinality-map}
        \cardmap : \powerset\setA \sto \natnumbers
    \end{equation}
    is monotone when considered as a map from the \SY{poset} $\powerposet\setA$ to the \SY{poset} $\natswithleq$.
    % \begin{equation}
    %     \defmap{
    %         \cardmap
    %     }{
    %         \powerposet{\setA}
    %     }{
    %         \sto
    %     }{
    %         \tup{\natnumbers,\leq}
    %     }{
    %         \subA
    %     }{
    %         \vert\subA\vert
    %     }
    % \end{equation}
    % is monotone.
    \Cref{fig:cardinality} shows a visualization of this map for the set~$\setA=\makeset{\posela,\poselb,\poselc}$.
    To prove this, recall that in the power \SY{poset} subsets are ordered by inclusion.
    Therefore, we need to show that
    \begin{equation}\label{eq:cardinality}
        \prfperiod{
            \subA \setsubseteq \subB
        }{
            \cardof \subA \leq \cardof \subB
        }
    \end{equation}

    This is easy to see that, because all elements of $\subA$ are also in $\subB$, the cardinality of~$\subA$ cannot be more than the cardinality of~$\subB$.
    Monotonicity depends on the partial order used on the domain and the codomain.
    To indicate that a map is monotone, we write it indicating the two \SY{posets} as the domain/codomain:
    \begin{equation}\label{eq:cardinality-as-posmap}
        \cardmap \colon \tup{\powerset\setA, \setsubseteq} \mto \tup{\natnumbers, {{\Nleq}}}.
    \end{equation}
\end{example}

\begin{gradedexercise}[\exname{WhichMapsMonotone}] [4 points]

    Consider the posets~$\posA = \tup{\posAset, \posleq_\posA}$ and~$\posB = \tup{\posBset, \posleq_\posB}$ described respectively by the following Hasse diagrams.

    \begin{equation}
        %\includegraphics[width=0.5\linewidth]{pics/ExamPosets.png}
        \includesag{exam_posets}
    \end{equation}

    The following diagrams show functions~$\posAset \mto \posBset$.
    We will call them~$\mapaindex{1}$,~$\mapaindex{2}$,~$\mapaindex{3}$ and~$\mapaindex{4}$, respectively.

    \begin{center}
        \setlength{\tabcolsep}{30pt}
        \begin{tabular}{cc}
            $\mapa_1$ &
            %\includegraphics[width=0.5\linewidth]{pics/ExamPosetMap1.png}
            \includesag{exam_f1} \\[+40pt]
            $\mapa_2$ &
            %\includegraphics[width=0.5\linewidth]{pics/ExamPosetMap2.png}
            \includesag{exam_f2} \\[+40pt]
            $\mapa_3$ &
            %\includegraphics[width=0.5\linewidth]{pics/ExamPosetMap3.png}
            \includesag{exam_f3} \\[+40pt]
            $\mapa_4$ &
            %\includegraphics[width=0.5\linewidth]{pics/ExamPosetMap4.png}
            \includesag{exam_f4}
        \end{tabular}
    \end{center}

    Which of the functions~$\mapa_1$,~$\mapa_2$,~$\mapa_3$ and~$\mapa_4$ are monotone maps?
\end{gradedexercise}
\solutionof{WhichMapsMonotone}

\begin{lemma}\label{lem:discrete-is-monotone}
    Consider a discrete poset~\posA and a poset~\posB.
    Any map~$\mapa \colon \posA\to \posB$ is monotone.
\end{lemma}
\newcommand{\samewidth}[1]{\makebox[3cm]{$#1$}}

\begin{gradedexercise}[\exname{FromDiscretePosets}]
    Prove \cref{lem:discrete-is-monotone}
\end{gradedexercise}
\solutionof{FromDiscretePosets}

\clearpage
\vfill
\begin{gradedexercise}[\exname{MonotoneMapCheck}]
    \label{ex:MonotoneMapCheck}

    Prove your answers to the following questions.
    \begin{enumerate}
        \item Is the function~$\mapa \colon \tup{\wnumbers, \leq} \to \tup{\wnumbers, \leq}, \ela \mapsto \ela^2$ monotone?
        \item Let~$\setA = \makeset{\setAel, \setBel, \setCel}$ and consider the posets~$\tup{\powerset{\setA}, \subseteq}$ and~$\tup{\natnumbers, \leq}$.
              Let
              \begin{equation*}
                  \defmap{\mapa}{\powerset{\setA}}{\to}{\natnumbers}{\subA}{\vert \subA \vert}
              \end{equation*}
              be the function which calculates the cardinality of any subset of $\setA$.
              Is $\mapa$ monotone?
        \item Consider the set of natural numbers which divide the number 36, equipped with the partial order ``$\preccurlyeq$'' such that~$\ela \preccurlyeq \elb$ if and only if~$\ela$ divides~$\elb$.
              Call this poset~$\posA = \tup{\posAset, \preccurlyeq}$, and let~$\mapa \colon \posAset \to \makeset{\false, \true}$ be defined by
              \begin{equation}
                  \mapa(\ela) =
                  \begin{cases}
                      \true  & \text{ if } \ela \text{ is an even number, } \\
                      \false & \text{ if } \ela \text{ is an odd number.
                      }
                  \end{cases}
              \end{equation}
              Is~$\mapa$ monotone if we equip~$\makeset{\false, \true}$ with the usual partial order such that~$\false \preceq \true$?
    \end{enumerate}
\end{gradedexercise}
\solutionof{MonotoneMapCheck}


