% !TEX root = chapter-standalone.tex

\section{Monotone maps}
\label{sec:monotonicity-monotone-maps}

%\linkvideo{spring2021-functors:semi-and-fun:mon-functions} % Monotone functions
% \linkvideo{spring2021-functors:semi-and-fun:mon-functions:fun-req-mon} % Functionalities and requirements
\linkvideo{spring2021-functors:semi-and-fun:mon-functions:mon-on-pos} % Monotone functions on posets

\todojira{80}{\bernina: @Andrea: Add some motivation from design.
    \eg generalization of ``non-decreasing'' and ``non-increasing'' from real numbers to partial orders}

\begin{definition}[Monotone map]
    \label{def:monotone}
    A \emph{\iindex{monotone map}} between two posets~$\posA=\tupp{\posAset, \posAleq}$ and~$\posB=\tupp{\posBset, \posBleq}$ is a map~$\mapa$ that preserves the ordering, in the sense that
    \begin{equation}
        \prfperiod{
            \posela \posAleq \poselb
        }{
            \mapa(\posela) \posBleq \mapa(\poselb)
        }
        %\posAel_1 \posAleq \posAel_2 \quad \Imp \quad \mapa(\posAel_1) \posBleq \mapa(\posAel_2).
    \end{equation}
\end{definition}

\begin{example}[Identity is monotone]
    Given a poset~$\posA$, the identity map~$\catid_{\posA}\colon \posA \mto \posA$ is monotone, since if~$\posela\posAleq \poselb$, then~$\catid_{\posA}(\posela)=\posela \posAleq  \poselb =  \catid_{\posA}(\poselb )$.
\end{example}

\begin{example}[Constant maps are monotone]
    Every constant map is monotone.
\end{example}

\begin{example}[Cardinality map]
    \todotext{Wrong reference}
    In \cref{def:power-poset} we presented the poset arising from the power set of a set~$\setA=\makeset{\posela,\poselb,\poselc}$ and ordered via subset inclusion.

    The map~$\vert \cdot \vert \colon \powerposet{\setA} \to \tup{\natnumbers,\leq}$ (cardinality), is a monotone map (\cref{fig:cardinality}).
    \begin{figure*}[h!]
        \centering
        \includesag{40_dpcatfig_exmonotone}
        \caption{The cardinality map is a monotone map. }
        \todographics{\alphubel: @Gioele: In this figure the arrows should be continuous with the color of morphisms.
            Also put $\mora$ near the arrows.
        }
        \label{fig:cardinality}
    \end{figure*}
\end{example}

\begin{lemma}
    Consider a discrete poset~$\posA$ and a poset~$\posB$.
    Any map~$\mapa \colon \posA\to \posB$ is monotone.
\end{lemma}
\newcommand{\samewidth}[1]{\makebox[3cm]{$#1$}}
\begin{proof}
    Since~$\posA$ is a discrete poset, we have
    \begin{equation}
        \prfdoubleperiod{
            \posAnel{1}\posAleq \posAnel{2}
        }{
            \posAnel{1}=\posAnel{2}
        }
    \end{equation}
    Therefore, we have
    \begin{equation}
        \prfperiod{
            \prftree{
                \prftree[r]{
                    \quad($P$ discrete)
                }{
                    \samewidth{\posAnel{1}\posAleq \posAnel{2}}
                }{
                    \samewidth{\posAnel{1}=\posAnel{2}}
                }
            }{
                \samewidth{\mapa(\posAnel{1})=\mapa(\posAnel{2})}
            }
        }{
            \samewidth{\mapa(\posAnel{1})\posBleq \mapa(\posAnel{2})}
        }
    \end{equation}
\end{proof}
% Unless indicated otherwise, in this book all maps between posets are assumed to be monotone or will turn out to be monotone.

\begin{gradedexercise}[\exname{MonotoneMapCheck}]
    \label{ex:MonotoneMapCheck?}

    Prove your answers to the following questions.
    \begin{enumerate}
        \item Is the function \todotext{@JL: use defmap comand to format on two lines}
              \begin{equation}
                  \mapa \colon \tup{\wnumbers, \leq} \to \tup{\wnumbers, \leq}, \ela \mapsto \ela^2
              \end{equation}
              monotone?
        \item Let $\setA = \makeset{a, b, c}$ and consider the posets $\tup{\powerset{\setA}, \subseteq}$ and $\tup{\natnumbers, \leq}$.
              \todotext{@JL: use defmap comand to format on two lines}
              Let $\mapa \colon \posA \to \natnumbers, \subA \mapsto \vert \subA \vert$ be the function which calculates the cardinality of any subset of $\setA$.
              Is $\mapa$ monotone?
        \item Consider the set of natural numbers which divide the number 36, equip\-ped with the partial order such that $\ela \leq \elb$ if and only if $\ela$ divides $\elb$.
              Call this poset $\posA = \tup{\posAset, \leq}$, and let $\mapa \colon \posAset \sto \boolset$ be defined by
              \begin{equation}
                  \mapa(\ela) =
                  \begin{cases}
                      \true  & \text{ if } \ela \text{ is an even number, } \\
                      \false & \text{ if } \ela \text{ is an odd number.
                      }
                  \end{cases}
              \end{equation}
              Is $\mapa$ monotone if we equip $\makeset{\false, \true}$ with the usual partial order such that $\false \leq \true$?
    \end{enumerate}
\end{gradedexercise}
\solutionof{MonotoneMapCheck}

