% !TEX root = chapter-standalone.tex

\section{Monotone maps}\label{sec:monotonicity-monotone-maps}
\todojira{80}{@Andrea: Add some motivation from design.
    \eg generalization of "non-decreasing" and "non-increasing" from real numbers to partial orders}

%\linkvideo{spring2021-functors:semi-and-fun:mon-functions} % Monotone functions
\linkvideo{spring2021-functors:semi-and-fun:mon-functions:fun-req-mon} % Functionalities and requirements
\linkvideo{spring2021-functors:semi-and-fun:mon-functions:mon-on-pos} % Monotone functions on posets

\begin{definition}[Monotone map]
    \label{def:monotone}
    A \emph{\iindex{monotone map}} between two posets~$\posA=\tupp{\posAset, \posAleq}$ and~$\posB=\tupp{\posBset, \posBleq}$ is a map~$\mapa$ that preserves the ordering, in the sense that
    \begin{equation*}
        \prfperiod{
            \posela \posAleq \poselb
        }{
            \mapa(\posela) \posBleq \mapa(\poselb)
        }
        %\posAel_1 \posAleq \posAel_2 \quad \Imp \quad \mapa(\posAel_1) \posBleq \mapa(\posAel_2).
    \end{equation*}
\end{definition}

\begin{remark}
    Given a poset~$\posA$, the map~$\catid_{\posA}\colon \posA \mto \posA$ is monotone, since if~$\posela\posAleq \poselb$, then~$\catid_{\posA}(\posela)=\posela \posAleq  \poselb =  \catid_{\posA}(\poselb )$.
\end{remark}

\begin{definition}[Antitone map]
    \label{def:antitone}
    An \emph{\iindex{antitone map}} between two posets~$\posA=\tupp{\posAset, \posAleq}$ and~$\posB=\tupp{\posBset, \posBleq}$ is a map~$\mapa$ that reverses the ordering, in the sense that
    \begin{equation*}
        \prfperiod{
            \posela \posAleq \poselb
        }{
            \mapa(\posela) \posgeq_\posB \mapa(\poselb)
        }
        %\posAel_1 \posAleq \posAel_2 \quad \Imp \quad \mapa(\posAel_1) \posBleq \mapa(\posAel_2).
    \end{equation*}
\end{definition}

\begin{example}[Unit cost, total cost]
    Assume that you want to produce some widgets, and that the manifacturing cost depends on the number of widgets.
    The function describing the total cost~$\stylemaps{t}\colon \natnumbers\to \nonNegReals$ is a map between the ordered sets~$\natnumbers$ and~$\nonNegReals$, and maps each quantity of widgets to a total manufacturing cost (\cref{fig:total_manufacturing}).
    Clearly,~$\stylemaps{t}$ is a monotone function.
    Conversely, the unit cost function~$\stylemaps{u}\colon \natnumbers\to \nonNegReals$ is antitone (\cref{fig:unit_manufacturing}).
\end{example}

\begin{figure}[h!]
    \subfloat[
        Unit cost  \vs  number of widgets.
        \label{fig:unit_manufacturing}
    ]{
        \includesag{unit_manufacturing}
    }
    \subfloat[
        Total cost \vs number of widgets.
        \label{fig:total_manufacturing}
    ]{
        \includesag{total_manufacturing}
    }
\end{figure}

\begin{example}[Rounding functions]
    \label{ex:rounding-functions}
    In this example we look at three rounding functions: \funceil~(\cref{fig:ceil}), \funfloor~(\cref{fig:floor}), and~$\rtntte$~(\cref{fig:rtntte}).
    Both the maps
    \begin{equation*}
        \defmapcomma{\funceil}{\tup{\reals,\leq}}{\to}{\tup{\natnumbers,\leq}}{\ela}{\elb \in \natnumbers \colon \elb-1<\ela \leq \elb}
    \end{equation*}
    and
    \begin{equation*}
        \defmapcomma{\funfloor}{\tup{\reals,\leq}}{\to}{\tup{\natnumbers,\leq}}{\ela}{\elb \in \natnumbers \colon \elb\leq \ela < \elb+1}
    \end{equation*}
    are monotone, since~$\ela\leq \elc$ implies both~$\funceil(\ela)\leq \funceil(\elc)$ and~$\funfloor(\ela)\leq\funfloor(\elc)$.
    Furthermore, the map ``round to nearest, ties to even''
    \begin{equation*}
        \defmapperiod{\rtntte}{\tup{\reals,\leq}}{\to}{\tup{\natnumbers,\leq}}{x}{\begin{cases}
                \funfloor(\ela), & \ela< (\funfloor(\ela)+\funceil(\ela))/2                                                 \\
                \funceil(\ela),  & \ela> (\funfloor(\ela)+\funceil(\ela))/2                                                 \\
                \funceil(\ela),  & (\ela= (\funfloor(\ela)+\funceil(\ela))/2 )\booland (\funceil(\ela) \text{ mod } 2 = 0)  \\
                \funfloor(\ela), & (\ela= (\funfloor(\ela)+\funceil(\ela))/2 )\booland (\funfloor(\ela) \text{ mod } 2 = 0)
            \end{cases}}
    \end{equation*}

    \begin{figure*}[h!
        ]
        \centering
        \subfloat[$\funceil$. \label{fig:ceil}]{
            \includesag{ceil}}
        \hfill
        \subfloat[$\funfloor$. \label{fig:floor}]{
            \includesag{floor}}
        \hfill
        \subfloat[$\rtntte$. \label{fig:rtntte}]{
            \includesag{rtntte}}
    \end{figure*}
\end{example}

\begin{example}[Cardinality map]
    In \cref{def:power-poset} we presented the poset arising from the power set of a set~$\setA=\{\posela,\poselb,\poselc\}$ and ordered via subset inclusion.

    The map~$\vert \cdot \vert \colon \powerposet{\setA} \to \tup{\natnumbers,\leq}$ (cardinality), is a monotone map (\cref{fig:cardinality}).
    \begin{figure*}[h!]
        \centering
        \includesag{40_dpcatfig_exmonotone}
        \caption{The cardinality map is a monotone map. }
        \label{fig:cardinality}
    \end{figure*}
\end{example}

\begin{lemma}
    Consider a discrete poset~$\posA$ and a poset~$\posB$.
    Any map~$\mapa \colon \posA\to \posB$ is monotone.
\end{lemma}
\newcommand{\samewidth}[1]{\makebox[3cm]{$#1$}}
\begin{proof}
    Since~$\posA$ is a discrete poset, one has
    \begin{equation*}
        \prfdoubleperiod{
            \posAel_1\posAleq \posAel_2
        }{
            \posAel_1=\posAel_2
        }
        %\posAel_1\posAleq \posAel_2 \iff \posAel_1=\posAel_2.
    \end{equation*}
    Therefore, one has
    \begin{equation*}
        \prfperiod{
            \prftree{
                \prftree[r]{\quad($P$ discrete)}{
                    \samewidth{\posAel_1\posAleq \posAel_2}
                }{
                    \samewidth{\posAel_1=\posAel_2}
                }
            }{
                \samewidth{\mapa(\posAel_1)=\mapa(\posAel_2)}
            }
        }{
            \samewidth{\mapa(\posAel_1)\posBleq \mapa(\posAel_2)}
        }
    \end{equation*}
\end{proof}
Unless indicated otherwise, in this book all maps between posets are assumed to be monotone or will turn out to be monotone.
In a similar way, one can define antitone maps.
