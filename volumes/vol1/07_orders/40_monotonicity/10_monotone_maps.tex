% !TEX root = chapter-standalone.tex

\section{Monotone maps}
\label{sec:monotonicity-monotone-maps}

%\linkvideo{spring2021-functors:semi-and-fun:mon-functions} % Monotone functions
% \linkvideo{spring2021-functors:semi-and-fun:mon-functions:fun-req-mon} % Functionalities and requirements
\linkvideo{spring2021-functors:semi-and-fun:mon-functions:mon-on-pos} % Monotone functions on posets

A monotone map is the generalization to posets of a ``non-decreasing'' function on real numbers.
The function $x \mapsto \max(0, 42 x)$ is non-decreasing on the real numbers because
\begin{equation}\label{eq:monotone-eq-1}
    \prfdoubleperiod{
        x \leq y
    }{
        \max(0, 42 x) \leq \max(0, 42 y)
    }
\end{equation}
Note that we use ``$\leq$'' and not ``$<$''.
Non-decreasing is weaker than ``increasing''.

The definition of monotone function on a poset is the direct generalization of this concept; the only change  is that we use the partial order rather than the total order on the reals.

\begin{definition}[Monotone map]
    \label{def:monotone}
    A \emph{\iindex{monotone map}} between two posets~$\posA=\tupp{\posAset, \posAleq}$ and~$\posB=\tupp{\posBset, \posBleq}$ is a map~$\mapa \colon \posAset \sto \posBset$ that preserves the ordering, in the sense that
    \begin{equation}
        \prfperiod{
            \posela \posAleq \poselb
        }{
            \mapa(\posela) \posBleq \mapa(\poselb)
        }
        %\posAel_1 \posAleq \posAel_2 \quad \Imp \quad \mapa(\posAel_1) \posBleq \mapa(\posAel_2).
    \end{equation}
\end{definition}

\begin{example}[Identity is monotone]
    Given a poset~\posA, the identity map $\catidat{\posAset}\colon \posAset \mto\linebreak[0] \posAset$ is monotone, since if~$\posela\posAleq \poselb$, then~$\mapidat{\posAset}(\posela)=\posela \posAleq  \poselb =  \mapidat{\posAset}(\poselb )$.
\end{example}

\begin{example}[Constant maps are monotone]
    Every constant map is monotone.
\end{example}

\begin{figure*}[b]
    \centering
    \includesag{40_dpcatfig_exmonotone}
    \caption{The cardinality map is a monotone map. }
    \label{fig:cardinality}
\end{figure*}

\begin{example}[Cardinality map]\label{exa:cardinality}
    Consider the power poset (\cref{def:power-poset}) $\powerposet\setA$ of a finite set \setA.
    The cardinality map
    \begin{equation}\label{eq:cardinality-map}
        \cardmap : \powerset\setA \sto \natnumbers
    \end{equation}
    is monotone when considered as a map from the poset $\powerposet\setA$ to the poset $\tup{\natnumbers,\Nleq}$.
    % \begin{equation}
    %     \defmap{
    %         \cardmap
    %     }{
    %         \powerposet{\setA}
    %     }{
    %         \sto
    %     }{
    %         \tup{\natnumbers,\leq}
    %     }{
    %         \subA
    %     }{
    %         \vert\subA\vert
    %     }
    % \end{equation}
    % is monotone.
    \Cref{fig:cardinality} shows a visualization of this map for the set~$\setA=\makeset{\posela,\poselb,\poselc}$.
    To prove this, recall that in the power poset subsets are ordered by inclusion.
    Therefore, we need to show that
    \begin{equation}\label{eq:cardinality}
        \prfdoubleperiod{
            \subA \setsubseteq \subB
        }{
            \cardof \subA  \leq  \cardof \subB
        }
    \end{equation}
    This is easy to see that, because all elements of $\subA$ are also in $\subB$, the cardinality of~$\subA$ cannot be more than the cardinality of~$\subB$.
    Monotonicity depends on the partial order used on the domain and the codomain.
    To indicate that a map is monotone, we write it indicating the two posets as the domain/codomain:
    \begin{equation}\label{eq:cardinality-as-posmap}
        \cardmap \colon \tup{\powerset\setA, \setsubseteq} \mto \tup{\natnumbers, \Nleq}.
    \end{equation}
\end{example}

\begin{lemma}\label{lem:discrete-is-monotone}
    Consider a discrete poset~\posA and a poset~\posB.
    Any map~$\mapa \colon \posA\to \posB$ is monotone.
\end{lemma}
\newcommand{\samewidth}[1]{\makebox[3cm]{$#1$}}
\begin{proof}
    Since~\posA is a discrete poset, we have
    \begin{equation}
        \prfdoubleperiod{
            \posAnel{1}\posAleq \posAnel{2}
        }{
            \posAnel{1}=\posAnel{2}
        }
    \end{equation}
    Therefore, we have
    \begin{equation}
        \prfperiod{
            \prftree{
                \prftree[r]{
                    \quad($P$ discrete)
                }{
                    \samewidth{\posAnel{1}\posAleq \posAnel{2}}
                }{
                    \samewidth{\posAnel{1}=\posAnel{2}}
                }
            }{
                \samewidth{\mapa(\posAnel{1})=\mapa(\posAnel{2})}
            }
        }{
            \samewidth{\mapa(\posAnel{1})\posBleq \mapa(\posAnel{2})}
        }
    \end{equation}
\end{proof}
% Unless indicated otherwise, in this book all maps between posets are assumed to be monotone or will turn out to be monotone.

\clearpage
\vfill
\begin{gradedexercise}[\exname{MonotoneMapCheck}]
    \label{ex:MonotoneMapCheck?}

    Prove your answers to the following questions.
    \begin{enumerate}
        \item Is the function
              \begin{equation}\label{eq:MonotoneMapCheck-1}
                  \defmap{\mapa}{\tup{\wnumbers,\Nleq}}{\to}{\tup{\wnumbers, \Nleq}}{\ela}{\ela^2}
              \end{equation}
              monotone?
              % \item Let~$\setA = \makeset{\ela, \elb, \elc}$ and consider the posets~$\tup{\powerset{\setA}, \setsubseteq}$ and~$\tup{\natnumbers, \leq}$.
              %       Let $\cardmap$ be
              %       \begin{equation}
              %           \defmap{\mapa}{\posA}{\to}{\natnumbers}{\subA}{\vert \subA \vert}
              %       \end{equation}
              %       be the function which calculates the cardinality of any subset of~$\setA$.
              %       Is~$\mapa$ monotone?
              %       \todotext{\alphubel: @JL: Note that this question cannot be used as exercise because it's an example.}
        \item Consider the set of natural numbers which divide the number 36, equip\-ped with the partial order such that~$\ela \leq \elb$ if and only if~$\ela$ divides~$\elb$.
              Call this poset~$\posA = \tup{\posAset, \posleq}$, and let~$\mapa \colon \posAset \sto \boolset$ be defined by
              \begin{equation}\label{eq:MonotoneMapCheck-2}
                  \mapa(\ela) =
                  \begin{cases}
                      \true  & \text{ if } \ela \text{ is an even number, } \\
                      \false & \text{ if } \ela \text{ is an odd number.
                      }
                  \end{cases}
              \end{equation}
              Is~$\mapa$ monotone if we equip~$\makeset{\false, \true}$ with the usual partial order such that~$\false \leq \true$?
    \end{enumerate}
\end{gradedexercise}
\solutionof{MonotoneMapCheck}

