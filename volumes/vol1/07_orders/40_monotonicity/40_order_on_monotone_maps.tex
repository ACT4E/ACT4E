\section{Order on monotone maps}

Fixed two poset



\begin{example}[Rounding functions]
    \label{ex:rounding-functions}
    In this example we look at three rounding functions: \funceil~(\cref{fig:ceil}), \funfloor~(\cref{fig:floor}), and~$\rtntte$~(\cref{fig:rtntte}).
    Both the maps
    \begin{equation}
        \defmapcomma{
            \funceil
        }{
            \tup{\reals,\leq}
        }{
            \to
        }{
            \tup{\natnumbers,\leq}
        }{
            \ela
        }{
            \elb \setin \natnumbers \colon \elb-1<\ela \leq \elb
        }
    \end{equation}
    and
    \begin{equation}
        \defmapcomma{
            \funfloor
        }{
            \tup{\reals,\leq}
        }{
            \to
        }{
            \tup{\natnumbers,\leq}
        }{
            \ela
        }{
            \elb \setin \natnumbers \colon \elb\leq \ela < \elb+1
        }
    \end{equation}
    are monotone, since~$\ela\leq \elc$ implies both~$\funceil(\ela)\leq \funceil(\elc)$ and~$\funfloor(\ela)\leq\funfloor(\elc)$.
    Furthermore, the map ``round to nearest, ties to even'' is defined as
    \begin{widepar}
        \begin{equation}
            \defmapperiod{
                \rtntte
            }{
                \tup{\reals,\leq}
            }{
                \to
            }{
                \tup{\natnumbers,\leq}
            }{
                x
            }{
                \begin{cases}
                    \funfloor(\ela), & \ela< (\funfloor(\ela)+\funceil(\ela))/2                                                 \\
                    \funceil(\ela),  & \ela> (\funfloor(\ela)+\funceil(\ela))/2                                                 \\
                    \funceil(\ela),  & (\ela= (\funfloor(\ela)+\funceil(\ela))/2 )\booland (\funceil(\ela) \text{ mod } 2 = 0)  \\
                    \funfloor(\ela), & (\ela= (\funfloor(\ela)+\funceil(\ela))/2 )\booland (\funfloor(\ela) \text{ mod } 2 = 0)
                \end{cases}
            }
        \end{equation}
    \end{widepar}
    \begin{figure*}[h!]
        \centering
        \subfloat[$\funceil$.
            \label{fig:ceil}]{
            \includesag{ceil}}
        \hfill
        \subfloat[$\funfloor$. \label{fig:floor}]{
            \includesag{floor}}
        \hfill
        \subfloat[$\rtntte$. \label{fig:rtntte}]{
            \includesag{rtntte}}
    \end{figure*}
\end{example}