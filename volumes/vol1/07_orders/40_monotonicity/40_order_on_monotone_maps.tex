\section{Order on monotone maps}

Fixed two \SY{posets} \posA and \posB, the set of \SY{monotone maps} $\posA \mto \posB$ form a \SY{poset} themselves.
We can order them point wise.

\begin{ctdefinition}[Order on monotone maps]
    \label{def:order-monotone-maps}
    \SYNDEF{order on monotone maps}
    Consider two \SY{monotone maps} $\mora,\morb\colon\posA\mto\posB$.
    We say that $\mora$ precedes $\morb$ if, point wise, the output of $\mora$ precedes the output of $\morb$ when presented with the same input:
    \begin{equation}
        \prfdoubleperiod{
            \mora \posleqof{\posA\mto\posB} \morb
        }{
            \forall \posela\setin\posAset \colon \mora(\posela) \posleqof\posB \morb(\posela)
        }
    \end{equation}
\end{ctdefinition}

\begin{example}[Rounding functions]
    \label{ex:rounding-functions}
    In this example we look at ``rounding functions'': these are functions that truncate a real number to an integer.
    You might already know the ceiling function \funceil~(\cref{fig:ceil}) and the floor function \funfloor~(\cref{fig:floor}),
    which are formally defined as
    \begin{equation}\label{eq:funceil}
        \defmapcomma{
            \funceil
        }{
            \realswithleq
        }{
            \mto
        }{
            \natswithleq
        }{
            \ela
        }{
            \min \makeset{\elb \setin \natnumbers \colon \elb \geq \ela }
            % \elb \setin \natnumbers \colon \elb-1<\ela \leq \elb
        }
    \end{equation}
    and
    \begin{equation}\label{eq:funfloor}
        \defmapperiod{
            \funfloor
        }{
            \realswithleq
        }{
            \mto
        }{
            \natswithleq
        }{
            \ela
        }{
            \max \makeset{\elb \setin \natnumbers \colon \elb \Nleq \ela }
        }
    \end{equation}
    The functions $\funceil$ and $\funfloor$ are monotone, since~$\ela\Rleq \elc$ implies both~$\funceil(\ela)\Nleq \funceil(\elc)$ and~$\funfloor(\ela)\Nleq\funfloor(\elc)$.

    There exist many other rounding functions, commonly used by computers.
    For example, the map ``round to nearest, ties to even'' \cite{P754:2008:ISF}
    rounds a number to the closest integer, and in case of ties it rounds to the even one (\cref{fig:rtntte}).
    For example, $3.2$ is mapped to $3$, $1.5$ is mapped to $2$, and $4.5$ is mapped to $4$.
    This is the formal definition:
    %
    \begin{widepar}
        \begin{equation}\label{eq:rtntte}
            \defmapperiod{
                \rtntte
            }{
                \realswithleq
            }{
                \mto
            }{
                \natswithleq
            }{
                x
            }{
                \begin{cases}
                    \funfloor(\ela), & \ela< (\funfloor(\ela)+\funceil(\ela))/2                                                \\
                    \funceil(\ela),  & \ela> (\funfloor(\ela)+\funceil(\ela))/2                                                \\
                    \funceil(\ela),  & (\ela= (\funfloor(\ela)+\funceil(\ela))/2 )\booland (\funceil(\ela) \text{ is even } )  \\
                    \funfloor(\ela), & (\ela= (\funfloor(\ela)+\funceil(\ela))/2 )\booland (\funfloor(\ela) \text{ is even } )
                \end{cases}
            }
        \end{equation}
    \end{widepar}
    \begin{marginfigure}
        \includesag{floorceilhasse}
        \caption{}
        \label{fig:floorceilhasse}
        \todographics{\bernina: @Gioele: can we align id and rtntte in the center?}
    \end{marginfigure}
    In this example, note that
    \begin{equation}\label{eq:funfloor-mapid-funceil}
        \funfloor \posleq \mapid \posleq \funceil
    \end{equation}
    and
    \begin{equation}\label{eq:funfloor-rtntte-funceil}
        \funfloor \posleq \rtntte \posleq \funceil,
    \end{equation}
    and $\mapid$ and $\rtntte$ are not comparable (see \cref{fig:floorceilhasse}).
\end{example}

\begin{figure*}[b]
    \centering
    \subfloat[$\funceil$.
        \label{fig:ceil}]{
        \includesag{ceil}}
    \hfill
    \subfloat[$\funfloor$. \label{fig:floor}]{
        \includesag{floor}}
    \hfill
    \subfloat[$\rtntte$. \label{fig:rtntte}]{
        \includesag{rtntte}}
    \caption{Comparison of three rounding methods.}
\end{figure*}
