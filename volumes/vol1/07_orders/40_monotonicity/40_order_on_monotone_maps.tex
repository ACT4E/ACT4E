\section{Order on monotone maps}

Fixed two poset $\posA$ and $\posB$, the set of monotone maps $\posA \mto \posB$ form a poset themselves.
We can order them point wise.

\begin{definition}[Order on monotone maps]
    \label{def:order-monotone-maps}
    Consider two monotone maps $\mora,\morb\colon\posA\mto\posB$.
    We say that $\mora$ precedes $\morb$ if:
    \begin{equation}
        \prfdoubleperiod{
            \mora \posleqof{\posA\mto\posB} \morb
        }{
            \forall \posela\setin\posAset \colon \mora(\posela) \posleqof\posB \mora(\poselb)
        }
    \end{equation}
\end{definition}

\begin{example}[Rounding functions]
    \label{ex:rounding-functions}
    In this example we look at three rounding functions: \funceil~(\cref{fig:ceil}), \funfloor~(\cref{fig:floor}), and~$\rtntte$~(\cref{fig:rtntte}).
    Both the maps
    \begin{equation}
        \defmapcomma{
            \funceil
        }{
            \tup{\reals,\leq}
        }{
            \to
        }{
            \tup{\natnumbers,\leq}
        }{
            \ela
        }{
            \min \makeset{\elb \setin \natnumbers \colon \elb \geq \ela }
            % \elb \setin \natnumbers \colon \elb-1<\ela \leq \elb
        }
    \end{equation}
    and
    \begin{equation}
        \defmapcomma{
            \funfloor
        }{
            \tup{\reals,\leq}
        }{
            \to
        }{
            \tup{\natnumbers,\leq}
        }{
            \ela
        }{
            % \elb \setin \natnumbers \colon \elb\leq \ela < \elb+1
            \max \makeset{\elb \setin \natnumbers \colon \elb \leq \ela }
        }
    \end{equation}
    are monotone, since~$\ela\leq \elc$ implies both~$\funceil(\ela)\leq \funceil(\elc)$ and~$\funfloor(\ela)\leq\funfloor(\elc)$.
    There are other rounding functions.
    For example, the map ``round to nearest, ties to even'' \cite{P754:2008:ISF} is defined as
    % 
    \begin{widepar}
        \begin{equation}
            \defmapperiod{
                \rtntte
            }{
                \tup{\reals,\leq}
            }{
                \to
            }{
                \tup{\natnumbers,\leq}
            }{
                x
            }{
                \begin{cases}
                    \funfloor(\ela), & \ela< (\funfloor(\ela)+\funceil(\ela))/2                                                \\
                    \funceil(\ela),  & \ela> (\funfloor(\ela)+\funceil(\ela))/2                                                \\
                    \funceil(\ela),  & (\ela= (\funfloor(\ela)+\funceil(\ela))/2 )\booland (\funceil(\ela) \text{ is even } ) \\
                    \funfloor(\ela), & (\ela= (\funfloor(\ela)+\funceil(\ela))/2 )\booland (\funfloor(\ela) \text{ is even } )
                \end{cases}
            }
        \end{equation}
    \end{widepar}
    \begin{figure*}[h!]
        \centering
        \subfloat[$\funceil$.
            \label{fig:ceil}]{
            \includesag{ceil}}
        \hfill
        \subfloat[$\funfloor$. \label{fig:floor}]{
            \includesag{floor}}
        \hfill
        \subfloat[$\rtntte$. \label{fig:rtntte}]{
            \includesag{rtntte}}
    \end{figure*}

    In this example, note that
    \begin{equation*}
        \funfloor \posleq \mapid \posleq \funceil
    \end{equation*}
    and
    \begin{equation*}
        \funfloor \posleq \rtntte \posleq \funceil,
    \end{equation*}
    and~$\mapid$ and $\rtntte$ are not comparable (see \cref{fig:floorceilhasse}).
    \begin{marginfigure}
    \includesag{floorceilhasse}
    \caption{}
    \label{fig:floorceilhasse}
    \end{marginfigure}
\end{example}