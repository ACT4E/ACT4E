\sectionexercises{Monotone maps}

The interface for a monotone map is shown in \cref{lst:MonotoneMap}.
It only specifies that it is a map between two posets.
The implied ``contract'' is the monotonicity property that must hold. (In programming languages with a richer type system than Python we could describe the contract using types; with Python, we need to rely on this implicit contract.)
% \begin{widepar}
%     \aligninner{%
%     \begin{minipage}{15cm}
\classlisting{MonotoneMap}
\classlisting{FiniteMonotoneMap}

% \begin{tabular}{cc}
%     \begin{minipage}{0.4\textwidth}
%         \classlisting{MonotoneMap}
%     \end{minipage}
%      &
%     \begin{minipage}{0.4\textwidth}
%         \classlisting{FiniteMonotoneMap}
%     \end{minipage}
% \end{tabular}
% \end{minipage}
%     }
% \end{widepar}

\codeboilerplate{FiniteMonotoneMapProperties}{
    Check if a map is monotone or antitone.
}
\begin{widepar}
    \aligninner{%
        \begin{minipage}{15cm}

            \classlisting{FiniteMonotoneMapProperties}
        \end{minipage}
    }
\end{widepar}

\begin{hint}
    You can reduce \funcname{is_antitone} to a call of \funcname{is_monotone} by using opposite posets.
\end{hint}
