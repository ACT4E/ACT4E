% !TEX root = chapter-standalone.tex
\section{Embeddings}

\begin{ctdefinition}[Order isomorphism]
    \label{def:order-isomorphism}
    A \SY{monotone map} is an \maindef{order isomorphism} if the other direction of the implication holds as well:
    \begin{equation}
        \prfdoubleperiod{
            \posAel \posleqof\posA \posBel
        }{
            \mapa(\posAel) \posleqof\posB \mapa(\posBel)
        }
    \end{equation}
\end{ctdefinition}

An order isomorphism is also called an \emph{embedding}.

\Cref{fig:order-embedding} shows the embedding relations among the \SY{poset} classes defined in \cref{fig:posets-up-to-four}.

\begin{figure*}
    \includegraphics[width=\textwidth]{order-embedding}
    \caption{Order embeddings for unlabeled \SY{posets} up to 4 elements}
    \label{fig:order-embedding}
    \todographicsjira{565}{\bernina: Is this the best visualization?
        It does not need to be a Hasse diagram.
        After deciding visualization, make tikz.
    }
\end{figure*}
