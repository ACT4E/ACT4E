% !TEX root = chapter-standalone.tex
\section{Monotone relations}

\todotext{J: introduce DPs already here}


\begin{definition}[monotone relation]
Let $\posAdefinition$ and~$\posBdefinition$ be posets. A monotone relation $\relA \colon \posA \mto \posB$ is a relation $\relA \colon \posAset \to \posBset$ such that for all $\ela, \ela' \in \posAset$ and all $\elb, \elb' \in \posBset$, 
\begin{enumerate}
\item $\tup{\ela, \elb} \in \relA, \ \ela' \leq_{\posA} \ela \ \implies \tup{\ela', \elb} \in \relA$;
\item $\tup{\ela, \elb} \in \relA, \ \elb \leq_{\posB} \elb' \ \implies \tup{\ela, \elb'} \in \relA$.
\end{enumerate}
\end{definition}

\subsection{Monotone relations and co-design}

A monotone relation
\begin{equation}
\adp \colon \funsp \mto \ressp
\end{equation}
can be used to model a relationship of``feasability'' between a poset $\funsp$ of ``functionalities'' and a poset $\ressp$ of ``requirements", in the sense that the relation describes whether a resource $\fun \setin \funsp$, seen as a functionality or service or product, is \emph{feasible} to obtain given a certain resource $\res \setin \ressp$, with $\res$ interpreted as a requirement or a cost.

The condition
\begin{equation}
\tup{\fun, \elb} \in \adp, \ \res \leq_{\ressp} \res' \ \implies \tup{\fun, \res'} \in \adp
\end{equation}
says that if $\fun$ is feasible to obtain using $\res$, then it is also feasible to obtain $\fun$ if we use more resources, $\res'$. 

The condition
\begin{equation}
\tup{\fun, \res} \in \adp, \ \fun' \leq_{\funsp} \fun \ \implies \tup{\fun', \res} \in \adp,
\end{equation}
on the other hand, says that if $\fun$ is feasible to obtain using $\res$ amount of resources, then it is also feasible to obtain less, $\fun'$, using the same resources $\res$. 