% !TEX root = chapter-standalone.tex
\section{Antitone maps}

Dually to \SY{monotone functions}, we can define \SY{antitone maps} as order \emph{reversing} functions.

\begin{definition}[Antitone map]
    \label{def:antitone}
    An \maindef{antitone map} between two \SY{posets}~$\posAdefinition$ and~$\posBdefinition$ is a map~$\mapa$ that reverses the ordering, in the sense that
    \begin{equation}
        \prfperiod{
            \posela \posAleq \poselb
        }{
            \mapa(\posela) \posgeqof\posB \mapa(\poselb)
        }
    \end{equation}
\end{definition}

\begin{example}[Unit cost, total cost]
    Assume that you want to produce some widgets, and that the manufacturing cost depends on the number of widgets.
    The function describing the total cost~$\stylemaps{t}\colon \natnumbers \sto \nonNegReals$ is a map between the ordered sets~\natnumbers and~\nonNegReals, and maps each quantity of widgets to a total manufacturing cost (\cref{fig:total_manufacturing}).
    Clearly,~$\stylemaps{t}$ is a \SY{monotone function}.
    Conversely, the unit cost function~$\stylemaps{u}\colon \natnumbers \sto \nonNegReals$ is antitone (\cref{fig:unit_manufacturing}).
\end{example}

\begin{marginfigure}
    \subfloat[
        Unit cost  \vs  number of widgets.
        \label{fig:unit_manufacturing}
    ]{
        \includesag{unit_manufacturing}
    }

    \subfloat[
        Total cost \vs number of widgets.
        \label{fig:total_manufacturing}
    ]{
        \includesag{total_manufacturing}
    }
    \caption{Unit and total costs vs. number of widgets.}
\end{marginfigure}

It is easy to see that an \SY{antitone map} $\mapa: \posA \mto \posB$ is the same thing as a \SY{monotone map} $\mapa: \posAop \mto \posB$.

\begin{lemma}\label{lem:antitone-is-monotone}
    An \SY{antitone map} $\mapa: \posA \mto \posB$ is a \SY{monotone map} $\mapa: \posAop \mto \posB$
    and a \SY{monotone map} $\mapa: \posA \mto \posB\posop$.
\end{lemma}
