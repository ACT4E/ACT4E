\section{Antitone maps}

Dually to monotone functions, we can define antitone maps as order \emph{reversing} functions.

\begin{definition}[Antitone map]
    \label{def:antitone}
    An \emph{\iindex{antitone map}} between two posets~$\posA=\tupp{\posAset, \posAleq}$ and~$\posB=\tupp{\posBset, \posBleq}$ is a map~$\mapa$ that reverses the ordering, in the sense that
    \begin{equation}
        \prfperiod{
            \posela \posAleq \poselb
        }{
            \mapa(\posela) \posgeq_\posB \mapa(\poselb)
        }
    \end{equation}
\end{definition}

\begin{example}[Unit cost, total cost]
    Assume that you want to produce some widgets, and that the manufacturing cost depends on the number of widgets.
    The function describing the total cost~$\stylemaps{t}\colon \natnumbers \sto \nonNegReals$ is a map between the ordered sets~$\natnumbers$ and~$\nonNegReals$, and maps each quantity of widgets to a total manufacturing cost (\cref{fig:total_manufacturing}).
    Clearly,~$\stylemaps{t}$ is a monotone function.
    Conversely, the unit cost function~$\stylemaps{u}\colon \natnumbers \sto \nonNegReals$ is antitone (\cref{fig:unit_manufacturing}).
\end{example}

\begin{figure}[h!]
    \subfloat[
        Unit cost  \vs  number of widgets.
        \label{fig:unit_manufacturing}
    ]{
        \includesag{unit_manufacturing}
    }
    \subfloat[
        Total cost \vs number of widgets.
        \label{fig:total_manufacturing}
    ]{
        \includesag{total_manufacturing}
    }
    \todographics{@Gioele: put y axis labels on the inside}
\end{figure}

It is easy to see that an antitone map $\mapa: \posA \mto \posB$ is the same thing as a monotone map $\mapa: \posA\op \mto \posB$.

\begin{lemma}\label{lem:antitone-is-monotone}
    An antitone map  $\mapa: \posA \mto \posB$ is a monotone map $\mapa: \posA\op \mto \posB$ 
    and a monotone map $\mapa: \posA \mto \posB\op$.
\end{lemma}