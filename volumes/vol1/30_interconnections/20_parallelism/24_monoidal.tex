% !TEX root = chapter-standalone.tex

\section{Monoidal stacking (semi)categories}\label{sec:moinoidal-stacking}

\todotext{Add some introduction.
    Why do we care about this?}
\begin{ctdefinition}[Strict monoidal stacking semicategory]
    \label{def:strict-monoidal-stacking-semicat}
    A \maindef{strict monoidal stacking semicategory} is a \SY{stacking semicategory} $\tup{\CatC, \mtimescatob, \mtimescatmor}$ with

    \constit

    \begin{enumerate}
        \item an object $\idmoncat \setin \Obof{\CatC}$, called the \emph{monoidal unit}
    \end{enumerate}

    \condit

    \begin{enumerate}
        \item For any object $\Obja$ of \CatC,
              \begin{equation}\label{eq:strict-mon-stacking-semicat-cond1}
                  \Obja \mtimescatob \idmoncat = \Obja \qquad \text{and} \qquad \idmoncat \mtimescatob \Obja = \Obja.
              \end{equation}
        \item The monoidal unit $\idmoncat$ has an \SY{identity morphism} $\catid_\idmoncat$, and for any morphism $\mora \colon \Obja \mto \Objb$,
              \begin{equation}\label{eq:strict-mon-stacking-semicat-cond2}
                  \mora \mtimescatmor \catid_\idmoncat = \mora \qquad \text{and} \qquad \catid_\idmoncat \mtimescatmor  \mora = \mora.
              \end{equation}
    \end{enumerate}
\end{ctdefinition}

\todotext{Talk about the graphical notation: we can write the 1 wire as just empty spaces. (Note that so far we always had morphisms as boxes with one wire on the left and right.)}

\begin{example}
    Consider the associative stacking category from \cref{ex:assoc-stacking-semicat-integers-max}, where objects are integers and stacking of objects is taking their maximum.
    There is no possible monoidal unit here: it would have to be a neutral element for the operation ``max'', but such does not exist for $\wnumbers$.
    However, we could modify this example, and replace $\wnumbers$ with a bounded set of numbers, such as $\natnumbers$.
    Then the smallest number in $\natnumbers$, namely $0$, serves a neutral element for ``max'' and provides a monoidal unit $\idmoncat$.
\end{example}

\begin{example}
    Consider the associative stacking category from \cref{ex:assoc-stacking-semicat-lists-concat}.
    A monoidal unit would need to be a neutral element for list concatenation.
    This would be the empty list.
    In \cref{ex:assoc-stacking-semicat-lists-concat} we specified that objects are only non-empty lists, hence we do not have a strict monoidal semicategory.
    However, this example can the be easily adjusted to include also the empty list, in which case we \emph{do} obtain a strict monoidal semicategory.
\end{example}

\begin{example}
    We can look at \Moore and ask whether it is a strict monoidal semicategory.
    The monoidal unit is given by the object
    $$\idmoncat = \cObj{}.
    $$
    Its identity morphism is the Moore machine
    \begin{equation*}
        \catid_\idmoncat=\tup{\cObj{},\cObj{},\cObj{},\prdyn_\idmoncat, \prreadout_\idmoncat, \tup{}},
    \end{equation*}
    where
    \begin{equation*}
        \defmapcomma{\prdyn_{\idmoncat}}
        {\cObj{} \cprod \cObj{}}
        {\sto}
        {\cObj{}}
        {\tup{}\tupconcat \tup{}}
        {\tup{}}
    \end{equation*}
    and
    \begin{equation*}
        \defmapperiod{\prreadout_{\idmoncat}}
        {\cObj{}}
        {\sto}
        {\cObj{}}
        {\tup{}}
        {\tup{}}
    \end{equation*}
    Clearly, $\prObja \cprod \cObj{}=\cObj{}\cprod \prObja=\prObja$ for every $\prObja\setin \Ob_\Moore$.
    Furthermore, consider a Moore machine $\mora\colon \prinL \mto \proutL$ with
    \begin{equation*}
        \mora=\tup{\prinL,\prstL, \proutL, \prdyn,\prreadout,\prstart}.
    \end{equation*}
    One has:
    \begin{equation*}
        \begin{aligned}
            \mora \mtimescatmor \catid_\idmoncat & =\tup{\prinL \cprod \cObj{},\prstL \cprod \cObj{},\proutL \cprod \cObj{},\prdyn_{\mora \mtimescatmor\catid_\idmoncat}, \prreadout_{\mora \mtimescatmor\catid_\idmoncat}, \prstart\tupconcat \tup{}} \\
                                                 & =\tup{\prinL,\prstL ,\proutL,\prdyn, \prreadout, \prstart}=\mora,
        \end{aligned}
    \end{equation*}
    where we used
    \begin{equation*}
        \begin{aligned}
            \prdyn_{\mora \mtimescatmor\catid_\idmoncat}\colon \prinL \cprod \cObj{}\cprod \prstL \cprod \cObj{} & \sto \prstL \cprod \cObj{} \\
            \prinel \tupconcat \tup{}\tupconcat \prstel \tupconcat \tup{}                                        & \mapsto \prdyn(\prinel,\prstel)\tupconcat \prdyn_\idmoncat(\tup{},\tup{})=\prdyn(\prinel,\prstel)
        \end{aligned}
    \end{equation*}
    and
    \begin{equation*}
        \begin{aligned}
            \prreadout_{\mora \mtimescatmor\catid_\idmoncat}\colon \prstL \cprod \cObj{} & \sto \proutL \cprod \cObj{} \\
            \prstel \tupconcat \tup{}                                                    & \mapsto \prreadout(\prstel)\tupconcat \prreadout_\idmoncat(\tup{})=\prreadout(\prstel)
        \end{aligned}
    \end{equation*}
    to show the equivalences $\prdyn=\prdyn_{\mora \mtimescatmor \catid_\idmoncat}$ and $\prreadout=\prreadout_{\mora \mtimescatmor \catid_\idmoncat}$.
    The argument for $\catid_\idmoncat \mtimescatmor \mora$ follows analogously.
\end{example}

\begin{ctdefinition}[Strict monoidal stacking category]
    \label{def:strict-monoidal-stacking-cat}
    A \maindef{strict monoidal stacking category} is a \SY{strict monoidal stacking semicategory} for which all identity morphisms exist.
\end{ctdefinition}

\begin{example}
    \LTI, equipped with previously described stacking operations and an appropriate unit, is a strict monoidal stacking category.
    The unit is given by the object 0, and its identity morphism is given by the LTI system
    \begin{equation*}
        \catid_\idmoncat=\tup{\mat{0}^{0\times 1}, \mat{0}^{0\times 0}, \mat{0}^{0\times 0}, \mat{0}^{0\times 0},\mat{0}^{0\times 0}}.
    \end{equation*}
    On the side of objects, clearly $l+0=0+l=l$ for any object $l\setin \Ob_\LTI$.
    Consider $\mora\colon l\mto m$.
    On the side of morphisms we have:
    \begin{equation*}
        \begin{aligned}
            \mora \mtimescatmor \catid_\idmoncat & =
            \langle\begin{bmatrix}\prstart \\ \mat{0}^{0\times 1}\end{bmatrix},
            \begin{bmatrix}\mat{A}&\mat{0}\\ \mat{0}&\mat{0}^{0\times 0}\end{bmatrix},
            \begin{bmatrix}\mat{B}&\mat{0}\\ \mat{0}&\mat{0}^{0\times 0}\end{bmatrix},
            \begin{bmatrix}\mat{C}&\mat{0}\\ \mat{0}&\mat{0}^{0\times 0}\end{bmatrix},
            \begin{bmatrix}\mat{D}&\mat{0}\\ \mat{0}&\mat{0}^{0\times 0}\end{bmatrix}\rangle \\
                                                 & =\genericlti{}
        \end{aligned}
    \end{equation*}
    Similarly:
    \begin{equation*}
        \begin{aligned}
            \catid_\idmoncat \mtimescatmor \mora & =
            \langle\begin{bmatrix}\mat{0}^{0\times 1}\\ \prstart\end{bmatrix},
            \begin{bmatrix}\mat{0}^{0\times 0}&\mat{0}\\\mat{0}&\mat{A}\end{bmatrix},
            \begin{bmatrix}\mat{0}^{0\times 0}&\mat{0}\\\mat{0}&\mat{B}\end{bmatrix},
            \begin{bmatrix}\mat{0}^{0\times 0}&\mat{0}\\\mat{0}&\mat{C}\end{bmatrix},
            \begin{bmatrix}\mat{0}^{0\times 0}&\mat{0}\\\mat{0}&\mat{D}\end{bmatrix}\rangle \\
                                                 & =\genericlti{}
        \end{aligned}
    \end{equation*}
\end{example}

\begin{gradedexercise}[\exname{StringDiagrams}]\label{ex:StringDiagrams}

    The following is a string diagram which can represent a composition of morphisms (having certain types) in any given monoidal category~$\tup{\CatC,\mtimescat,\idmoncat}$
    (We read the diagram left-to-right for series composition, and top-to-bottom for parallel composition).
    The resulting morphism described by the total diagram -- call it $\morc$ -- is one of the type~$\morc : \Obja \mto \Objc$.

    \begin{center}
        \label{eq:ExamStringDiagram1}
        %\includegraphics[width=0.5\linewidth]{pics/ExamStringDiagram1.png}
        \includesag{string_diag}
    \end{center}

    In each part of this exercise, we will specify a monoidal category and specific objects and morphisms to plug into the variables~$\Obja, \Objb, \Objc$ and~$\mora, \morb$ in this diagram.
    Your task is to compute the respective resulting morphism $\morc$ as dictated by the diagram.

    \begin{enumerate}
        \item In this part, let~$\tup{\CatC,\mtimescat,\idmoncat}$ be the monoidal category where~\CatC is the category~\Set of sets and functions,~$\mtimescat$ is the cartesian product of sets and functions, and~$\idmoncat$ is a chosen~$1$-element set that we denote by~$\singleton$.
              In the string diagram above, let~$\Obja = \wnumbers$,~$\Objb = \natnumbers$, and~$\Objc = \wnumbers$.
              Furthermore, let
              \begin{equation}
                  \mapa\colon \singleton \mto \Objb
              \end{equation}
              be the function with~$\mapa(\singletonel) = 5$, and let
              \begin{equation}
                  \defmapperiod{\mapb}{\natnumbers\cartprod \wnumbers}{\mto}{\wnumbers}{\tup{\elb,\ela}}{\elb+\ela}
              \end{equation}
              Compute the composite morphism $\morc$ described by the string diagram in this case.

        \item In this part, let~$\tup{\CatC,\mtimescat,\idmoncat}$ be the monoidal category where~\CatC is the category~\Rel of sets and relations,~$\mtimescat$ is the cartesian product of sets and relations, and~$\idmoncat$ is a chosen~$1$-element set that we again denote by~$\singleton$.
              In the string diagram, let~$\Obja = \wnumbers$,~$\Objb = \wnumbers$, and~$\Objc = \wnumbers$.
              Furthermore, let~$\mapa\colon \singleton \mto \wnumbers$ be the relation
              \begin{equation}
                  \mapa = \makeset{ \tup{\singletonel, \elb} \setin \singleton \cartprod \wnumbers \mid \elb \text{ is an even number}}
              \end{equation}
              and let~$\mapb \colon \wnumbers \cartprod \wnumbers \mto \wnumbers$ be the relation
              \begin{equation}
                  \mapb = \makeset{ \tup{\tup{\elb, \ela}, \elc} \setin (\wnumbers \cartprod \wnumbers) \cartprod \wnumbers \mid \elb = \ela = \elc}.
              \end{equation}

              Compute the composite morphism $\morc$ described by the string diagram in this case.

        \item In this part, let~$\tup{\CatC,\mtimescat,\idmoncat}$ be the monoidal category where~\CatC is the category of real vector spaces and real linear maps,~$\mtimescat$ is the direct sum, and~$\idmoncat$ is the~$0$-dimensional real vector space~$\makeset{0}$.
              In the string diagram, let~$\Obja = \Objb = \Objc = \reals^3$.
              Furthermore, let
              \begin{equation}
                  \mapa\colon \makeset{0} \mto \reals^3
              \end{equation}
              be the linear function with~$\mapa(0) = 0$, and let
              \begin{equation}
                  \defmapperiod{\mapb}{\reals^3\oplus \reals^3}{\mto}{\reals^3}{\tup{\elb,\ela}}{\elb+\ela}
              \end{equation}
              Compute the composite morphism $\morc$ described by the string diagram in this case.
    \end{enumerate}
\end{gradedexercise}
\solutionof{StringDiagrams}
