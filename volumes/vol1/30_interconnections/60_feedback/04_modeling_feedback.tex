% !TEX root = chapter-standalone.tex

\section{Modeling feedback}
\label{sec:modeling-feedback}

\todotext{write intro to this section}

\devel{
    \subsection{Feedback of Moore machines}

    \todotext{give an example here to illustrate}

    \subsection{Feedback of LTI systems}

    \todotext{give an example here to illustrate}
}

\subsection{Ships and toilets}

Consider the case in which you are designing the toilets of a cruise ship.
You know that you need a toilet every 10 passengers (\ie, if you have 11 passengers, you need 2 toilets).
Furthermore, you know that each toilet needs an employee for its service, \ie, an extra passenger.
Now, the problem of maximizing the number of people you can put on the ship, by minimizing the number of toilets you need to install, is a design problem.
The resource \SY{poset} is the one describing the number of toilets needed $\R{n_\mathsf{toilets}}$, and the functionality \SY{poset} is the one describing the number of people you can accommodate on the ship $\F{n_\mathsf{passengers}}$.
This can also be written diagrammatically as
\equationsag{06_ship_design}{eq:06_ship_design}
\todojira{193}{\alphubel: @Andrea: Finish the example and correct it.
}

\todotext{give an example here to illustrate}

\subsection{Feedback and trace}

In the following we will introduce a category-theoretic formalization of feedback which is called a \emph{trace}.
The origin of the name is that, in a special case, this notion of feedback corresponds to taking the trace of a matrix.

\todotext{explain better why we call feedback operations by the name traces, detail the situation in terms of linear algebra}

