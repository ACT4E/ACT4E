% !TEX root = chapter-standalone.tex


\section{An example}
\devel{
\JL{An example of a Galois connection between functionalities and resources could be treated here to motivate the adjunction discussion}}


\section{Adjunctions: hom-set definition}
In this section we give a definition of adjunction which can be viewed as an analogy with the following situation in linear algebra. Suppose~$V$ and~$W$ are finite-dimensional real vector spaces, equipped with inner products~$(-, -)_V$ and~$(-, -)_W$, respectively. The adjoint of a linear map~$F\colon V \to W$ is a linear map~$F^*\colon W \sto V$ such that
\begin{equation*}
  (Fv, w)_W = (v, F^*w)_V, \quad \forall v \in V, w \in W.
\end{equation*}


\begin{ctdefinition}[Adjunction, Version 1]
  \label{def:adj-iso}
  \label{def:cat-adjunction-v1}
  Let \CatC and \CatD be categories. An \emph{\iindex{adjunction}} from \CatC to \CatD is given by the following data:
  \begin{compactenum}
    \item A functor~$\funl\colon \CatC \fto \CatD$ (the \emph{left adjoint});
    \item A functor~$\funr\colon \CatD \fto \CatC$ (the \emph{right adjoint});
    \item A natural isomorphism~$\adjtau: \HomSet\CatD{\funl -}{-} \nto \HomSet\CatC{-}{\funr}$
  \end{compactenum}
  We use the notation~$\funl \adjunction  \funr$ to indicate that~$\funl$ and~$\funr$ form an adjunction, with $\funl$ the left adjoint and $\funr$ the right adjoint.
\end{ctdefinition}

\begin{remark}
  Note that~$\adjtau$ is a natural isomorphism between functors of the form
  \begin{equation}
    \CatC\op \Ctimes \CatD \ftolong   \Set
  \end{equation}
\end{remark}




