% !TEX root = chapter-standalone.tex
\section{Relating the two definitions}
\label{relate-adj-defs}

Let's start first with the ``hom-set definition'' of adjunction, and show how to obtain the ``(co)unit definition''.
Given an adjunction~$F \dashv G$ from a category \CatC to a category \CatD, we have, by \cref{def:adj-iso} a natural isomorphism~$\tau$ with components
\begin{equation*}
  \tau_{X,Y} \colon \HomSet{\CatD}{F(X)}{Y} \to \HomSet{\CatC}{X}{G(Y)}.
\end{equation*}
From this data we can construct the unit and counit of the adjunction as follows.

Given an object~$A$ of~\CatC, we define
\begin{equation*}
  \eta_C \colon A \to G(F(A))
\end{equation*}
to be the image under~$\tau_{A, F(A)}$ of~$\catid_{F(A)} \in \HomSet{\CatD}{F (A)}{F(A)}$.

And given an object~$B$ of~\CatD, we define
\begin{equation*}
  \epsilon_B \colon F(G(B)) \to B
\end{equation*}
to the the image under~$\tau_{G(B), B}^{-1}$ of~$\catid_{G(B)} \in \HomSet\CatD{G(B)}{G(B)}$.

\begin{exercise}\label{ex:eta-epsilon}
  Show that if we define~$\eta$ and~$\epsilon$ in terms of their components as above, then they do indeed define natural transformations
  \begin{equation*}
    \eta\colon \catid_{\CatC} \nto G\then F
  \end{equation*}
  and
  \begin{equation*}
    \epsilon\colon G \then F \nto \catid_{\CatD}
  \end{equation*}
  respectively. In other words, check the naturality conditions for~$\eta$ and~$\epsilon$.
\end{exercise}
\begin{solution}
  \todotext{Solution of \cref{ex:eta-epsilon}.}
\end{solution}

\begin{exercise}\label{ex:eta-epsilon-triangle}
  Show that~$\eta$ and~$\epsilon$, as defined above, satisfy the triangle identites stated in \cref{def:adj-counit}.
\end{exercise}
\begin{solution}
  \todotext{Solution of \cref{ex:eta-epsilon-triangle}.}
\end{solution}

Now let's start with the ``(co)unit definition'' of adjunction and see how to obtain the ``hom-set definition''.

Given the unit $\eta$ and counit $\epsilon$, we can construct the components~$\tau_{X,Y}$ of the natural transformation~$\tau$ as follows. Given~$f \in \Hom_{\CatD}(F(X),Y)$, we define
\begin{equation*}
  \tau_{X,Y}(f) = \eta_X \then G(f).
\end{equation*}
Similarly, given~$g \in \Hom_{\CatC}(X,G(Y))$, the inverse component is given by
\begin{equation*}
  \tau_{X,Y}^{-1}(g) = F(g) \then \epsilon_Y.
\end{equation*}

\begin{exercise}\label{ex:tau}
  Show that~$\tau_{X,Y}$ and~$\tau_{X,Y}^{-1}$ are indeed functions which are inverses of each other.
\end{exercise}
\begin{solution}
  \todotext{Solution of \cref{ex:tau}.}
\end{solution}

\begin{exercise}\label{ex:tau2}
  Show that the functions~$\tau_{X,Y}$ do assemble to a natural transformation
  \begin{equation*}
    \tau  \colon \Hom_{\CatD}(F(-) ,- ) \to \Hom_{\CatC}(-  , G(- ) )
  \end{equation*}
  between functors~$\CatC\op \times \CatD \to \Set $.
\end{exercise}
\begin{solution}
  \todotext{Solution of \cref{ex:tau2}.}
\end{solution}
