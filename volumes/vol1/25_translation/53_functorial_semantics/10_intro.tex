% !TEX root = chapter-standalone.tex

\section{Specification verses behavior}

\subsection{Systems}

In the following discussion we will often use the word ``system''.
This term here is not a precise one; however, it is often a way of saying something like ``engineering-flavored morphism'' or ``system component" or ``dynamical system'', but without necessarily getting precise about which kind of things we are exactly referring to or which (semi)categories might be involved.

We will think of a system as something that has input and output ports with which it can interact with other systems (or its broader environment), and that a system in some way establishes a relationship between input and output signals.
This relationship might be a deterministic, causal relationship -- inputs determining outputs -- or it might be another form of lawfulness.

A typical notation to depict a system diagrammatically is to draw it as a box, with externally extended wires indicating the input and output ports.
We will usually orient such diagrams horizontally, and assume that the left-hand wires indicate input ports, and right-hand wires indicate output ports.

\todotext{insert diagam of a single system here}

\subsection{Composing systems}

We assume that our concept of system is compositional: systems can be connected together to build larger, composite systems.

An illustrative example of systems are \SY{Moore machines}, which we discussed in some length in \cref{chap:processes}.
Given a Moore machine
\begin{equation}
    \label{eq:moore-machine-specification}
    \tup{\prinL,\prstL,\proutL,\prdyn,\prreadout,\prstart}
\end{equation}
with an input set $\prinL$ and an output $\proutL$.
When the output set of one machine matches the input set of another, we have already seen how to compose \SY{Moore machines} in series such that the result is again a \SY{Moore machine}.
%We will soon also formalize parallel composition and feedback for \SY{Moore machines}.

\todotext{insert illustrative figure of composing/composed \SY{Moore machines} here}

\subsection{System specification vs. system behavior}

We will make a distinction between ways of specifying a system, and ways of describing how a system might behave.

To see what we mean, consider the example of \SY{Moore machines}.
A way to specify a \SY{Moore machine}, according to our \cref{def:moore-machine}, is to specify a tuple of the form \cref{eq:moore-machine-specification}.

On the other hand, we saw that \SY{Moore machines} can act on sequences or lists of signals in various ways.
We think of these actions as encoding ``behaviors'' that a \SY{Moore machine} can exhibit.
The idea is that an action encodes a way that a \SY{Moore machine} ``does something'' -- how it relates input signals to output signals.

Strictly speaking, we will think of just the specific relation between inputs and outputs as a behavior, and an action is a way of associating specified machines to specified behaviors.

\todotext{insert diagram illustrating the situation}

For example, given a machine
\begin{equation}
    \mora \colon \prinL \to \proutL
\end{equation}
specified by $\cref{eq:moore-machine-specification}$, the standard action \cref{def:moore-standard-action-on-sequences} associates to it a behavior
\begin{equation}
    \act(\mora) \colon \streamsof{\prinL} \to \streamsof{\proutL}.
\end{equation}
We can study specifications of machines and possible behaviors of machines each in their own right, and we can study ways that specifications and behaviors can be connected.

In \cref{sec:different-actions-of-moore-machines} we saw that different kinds of behaviors are possible, via different actions.
For example, not only behaviors of the type
\begin{equation}
    \streamsof{\prinL}  \to \streamsof{\proutL}
\end{equation}
but also of the type
\begin{equation}
    \listsof{\prinL} \to \listsof{\proutL}
\end{equation}
are possible.

In \cref{sec:LTI-systems} below on LTI systems, we will see an example of how a single system might be specified in different ways.

