% !TEX root = chapter-standalone.tex

\section{Subcategories of endomorphisms}
\label{sec:Draw}

\todographics{@Gioele: put here a figure from the exercises to illustrate the concept of objects and morphisms in Draw}
\begin{definition}[Drawings]
    \label{def:Draw}
    \SYNDEF{category of drawings}
    There exists a category~\Draw in which:
    \begin{enumerate}
        \item An object in~$\alpha\setin \Obof\Draw$ is a black-and-white drawing,
              that is a function~$\alpha \colon \reals^2 \sto \boolset$.
        \item A morphism in~$\HomSet\Draw{\alpha}{\beta}$ between two drawings~$\alpha$ and~$\beta$ is an invertible map~$f\colon \reals^2 \sto \reals^2$ such that~$\alpha(x) = \beta(f(x))$.
        \item The identity function at any object~$\alpha$ is the identity map on~$\reals^2$.
        \item Composition is given by function composition.
    \end{enumerate}
\end{definition}

\begin{exercise}
    \label{ex:draw}
    Check whether just considering
    \begin{itemize}
        \item affine invertible transformations, or
        \item rototranslations, or
        \item scalings, or
        \item translations, or
        \item rotations,
    \end{itemize}
    as morphisms forms a \SY{subcategory} of~\Draw.
\end{exercise}
\begin{solution}
    \begin{marginfigure}
        \begin{center}
            \includesag{scaling}
        \end{center}
        \caption{Example of scaling.}
    \end{marginfigure}

    \begin{marginfigure}
        \begin{center}
            \includesag{translation}
        \end{center}
        \caption{Example of translation.}
    \end{marginfigure}

    \begin{marginfigure}
        \begin{center}
            \includesag{rotation}
        \end{center}
        \caption{Example of rotation.}
    \end{marginfigure}

    \begin{marginfigure}
        \begin{center}
            \includesag{rototranslation}
        \end{center}
        \caption{Example of rototranslation.}
    \end{marginfigure}

    \begin{marginfigure}
        \begin{center}
            \includesag{affinetrafo}
        \end{center}
        \caption{Example of affine transformation with~$A=1.5\begin{bmatrix}
                    \cos(\pi/4) & \sin(\pi/4) \\-\sin(\pi/4)& \cos(\pi/4)
                \end{bmatrix}$.}
    \end{marginfigure}
    We check the specializations one by one.
    In all specializations, we consider the same objects as in~\Draw.
    \begin{itemize}
        \item \textbf{Scalings.}
              Let~$s,t\setin \reals$.
              Scalings can be represented as functions of the form
              \begin{equation}
                  \defmapperiod{
                      \scaling_{s,t}
                  }{
                      \reals^2
                  }{
                      \to
                  }{
                      \reals^2
                  }{
                      \tup{x,y}
                  }{
                      \tup{s x,t y}
                  }
              \end{equation}
              By just considering morphisms which are scalings, we are considering a subset of all morphisms.
              Furthermore, the composition of two scalings is again a scaling.
              Indeed, consider scalings~$\scaling_{s,t}$,~$\scaling_{u,v}$.
              We have
              \begin{equation}
                  \begin{aligned}
                      (\scaling_{s,t}\mthen \scaling_{u,v})(x,y)
                       & =\scaling_{u,v}(sx, ty) \\
                       & =\tup{usx, vty} \\
                       & =\scaling_{us,vt}.
                  \end{aligned}
              \end{equation}
              Finally, the \SY{identity morphism} in \Draw corresponds to a scaling of the form $\scaling_{1,1}$.
        \item \textbf{Translations.}
              Let~$s,t\setin \reals$.
              Translations are functions of the form
              \begin{equation}
                  \defmapperiod{
                      \translation_{s,t}
                  }{
                      \reals^2
                  }{
                      \to
                  }{
                      \reals^2
                  }{
                      \tup{x,y}
                  }{
                      \tup{x+s,y+t}
                  }
              \end{equation}
              By just considering morphisms which are translations, we are considering a subset of all morphisms.
              Furthermore, the composition of two translations is again a translation.
              Indeed, consider scalings~$\translation_{s,t}$,~$\translation_{u,v}$.
              We have
              \begin{equation}
                  \begin{aligned}
                      (\translation_{s,t}\mthen \translation_{u,v})(x,y)
                       & =\translation_{u,v}(x+s, y+t) \\
                       & =\tup{x+s+u, y+t+v} \\
                       & =\translation_{s+u,t+v}.
                  \end{aligned}
              \end{equation}
              Finally, the \SY{identity morphism} in \Draw corresponds to a translation of the form $\translation_{0,0}$.
        \item \textbf{Rotations.}
              Let~$\theta \setin [0,2\pi)$.
              Rotations are functions of the form
              \begin{equation}
                  \defmapperiod{
                      \rotation_{\theta}
                  }{
                      \reals^2
                  }{
                      \to
                  }{
                      \reals^2
                  }{
                      \tup{x,y}
                  }{
                      \tup{x\cos(\theta)+y\sin(\theta), y\cos(\theta)-x\sin(\theta)}
                  }
              \end{equation}
              By just considering morphisms which are rotations, we are considering a subset of all morphisms.
              Furthermore, the composition of two rotations is again a rotation.
              Indeed, consider rotations~$\rotation_{\theta}$,~$\rotation_{\phi}$.
              We have
              \begin{equation}
                  \begin{aligned}
                      (\rotation_{\theta}\mthen \rotation_{\phi})(x,y)
                       & =\rotation(\tup{x\cos(\theta)+y\sin(\theta), y\cos(\theta)-x\sin(\theta)}) \\
                       & =\tup{x\cos(\theta+\phi)+y\sin(\theta+\phi), y\cos(\theta+\phi)-x\sin(\theta+\phi)} \\
                       & =\rotation_{\theta+\phi}.
                  \end{aligned}
              \end{equation}
              Finally, the \SY{identity morphism} in \Draw corresponds to a rotation of the form $\rotation_{0}$.
        \item \textbf{Rototranslations.}
              Let~$s,t\setin \reals$ and~$\theta \setin [0,2\pi)$.
              Rototranslations are functions arising from the combination of rotations and translations, and are therefore of the form:
              \begin{equation}
                  \defmapperiod{
                      \rototrans_{\theta,s,t}
                  }{
                      \reals^2
                  }{
                      \to
                  }{
                      \reals^2
                  }{
                      \tup{x,y}
                  }{
                      \tup{x\cos(\theta)+y\sin(\theta)+s, y\cos(\theta)-x\sin(\theta)+t}
                  }
              \end{equation}
              By just considering morphisms which are rotations, we are considering a subset of all morphisms.
              Furthermore, the composition of two rototranslations is again a rototranslation.
              Consider rototranslations~$\rototrans_{\theta,s,t}$,~$\rototrans_{\phi,u,v}$.
              We have:
              \begin{equation}
                  \begin{aligned}
                       & (\rototrans_{\theta,s,t}\mthen \rototrans_{\phi,u,v})(x,y) \\
                       & =\rototrans_{\phi,u,v}(x\cos(\theta)+y\sin(\theta)+s, y\cos(\theta)-x\sin(\theta)+t) \\
                       & =\langle(x\cos(\theta)+y\sin(\theta)+s)\cos(\phi)+(y\cos(\theta)-x\sin(\theta)+t))\sin(\phi)+u, \\
                       & (y\cos(\theta)-x\sin(\theta)+t)\cos(\phi) - (x\cos(\theta)+y\sin(\theta)+s)\sin(\phi)+v\rangle \\
                       & =\langle x\cos(\theta+\phi)+y\sin(\theta+\phi)+s\cos(\phi)+t\sin(\phi)+u, \\
                       & y\cos(\theta+\phi)-x\sin(\theta+\phi)+t\cos(\phi)-s\sin(\phi)+v\rangle \\
                       & =\rototrans_{\theta+\phi,s\cos(\phi)+t\sin(\phi)+u, t\cos(\phi)-s\sin(\phi)+v}(x,y).
                  \end{aligned}
              \end{equation}
              Finally, the \SY{identity morphism} in \Draw corresponds to a rotation of the form $\rototrans_{0,0,0}$.
        \item \textbf{Affine transformations.}
              Let~$\mat{A}\setin \reals^{\matdim{2}{2}}$ and~$\mat{b}\setin \reals^{\matdim{2}{1}}$.
              Affine transformations are functions that could arise from the combination of rotations and translations, and scalings, and are therefore of the form:
              \begin{equation}
                  \defmapcomma{
                  \affinetrafo_{\mat{A},\mat{b}}
                  }{
                  \reals^2
                  }{
                  \to
                  }{
                  \reals^2
                  }{
                  \tup{x,y}
                  }{
                  \tup{a_{11}x+a_{12}y+b_{11},a_{21}x+a_{22}y+b_{21}}
                  }
              \end{equation}
              where~$*_{ij}$ represents the element at the~$i$-th row and~$j$-th column of~$*$.

              Some special cases are:
              \begin{itemize}
                  \item With~$\mat{A}=\mathbb{1}$ we obtain translations~$\translation_{b_{11},b_{21}}$;
                  \item With~$\mat{b}=\begin{bmatrix}
                                0 & 0
                            \end{bmatrix}\mattransp$ and~$A=\begin{bmatrix}
                                s & 0 \\0& t
                            \end{bmatrix}$ we obtain scalings~$\scaling_{s,t}$;
                  \item With~$\mat{b}=\begin{bmatrix}
                                0 & 0
                            \end{bmatrix}\mattransp$ and~$A=\begin{bmatrix}
                                \cos(\theta) & \sin(\theta) \\-\sin(\theta)& \cos(\theta)
                            \end{bmatrix}$ we obtain rotations~$\rotation_{\theta}$.
              \end{itemize}
              By just considering morphisms which are affine transformations, we are considering a subset of all morphisms.
              Furthermore, the composition of two affine transformations is again an affine transformation.
              Clearly, the composition of affine transformations~$\affinetrafo_{\mat{A},\mat{b}}$,~$\affinetrafo_{\mat{C},\mat{d}}$ is~$\affinetrafo_{\mat{CA},\mat{Cb}+\mat{d}}$
              Finally, the \SY{identity morphism} in \Draw corresponds to an affine transformation of the form~$\affinetrafo_{\mathbb{1},\mathbf{0}^{\matdim{2}{1}}}$.
    \end{itemize}
\end{solution}

