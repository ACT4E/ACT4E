% !TEX root = chapter-standalone.tex

\section[Other examples]{Other examples of subcategories in engineering}

In engineering it is very common to look at specific types of functions; in many cases, the properties of a certain type of function are preserved by function composition, and so they form a category.

\subsubsection{\Injset forms a subcategory of \Set}
\begin{definition}[\iindex{Injective function}]
	\label{def:injective-function}
	Let~$\mapa \colon \setA\to \setB$ be a function.
	The function~$\mapa$ is \emph{injective} if, for all~$\ela,\elb\in \setA$ holds:
	\begin{equation*}
		\prfperiod{\mapa(\ela)=\mapa(\elb)}{\ela=\elb}
	\end{equation*}
\end{definition}

\begin{example}
	\label{ex:Injset}
	We can define a category \iindex{\Injset} which has the same objects as \Set but restricts the morphisms to be \emph{injective functions}.
	We want to show that \Injset is a subcategory of \Set.
	Composition and identity morphisms are defined as in \Set.

	Since~$\Ob_{\Injset}=\Ob_{\Set}$, the first condition of \cref{def:subcategory} is satisfied.
	Injective functions are a particular type of functions: this satisfies the second condition.
	Given~$\Obja\in \Ob_{\Injset}$, the identity morphism~$\catid_\Obja\in \HomSet{\Set}{\Obja}{\Obja}$ corresponds to the identity morphism in~$\HomSet{\Injset}{\Obja}{\Obja}$: the identity function is injective.
	This proves the third condition.
	To check the fourth condition, consider two morphisms~$\mora \in \HomSet{\Set}{\Obja}{\Objb}$,~$\morb \in \HomSet{\Set}{\Objb}{\Objc}$ such that~$\mora \in \HomSet{\Injset}{\Obja}{\Objb}$ and~$\morb\in \HomSet{\Injset}{\Objb}{\Objc}$.
	From the injectivity of~$\mora,\morb$, we know that given~$\ela,\ela\styleelements{'}\in \Obja$,
	\begin{equation*}
		\prfdoublecomma{
			\mora(\ela)=\mora(\ela\styleelements{'})
		}{
			\ela=\ela\styleelements{'}
		}
	\end{equation*}
	and~$\elb,\elb\styleelements{'}\in \Obja$,
	\begin{equation*}
		\prfdoubleperiod{
			\morb(\elb)=\morb(\elb\styleelements{'})
		}{
			\elb=\elb\styleelements{'}
		}
	\end{equation*}
	Furthermore, we have:
	\begin{equation*}
		\prfcomma{(\mora \mthen \morb)(\ela)
			=(\mora \mthen \morb)(\ela\styleelements{'})}{\prftree{\mora(\ela)=\mora(\ela\styleelements{'})}{\ela=\ela\styleelements{'}}}
	\end{equation*}
	which proves the fourth condition of \cref{def:subcategory}: the composition of injective functions is injective.
\end{example}

\devel{
	\todojira{462}{Put the following in monoids}
	\begin{definition}[Continuous functions]
		Let~$\mapa \colon \reals\to \reals$ be a function.
		We call~$\mapa$ \emph{continuous} at~$c\in \reals$ if
		\begin{equation}
			\lim_{x\to c}\mapa(x)=f(c);
		\end{equation}
		$\mapa$ is continuous over~\reals if the condition is satisfied for all~$c\in \reals$.

	\end{definition}

	\begin{example}
		We can define a category~$\Cat{Cont}$ in which~
		\begin{equation}
			\Ob_\Cat{Cont}=\{\reals, \reals^2, \reals^3, \dots \}
		\end{equation}
		and in which the morphisms are given by continuous functions.
		Composition and identity are as in~\Set.
		We want to show that $\Cat{Cont}$ is a subcategory of~\Set.
		\todojira{126}{write down formally and use that composition of continuous is continuous}
	\end{example}

	\todojira{127}{Differentiable functions: Set to Manifolds}

	\begin{definition}[Differentiable functions]
		A function $f\colon U\subset \reals\to \reals$, defined on an open set $U$, is \emph{differentiable} at $a\in U$ if the derivative
		\begin{equation}
			f'(a)=\lim_{h\to 0} \frac{f(a+h)-f(a)}{h}
		\end{equation}
		exists; $f$ is differentiable on $U$ if it is differentiable at every point of $U$.
	\end{definition}

	\begin{example}
		the composition of differentiable functions is differentiable
	\end{example}

	\todojira{128}{Lipschitz bounded}

	\begin{definition}[Lipschitz continuous function]
		A real valued function $f\colon \reals\to \reals$ is called \emph{Lipschitz} continuous if there exists a positive real constant $\kappa$ such that, for all $x_1,x_2\in \reals$:
		\begin{equation}
			\vert f(x_1)-f(x_2)\vert \leq \kappa \vert x_1-x_2\vert.
		\end{equation}
	\end{definition}

	\begin{example}
		the composition of differentiable functions is differentiable
	\end{example}

	\todojira{129}{smooth; cont diff, composition is compdiff}
	%\begin{exercise}
	%Check that \Set, as specified above, does in fact define a category.
	%\end{exercise}

	\subsubsection{Generalization outside of $\reals$}
	Generalization to more general spaces.
	We used the fact thath R is:
	\begin{itemize}
		\item For defining continuous functions, we used the fact that~$\reals$ is topological space (minimum needed for defining a continuous function).
		      In fact, the real definition of continuous function is:

		      \todojira{130}{ADD: continuous function}

		\item To define Lipschitz we needed the fact that~ $\reals$ is a metric space
		\item For differentiable, smooth, you define this on Manifolds.
		      Exists tangent space.
	\end{itemize}

	Hence in general, the objects of these are different, so it's not really a relation of subcategory, that requires the objects to be the same.
	However, one can generalize this notion using functors.

	\todojira{131}{functor F:C→D that is both injective on objects and a faithful functor.}
}

\todojira{461}{Add set and pos example for subcats.}