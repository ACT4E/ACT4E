% !TEX root = chapter-standalone.tex

\section[Other examples]{Other examples of \SY{subcategories} in engineering}

In engineering, it is very common to look at specific types of functions; in many cases, the properties of a certain type of function are preserved by function composition, and so they form a category.

\subsubsection{\Injset is a \SY{subcategory} of \Set}
% \todostructure{\alphubel: We have alrady defined informally \SY{injective} function in the preliminaries.
%     Move this there?}
% \begin{definition}[Injective function]
%     \label{def:injective-function}
%     Let~$\mapa \colon \setA \sto \setB$ be a function.
%     The function~$\mapa$ is \emph{injective} if, for all~$\ela,\elb\setin \setA$ holds:
%     \begin{equation}
%         \prfperiod{
%             \mapa(\ela)=\mapa(\elb)
%         }{
%             \ela=\elb
%         }
%     \end{equation}
% \end{definition}

\begin{example}
    \SYNDEF{category of sets and injective functions}
    \label{ex:Injset}
    We can define a category \Injset that has the same objects as \Set but restricts the morphisms to be \emph{\SY{injective functions}} (\cref{def:injective-function}).
    We want to show that \Injset is a \SY{subcategory} of \Set.
    Composition and \SY{identity morphisms} are defined as in \Set.

    Since~$\Obof{\Injset}=\Obof{\Set}$, the first condition of \cref{def:subcategory} is satisfied.
    Injective functions are a particular type of functions: this satisfies the second condition.
    Given~$\Obja\setin \Obof{\Injset}$, the \SY{identity morphism}~$\catidat\Obja\setin \HomSet{\Set}{\Obja}{\Obja}$ corresponds to the \SY{identity morphism} in~$\HomSet{\Injset}{\Obja}{\Obja}$: the identity function is \SY{injective}.
    This proves the third condition.
    To check the fourth condition, consider two morphisms~$\mora \setin \HomSet{\Set}{\Obja}{\Objb}$,~$\morb \setin \HomSet{\Set}{\Objb}{\Objc}$ such that~$\mora \setin \HomSet{\Injset}{\Obja}{\Objb}$ and~$\morb\setin \HomSet{\Injset}{\Objb}{\Objc}$.
    From the \SY{injectivity} of~$\mora,\morb$, we know that given~$\ela,\ela\elprime\setin \Obja$,
    \begin{equation}
        \prfdoublecomma{
            \mora(\ela)=\mora(\ela\elprime)
        }{
            \ela=\ela\elprime
        }
    \end{equation}
    and~$\elb,\elb\elprime\setin \Obja$,
    \begin{equation}
        \prfdoubleperiod{
            \morb(\elb)=\morb(\elb\elprime)
        }{
            \elb=\elb\elprime
        }
    \end{equation}
    Furthermore, we have:
    \begin{equation}
        \prfcomma{
            (\morab)(\ela)
            =(\morab)(\ela\elprime)
        }{
            \prftree{
                \mora(\ela)=\mora(\ela\elprime)
            }{
                \ela=\ela\elprime
            }
        }
    \end{equation}
    which proves the fourth condition of \cref{def:subcategory}: the composition of \SY{injective} functions is \SY{injective}.
\end{example}

\devel{
    \todojira{462}{Put the following in monoids}
    \begin{definition}[Continuous functions]\label{def:continuous-function}

        Let~$\mapa \colon \reals\sto \reals$ be a function.
        We call~$\mapa$ \emph{continuous} at~$c\setin \reals$ if
        %
        \begin{equation}
            \lim_{x\to c}\mapa(x)=f(c);
        \end{equation}
        %
        $\mapa$ is continuous over~\reals if the condition is satisfied for all~$c\setin \reals$.

    \end{definition}

    \begin{example}
        We can define a category~\Cont in which
        \begin{equation}
            \Obof\Cont=\makeset{\reals, \reals^2, \reals^3, \dots }
        \end{equation}
        and in which the morphisms are given by continuous functions.
        Composition and identity are as in~\Set.
        We want to show that \Cont is a \SY{subcategory} of~\Set.
        \todojira{126}{write down formally and use that composition of continuous is continuous}
    \end{example}

    \todojira{127}{Differentiable functions: Set to Manifolds}

    \begin{definition}[Differentiable functions]\label{def:differentiable}
        A function $f\colon U\subset \reals\sto \reals$, defined on an open set $U$, is \emph{differentiable} at $a\setin U$ if the derivative
        \begin{equation}
            f'(a)=\lim_{h\to 0} \frac{f(a+h)-f(a)}{h}
        \end{equation}
        exists; $f$ is differentiable on $U$ if it is differentiable at every point of $U$.
    \end{definition}

    \begin{example}
        The composition of differentiable functions is differentiable.
    \end{example}

    \todojira{128}{Lipschitz bounded}

    \begin{definition}[Lipschitz continuous function]\label{def:Lipschitz}
        A real valued function $f\colon \reals\sto \reals$ is called \emph{Lipschitz} continuous if there exists a positive real constant $\kappa$ such that, for all $x_1,x_2\setin \reals$:
        \begin{equation}
            \absvalueof{f(x_1)-f(x_2)} \leq \kappa \absvalueof{ x_1-x_2}.
        \end{equation}
    \end{definition}

    \begin{example}
        The composition of differentiable functions is differentiable
    \end{example}

    \todojira{129}{smooth; cont diff, composition is compdiff}
    %\begin{exercise}
    %Check that \Set, as specified above, does in fact define a category.
    %\end{exercise}

    \subsubsection{Generalization outside \reals}
    Generalization to more general spaces.
    We used the fact that R is:
    \begin{itemize}
        \item For defining continuous functions, we used the fact that~\reals is topological space (minimum needed for defining a continuous function).
              In fact, the real definition of continuous function is:

              \todojira{130}{ADD: continuous function}

        \item To define Lipschitz we needed the fact that~ \reals is a metric space
        \item For differentiable, smooth, you define this on Manifolds.
              Exists tangent space.
    \end{itemize}

    Hence, in general, the objects of these are different, so it's not really a relation of \SY{subcategory}, that requires the objects to be the same.
    However, we can generalize this notion using \SY{functors}.

    \todojira{131}{functor F:C→D that is both \SY{injective} on objects and a \SY{faithful functor}.}
}

\todojira{461}{\alphubel: Add set and pos example for subcats.}
