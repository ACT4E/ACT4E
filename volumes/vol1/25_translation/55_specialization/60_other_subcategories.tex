% !TEX root = chapter-standalone.tex
\devel{
\section{Other examples of subcategories in engineering}

In engineering it is very common to look at specific types of functions; in many cases, the properties of a certain type of function are preserved by function composition, and so they form a category.



    \begin{definition}[Continuous functions]
        Let~$\mapa \colon \reals\to \reals$ be a function.
        We call~$\mapa$ \emph{continuous} at~$c\in \reals$ if~$\lim_{x\to c}\mapa(x)=f(c)$;~$\mapa$ is continuous over~\reals if the condition is satisfied for all~$c\in \reals$.

        \begin{example}
            We can define a category~$\Cat{Cont}$ which~$\Ob_\Cat{Cont}=\{\reals, \reals^2, \reals^3, \dots \}$ and in which the morphisms are given by continuous functions.
            Composition and identity are as in~\Set.
            We want to show that $\Cat{Cont}$ is a subcategory of~\Set.
            \todojira{126}{write down formally and use that composition of continuous is continuous}
        \end{example}
    \end{definition}

    \todojira{127}{Differentiable functions: Set to Manifolds}

    \begin{definition}[Differentiable functions]
        A function $f\colon U\subset \reals\to \reals$, defined on an open set $U$, is \emph{differentiable} at $a\in U$ if the derivative
        \begin{equation}
            f'(a)=\lim_{h\to 0} \frac{f(a+h)-f(a)}{h}
        \end{equation}
        exists; $f$ is differentiable on $U$ if it is differentiable at every point of $U$.
    \end{definition}

    \begin{example}
        the composition of differentiable functions is differentiable
    \end{example}

    \todojira{128}{Lipschitz bounded}

    \begin{definition}[Lipschitz continuous function]
        A real valued function $f\colon \reals\to \reals$ is called \emph{Lipschitz} continuous if there exists a positive real constant $\kappa$ such that, for all $x_1,x_2\in \reals$:
        \begin{equation}
            \vert f(x_1)-f(x_2)\vert \leq \kappa \vert x_1-x_2\vert.
        \end{equation}
    \end{definition}

    \begin{example}
        the composition of differentiable functions is differentiable
    \end{example}

    \todojira{129}{smooth; cont diff, composition is compdiff}
    %\begin{exercise}
    %Check that \Set, as specified above, does in fact define a category.
    %\end{exercise}

    \subsubsection{Generalization outside of $\reals$}
    Generalization to more general spaces.
    We used the fact thath R is:
    \begin{itemize}
        \item For defining continuous functions, we used the fact that~$\reals$ is topological space (minimum needed for defining a continuous function).
              In fact, the real definition of continuous function is:

              \todojira{130}{ADD: continuous function}

        \item To define Lipschitz we needed the fact that~ $\reals$ is a metric space
        \item For differentiable, smooth, you define this on Manifolds.
              Exists tangent space.
    \end{itemize}

    Hence in general, the objects of these are different, so it's not really a relation of subcategory, that requires the objects to be the same.
    However, one can generalize this notion using functors.

    \todojira{131}{functor F:C→D that is both injective on objects and a faithful functor.}
}
