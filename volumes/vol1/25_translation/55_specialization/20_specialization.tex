% !TEX root = chapter-standalone.tex

\section{Subcategories}
\label{sec:subcategories}

\todojira{123}{@Gioele: Modify config file and add video from first edition.}

\todographics{Needs graphics for subcategorie}

\index{subcategory}
\begin{ctdefinition}[Subcategory]
    \label{def:subcategory}
    A \emph{sub(semi)category}~\CatD of a (semi)category~\CatC is a category for which:
    \begin{compactenum}
        \item All the objects in~$\ObD$ are in~$\ObC$;
        \item For any two objects~$\Obja,\Objb\in \ObD$,
        the morphisms of \CatD between of them are a subsets of the morphisms of \CatC:
        \begin{equation}
            \HomSet{\CatD}{\Obja}{\Objb}\subseteq \HomSet{\CatC}{ \Obja}{\Objb};
        \end{equation}
        \item If~$\mora \colon \Obja \mto \Objb$ and~$\morb\colon \Objb\mto \Objc$ in \CatD, then the composite~$\mora \mthen \morb$ in \CatC is in~\CatD and represents the composite in~\CatD
        \item (Categories) If~$\Obja\in \ObD$, then~$\catid_\Obja \in \HomSet{\CatC}{\Obja}{\Obja}$ is in~$\HomSet{\CatD}{\Obja}{\Obja}$ and acts as its identity morphism.
    \end{compactenum}
\end{ctdefinition}

\section{Subcategories of \Rel and \Set}

Two important examples of subcategory are the following.

\begin{example}[Finite Sets]
    \label{ex:FinSet}
    \iindex{\FinSet} is the category of finite sets and all functions between them.
    It is a subcategory of the category \Set of sets and functions.
    While an object~$\Obja \in \Obof\Set$ is a set with arbitrary cardinality,~$\Ob_{\FinSet}$ only includes sets which have finitely many elements.
    Objects of \FinSet are in \Set, but the converse is not true.
    Furthermore, given~$\Obja,\Objb\in \Obof\FinSet$, we take~$\HomSet{\FinSet}{\Obja}{\Objb} =\HomSet{\Set}{\Obja}{\Objb}$.
\end{example}

\begin{example}[\Set and \Rel]
    The category \Set is a subcategory of \Rel.
    To show this, we need to prove the conditions presented in \cref{def:subcategory}.
    \begin{enumerate}
        \item In both \Rel and \Set, the collection of objects is all sets.
        \item Given~$\Obja,\Objb\in \Ob_{\Set}$, we know that~$\HomSet{\Set}{\Obja}{\Objb}\subseteq \HomSet{\Rel}{\Obja}{\Objb}$, since all functions between sets~$\Obja,\Objb$ are a particular subset of all relations between~$\Obja,\Objb$.
        \item For each~$\Obja \in \Ob_{\Set}$, the identity relation~$\catid_\Obja=\{\tup{\ela,\ela'}\in \Obja\cartprod \Obja \mid \ela=\ela'\}$ corresponds to the identity function~$\catid_\Obja \colon \Obja \mto \Obja$ in \Set.
        \item Let~$\relA\subseteq \Obja\cartprod \Objb$ and~$\relB\subseteq \Objb \cartprod \Objc$ be relations which are functions.
              We need to show that their composition in \Rel, expressed as~$\relA\mthen \relB\subseteq \Obja\cartprod \Objc$, is again a function.
              This was proven in \cref{lem:comprelfun}.
    \end{enumerate}
\end{example}





\subsubsection{\Injset forms a subcategory of \Set}
\begin{definition}[\iindex{Injective function}]
    \label{def:injective-function}
    Let~$\mapa \colon \setA\to \setB$ be a function.
    The function~$\mapa$ is \emph{injective} if, for all~$\ela,\elb\in \setA$ holds:
    \begin{equation*}
        \prftree[r]{.}{
            \mapa(\ela)=\mapa(\elb)
        }{
                \ela=\elb
        }
    \end{equation*}
    %~$\mapa(\ela)=\mapa(\ela')\implies \ela=\ela'$.
\end{definition}

\begin{example}[Set and injective functions]
    \label{ex:Injset}
    We can define a category \iindex{\Injset} which has the same objects as~\Set but restricts the morphisms to be \emph{injective functions}.
    We want to show that \Injset is a subcategory of \Set.
    Composition and identity morphisms are defined as in \Set.

    Since~$\Ob_{\Injset}=\Ob_{\Set}$, the first condition of \cref{def:subcategory} is satisfied.
    Injective functions are a particular type of functions: this satisfies the second condition.
    Given~$\Obja\in \Ob_{\Injset}$, the identity morphism~$\catid_\Obja\in \HomSet{\Set}{\Obja}{\Obja}$ corresponds to the identity morphism in~$\HomSet{\Injset}{\Obja}{\Obja}$: the identity function is injective.
    This proves the third condition.
    To check the fourth condition, consider two morphisms~$\mora \in \HomSet{\Set}{\Obja}{\Objb}$,~$\morb \in \HomSet{\Set}{\Objb}{\Objc}$ such that~$\mora \in \HomSet{\Injset}{\Obja}{\Objb}$ and~$\morb\in \HomSet{\Injset}{\Objb}{\Objc}$.
    From the injectivity of~$\mora,\morb$, we know that given~$\ela,\ela'\in \Obja$,
    \begin{equation*}
        \prftree[double line]{\mora(\ela)=\mora(\ela')}{\ela=\ela'}
    \end{equation*}
    and~$\elb,\elb'\in \Obja$,
    \begin{equation*}
        \prftree[double line]{\morb(\elb)=\morb(\elb')}{\elb=\elb'}
    \end{equation*}
    Furthermore, we have:
    \begin{equation*}
        \prftree[r]{,}{
            (\mora \mthen \morb)(\ela) = (\mora \mthen \morb)(\ela')
        }{
            \prftree{
                \mora(\ela)=\mora(\ela')
            }{
                \ela=\ela'
            }
        }
    \end{equation*}
    which proves the fourth condition of \cref{def:subcategory}: the composition of injective functions is injective.
\end{example}
