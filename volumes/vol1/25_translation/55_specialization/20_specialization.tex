% !TEX root = chapter-standalone.tex

\section{Subcategories}
\label{sec:subcategories}

\todographicsjira{429}{\alphubel: @Gioele: Needs graphics for subcategorie}

\begin{ctdefinition}[Subcategory]
    \label{def:subcategory}
    \SYNDEF{subcategory}
    A \emph{sub(semi)category}~\CatD of a (semi)category~\CatC is a category for which:
    \begin{enumerate}
        \item All the objects in~$\ObD$ are in~$\ObC$;
        \item For any two objects~$\Obja,\Objb\setin \ObD$,
              the morphisms of \CatD between them are a subset of the morphisms of \CatC:
              \begin{equation}
                  \HomSet{\CatD}{\Obja}{\Objb}\setsubseteq \HomSet{\CatC}{ \Obja}{\Objb};
              \end{equation}
        \item If~$\mora \colon \Obja \mto \Objb$ and~$\morb\colon \Objb\mto \Objc$ in \CatD, then the composite~$\morab$ in \CatC is in~\CatD and represents the composite in~\CatD.
        \item (Categories) If~$\Obja\setin \ObD$, then the identity $\catidat\Obja$ in $\HomSet{\CatC}{\Obja}{\Obja}$ is also in~$\HomSet{\CatD}{\Obja}{\Obja}$ and acts as its \SY{identity morphism}.
    \end{enumerate}
\end{ctdefinition}

\subsection{Subcategories of \Rel and \Set}

Two important examples of \SY{subcategory} are the following.

\begin{example}[\FinSet]
    \SYNDEF{category of finite sets and functions}
    \label{ex:FinSet}
    \FinSet is the category of finite sets and all functions between them.
    It is a \SY{subcategory} of the category \Set of sets and functions.
    While an object~$\Obja \setin \Obof\Set$ is a set with arbitrary cardinality,~$\Obof{\FinSet}$ only includes sets which have finitely many elements.
    Objects of \FinSet are in \Set, but the converse is not true.
    Furthermore, given~$\Obja,\Objb\setin \Obof\FinSet$, we take~$\HomSet{\FinSet}{\Obja}{\Objb} =\HomSet{\Set}{\Obja}{\Objb}$.
\end{example}

\begin{example}[\Set and \Rel]
    The category \Set is a \SY{subcategory} of \Rel.
    To show this, we need to prove the conditions presented in \cref{def:subcategory}.
    \begin{enumerate}
        \item In both \Rel and \Set, the collection of objects is all sets.
        \item Given~$\Obja,\Objb\setin \Obof{\Set}$, we know that~$\HomSet{\Set}{\Obja}{\Objb}\setsubseteq \HomSet{\Rel}{\Obja}{\Objb}$, since all functions between sets~$\Obja,\Objb$ are a particular subset of all relations between~$\Obja,\Objb$.
        \item Let~$\relA\setsubseteq \Obja\cartprod \Objb$ and~$\relB\setsubseteq \Objb \cartprod \Objc$ be relations which are functions.
              We need to show that their composition in \Rel, expressed as~$\relA\mthen \relB\setsubseteq \Obja\cartprod \Objc$, is again a function.
              This was proven in \cref{lem:comprelfun}.
        \item For each~$\Obja \setin \Obof{\Set}$, the identity relation~$\catidat\Obja=\makeset{\tupp{\ela,\ela\elprime}\setin {\Obja} \cartprod \Obja \mid \ela=\ela\elprime}$ corresponds to the identity function~$\catidat\Obja \colon \Obja \mto \Obja$ in \Set.
    \end{enumerate}
\end{example}
