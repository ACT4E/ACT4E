% !TEX root = chapter-standalone.tex
\label{sec:specialization}


\section{Notion of subcategory}
\todojira{123}{@Gioele: Modify config file and add video from first edition.}
\index{subcategory}
\begin{ctdefinition}[Subcategory]
    \label{def:subcategory}
    A \emph{\iindex{subcategory}}~\CatD of a category~\CatC is a category for which:
    \begin{compactenum}
        \item All the objects in~$\ObD$ are in~$\ObC$;
        \item For any objects~$\Obja,\Objb\in \ObD$,~$\HomSet{\CatD}{\Obja}{\Objb}\subseteq \HomSet{\CatC}{ \Obja}{\Objb}$;
        \item If~$\Obja\in \ObD$, then~$\catid_\Obja \in \HomSet{\CatC}{\Obja}{\Obja}$ is in~$\HomSet{\CatD}{\Obja}{\Obja}$ and acts as its identity morphism;
        \item If~$\mora \colon \Obja \mto \Objb$ and~$\morb\colon \Objb\mto \Objc$ in \CatD, then the composite~$\mora \mthen \morb$ in \CatC is in~\CatD and represents the composite in~\CatD.
    \end{compactenum}
\end{ctdefinition}

Two important examples of subcategory are the following.

\begin{example}[Finite Sets]
    \label{ex:FinSet}
    \iindex{\FinSet} is the category of finite sets and all functions between them.
It is a subcategory of the category \Set of sets and functions.
While an object~$\Obja \in \Ob_\Set$ is a set with arbitrary cardinality,~$\Ob_{\FinSet}$ only includes sets which have finitely many elements.
Objects of \FinSet are in \Set, but the converse is not true.
Furthermore, given~$\Obja,\Objb\in \Ob_\FinSet$, we take~$\HomSet{\FinSet}{\Obja}{\Objb} =\HomSet{\Set}{\Obja}{\Objb}$.
\end{example}

\begin{example}[\Set and \Rel]
    The category \Set is a subcategory of \Rel.
To show this, we need to prove the conditions presented in \cref{def:subcategory}.
    \begin{enumerate}
        \item In both \Rel and \Set, the collection of objects is all sets.
        \item Given~$\Obja,\Objb\in \Ob_{\Set}$, we know that~$\HomSet{\Set}{\Obja}{\Objb}\subseteq \HomSet{\Rel}{\Obja}{\Objb}$, since all functions between sets~$\Obja,\Objb$ are a particular subset of all relations between~$\Obja,\Objb$.
        \item For each~$\Obja \in \Ob_{\Set}$, the identity relation~$\catid_\Obja=\{\tup{\ela,\ela'}\in \Obja\times \Obja \mid \ela=\ela'\}$ corresponds to the identity function~$\catid_\Obja \colon \Obja \mto \Obja$ in \Set.
        \item Let~$\relA\subseteq \Obja\times \Objb$ and~$\relB\subseteq \Objb \times \Objc$ be relations which are functions.
We need to show that their composition in \Rel, expressed as~$\relA\mthen \relB\subseteq \Obja\times \Objc$, is again a function.
This was proven in \cref{lem:comprelfun}.
    \end{enumerate}

\end{example}

