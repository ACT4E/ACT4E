% !TEX root = chapter-standalone.tex

\section{Subcategories of Berg}
\label{sec:subcat_berg}

Recall the category \Berg presented in \cref{sec:trekking}.
In the following, we want to give both a positive and a negative example of subcategories related to \Berg.

%We first start our discussion by introducing an \emph{amateur} version of \Berg, called \Bergama, which only considers paths (morphisms) in \Berg, whose steepness does not exceed a critical value, say 1/2.
%Is \Bergama a subcategory of \Berg?
We start our discussion by introducting a \emph{limited} version of \Berg, called~$\Berg_\alpha$, which only considers paths (morphisms) in \Berg, whose steepness does not exceed the critical value~$\alpha\setin [0,1]$.
Is~$\Berg_\alpha$ a subcategory of \Berg?
Let's check the different conditions:
\begin{enumerate}
    \item The constraint on the maximum steepness restricts the objects which are acceptable in~$\Berg_\alpha$ via the identity morphisms of \Berg.
          Indeed, recall that given an object~$\tup{\styleobj{p},\styleobj{v}}\setin \Ob_{\Berg}$, the identity morphism is defined as~$\stylemorph{1}_{\tup{\styleobj{p},\styleobj{v}}}=\tup{\gamma,0}$, with~$\gamma(0)=p$ and~$\dot{\gamma}(0)=v$.
          The steepness is computed via~$\styleobj{v}$.
          In particular,~$\Berg_\alpha$ will only contain objects whose identity morphisms do not exceed the steepness constraint, In other words~$\Ob_{\Berg_\alpha} \setsubseteq \Ob_\Berg$.
    \item For~$\Obja,\Objb\setin \Ob_{\Bergama}$, we know that paths satisfying the steepness constraint are specific paths in \Berg:~$\Hom_{\Berg_\alpha}\setsubseteq \Hom_{\Berg}$.
    \item The identity morphisms in \Berg which satisfy the steepness constraint are, by definition, in~$\Berg_\alpha$ and they act as identities there.
    \item Given two morphisms~$\mora,\morb$ which can be composed in~$\Berg_\alpha$, the maximum steepness of their composition~$\mora \mthen \morb$ is given by:
          \begin{equation*}
              \MaxSteepness(\mora \mthen \morb)
              =
              \max \ \{
              \MaxSteepness(\mora),
              \MaxSteepness(\morb)
              \}
              <
              \alpha.
          \end{equation*}
\end{enumerate}

This shows that~$\Berg_\alpha$ is a subcategory of \Berg.

What would an example of non-subcategory of \Berg be?
Let's define a new category \Berglazy, which now discriminates morphisms based on the lengths of the paths they represent.
For instance, assume that as amateur hikers, we don't want to consider morphisms which are more than \unit[1]{km} long.
By concatenating two paths (morphisms) of length \unit[0.6]{km} in \Berglazy, the resulting composition will be \unit[1.2]{km}, violating the posed constraint and hence not being in \Berglazy.
This violates the fourth property of \cref{def:subcategory}.

% \book{
% \section{Some counterexamples}

% C: Endomorphisms of the plane

% % f: R^2 -> R^2

% functions modulo squared

% composition of functions is given by

% %   f;g =  f( g(x) ) modulo Square

% %   f; (g;h) =  f(g;h(x)) mod S) mod S
% %           =  f(g(h(x mod S) mod S)) mod S) mod S
% %           =  f(g(h(x mod S)))

% %   (f;g); h =   (f;g)( h(x) mod S) mod S
% %           =    f( g( h(x) mod S) mod S) mod S
% %           =    f( g( h(x) mod S)

% %  Z
% }
