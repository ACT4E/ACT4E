% !TEX root = chapter-standalone.tex

\section{Full and faithful functors}

\begin{ctdefinition}[Full and faithful functors]
    \label{def:functorfullfaith}
    \SYNDEF{full functor}
    \SYNDEF{faithful functor}
    A \SY{functor}~$\funa\colon \CatC \fto \CatD$ is \emph{full} (respectively \emph{faithful}) if for each pair of objects~$\Obja,\Objb\setin \CatC$, the function
    \begin{equation}
        \funa \colon \HomSet{\CatC}{\Obja}{\Objb} \to \HomSet{\CatD}{\funa(\Obja)}{\funa(\Objb)}
    \end{equation}
    is \SY{surjective} (respectively \SY{injective}).
\end{ctdefinition}


\begin{marginfigure}
    \centering
    \includesag{095_full_faithful_1}
    \caption{}
    \label{fig:ex_full_faithful_1}
    \todomistake{\alphubel: This figure is wrong}
\end{marginfigure}
\begin{example}
    Let~\CatC be the category depicted in \cref{fig:ex_full_faithful_1}.
    Let~$\funa\colon \CatC \fto \CatC$ be the \SY{endofunctor} which maps object~$\Obja\setin \Obof{\CatC}$ to~$\Obja$ and object~$\Objb\setin \Obof{\CatC}$ to~$\Obja$.
    This \SY{functor} is \SY{full} and \SY{faithful}.
    Note that the map of objects and the map of morphisms are neither \SY{surjective} nor \SY{injective}.
\end{example}

\begin{example}
    Consider a \SY{functor} which maps a category \CatC to its \SY{pre-order} structure \CatD, that is a category with the same objects as \CatC and a \emph{unique} morphism from~$\Obja\setin \Obof\CatD$ to~$\Objb\setin \Obof\CatD$ if and only if there is at least one morphism from~$\Obja$ to~$\Objb$ in \CatC.
    This \SY{functor} is \SY{full} by construction, and it is \SY{faithful} if and only if \CatC is a \SY{pre-order}.
    \todojira{136}{maybe add pic.
        OK, so what?
        Examples?
        Why is it useful?
    }
\end{example}

