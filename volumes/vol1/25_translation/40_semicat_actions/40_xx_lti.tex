% !TEX root = chapter-standalone.tex

\section{Discrete-time linear systems}
\todotext{I think the material below is outdated}

\begin{definition}[Discrete-time linear systems]
    \label{def:discrete-time-linear-system}
    A \maindef{discrete-time linear time-invariant open system} is defined by three matrices~$\mat{A},\mat{B},\mat{C}$.
    Together they give a recurrence of the type
    \begin{equation}
        \label{eq:discrete-time-dynamics}
        \begin{aligned}
            \mat{x}_{k+1} & = \mat{A} \mat{x}_k + \mat{B} \mat{u}_k, \\
            \mat{y}_{k}   & = \mat{C} \mat{x}_k + \mat{D} \mat{u}_k.
        \end{aligned}
    \end{equation}
    If~$\mat{x}$ has dimension~$n\geq1$,~$\mat{u}$ dimension~$m\geq1$ and~$\mat{y}$ dimension~$p\geq1$, then~$\mat{A}$ has dimension~$\ntimesn$,~$\mat{B}$ has dimension~$\matdim{n}{m}$, and~$\mat{C}$ has dimension~$\matdim{p}{n}$.
\end{definition}

\begin{marginfigure}
    \centering
    \prftree{\includesag{20_dyn_1}}{\includesag{20_dyn_2}}{\includesag{20_dyn_1_2}}
    \caption{Composition of discrete-time linear systems.}
    \label{fig:comp_dyn_syst}
\end{marginfigure}

Consider now the systems where~$m=p$ (but possibly different~$n$):
these are systems with input and output of the same size.
Hence, we can compose them in series.
Series composition is the composition in which the output of a system is the input of another system (\cref{fig:comp_dyn_syst}).
The composition forms a new discrete-time linear system, which has the input of the first system and the output of the second system.
Consider the first system~$\mora$:
%
\begin{equation}
    \label{eq:dlin-1}
    \begin{aligned}
        \mat{x}_{k+1} & = \mat{A}_\mora \mat{x}_k + \mat{B}_\mora \mat{u}_k, \\
        \mat{y}_{k}   & = \mat{C}_\mora \mat{x}_k + \mat{D}_\mora \mat{u}_k,
    \end{aligned}
\end{equation}
%
and the second system (which has~$\mat{y}$ as input)~$\morb$:
%
\begin{equation}
    \label{eq:dlin-2}
    \begin{aligned}
        \mat{z}_{k+1} & = \mat{A}_\morb \mat{z}_k + \mat{B}_\morb \mat{y}_k, \\
        \mat{v}_{k}   & = \mat{C}_\morb \mat{z}_k+ \mat{D}_\morb \mat{y}_k.
    \end{aligned}
\end{equation}

We can write their composition~$\mora\mtimes \morb$ compactly as a discrete linear system with input~$\mat{u}$ and output~$\mat{v}$:
%
\begin{equation}
    \label{eq:dlin-3}
    \begin{aligned}
        \begin{bmatrix}
            \mat{x}_{k+1} \\
            \mat{z}_{k+1}
        \end{bmatrix} & =
        \begin{bmatrix}
            \mat{A}_\mora               & \mat{0}       \\
            \mat{B}_\morb \mat{C}_\mora & \mat{A}_\morb
        \end{bmatrix}
        \begin{bmatrix}
            \mat{x}_k \\ \mat{z}_k
        \end{bmatrix}
        +
        \begin{bmatrix}
            \mat{B}_\mora \\ \mat{B}_\morb \mat{D}_\mora
        \end{bmatrix}\mat{u}_k, \\
        \mat{v}_k        & =
        \begin{bmatrix}
            \mat{D}_\morb \mat{C}_\mora & \mat{C}_\morb
        \end{bmatrix}
        \begin{bmatrix}
            \mat{x}_k \\ \mat{z}_k
        \end{bmatrix}+ \mat{D}_\morb \mat{D}_\mora \mat{u}_k.
    \end{aligned}
\end{equation}

We now want to show that the composition of discrete linear systems is \SY{associative} (graphically reported in \cref{fig:ass_dyn_syst}).
To do so, we need a third system,~$\morc$, with input~$\mat{v}$ and output~$\mat{w}$:
\begin{equation}
    \label{eq:dlin-4}
    \begin{aligned}
        \mat{s}_{k+1} & = \mat{A}_\morc \mat{s}_k + \mat{B}_\morc \mat{v}_k, \\
        \mat{w}_{k}   & = \mat{C}_\morc \mat{s}_k+ \mat{D}_\morc \mat{v}_k.
    \end{aligned}
\end{equation}

Furthermore, we want to show that~$(\dtsysa\mtimes \dtsysb)\mtimes \dtsysc=\dtsysa\mtimes (\dtsysb\mtimes \dtsysc)$.
The check is analogous to the one made for continuous-time LTI systems.

\todojira{649}{\alphubel: @Gioele: J: This equality is maybe not formally true on the nose... but maybe yes... i guess we should try to make a more precise definition?
    If we want to see these as a special case of \SY{Moore machines} (which seems desireable), then we won't have equality on the nose unless we use the trick AC suggested, using \textbf{SetL}.
}

% \begin{comment}
% \begin{equation}
%     \label{eq:dlin-5}
%     \begin{aligned}
%         \begin{bmatrix}
%             \mat{x}_{k+1} \\
%             \mat{z}_{k+1} \\
%             \mat{s}_{k+1}
%         \end{bmatrix} & =
%         \begin{bmatrix}
%             \mat{A}  & \mat{0}  & \mat{0} \\
%             \mat{FC} & \mat{E}  & \mat{0} \\
%             \mat{0}  & \mat{IG} & \mat{H}
%         \end{bmatrix}
%         \begin{bmatrix}
%             \mat{x}_k \\ \mat{z}_k\\ \mat{s}_k
%         \end{bmatrix}+
%         \begin{bmatrix}
%             \mat{B} \\ \mat{0}\\ \mat{0}
%         \end{bmatrix}\mat{u}_k, \\
%         \mat{w}_k                         & =
%         \begin{bmatrix}
%             \mat{0} & \mat{0} & \mat{J}
%         \end{bmatrix}
%         \begin{bmatrix}
%             \mat{x}_k \\ \mat{z}_k\\ \mat{s}_k
%         \end{bmatrix}.
%     \end{aligned}
% \end{equation}
% We now want to write the composition~$\dtsysa\mtimes (\dtsysb\mtimes \dtsysc)$.
% To do so, we start by writing~$(\dtsysb\mtimes \dtsysc)$ as:
% \begin{equation}
%     \label{eq:dlin-6}
%     \begin{aligned}
%         \begin{bmatrix}
%             \mat{z}_{k+1} \\
%             \mat{s}_{k+1}
%         \end{bmatrix} & =
%         \begin{bmatrix}
%             \mat{E}  & \mat{0} \\
%             \mat{IG} & \mat{H}
%         \end{bmatrix}
%         \begin{bmatrix}
%             \mat{z}_k \\ \mat{s}_k
%         \end{bmatrix}+
%         \begin{bmatrix}
%             \mat{F} \\ \mat{0}
%         \end{bmatrix}\mat{y}_k, \\
%         \mat{w}_k        & =
%         \begin{bmatrix}
%             \mat{0} & \mat{J}
%         \end{bmatrix}
%         \begin{bmatrix}
%             \mat{z}_k \\ \mat{s}_k
%         \end{bmatrix}.
%     \end{aligned}
% \end{equation}
% From this we can write~$\dtsysa\mtimes (\dtsysb\mtimes \dtsysc)$ as:
% \begin{equation}
%     \label{eq:dlin-7}
%     \begin{aligned}
%         \begin{bmatrix}
%             \mat{x}_{k+1} \\
%             \mat{z}_{k+1} \\
%             \mat{s}_{k+1}
%         \end{bmatrix} & =
%         \begin{bmatrix}
%             \mat{A}  & \mat{0}  & \mat{0} \\
%             \mat{FC} & \mat{E}  & \mat{0} \\
%             \mat{0}  & \mat{IG} & \mat{H}
%         \end{bmatrix}
%         \begin{bmatrix}
%             \mat{x}_k \\ \mat{z}_k\\ \mat{s}_k
%         \end{bmatrix}+
%         \begin{bmatrix}
%             \mat{B} \\ \mat{0}\\ \mat{0}
%         \end{bmatrix}\mat{u}_k, \\
%         \mat{w}_k                         & =
%         \begin{bmatrix}
%             \mat{0} & \mat{0} & \mat{J}
%         \end{bmatrix}
%         \begin{bmatrix}
%             \mat{x}_k \\ \mat{z}_k\\ \mat{s}_k
%         \end{bmatrix},
%     \end{aligned}
% \end{equation}
% showing associativity.
% \end{comment}

\begin{figure}[tbh]
    \centering
    \prflinepadbefore=5pt
    \prflinepadafter=5pt
    \prfdouble{
        \prftree{\prftree{\includesag{20_dyn_1}}{\includesag{20_dyn_2}}{\includesag{20_dyn_1_2}}}{\prftree{\includesag{20_dyn_3}}{\includesag{20_dyn_3}}}{\includesag{20_dyn_12_3}}
    }
    {\prftree{\prftree{\includesag{20_dyn_1}}{\includesag{20_dyn_1}}}{\prftree{\includesag{20_dyn_2}}{\includesag{20_dyn_3}}{\includesag{20_dyn_2_3}}}{\includesag{20_dyn_1_23}}
    }
    \caption{Associativity law for the composition of discrete-time linear systems. }
    \label{fig:ass_dyn_syst}
\end{figure}

\

\

\begin{exercise}[Discrete-time linear systems]
    In \cref{def:discrete-time-linear-system} we have constructed the \SY{semigroup} of linear discrete-time dynamical systems.
    Does it form a monoid?
    If not, please explain the reason and suggest an extension to \cref{def:discrete-time-linear-system} so that you can obtain a \SY{monoid}.
\end{exercise}
\begin{solution}
    By considering \cref{def:discrete-time-linear-system}, we are looking for a discrete-time linear system which, pre- or post-composed with another one does not change it.
    This is a problem for two reasons.
    First, we need to extend \cref{def:discrete-time-linear-system} to account for cases in which~$n=0$.
    Indeed, when composing two systems the state space dimension grows by at least 1, which is not acceptable for the identity case.
    Second, it is not possible to find an identity.
    One can solve this issue by considering the formalization
    \begin{equation}
        \label{eq:discrete-time-dynamics-D}
        \begin{aligned}
            x_{k+1} & = \mathbf{A} x_k & + \mathbf{B} u_k, \\
            y_{k}   & = \mathbf{C} x_k & + \mathbf{D} u_k.
        \end{aligned}
    \end{equation}
    This introduces a direct input/output link that allows to not have delay (introduced by the state dynamics).
\end{solution}

