
\section{A category of categories}
\linkvideo{spring2021-functors:semi-and-fun:cat-of-cat} % A category of categories

Given the existence of an identity functor and the ability of functors to compose, we can define a category of categories \Category.
In order to avoid set-theoretic technicalities, we restrict our attention to so-called ``small'' categories: these are categories whose collection of objects from a set (and not a proper class).\footnote{See \cref{sec:technical-terms}, where we discuss these terms briefly.}

\begin{ctdefinition}[Category of small categories]
    \label{def:Category}
    There is a category, called \Category, which is constituted of
    \begin{compactitem}
        \item Objects: categories;
        \item Morphisms: functors;
        \item Identity morphisms: identity functors;
        \item Composition: composition of functors.
    \end{compactitem}
\end{ctdefinition}

\begin{gradedexercise}[\exname{CatProductCategorical}]
    Prove that the product category~$\CatC \Ctimes \CatD$ of two small categories ``is'' the categorical product of~$\CatC$ and~$\CatD$ within the category of small categories.
\end{gradedexercise}

\solutionof{CatProductCategorical}
