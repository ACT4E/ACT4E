% !TEX root = chapter-standalone.tex

\section{Functor composition}

\begin{ctdefinition}[Functor composition]
    \label{def:functor_composition}
    Consider categories~$\CatA,\CatB,\CatC$ and functors~$\funa\colon \CatA\fto \CatB$,~$\funb \colon \CatB\fto \CatC$.
    Functor composition is given by~$\funa\fthen \funb\colon \CatA \fto \CatC$, where:
    \begin{compactitem}
        \item Given~$\Obja\in \Ob_\CatA$, one has~$(\funa\fthen \funb)(\Obja)\coloneqq\funb(\funa(\Obja))$;
        \item Given~$\mora \in \HomSet{\CatA}{\Obja}{\Objb}$, one has~$(\funa\fthen \funb)(\mora)\coloneqq \funb(\funa(\mora))$.
    \end{compactitem}
\end{ctdefinition}

\begin{lemma}
    \label{lem:functors_compose}
    The composition of functors is a functor.
\end{lemma}

\begin{exercise}
    Prove \cref{lem:functors_compose}.
\end{exercise}
\begin{solution}
    In the following, we want to show that functors compose.
    Given categories~$\CatA,\CatB,\CatC$ and functors~$\funa\colon \CatA\fto \CatB$,~$\funb \colon \CatB\fto \CatC$, we want to show that~$\funa\fthen\funb$ is a functor.
    To do this, we show that~$\funa\fthen\funb$ preserves identities and compositions.
    \begin{compactitem}
        \item Given an object~$\Obja\in \CatA$, we have:
              \begin{equation*}
                  \begin{aligned}
                      (\funa\fthen\funb)(\catid_\Obja)
                       & =\funb(\funa(\catid_\Obja))    \\
                       & =\funb(\catid_{\funa(\Obja)})  \\
                       & =\catid_{\funb(\funa(\Obja))},
                  \end{aligned}
              \end{equation*}
              where we used that~$\funa$ and~$\funb$ are functors (they preserve identities).
        \item Furthermore, given composable morphisms~$\mora,\morb\in \CatA$, one has:
              \begin{equation*}
                  \begin{aligned}
                      (\funa\fthen\funb)(\mora\mthen\morb)
                       & =\funb(\funa(\mora)\mthen \funa(\morb))         \\
                       & =\funb(\funa(\mora))\mthen \funb(\funa(\morb)),
                  \end{aligned}
              \end{equation*}
              where again we used that~$\funa,\funb$ are functors (they preserve composition).
    \end{compactitem}
\end{solution}

We can define an identity functor.

\begin{ctdefinition}[Identity functor]
    \label{def:identity_functor}
    Given a category~\CatC, we define it as~$\funid_\CatC\colon \CatC \fto \CatC$, where
    \begin{compactitem}
        \item Given~$\Obja\in \Ob_\CatC$, one has~$\funid_\CatC(\Obja)\coloneqq \Obja$;
        \item Given~$\mora \in \HomSet{\CatC}{\Obja}{\Objb}$, one has~$\funid_\CatC(\mora)\coloneqq \mora$.
    \end{compactitem}
\end{ctdefinition}

\begin{lemma}
    \label{lem:identity_functor_is_functor}
    The identity functor is indeed a functor.
\end{lemma}

\begin{exercise}
    Prove \cref{lem:identity_functor_is_functor}
\end{exercise}
\begin{solution}
    To show that this is a valid functor, we need to show that it preserves identities and composition:
    \begin{compactitem}
        \item Given any~$\Obja \in \Ob_\CatC$, we have:
              \begin{equation*}
                  \begin{aligned}
                      \funid_{\CatC}(\catid_\Obja) & =\catid_\Obja                   \\
                                                   & =\catid_{\funid_{\CatC}(\Obja)}
                  \end{aligned}
              \end{equation*}
              Furthermore, given composable morphisms~$\mora,\morb$ in~\CatC, we have:
              \begin{equation*}
                  \begin{aligned}
                      \funid_{\CatC}(\mora\mthen\morb) & =\mora\mthen\morb                                    \\
                                                       & =\funid_{\CatC}(\mora)\mthen \funid_{\CatC} (\morb).
                  \end{aligned}
              \end{equation*}
    \end{compactitem}
\end{solution}
