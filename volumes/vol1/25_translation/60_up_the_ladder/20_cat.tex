% !TEX root = chapter-standalone.tex
\section{A category of categories}
\linkvideo{spring2021-functors:semi-and-fun:cat-of-cat} % A category of categories

Given the existence of an \SY{identity functor} and the ability of \SY{functors} to compose, we can define a category of categories \Category.
In order to avoid set-theoretic technicalities, we restrict our attention to so-called ``small'' categories: these are categories whose collection of objects form a set (and not a proper class).
%\footnote{See \cref{sec:foundations}, where we discuss these terms briefly.}

\begin{ctdefinition}[Category of small categories]
    \label{def:Category}
    \SYNDEF{category of small categories}
    There is a category, called \Category, which is constituted of
    \begin{itemize}
        \item Objects: small categories;
        \item Morphisms: \SY{functors};
        \item \SY{Identity morphisms}: \SY{identity functors};
        \item Composition: \SY{composition of functors}.
    \end{itemize}
\end{ctdefinition}
\vfill
\begin{gradedexercise}[\exname{CatProductCategorical}]
    Prove that the product category~$\CatC \Ctimes \CatD$ of two small categories ``is'' the \SY{categorical product} of~\CatC and~\CatD within the category of small categories.
\end{gradedexercise}
\solutionof{CatProductCategorical}
\todojira{460}{\bernina: Is DP the \SY{join} of rel and pos when using embedding?}
\todojira{463}{\alphubel: @Gioele: Cats picture}
