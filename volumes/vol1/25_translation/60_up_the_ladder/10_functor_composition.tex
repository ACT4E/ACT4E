% !TEX root = chapter-standalone.tex

\section{Functor composition}
\begin{ctdefinition}[Functor composition]
    \label{def:functor_composition}
    Consider categories~$\CatA,\CatB,\CatC$ and functors~$\funa\colon \CatA\fto \CatB$,~$\funb \colon \CatB\fto \CatC$.
    Functor composition is given by~$\funa\fthen \funb\colon \CatA \fto \CatC$, where:
    \begin{itemize}
        \item Given~$\Obja\setin \Obof\CatA$, we have
              \begin{equation}
                  (\funa\fthen \funb)(\Obja)
                  \definedas\funb(\funa(\Obja)).
              \end{equation}
        \item Given~$\mora \setin \HomSet{\CatA}{\Obja}{\Objb}$, we have
              \begin{equation}
                  (\funa\fthen \funb)(\mora)
                  \definedas \funb(\funa(\mora)).
              \end{equation}
    \end{itemize}
    \todotextjira{741}{\alphubel: @Gioele: use specialized funob/funmor notation $\funobspace{F}, \funmorspace{G}$ notation}
\end{ctdefinition}

\begin{lemma}
    \label{lem:functors_compose}
    The composition of functors is a functor.
\end{lemma}

\begin{exercise}
    Prove \cref{lem:functors_compose}.
\end{exercise}
\begin{solution}
    In the following, we want to show that functors compose.
    Given categories~$\CatA,\CatB,\CatC$ and functors~$\funa\colon \CatA\fto \CatB$,~$\funb \colon \CatB\fto \CatC$, we want to show that~$\funa\fthen\funb$ is a functor.
    To do this, we show that~$\funa\fthen\funb$ preserves identities and compositions.
    \begin{itemize}
        \item Given an object~$\Obja\setin \CatA$, we have:
              \begin{equation}
                  \begin{aligned}
                      (\funa\fthen\funb)(\catidat\Obja)
                       & =\funb(\funa(\catidat\Obja)) \\
                       & =\funb(\catidat{\funa(\Obja)}) \\
                       & =\catidat{\funb(\funa(\Obja))},
                  \end{aligned}
              \end{equation}
              where we used that~$\funa$ and~$\funb$ are functors (they preserve identities).
        \item Furthermore, given composable morphisms~$\mora,\morb\setin \CatA$, we have:
              \begin{equation}
                  \begin{aligned}
                      (\funa\fthen\funb)(\morab)
                       & =\funb(\funa(\mora)\mthen \funa(\morb)) \\
                       & =\funb(\funa(\mora))\mthen \funb(\funa(\morb)),
                  \end{aligned}
              \end{equation}
              where again we used that~$\funa,\funb$ are functors (they preserve composition).
    \end{itemize}
\end{solution}

We can define an identity functor.

\begin{ctdefinition}[Identity functor]
    \label{def:identity_functor}
    Given a category~\CatC, we define it as~$\funidC\colon \CatC \fto \CatC$, where
    \begin{itemize}
        \item Given~$\Obja\setin \ObC$, we have~$\funidC(\Obja)\definedas \Obja$;
        \item Given~$\mora \setin \HomSet{\CatC}{\Obja}{\Objb}$, we have~$\funidC(\mora)\definedas \mora$.
    \end{itemize}
\end{ctdefinition}

\begin{lemma}
    \label{lem:identity_functor_is_functor}
    The identity \SY{functor} is indeed a functor.
\end{lemma}

\begin{exercise}
    Prove \cref{lem:identity_functor_is_functor}
\end{exercise}
\begin{solution}
    To show that this is a valid functor, we need to show that it preserves identities and composition:
    \begin{itemize}
        \item Given any~$\Obja \setin \ObC$, we have:
              \begin{equation}
                  \begin{aligned}
                      \funid_{\CatC}(\catidat\Obja) & =\catidat\Obja \\
                                                    & =\catidat{\funid_{\CatC}(\Obja)}
                  \end{aligned}
              \end{equation}
              Furthermore, given composable morphisms~$\mora,\morb$ in~\CatC, we have:
              \begin{equation}
                  \begin{aligned}
                      \funid_{\CatC}(\morab) & =\morab \\
                                             & =\funid_{\CatC}(\mora)\mthen \funid_{\CatC} (\morb).
                  \end{aligned}
              \end{equation}
    \end{itemize}
\end{solution}
