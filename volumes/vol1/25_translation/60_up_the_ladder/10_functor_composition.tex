% !TEX root = chapter-standalone.tex

\section{Functor composition}
\begin{ctdefinition}[Functor composition]
    \label{def:functor-composition}
    \SYNDEF{functor composition}
    Consider categories~$\CatA,\CatB,\CatC$ and \SY{functors}~$\funa\colon \CatA\fto \CatB$,~$\funb \colon \CatB\fto \CatC$.
    Functor composition is given by~$\funa\fthen \funb\colon \CatA \fto \CatC$, where:
    \begin{itemize}
        \item Given~$\Obja\setin \Obof\CatA$, we have
              \begin{equation}
                  (\funaob\fthen \funbob)(\Obja)
                  \definedas\funbob(\funaob(\Obja)).
              \end{equation}
        \item Given~$\mora \setin \HomSet{\CatA}{\Obja}{\Objb}$, we have
              \begin{equation}
                  (\funamor\fthen \funbmor)(\mora)
                  \definedas \funbmor(\funamor(\mora)).
              \end{equation}
    \end{itemize}
\end{ctdefinition}

\todotextjira{741}{\alphubel: @Gioele: use specialized funob/funmor notation $\funobspace{F}, \funmorspace{G}$ notation}

\begin{lemma}
    \label{lem:functors_compose}
    The composition of \SY{functors} is a \SY{functor}.
\end{lemma}

\begin{exercise}
    Prove \cref{lem:functors_compose}.
\end{exercise}
\begin{solution}
    In the following, we want to show that \SY{functors} compose.
    Given categories~$\CatA,\CatB,\CatC$ and \SY{functors}~$\funa\colon \CatA\fto \CatB$,~$\funb \colon \CatB\fto \CatC$, we want to show that~$\funa\fthen\funb$ is a \SY{functor}.
    To do this, we show that~$\funa\fthen\funb$ preserves identities and compositions.
    \begin{itemize}
        \item Given an object~$\Obja\setin \CatA$, we have:
              \begin{equation}\label{eq:functor-comp-proof-1}
                  \begin{aligned}
                      (\funamor\fthen\funbmor)(\catidat\Obja)
                       & =\funbmor(\funamor(\catidat\Obja)) \\
                       & =\funbmor(\catidat{\funaob(\Obja)}) \\
                       & =\catidat{\funbob(\funaob(\Obja))},
                  \end{aligned}
              \end{equation}
              where we used that~$\funa$ and~$\funb$ are \SY{functors} (they preserve identities).
        \item Furthermore, given composable morphisms~$\mora,\morb\setin \CatA$, we have:
              \begin{equation}\label{eq:functor-comp-proof-2}
                  \begin{aligned}
                      %   (\funamor\fthen\funbmor)(\morab)
                      %    & =\funbmor(\funamor(\mora)\mthen \funamor(\morb)) \\
                      %    & =\funbmor(\funamor(\mora))\mthen \funbmor(\funamor(\morb)),
                      (\funa\fthen\funb)(\morab)
                       & =\funb(\funa(\mora)\mthen \funa(\morb)) \\
                       & =\funb(\funa(\mora))\mthen \funb(\funa(\morb)) \\
                       & = (\funa\fthen\funb)(\mora) \mthen (\funa\fthen\funb)(\morb),
                  \end{aligned}
              \end{equation}
              where again we used that~$\funa,\funb$ are \SY{functors} (they preserve composition).
    \end{itemize}
\end{solution}

% We can define an \SY{identity functor}.

% \begin{ctdefinition}[Identity functor]
%     \label{def:identity_functor}
%     Given a category~\CatC, we define the \maindef{identity functor} ~$\funidC\colon \CatC \fto \CatC$, where
%     \begin{itemize}
%         \item Given~$\Obja\setin \ObC$, we have~$\funidC(\Obja)\definedas \Obja$;
%         \item Given~$\mora \setin \HomSet{\CatC}{\Obja}{\Objb}$, we have~$\funidC(\mora)\definedas \mora$.
%     \end{itemize}
% \end{ctdefinition}
