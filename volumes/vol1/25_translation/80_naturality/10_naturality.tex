% !TEX root = chapter-standalone.tex

\section{Natural transformations}

\linkvideo{spring2021-nat-trafos:diagrams} % Diagrams
%\linkvideo{spring2021-nat-trafos:natural-trafos} % Natural transformations

The name ``natural transformation'' has historical roots -- do not worry if the concept itself does not seem natural to you at first! We will try to explain it in a way that is hopefully at least clear. 

The situation we start with is when we have two functors $\funa : \CatC \fto \CatD$  and $\funb : \CatC \fto \CatD$, sharing the same source and target, respectively. A natural transformation from $\funa$ to $\funb$ is a kind of ``map'' that relates the two functors. How might one define such a thing? 

Consider a morphism $\mora : \Obja \mto \Objb$ in the category $\CatC$. Under the functor $\funa$ it will be mapped to some morphism $\funa(\mora): \funa(\Obja) \mto \funb(\Objb)$ in $\CatD$, and under the functor $\funb$ it will be mapped to some other morphism $\funb(\mora): \funb(\Obja) \mto \funb(\Objb)$ in $\CatD$. 

\todo{Insert diagram illustrating the situation thus far. }

We can think of $\funa(\mora): \funa(\Obja) \mto \funb(\Objb)$ and $\funb(\mora): \funb(\Obja) \mto \funb(\Objb)$ as each being very small diagrams (directed graphs) in $\CatD$. One straightforward way to relate the one to the other is if there are morphisms $\alpha_\Obja$ and $\alpha_\Objb$ in $\CatD$ forming the square diagram depicted in [REF]. Additionally, we will assume that $\alpha_\Obja$ and $\alpha_\Objb$ make the diagram in [REF] commutative. (We will discuss in a moment why this is a convenient assumption to make.) 
 
\todo{Insert diagram illustrating  naturality square.}

Now consider not only a single morphism $\mora : \Obja \mto \Objb$ in $\CatC$ being mapped by $\funa$ and $\funb$, respectively, but all of the category $\CatC$. Under $\funa$, the category $\CatC$ is mapped to a -- possibly very complicated --  diagram in $\CatD$ (a directed graph of objects and morphism comprising the image of $\funa$), and similarly, under $\funb$, the category $\CatC$ is mapped to another diagram in $\CatD$ (the image of $\funb$). To relate the image of $\funa$ to the image of $\funb$ we can proceed in the same way as above: for each object $\Obja$ in $\CatC$, we choose a morphism $\alpha_\Obja : \funa(\Obja) \mto \funb{\Objb}$ in $\CatD$. This will give rise to lots of squares of the kind in [REF]. In [REF] we have illustrated a situation involving three objects and two morphisms in $\CatC$. We have ``glued'' the two squares together since they share an edge (this a more compact way of drawing them). 

\todo{Insert diagram illustrating two adjacent naturality squares.}



\

\

\linkvideo{spring2021-nat-trafos:natural-trafos:nat-trafo-def} % Definition of natural transformation
\linkvideo{spring2021-nat-trafos:natural-trafos:nattrafos-as-mor} % Natural transformations are morphisms between functors

\begin{ctdefinition}[Natural transformation]
    \label{def:natural-transformation}
    Let \CatC and \CatD be categories, and let~$\funa,\funb\colon \CatC\fto \CatD$ be functors.
    A \emph{\iindex{natural transformation}}~$\ntrafoa \colon \funa\nto \funb$ is specified by:

    \constit
    \begin{enumerate}
        \item For each object~$\Obja\in \CatC$, a morphism $\ntrafoa_\Obja \colon \funa(\Obja)\mto \funb(\Obja)$ in \CatD, called the $\Obja$\emph{-component} of $\ntrafoa$.
    \end{enumerate}
    \condit
    \begin{enumerate}
        \item For every morphism~$\mora\colon \Obja\mto \Objb$ in \CatC, the components of $\ntrafoa$ must satisfy the \emph{naturality condition}
              \begin{equation}
                  \funa(\mora)\mthen \ntrafoa_\Objb = \ntrafoa_\Obja\mthen \funb(\mora).
              \end{equation}
              In other words, the following diagram must commute:
              \begin{equation}
                  \middlesag{55_natural_2}.
              \end{equation}
    \end{enumerate}f

\end{ctdefinition}
A natural transformation~$\ntrafoa \colon \funa\nto \funb$ is denoted visually as follows:
\begin{equation}
    \middlesag{55_natural_1}.
\end{equation}
\begin{figure}[h!]
    \centering
    \begin{ctdefinitionshade}
        \includesag{096_natural_graphically}
    \end{ctdefinitionshade}
    \caption{}
    \label{fig:nat_trans_graphically}
\end{figure}

\todo{@J: add a remark about the origins of the name ``natural transformation'' and assure the reader that the concept need not feel particularly ``natural''.}

\begin{ctdefinition}[Natural isomorphism]
    \label{def:nat_iso}
    A natural transformation~$\ntrafoa \colon \funa \nto \funb $ is called a \emph{\iindex{natural isomorphism}} if each component morphism ~$\ntrafoa_\Obja$ in \CatD is an isomorphism.
\end{ctdefinition}
\todotextjira{473}{@J: seems like a lot of things are missing here.}
\clearpage

\section{Examples}
\linkvideo{spring2021-nat-trafos:natural-trafos:double-dual} % Double dual

\begin{example}
    \label{ex:Vect}
    Consider the category~$\Vect_{\reals}$ whose objects are real vector spaces and whose morphisms are linear maps.
    (For convenience, in the following we sometimes omit reference to the ground field.)
    Recall that the \emph{dual} of a vector space~$V$ is the vector space describing all linear maps from~$V$ to~\reals:
    \begin{equation}
        \label{eq:dual-vector-space}
        V^* \definedas \HomSet{\Vect}{V}{\reals},
    \end{equation}
    Also, recall that if~$\mapa\colon V \mto W$ is a linear map, then its dual is a linear map~$\mapa^* \lb  \colon W^* \mto V^*$.

    Applying the above duality construction twice to a vector space or a linear map gives their double dual.
    It turns out that this is a functorial operation.
    That is, there is a functor
    \begin{equation}
        \label{eq:double-dual-functor}
        \text{\stylefunctors{Double dual}}\colon \Vect \fto \Vect
    \end{equation}
    that maps every vector space and every linear map to its double dual.

    Furthermore, for any vector space~$V$, there is a ``canonical'' or ``natural'' map~
    \begin{equation}
        \ntrafoa_V \colon V \nto V^{**}
    \end{equation}
    defined by
    \begin{equation}
        \label{eq:natural-trafo-to-double-dual}
        \ntrafoa_V(v)(l) = l(v), \quad  v \in V, l \in V^*.
    \end{equation}
    These form the components of a natural transformation from the identity functor on \Vect to the double dual functor.
    \begin{center}
        \includesag{nat-trafo-ddual}
    \end{center}
\end{example}

\begin{example}
    Consider the powerset functor from \cref{ex:powerset_functor}, and denote it~$\funa$.
    As a reminder, the functor maps each set to its powerset, and each map to a map distributed over subsets of the powerset.
    We now look at a natural transformation~$\ntrafoa\colon \funid_\Set \nto \funa$, specified by
    \begin{equation*}
        \begin{aligned}
            \ntrafoa_\setA\colon \funid_\Set(\setA) & \mto \funa(\setA)     \\
            \setA                                   & \mto \powerset(\setA) \\
            \setAel                                 & \mapsto \{\setAel\}.
        \end{aligned}
    \end{equation*}
    In other words, the natural transformation embeds each element of~$\setA$ into the power set~$\powerset(\setA)$.
    To check that this is a valid natural transformation, consider $\mora\colon \setA\mto \setB$.
    One has
    \begin{equation*}
        \begin{aligned}
            (\ntrafoa_\setA \mthen \funa(\mora))(\setAel)
             & =\{\setAel\} \mthen \funa(\mora)         \\
             & =\{\mora(\setAel)\}                      \\
             & =\mora(\setAel)\mthen \ntrafoa_\setB     \\
             & =(\mora \mthen \ntrafoa_\setB)(\setAel).
        \end{aligned}
    \end{equation*}
\end{example}

\vfill

\begin{marginfigure}
    \centering
    \includesag{graph-cat}
    \caption{}
    \label{fig:graph-cat-again}
\end{marginfigure}
\begin{gradedexercise}[\exname{NatTrafosGraphs}]
    \label{ex:NatTrafosGraphs}
    This exercise builds on \cref{ex:GraphsViaFunctors}.
    There, we defined a category~$\Cat{G}$ which has precisely two objects and four morphisms, see \cref{fig:graph-cat-again} (the two identity morphisms are not drawn).
    The task there was to understand how specifying a functor from this category $G$ into the category of sets is `the same thing' as specifying a directed graph.

    Now consider two functors~$\funaA, \funaB \colon \Cat{G} \fto \Set$.
    Spell out what it means to have a natural transformation~$\ntrafoa\colon \funaA \nto \funaB$.
    What does this correspond to in the language of directed graphs?
\end{gradedexercise}

\solutionof{NatTrafosGraphs}

\begin{gradedexercise}[\exname{UpperSetsNatTrafos}]
    \label{ex:UpperSetsNatTrafos}
    This exercise builds on \cref{ex:UpperSetsViaFunctors}.
    There we fixed a poset~$\posA$, viewed it as a category~$\Cat{P}$, and saw that functors~$\Cat{P} \fto \Bool$ encode upper sets in~$\posA$.
    Suppose we have two functors~$\funaA, \funaB \colon \Cat{P} \fto \Bool$.
    What does a natural transformation~$\ntrafoa \colon \funaA \nto \funaB$ correspond to in terms of the upper sets encoded by~$\funaA$ and~$\funaB$, respectively?
\end{gradedexercise}

\solutionof{UpperSetsNatTrafos}

\todojira{464}{AC: References for functional programming}

\devel{
    \section{To add}
    \todojira{141}{Add parts corresponding to listed videos.}
    \linkvideo{spring2021-nat-trafos:natural-trafos:horizontal-composition} % Horizontal composition
    \linkvideo{spring2021-nat-trafos:natural-trafos:interchange-law} % Interchange law
    \linkvideo{spring2021-nat-trafos:natural-trafos:rel-mon-maps} % Relating monotone maps
    \linkvideo{spring2021-nat-trafos:natural-trafos:eq-maps-gr-actions} % Equivariant maps between group actions
}

\showslides{
    \begin{forslides}
        \begin{equation}
            \label{eq:ex_nt_1}
            \ntrafoa \colon \funid_\Set \nto \funa
        \end{equation}
        \begin{equation}
            \label{eq:ex_nt_2}
            \begin{aligned}
                \ntrafoa_\setA \colon \funid_\Set(\setA) & \mto \funa(\setA)   \\
                \setAel                                  & \mapsto \{\setAel\}
            \end{aligned}
        \end{equation}
    \end{forslides}
}
