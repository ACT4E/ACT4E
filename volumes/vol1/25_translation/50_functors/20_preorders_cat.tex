% !TEX root = chapter-standalone.tex


\section{A poset as a category}
\label{sec:posetsarecats}
A single poset~$\tup{\posA, \posleq}$ can be described as a category, in which each point~$\posAel\in \posA$ is an object, and there is a morphism between~$\posAel_1$ and~$\posAel_2$ if and only if~$\posAel_1 \posAleq \posAel_2$. This is a ``thin'' category, wich means that there is at most one morphism
between two objects: For any~$\posAel_1,\posAel_2\in \posA$, there exist only one relation~$\posAel_1\posAleq \posAel_2$ in~$\posA$ (\cref{def:poset}). The identity morphism is given by the reflexivity property of posets: for any~$\posAel\in \posA$, we have~$\posAel \posAleq \posAel$. Furthermore, composition is given by the transitivity property of posets: for~$\posAel_1,\posAel_2,\posAel_3 \in \posA$,~$\posAel_1\posAleq \posAel_2$ and~$\posAel_2\posAleq \posAel_3$ implies~$\posAel_1\posAleq \posAel_3$.

\begin{example}
  Let's revisit \cref{ex:hasseinclusion}, in which we had a poset~$\powerset{\left(\{a,b,c\}\right)}$ with order given by inclusion (\cref{fig:posetascat}).

  \begin{figure}[h!]
    \begin{center}
      \includesag{40_dpcatfig_power}
    \end{center}
    \caption{Power set~$\powerset{\{a,b,c\}}$ as a category. \label{fig:posetascat}}
  \end{figure}

  This is a category~\CatC, with~$\ObC=\powerset\left(\{a,b,c\}\right)$, and morphisms given by the inclusions. Note that we omit to draw self-arrows for the identity morphisms. Composition is given by the transitivity law of posets. For instance, since~$\{a\}\subseteq \{a,b\}$ and~$\{a,b\} \subseteq \{a,b,c\}$, we can say that~$\{a\}\subseteq \{a,b,c\}$.
\end{example}

\subsection{Monotone maps are functors}
Note that morphisms in \Pos are morphisms between posets, and we have just discovered that posets are examples of categories.
\begin{lemma}
  \label{lem:posetfunctor}
  A monotone map~$\mapa$ between posets~$\posA,\posB$ is a functor between the ``posetal categories''~$\posA$ and~$\posB$.
\end{lemma}
\begin{proof}
  We start by specifying the functor~$\mapa$ and two posets~$\posA$ and~$\posB$. We first specify the action of~$\mapa$ on objects (elements of a poset) and on morphisms (order relations). A monotone function maps each element of a poset~$\posAel\in \posA$ to~$\mapa(\posAel) \in \posB$, and it guarantees that for every~$\posAel_1,\posAel_2\in \posA$, if $\posAel_1\posAleq \posAel_2$ then~$\mapa(\posAel_1)\posAleq \mapa(\posAel_2)$. We now need to check the two conditions that a functor must satisfy. First, consider the identity morphism for~$\posAel\in \posA$, namely~$\posAel \posAleq \posAel$. The application of the map~$\mapa$ results in the condition~$\mapa(\posAel)\posAleq \mapa(\posAel)$, which is the identity morphism on~$\posB$. Second, morphisms~$\posAel_1\posAleq \posAel_2$ and~$\posAel_2\posleq \posAel_3$ in~$\posA$, by applying the map~$\mapa$ to the morphism composition~$\posAel_1\posAleq \posAel_3$ one obtains~$\mapa(\posAel_1)\posAleq \mapa(\posAel_3)$, which is the same as the composition of~$\mapa(\posAel_1)\posAleq \mapa(\posAel_2)$ and~$\mapa(\posAel_2)\posAleq \mapa(\posAel_3)$.
\end{proof}

\includepdf[scale=0.8,pages={25-26},nup=1x3,frame,pagecommand={}]{ACT4E-06-posets.pdf} % preorder as category
