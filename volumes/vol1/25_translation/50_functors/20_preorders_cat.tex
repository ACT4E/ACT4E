% !TEX root = chapter-standalone.tex

\subsection[\dots as \SY{semicategory actions}]{Functors generalize \SY{semicategory actions}}

\linkvideo{spring2021-functors:semi-and-fun:ex-semigroup-semifun} % Semigroup morphisms as semi-functors

Semi-functors are a generalization of the various \SY{semigroup morphisms} that we saw in the previous chapter.

\linkvideo{spring2021-functors:semi-and-fun:ex-semicat-act-fun} % Semi-category actions as functors

In particular, they are a generalization of \SY{semicategory actions} (\cref{def:semicategory-action}), which we can re-define as follows.

\begin{ctdefinition}[Semicategory actions, redefined]\label{def:semicat-action-redefined}
    A \maindef{semicategory action} of~\CatC is a \SY{semifunctor}~$\funa\colon \CatC \fto \Set$.
\end{ctdefinition}

\subsection[\dots as  monotone maps]{Functors generalize monotone maps.}
\label{sec:posetsarecats}

\linkvideo{spring2021-tradeoffs:tradeoffs:orders:preorder-as-cat} % Pre-order as a category
\linkvideo{spring2021-tradeoffs:tradeoffs:orders:preorder-poset} % The skeleton of a pre-order is a poset

Recall that a single \SY{poset}~$\posAdefinition$ can be viewed as a category $\poscat{\posA}$, in which each element of the poset is an object, and there is a morphism between two objects~$\posela$ and~$\poselb$ if and only if~$\posela \posAleq \poselb$ (\cref{def:poscat}).

% \subsection{Monotone maps are functors}

\linkvideo{spring2021-functors:semi-and-fun:mon-functions:mon-fun-as-func} % Monotone functions as functors

\begin{lemma}
    \label{lem:posetfunctor}
    A \SY{monotone map} between \SY{posets}~$\posA,\posB$ is isomorphic to a \SY{functor} between the ``posetal categories''~$\poscat{\posA}$ and~$\poscat{\posB}$.
\end{lemma}
\begin{proof}
    We start by specifying the \SY{functor}~$\funa$ and two posetal categories~$\poscat{\posA}$ and~$\poscat{\posB}$.
    We first specify the action of~$\funa$ on objects (elements of a \SY{poset} considered as objects of the posetal category) and on morphisms (order relations considered as morphisms of the posetal category).
    A \SY{monotone function} maps each element of a poset~$\posela\setin \posAset$ to~$\funa(\posela) \setin \posBset$, and it guarantees that for every~$\posela,\poselb\setin \posAset$, if~$\posela\posAleq \poselb$ then~$\funa(\posela)\posBleq \funa(\poselb)$.
    We now need to check the two conditions that a \SY{functor} must satisfy.
    First, consider the \SY{identity morphism} for~$\posela\setin \posAset$, namely~$\posela \posAleq \posela$.
    The application of the map~$\funa$ results in the condition~$\funa(\posela)\posBleq \funa(\posela)$, which is the \SY{identity morphism} on~\posB.
    Second, morphisms~$\posela\posAleq \poselb$ and~$\poselb\posleq \poselc$ in~\posA, by applying the map~$\funa$ to the morphism composition~$\posela\posAleq \poselc$ we obtain~$\funa(\posela)\posBleq \funa(\poselc)$, which is the same as the composition of~$\funa(\posela)\posBleq \funa(\poselb)$ and~$\funa(\poselb)\posBleq \funa(\poselc)$.
\end{proof}

\todomistake{J: I find the statement of the lemma, and its proof, confusing.
    In particular, I would not use the word ``isomorphic" in the statement of the lemma, since I think this cannot be made into a precise/correct statement.
}

%\devel{\includepdf[scale=0.8,pages={25-26},nup=1x3,frame,pagecommand={}]{ACT4E-06-posets.pdf}} % preorder as category
