% !TEX root = chapter-standalone.tex

\section[\dots as semicategory actions]{Functors generalize semicategory actions}

\linkvideo{spring2021-functors:semi-and-fun:ex-semigroup-semifun} % Semigroup morphisms as semi-functors

Semi-functors are a generalization of the various semigroup morphisms that we saw in the previous chapter.

\linkvideo{spring2021-functors:semi-and-fun:ex-semicat-act-fun} % Semi-category actions as functors

In particular, they are a generalization of semicategory actions (\cref{def:semicategory-action}), which we can re-define as follows.

\begin{ctdefinition}[Semicategory actions, redefined]
    A semicategory action of~$\CatC$ is a semi-functor~$\funa\colon \CatC \fto \Set$.
\end{ctdefinition}

\section[\dots as monotone maps]{Functors generalize monotone maps}
\label{sec:posetsarecats}

\todostructure{Video "Preorder as a category" should be moved to where we talk about categorification.}
\linkvideo{spring2021-tradeoffs:tradeoffs:orders:preorder-as-cat} % Pre-order as a category
\linkvideo{spring2021-tradeoffs:tradeoffs:orders:preorder-poset} % The skeleton of a pre-order is a poset

A single poset~$\posA=\tup{\posAset, \posleq}$ can be described as a category, in which each point is an object, and there is a morphism between two objects~$\posela$ and~$\poselb$ if and only if~$\posela \posAleq \poselb$.

\todotextjira{514}{@Gioele: Typographical conventions not followed in this definition.}
\begin{ctdefinition}[Categorification of a poset]
    Given a poset~$\posA$, we consider its \emph{categorification}~$\poscat{\posA}$, which is a category with
    \begin{compactenum}
        \item \emph{Objects}: Objects of~$\poscat{\posA}$ are elements of~$\posAset$;
        \item \emph{Morphisms}: Given~$\posAel,\posBel\in \poscat{\posA}$, there exists a morphism~$\mora\colon \posAel\mto \posBel$ if and only if~$\posAel \posAleq \posBel$.
              Note that because~$\posA$ is a poset, this is a ``thin'' category, wich means that there is at most one morphism between two objects.
              .
        \item \emph{Identity morphism}: Given any~$\posAel \in \poscat{\posA}$, the identity morphism is given by \emph{reflexivity} in posets.
        \item \emph{Composition of morphisms}: Composition is given by \emph{transitivity}
    \end{compactenum}
\end{ctdefinition}

\begin{marginfigure}
    \centering
    \includesag{40_dpcatfig_power}
    \caption{Power set~$\powerset{\{\setAel,\setBel,\setCel\}}$ as a poset. \label{fig:posetascat}}
\end{marginfigure}

\begin{example}
    Let's revisit \cref{ex:hasseinclusion}, in which we had a poset~$\posA$ on $\powerset \makeset{\setAel,\setBel,\setCel}$ with order given by inclusion (\cref{fig:posetascat}).
    Its categorification~$\poscat{\posA}$ is a category, with $\Ob_{\poscat{\posA}}=\powerset\left(\makeset{\setAel,\setBel,\setCel}\right)$, and morphisms given by the inclusions.
    Note that we omit to draw self-arrows for the identity morphisms.
    Composition is given by the transitivity law of posets.
    For instance, since~$\{\setAel\}\subseteq \{\setAel,\setBel\}$ and~$\{\setAel,\setBel\} \subseteq \{\setAel,\setBel,\setCel\}$, we can say that~$\{\setAel\}\subseteq \{\setAel,\setBel,\setCel\}$.
\end{example}

% \subsection{Monotone maps are functors}

\linkvideo{spring2021-functors:semi-and-fun:mon-functions:mon-fun-as-func} % Monotone functions as functors

Note that morphisms in \Pos are morphisms between posets, and we have just discovered that posets are examples of categories.
\begin{lemma}
    \label{lem:posetfunctor}
    A monotone map~$\funa$ between posets~$\posA,\posB$ is isomorphic to a functor between the ``posetal categories''~$\poscat{\posA}$ and~$\poscat{\posB}$.
\end{lemma}
\begin{proof}
    \todo{adjust here}
    We start by specifying the functor~$\funa$ and two posets~$\posA$ and~$\posB$.
    We first specify the action of~$\funa$ on objects (elements of a poset) and on morphisms (order relations).
    A monotone function maps each element of a poset~$\posela\in \posA$ to~$\funa(\posela) \in \posB$, and it guarantees that for every~$\posela,\poselb\in \posA$, if~$\posela\posAleq \poselb$ then~$\funa(\posela)\posBleq \funa(\poselb)$.
    We now need to check the two conditions that a functor must satisfy.
    First, consider the identity morphism for~$\posela\in \posA$, namely~$\posela \posAleq \posela$.
    The application of the map~$\funa$ results in the condition~$\funa(\posela)\posBleq \funa(\posela)$, which is the identity morphism on~$\posB$.
    Second, morphisms~$\posela\posAleq \poselb$ and~$\poselb\posleq \poselc$ in~$\posA$, by applying the map~$\funa$ to the morphism composition~$\posela\posAleq \poselc$ one obtains~$\funa(\posela)\posBleq \funa(\poselc)$, which is the same as the composition of~$\funa(\posela)\posBleq \funa(\poselb)$ and~$\funa(\poselb)\posBleq \funa(\poselc)$.
\end{proof}

%\devel{\includepdf[scale=0.8,pages={25-26},nup=1x3,frame,pagecommand={}]{ACT4E-06-posets.pdf}} % preorder as category
