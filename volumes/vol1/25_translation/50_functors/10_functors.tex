% !TEX root = chapter-standalone.tex
\section{Functors}

We give a first definition for semi-categories.

%\linkvideo{spring2021-functors:semi-and-fun} % Semifunctors and functors

\begin{figure*}
    \begin{ctdefinitionshade}
        \subfloat[\label{fig:functor_detail}Functor diagram] {
            \centering
            \includesag{095_functor_detail}
        }
        \subfloat[\label{fig:functor_representation}Synthetic notation] {
            \centering
            \includesag{095_functor}
        }
    \end{ctdefinitionshade}
    \caption{Commuting diagrams for semi-functors, with verbose notation (left) and synthetic notation (right).}
\end{figure*}

\linkvideo{spring2021-functors:semi-and-fun:semi-fun-def} % Definition of semi-functor
\begin{ctdefinition}[\iindex{Semi-functor}]
    \label{def:semi-functor}
    Given two semi-categories~\CatC and~\CatD, a \emph{\iindex{semi-functor}}~$\funa\colon \CatC\fto \CatD$ from~\CatC to~\CatD is defined by the following constituents and conditions. \\
    \underline{Constituents:}
    \begin{compactenum}
        [i)]
        \item A map
              \begin{equation}
                  \funaob\colon \ObC \to \ObD.
              \end{equation}
        \item For every pair of objects~$\Obja, \Objb\in \Obof\CatC$ a map
              \begin{equation}
                  \funamor\colon \HomSet{\CatC}{\Obja}{\Objb} \to \HomSet{\CatD}{\funaob(\Obja)}{\funaob(\Objb)}.
              \end{equation}
    \end{compactenum}
    \underline{Conditions:}
    \begin{compactenum}
        \item It holds that the functor application to morphisms commutes with category composition:
              \begin{equation}
                  \prftree[r]{.}{
                      \mora\colon \Obja \mto_{\CatC} \Objb
                  }{
                      \morb\colon \Objb \mto_{\CatC} \Objc
                  }{
                      \funamor(\mora \mthen \morb)=\funamor(\mora)\mthen \funamor(\morb)
                  }
              \end{equation}
    \end{compactenum}
\end{ctdefinition}

\showslides{
    \begin{forslides}
        \begin{ctdefinition}[\iindex{Semi-functor}]
            \label{def:semi-functor-compact}
            A \emph{\iindex{semi-functor}}~$\funa\colon \CatC\fto \CatD$ between two semi-categories is defined by a map
            \begin{equation}
                \funaob: \ObC \to \ObD.
            \end{equation}
            and, for every pair of objects $\Obja, \Objb$, a map
            \begin{equation}
                \funamor: \HomSet{\CatC}{\Obja}{\Objb} \to \HomSet{\CatD}{\funaob(\Obja)}{\funaob(\Objb)}
            \end{equation}
            such that
            \begin{equation}
                \label{eq:semifunctor-condition}
                \prftree{\mora: \Obja \to_{\CatC} \Objb}{\morb: \Objb \to_{\CatC} \Objc}{
                    \funamor(\mora \mthen \morb)=\funamor(\mora)\then \funamor(\morb)
                }.
            \end{equation}
        \end{ctdefinition}
        %\begin{equation}\label{eq:functor}
        %\includesag{95_functor}
        %\end{equation}
        %\begin{equation}\label{eq:functor_detail}
        %\includesag{95_functor_detail}
        %\end{equation}
        \begin{equation}
            \label{eq:slides_Ff}
            \funa(\mora)
        \end{equation}
        \begin{equation}
            \label{eq:slides_Fg}
            \funa(\morb)
        \end{equation}
        \begin{equation}
            \label{eq:slides_Fh}
            \funa(\morc)
        \end{equation}
        \begin{equation}
            \label{eq:slides_Ffg}
            \funa(\mora\mthen\morb)
        \end{equation}
        \begin{equation}
            \label{eq:mono1}
            \funa: \posA \to \posB
        \end{equation}
        \begin{equation}
            \label{eq:mono2}
            \prftree{\posela \posAleq \poselb}{\funa(\posela) \posBleq \funa(\poselb)}
        \end{equation}
        \begin{equation}
            \label{eq:mono3}
            \prftree[double]{\posela \posAleq \poselb}{\funa(\posela) \posBleq \funa(\poselb)}
        \end{equation}
        \begin{equation}
            \label{eq:functor-composition1}
            \prftree{\funa\colon \CatC \fto \CatD}{\funb\colon\CatD\fto\CatE}{\funa\fthen\funb\colon \CatC \fto \CatE}
        \end{equation}
        \begin{equation}
            \label{eq:slides_mora}
            \mora\colon \Obja\to\Objb
        \end{equation}
        \begin{equation}
            \label{eq:slides_mora_obja}
            \Obja
        \end{equation}
        \begin{equation}
            \label{eq:slides_mora_objb}
            \Objb
        \end{equation}
        \begin{equation}
            \label{eq:slides_funa}
            \funa\colon\CatC\fto\CatD
        \end{equation}
        \begin{equation}
            \label{eq:slides_ntrafoa}
            \ntrafoa\colon\funa\nto\funb
        \end{equation}
        \begin{equation}
            \label{eq:slides_hom}
            \HomSet\Category\CatC\CatD
        \end{equation}
        \begin{equation}
            \label{eq:pow_fun_1}
            \funa\colon \Set \fto \Set
        \end{equation}
        \begin{equation}
            \label{eq:pow_fun_2}
            \funaob
        \end{equation}
        \begin{equation}
            \label{eq:pow_fun_3}
            \begin{aligned}
                \funamor(\mora)\colon \powerset(\setA) & \fto \powerset(\setB)                              \\
                \setC                                  & \mapsto \{ \mora(\setCel) \mid \setCel \in \setC\}
            \end{aligned}
        \end{equation}
        \begin{equation}
            \label{eq:pow_fun_4}
            \setA=\{ 1,2,3\} \text{ and }\setB=\{3,6,9\}
        \end{equation}
        \begin{equation}
            \label{eq:idfun}
            \funid_\CatC \colon \CatC \fto \CatC
        \end{equation}
    \end{forslides}
}

This situation is graphically reported in \cref{fig:functor_detail}.
% 
It is common to overload the notation and use~$\funa$ to mean both~$\funaob$ and~$\funamor$.
The diagram with this notation overload is in \cref{fig:functor_representation}.

For categories we have the stronger concept of \emph{functor}s.
Categories have identities, and functors are required to preserve the identities.

\linkvideo{spring2021-functors:semi-and-fun:fun-def} % Definition of functor
\begin{ctdefinition}[\iindex{Functor}]
    \label{def:functor}
    A functor from category~\CatC to category~\CatD is a semi-functor~$\funa\colon \CatC \fto \CatD$
    that satisfies the condition
    \begin{equation}
        \label{eq:functor-condition}
        \funamor(\catid_\Obja)=\catid_{\funaob(\Obja)}.
    \end{equation}
\end{ctdefinition}

A functor from a category to itself is called an \emph{\iindex{endofunctor}}.
The simplest example of an endofunctor is the identity functor.

\begin{definition}[Identity functor]
    For any (semi-)category \CatC, one can define the \emph{identity (semi-)functor}
    \begin{equation}
        \funid_\CatC \colon \CatC \fto \CatC,
    \end{equation}
    which maps each object to itself and each morphism to itself.
\end{definition}

To see that this is indeed a functor, notice that $\funid_\CatC(\mora \mthen \morb)$ is equal to
$\funid_\CatC(\mora)\mthen \funid_\CatC(\morb)$ because they are both equal to $\mora \mthen \morb$.
Moreover, identities are preserved because~$\funid_\CatC(\catid_\CatC)=\catid_\CatC$.