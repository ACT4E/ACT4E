% !TEX root = chapter-standalone.tex

\section{Semifunctors}

\linkvideo{spring2021-functors:semi-and-fun:semi-fun-def} % Definition of semi-functor


%\linkvideo{spring2021-functors:semi-and-fun} % Semifunctors and functors

\begin{ctdefinition}[Semifunctor]
    \label{def:semifunctor}
    Given \SY{semicategories}~\CatC and~\CatD, a \maindef{semifunctor}~$\funa\colon \CatC\fto \CatD$ from~\CatC to~\CatD is:
    \\
  
   \constit
   
    \begin{enumerate}
        \item A map
              \begin{equation}
                  \funaob \colon \ObC \to \ObD.
              \end{equation}
        \item For every pair of objects~$\Obja, \Objb\setin \Obof\CatC$ a map
              \begin{equation}
                  \funamor\colon \HomSet{\CatC}{\Obja}{\Objb} \to \HomSet{\CatD}{\funaob(\Obja)}{\funaob(\Objb)}.
              \end{equation}
    \end{enumerate}
    
    \condit
    
    \begin{enumerate}
        \item \SYN{semifunctor}{Semifunctor} application to morphisms is compatible with the respective category composition operations:
              \begin{equation}
                  \prfperiod{
                      \mora\colon \Obja \mtoin{\CatC} \Objb
                  }{\quad}{
                      \morb\colon \Objb \mtoin{\CatC} \Objc
                  }{
                      \funamor(\mora \mthenof{\CatC} \morb)=\funamor(\mora) \mthenof{\CatD} \funamor(\morb)
                  }
              \end{equation}
    \end{enumerate}
\end{ctdefinition}

This situation is depicted graphically in \cref{fig:functor_detail}.
%
It is common to overload the notation and use~$\funa$ to mean both~$\funaob$ and~$\funamor$.
The diagram with this overloaded ``synthetic notation'' is in \cref{fig:functor_representation}.

\vfill

\begin{figure}[h!]
    % \begin{ctdefinitionshade}
    \subfloat[\label{fig:functor_detail}
        Functor diagram] {
        \centering
        \includesag{095_functor_detail}
    }
    \subfloat[\label{fig:functor_representation}Synthetic notation] {
        \centering
        \includesag{095_functor}
    }
    % \end{ctdefinitionshade}
    \caption{
        Commuting diagrams for \SY{semifunctors}, with verbose notation (left) and synthetic notation (right).
    }
    \todographics{@Gioele: Label categories outside.
        Grey borders are too tight.
    }
\end{figure}

\vspace{2cm}


\section{Functors}
\label{sec:functors}
\linkvideo{spring2021-functors:semi-and-fun:fun-def} % Definition of functor

For categories, we have the stronger concept of \SY{functor}.
Categories have identities, and \SY{functors} are required to preserve the identities.

\begin{ctdefinition}[Functor]
    \label{def:functor}
    A \maindef{functor} from category~\CatC to a category~\CatD is a \SY{semifunctor}~$\funa\colon \CatC \fto \CatD$
    that satisfies the condition
    \begin{equation}
        \label{eq:functor-condition}
        \funamor(\catidat\Obja)=\catidat{\funaob(\Obja)}
    \end{equation}
    for all objects $\Obja$ in $\CatC$. 
\end{ctdefinition}

\todotextjira{493}{@JL: we need some pedagogical examples here}

A \SY{functor} from a category to itself is called an \maindef{endofunctor}.
The simplest example of an \SY{endofunctor} is the \SY{identity functor}.

\begin{definition}[Identity (semi)functor]\label{def:identity-semifunctor}
    \SYNDEF{identity semifunctor}
    For any (semi)category \CatC, we can define the \emph{identity (semi)functor}
    \begin{equation}
        \funidC \colon \CatC \fto \CatC,
    \end{equation}
    which maps each object to itself and each morphism to itself.
\end{definition}

\begin{exercise}
    Check that the identity functor is a functor.
\end{exercise}
\begin{solution}
    To show that this is a valid \SY{functor}, we need to show that it preserves identities and composition:
    \begin{itemize}
        \item Given any~$\Obja \setin \ObC$, we have:
              \begin{equation}
                  \begin{aligned}
                      \funid_{\CatC}(\catidat\Obja) & =\catidat\Obja \\
                                                    & =\catidat{\funid_{\CatC}(\Obja)}
                  \end{aligned}
              \end{equation}
              Furthermore, given composable morphisms~$\mora,\morb$ in~\CatC, we have:
              \begin{equation}
                  \begin{aligned}
                      \funid_{\CatC}(\morab) & =\morab \\
                                             & =\funid_{\CatC}(\mora)\mthen \funid_{\CatC} (\morb).
                  \end{aligned}
              \end{equation}
    \end{itemize}
\end{solution}

% To see that this is indeed a \SY{functor}, notice that~$\funidC(\morab)$ is equal to~$\funidC(\mora)\mthen \funidC(\morb)$ because they are both equal to~$\morab$.
% Moreover, identities are preserved because~$\funidC(\catidof\CatC)=\catidof\CatC$.

\todojira{620}{\alphubel: @JL: let's discuss the ``image'' of a \SY{functor} somewhere}
