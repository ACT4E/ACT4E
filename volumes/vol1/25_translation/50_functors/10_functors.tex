% !TEX root = chapter-standalone.tex

\section{Semifunctors}

\linkvideo{spring2021-functors:semi-and-fun:semi-fun-def} % Definition of semi-functor
We give a first definition for \SY{semicategories}.

%\linkvideo{spring2021-functors:semi-and-fun} % Semifunctors and functors

\begin{ctdefinition}[Semifunctor]
    \label{def:semifunctor}
    Given two \SY{semicategories}~\CatC and~\CatD, a \maindef{semifunctor}~$\funa\colon \CatC\fto \CatD$ from~\CatC to~\CatD is defined by the following constituents and conditions.
    \\
    \underline{Constituents:}
    \begin{enumerate}
        [i)]
        \item A map
              \begin{equation}
                  \funaob \colon \ObC \to \ObD.
              \end{equation}
        \item For every pair of objects~$\Obja, \Objb\setin \Obof\CatC$ a map
              \begin{equation}
                  \funamor\colon \HomSet{\CatC}{\Obja}{\Objb} \to \HomSet{\CatD}{\funaob(\Obja)}{\funaob(\Objb)}.
              \end{equation}
    \end{enumerate}
    \underline{Conditions:}
    \begin{enumerate}
        \item It holds that the \SY{functor} application to morphisms commutes with category composition:
              \begin{equation}
                  \prfperiod{
                      \mora\colon \Obja \mtoin{\CatC} \Objb
                  }{\quad}{
                      \morb\colon \Objb \mtoin{\CatC} \Objc
                  }{
                      \funamor(\morab)=\funamor(\mora)\mthen \funamor(\morb)
                  }
              \end{equation}
    \end{enumerate}
\end{ctdefinition}

This situation is graphically reported in \cref{fig:functor_detail}.
%
It is common to overload the notation and use~$\funa$ to mean both~$\funaob$ and~$\funamor$.
The diagram with this notation overload is in \cref{fig:functor_representation}.

\vfill

\begin{figure*}[h!]
    % \begin{ctdefinitionshade}
    \subfloat[\label{fig:functor_detail}
        Functor diagram] {
        \centering
        \includesag{095_functor_detail}
    }
    \subfloat[\label{fig:functor_representation}Synthetic notation] {
        \centering
        \includesag{095_functor}
    }
    % \end{ctdefinitionshade}
    \caption{
        Commuting diagrams for \SY{semifunctors}, with verbose notation (left) and synthetic notation (right).
    }
\end{figure*}

\vspace{2cm}

\showslides{
    \begin{forslides}
        \begin{ctdefinition}[Semifunctor]
            \label{def:semi-functor-compact}
            A \emph{semifunctor}~$\funa\colon \CatC\fto \CatD$ between two \SY{semicategories} is defined by a map
            \begin{equation}
                \funaob: \ObC \to \ObD
            \end{equation}
            and, for every pair of objects $\Obja, \Objb$, a map
            \begin{equation}
                \funamor: \HomSet{\CatC}{\Obja}{\Objb} \to \HomSet{\CatD}{\funaob(\Obja)}{\funaob(\Objb)}
            \end{equation}
            such that
            \begin{equation}
                \label{eq:semifunctor-condition}
                \prftree{\mora \colon \Obja \to_{\CatC} \Objb}{\morb: \Objb \to_{\CatC} \Objc}{
                    \funamor(\morab)=\funamor(\mora)\mthen \funamor(\morb)
                }.
            \end{equation}
        \end{ctdefinition}
        %\begin{equation}\label{eq:functor}
        %\includesag{95_functor}
        %\end{equation}
        %\begin{equation}\label{eq:functor_detail}
        %\includesag{95_functor_detail}
        %\end{equation}
        \begin{equation}
            \label{eq:slides_Ff}
            \funa(\mora)
        \end{equation}
        \begin{equation}
            \label{eq:slides_Fg}
            \funa(\morb)
        \end{equation}
        \begin{equation}
            \label{eq:slides_Fh}
            \funa(\morc)
        \end{equation}
        \begin{equation}
            \label{eq:slides_Ffg}
            \funa(\morab)
        \end{equation}
        \begin{equation}
            \label{eq:mono1}
            \funa: \posA \to \posB
        \end{equation}
        \begin{equation}
            \label{eq:mono2}
            \prftree{\posela \posAleq \poselb}{\funa(\posela) \posBleq \funa(\poselb)}
        \end{equation}
        \begin{equation}
            \label{eq:mono3}
            \prfdouble{\posela \posAleq \poselb}{\funa(\posela) \posBleq \funa(\poselb)}
        \end{equation}
        \begin{equation}
            \label{eq:functor-composition1}
            \prftree{\funa\colon \CatC \fto \CatD}{\funb\colon\CatD\fto\CatE}{\funa\fthen\funb\colon \CatC \fto \CatE}
        \end{equation}
        \begin{equation}
            \label{eq:slides_mora}
            \mora\colon \Obja\mto\Objb
        \end{equation}
        \begin{equation}
            \label{eq:slides_mora_obja}
            \Obja
        \end{equation}
        \begin{equation}
            \label{eq:slides_mora_objb}
            \Objb
        \end{equation}
        \begin{equation}
            \label{eq:slides_funa}
            \funa\colon\CatC\fto\CatD
        \end{equation}
        \begin{equation}
            \label{eq:slides_ntrafoa}
            \ntrafoa\colon\funa\nto\funb
        \end{equation}
        \begin{equation}
            \label{eq:slides_hom}
            \HomSet\Category\CatC\CatD
        \end{equation}
        \begin{equation}
            \label{eq:pow_fun_1}
            \funa\colon \Set \fto \Set
        \end{equation}
        \begin{equation}
            \label{eq:pow_fun_2}
            \funaob
        \end{equation}
        \begin{equation}
            \label{eq:pow_fun_3}
            \begin{aligned}
                \funamor(\mora)\colon \powerset(\setA) & \fto \powerset(\setB) \\
                \setC                                  & \mapsto \makeset{ \mora(\setCel) \mid \setCel \setin \setC}
            \end{aligned}
        \end{equation}
        \begin{equation}
            \label{eq:pow_fun_4}
            \setA=\makeset{ 1,2,3} \qqand \setB=\makeset{3,6,9}
        \end{equation}
        \begin{equation}
            \label{eq:idfun}
            \funidC \colon \CatC \fto \CatC
        \end{equation}
    \end{forslides}
}

\section{Functors}
\label{sec:functors}
\linkvideo{spring2021-functors:semi-and-fun:fun-def} % Definition of functor

For categories, we have the stronger concept of \SY{functors}.
Categories have identities, and \SY{functors} are required to preserve the identities.

\begin{ctdefinition}[Functor]
    \label{def:functor}
    A \maindef{functor} from category~\CatC to category~\CatD is a \SY{semifunctor}~$\funa\colon \CatC \fto \CatD$
    that satisfies the condition
    \begin{equation}
        \label{eq:functor-condition}
        \funamor(\catidat\Obja)=\catidat{\funaob(\Obja)}.
    \end{equation}
\end{ctdefinition}

\todotextjira{493}{@JL: we need some pedagogical examples here}

A \SY{functor} from a category to itself is called an \maindef{endofunctor}.
The simplest example of an \SY{endofunctor} is the \SY{identity functor}.

\begin{definition}[Identity (semi-)functor]\label{def:identity-semifunctor}
    \SYNDEF{identity semifunctor}
    For any (semi-)category \CatC, we can define the \emph{identity (semi-)functor}
    \begin{equation}
        \funidC \colon \CatC \fto \CatC,
    \end{equation}
    which maps each object to itself and each morphism to itself.
\end{definition}

\begin{exercise}
    Check that the identity functor is a functor.
\end{exercise}
\begin{solution}
    To show that this is a valid \SY{functor}, we need to show that it preserves identities and composition:
    \begin{itemize}
        \item Given any~$\Obja \setin \ObC$, we have:
              \begin{equation}
                  \begin{aligned}
                      \funid_{\CatC}(\catidat\Obja) & =\catidat\Obja \\
                                                    & =\catidat{\funid_{\CatC}(\Obja)}
                  \end{aligned}
              \end{equation}
              Furthermore, given composable morphisms~$\mora,\morb$ in~\CatC, we have:
              \begin{equation}
                  \begin{aligned}
                      \funid_{\CatC}(\morab) & =\morab \\
                                             & =\funid_{\CatC}(\mora)\mthen \funid_{\CatC} (\morb).
                  \end{aligned}
              \end{equation}
    \end{itemize}
\end{solution}

% To see that this is indeed a \SY{functor}, notice that~$\funidC(\morab)$ is equal to~$\funidC(\mora)\mthen \funidC(\morb)$ because they are both equal to~$\morab$.
% Moreover, identities are preserved because~$\funidC(\catidof\CatC)=\catidof\CatC$.

\todojira{620}{\alphubel: @JL: let's discuss the ``image'' of a \SY{functor} somewhere}
