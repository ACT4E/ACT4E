% !TEX root = chapter-standalone.tex

\clearpage
\section{Functors}
\linkvideo{spring2021-functors:semi-and-fun} % Semifunctors and functors

\begin{marginfigure}
  \begin{center}
    \includesag{095_functor_detail}
  \end{center}
  \caption{Commuting diagram for semi-functors}
  \label{fig:functor_detail}
\end{marginfigure}


\linkvideo{spring2021-functors:semi-and-fun:semi-fun-def} % Definition of semi-functor
\begin{ctdefinition}[\iindex{Semi-functor}]
  \label{def:semi-functor}
  Given two semi-categories~\CatC and~\CatD, a \emph{\iindex{semi-functor}}~$\funa\colon \CatC\fto \CatD$ from~\CatC to ~\CatD is defined by the following data and conditions. \\
  \underline{Data:}
  \begin{compactenum}
    [i)]
    \item A map
    \begin{equation}
      \funaob\colon \ObC \to \ObD.
     \end{equation}
    \item For every pair of objects~$\Obja, \Objb$ of~\CatC a map
    \begin{equation}
      \funamor\colon \HomSet{\CatC}{\Obja}{\Objb} \to \HomSet{\CatD}{\funaob(\Obja)}{\funaob(\Objb)}
    \end{equation}
  \end{compactenum}
  \underline{Conditions:}
  \begin{compactenum}
    \item It holds that
    \begin{equation}
      \prftree{\mora\colon \Obja \to_{\CatC} \Objb}{\morb\colon \Objb \to_{\CatC} \Objc}{
      \funamor(\mora \mthen \morb)=\funamor(\mora)\then \funamor(\morb)
      }.
    \end{equation}
  \end{compactenum}
\end{ctdefinition}

\devel{
\begin{forslides}
\begin{ctdefinition}[\iindex{Semi-functor}]
  \label{def:semi-functor-compact}
  A \emph{\iindex{semi-functor}}~$\funa\colon \CatC\fto \CatD$ between two semi-categories is defined by a map
    \begin{equation}
      \funaob: \ObC \to \ObD.
     \end{equation}
  and, for every pair of objects $\Obja, \Objb$, a map
    \begin{equation}
      \funamor: \HomSet{\CatC}{\Obja}{\Objb} \to \HomSet{\CatD}{\funaob(\Obja)}{\funaob(\Objb)}
    \end{equation}
  such that
    \begin{equation}\label{eq:semifunctor-condition}
      \prftree{\mora: \Obja \to_{\CatC} \Objb}{\morb: \Objb \to_{\CatC} \Objc}{
      \funamor(\mora \mthen \morb)=\funamor(\mora)\then \funamor(\morb)
      }.
    \end{equation}
\end{ctdefinition}
%\begin{equation}\label{eq:functor}
%\includesag{95_functor}
%\end{equation}
%\begin{equation}\label{eq:functor_detail}
%\includesag{95_functor_detail}
%\end{equation}
\begin{equation}\label{eq:slides_Ff}
\funa(\mora)
\end{equation}
\begin{equation}\label{eq:slides_Fg}
\funa(\morb)
\end{equation}
\begin{equation}\label{eq:slides_Fh}
\funa(\morc)
\end{equation}
\begin{equation}\label{eq:slides_Ffg}
\funa(\mora\mthen\morb)
\end{equation}
\begin{equation}\label{eq:mono1}
\funa: \posA \to \posB
\end{equation}
\begin{equation}\label{eq:mono2}
\prftree{\posela \posAleq \poselb}{\funa(\posela) \posBleq \funa(\poselb)}
\end{equation}
\begin{equation}\label{eq:mono3}
\prftree[double]{\posela \posAleq \poselb}{\funa(\posela) \posBleq \funa(\poselb)}
\end{equation}
\begin{equation}\label{eq:functor-composition1}
\prftree{\funa\colon \CatC \fto \CatD}{\funb\colon\CatD\fto\CatE}{\funa\fthen\funb\colon \CatC \fto \CatE}
\end{equation}
\begin{equation}\label{eq:slides_mora}
\mora\colon \Obja\to\Objb
\end{equation}
\begin{equation}\label{eq:slides_mora_obja}
 \Obja
\end{equation}
\begin{equation}\label{eq:slides_mora_objb}
 \Objb
\end{equation}
\begin{equation}\label{eq:slides_funa}
 \funa\colon\CatC\fto\CatD
\end{equation}
\begin{equation}\label{eq:slides_ntrafoa}
 \ntrafoa\colon\funa\nto\funb
\end{equation}
\begin{equation}\label{eq:slides_hom}
 \HomSet\Category\CatC\CatD
\end{equation}
  \begin{equation}\label{eq:pow_fun_1}
    \funa\colon \Set \fto \Set
\end{equation}
    \begin{equation}
\label{eq:pow_fun_2}
    \funaob
\end{equation}
      \begin{equation}
\label{eq:pow_fun_3}
\begin{aligned}
    \funamor(\mora)\colon \powerset(\setA)&\fto \powerset(\setB)\\
    \setC&\mapsto \{ \mora(\setCel) \mid \setCel \in \setC\}
\end{aligned}
\end{equation}
\begin{equation}
  \label{eq:pow_fun_4}
    \setA=\{ 1,2,3\} \text{ and }\setB=\{3,6,9\}
\end{equation}
  \begin{equation}
  \label{eq:idfun}
    \funid_\CatC \colon \CatC \fto \CatC
\end{equation}
\end{forslides}}

This situation is graphically reported in \cref{fig:functor_detail}.

It is common to overload the notation and use~$\funa$ to mean both~$\funaob$ and~$\funamor$. The diagram with this notation overload is in \cref{fig:functor_representation}.


\begin{marginfigure}
  \begin{center}
    \includesag{095_functor}
  \end{center}
  \caption{Commuting diagram for semi-functors, overloading the notation.}
  \label{fig:functor_representation}
\end{marginfigure}

\linkvideo{spring2021-functors:semi-and-fun:ex-semigroup-semifun} % Semigroup morphisms as semi-functors
Semi-functors are a generalization of the various semigroup morphisms that we saw in the previous chapter.


\linkvideo{spring2021-functors:semi-and-fun:ex-semicat-act-fun} % Semi-category actions as functors
In particular, they are a generalization of semi-category actions (\cref{def:semicategory-action}), which we can re-define as follows.

\begin{ctdefinition}[Semi-category actions, redefined]
  A semi-category action of~$\CatC$ is a semi-functor~$\funa\colon \CatC \fto \Set$.
\end{ctdefinition}


\linkvideo{spring2021-functors:semi-and-fun:embedding-fun} % Embedding functors
\begin{ctdefinition}[Embedding functor]\label{def:embedding-functor}
A functor~$\funa\colon \CatC \fto \CatD$ is an \emph{embedding} if~$\funaob$ and~$\funamor$ are injective.
\end{ctdefinition}

For categories we have the stronger concept of \emph{functor}s.
Categories have identities, and functors are required to preserve the identities.


\linkvideo{spring2021-functors:semi-and-fun:fun-def} % Definition of functor
\begin{ctdefinition}[\iindex{Functor}]
  \label{def:functor}
  A functor from category~\CatC to category~\CatD is a semi-functor~$\funa\colon \CatC \fto \CatD$
  that satisfies the condition
  \begin{equation}\label{eq:functor-condition}
    \funamor(\catid_\Obja)=\catid_{\funaob(\Obja)}.
  \end{equation}
\end{ctdefinition}

\begin{remark}
  A functor from a category to itself is called an \emph{\iindex{endofunctor}}.
\end{remark}

\todo{Write example from recital of power set functor. Instrumental for later nat trafo}


\begin{example}\label{exa:double-dual-functor}
Let~$\CatC$ be the category whose objects are all real vector spaces and whose morphisms are~$\reals$-linear maps.
Composition is the usual composition of linear maps. There is an endofunctor~$\funa \colon \CatC \mto \CatC$ whose action on objects is
\begin{equation}
\funa ( \vecspB ) = \vecspB^{**}.
\end{equation}
(Recall that~$\vecspB^{**} = \{ \text{linear maps } \vecspB^* \mto \reals \} =  \HomSet{\CatC}{\vecspB^*}{\reals}$).
The action of~$\funa$ on morphisms is as follows. Given a linear map~$\mora \colon \vecspB \mto \vecspC$,
\begin{equation}
\funa ( \mora ) \colon \vecspB^{**} \mto \vecspC^{**}, \quad \xi \mapsto [ l \mapsto \xi (f \then l)].
\end{equation}
\end{example}

\begin{gradedexercise}[\exname{DoubleDualFunctor}]\label{ex:DoubleDualFunctor}
Prove that $\funa$ as defined in \cref{exa:double-dual-functor} is in fact a functor. 
\end{gradedexercise}
\solutionof{DoubleDualFunctor}

\begin{gradedexercise}[\exname{GraphsViaFunctors}]\label{ex:GraphsViaFunctors}
Consider the following category, which has two objects,~$\styleobj{V}$ and~$\styleobj{A}$, and four morphisms:
besides the identity morphisms, there are two morphisms,~$\stylemorph{s}$ and~$\stylemorph{t}$, from~$\styleobj{A}$ to~$\styleobj{V}$.
See \cref{fig:graph-cat}. Call this category~$\Cat{G}$.

Can you explain the following statement? ``Specifying a functor~$\Cat{G} \mto \Set$ is the `same thing' as specifying a directed graph''.
\end{gradedexercise}
\solutionof{GraphsViaFunctors}


\begin{marginfigure}
\begin{center}
\includesag{graph-cat}
\end{center}
\caption{}
  \label{fig:graph-cat}
\end{marginfigure}

\begin{gradedexercise}[\exname{UpperSetsViaFunctors}]\label{ex:UpperSetsViaFunctors}
Recall that~$\Bool$ denotes the category with two objects,~$\true$ and~$\false$, and with precisely one non-identity morphism which goes from~$\false$ to~$\true$.
Let~$\posA$ be a poset. View it as a category~$\Cat{P}$, and let~$\funa \colon \Cat{P} \mto \Bool$ be a functor. In other words,~$\funaMap = \funaob$ is a monotone function.
Prove that the set
\begin{equation}
\subA \coloneqq \{ \posAel \in \posA \mid \funaMap(\posAel) = \true \} \subseteq \posA
\end{equation}
is an upper set. 
\end{gradedexercise}
\solutionof{UpperSetsViaFunctors}

%\begin{comment}
%%%%%%%%%
%\clearpage
%\section{Staring at Pareto fronts}
%
%\section{Chains and Antichains} \label{sec:chains-antichains}
%
%
%\begin{definition}[Chain in a poset]
%\label{def:chain}
%Given a poset~$S$, a \emph{chain} is a sequence of elements~${s_i}$ in~$S$ where two successive elements are comparable, \ie :
%\begin{equation}
%    i \ordleq j \Rightarrow s_i \ordleq s_j.
%\end{equation}
%\end{definition}
%
%\begin{definition}[Antichain in a poset]
%\label{def:antichain}
%An \emph{antichain} is a subset~$S$ of a poset where no elements are comparable. If~$a,b \in S$, then~$a \ordleq b$ implies~$a=b$.
%\end{definition}
%\begin{remark}
%We denote the set of antichains of a poset~$P$ by~$\antichains P$.
%\end{remark}
%\begin{remark}
%Note that given a poset~$\tup{P,\ordleq}$,~$\emptyset \in \antichains P$ since~$\emptyset$ contains no elements (and hence no comparable elements).
%\end{remark}
%
%In the context of pizza recipes, consider the diagram reported in~\cref{fig:antichain}. The blue points represent an antichain of recipes~$\{\tup{\unit[1]{USD},\unit[2]{h}},\tup{\unit[2]{USD},\unit[1]{h}}\}$, \ie  recipes which do not dominate each other (one is cheaper but takes longer and the other is more expensive but quicker). The red point represents a recipe which cannot be part of the antichain, since it is dominated by~$\tup{\unit[2]{USD},\unit[1]{h}}$.
%
%\begin{figure}[h!]
%\begin{center}
%\includesag{70_antichain}
%\end{center}
%\caption{Example of an antichain of pizza recipes. \label{fig:antichain}}
%\end{figure}
%
%
%\begin{example}
%Let's consider the poset~$\tup{P,\ordleq}$ where~$a\ordleq b$ if~$a$ is a divisor of~$b$ and~$P=\{1,5,10,11,13,15\}$. A chain of~$P$ is~$\{1,5,10,15\}$. An antichain of~$P$ is~$\{10,11,13\}$.
%\end{example}
%
%\begin{example}
%Consider \cref{ex:hasseinclusion}. Examples of chains are
%\begin{equation}
%    \{\varnothing,\{a\},\{a,b\},\{a,b,c\}\}, \quad  \{\varnothing,\{b\},\{b,c\},\{a,b,c\}\}.
%\end{equation}
%Examples of antichains are
%\begin{equation}
%    \{\{a\},\{b\},\{c\}\}, \quad \{ \{a,b\},\{a,c\}, \{b,c\}\}.
%\end{equation}
%\end{example}
%
%\begin{example}
%\label{ex:battery}
%Suppose you have to choose a battery model based on its cost and its weight, both to be minimized. There may be models which dominate others. For instance, a model~$\tup{\unit[10]{USD},\unit[1]{kg}}$ is always better than a model~$\tup{\unit[11]{USD},\unit[1.1]{kg}}$. Also, there may be models which are incomparable, \ie  which form an antichain. For example, you cannot say whether~$\tup{\unit[10]{USD},\unit[1]{kg}}$ is better than~$\tup{\unit[5]{USD},\unit[2]{kg}}$. The incomparable models form an antichain.
%\end{example}
%
%\section{Upper and lower sets}
%
%\begin{definition}[Upper set]
%\label{def:upperset}
%An \emph{upper set} is a subset~$U$ of a poset~$P$ such
%that, if an element is inside, all elements above it are inside as well.
%In formulas:
%\begin{equation}
%\text{$U$ is an upperset} \equiv \forall x\in U, \forall y\in P\colon x\ordleq y \Imp y\in U.
%\end{equation}
%\end{definition}
%\begin{remark}
%We call~$\mathsf{U}P$ the set of upper sets of~$P$.
%\end{remark}
%
%\begin{definition}[Lower set]
%\label{def:lowerset}
%A \emph{lower set} is a subset~$L$ of a poset~$P$ if, if a point is inside, all points below it are inside as well. In formulas:
%\begin{equation}
%\text{$L$ is a lower set} \equiv \forall x\in L, \forall y\in P\colon y\ordleq x \Imp y\in L.
%\end{equation}
%\end{definition}
%\begin{remark}
%We call~$\mathsf{L}P$ the set of lower sets of~$P$.
%\end{remark}
%
%\begin{remark}
%Note that if~$A$ is an antichain of a poset~$P$, then the set
%\begin{equation}
%    I(A)=\{x\colon x\ordleq y, y\in A\}
%\end{equation}
%is a lower set of~$P$.
%\end{remark}
%
%Consider the blue poset of pizza recipes from before. The upper and lower sets of this poset can be represented as in~\cref{fig:upperset}. The upper set can be interpreted as all the potential pizza recipes for which we can find better alternatives in the poset. Similarly, the lower set can be interpreted as all the potential pizza recipes which would be better than the ones in the poset.
%
%\begin{figure}[h!]
%\begin{center}
%\includesag{70_upper_lower_set}
%\end{center}
%\caption{Example of upper and lower sets of a poset of pizza recipes. \label{fig:upperset}}
%\end{figure}
%\begin{example}[Upper and lower sets in~\Bool]
%The booleans~$\{\false, \true \}$ form a poset with~$\false \leq \true\colon(\Bool,\ordleq)$ . The subset~$\{\false\} \subseteq \Bool$ is not an upper set, since~$\false \leq \true$ and~$\true \notin \{\false \}$.
%\end{example}
%
%\begin{lemma}
%$\UR$ is a bounded lattice (\cref{def:lattice}) with
%\begin{equation}
%    \{\UR,\leq_{\UR},\bot_{\UR},\top_{\UR},\vee_{\UR},\wedge_{\UR}\}=\{\UR,\supseteq,R,\emptyset,\cap,\cup\}.
%\end{equation}
%\begin{proof}
%Consider the poset~$\tup{\UR,\supseteq}$ and~$P,Q\in \UR$.
%\begin{itemize}
%    \item First, we need to show that~$P\cap Q\in \UR$. One has~$P \subseteq \UR$ and $Q\subseteq \UR$, meaning that by definition, if~$x\in P\cap Q$, we have~$x\in P \wedge x\in Q$. It follows that~$x\in \UR$ for all~$x\in P\cap Q$. Furthermore, we need to show that~$P\cap Q$ is the least upper bound of $P,Q$. Assume this is not true, \ie  there exists a~$T\in \UR$,~$T\neq P\cap Q$, such that~$P\supseteq T\supseteq P\cap Q$ and~$Q\supseteq T\supseteq P\cap Q$. Using the fact that intersection preserves inclusions, one has
%\begin{equation}
%\begin{aligned}
%    P\cap Q &\supseteq T\cap T \supseteq P\cap Q\\
%    P\cap Q &\supseteq T \supseteq P\cap Q\\
%    T&= P\cap Q,
%\end{aligned}
%\end{equation}
%which contradicts the assumption. Therefore,~$P\cap Q$ is the least upper bound of~$P,Q$.
%\item Second, we need to show that~$P\cup Q\in \LF$. One has~$P\subseteq \UR$ and~$Q\subseteq \UR$, meaning that by definition, if~$x\in P\cup Q$, we have either~$x\in P$ or~$x\in Q$. If~$x\in P$, then~$x\in \UR$. If~$x\in Q$, then~$x\in \UR$. It follows that~$x\in \UR$ for all~$x\in P\cup Q$.  Furthermore, we need to show that~$P\cup Q$ is the greatest lower bound of~$P,Q$. Assume this is not true, \ie  there exists a~$T\in \UR$,~$T\neq P\cup Q$, such that~$P\cup Q\supseteq T\supseteq P$ and~$P\cup Q\supseteq T\supseteq Q$. Using the fact that union preserves inclusions, one has
%\begin{equation}
%    \begin{aligned}
%    (P\cup Q)\cup (P\cup Q) &\supseteq T \cup T \supseteq P\cup Q\\
%    P\cup Q &\supseteq T\supseteq P\cup Q\\
%    T&=P\cup Q,
%    \end{aligned}
%\end{equation}
%which contradicts the assumption.  Therefore,~$P\cup Q$ is the greatest lower bound of~$P,Q$.
%\end{itemize}
%We have therefore proved that~$\tup{\UR,\supseteq}$ is a lattice. To show that it is bounded, we notice that~$\emptyset \subseteq T$ for any~$T\in \UR$, meaning that~$\emptyset$ is the top. Furthermore, we notice that~$T\subseteq R$ for any~$T\in \UR$, meaning that~$R$ is a bottom. Therefore, the lattice is bounded.
%\end{proof}
%\end{lemma}
%
%\begin{lemma}
%$\LF$ is a bounded lattice (\cref{def:lattice}) with
%\begin{equation}
%    \{\LF,\leq_{\LF},\bot_{\LF},\top_{\LF},\vee_{\LF},\wedge_{\LF}\}=\{\LF,\subseteq,\emptyset,F,\cup,\cap\}.
%\end{equation}
%\end{lemma}
%\begin{proof}
%Consider the poset~$\tup{\LF,\subseteq}$ and~$P,Q\in \LF$.
%\begin{itemize}
%    \item First, we need to show that~$P\cup Q\in \LF$. One has~$P \subseteq \LF$ and~$Q\subseteq \LF$, meaning that by definition, if~$x\in P\cup Q$, either~$x\in P$ or~$x\in Q$. If~$x\in P$, then~$x\in \LF$. If~$x\in Q$, then~$x\in \LF$. It follows that~$x\in \LF$ for all~$x\in P\cup Q$. Furthermore, we need to show that~$P\cup Q$ is the least upper bound of~$P,Q$. Assume this is not true, \ie  there exists a~$T\in \LF$,~$T\neq P\cup Q$, such that~$P\subseteq T\subseteq P\cup Q$ and~$Q\subseteq T\subseteq P\cup Q$. Using the fact that union preserves inclusions, one has
%\begin{equation}
%\begin{aligned}
%    P\cup Q &\subseteq T\cup T \subseteq P\cup Q\\
%    P\cup Q &\subseteq T \subseteq P\cup Q\\
%    T&= P\cup Q,
%\end{aligned}
%\end{equation}
%which contradicts the assumption. Therefore,~$P\cup Q$ is the least upper bound of~$P,Q$.
%\item Second, we need to show that~$P\cap Q\in \LF$. One has~$P\subseteq \LF$ and~$Q\subseteq \LF$, meaning that by definition, if~$x\in P\cap Q$, we have~$x\in P\wedge x\in Q$, \ie ~$x\in \LF$, for all~$x\in P\cap Q$. Furthermore, we need to show that~$P\cap Q$ is the greatest lower bound of~$P,Q$. Assume this is not true, \ie  there exists a~$T\in \LF$,~$T\neq P\cap Q$, such that~$P\cap Q\subseteq T\subseteq P$ and~$P\cap Q\subseteq T\subseteq Q$. Using the fact that intersection preserves inclusions, oen has
%\begin{equation}
%    \begin{aligned}
%    (P\cap Q)\cap (P\cap Q) &\subseteq T \cap T \subseteq P\cap Q\\
%    P\cap Q &\subseteq T\subseteq P\cap Q\\
%    T&=P\cap Q,
%    \end{aligned}
%\end{equation}
%which contradicts the assumption.  Therefore,~$P\cap Q$ is the greatest lower bound of~$P,Q$.
%\end{itemize}
%We have therefore proved that~$\tup{\LF,\subseteq}$ is a lattice. To show that it is bounded, we notice that~$\emptyset \subseteq T$ for any~$T\in \LF$, meaning that~$\emptyset$ is the bottom. Furthermore, we notice that~$T\subseteq F$ for any~$T\in \LF$, meaning that~$F$ is a top. Therefore, the lattice is bounded.
%\end{proof}
%
%\GZ{Discuss the above with JL}
%\section{From antichains to uppersets, and viceversa}
%\begin{definition}[Upper closure operator]
%\label{def:upperclosure}
%The \emph{upper closure operator} $\uparrow$ maps a subset to the smallest upper set that includes it, \ie :
%\begin{equation}
%    \begin{aligned}
%    \uparrow \colon \powerset(P)&\to \mathsf{U}P\\
%    S&\mapsto \{y\in P \mid \exists x\in S \colon x\ordleq y\}.
%    \end{aligned}
%\end{equation}
%\end{definition}
%\begin{remark}
%Note that, by definition, an upper set is closed to upper closure.
%\end{remark}
%\begin{remark}
%For any~$S\in \powerset(P)$,~$\uparrow S$ is in fact an upper set.
%\begin{proof}
%Suppose~$y\in \uparrow S$ and~$z\in P$, and suppose $y\ordleq z$. By definition~$\exists x$ s.t.~$x\ordleq y$, meaning that~$x\ordleq z$. Thus,~$z\in \uparrow S$, as was to be shown.
%\end{proof}
%\end{remark}
%
%\begin{lemma}
%The upper closure operator~$\uparrow$ is a monotone map.
%\end{lemma}
%\begin{proof}
%Consider the posets~$\tup{\powerset(P),\subseteq}$ and~$\tup{\mathsf{U}P,\supseteq}$, and~$S_1,S_2\in \powerset(P)$. It is clear that given~$S_1\subseteq S_2$, one has
%\begin{equation}
%    \{y\in P\mid \exists x\in S_1\colon x\ordleq y\} \supseteq \{y\in P\mid \exists x\in S_2\colon x\ordleq y\}.
%\end{equation}
%Therefore,~$\uparrow S_1\supseteq \ \uparrow S_2$, satisfing the monotonicity property for~$\uparrow$.
%\end{proof}
%
%In the example of the pizza recipes, first, consider the upper set of a single element of the poset, \eg ~$p_1=\tup{\unit[1]{USD},\unit[2]{h}}$  (\cref{fig:upperclosure_1}).
%\begin{figure}[h!]
%\begin{center}
%\includesag{70_upper_closure_1}
%\end{center}
%\caption{The upper closure of a singleton set of pizza recipes. \label{fig:upperclosure_1}}
%\end{figure}
%Then, consider the case of two elements, with~$p_2=\tup{\unit[2]{USD},\unit[1]{h}}$ (\cref{fig:upperclosure_2}).
%
%\begin{figure}[h!]
%\begin{center}
%    \includesag{70_upper_closure_2}
%\end{center}
%\caption{The upper closure of a set of pizza recipes. \label{fig:upperclosure_2}}
%\end{figure}
%Note that the upper set of the subset formed by the two elements is the union of the upper sets of the single elements.
%
%\begin{definition}[Lower closure operator]
%The \emph{lower closure operator} $\downarrow$ maps a subset to the smallest lower set that includes it, \ie
%\begin{equation}
%    \begin{aligned}
%    \downarrow\colon \powerset(P)&\to \mathsf{L}P\\
%    S&\mapsto \{ y\in P \mid \exists x\in S \colon y\ordleq x\}.
%    \end{aligned}
%\end{equation}
%\end{definition}
%
%\begin{lemma}
%The lower closure operator $\downarrow$ is a monotone map.
%\end{lemma}
%\begin{proof}
%Consider the posets~$\tup{\powerset(P),\subseteq}$ and~$\tup{\mathsf{L}P,\subseteq}$, and let~$S_1,S_2\in \powerset(P)$. It is clear that given~$S_1\subseteq S_2$, one has
%\begin{equation}
%    \{y\in P\mid \exists x\in S_1\colon y\ordleq x\} \subseteq \{y\in P\mid \exists x\in S_2\colon y\ordleq x\}.
%\end{equation}
%Therefore,~$\uparrow S_1\subseteq \ \uparrow S_2$, satisfing the monotonicity property for~$\downarrow$.
%\end{proof}
%
%\begin{definition}[Min]
%\label{def:Min}
%$\Min \colon \powerset(P) \to \antichains P$ is the map that sends a subset~$S$ of a poset to the minimal elements of that subset, \ie , those elements~$a \in S$ such that~$a \ordleq b$ for all~$b \in S$. In formulas:
%\begin{equation}
%    \begin{aligned}
%    \Min \colon \powerset(P) &\to \antichains P\\
%    S&\mapsto \{ x\in S\colon (y\in S)\wedge(y\ordleq x)\Rightarrow (x=y)\}.
%    \end{aligned}
%\end{equation}
%Note that~$\Min(S)$ could be empty.
%\end{definition}
%
%\begin{definition}[Max]
%\label{def:Max}
%$\Max \colon \powerset(P) \to \antichains P$ is the map that sends a subset~$S$ of a poset to the maximal elements of that subset, \ie , those elements~$a \in S$ such that~$a \ordgeq b$ for all~$b \in S$. In formulas:
%\begin{equation}
%    \begin{aligned}
%    \Max \colon \powerset(P) &\to \antichains P\\
%    S&\mapsto \{ x\in S\colon (y\in S)\wedge(y\ordgeq x)\Rightarrow (x=y)\}.
%    \end{aligned}
%\end{equation}
%Note that~$\Max(S)$ could be empty.
%\end{definition}
%
%
%\begin{definition}[Downward closed set]
%\label{def:downward-closed-upperset}
%An upper set~$S$ is \emph{downward-closed} in a poset~$P$ if
%\begin{equation}
%    S =\, \uparrow \Min S.
%\end{equation}
%\end{definition}
%\begin{remark}
%The set of downward-closed upper sets of~$P$ is denoted~$\underline{\mathsf{U}}P$.
%\end{remark}
%
%\begin{example}
%Consider the battery example of~\cref{ex:battery}, and the antichain given by the battery models~$a=\tup{\unit[10]{USD},\unit[1]{kg}}$,~$b=\tup{\unit[20]{USD},\unit[0.5]{kg}}$, and~$c=\tup{\unit[30]{USD},\unit[0.25]{kg}}$ (\cref{fig:examplebatt}).
%The upper closure~$\uparrow \{a,b,c\}$ represents all the existing battery models dominated by elements in~$\{a,b,c\}$. The lower closure uperator~$\downarrow\{a,b,c\}$ represents all the battery models which, if existing, would dominate~$\{a,b,c\}$.
%\begin{figure}[h!]
%\begin{center}
%    \includesag{70_battery_1}
%\end{center}
%\caption{Battery example. From the left: antichain, upper set, and lower set. \label{fig:examplebatt}}
%\end{figure}
%\end{example}
%
%\begin{lemma}
%Given a poset~$\tup{P,\ordleq}$,~$\tup{\antichains P,\ordleq_{\antichains P}}$ is a poset with
%\begin{equation}
%\label{eq:orderantichain}
%    A\ordleq_{\antichains P} B \text{ if and only if } \uparrow A \supseteq \ \uparrow B.
%\end{equation}
%Furthermore, it is bounded by the top~$\top_{\antichains P}=\emptyset$ and the bottom~$\bot_{\antichains P}=\{\bot_P\}$.
%\end{lemma}
%
%\begin{proof}
%We need to show the poset properties (\cref{def:poset}).
%We can prove the following:
%\begin{compactitem}
%\item \emph{Reflexivity}: From~$\tup{P,\ordleq}$ being a poset we know that
%\begin{equation}
%\begin{aligned}
%\{y\in P \mid \exists x\in A \colon x\ordleq y\} &\supseteq \{y\in P \mid \exists x\in A \colon x\ordleq y\},\\
%\uparrow A =\ \uparrow A
%\end{aligned}
%\end{equation}
%and hence~$A\ordleq_{\antichains P}A$.
%\item \emph{Antisymmetry}: One has
%\begin{equation}
%    \begin{aligned}
%    \left(A\ordleq_{\antichains P} B\right) \wedge \left(B\ordleq_{\antichains P} A\right)
%    &\Leftrightarrow \left(\uparrow A \supseteq \ \uparrow \ B\right) \wedge \left( \uparrow  B\supseteq \ \uparrow \ A\right)\\
%    &\Leftrightarrow A=_{\antichains P} B.
%    \end{aligned}
%\end{equation}
%\item \emph{Transitivity}: One has
%\begin{equation}
%    \begin{aligned}
%    \left(A\ordleq_{\antichains P} B\right) \wedge \left(B\ordleq_{\antichains P} C\right)&\Leftrightarrow  \left(\uparrow A \supseteq \ \uparrow \ B\right) \wedge \left( \uparrow  B\supseteq \ \uparrow C\right)\\
%    &\Imp \ \uparrow A\supseteq \ \uparrow C\\
%    &\Imp A\ordleq_{\antichains P}C.
%    \end{aligned}
%\end{equation}
%In order to find the top, we need to find the smallest set~$\top_{\antichains P}$ such that~$A\ordleq_{\antichains P} \top_{\antichains P}$ for all~$A\in \antichains P$. In other words, such that~$\uparrow A\supseteq \ \uparrow \top_{\antichains P}$ for all~$A\in \antichains P$. This is clearly~$\emptyset$, since~$\uparrow \emptyset = \emptyset$. Similarly, in order to find the bottom, we need to find the set~$\bot_{\antichains P}$ such that~$\bot_{\antichains P} \ordleq_{\antichains P} A$ for all~$A\in \antichains P$. In other words, such that~$\uparrow \bot_{\antichains P} \supseteq \ \uparrow A$ for all~$A\in \antichains P$, which is by definition~$\bot_P$.
%\end{compactitem}
%\end{proof}
%\end{comment}