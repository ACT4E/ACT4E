% !TEX root = standalone.tex


\section{Other examples of functors}

\subsection{The list functor}
In the following, we present a concept related to the monoidal structure introduced.

\todo{This is the free monoid functor and we need the concept of monoid (or at least it is elegant if we have it). This is now at section 21 though...}
\todotext{to write}

\subsection{Planning as the search of a functor}
\begin{example}[ex:planning-as-search-functor]
  Recall the category \Berg introduced in \cref{sec:trekking} and define a category \Plans where objects are specific areas of the mountain and morphisms describing visiting order constraints, illustrated in \cref{fig:visiting_order_constraints}.
  \begin{figure}[h!]
    \begin{center}
      \includesag{095_plans}
    \end{center}
    \caption{Example of visiting order constraints on a mountain.}
    \label{fig:visiting_order_constraints}
  \end{figure}
  For instance, there is a morphism from ``mountain lodge'' to ``panoramic lake'', which desribes the plan of going from the lodge area to the lake area. We call such morphisms \emph{plans}. Plans can be composed via concatenation. For instance, given a plan to go from ``mountain lodge'' to ``panoramic lake'', and a plan to go from ``panoramic lake area'' to ``peak'', their composition is the plan of going from ``mountain lodge'' to the ``peak'', passing through the ``panoramic lake'' (concatenation).

  When we talk about \textbf{planning} in this context, we refer to the action of finding a functor from \Plans to \Berg. Let's look at this in more detail. The objects of \Berg are tuples~$\tup{p,v}$, where~$p$ represent coordinates of a specific location and~$v\in \reals^3$ represents velocities. Morphisms in \Berg are paths that connect locations. For the sake of our planning, we can identify areas of the mountain as sets of locations. Such areas are, for instance, the ``mountain lodge'', ``panoramic lake'', and the
  ``peak'' (note that the ``peak'' represents an area corresponding to a single location). Given some plans as in \cref{fig:visiting_order_constraints}, we want to find a map~$P$ which maps each object in \Plans (area of the mountain) to an object of \Berg (specific location and velocity). Similarly, it must map each morphism in \Plans (visiting order constraints) to a morphism in \Berg (specific paths). This is illustrated in \cref{fig:plans_functor}.
\end{example}

\begin{figure}[h!]
  \centering
  \caption{\todo{to create}}
  \label{fig:plans_functor}
\end{figure}

