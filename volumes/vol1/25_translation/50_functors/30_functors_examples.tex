% !TEX root = chapter-standalone.tex

\section{More examples of functors}

\begin{example}[Powerset functor]
    \label{ex:powerset_functor}
    We consider the powerset \SY{functor}~$\funa\colon \Set \fto \Set$, which maps each set to its power set.
    The action on morphisms is given by:
    \begin{equation}
        \begin{aligned}
            \funamor(\mora)\colon \powersetof \setA & \to \powersetof \setB \\
            \setC                                   & \mapsto \makeset{\mora(\setCel) \mid \setCel \setin \setC}.
        \end{aligned}
    \end{equation}
    Here is a practical example.
    Consider the two sets~$\setA = \makeset{\sfondue, \sbretzel, \schoco}$ and~$\setB = \makeset{\scheese,\sgrapes,\sapple}$.
    By applying the \SY{functor} to~\setA one would have:
    \begin{equation}
        \funaob(\setA)
        =
        \makeset{
            \Emptyset,
            \makeset{\sfondue},
            \makeset{\sbretzel},
            \makeset{\schoco},
            \makeset{\sfondue,\sbretzel},
            \makeset{\sfondue,\schoco},
            \makeset{\sbretzel,\schoco},
            \makeset{\sfondue,\sbretzel,\schoco}
        }.
    \end{equation}
    Furthermore, consider the map
    %
    \begin{equation}
        \begin{aligned}
            \mora \colon \setA & \sto \setB, \\
            \sfondue           & \mapsto \scheese, \\
            \sbretzel          & \mapsto \sgrapes, \\
            \schoco            & \mapsto \sapple.
        \end{aligned}
    \end{equation}
    %
    This would for instance give~$\funamor(\mora)(\makeset{\sfondue,\sbretzel})=\makeset{\mora(\sfondue),\mora(\sbretzel)}=\makeset{\scheese,\sgrapes}$.

    Is the aforementioned structure a functor?
    It is, let's check it explicitly.
    First, consider~$\mora\colon \setA\fto \setB$,~$\morb\colon \setB \fto \setC$.
    First, we have
    \begin{equation}
        \begin{aligned}
            \funa(\morab)(\setC) & =\makeset{ \morb(\mora(\setCel))\mid \setCel \setin \setC},
        \end{aligned}
    \end{equation}
    and
    \begin{equation}
        \begin{aligned}
            (\funa(\mora)\mthen \funa(\morb))(\setC)
             & =\makeset{ \mora(\setCel) \mid \setCel \setin \setC}\mthen \funa(\morb) \\
             & =\makeset{g(\setDel)\mid \setDel \setin \makeset{\mora(\setCel)\mid \setCel\setin \setC}} \\
             & =\makeset{ \morb(\mora(\setCel))\mid \setCel \setin \setC}.
        \end{aligned}
    \end{equation}
    Second, we have
    \begin{equation}
        \begin{aligned}
            \funa(\catidat\setA)(\setC) & = \makeset{\catidat\setA(\setCel)\mid \setCel \setin \setC} \\
                                        & =\catidat{\funa(\setC)}.
        \end{aligned}
    \end{equation}
\end{example}

\begin{example}[Vector spaces]
    \label{exa:double-dual-functor}
    Let~$\VectRs$ be the category whose objects are all real \SY{vector spaces} and whose morphisms are~\reals-linear maps.
    Composition is the usual composition of linear maps.

    There is an \SY{endofunctor}~$\funa \colon \VectRs \mto \VectRs$ whose action on objects is
    \begin{equation}
        \funa ( \vecspB ) = \vecspB^{**}.
    \end{equation}
    (Recall that~$\vecspB^{**} = \makeset{ \text{linear maps } \vecspB^* \mto \reals } =  \HomSet{\VectRs}{\vecspB^*}{\reals}$).
    The action of~$\funa$ on morphisms is as follows.
    Given a linear map~$\mora \colon \vecspB \mto \vecspC$,
    \begin{equation}
        \funa ( \mora ) \colon \vecspB^{**} \mto \vecspC^{**}, \quad \xi \mapsto [ l \mapsto \xi (f \mthen l)].
    \end{equation}
\end{example}

\subsection{The list functor}
In the following, we present a concept related to \SY{monoids}.

We now consider the \SY{functor}
\begin{equation}
    \listfun\colon \Set \fto \Mon
\end{equation}
from the \SY{category of sets} to the \SY{category of monoids}.

Given a set~\setA, the \SY{functor} returns a specific \SY{monoid}
\begin{equation}
    \listfun(\setA)\definedas\tupp{\listsof{\setA},\emptylist, \listconcat}.
\end{equation}

Given a map~$\mapa\colon \setA \sto \setB$, we have
% \begin{equation}
%     \defmapcomma{
%         \listfun(\mapa)
%     }{
%         \listsof{\setA}
%     }{
%         \to
%     }{
%         \listsof{\setB}
%     }{
%         \makelist{\setAel_1, \ldots, \setAel_n}
%     }{
%         \makelist{\mapa(\setAeln{1}), \ldots, \mapa(\setAeln{n})}
%     }
% \end{equation}
\begin{equation}
    \defmapcomma{
        \listfun(\mapa)
    }{
        \listsof{\setA}
    }{
        \to
    }{
        \listsof{\setB}
    }{
        \makegenlist{ \setAel_i }
    }{
        \makegenlist{  \mapa(\setAeln{i}) }
    }
\end{equation}
which applies~$\mapa$ element-wise in the list.
The empty list in~$\listsof{\setA}$ is mapped to the empty list in~$\listsof{\setB}$.
    {}
$\listfun$ is a \SY{functor}, given that the identity map is mapped to the identity map on lists, and that
% \begin{equation}
%     \begin{aligned}
%         \listfun(\mapa \mthen \mapb)(\makelist{\setAeln{1},\ldots,\setAeln{n}}) & =\makelist{(\mapa \mthen \mapb)(\setAeln{1}), \ldots, (\mapa \mthen \mapb)(\setAeln{n})} \\
%                                                                                 & =\listfun(\mapb)(\makelist{\mapa(\setAeln{1}), \ldots, \mapa(\setAeln{n})}) \\
%                                                                                 & =(\listfun(\mapa)\mthen \listfun(\mapb))(\makelist{\setAeln{1}, \ldots, \setAeln{n}}).
%     \end{aligned}
% \end{equation}
\begin{equation}
    \begin{aligned}
         & \listfun(\mapa \mthen \mapb)(\makegenlist{\setAeln{i}}) \\ & =\makegenlist{ (\mapa \mthen \mapb)(\setAeln{i})} \\
         & =\listfun(\mapb)(\makegenlist{\mapa(\setAeln{i})}) \\
         & =(\listfun(\mapa)\mthen \listfun(\mapb))(\makegenlist{\setAeln{i}}).
    \end{aligned}
\end{equation}

\devel{

    \subsection{The elevation functor}

    \begin{figure}
        \includegraphics[width=8cm]{berg_elevation}
    \end{figure}

}

\subsection{Planning as the search of a functor}

\begin{example}
    \label{ex:planning-as-search-functor}
    Recall the category \Berg introduced in \cref{sec:trekking} and define a category \Plans where objects are activities and morphisms describe activities order constraints, illustrated in \cref{fig:plans_functor} (left).

    For instance, there is a morphism from ``mountain lodge'' to ``panoramic lake'', which describes the plan of going from the lodge area to the lake area.
    We call such morphisms \emph{plans}.
    Plans can be composed via concatenation.
    For instance, given a plan to go from ``mountain lodge'' to ``panoramic lake'', and a plan to go from ``panoramic lake'' to ``peak'', their composition is the plan of going from ``mountain lodge'' to the ``peak'', passing through the ``panoramic lake''.

    When we talk about ``planning'' in this context, we refer to the action of finding a \SY{functor}~$\funa$ from \Plans to \Berg.
    % Let's look at this in more detail.
    The objects of \Berg are tuples~$\tup{p,v}$, where~$p$ represent coordinates of a specific location and~$v\setin \reals^3$ represents velocities.
    Morphisms in \Berg are paths that connect locations.
    For the sake of our planning, we can identify areas of the mountain as sets of locations.
    Such areas are, for instance, the ``mountain lodge'', ``panoramic lake'', and the
    ``peak'' (note that the ``peak'' represents an area corresponding to a single location).
    Given some plans as in \cref{fig:plans_functor} (left), we want to find a map~$P$ which maps each object in \Plans (activity) to an object of \Berg (specific location and velocity).
    Similarly, it must map each morphism in \Plans (activity order constraints) to a morphism in \Berg (specific paths).
    This is illustrated in \cref{fig:plans_functor}.
\end{example}

\begin{figure*}[h!]
    \centering
    \includesag{planning_berg}
    \caption{Planning \SY{functor}.}
    \label{fig:plans_functor}
\end{figure*}

\vfill
\clearpage
\vfill

\begin{gradedexercise}[\exname{DoubleDualFunctor}]
    \label{ex:DoubleDualFunctor}
    Prove that $\funa$ as defined in \cref{exa:double-dual-functor} is in fact a \SY{functor}.
\end{gradedexercise}
\solutionof{DoubleDualFunctor}

\begin{marginfigure}
    \centering
    \includesag{graph-cat}
    \caption{}
    \label{fig:graph-cat}
\end{marginfigure}

\begin{gradedexercise}[\exname{GraphsViaFunctors}]
    \label{ex:GraphsViaFunctors}
    Consider the following category, which has two objects,~$\styleobj{V}$ and~$\styleobj{A}$, and four morphisms: besides the \SY{identity morphisms}, there are two morphisms,~$\stylemorph{s}$ and~$\stylemorph{t}$, from~$\styleobj{A}$ to~$\styleobj{V}$.
    See \cref{fig:graph-cat}.
    Call this category~$\CatSymbol{G}$.

    Can you explain the following statement?
    ``Specifying a \SY{functor}~$\CatSymbol{G} \fto \Set$ is the ``same thing'' as specifying a directed graph''.
\end{gradedexercise}
\solutionof{GraphsViaFunctors}

\begin{gradedexercise}[\exname{UpperSetsViaFunctors}]
    \label{ex:UpperSetsViaFunctors}
    Recall that~$\Bool$ denotes the category with two objects,~$\true$ and~$\false$, and with precisely one non-identity morphism which goes from~$\false$ to~$\true$.
    Let~\posA be a poset.
    View it as a category~$\CatSymbol{P}$, and let~$\funa \colon \CatSymbol{P} \mto \Bool$ be a \SY{functor}.
    In other words,~$\funaMap = \funaob$ is a \SY{monotone function}.
    Prove that the set
    \begin{equation}
        \subA \definedas \makeset{ \posAel \setin \posAset \mid \funaMap(\posAel) = \true } \setsubseteq \posA
    \end{equation}
    is an \SY{upper set}.
\end{gradedexercise}
\solutionof{UpperSetsViaFunctors}

\todojira{484}{\alphubel: @Gioele: Show that there is a \SY{functor} from reals to cat, which maps an alpha to the category bergalpha}
