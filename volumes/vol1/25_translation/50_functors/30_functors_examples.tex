% !TEX root = chapter-standalone.tex

\section{More examples of functors}

\begin{example}[Powerset functor]
    \label{ex:powerset_functor}
    We consider the power set functor~$\funa\colon \Set \fto \Set$, which maps each set to its power set.
    The action on morphisms is given by:
    \begin{equation*}
        \begin{aligned}
            \funamor(\mora)\colon \powerset(\setA) & \to \powerset(\setB)                               \\
            \setC                                  & \mapsto \{\mora(\setCel) \mid \setCel \in \setC\}.
        \end{aligned}
    \end{equation*}
    Let's see a practical example.
    For instance, consider the two sets~$\setA = \makeset{\sfondue, \sbretzel, \schoco}$ and~$\setB = \makeset{\scheese,\sgrapes,\sapple}$.
    By applying the functor to~$\setA$ one would have:
    \begin{equation*}
        \funaob(\setA)
        =
        \{
        \{\emptyset\},
        \{\sfondue\},
        \{\sbretzel\},
        \{\schoco\},
        \{\sfondue,\sbretzel\},
        \{\sfondue,\schoco\},
        \{\sbretzel,\schoco\},
        \{\sfondue,\sbretzel,\schoco\}
        \}.
    \end{equation*}
    Furthermore, consider the map
    %
    \begin{equation*}
        \begin{aligned}
            \mora \colon \setA & \to \setB        \\
            \sfondue           & \mapsto \scheese \\
            \sbretzel          & \mapsto \sgrapes \\
            \schoco            & \mapsto \sapple.
        \end{aligned}
    \end{equation*}
    %
    This would for instance give~$\funamor(\mora)(\{\sfondue,\sbretzel\})=\{\mora(\sfondue),\mora(\sbretzel)\}=\{\scheese,\sgrapes\}$.

    Is the aforementioned structure a functor?
    It is, let's check it explicitly.
    First, consider~$\mora\colon \setA\fto \setB$,~$\morb\colon \setB \fto \setC$.
    First, one has
    \begin{equation*}
        \begin{aligned}
            \funa(\mora\mthen \morb)(\setC) & =\{ \morb(\mora(\setCel))\mid \setCel \in \setC\},
        \end{aligned}
    \end{equation*}
    and
    \begin{equation*}
        \begin{aligned}
            (\funa(\mora)\mthen \funa(\morb))(\setC)
             & =\{ \mora(\setCel) \mid \setCel \in \setC\}\mthen \funa(\morb)          \\
             & =\{g(\setDel)\mid \setDel \in \{\mora(\setCel)\mid \setCel\in \setC\}\} \\
             & =\{ \morb(\mora(\setCel))\mid \setCel \in \setC\}.
        \end{aligned}
    \end{equation*}
    Second, one has
    \begin{equation*}
        \begin{aligned}
            \funa(\catid_\setA)(\setC) & = \{\catid_\setA(\setCel)\mid \setCel \in \setC\} \\
                                       & =\catid_{\funa(\setC)}.
        \end{aligned}
    \end{equation*}
\end{example}

\begin{example}[Vector spaces]
    \label{exa:double-dual-functor}
    Let~$\CatC$ be the category whose objects are all real vector spaces and whose morphisms are~$\reals$-linear maps.
    Composition is the usual composition of linear maps.
    \todotextjira{403}{@J: Can you give a specific name to this category?}
    There is an endofunctor~$\funa \colon \CatC \mto \CatC$ whose action on objects is
    \begin{equation}
        \funa ( \vecspB ) = \vecspB^{**}.
    \end{equation}
    (Recall that~$\vecspB^{**} = \{ \text{linear maps } \vecspB^* \mto \reals \} =  \HomSet{\CatC}{\vecspB^*}{\reals}$).
    The action of~$\funa$ on morphisms is as follows.
    Given a linear map~$\mora \colon \vecspB \mto \vecspC$,
    \begin{equation}
        \funa ( \mora ) \colon \vecspB^{**} \mto \vecspC^{**}, \quad \xi \mapsto [ l \mapsto \xi (f \then l)].
    \end{equation}
\end{example}

\begin{gradedexercise}[\exname{DoubleDualFunctor}]
    \label{ex:DoubleDualFunctor}
    Prove that $\funa$ as defined in \cref{exa:double-dual-functor} is in fact a functor.
\end{gradedexercise}
\solutionof{DoubleDualFunctor}

\begin{gradedexercise}[\exname{GraphsViaFunctors}]
    \label{ex:GraphsViaFunctors}
    Consider the following category, which has two objects,~$\styleobj{V}$ and~$\styleobj{A}$, and four morphisms: besides the identity morphisms, there are two morphisms,~$\stylemorph{s}$ and~$\stylemorph{t}$, from~$\styleobj{A}$ to~$\styleobj{V}$.
    See \cref{fig:graph-cat}.
    Call this category~$\Cat{G}$.

    Can you explain the following statement?
    ``Specifying a functor~$\Cat{G} \mto \Set$ is the `same thing' as specifying a directed graph''.
\end{gradedexercise}
\solutionof{GraphsViaFunctors}

\begin{marginfigure}
    \centering
    \includesag{graph-cat}
    \caption{}
    \label{fig:graph-cat}
\end{marginfigure}

\begin{gradedexercise}[\exname{UpperSetsViaFunctors}]
    \label{ex:UpperSetsViaFunctors}
    Recall that~$\Bool$ denotes the category with two objects,~$\true$ and~$\false$, and with precisely one non-identity morphism which goes from~$\false$ to~$\true$.
    Let~$\posA$ be a poset.
    View it as a category~$\Cat{P}$, and let~$\funa \colon \Cat{P} \mto \Bool$ be a functor.
    In other words,~$\funaMap = \funaob$ is a monotone function.
    Prove that the set
    \begin{equation}
        \subA \coloneqq \{ \posAel \in \posA \mid \funaMap(\posAel) = \true \} \subseteq \posA
    \end{equation}
    is an upper set.
\end{gradedexercise}
\solutionof{UpperSetsViaFunctors}

\devel{
    \subsection{The list functor}
    In the following, we present a concept related to the monoidal structure introduced.

    \todojira{119}{
        @Gioele: This is the free monoid functor.
        Recall concept of monoid and add.
    }
}

\devel{
    \subsection{The elevation functor}

    \begin{figure}
        \includegraphics[width=8cm]{berg_elevation}
    \end{figure}

}

\subsection{Planning as the search of a functor}

\begin{example}
    \label{ex:planning-as-search-functor}
    Recall the category \Berg introduced in \cref{sec:trekking} and define a category \Plans where objects are activities and morphisms describe activities order constraints, illustrated in \cref{fig:plans_functor} (left).

    For instance, there is a morphism from ``mountain lodge'' to ``panoramic lake'', which describes the plan of going from the lodge area to the lake area.
    We call such morphisms \emph{plans}.
    Plans can be composed via concatenation.
    For instance, given a plan to go from ``mountain lodge'' to ``panoramic lake'', and a plan to go from ``panoramic lake area'' to ``peak'', their composition is the plan of going from ``mountain lodge'' to the ``peak'', passing through the ``panoramic lake'' (concatenation).

    When we talk about \textbf{planning} in this context, we refer to the action of finding a functor~$\funa$ from \Plans to \Berg.
    Let's look at this in more detail.
    The objects of \Berg are tuples~$\tup{p,v}$, where~$p$ represent coordinates of a specific location and~$v\in \reals^3$ represents velocities.
    Morphisms in \Berg are paths that connect locations.
    For the sake of our planning, we can identify areas of the mountain as sets of locations.
    Such areas are, for instance, the ``mountain lodge'', ``panoramic lake'', and the
    ``peak'' (note that the ``peak'' represents an area corresponding to a single location).
    Given some plans as in \cref{fig:plans_functor} (left), we want to find a map~$P$ which maps each object in \Plans (activity) to an object of \Berg (specific location and velocity).
    Similarly, it must map each morphism in \Plans (activity order constraints) to a morphism in \Berg (specific paths).
    This is illustrated in \cref{fig:plans_functor}.
\end{example}

\begin{figure*}[h!]
    \centering
    \includesag{planning_berg}
    \caption{Planning functor.}
    \todographics{@Gioele: (left side) can we have some space between the arrowheads and the points?
        Also the red does not look good against the image.
        Add a border/shadow to font.
        Also, let's add  a $\funaob$ arrow from ``touch the sky'' to ``peak''.
    }
    \label{fig:plans_functor}
\end{figure*}
