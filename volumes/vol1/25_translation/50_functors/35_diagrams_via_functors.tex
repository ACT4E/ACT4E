% !TEX root = chapter-standalone.tex

\section{Diagrams via functors}

\todojira{368}{@J: Explain how diagrams can be encoded using functors}

So far, we have been thinking of diagrams in a category $\CatC$ as directed graphs, with nodes labeled by objects of $\CatC$ and edges labeled by morphisms of $\CatC$. There is another way of thinking about diagrams, using functors. 

To see how this works, consider the diagram in \cref{fig:diagram-as-functor-example}. We might say that this diagram has a certain ``shape'', which is independent of the specific labels that appear. Namely, the diagram involves three objects, and the arrows between them are arranged in a certain way. The idea is that we can encode this shape abstractly as a category $\CatJ$ and then we think of the diagram in \cref{fig:diagram-as-functor-example} as the ``imprint'' of a functor $\stylefunctors{D}: \CatJ \fto \CatC$. (We call it ``$\stylefunctors{D}$'' for ``diagram''.)

\begin{marginfigure}
    \centering
    \includesag{diagram_as_functor_example}
    \caption{}
    \label{fig:diagram-as-functor-example}
\end{marginfigure}

\begin{marginfigure}
    \centering
    \includesag{shape_diagram_example}
    \caption{}
    \label{fig:shape-diagram-example}
\end{marginfigure}



In the case of the diagram in \cref{fig:diagram-as-functor-example}, the category $\CatJ$ that we want to use looks as in \cref{fig:shape-diagram-example}: we take it to have three objects (which we label $1$, $2$, and $3$, respectively) and, beyond identity morphisms, we let $\CatJ$ have only two other morphisms: one from $2$ to $1$, and one from $2$ to $3$. The functor $\stylefunctors{D}: \CatJ \fto \CatC$ that encodes \cref{fig:diagram-as-functor-example} is specified by 
\begin{align*}
& \stylefunctors{D}(\styleobj{1}) = \Obja, \quad \stylefunctors{D}(\styleobj{2}) = \Objb, \quad  \stylefunctors{D}(\styleobj{3}) = \Objc;  \\
& \stylefunctors{D}(\stylemorph{a}) = \mora, \quad \stylefunctors{D}(\stylemorph{b}) = \morb.
\end{align*}
\todo{insert nice figure showing the shape category being mapped to the diagram}

Note that we cannot think every diagram directly as the ``imprint'' of a functor. For example, consider the diagram in [REF]. Intuitively, if we want to think of it in terms of a functor from a shape category $\CatJ$, then $\CatJ$ should be as in [REF]. However, ... 

\todo{@J: finish writing}