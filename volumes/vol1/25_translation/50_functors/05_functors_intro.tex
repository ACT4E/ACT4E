% !TEX root = chapter-standalone.tex

\section{Layers of abstraction}
\label{sec:functors}

\linkvideo{spring2021-functors:bigpic} % Big picture, zipping

We can think of a given category \CatC as a ``compositional world'': inside of \CatC we have objects, morphisms between them, and a way to talk about composing morphisms.
Now we will zoom out a level, and consider different categories -- different worlds -- simultaneously, and how to relate them to each other.

The most basic notion of how to ``map'' one category to another is given by the concept of a \emph{functor}.

Just like a morphism
% 
\begin{equation}
    \mora \colon \Obja \mto \Objb
\end{equation}
% 
is an arrow between objects in a category, a functor
% 
\begin{equation}
    \funa \colon \CatC \fto \CatD
\end{equation}
% 
is an arrow between two categories.
In fact, we will see in the next chapters that \emph{functors are morphisms in a category of categories}.

Later on, we will see that there is yet another level of generalization: we can define \emph{arrows between functors}, called natural transformations.

% \todotextjira{118}{@Andrea: add introduction here, with several examples of translations}

