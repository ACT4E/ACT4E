% !TEX root = standalone.tex


\section{Interconnection of DPIs}

\todo{Do here the interconnection of two DPIs and their semantics.}

Graphically, one is allowed to connect only edges of different
color, and of the same type. This interconnection is indicated with the symbol~``$\posleq$''
in a rounded box~(\cref{fig:connection}).

\begin{figure}[h]
  \centering
  \includesag{520_dp_unc_conn}
  \caption{\label{fig:connection}}
\end{figure}

%\captionsideleft{\label{fig:connection}}{\includegraphics[scale=0.33]{papers/uncertainty/unc_connection.pdf}}

\noindent The semantics of the interconnection is that the resources
required by the first DPI are provided by the second DPI. This is
a partial order inequality constraint of the type~$\res_{1}\posleq\fun_{2}$.


\begin{definition}[$\dpseries$ composition]
  \label{def:series-composition}The series composition of two DPIs
  ~$\dprob_{1}=\langle\funsp_{1},\ressp_{1},\impsp_{1},\prov_{1},\req_{1}\rangle$
  and~$\dprob_{2}=\langle\funsp_{2},\ressp_{2},\impsp_{2},\prov_{2}$,
  ~$\req_{2}\rangle$, for which~$\funsp_{2}=\ressp_{1}$, is
  \begin{equation}
    \dpseries(\dprob_{1},\dprob_{2})\definedas\left\langle \funsp_{1},\ressp_{2},\impsp,\prov,\req\right\rangle ,
  \end{equation}
  where:
  \begin{eqnarray}
    \impsp & = & \{\left\langle \imp_{1},\imp_{2}\right\rangle \in\impsp_{1}\times\impsp_{2}\mid\req_{1}(\imp_{1})\posleq_{\ressp_{1}}\prov_{2}(\imp_{2})\},\\
    \prov & : & \left\langle \imp_{1},\imp_{2}\right\rangle \mapsto\prov_{1}(\imp_{1}),\\
    \req & : & \left\langle \imp_{1},\imp_{2}\right\rangle \mapsto\req_{2}(\imp_{2}).
  \end{eqnarray}
\end{definition}
\captionsideleft{\label{fig:composition-2}}{
  \includegraphics[scale=0.33]{gmcdp_series3}
}




\todo{Show that composition so defined does not make DPI a category, but it makes it a bicategory (associative
up to isomorphism)}


\todo{discussion about \emph{decomposition is not decoupling}}
