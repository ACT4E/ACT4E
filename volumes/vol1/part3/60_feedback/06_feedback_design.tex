% !TEX root = standalone.tex


\section{Feedback in design problems}
\label{sec:feedbackindesign}
\todo{Motivation}

\subsection{Rockets and toilets}

Consider the case in which you are designing the toilets of a cruise ship. You know that you need a toilet every 10 passengers (i.e., if you have 11 passengers, you need 2 toilets). Furthermore, you know that each toilet needs an employee for its service, i.e. an extra passenger. Now, the problem of maximizing the number of people you can put on the ship, by minimizing the number of toilets you need to install, is a design problem. The resource poset is the one describing the number of toilets needed $\R{n_\mathsf{toilets}}$, and the functionality poset is the one describing the number of people you can accomodate on the ship $\F{n_\mathsf{passengers}}$.
This can also be written diagramatically, as
\begin{center}
    \begin{tikzpicture}[DP]
        \node[dp={2}{1}] (f) {Ship Design};
        \draw[rconn,rcname={$ \R{n_\mathsf{toilets}}\cdot \frac{\mathsf{cleaners}}{\mathsf{toilets}}$},fcname={$\F{n_\mathsf{cleaners}}$},feedback=1,loos=3] (f_res1) -- ($(f)+(0,9)$) |- (f_fun1);
        \draw[funconn, funame={$n_\mathsf{passengers}$}] (f_fun2);
    \end{tikzpicture}
\end{center}
\todo{Don't want to introduce sum and multiplication}


\subsection{Feedback in DPI}

\todo{define here $\dploopb$. Later we define $\dploop$}



\subsection{Feedback in DP}

