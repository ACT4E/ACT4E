% !TEX root = standalone.tex


\section{Monoids}\label{sec:parallelism-monoids}

\begin{definition}[Monoid]
  \label{def:monoid}
  A \index{\emph{monoid}} is a set~$M$ with a binary operation~$\otimes \colon M\times M\to M$, and a \emph{neutral element}~$\idmon\in M$, which satisfy:
  \begin{enumerate}
    \item Associative law:~$(x\otimes y)\otimes z=x\otimes (y\otimes z)$;
    \item Unit Laws:~$\idmon\otimes x=x=x\otimes \idmon$.
  \end{enumerate}
\end{definition}
\begin{example}
  Consider~$\tup{\reals,+,0}$. This is a monoid, since, for all~$x,y\in \reals$, we have:
  \begin{equation*}
  (x+y)
    +z=x+(y+z),
  \end{equation*}
  and
  \begin{equation*}
    x+0=0=0+x.
  \end{equation*}
\end{example}

\begin{example}
  Consider~$\tup{\nonNegReals,\max,0}$. This is a monoid, since, for all~$x,y\in \nonNegReals$, we have:
  \begin{equation*}
    \max(\max(x,y),z)=\max(x,\max(y,z)),
  \end{equation*}
  and
  \begin{equation*}
    \max(x,0)=x=\max(0,x).
  \end{equation*}
\end{example}

\begin{example}
  In this example we look at sequences. A sequence is a function whose domain is a subset of~$\natnumbers$, and are called \emph{finite} if the domain of the function is finite. Often finite sequences are referred to as \emph{lists}. Given a set~$S$, we denote the set of all lists on~$S$ by $S^\ast$. This can be made into a monoid, by considering \emph{concatenation} as the operation, and the empty list as the neutral element. Specifically, a list is an element $s\in S^\ast$ consists of a $n\in \natnumbers$ and a function~$f\colon [n]\to S$, where~$[n]=\{i\colon \natnumbers\mid i<n\}\subseteq \natnumbers$. The empty list, denoted~$()$, is the unique list of length 0. Given~$n>0$, we can write the list which assigns~$0,\ldots,n-1$ to~$s_0,s_1,\ldots,s_{n-1}$ as $(s_0,s_1,\ldots,s_{n-1})$. Given a list~$x\in S^\ast$ of length~$m$ and a list~$y\in S^\ast$ of length~$n$, we can define their \emph{concatenation}~$x*y$ as list of length~$m+n$ with:
  \begin{equation*}
    i\mapsto
    \begin{cases}
      x_i&\text{if }i<m\\
      y_{i-m}&\text{if }i\geq m.
    \end{cases}
  \end{equation*}
  Clearly, this definition of concatenation satisfies associativity and unitality, making this construction a monoid. This is often referred to as the \emph{free monoid on~$S$}.
\end{example}

\begin{example}
  Given any category~\CatC, and any object~$\Obja\in \CatC$, the set of \emph{endomorphisms}~$\Hom_{\CatC}(\Obja,\Obja)$ is a monoid. The category depicted in \cref{fig:monoid_endomorphisms} has three objects~$\Obja,\Objb,\Objc$ and several morphisms.~$\Obja$ has four endomorphisms,~$\Objb$ two, and~$\Objc$ three (including identity morphisms). Let's now take the binary operation~$\otimes$ to be the composition~$\then$ in~\CatC, and the neutral element to be the identity~$\id_\Obja$. The associativity and unitality laws of the category~\CatC coincide with the ones of the monoid's definition, and are satisfied. Therefore, we can identity a monoid as a one-object category.
\end{example}

\begin{figure}[h!]
  \begin{center}
    \includesag{043_monoid_endomorphisms}
    \caption{\label{fig:monoid_endomorphisms}}
  \end{center}
\end{figure}


%\section{Dynamical systems and monoids}

\AC{in the end I would make this only a simple example of monoid - no introduction of group etc.}
\JL{inserting this here as an un-baked idea for a subsection. maybe it could be the first subsection of this chapter; that way idendity laws and associative laws can be introduced before talking about categories}
\gray{
  What are the simplest kinds of mathematical models of a dynamical system that we can think of?

  One possible answer is something like this: we can describe a dynamical system as a set $S$ of possible states, together with a description of how states change over time. For the latter, consider time to be labeled by distinct ``points in time''. Then, we can just think in terms of time-steps, e.g. seconds, or we can think of points in time where e.g. an action is triggered and the system passes to a new state.

  One thing we want to describe is how the state of our system changes over time, and in particular from one moment in time to the next. For any time step, we will not assume that we know what specific state the system is in, but rather we will describe, at once, all possible evolutions during that time step, i.e. we consider all possible initial conditions at once. Given two consecutive moments in time, we might describe the possible changes in the system by a function $T : S \rightarrow S$, which maps each state $s \in S$ to a next state $T(s) \in S$. This is a deterministic change of state: given $s$, the function $T$ determines the next state $T(s)$. The function $T$ is like a rule. Let's call $T$ an ``evolution operator'', because it describes how the system states might evolve over a time step.

  We might want to consider various possible evolution operators. We could consider functions $T_a$, $T_b$, $T_c$, etc. We can also compose these functions: given $T_a$ and $T_b$, we might have, over the course of two time steps, the change described by $T_a \circ T_b$. For simplicity, let's suppose we work with three evolution operations $T_a$, $T_b$, and $T_c$.

  \

  -> introduce semigroups (implicitly or explicitly)

  \


  -> introduce monoids
}

\AC{
  For me the basic example of monoid with dynamical systems is taking the transition function.

  Let $E^s_t: X \to X$ be the evolution function from $s$ to $t$. Then states evolve like this: $x_t = E^s_t (x_s)$.

  If you assume that the system is time invariant, then the evolution only depends on the difference $\delta = s-t$.
  You have now a communative monoid of transition functions $T_\delta$ where $T_0 = \text{identity}$.

  (No need to do semigroups.)

}





