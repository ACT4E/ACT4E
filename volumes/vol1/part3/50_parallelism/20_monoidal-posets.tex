% !TEX root = standalone.tex


\section{Monoidal posets}\label{sec:parallelism-monoidal-posets}

\todo{Move when we discuss monoidal categories as a basic example.}
\begin{definition}[Monoidal poset]
  \label{def:monoidal_poset}
  A \emph{monoidal structure} on a poset~$\tup{P,\ordleq}$ consists of:
  \begin{compactenum}
    \item An element~$\idmon\in P$, called \emph{monoidal unit}, and
    \item a function~$\otimes\colon P\times P\to P$, called the \emph{monoidal product}. Note that we write
    \begin{equation*}
      \otimes(p_1,p_2)=p_1\otimes p_2, \quad p_1,p_2\in P.
    \end{equation*}
  \end{compactenum}
  The constituents must satisfy the following properties:
  \begin{compactenum}[(a)]
    \item \emph{Monotonicity}: For all~$p_1,p_2,q_1,q_2\in P$, if~$p_1\ordleq q_1$ and~$p_2\ordleq q_2$, then
    \begin{equation*}
      p_1\otimes p_2\ordleq q_1\otimes q_2.
    \end{equation*}
    \item \emph{Unitality}: For all~$p\in P$,~$\idmon\otimes p=p$ and~$p\otimes \idmon=p$.
    \item \emph{Associativity}: For all~$p,q,r\in P$, $(p\otimes q)\otimes r=p\otimes (q\otimes r)$.
  \end{compactenum}
  A poset equipped with a monoidal structure $\tup{P,\ordleq,\idmon,\otimes}$ is called a \emph{monoidal poset}.
\end{definition}

\begin{example}
  \label{ex:monoidal_pos_reals}
  Consider the real numbers~$\reals$ with the poset structure given the usual ordering. Consider 0 as monoidal unit and the operation~$+\colon \reals\times \reals\to \reals$ as mononidal product. It is easy to see that the conditions of~\cref{def:monoidal_poset} are satisfied:
  \begin{compactenum}[(a)]
    \item If~$p_1\ordleq p_2$ and $q_1\ordleq q_2$, it is true that~$p_1+p_2\ordleq q_1+q_2$,~$\forall p_1,p_2,q_1,q_2\in \reals$.
    \item $0+p=p+0=0$,~$\forall p\in \reals$.
    \item $(p+q)+r=p+(q+r)$,~$\forall p,q,r\in \reals$.
  \end{compactenum}
\end{example}

\begin{example}
  Someone proposes now to substite the monoidal unit in \cref{ex:monoidal_pos_reals} with 1 and the monoidal product with ``*''. This does not form a monoidal poset anymore. To see a simple counterexample, consider the fact that~$-5\leq 0$ and~$-4\leq 3$. However,~$(-5)\cdot (-4) \nleq 0 \cdot 3$.
\end{example}

\begin{example}
  Consider now~$\tup{\Bool,\leq_{\Bool},\true,\wedge}$. The action of the monoidal product ``$\wedge$'' can be summarized in a table:
  \begin{center}
    \begin{tabular}{c|cc}
      $\wedge$ & $\false$ & $\true$  \\
      \hline
      $\false$ & $\false$ & $\false$ \\
      $\true$  & $\false$ & $\true$
    \end{tabular}
  \end{center}
  From this table, it is clear that given~$x_1\leq_{\Bool}y_1$ and~$x_2\leq_{\Bool} y_2$, one has~$x_1\wedge x_2\leq_{\Bool} y_1\wedge y_2$ (if you do not believe it, try all possible combinations). Furthermore,~$x\wedge \true=x=\true \wedge x$.
\end{example}
