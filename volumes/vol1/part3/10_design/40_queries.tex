\section{Queries in design}

Suppose that we have a model with a functionality space~\funsp, a requirements space~\ressp, and an implementation space~$\impsp$.

There are several queries we can ask of a model. They all look at the same phenomenon from different angles, so they look similar; however the computational cost of answering each one might be very different.

The first kind of query is one that asks if the design if feasible when fixed all variables

\begin{problem}[Feasibility problem]
  Given a triplet of implementation~$\imp\in\impsp$, functionality~$\fun\in\funsp$, requirements~$\res\in\ressp$, determine if the design is feasible.
\end{problem}

The second type of query is that which fixes the boundary conditions of functionality and requirements, and asks to find a solution.

\begin{problem}[Find implementation]
  Given a pair of minimal requested functionality~$\fun\in\funsp$ and maximum allowed requirements~$\res\in\ressp$, determine if there is a an implementation~$\imp\in\impsp$ that is feasible.
\end{problem}

A different type of query is the one in which the design objective (the functionality)
is fixed, and we ask what are the least resources necessary.


\begin{problem}[FixFunMinReq]
  Given a certain functionality~$\fun\in\funsp$, find the set of ``minimal'' resources in~\ressp that are needed to realize it (along with the implementations), or provide a proof that there are none.
\end{problem}


Dually, we can ask, fixed the resources available, what are the functionalities that can be required.

\begin{problem}[FixReqMinFun]
  Given a certain requirement~$\res\in\ressp$, find the set of ``maximal'' functionalities in that can be realize it (along with the implementations), or provide a proof that there are none.
\end{problem}

It is very natural to talk about the ``minimal'' requirements and ``maximal'' functionalities;
after all, we always want to minimize costs and maximize performance. In the next chapter
we start to put more mathematical scaffolding in place, starting from definining functionality
and requirements as posets.

% The design queries we will present throughout this paper are the following:
% \begin{item}
% \item

% \item
% \item Given certain resources, find the maximal functionality that can be realized, or provide a proof that there are none.
% \end{item}
