% !TEX root = chapter-standalone.tex

\section[\DP is symmetric monoidal]{\DP is a symmetric monoidal category}
\label{sec:parallelism-DP-monoidal}

\linkvideo{spring2021-functorial-comp-b:solving-queries:solving-parallel} % Parallel composition
%
%
\begin{example}
    After the X101 spontaneously combusted in low Earth orbit, the astronauts at Jeb's Spaceship Parts go on strike.
    They demand that the engineers take into account safety and living conditions on the future X102.
    As long as the propulsion and life support systems of the X102 do not interact, we can simply tensor the two design problems representing these systems into one, big co-design problem (\cref{fig:examplemonoidal}).
    \begin{figure}[h!]
        \centering
        \includesag{50_engine_tensor_1_2}
        \caption{Example of tensor of design problems. }
        \label{fig:examplemonoidal}
    \end{figure}
\end{example}
In~\DP, putting two design problems in parallel corresponds to their \emph{monoidal product}.

\begin{definition}[Monoidal product in~\DP]
    \label{def:monoidalproduct}
    Given two design problems~$\adpa \colon \funposA \profto \resposB$ and~$\adpb \colon \funposC \profto \resposD$, their \emph{monoidal product}~$\adpa \mtimescat \adpb \colon \funposA \cartprod \funposC \profto \resposB\cartprod \resposD$ is their conjunction:
    \begin{equation}
        \label{eq:monoidalprod_dp}
        \begin{aligned}
            \adpa \mtimescat \adpb \colon (\funposA \cartprod \funposC)\op \cartprod (\resposB \cartprod \resposD) & \toinPos \Bool,                                                                 \\
            \tup{\tup{\funposAel,\funposCel}\Fop,\tup{\resposBel,\resposDel}}                                      & \mapsto \adpa(\funposAopel,\resposBel) \booland \adpb(\funposCopel,\resposDel).
        \end{aligned}
    \end{equation}
\end{definition}
The diagrammatic representation of the monoidal product is reported in~\cref{fig:dpmonoidal}.

\todographicsjira{356}{I don't think we use this notation with the explicit $\mtimescat$ at all.
    Rather the outer most box should be dotted, do not have the $A\cartprod C$ or $B\cartprod D$
    or $\mtimescat$ in the middle.
    Also, it should be coherent with \cref{fig:examplemonoidal}.
}

\begin{figure}[h!]
    \centering
    \includesag{50_monoidal}
    \caption{Monoidal product of design problems.}
    \label{fig:dpmonoidal}
\end{figure}

\begin{remark}
    For~$\adpa \colon \funposA\profto \resposB$ and~$\adpb \colon \funposC \profto \resposD$, the monoidal product
    \begin{equation}
        \left(\adpa \mtimescat \adpb\right)(\tup{\funposAopel, \funposCopel}, \tup{\resposBel,\resposDel})
    \end{equation}
    is true if \emph{both}~$\adpa(\funposAopel,\resposBel)$ and~$\adpb(\funposCopel,\resposDel)$ are true, and false otherwise.
\end{remark}

\begin{lemma}
    \label{lem:monoidal_functorial}
    The monoidal product~$\mtimescat$ is functorial (\cref{def:functor}) in~\DP.
\end{lemma}
\begin{proof}
    First, consider posets~$\posA,\posB\in \Ob_\DP$.
    We show that~$\catid_\posA\mtimescat \catid_\posB = \catid_{\posA\cartprod \posB}$
    It holds
    \begin{equation}
        \begin{aligned}
            \left( \catid_{\posA}\mtimescat \catid_{\posB}\right)
            \left( \tup{\F{\posgenAel_1},\F{\posgenBel_1}}\Fop,\tup{\R{\posgenAel_2},\R{\posgenBel_2}}\right) & =
            \catid_\posgenA(\F{\posgenAel_1^*},\R{\posgenAel_2})\booland \catid_\posgenB(\F{\posgenBel_1^*},\R{\posgenBel_2})                                                                                                                        \\
                                                                                                              & =\left( \F{\posgenAel_1}\posleqof\posA \R{\posgenAel_2}\right)\booland \left( \F{\posgenBel_1}\posleqof\posB \R{\posgenBel_2}\right) \\
                                                                                                              & =\tup{\F{\posgenAel_1},\F{\posgenBel_1}}\posleqof{\posA\cartprod \posB}\tup{\R{\posgenAel_2},\R{\posgenBel_2}}                       \\
                                                                                                              & =\catid_{\posA\cartprod \posB}\left( \tup{\F{\posgenAel_1},\F{\posgenBel_1}}^*,\tup{\R{\posgenAel_2},\R{\posgenBel_2}}\right).
        \end{aligned}
    \end{equation}
    Furthermore, consider the design problems
    \begin{equation*}
        \adpa\colon \funposA_1 \profto \resposB_1,\quad \adpb\colon \funposA_2\profto \resposB_2, \quad \adpc\colon \funposB_1\profto \resposC_1,\quad \adpd\colon \funposB_2\profto \resposC_2.
    \end{equation*}
    We need to show that
    \begin{equation}
        \underbrace{\left( (\adpa\mthen \adpc) \mtimescat (\adpb\mthen \adpd)\right)}_{\star}=\left( (\adpa\mtimescat \adpb)\mthen (\adpc\mtimescat \adpd)\right).
    \end{equation}
    It holds
    \begin{equation}
        \begin{aligned}
             & \star \left( \tup{\F{\posgenAel_1},\F{\posgenAel_2}}\Fop,\tup{\R{\posgenCel_1},\R{\posgenCel_2}}\right)=
            (\adpa\mthen \adpc)(\F{\posgenAel_1}\Fop,\R{\posgenCel_1})\booland (\adpa\mthen \adpd)(\F{\posgenAel_2}\Fop,\R{\posgenCel_2})                                                                                                                                                                                                                                                   \\
             & =\left(\bigvee_{\posBel_\styleelements{1}\in \posB_\stylepos{1}}\left( \adpa(\F{\posgenAel_1^*},\R{\posgenBel_1})\booland \adpc(\F{\posgenBel_1^*},\R{\posgenCel_1})\right)\right) \booland\left(\bigvee_{\posBel_\styleelements{2}\in \posB_\stylepos{2}}\left( \adpb(\F{\posgenAel_2^*},\R{\posgenBel_2})\booland \adpd(\F{\posgenBel_2^*},\R{\posgenCel_2})\right)\right) \\
             & =\bigvee_{\posBel_\styleelements{1}\in \posB_\stylepos{1}}\bigvee_{\posBel_\styleelements{2}\in \posB_\stylepos{2}} \left(\adpa(\F{\posgenAel_1^*},\R{\posgenBel_1})\booland \adpc(\F{\posgenBel_1^*},\R{\posgenCel_1})\booland \adpa(\F{\posgenAel_2^*},\R{\posgenBel_2})\booland \adpd(\F{\posgenBel_2^*},\R{\posgenCel_2}) \right)                                        \\
             & =\bigvee_{\tup{\posBel_\styleelements{1},\posBel_\stylepos{2}}\in \posB_\stylepos{1}\cartprod \posB_\stylepos{2}} \left(\adpa(\F{\posgenAel_1^*},\R{\posgenBel_1})\booland \adpb(\F{\posgenAel_2^*},\R{\posgenBel_2})\booland \adpc(\F{\posgenBel_1^*},\R{\posgenCel_1})\booland \adpd(\F{\posgenBel_2^*},\R{\posgenCel_2}) \right)                                          \\
             & =\left( (\adpa\mtimescat \adpb)\mthen (\adpc\mtimescat \adpd)\right)\left(\tup{\F{\posgenAel_1},\F{\posgenAel_2}}\Fop,\tup{\R{\posgenCel_1},\R{\posgenCel_2}} \right).
        \end{aligned}
    \end{equation}
    Therefore,~$\mtimescat$ is functorial in~\DP.
\end{proof}

\begin{lemma}
    $\tup{\DP,\mtimescat, \singleton}$ is a monoidal category.
\end{lemma}
\begin{proof}
    To show that~\DP is a monoidal category, we have to first identify the constituents presented in \cref{def:monoidal-cat}.
    First, recall~$\singleton$ to be singleton: this is the monoidal unit.
    In \cref{lem:monoidal_functorial} we have shown that~$\mtimescat$ is a functor.
    Furthermore, we identify
    \begin{itemize}
        \item $\leftunitor_\posA \colon \F{\singleton} \cartprod \funposA \profto \resposA$, for all~$\posA\in \Ob_\DP$, is the left unitor.
              This is given by
              \begin{equation}
                  \leftunitor_\posA\left( \disunionA{\F{\posgenAel_1}}\Fop,\R{\posgenAel_2}\right)\definedas \F{\posgenAel_1}\posleqof\posA \R{\posgenAel_2}.
              \end{equation}
              To prove that this is an isomorphism, we define its inverse~$\leftunitor_\posA^{-1}\colon \funposA\profto \R{\singleton} \cartprod \resposA$ and show that~$\leftunitor_\posA\mthen \leftunitor_\posA^{-1}=\id_{\singleton\cartprod \posA}$ and~$\leftunitor_\posA^{-1}\then \leftunitor_\posA=\catid_{\posA}$.
              One has
              \begin{equation}
                  \begin{aligned}
                      \left( \leftunitor_\posA^{-1}\mthen \leftunitor_\posA\right)(\tup{\F{\posgenAel_1^*},\R{\posgenAel_2}}) & =
                      \bigvee_{\tup{\singletonel,\posAel}\in  \singleton\cartprod \posA} \leftunitor_\posA^{-1}(\F{\posgenAel_1^*},\tup{\R{\singletonel},\R{\posgenAel}})\booland \leftunitor_\posA(\tup{\F{\singletonel},\FposgenAel}\Fop,\R{\posgenAel_2})                              \\
                                                                                                                              & = \bigvee_{\tup{\singletonel,\posAel}\in  \singleton\cartprod \posA}(\F{\posgenAel_1}\posleq \R{\posgenAel}) \booland \FposgenAel\posleq \R{\posgenAel_2} \\
                                                                                                                              & =\F{\posgenAel_1}\posleq \R{\posgenAel_2}                                                                                                                 \\
                                                                                                                              & =\id_\posA(\F{\posgenAel_1^*},\R{\posgenAel_2}).
                  \end{aligned}
              \end{equation}
              Similarly, one can show that~$\leftunitor_\posA\mthen \leftunitor_\posA^{-1}=\id_{\singleton \cartprod \posA}$.
        \item $\rightunitor_\posA\colon \funposA \cartprod \R{\singleton} \profto \resposA$, for all~$\posA\in \Ob_\DP$, is the right unitor.
              This is given by
              \begin{equation}
                  \rightunitor\left( \tup{\F{\posgenAel_1},\F{\singletonel}}\Fop,\R{\posgenAel_2}\right)\definedas \F{\posgenAel_1}\posleqof\posgenA \R{\posgenAel_2}.
              \end{equation}
              The proof that~$\rightunitor_\posgenA$ is an isomorphism is analogous to the one for~$\leftunitor_\posgenA$.
        \item $\associator_{\posgenA,\posgenB,\posgenC}\colon (\funposA\cartprod \funposB)\cartprod \funposC \profto \resposA\cartprod (\resposB\cartprod \resposC)$ for all~$\posgenA,\posgenB,\posgenC \in \Ob_\DP$, is the associator.
              It is given by
              \begin{equation}
                  \begin{aligned}
                      \associator_{\posA,\posB,\posC} & (\tup{\tup{\F{\posgenAel_1},\F{\posgenBel_1}},\F{\posgenCel_1}}\Fop,\tup{\R{\posgenAel_2},\tup{\R{\posgenBel_2},\R{\posgenCel_2}}})                                                 \\
                                                      & \definedas (\F{\posgenAel_1}\posleqof\posA \R{\posgenAel_2}) \booland (\F{\posgenBel_1} \posleqof\posB \R{\posgenBel_2})\booland (\F{\posgenCel_1}\posleqof\posC \R{\posgenCel_2}).
                  \end{aligned}
              \end{equation}
              To prove that~$\associator_{\posA,\posB,\posC}$ is an isomorphism, we first define its inverse
              \begin{equation}
                  \associator_{\posA,\posB,\posC}^{-1}\colon \funposA\cartprod (\funposB\cartprod \funposC) \profto (\resposA\cartprod \resposB)\cartprod \resposC
              \end{equation}
              and show~$\associator_{\posA,\posB,\posC}^{-1}\mthen \associator_{\posA,\posB,\posC}=\associator_{\posA,\posB,\posC}\mthen \associator_{\posA,\posB,\posC}^{-1}= \catid_{\posA\cartprod \posB\cartprod \posC}$.
              One has
              \begin{equation}
                  \begin{aligned}
                       & \left( \associator_{\posA,\posB,\posC}^{-1}\mthen \associator_{\posA,\posB,\posC} \right)(\tup{\F{\posgenAel_1},\tup{\F{\posgenBel_1},\F{\posgenCel_1}}}\Fop,\tup{\R{\posgenAel_2},\tup{\R{\posgenBel_2},\R{\posgenCel_2}}})                           \\
                       & =\bigvee_{\tup{\tup{\posAel,\posBel},\posCel}\in (\posA\cartprod \posB)\cartprod \posC}
                      \associator_{\posA,\posB,\posC}^{-1}(\tup{\F{\posgenAel_1},\tup{\F{\posgenBel_1},\F{\posgenCel_1}}}\Fop,\tup{\tup{\R{\posgenAel},\RposgenBel},\R{\posgenCel}})\booland                                                                                    \\
                       & \associator_{\posA,\posB,\posC}(\tup{\tup{\FposgenAel,\FposgenBel},\F{\posgenCel}}\Fop,\tup{\R{\posgenAel_2},\tup{\R{\posgenBel_2},\R{\posgenCel_2}}})                                                                                                 \\
                       & =\bigvee_{\tup{\tup{\posAel,\posBel},\posCel}\in (\posA\cartprod \posB)\cartprod \posC}\left( (\F{\posgenAel_1}\posleq \R{\posgenAel}) \booland (\F{\posgenBel_1}\posleq \RposgenBel) \booland (\F{\posgenCel_1}\posleq \R{\posgenCel})\right)\booland \\
                       & \left((\FposgenAel\posleq \R{\posgenAel_2})\booland (\FposgenBel\posleq \R{\posgenBel_2}) \booland (\F{\posgenCel}\posleq \R{\posgenCel_2}\right)                                                                                                      \\
                       & =(\F{\posgenAel_1}\posleq \R{\posgenAel_2}) \booland (\F{\posgenBel_1}\posleq \R{\posgenBel_2}) \booland (\F{\posgenCel_1}\posleq \R{\posgenCel_2})                                                                                                    \\
                       & =\catid_{\posA\cartprod \posB\cartprod \posC}(\tup{\F{\posgenAel_1},\F{\posgenBel_1},\F{\posgenCel_1}}\Fop,\tup{\R{\posgenAel_2},\R{\posgenBel_2},\R{b_3}}).
                  \end{aligned}
              \end{equation}
              The proof for~$\associator_{\posA,\posB,\posC}\mthen \associator_{\posA,\posB,\posC}^{-1}$ is analogous.
    \end{itemize}
    Therefore,~$\tup{\DP,\mtimescat, \singleton}$ is a monoidal category.
\end{proof}

\subsubsection*{\DP is a symmetric monoidal category}
\begin{lemma}
    \label{lem:symmetricmonoidaldp}
    For any~$\posA,\posB \in\Ob_\DP$, the design problem~$\braiding_{\posA,\posB}\colon \F{\posgenA} \cartprod \F{\posgenB} \profto \R{\posgenB} \cartprod \R{\posgenA}$ given by
    \begin{equation}
        \braiding_{\posA,\posB}(\tup{\F{\posgenAel_1},\F{\posgenBel_1}}\Fop,\tup{\R{\posgenBel_2},\R{\posgenAel_2}})\definedas \left(\F{\posgenAel_1}\posleqof\posA \R{\posgenAel_2}\right)\booland \left(\F{\posgenBel_1}\posleqof\posB \R{\posgenBel_2}\right)
    \end{equation}
    constitutes the braiding operation for a symmetric monoidal structure on~$\tupp{\DP,\mtimescat,\singleton}$.
    In other words,~$\tup{\DP, \mtimescat, \singleton, \braiding}$ is a symmetric monoidal category.
\end{lemma}

\begin{proof}
    In this proof, given two elements~$\posAel_1,\posBel_2$ of a poset~$\posA$, we denote for brevity~$\posAel_\styleelements{1} \posleqof\posA \posAel_\styleelements{2}$ by~$\posAel_\styleelements{1} \posleq \posAel_\styleelements{2}$.
    To prove that~$\braiding_{\posA,\posB}$ is an isomorphism, we use \cref{def:monoidal-cat} and show~$\braiding_{\posA,\posB}\then \braiding_{\posB,\posA}=\id_{\posA\cartprod \posB}$.
    One has
    \begin{equation}
        \begin{aligned}
             & \left( \braiding_{\posA,\posB}\mthen \braiding_{\posB,\posA}\right) \left( \tup{\F{\posgenAel_1},\F{\posgenBel_1}}\Fop,\tup{\R{\posgenAel_2},\R{\posgenBel_2}}\right)                                                                                                    \\
             & =\bigvee_{\tup{\posBel,\posAel}\in \posB\cartprod \posA}\braiding_{\posA,\posB}(\tup{\F{\posgenAel_1},\F{\posgenBel_1}}\Fop,\tup{\RposgenBel,\R{\posgenAel}})\booland \braiding_{\posB,\posA}(\tup{\FposgenBel,\FposgenAel}\Fop,\tup{\R{\posgenAel_2},\R{\posgenBel_2}}) \\
             & =\left( (\F{\posgenAel_1}\posleq \R{\posgenAel}) \booland (\F{\posgenBel_1}\posleq \RposgenBel)\right)\booland \left((\FposgenAel\posleq \R{\posgenAel_2}) \booland (\FposgenBel\posleq \R{\posgenBel_2})\right)                                                         \\
             & =(\F{\posgenAel_1}\posleq \R{\posgenAel_2})\booland (\F{\posgenBel_1}\posleq \R{\posgenBel_2})                                                                                                                                                                           \\
             & =\id_{\posA\cartprod \posB}(\tup{\F{\posgenAel_1},\F{\posgenBel_1}}\Fop,\tup{\R{\posgenAel_2},\R{\posgenBel_2}}).
        \end{aligned}
    \end{equation}
    This also shows the second triangle identity:~$\braiding_{\posA,\posB}$ is its own identity.
    For naturality, let's consider two morphisms (design problems)~$\adpa\colon \funposA_\F{1}\profto \resposB_\R{1}$ and $\adpb\colon \funposA_\F{2}\profto \resposB_\R{2}$.
    For brevity, denote~$\braiding_{\posB_\stylepos{1}\cartprod \posB_\stylepos{2},\posB_\stylepos{2}\cartprod \posB_\stylepos{1}}$ by~$\braiding_\posB$ and~$\braiding_{\posA_\stylepos{1}\cartprod \posA_\stylepos{2},\posA_\stylepos{2}\cartprod \posA_\stylepos{1}}$ by~$\braiding_\posA$.
    One has
    \begin{equation}
        \begin{aligned}
             & \left((\adpa\mtimescat \adpb)\mthen \braiding_\posgenB \right)\left( \tup{\F{\posgenAel_1},\F{\posgenAel_2}}\Fop,\tup{\R{\posgenBel_2},\R{\posgenBel_1}}\right)                                                                                                                                                                                       \\
             & =\bigvee_{\tup{\posBel,\posBel\styleelements{'}}\in \posB_\stylepos{1}\cartprod \posB_\stylepos{2}} \left(\adpa\mtimescat \adpb\right) \left( \tup{\F{\posgenAel_1},\F{\posgenAel_2}}\Fop,\tup{\RposgenBel,\R{\posgenBel'}}\right)\booland \braiding_\posB\left(\tup{\FposgenBel,\F{\posgenBel'}}\Fop,\tup{\R{\posgenBel_2},\R{\posgenBel_1}} \right) \\
             & =\bigvee_{\tup{\posBel,\posBel\styleelements{'}}\in \posB_\stylepos{1}\cartprod \posB_\stylepos{2}}(\adpa(\F{\posgenAel_1^*},\RposgenBel)\booland \adpb(\F{\posgenAel_2^*},\R{\posgenBel'}))\booland (\left(\FposgenBel\posleq \R{\posgenBel_1}\right) \booland \left(\F{\posgenBel'}\posleq \R{\posgenBel_2}\right))                                 \\
             & = \adpa(\F{\posgenAel_1^*},\R{\posgenBel_1}) \booland \adpb(\F{\posgenAel_2^*},\R{\posgenBel_2}),
        \end{aligned}
    \end{equation}
    where the last step comes from the monotonicity of~$\adpa$ and~$\adpb$.
    Similarly,
    \begin{equation}
        \begin{aligned}
             & \left( \braiding_\posA \mthen (\adpb\mtimescat \adpa)\right)\left( \tup{\F{\posgenAel_1},\F{\posgenAel_2}}\Fop,\tup{\R{\posgenBel_2},\R{\posgenBel_1}}\right)                                                                                                                                                                                              \\
             & =\bigvee_{\tup{\posAel,\posAel\styleelements{'}}\in \posA_\stylepos{2}\cartprod \posA_\stylepos{1}}\braiding_\posgenA\left(\tup{\F{\posgenAel_1},\F{\posgenAel_2}}\Fop,\tup{\R{\posgenAel},\R{\posgenAel}'} \right)\booland \left(\adpb\mtimescat \adpa\right) \left( \tup{\FposgenAel,\F{\posgenAel'}}\Fop,\tup{\R{\posgenBel_2},\R{\posgenBel_1}}\right) \\
             & =\bigvee_{\tup{\posAel,\posAel\styleelements{'}}\in \posA_\stylepos{2}\cartprod \posA_\stylepos{1}}(\left(\F{\posgenAel_1}\posleq \R{\posgenAel'}\right)\booland \left(\F{\posgenAel_2}\posleq \R{\posgenAel}\right)) \booland (\adpb(\FposgenAelop,\R{\posgenBel_2})\booland \adpa(\F{\posgenAel'^*},\R{\posgenBel_1}))                                   \\
             & = \adpb(\F{\posgenAel_1^*},\R{\posgenBel_1}) \booland \adpa(\R{\posgenAel_2^*},\R{\posgenBel_2}).
        \end{aligned}
    \end{equation}
    To show the first triangle identity, we write
    \begin{equation}
        \begin{aligned}
             & \left(\braiding_{\singleton \cartprod \posA}\mthen \rightunitor_\posA\right)\left( \tup{\F{\singletonel},\F{\posgenAel_1}}\Fop,\R{\posgenAel_2}\right)                                                                                                                                                                   \\
             & =\bigvee_{\tup{\F{\posgenAel'},\R{\singletonel}}\in \posA\cartprod \singleton}\braiding_{\singleton \cartprod \posA}\left( \tup{\F{\singletonel},\F{\posgenAel_1}}\Fop,\tup{\R{\posgenAel'},\R{\singletonel}}\right)\booland \rightunitor_\posA\left( \tup{\F{\posgenAel'},\F{\singletonel}}\Fop,\R{\posgenAel_2}\right) \\
             & =\bigvee_{\tup{\posAel\styleelements{'},\singletonel}\in \posA\cartprod \singleton} \left(\F{\singletonel}\posleq \R{\singletonel}\right) \booland \left(\F{\posgenAel_1}\posleq \R{\posgenAel'}\right)\booland \left(\F{\posgenAel'}\posleq \R{\posgenAel_2}\right)                                                     \\
             & =\F{\posgenAel_1}\posleq \R{\posgenAel_2}                                                                                                                                                                                                                                                                                \\
             & =\leftunitor_\posA\left( \tup{\F{\singletonel},\F{\posgenAel_1}}\Fop,\R{\posgenAel_2}\right).
        \end{aligned}
    \end{equation}
    The hexagon identities are more verbose.
    Consider~$\posA,\posB,\posC\in \Ob_\DP$.
    For brevity, we denote~$\associator_{\posA,\posB,\posC}$ by~$\associator$,~$\braiding_{\posA,\posB}\mtimescat \id_\posC$ by~$\braiding'$,~$\id_\posB \mtimescat \braiding_{\posA,\posC}$ by~$\braiding''$,~$(\posB\cartprod \posA)\cartprod \posC$ as~$\Diamond$, and~$\posB\cartprod (\posA\cartprod \posC)$ as~$\Delta$.

    Recall that
    \begin{equation}
        \begin{aligned}
            \braiding' \left(\tup{\tup{\F{\posgenAel_1},\F{\posgenBel_1}},\F{\posgenCel_1}}\Fop,\tup{\R{\posgenBel_2},\tup{\R{\posgenAel_2},\R{\posgenCel_2}}} \right) & =
            \left( (\F{\posgenAel_1}\posleq \R{\posgenAel_2})  \booland (\F{\posgenBel_1}\posleq \R{\posgenBel_2})\right)\booland (\F{\posgenCel_1}\posleq \R{\posgenCel_2})                                                                                                                                                    \\
                                                                                                                                                                       & = (\F{\posgenAel_1}\posleq \R{\posgenAel_2})  \booland (\F{\posgenBel_1}\posleq \R{\posgenBel_2}) \booland (\F{\posgenCel_1}\posleq \R{\posgenCel_2}).
        \end{aligned}
    \end{equation}
    One has
    \begin{equation}
        \begin{aligned}
             & \left(\braiding' \mthen \associator \right) \left(\tup{\tup{\F{\posgenAel_1},\F{\posgenBel_1}},\F{\posgenCel_1}}\Fop,\tup{\R{\posgenBel_2},\tup{\R{\posgenAel_2},\R{\posgenCel_2}}} \right)                                                                                                                                                                         \\
             & =\bigvee_{\tup{\tup{\posBel,\posAel},\posCel}\in \Diamond}\braiding' \left( \tup{\tup{\F{\posgenAel_1},\F{\posgenBel_1}},\F{\posgenCel_1}}\Fop,\tup{\tup{\RposgenBel,\R{\posgenAel}},\R{\posgenCel}}\right)\booland \associator \left( \tup{\tup{\FposgenBel,\FposgenAel},\F{\posgenCel}}\Fop,\tup{\R{\posgenBel_2},\tup{\R{\posgenAel_2},\R{\posgenCel_2}}}\right) \\
             & =\bigvee_{\tup{\tup{\posBel,\posAel},\posCel}\in \Diamond} \left(\left(\F{\posgenAel_1}\posleq \R{\posgenAel} \right)\booland \left( \F{\posgenBel_1}\posleq \RposgenBel\right)\booland \left(\F{\posgenCel_1}\posleq \R{\posgenCel}\right)\right)\booland                                                                                                          \\
             & \left(\left(\FposgenBel\posleq \R{\posgenBel_2}\right)\booland \left( \FposgenAel\posleq \R{\posgenAel_2}\right)\booland \left(\F{\posgenCel}\posleq \R{\posgenCel_2}\right)\right)                                                                                                                                                                                 \\
             & =\left(\F{\posgenBel_1}\posleq \R{\posgenBel_2} \right)\booland \left(\F{\posgenAel_1}\posleq \R{\posgenAel_2} \right)\booland \left( \F{\posgenCel_1}\posleq \R{\posgenCel_2}\right)                                                                                                                                                                               \\
             & =\underbrace{\associator\left(\tup{\tup{\F{\posgenBel_1},\F{\posgenAel_1}},\F{\posgenCel_1}}^*,\tup{\R{\posgenBel_2},\tup{\R{\posgenAel_2},\R{\posgenCel_2}}}\right)}_{\star}.
        \end{aligned}
    \end{equation}
    Furthermore, consider
    \begin{equation}
        \begin{aligned}
             & \left( \star \mthen \braiding''\right)\left( \tup{\tup{\F{\posgenAel_1},\F{\posgenBel_1}},\F{\posgenCel_1}}\Fop,\tup{\R{\posgenBel_3},\tup{\R{\posgenCel_3},\R{\posgenAel_3}}}\right)                                                                                                                                  \\
             & =\bigvee_{\tup{\posBel_\styleelements{2},\tup{\posAel_\styleelements{2},\posCel_\styleelements{2}}}\in \Delta} \star \left(\tup{\tup{\F{\posgenAel_1},\F{\posgenBel_1}},\F{\posgenCel_1}}\Fop, \tup{\R{\posgenBel_2},\tup{\R{\posgenAel_2},\R{\posgenCel_2}}} \right)\booland                                          \\
             & \braiding'' \left(\tup{\F{\posgenBel_2},\tup{\F{\posgenAel_2},\F{\posgenCel_2}}}^*,\tup{\R{\posgenBel_3},\tup{\R{\posgenCel_3},\R{\posgenAel_3}}} \right)                                                                                                                                                              \\
             & =\bigvee_{\tup{\posgenBel_\styleelements{2},\tup{\posgenAel_\styleelements{2},\posgenCel_\styleelements{2}}}\in \Delta}(\left(\F{\posgenBel_1}\posleq \R{\posgenBel_2} \right)\booland \left(\F{\posgenAel_1}\posleq \R{\posgenAel_2} \right)\booland \left( \F{\posgenCel_1}\posleq \R{\posgenCel_2}\right)) \booland \\
             & \left(\left(\F{\posgenBel_2}\posleq \R{\posgenBel_3}\right)\booland \left(\F{\posgenAel_2}\posleq \R{\posgenAel_3}\right) \booland \left(\F{\posgenCel_2}\posleq \R{\posgenCel_3}\right)\right)                                                                                                                        \\
             & =(\F{\posgenAel_1}\posleq \R{\posgenAel_3}) \booland (\F{\posgenBel_1}\posleq \R{\posgenBel_3}) \booland (\F{\posgenCel_1}\posleq \R{\posgenCel_3}).
        \end{aligned}
    \end{equation}
    It is easy to see that the other direction in the hexagon commutative diagram commutes as well.
    With this we have proved that~$\braiding$ is a valid braiding operation and hence that~$\tup{\DP, \mtimescat, \singleton, \braiding}$ is a symmetric monoidal category.
\end{proof}
