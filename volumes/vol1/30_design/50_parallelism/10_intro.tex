% !TEX root = chapter-standalone.tex


\section{Modeling parallelism}
\label{sec:modeling-parallelism}


[ We are working on this section, and more content will appear. ]

\

We have talked a lot about composition, and considered many different examples. However, the types of compositions we studied were, so far, mostly of the ``in series composition'' kind. For instance, we considered the series composition of travel routes (\cref{exa:car-category}) and trekking routes (\cref{sec:trekking}), functions (\cref{sec:functions}) and relations (\cref{sec:connection-relations}), engineering component dependencies (\cref{sec:dependencies-design}) and Moore machines (\cref{sec:moore-machines}), etc. 


In this chapter, we will think about compositing things both in series and \emph{in parallel}. For example, functions $\mora: \setA \mto \setB$ and $\morb: \setB \mto \setC$ may be composed is series to obtain a function $\mora \then \morb: \setA \mto \setC$, while functions $\mora_1 : \setA \mto \setB$ and $\mora_2 : \setC \mto \setD$ may be composed ``in parallel'' by taking their cartesian product: we obtain the function $\mora_1 \times \mora_2: \setA \times \setB \mto \setC \times \setD$. We note that the series composition here relies on the target set of the function $\mora$ being the same as the source set of $\morb$. The parallel composition of $\mora_1$ and $\morb_2$ has no such restriction, but it relies on the ``additional structure'' provided by the cartesian product. This ``additional structure'' will be formalized in this chapter using the notion of a monoidal product. 

Composing components in parallel is of course a very familiar notion in engineering, and the mathematical concepts we develop here will, in particular, model parallel composition in this engineering sense. In the context of co-design of complex systems, for example, we have seen that series composition corresponds to relating the functionalities of one component to the required resources of a next component. 

\todo{Insert figure here}

Parallel composition, on the other hand, will correspond to taking two components and thinking of them as a single component whose functionality and resource space are given by the cartesian products of the respective constituent functionality and resource spaces of the original two components. 

\todo{Insert figure here}


