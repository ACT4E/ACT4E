% !TEX root = chapter-standalone.tex


\section{Monoidal functors}
\label{sec:monoidal-functors}
\begin{ctdefinition}[\iindex{Strong monoidal functor}]
    \label{def:strong-monoidal-functor}

    Let~$\tup{\CatC,\mtimesC,\idmoncat_{\CatC}}$ and~$\tup{\CatD,\mtimesD,\idmoncat_{\CatD}}$ be two monoidal categories.
    A \emph{strong monoidal functor} between \CatC and \CatD is given by:
    \begin{enumerate}
        \item A functor
        \begin{equation}
            \funa \colon \CatC \fto \CatD;
        \end{equation}
        \item An isomorphism
        \begin{equation}
            \strongeps\colon \idmoncat_{\CatD}\mto \funa(\idmoncat_{\CatC});
        \end{equation}
        \item A natural isomorphism $\strongmu$
        \begin{equation}
            \strongmu_{\Obja,\Objb}\colon \funa(\Obja)\mtimesD \funa(\Objb) \mto \funa(\Obja \mtimesC \Objb),\quad \forall \Obja,\Objb\in \CatC,
        \end{equation}
    \end{enumerate}
    satisfying the following conditions:
    \begin{enumerate}
        \item[a)] \emph{Associativity}: For all objects~$\Obja,\Objb,\Objc\in \CatC$,
        there are  \emph{associators}~$\associator^\CatC$ and~$\associator^\CatD$ such that
        the following diagram commutes.
        \begin{equation}
            \label{eq:120_natural_associativity}
            \includesag{120_natural_associativity}
        \end{equation}

        \item[b)] \emph{Unitality}: For all~$\Obja\in \CatC$, the following diagrams commute:
        \begin{equation}
            \label{eq:120_natural_associativity}
            \includesag{120_natural_associativity}
        \end{equation}
        where~$\leftunitor^\CatC$ and~$\rightunitor^\CatC$ represent the left and right \emph{unitors}.
    \end{enumerate}
\end{ctdefinition}

\todo{Do example here of a compiler having to be a functor to a pre-monoidal category.}
