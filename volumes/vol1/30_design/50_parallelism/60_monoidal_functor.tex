% !TEX root = standalone.tex


\section{Monoidal functors}
\label{sec:monoidal-functors}
\begin{ctdefinition}[\iindex{Strong monoidal functor}]
  \label{def:strong-monoidal-functor}

  Let~$\tup{\CatC,\mtimesC,\idmoncat_{\CatC}}$ and~$\tup{\CatD,\mtimesD,\idmoncat_{\CatD}}$ be two monoidal categories. A \emph{strong monoidal functor} between \CatC and \CatD is given by:
  \begin{enumerate}
    \item A functor
    \begin{equation}
      \funa \colon \CatC \mto \CatD;
    \end{equation}
    \item An isomorphism
    \begin{equation}
      \strongeps\colon \idmoncat_{\CatD}\mto \funa(\idmoncat_{\CatC});
    \end{equation}
    \item A natural isomorphism
    \begin{equation}
      \strongmu_{\obja,\objb}\colon \funa(\obja)\mtimesD \funa(\objb) \to \funa(\obja \mtimesC \objb),\quad \forall \obja,\objb\in \CatC,
    \end{equation}
  \end{enumerate}
  satisfying the following conditions:
  \begin{enumerate}
    \item[a)] \emph{Associativity}: For all objects~$\obja,\objb,\objc\in \CatC$, the following diagram commutes.
    \begin{center}
      \includesag{120_natural_associativity}
    \end{center}
    where~$\associator^\CatC$ and~$\associator^\CatD$ are called \emph{associators}.
    \item[b)] \emph{Unitality}: For all~$\obja\in \CatC$, the following diagrams commute:
    \begin{center}
      \includesag{120_natural_unitality}
    \end{center}
    where~$\leftunitor^\CatC$ and~$\rightunitor^\CatC$ represent the left and right \emph{unitors}.
  \end{enumerate}
\end{ctdefinition}


\todo{Do example here of a compiler having to be a functor to a pre-monoidal category.}
