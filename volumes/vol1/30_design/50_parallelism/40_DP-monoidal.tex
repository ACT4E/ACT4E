% !TEX root = chapter-standalone.tex


\section{\DP is a symmetric monoidal category}\label{sec:parallelism-DP-monoidal}

\linkvideo{spring2021-functorial-comp-b:solving-queries:solving-parallel} % Parallel composition

\begin{example}
    After the X101 spontaneously combusted in low Earth orbit, the astronauts at Jeb's Spaceship Parts go on strike.
    They demand that the engineers take into account safety and living conditions on the future X102.
    As long as the propulsion and life support systems of the X102 do not interact, we can simply tensor the two design problems representing these systems into one, big co-design problem (\cref{fig:examplemonoidal}).
    \begin{figure}[h!]
        \centering
        \includesag{50_engine_tensor_1_2}
        \caption{Example of tensor of design problems. }
        \label{fig:examplemonoidal}
    \end{figure}
\end{example}
In~\DP, putting two design problems in parallel corresponds to their \emph{monoidal product}.

\begin{definition}[Monoidal product in~\DP]
    \label{def:monoidalproduct}
    Given two design problems~$\adpa \colon \funposA \profto \resposB$ and~$\adpb \colon \funposC \profto \resposD$, their \emph{monoidal product}~$\adpa \mtimescat \adpb \colon \funposA \cartprod \funposC \profto \resposB\cartprod \resposD$ is their conjunction:
    \begin{equation}
        \label{eq:monoidalprod_dp}
        \begin{aligned}
            \adpa \mtimescat \adpb \colon (\funposA \cartprod \funposC)\op \cartprod (\resposB \cartprod \resposD) & \toinPos \Bool, \\
            \tup{\tup{\funposAel,\funposCel}^*,\tup{\resposBel,\resposDel}} &\mapsto \adpa(\funposAel^*,\resposBel) \booland \adpb(\funposCel^*,\resposDel).
        \end{aligned}
    \end{equation}
    The diagrammatic representation of the monoidal product is reported in~\cref{fig:dpmonoidal}.
\end{definition}

\todographicsjira{356}{I don't think we use this notation with the explicit $\mtimescat$ at all.
Rather the outer most box should be dotted, do not have the $A\cartprod C$ or $B\cartprod D$
    or $\mtimescat$ in the middle.
    Also, it should be coherent with \cref{fig:examplemonoidal}.
}

\begin{figure}[h!]
    \centering
    \includesag{50_monoidal}
    \caption{Monoidal product of design problems.}
    \label{fig:dpmonoidal}
\end{figure}


\begin{remark}
    For~$\adpa \colon \funposA\profto \resposB$ and~$\adpb \colon \funposC \profto \resposD$, the monoidal product
    \begin{equation}
        \left(\adpa \mtimescat \adpb\right)(\tup{\funposAel^*, \funposCel^*}, \tup{\resposBel,\resposDel})
    \end{equation}
    is true if \emph{both}~$\adpa(\funposAel^*,\resposBel)$ and~$\adpb(\funposCel^*,\resposDel)$ are true, and false otherwise.
\end{remark}

\begin{lemma}
    \label{lem:monoidal_functorial}
    The monoidal product~$\mtimescat$ is functorial (\cref{def:functor}) in~\DP.
\end{lemma}
\begin{proof}
    First, consider posets~$\posgenA,\posgenB\in \Ob_\DP$.
    We show that~$\catid_\posgenA\mtimescat \catid_\posgenB = \catid_{\posgenA\cartprod \posgenB}$
    It holds
    
    \begin{equation}
        \begin{aligned}
            \left( \catid_{\posgenA}\mtimescat \catid_{\posgenB}\right)
            \left( \tup{\F{\posgenAel_1},\F{\posgenBel_1}}^*,\tup{\R{\posgenAel_2},\R{\posgenBel_2}}\right)&=
            \catid_\posgenA(\F{\posgenAel_1}^*,\R{\posgenAel_2})\booland \catid_\posgenB(\F{\posgenBel_1}^*,\R{\posgenBel_2})\\
            &=\left( \F{\posgenAel_1}\posleq_\posgenA \R{\posgenAel_2}\right)\booland \left( \F{\posgenBel_1}\posleq_\posgenB \R{\posgenBel_2}\right)\\
            &=\tup{\F{\posgenAel_1},\F{\posgenBel_1}}\posleq_{\posgenA\cartprod \posgenB}\tup{\R{\posgenAel_2},\R{\posgenBel_2}}\\
            &=\catid_{\posgenA\cartprod \posgenB}\left( \tup{\F{\posgenAel_1},\F{\posgenBel_1}}^*,\tup{\R{\posgenAel_2},\R{\posgenBel_2}}\right).
        \end{aligned}
    \end{equation}
    Furthermore, consider the design problems
    \begin{equation*}
        \adpa\colon \funposA_1 \profto \resposB_1,\quad \adpb\colon \funposA_2\profto \resposB_2, \quad \adpc\colon \funposB_1\profto \resposC_1,\quad \adpd\colon \funposB_2\profto \resposC_2.
    \end{equation*}
    We need to show that
    \begin{equation}
        \underbrace{\left( (\adpa\mthen \adpc) \mtimescat (\adpb\mthen \adpd)\right)}_{\star}=\left( (\adpa\mtimescat \adpb)\mthen (\adpc\mtimescat \adpd)\right).
    \end{equation}
    It holds
    \begin{equation}
        \begin{aligned}
            &\star \left( \tup{\F{\posgenAel_1},\F{\posgenAel_2}}^*,\tup{\R{\posgenCel_1},\R{\posgenCel_2}}\right)=
            (\adpa\mthen \adpc)(\F{\posgenAel_1}^*,\R{\posgenCel_1})\booland (\adpa\mthen \adpd)(\F{\posgenAel_2}^*,\R{\posgenCel_2})\\
            &=\left(\bigvee_{\posgenBel_1\in \posgenB_1}\left( \adpa(\F{\posgenAel_1}^*,\R{\posgenBel_1})\booland \adpc(\F{\posgenBel_1}^*,\R{\posgenCel_1})\right)\right) \booland\left(\bigvee_{\posgenBel_2\in \posgenB_2}\left( \adpb(\F{\posgenAel_2}^*,\R{\posgenBel_2})\booland \adpd(\F{\posgenBel_2}^*,\R{\posgenCel_2})\right)\right)\\
            &=\bigvee_{\posgenBel_1\in \posgenB_1}\bigvee_{\posgenBel_2\in \posgenB_2} \left(\adpa(\F{\posgenAel_1}^*,\R{\posgenBel_1})\booland \adpc(\F{\posgenBel_1}^*,\R{\posgenCel_1})\booland \adpa(\F{\posgenAel_2}^*,\R{\posgenBel_2})\booland \adpd(\F{\posgenBel_2}^*,\R{\posgenCel_2}) \right)\\
            &=\bigvee_{\tup{\posgenBel_1,\posgenBel_2}\in \posgenB_1\cartprod \posgenB_2} \left(\adpa(\F{\posgenAel_1}^*,\R{\posgenBel_1})\booland \adpb(\F{\posgenAel_2}^*,\R{\posgenBel_2})\booland \adpc(\F{\posgenBel_1}^*,\R{\posgenCel_1})\booland \adpd(\F{\posgenBel_2}^*,\R{\posgenCel_2}) \right)\\
            &=\left( (\adpa\mtimescat \adpb)\mthen (\adpc\mtimescat \adpd)\right)\left(\tup{\F{\posgenAel_1},\F{\posgenAel_2}}^*,\tup{\R{\posgenCel_1},\R{\posgenCel_2}} \right).
        \end{aligned}
    \end{equation}
    Therefore,~$\mtimescat$ is functorial in~\DP.
\end{proof}


\begin{lemma}
    $\tup{\DP,\mtimescat, \singleton}$ is a monoidal category.
\end{lemma}
\begin{proof}
    To show that~\DP is a monoidal category, we have to first identify the constituents presented in \cref{def:monoidal-cat}.
    First, recall~$\singleton$ to be singleton: this is the monoidal unit.
    In \cref{lem:monoidal_functorial} we have shown that~$\mtimescat$ is a functor.
    Furthermore, we identify
    \begin{compactitem}
        \item $\leftunitor_\posgenA \colon \F{\singleton} \cartprod \funposA \profto \resposA$, for all~$\posgenA\in \Ob_\DP$, is the left unitor.
        This is given by
        \begin{equation}
            \leftunitor_\posgenA\left( \disunionA{\F{\posgenAel_1}}^*,\R{\posgenAel_2}\right)\definedas \F{\posgenAel_1}\posleq_\posgenA \R{\posgenAel_2}.
        \end{equation}
        To prove that this is an isomorphism, we define its inverse~$\leftunitor_\posgenA^{-1}\colon \funposA\profto \R{\singleton} \cartprod \resposA$ and show that~$\leftunitor_\posgenA\then \leftunitor_\posgenA^{-1}=\id_{\singleton\cartprod \posgenA}$ and~$\leftunitor_\posgenA^{-1}\then \leftunitor_\posgenA=\catid_{\posgenA}$.
        One has
        \begin{equation}
            \begin{aligned}
                \left( \leftunitor_\posgenA^{-1}\mthen \leftunitor_\posgenA\right)(\tup{\F{\posgenAel_1}^*,\R{\posgenAel_2}})&= \bigvee_{\disunionA{\posgenAel}\in  \singleton\cartprod \posgenA} \leftunitor_\posgenA^{-1}(\F{\posgenAel_1}^*,\disunionA{\R{\posgenAel}})\booland \leftunitor_\posgenA(\disunionA{\F{\posgenAel}}^*,\R{\posgenAel_2})\\
                &= \bigvee_{\disunionA{\posgenAel}\in  \singleton\cartprod \posgenA}(\F{\posgenAel_1}\posleq \R{\posgenAel}) \booland \F{\posgenAel}\posleq \R{\posgenAel_2}\\
                &=\F{\posgenAel_1}\posleq \R{\posgenAel_2}\\
                &=\id_\posgenA(\tup{\F{\posgenAel_1}^*,\R{\posgenAel_2}}).
            \end{aligned}
        \end{equation}
        Similarly, one can show that~$\leftunitor_\posgenA\mthen \leftunitor_\posgenA^{-1}=\id_{\singleton \cartprod \posgenA}$.
        \item $\rightunitor_\posgenA\colon \funposA \cartprod \R{\singleton} \profto \resposA$, for all~$\posgenA\in \Ob_\DP$, is the right unitor.
        This is given by
        \begin{equation}
            \rightunitor\left( \tup{\F{\posgenAel_1},1}^*,\R{\posgenAel_2}\right)\definedas \F{\posgenAel_1}\posleq_\posgenA \R{\posgenAel_2}.
        \end{equation}
        The proof that~$\rightunitor_\posgenA$ is an isomorphism is analogous to the one for~$\leftunitor_\posgenA$.
        \item $\associator_{\posgenA,\posgenB,\posgenC}\colon (\funposA\cartprod \funposB)\cartprod \funposC \profto \resposA\cartprod (\resposB\cartprod \resposC)$ for all~$\posgenA,\posgenB,\posgenC \in \Ob_\DP$, is the associator.
        It is given by
        \begin{equation}
            \begin{aligned}
            \associator_{\posgenA,\posgenB,\posgenC}&(\tup{\tup{\F{\posgenAel_1},\F{\posgenBel_1}},\F{\posgenCel_1}}^*,\tup{\R{\posgenAel_2},\tup{\R{\posgenBel_2},\R{\posgenCel_2}}})\\
            &\definedas (\F{\posgenAel_1}\posleq_\posgenA \R{\posgenAel_2}) \booland (\F{\posgenBel_1} \posleq_B \R{\posgenBel_2})\booland (\F{\posgenCel_1}\posleq_C \R{\posgenCel_2}).
            \end{aligned}
            \end{equation}
        To prove that~$\associator_{\posgenA,\posgenB,\posgenC}$ is an isomorphism, we first define its inverse
        \begin{equation}
            \associator_{\posgenA,\posgenB,\posgenC}^{-1}\colon \funposA\cartprod (\funposB\cartprod \funposC) \profto (\resposA\cartprod \resposB)\cartprod \resposC
            \end{equation}
        and show~$\associator_{\posgenA,\posgenB,\posgenC}^{-1}\mthen \associator_{\posgenA,\posgenB,\posgenC}=\associator_{\posgenA,\posgenB,\posgenC}\then \associator_{\posgenA,\posgenB,\posgenC}^{-1}= \catid_{\posgenA\cartprod \posgenB\cartprod \posgenC}$.
        One has
        \begin{equation}
            \begin{aligned}
                &\left( \associator_{\posgenA,\posgenB,\posgenC}^{-1}\mthen \associator_{\posgenA,\posgenB,\posgenC} \right)(\tup{\F{\posgenAel_1},\tup{\F{\posgenBel_1},\F{\posgenCel_1}}}^*,\tup{\R{\posgenAel_2},\tup{\R{\posgenBel_2},\R{\posgenCel_2}}})\\
                &=\bigvee_{\tup{\tup{\posgenAel,\posgenBel},\posgenCel}\in (\posgenA\cartprod \posgenB)\cartprod \posgenC}
                \associator_{\posgenA,\posgenB,\posgenC}^{-1}(\tup{\F{\posgenAel_1},\tup{\F{\posgenBel_1},\F{\posgenCel_1}}}^*,\tup{\tup{\R{\posgenAel},\R{\posgenBel}},\R{\posgenCel}})\booland\\
                &\associator_{\posgenA,\posgenB,\posgenC}(\tup{\tup{\F{\posgenAel},\F{\posgenBel}},\R{\posgenCel}}^*,\tup{\R{\posgenAel_2},\tup{\R{\posgenBel_2},\R{\posgenCel_2}}})\\
                &=\bigvee_{\tup{\tup{\posgenAel,\posgenBel},\posgenCel}\in (\posgenA\cartprod \posgenB)\cartprod \posgenC}\left( (\F{\posgenAel_1}\posleq \R{\posgenAel}) \booland (\F{\posgenBel_1}\leq \R{\posgenBel}) \booland (\F{\posgenCel_1}\posleq \R{\posgenCel})\right)\booland \\
                & \left((\F{\posgenAel}\posleq \R{\posgenAel_2})\booland (\F{\posgenBel}\posleq \R{\posgenBel_2}) \booland (\F{\posgenCel}\posleq \R{\posgenCel_2}\right)\\
                &=(\F{\posgenAel_1}\posleq \R{\posgenAel_2}) \booland (\F{\posgenBel_1}\posleq \R{\posgenBel_2}) \booland (\F{\posgenCel_1}\posleq \R{\posgenCel_2})\\
                &=\catid_{\posgenA\cartprod \posgenB\cartprod \posgenC}(\tup{\F{\posgenAel_1},\F{\posgenBel_1},\F{\posgenCel_1}}^*,\tup{\R{\posgenAel_2},\R{\posgenBel_2},\R{b_3}}).
            \end{aligned}
        \end{equation}
        The proof for~$\associator_{\posgenA,\posgenB,\posgenC}\mthen \associator_{\posgenA,\posgenB,\posgenC}^{-1}$ is analogous.
    \end{compactitem}
    Therefore,~$\tup{\DP,\mtimescat, \singleton}$ is a monoidal category.
\end{proof}

\subsubsection*{\DP is a symmetric monoidal category}
\begin{lemma}
    \label{lem:symmetricmonoidaldp}
    For any~$\posgenA,\posgenB \in\Ob_\DP$, the design problem~$\braiding_{\posgenA,\posgenB}\colon \F{\posgenA} \cartprod \F{\posgenB} \profto \R{\posgenB} \cartprod \R{\posgenA}$ given by
    \begin{equation}
        \braiding_{\posgenA,\posgenB}(\tup{\F{\posgenAel_1},\F{\posgenBel_1}}^*,\tup{\R{\posgenBel_2},\R{\posgenAel_2}})\definedas \left(\F{\posgenAel_1}\posleq_\posgenA \R{\posgenAel_2}\right)\wedge \left(\F{\posgenBel_1}\posleq_\posgenB \R{\posgenBel_2}\right)
    \end{equation}
    constitutes the braiding operation for a symmetric monoidal structure on~$\tup{\DP,\mtimescat,\singleton}$.
    In other words,~$\tup{\DP, \mtimescat, \singleton, \braiding}$ is a symmetric monoidal category.
\end{lemma}

\begin{proof}
    In this proof, given two elements~$\posgenAel_1,\posgenBel_2$ of a poset~$\posgenA$, we denote for brevity~$\posgenAel_1 \posleq_\posgenA \posgenAel_2$ by~$\posgenAel_1 \posleq \posgenAel_2$.
    To prove that~$\braiding_{\posgenA,\posgenB}$ is an isomorphism, we use \cref{def:monoidal-cat} and show~$\braiding_{\posgenA,\posgenB}\then \braiding_{\posgenB,\posgenA}=\id_{\posgenA\cartprod \posgenB}$.
    One has
    \begin{equation}
        \begin{aligned}
            &\left( \braiding_{\posgenA,\posgenB}\mthen \braiding_{\posgenB,\posgenA}\right) \left( \tup{\F{\posgenAel_1},\F{\posgenBel_1}}^*,\tup{\R{\posgenAel_2},\R{\posgenBel_2}}\right)\\
            &=\bigvee_{\tup{\posgenBel,\posgenAel}\in \posgenB\cartprod \posgenA}\braiding_{\posgenA,\posgenB}(\tup{\F{\posgenAel_1},\F{\posgenBel_1}}^*,\tup{\R{\posgenBel},\R{\posgenAel}})\booland \braiding_{\posgenB,\posgenA}(\tup{\F{\posgenBel},\F{\posgenAel}}^*,\tup{\R{\posgenAel_2},\R{\posgenBel_2}})\\
            &=\left( (\F{\posgenAel_1}\posleq \R{\posgenAel}) \booland (\F{\posgenBel_1}\posleq \R{\posgenBel})\right)\booland \left((\F{\posgenAel}\posleq \R{\posgenAel_2}) \booland (\F{\posgenBel}\posleq \R{\posgenBel_2})\right)\\
            &=(\F{\posgenAel_1}\posleq \R{\posgenAel_2})\booland (\F{\posgenBel_1}\posleq \R{\posgenBel_2})\\
            &=\id_{\posgenA\cartprod \posgenB}(\tup{\F{\posgenAel_1},\F{\posgenBel_1}}^*,\tup{\R{\posgenAel_2},\R{\posgenBel_2}}).
        \end{aligned}
    \end{equation}
    This also shows the second triangle identity:~$\braiding_{\posgenA,\posgenB}$ is its own identity.
    For naturality, let's consider two morphisms (design problems)~$\adpa\colon \funposA_1\profto \resposB_1$,~$\adpb\colon \funposA_2\profto \resposB_2$.
    For brevity, denote~$\braiding_{\posgenB_1\cartprod \posgenB_2,\posgenB_2\cartprod \posgenB_1}$ by~$\braiding_\posgenB$ and~$\braiding_{\posgenA_1\cartprod \posgenA_2,\posgenA_2\cartprod \posgenA_1}$ by~$\braiding_\posgenA$.
    One has
    \begin{equation}
        \begin{aligned}
            &\left((\adpa\mtimescat \adpb)\then \braiding_\posgenB \right)\left( \tup{\F{\posgenAel_1},\F{\posgenAel_2}}^*,\tup{\R{\posgenBel_2},\R{\posgenBel_1}}\right)\\
            &=\bigvee_{\tup{\posgenBel,\posgenBel'}\in \posgenB_1\cartprod \posgenB_2} \left(\adpa\mtimescat \adpb\right) \left( \tup{\F{\posgenAel_1},\F{\posgenAel_2}}^*,\tup{\R{\posgenBel},\R{\posgenBel'}}\right)\booland \braiding_\posgenB\left(\tup{\F{\posgenBel},\F{\posgenBel'}},\tup{\R{\posgenBel_2},\R{\posgenBel_1}} \right)\\
            &=\bigvee_{\tup{\posgenBel,\posgenBel'}\in \posgenB_1\cartprod \posgenB_2}(\adpa(\F{\posgenAel_1}^*,\R{\posgenBel})\booland \adpb(\F{\posgenAel_2}^*,\R{\posgenBel'}))\booland (\left(\F{\posgenBel}\posleq \R{\posgenBel_1}\right) \booland \left(\F{\posgenBel'}\posleq \R{\posgenBel_2}\right))\\
            &= \adpa(\F{\posgenAel_1}^*,\R{\posgenBel_1}) \booland \adpb(\F{\posgenAel_2}^*,\R{\posgenBel_2}),
        \end{aligned}
    \end{equation}
    where the last step comes from the monotonicity of~$\adpa$ and~$\adpb$.
    Similarly,
    \begin{equation}
        \begin{aligned}
            &\left( \braiding_\posgenA \then (\adpb\mtimescat \adpa)\right)\left( \tup{\F{\posgenAel_1},\F{\posgenAel_2}}^*,\tup{\R{\posgenBel_2},\R{\posgenBel_1}}\right)\\
            &=\bigvee_{\tup{\posgenAel,\posgenAel'}\in \posgenA_2\cartprod \posgenA_1}\braiding_\posgenA\left(\tup{\F{\posgenAel_1},\F{\posgenAel_2}}^*,\tup{\R{\posgenAel},\R{\posgenAel}'} \right)\booland \left(\adpb\mtimescat \adpa\right) \left( \tup{\F{\posgenAel},\F{\posgenAel}'}^*,\tup{\R{\posgenBel_2},\R{\posgenBel_1}}\right)\\
            &=\bigvee_{\tup{\posgenAel,\posgenAel'}\in \posgenA_2\cartprod \posgenA_1}(\left(\F{\posgenAel_1}\posleq \R{\posgenAel}'\right)\booland \left(\F{\posgenAel_2}\posleq \R{\posgenAel}\right)) \booland (\adpb(\F{\posgenAel}^*,\R{\posgenBel_2})\booland \adpa(\F{\posgenAel}'^*,\R{\posgenBel_1}))\\
            &= \adpb(\F{\posgenAel_1}^*,\R{\posgenBel_1}) \booland \adpa(\R{\posgenAel_2}^*,\R{\posgenBel_2}).
        \end{aligned}
    \end{equation}
    To show the first triangle identity, we write
    \begin{equation}
        \begin{aligned}
            &\left(\braiding_{\singleton \cartprod \posgenA}\then \rightunitor_\posgenA\right)\left( \tup{\F{1},\F{\posgenAel_1}}^*,\R{\posgenAel_2}\right)\\
            &=\bigvee_{\tup{\F{\posgenAel'},\R{\singletonel}}\in \posgenA\cartprod \singleton}\braiding_{\singleton \cartprod \posgenA}\left( \tup{\F{\singletonel},\F{\posgenAel_1}}^*,\tup{\R{\posgenAel'},\R{\singletonel}}\right)\booland \rightunitor_\posgenA\left( \tup{\F{\posgenAel'},\F{\singletonel}}^*,\R{\posgenAel_2}\right)\\
            &=\bigvee_{\tup{\posgenAel',\singletonel}\in \posgenA\cartprod \singleton} \left(\F{\singletonel}\posleq \R{\singletonel}\right) \booland \left(\F{\posgenAel_1}\posleq \R{\posgenAel'}\right)\booland \left(\F{\posgenAel'}\posleq \R{\posgenAel_2}\right)\\
            &=\F{\posgenAel_1}\posleq \R{\posgenAel_2}\\
            &=\leftunitor_\posgenA\left( \tup{\F{\singletonel},\F{\posgenAel_1}}^*,\R{\posgenAel_2}\right).
        \end{aligned}
    \end{equation}
    The hexagon identities are more verbose.
    Consider~$\posgenA,\posgenB,\posgenC\in \Ob_\DP$.
    For brevity, we denote~$\associator_{\posgenA,\posgenB,\posgenC}$ by~$\associator$,~$\braiding_{\posgenA,\posgenB}\mtimescat \id_\posgenC$ by~$\braiding'$,~$\id_\posgenB \mtimescat \braiding_{\posgenA,\posgenC}$ by~$\braiding''$,~$(\posgenB\cartprod \posgenA)\cartprod \posgenC$ as~$\Diamond$, and~$\posgenB\cartprod (\posgenA\cartprod \posgenC)$ as~$\Delta$.
    
    Recall that
    \begin{equation}
        \begin{aligned}
            \braiding' \left(\tup{\tup{\F{\posgenAel_1},\F{\posgenBel_1}},\F{\posgenCel_1}}^*,\tup{\R{\posgenBel_2},\tup{\R{\posgenAel_2},\R{\posgenCel_2}}} \right)&=\left( (\F{\posgenAel_1}\posleq \R{\posgenAel_2})  \booland (\F{\posgenBel_1}\posleq \R{\posgenBel_2})\right)\booland (\F{\posgenCel_1}\posleq \R{\posgenCel_2})\\
            &= (\F{\posgenAel_1}\posleq \R{\posgenAel_2})  \booland (\F{\posgenBel_1}\posleq \R{\posgenBel_2}) \booland (\F{\posgenCel_1}\posleq \R{\posgenCel_2}).
        \end{aligned}
    \end{equation}
    One has
    \begin{equation}
        \begin{aligned}
            &\left(\braiding' \then \associator \right) \left(\tup{\tup{\F{\posgenAel_1},\F{\posgenBel_1}},\F{\posgenCel_1}}^*,\tup{\R{\posgenBel_2},\tup{\R{\posgenAel_2},\R{\posgenCel_2}}} \right)\\
            &=\bigvee_{\tup{\tup{\posgenBel,\posgenAel},\posgenCel}\in \Diamond}\braiding' \left( \tup{\tup{\F{\posgenAel_1},\F{\posgenBel_1}},\F{\posgenCel_1}}^*,\tup{\tup{\R{\posgenBel},\R{\posgenAel}},\R{\posgenCel}}\right)\booland \associator \left( \tup{\tup{\F{\posgenBel},\F{\posgenAel}},\F{\posgenCel}}^*,\tup{\R{\posgenBel_2},\tup{\R{\posgenAel_2},\R{\posgenCel_2}}}\right)\\
            &=\bigvee_{\tup{\tup{\posgenBel,\posgenAel},\posgenCel}\in \Diamond} \left(\left(\F{\posgenAel_1}\posleq \R{\posgenAel} \right)\booland \left( \F{\posgenBel_1}\posleq \R{\posgenBel}\right)\booland \left(\F{\posgenCel_1}\posleq \R{\posgenCel}\right)\right)\booland \\
            &  \left(\left(\F{\posgenBel}\posleq \R{\posgenBel_2}\right)\booland \left( \F{\posgenAel}\posleq \R{\posgenAel_2}\right)\booland \left(\F{\posgenCel}\posleq \R{\posgenCel_2}\right)\right)\\
            &=\left(\F{\posgenBel_1}\posleq \R{\posgenBel_2} \right)\booland \left(\F{\posgenAel_1}\posleq \R{\posgenAel_2} \right)\booland \left( \F{\posgenCel_1}\posleq \R{\posgenCel_2}\right)\\
            &=\underbrace{\associator\left(\tup{\tup{\F{\posgenBel_1},\F{\posgenAel_1}},\F{\posgenCel_1}}^*,\tup{\R{\posgenBel_2},\tup{\R{\posgenAel_2},\R{\posgenCel_2}}}\right)}_{\star}.
        \end{aligned}
    \end{equation}
    Furthermore, consider
    \begin{equation}
        \begin{aligned}
            &\left( \star \mthen \braiding''\right)\left( \tup{\tup{\F{\posgenAel_1},\F{\posgenBel_1}},\F{\posgenCel_1}}^*,\tup{\R{\posgenBel_3},\tup{\R{\posgenCel_3},\R{\posgenAel_3}}}\right)\\
            &=\bigvee_{\tup{\posgenBel_2,\tup{\posgenAel_2,\posgenCel_2}}\in \Delta} \star \left(\tup{\tup{\F{\posgenAel_1},\F{\posgenBel_1}},\F{\posgenCel_1}}^*, \tup{\R{\posgenBel_2},\tup{\R{\posgenAel_2},\R{\posgenCel_2}}} \right)\booland\\
            &\braiding'' \left(\tup{\F{\posgenBel_2},\tup{\F{\posgenAel_2},\F{\posgenCel_2}}}^*,\tup{\R{\posgenBel_3},\tup{\R{\posgenCel_3},\R{\posgenAel_3}}} \right)\\
            &=\bigvee_{\tup{\posgenBel_2,\tup{\posgenAel_2,\posgenCel_2}}\in \Delta}(\left(\F{\posgenBel_1}\posleq \R{\posgenBel_2} \right)\booland \left(\F{\posgenAel_1}\posleq \R{\posgenAel_2} \right)\booland \left( \F{\posgenCel_1}\posleq \R{\posgenCel_2}\right)) \booland\\
            & \left(\left(\F{\posgenBel_2}\posleq \R{\posgenBel_3}\right)\booland \left(\F{\posgenAel_2}\posleq \R{\posgenAel_3}\right) \booland \left(\F{\posgenCel_2}\posleq \R{\posgenCel_3}\right)\right)\\
            &=(\F{\posgenAel_1}\posleq \R{\posgenAel_3}) \booland (\F{\posgenBel_1}\posleq \R{\posgenBel_3}) \booland (\F{\posgenCel_1}\posleq \R{\posgenCel_3}).
        \end{aligned}
    \end{equation}
    It is easy to see that the other direction in the hexagon commutative diagram commutes as well.
    With this we have proved that~$\braiding$ is a valid braiding operation and hence that~$\tup{\DP, \mtimescat, \singleton, \braiding}$ is a symmetric monoidal category.
\end{proof}
