% !TEX root = chapter-standalone.tex


\section{Dualizable objects}
\label{sec:dual-objects}

\todotext{Fill in a minimum of theory about duals in a monoidal category}

There is a concept of ``duality'' for objects in monoidal category which we will introduce with an illustrative example. We have seen in [REF] that the category $\CatC = \Vect_{\reals}$ of real vector spaces is symmetric monoidal, with tensor product as the monoidal product. Given a vector space $\vecspB$, its \emph{linear dual} is the real vector space
\begin{equation}
\vecspB^{*} := \{ \text{ linear maps } \vecspB \mto \reals \} = \Hom_{\CatC}(\vecspB, \reals). 
\end{equation}

Recall from linear algebra the following fact about any vector space $\vecspB$: 
\begin{equation}\label{eq:double-dual-fin-dim}
\vecspB \simeq (\vecspB^{*})^* \text{ if and only if } \dim \vecspB < \infty.  
\end{equation}
One might say that the finite-dimensional real vector spaces are characterizable based on their behavior in this way with respect to the operation of taking the linear dual. 

We will develop an alternative formulation of this fact, based on the notion of a \emph{dualizable object}. This notion will make sense in the setting of any (symmetric) monoidal category, and we will see then, that \cref{eq:double-dual-fin-dim} translates to the statement
\begin{equation}\label{eq:dualizable-fin-dim}
\vecspB \in \Ob_{\Vect_\reals} \text{ is dualizable if and only if } \dim \vecspB < \infty.  
\end{equation}

Key protagonists in this reformulation are \emph{evaluation} and \emph{coevaluation} maps. 

Given finite-dimensional real vector space $\vecspB$, these look as follows. The evaluation map $ev_\vecspB$ associated to $\vecspB$ is
\begin{equation}\label{eq:evaluation-vecsp}
ev_\vecspB : V^* \otimes V \mto \reals, \ \tup{l, v} \mapsto l(v).
\end{equation}
In other words, given $\tup{l, v}$, the $ev_\vecspB$ evaluates $l$ at $v$. 

The coevaluation map $coev_\vecspB$ associated to $\vecspB$ is slightly trickier to understand. Let $\{e_1, ...., e_n \}$ be a basis of $\vecspB$, and let $\{e_1^*, ...., e_n^* \}$ be the corresponding dual basis of $\vecspB^*$. Then
\begin{equation}
coev_\vecspB : \reals \mto V \otimes V^*, \ \lambda \mapsto \lambda \sum_{i=1}^n e_i \otimes e_i^*.
\end{equation}
It turns out that this map is independent of the choice of basis. One way to think of this coevaluation map is to recall that $V \otimes V^* \simeq \Hom(V, V)$. Under this identification, $coev_\vecspB$ maps the scalar $\lambda$ to the linear endomorphism of $V$ which is ``multiplication by $\lambda$''. (In terms of matrices, this is a diagonal matrix, with $\lambda$ at every entry of the diagonal.)
 
Recall that as part of the monoidal structure on $\Vect_\reals$ we have the left and right unitors
\begin{equation}
....
\end{equation}
The evalution and coevaluation maps defined above satisfy the following equations:
\begin{equation}
..
\end{equation}
and
\begin{equation}
...
\end{equation}

\begin{gradedexercise}
Check [REF] and [REF] by direct calculation. 
\end{gradedexercise}

The equations [REF] and [REF] form the basis for the general definition of a dualizable object. 

\todotext{state definition of dualizability for SMC case}
\todotext{make remark about how to generalize to non-symmetric case}
