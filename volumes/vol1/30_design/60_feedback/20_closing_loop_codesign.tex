% !TEX root = chapter-standalone.tex

\todo{some mess here - to reorder}

\begin{example}
  Consider a vehicle motor that weighs a certain amount but can also carry some weight (\cref{fig:examplefeedback}).
  \begin{figure}[h!]
    \begin{center}
      \includesag{50_motor1}
    \end{center}
    \caption{Example of feedback in design problems.}
    \label{fig:examplefeedback}
  \end{figure}
  In the diagram, we haven't added anything to the weight of the motor, so currently the only thing the motor is carrying is itself. Also, note that we are considering~$\text{CO}_2\op$ since we want to optimize toward \emph{less} CO$_2$. Fix a given amount of CO$_2$ and fuel. In that case, closing the loop corresponds to drawing a line~$(c^\ast = c)$ in the graph of~$\text{weight}\op \times \text{weight}$ and taking only solutions under the line in \cref{fig:weightcarrier}.
\end{example}
\begin{figure}[h!]
  \centering
  \includesag{50_carrier}
  \caption{The shaded area marks a portion of the feasibility set of a traced design problem, `motor'. Note that it is not an upper set in the subspace ``weight $\times$ weight'' of `motor'.}
  \label{fig:weightcarrier}

\end{figure}

Note that the shaded area is \emph{not} an upper set. This is admissible, since the actual feasible set of `motor' is a subset of CO$_2 \times$ fuel, and there, it is an upper set.
\todo{change this example}
Suppose that we are given a design problem with a resource and a functionality of the same type $C$ (\cref{fig:extrace_1}).

\begin{figure}[h!]
  \begin{center}
    \includesag{50_trace}
  \end{center}
  \caption{Design problem with a resource and a functionality of the same type.
  }
  \label{fig:extrace_1}
\end{figure}

Can we ``close the loop'', as in the diagram reported in~\cref{fig:extrace_2}?
\begin{figure}[h!]
  \begin{center}
    \includesag{50_trace2}
  \end{center}
  \caption{Closing the loop in the design problem.}
  \label{fig:extrace_2}
\end{figure}

\todotext{adapt style and macroize everything}
It turns out that we can give a well-defined semantics to this loop-closing operation, which coincides with the notion of a \emph{trace} in category theory.

The following is the formal definition of the trace operation for design problems.

\begin{definition}[Trace of a design problem]
  \label{def:dp-trace}
  Given a design problem~$\mora\colon \funposC \times \funposA \profto \resposC \times \resposB$, we can define
  its \emph{trace}~$\Tr_{\funposA,\resposB}^\posgenC(\mora) \colon \funposA \profto \resposB$ as follows:
%
  \begin{equation}
    \label{eq:tracedef}
    \begin{aligned}
      \Tr_{\funposA,\resposB}^\posgenC(\mora) \colon  \funposA\op \times \resposB &\toinPos \Bool \\
      \tup{\F{a}^*, \R{b}} &\mapsto \bigvee_{c\in \posgenC}
      f(\tup{\F{a}^*, \R{c}},
      \tup{\F{b}^*, \R{c}}).
    \end{aligned}
  \end{equation}
\end{definition}


Think of drawing a loop as a way of writing down the following requirement: Something that produces~$C$ should not use up more of~$C$ than it produces.

\paragraph{Trace axioms}
We will show that the loop operation~$\Tr_{A,B}^C$ corresponds to the \emph{trace} in~\DP.
Intuitively, forming a loop models the idea of feedback in a control-theoretic setting--the output of a process influences the choice of input--while the idea of ``trace'' of a monoidal category comes from the trace of a square matrix~$(\Tr A = \sum_i a_{ii})$, which defines the categorical trace in the (monoidal) category of vector spaces.
The connection between the two is more apparent if one decomposes the trace of a square matrix as a set of properties that any linear map from a space to itself should satisfy. One can find the trace axioms in any standard text on category theory, \eg  \cite{mac2013categories}; these are equivalent to certain diagrammatic conditions \cite{joyal96}, as in \cref{tab:traceaxioms}.


\begin{table}[h!]
  \begin{center}
    \adjustbox{max width=\textwidth}{
      \begin{tabular}{cc}
        Vanishing I                     & Vanishing II                    \\
        \includesag{50_vanishing_1a_1b} & \includesag{50_vanishing_2a_2b} \\
        Superposing                     & Yanking                         \\
        \includesag{50_superposing_1_2} & \includesag{50_yanking}
      \end{tabular}
    }
  \end{center}
  \caption{The trace axioms in diagrammatic form from \cite{joyal96} in \DP.
  \label{tab:traceaxioms}}
\end{table}




\begin{lemma}
  Trace as in~\cref{def:dp-trace} satisfies the trace axioms. In other words,~$\tup{\DP, \otimes, \singleton, \sigma}$ is a traced monoidal category, with trace as in~\cref{eq:tracedef}.
\end{lemma}
\begin{proof}
  We have already shown that~$\tup{\DP,\otimes,\singleton,\sigma}$ is a symmetric monoidal category (\cref{lem:symmetricmonoidaldp}).
  We prove the trace axioms one by one, starting from vanishing (\cref{eq:vanishing_1}, \cref{eq:vanishing_2}). Given any~$A,B\in \Ob_\DP$ and~$f\colon \F{A}\times \F{\singleton}\profto \R{B}\times \R{\singleton}$ in~\DP, we have
  \begin{equation}
    \begin{aligned}
      \Tr_{\F{A},\R{B}}^{\singleton}(f)(\F{a^*},\R{b})&=\bigvee_{c\in \singleton}f(\tup{\F{a},\F{c}}^*,\tup{\R{b},\R{c}})\\
      &=f(\tup{\F{a},\F{1}}^*,\tup{\R{b},\R{1}})\\
      &=f(\F{a}^*,\R{b}).
    \end{aligned}
  \end{equation}
  Furthermore, for any morphism~$f\colon \F{A}\times \F{X}\times \F{Y} \profto \R{B}\times \R{X} \times \R{Y}$ in \DP, one has
  \begin{equation}
    \begin{aligned}
      \Tr_{\F{A},\R{B}}^{X\times Y}(f)(\F{a}^*,\R{b})&=\bigvee_{\tup{x,y} \in X\times Y} f(\tup{\F{a},\F{x},\F{y}}^*,\tup{\R{b},\R{x},\R{y}})\\
      &=\bigvee_{x \in X}\left(\bigvee_{y \in Y} f(\tup{\F{a},\F{x},\F{y}}^*,\tup{\R{b},\R{x},\R{y}})\right)\\
      &=\Tr_{\F{A},\R{B}}^X\left(
      \Tr_{\F{A}\times \F{X},\R{B}\times \R{X}}^Y(f)(\tup{\F{a},\F{x}}^*,\tup{\R{b},\R{x}})\right).
    \end{aligned}
  \end{equation}
  For the superposing axiom (\cref{eq:superposing}), consider~$f\colon \F{A}\times \F{X}\profto \R{B}\times \R{X}$ in \DP. One has
  \begin{equation}
    \begin{aligned}
      \Tr_{C\times \F{A},C\times \R{B}}^{X}(\id_C\otimes f)(\tup{\F{c_1},\F{a}}^*,\tup{\R{c_2},\R{b}})&=\bigvee_{x\in X} \id_C(\F{c_1}^*,\R{c_2})\wedge f(\tup{\F{a},\F{x}}^*,\tup{\R{b},\R{x}})\\
      &=\id_C(\F{c_1}^*,\R{c_2}) \wedge \bigvee_{x\in X} f(\tup{\F{a},\F{x}}^*,\tup{\R{b},\R{x}})\\
      &=(\id_C \otimes \Tr_{\F{A},\R{B}}^X(f))(\tup{\F{c_1},\F{a}}^*,\tup{\R{c_2},\R{b}}).
    \end{aligned}
  \end{equation}
  Finally, for yanking (\cref{eq:yanking}) consider~$\sigma_{X,X}$. One has
  \begin{equation}
    \begin{aligned}
      \Tr_{X,X}^{X}(\sigma_{X,X})(\F{x_1}^*,\R{x_2})&=\bigvee_{x\in X} \sigma_{X,X}(\tup{\F{x_1},\F{x}}^*,\tup{\R{x},\R{x_2}}) \\
      &=\bigvee_{x\in X} \F{x_1} \ordleq \R{x_2} \wedge \F{x}\ordleq \R{x}\\
      &=\bigvee_{x\in X} \F{x_1} \ordleq \R{x_2}\\
      &=\id_X(\F{x_1}^*,\R{x_2}).
    \end{aligned}
  \end{equation}
\end{proof}
