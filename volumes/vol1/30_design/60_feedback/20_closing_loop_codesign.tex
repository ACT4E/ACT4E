% !TEX root = chapter-standalone.tex

\todojira{201}{Change example}

\begin{example}
    Consider a vehicle motor that weighs a certain amount but can also carry some weight (\cref{fig:examplefeedback}).
    \begin{figure}[h!]
        \begin{center}
            \includesag{50_motor1}
        \end{center}
        \caption{Example of feedback in design problems.}
        \label{fig:examplefeedback}
    \end{figure}
    In the diagram, we haven't added anything to the weight of the motor, so currently the only thing the motor is carrying is itself. Also, note that we are considering~$\text{CO}_2\op$ since we want to optimize toward \emph{less} CO$_2$. Fix a given amount of CO$_2$ and fuel. In that case, closing the loop corresponds to drawing a line~$(c^\ast = c)$ in the graph of~$\text{weight}\op \times \text{weight}$ and taking only solutions under the line in \cref{fig:weightcarrier}.
\end{example}
\begin{figure}[h!]
    \centering
    \includesag{50_carrier}
    \caption{The shaded area marks a portion of the feasibility set of a traced design problem, `motor'. Note that it is not an upper set in the subspace ``weight $\times$ weight'' of `motor'.}
    \label{fig:weightcarrier}

\end{figure}

Note that the shaded area is \emph{not} an upper set. This is admissible, since the actual feasible set of `motor' is a subset of CO$_2 \times$ fuel, and there, it is an upper set.

Suppose that we are given a design problem with a resource and a functionality of the same type~$\posgenC$ (\cref{fig:extrace_1}).

\begin{figure}[h!]
    \begin{center}
        \includesag{50_trace}
    \end{center}
    \caption{Design problem with a resource and a functionality of the same type.
    }
    \label{fig:extrace_1}
\end{figure}

Can we ``close the loop'', as in the diagram reported in~\cref{fig:extrace_2}?
\begin{figure}[h!]
    \begin{center}
        \includesag{50_trace2}
    \end{center}
    \caption{Closing the loop in the design problem.}
    \label{fig:extrace_2}
\end{figure}

It turns out that we can give a well-defined semantics to this loop-closing operation, which coincides with the notion of a \emph{trace} in category theory.

The following is the formal definition of the trace operation for design problems.


\linkvideo{spring2021-functorial-comp-b:solving-queries:solving-loop} % Loop composition
\begin{definition}[Trace of a design problem]
    \label{def:dp-trace}
    Given a design problem~$\adpa\colon \funposA \times \funposC \profto \resposB \times \resposC$, we can define
    its \emph{trace}~$\Tr_{\funposA,\resposB}^\posgenC(\adpa) \colon \funposA \profto \resposB$ as follows:
%
    \begin{equation}
        \label{eq:tracedef}
        \begin{aligned}
            \Tr_{\funposA,\resposB}^\posgenC(\adpa) \colon  \funposA\op \times \resposB &\toinPos \Bool \\
            \tup{\F{a}^*, \R{b}} &\mapsto \bigvee_{c\in \posgenC}
            \adpa(\tup{\F{a}, \F{c}}^*,
            \tup{\R{b}, \R{c}}).
        \end{aligned}
    \end{equation}
\end{definition}


Think of drawing a loop as a way of writing down the following requirement: Something that produces~$\posgenC$ should not use up more of~$\posgenC$ than it produces.

\todotextjira{164}{The following section/paragraph on trace axioms needs attention: we don't use quite the same axioms in our definition of traced monoidal category as are used below in the context of DP.  I suggest we keep the definition of traced monoidal category that we currently have and change/add diagrams below for DP and/or add a remark.}

\paragraph{Trace axioms}
We will show that the loop operation~$\Tr_{\funposA,\resposB}^\posgenC$ corresponds to the \emph{trace} in~\DP.
Intuitively, forming a loop models the idea of feedback in a control-theoretic setting--the output of a process influences the choice of input--
while the idea of ``trace'' of a monoidal category comes from the trace of a square matrix~$(\Tr\mat{A} = \sum_i a_{ii})$, which defines the categorical trace in the (monoidal) category of vector spaces, as previously shown.
The connection between the two is more apparent if one decomposes the trace of a square matrix as a set of properties that any linear map from a space to itself should satisfy. One can find the trace axioms in  \cite{mac2013categories};
these are equivalent to certain diagrammatic conditions \cite{joyal96}, as in \cref{tab:traceaxioms}.


\begin{table}[h!]
    \begin{center}
        \adjustbox{max width=\textwidth}{
            \begin{tabular}{cc}
                Vanishing I                     & Vanishing II                    \\
                \includesag{50_vanishing_1a_1b} & \includesag{50_vanishing_2a_2b} \\
                Superposing                     & Yanking                         \\
                \includesag{50_superposing_1_2} & \includesag{50_yanking}
            \end{tabular}
        }
    \end{center}
    \caption{The trace axioms in diagrammatic form from \cite{joyal96} in \DP.
    \label{tab:traceaxioms}}
\end{table}




\begin{lemma}
    Trace as in~\cref{def:dp-trace} satisfies the trace axioms. In other words,~$\tup{\DP, \otimes, \singleton, \sigma}$ is a traced monoidal category, with trace as in~\cref{eq:tracedef}.
\end{lemma}
\begin{proof}
    We have already shown that~$\tup{\DP,\otimes,\singleton,\sigma}$ is a symmetric monoidal category (\cref{lem:symmetricmonoidaldp}).
    We prove the trace axioms one by one, starting from vanishing (\cref{eq:vanishing_1}, \cref{eq:vanishing_2}).
    Given any~$\posgenA,\posgenB\in \Ob_\DP$ and~$\adpa\colon \F{\posgenA}\times \F{\singleton}\profto \R{\posgenB}\times \R{\singleton}$ in~\DP, we have
    \begin{equation}
        \begin{aligned}
            \Tr_{\F{\posgenA},\R{\posgenB}}^{\singleton}(\adpa)(\F{a^*},\R{b})&=\bigvee_{c\in \singleton}\adpa(\tup{\F{a},\F{c}}^*,\tup{\R{b},\R{c}})\\
            &=\adpa(\tup{\F{a},\F{1}}^*,\tup{\R{b},\R{1}})\\
            &=\adpa(\F{a}^*,\R{b}).
        \end{aligned}
    \end{equation}
    Furthermore, for any morphism~$\adpa\colon \F{\posgenA}\times \F{\posgenX}\times \F{\posgenY} \profto \R{\posgenB}\times \R{\posgenX} \times \R{\posgenY}$ in \DP, one has
    \begin{equation}
        \begin{aligned}
            \Tr_{\F{\posgenA},\R{\posgenB}}^{\posgenX\times \posgenY}(\adpa)(\F{a}^*,\R{b})&=\bigvee_{\tup{x,y} \in \posgenX\times \posgenY} \adpa(\tup{\F{a},\F{x},\F{y}}^*,\tup{\R{b},\R{x},\R{y}})\\
            &=\bigvee_{x \in \posgenX}\left(\bigvee_{y \in \posgenY} \adpa(\tup{\F{a},\F{x},\F{y}}^*,\tup{\R{b},\R{x},\R{y}})\right)\\
            &=\Tr_{\F{\posgenA},\R{\posgenB}}^X\left(
            \Tr_{\F{\posgenA}\times \F{\posgenX},\R{\posgenB}\times \R{\posgenX}}^Y(\adpa)(\tup{\F{a},\F{x}}^*,\tup{\R{b},\R{x}})\right).
        \end{aligned}
    \end{equation}
    For the superposing axiom (\cref{eq:superposing}), consider~$\adpa\colon \F{\posgenA}\times \F{\posgenX}\profto \R{\posgenB}\times \R{\posgenX}$ in \DP.
    One has
    \begin{equation}
        \begin{aligned}
            \Tr_{\R{\posgenC}\times \F{\posgenA},\R{\posgenC}\times \R{\posgenB}}^{\posgenX}(\id_\posgenC\mtimescat \adpa)(\tup{\F{c_1},\F{a}}^*,\tup{\R{c_2},\R{b}})&=
            \bigvee_{x\in \posgenX} \id_\posgenC(\F{c_1}^*,\R{c_2})\wedge \adpa(\tup{\F{a},\F{x}}^*,\tup{\R{b},\R{x}})\\
            &=\id_\posgenC(\F{c_1}^*,\R{c_2}) \wedge \bigvee_{x\in \posgenX} \adpa(\tup{\F{a},\F{x}}^*,\tup{\R{b},\R{x}})\\
            &=(\id_\posgenC \mtimescat \Tr_{\F{\posgenA},\R{\posgenB}}^\posgenX(\adpa))(\tup{\F{c_1},\F{a}}^*,\tup{\R{c_2},\R{b}}).
        \end{aligned}
    \end{equation}
    Finally, for yanking (\cref{eq:yanking}) consider~$\sigma_{\posgenX,\posgenX}$.
    One has
    \begin{equation}
        \begin{aligned}
            \Tr_{\posgenX,\posgenX}^{\posgenX}(\sigma_{\posgenX,\posgenX})(\F{x_1}^*,\R{x_2})&=\bigvee_{x\in \posgenX} \sigma_{\posgenX,\posgenX}(\tup{\F{x_1},\F{x}}^*,\tup{\R{x},\R{x_2}}) \\
            &=\bigvee_{x\in \posgenX} \F{x_1} \ordleq \R{x_2} \wedge \F{x}\ordleq \R{x}\\
            &=\bigvee_{x\in \posgenX} \F{x_1} \ordleq \R{x_2}\\
            &=\id_\posgenX(\F{x_1}^*,\R{x_2}).
        \end{aligned}
    \end{equation}
\end{proof}
