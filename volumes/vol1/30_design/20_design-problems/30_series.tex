% !TEX root = chapter-standalone.tex


\section{The semi-category \DPI}

\devel{
\begin{forslides}
\begin{equation}\label{eq:adpa}
\adpa
\end{equation}
\begin{equation}\label{eq:adpb}
\adpb
\end{equation}
\begin{equation}\label{eq:adp1}
\adpa: \FB{A} \profto  \RB{B}
\end{equation}

\begin{equation}\label{eq:adp1pf}
\adpa: \FB{A}^{\op} \times  \RB{B} \to_{\Pos} \Bool
\end{equation}

\begin{equation}\label{eq:adp1us}
\adpa: \FB{A} \to \uppersets \RB{B}
\end{equation}

\begin{equation}\label{eq:adp2pf}
\adpb: \FB{B}^{\op} \times  \RB{C} \to_{\Pos} \Bool
\end{equation}

\begin{equation}\label{eq:adp2us}
\adpb: \FB{B} \to \uppersets \RB{C}
\end{equation}


\begin{equation}\label{eq:adp3us}
\adpb: \FB{C} \to \uppersets \RB{D}
\end{equation}

\begin{equation}\label{eq:adp12pf}
(\adpa\fthen\adpb): \FB{A}^{\op} \times  \RB{C} \to_{\Pos} \Bool
\end{equation}

\begin{equation}\label{eq:adp12us}
(\adpa\fthen\adpb): \FB{A}  \to\uppersets \RB{C}
\end{equation}


\begin{equation}\label{eq:adpM12us}
(\adpa\fthen\adpb):(\FB{A} \times \FB{C})  \to\uppersets (\RB{C} \times \RB{D})
\end{equation}

\begin{equation}\label{eq:adp12pfexpr}
\langle\F{a}^{\text{op}}, \R{c}\rangle \mapsto \bigvee_{b_1\leq b_2} \adpa(\F{a}^{\text{op}}, b_1) \wedge \adpb(b_2^{\text{op}}, \R{c})
\end{equation}

\begin{equation}\label{eq:adp2}
\adpb: \FB{B} \profto  \RB{C}
\end{equation}
\begin{equation}\label{eq:adp1adp2}
\adpa\fthen\adpb: \FB{A} \profto  \RB{C}
\end{equation}
\begin{equation}\label{eq:FA}
\FB{A}
\end{equation}
\begin{equation}\label{eq:FB}
\FB{B}
\end{equation}
\begin{equation}\label{eq:FC}
\FB{C}
\end{equation}
\begin{equation}\label{eq:FD}
\FB{D}
\end{equation}
\begin{equation}\label{eq:RA}
\RB{A}
\end{equation}
\begin{equation}\label{eq:RB}
\RB{B}
\end{equation}
\begin{equation}\label{eq:RC}
\RB{C}
\end{equation}
\begin{equation}\label{eq:RD}
\RB{D}
\end{equation}
\end{forslides}
}

\todo{Do here the interconnection of two DPIs and their semantics.}

Graphically, one is allowed to connect only edges of different colors and of the same type.
This interconnection is indicated with the symbol~``$\posleq$'' in a rounded box~(\cref{fig:connection}).

\begin{figure}[h]
  \centering
  \includesag{520_dp_unc_conn}
  \caption{\label{fig:connection}}
\end{figure}

\begin{equation}
  % \label{eq:dp_conn}
  \includesag{520_dp_unc_conn}
\end{equation}
\begin{equation}
  \label{eq:impsp1}
  \impsp_1
\end{equation}
\begin{equation}
  \label{eq:impsp2}
  \impsp_2
\end{equation}
\begin{equation}\label{eq:impsp12}
\impsp_{1} \setconcat \impsp_{2}
\end{equation}
%\captionsideleft{\label{fig:connection}}{\includegraphics[scale=0.33]{papers/uncertainty/unc_connection.pdf}}

The semantics of the interconnection is that the second DPI provides the resources required by the first DPI.
This is a partial order inequality constraint of the type~$\res_{1}\posleq\fun_{2}$.


\begin{definition}[DPI composition]
  \label{def:series-composition}The series composition of two DPIs
  \begin{equation}
    \begin{aligned}
    \dprob_{1}&=&\tup{\funsp_{1},\ressp_{1},\impsp_{1},\prov_{1},\req_{1}},
  \\
    \dprob_{2}&=&\tup{\funsp_{2},\ressp_{2},\impsp_{2},\prov_{2},\req_{2}},
  \end{aligned}
  \end{equation}
  for which $\funsp_{2}=\ressp_{1},$ is
  \begin{equation}
    (\dprob_{1} \fthen \dprob_{2})
    \definedas
    \tup{ \funsp_{1},\ressp_{2},\impsp,\prov,\req},
  \end{equation}
  % where:
  \begin{equation}
    \impsp  =  \{  \tupcat {\imp_{1}} {\imp_{2}} \in (\impsp_{1} \setconcat \impsp_{2})\mid\req_{1}(\imp_{1})\posleq_{\ressp_{1}}\prov_{2}(\imp_{2})\},
  \end{equation}
  \begin{equation}
  \begin{aligned}
    \prov & : & \tupcat {\imp_{1}} {\imp_{2}}   \mapsto\prov_{1}(\imp_{1}),\\
    \req & : & \tupcat {\imp_{1}} {\imp_{2}} \mapsto\req_{2}(\imp_{2}).
  \end{aligned}
  \end{equation}
\end{definition}
\captionsideleft{\label{fig:composition-2}}{
  \includegraphics[scale=0.33]{gmcdp_series3}
}

\begin{forslides}
\begin{equation}\label{eq:dproba}
  \dprob \colon \funsp \profto \ressp
\end{equation}

\begin{equation}\label{eq:dprob2}
  \ftor \colon \funsp \profto \Uressp
\end{equation}
\end{forslides}

\begin{lemma}
  Series composition is associative.
\end{lemma}
\begin{proof}
  \todo{Write proof of this lemma}
\end{proof}

These two properties are sufficient to conclude that there exists a semi-category of design problems.

\begin{definition}[Semi-category \DPI]
  \label{def:DPIcat}
  There is a semi-category \DPI where
  \begin{itemize}
    \item The objects are posets.
    \item The morphisms are DPIs $\tup{\funsp,\ressp,\impsp,\prov,\req}$.
  \end{itemize}
\end{definition}

\begin{lemma}
\DPI is not a category, because we cannot find identities.
\end{lemma}
\begin{proof}
By contradiction. Suppose we can find a DPI that works as an identity for interconnection for any other DPI.
Therefore, we have
\begin{equation}
  \impsp_{1} \setconcat \impsp_{2} = \impsp_{1}.
\end{equation}
This implies that $\impsp_{2}$ must be an empty list of sets,  that is inhabited by only one element, the empty list of elements. Therefore, $\req_2$ of the identity is necessarily a constant because there is an .
\end{proof}
% We also have an identity.

% \begin{definition}[Identity for DPI] For any poset $\posAn$, we can define a DPI
%   \begin{equation}
%     \funid_\posAn = \tup{\funsp,\ressp,\impsp,\prov,\req}
%   \end{equation}
%   as follows:
%   \begin{equation}
%     \funsp = \ressp = \impsp = \posAn,
%   \end{equation}
%   \begin{equation}
%     \prov = \req = \catid_{\posAset},
%   \end{equation}
%   where $\catid_{\posAset}$ is the identity on the set $\posAset$.

%   %   \begin{equation}
%   %   \definemap{\prov}{\impsp}{\funsp}{a}{a}
%   % \end{equation}
%   % \begin{equation}
%   %   \definemap{\req}{\impsp}{\ressp}{a}{a}
%   % \end{equation}
% \end{definition}

% \begin{lemma}
%   $\dprob \fthen \funid_{\ressp} = \dprob$
% \end{lemma}
% \begin{proof}
% Take
% \begin{equation}
% \dprob = \tup{\funsp,\posAn,\impsp,\prov,\req}
% \end{equation}
% We have for the series
% \begin{equation}
%   \impsp =
%   \{
%     \tupcat {\imp_1} {\imp_2} \in (\impsp \setconcat \posAn)
%       \mid
%       \req_{1}(\imp_{1}) \posleq_{\posAn} \prov_{2}(\imp_{2})
% \}
% \end{equation}
% \end{proof}

\todo{discussion about \emph{decomposition is not decoupling}}
