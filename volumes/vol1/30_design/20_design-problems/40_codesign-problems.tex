% !TEX root = chapter-standalone.tex


\section{Co-design problems}
\label{sec:Co-design-problems}

A ``co-design problem'' will be defined as a \emph{multigraph} of design
problems.
\begin{definition}[Co-design problem with implementation]
    \label{def:cdpi}
    A \emph{Co-Design Problem with Implementation} (CDPI)
    is a tuple~$\tup{\funsp,\ressp,\tup{\cdpiN,\cdpiE}}$,
    where~\funsp and~\ressp are two posets, and~$\tup{ \cdpiN,\cdpiE} $
    is a multigraph of DPIs.
    Each node~$\cdpin\in\cdpiN$ is a
    DPI~$\cdpin=\tup{\funsp_{\cdpin},\ressp_{\cdpin},\impsp_{\cdpin},\prov_{\cdpin},\req_{\cdpin}}.$
    An edge~$\cdpie\in\cdpiE$ is a tuple $\cdpie=\tup{ \tup{ \cdpinA,\cdpiresindA} ,\tup{\cdpinB,\cdpifunindB}}$,
    where~$\cdpinA,\cdpinB\in\cdpiN$ are two nodes and~$\cdpiresindA$
    and~$\cdpifunindB$ are the indices of the components of the functionality
    and resources to be connected, and it holds that~$\pi_{\cdpiresindA}\ressp_{\cdpinA}=\pi_{\cdpifunindB}\funsp_{\cdpinB}$~(\cref{fig:mcdps}).

    \begin{figure}[h]
        \centering
        \includegraphics[scale=0.33]{gmcdptro_cdpi}
        \caption{\todographics{redo in tikz}\label{fig:mcdps}}
    \end{figure}

\end{definition}

A CDPI is equivalent to a DPI with an implementation space~\impsp
that is a subset of the product $\prod_{\cdpin\in\cdpiN}\impsp_{\cdpin}$,
and contains only the tuples that satisfy the co-design constraints.
An implementation tuple~$\imp\in\prod_{\cdpin\in\cdpiN}\impsp_{\cdpin}$
belongs to~\impsp iff it respects all functionality--resources
constraints on the edges, in the sense that, for all edges~$\tup{\tup{ \cdpinA,\cdpiresindA} ,\tup{\cdpinB,\cdpifunindB}}$
in~$\cdpiE$, it holds that
\begin{equation*}
    \pi_{\cdpiresindA}\req_{\cdpinA}(\pi_{\cdpinA}\imp)\posleq\pi_{\cdpifunindB}\prov_{\cdpinB}(\pi_{\cdpinB}\imp).
\end{equation*}
The posets~$\funsp,\ressp$ for the entire CDPI are the products
of the functionality and resources of the nodes that remain \emph{unconnected}.
For a node~$\cdpin$, let~$\unconnectedfun_{\cdpin}$ and~$\unconnectedres_{\cdpin}$
be the set of unconnected functionalities and resources.
Then~$\funsp$ and~$\ressp$ for the CDPI are defined as the product of the unconnected functionality and resources of all DPIs:
~$\funsp=\prod_{\cdpin\in\cdpiN}\prod_{\cdpifunind\in\unconnectedfun_{\cdpin}}\pi_{\cdpifunind}\funsp_{\cdpin}$
and~$\ressp=\prod_{\cdpin\in\cdpiN}\prod_{\cdpiresind\in\unconnectedres_{\cdpin}}\pi_{\cdpiresind}\ressp_{\cdpin}.$
The maps~$\prov,\req$ return the values of the unconnected functionality
and resources:
\begin{equation*}
\begin{aligned}
    \prov\colon \imp & \mapsto{\scriptstyle {\displaystyle \prod_{\cdpin\in\cdpiN}\prod_{\cdpifunind\in\unconnectedfun_{\cdpin}}}}\pi_{\cdpifunind}\prov_{\cdpin}(\pi_{\cdpin}\imp),\\
    \req\colon \imp & \mapsto{\displaystyle \prod_{\cdpin\in\cdpiN}\prod_{\cdpiresind\in\unconnectedres_{\cdpin}}}\pi_{\cdpiresind}\req_{\cdpin}(\pi_{\cdpin}\imp).
\end{aligned}
\end{equation*}

\todotext{macro for unconnected a bit similar to upper sets?}
\todo{Re-do a concrete example like this}

\begin{example}
    The MCDP in~\cref{fig:exampleq} is the interconnection of 3
    DPs~$\ftor_{a},\ftor_{b},\ftor_{c}.$ The semantics of the MCDP as
    an optimization problem is given by:
    \begin{equation*}
    \fun_1\mapsto \begin{cases}
                      \Min \res_4,& \res_1\in \ftor_a(\fun_1,\fun_2),\quad \res_1\posleq \fun_3, \quad \fun_3=\tup{\fun_4,\fun_5},\\
                      &\res_2\in \ftor_b(\fun_4),\quad \res_4\posleq \fun_2,\quad \res_4=\tup{\res_2,\res_3},\\
                      &\res_3\in \ftor_c(\fun_5).
\end{cases}
\end{equation*}
\end{example}

\captionsideleft{\todographics{redo in tikz}\label{fig:exampleq}}{ \includegraphics[scale=0.33]{unc_atoms_g_v_graph}}


%\captionsideleft{\label{fig:example-semanticsq}}{\includegraphics[scale=0.33]{unc_semantics}}


\begin{example}
    \label{exa:chassis_plus_motor}Consider the co-design of chassis (\cref{exa:chassis})
    plus motor (\cref{exa:motor}). The design problem for a motor has \F{speed}
    and \F{torque} as the provided functionality (what the motor must
    provide), and \R{cost}, \R{mass}, \R{voltage}, and \R{current}
    as the required resources~(\cref{fig:motor}).

    \begin{figure}[h]
        \centering
        \includesag{520_motor_dp}
        \caption{\label{fig:motor}}
    \end{figure}

%\captionsideleft{\label{fig:motor}}{\includegraphics[scale=0.33]{gmcdp_motor.pdf}}

    \noindent For the chassis (\cref{fig:gmcdp_chassis}), the provided
    functionality is parameterized by the \F{mass} of the payload and
    the platform \F{velocity}. The required resources include the \R{cost},
    \R{total mass}, and what the chassis needs from its motor(s), such
    as \R{speed} and \R{torque}.

    \begin{figure}[h]
        \centering
        \includesag{520_dp_chassis}
        \caption{\label{fig:gmcdp_chassis}}
    \end{figure}

%\captionsideleft{\label{fig:gmcdp_chassis}}{\includegraphics[scale=0.33]{gmcdp_chassis.pdf}}

    \noindent The two design problem can be connected at the edges for
    torque and speed~(\cref{fig:gmcdp_chassis_plus_motor_series}). The
    semantics is that the motor needs to have\emph{ at least }the given
    torque and speed.

    %\begin{figure}[h]
    %  \centering
    %  \includegraphics[scale=0.33]{gmcdp_chassis_plus_motor_series}
    %  \caption{\label{fig:gmcdp_chassis_plus_motor_series}}
    %\end{figure}

    \begin{figure}[h!]
        \centering
        \includesag{dp_chassis_motor}
        \caption{\label{fig:gmcdp_chassis_plus_motor_series}}
    \end{figure}
%\captionsideleft{\label{fig:gmcdp_chassis_plus_motor_series}}{\includegraphics[scale=0.33]{gmcdp_chassis_plus_motor_series.pdf}}

    Resources can be summed together using a trivial DP corresponding
    to the map $\ftor:\left\langle \fun_{1},\fun_{2}\right\rangle \mapsto\{\fun_{1}+\fun_{2}\}$
    (\cref{fig:total_cost}).

    \begin{marginfigure}
        \centering
        \includesag{520_dp_sum_costs}
        \caption{\label{fig:total_cost}}
    \end{marginfigure}


%\captionsideleft{\label{fig:total_cost}}{\includegraphics[scale=0.33]{gmcdp_weightsum.pdf}}

    A co-design problem might contain recursive co-design constraints.
    For example, if we set the payload to be transported to be the sum
    of the motor mass plus some extra payload, a cycle appears in the
    graph~(\cref{fig:gmcdp_chassis_plus_motor}).


    \begin{figure}[h]
        \centering{}\includegraphics[scale=0.33]{gmcdp_chassis_plus_motor}
        \caption{\label{fig:gmcdp_chassis_plus_motor}}
    \end{figure}

    \devel{
        \begin{center}
            \includesag{dp_chassis_motor_loop}
        \end{center}
        \todographics{finish}}

    This formalism makes it easy to abstract away the details
    in which we are not interested. Once a diagram like~\cref{fig:gmcdp_chassis_plus_motor}
    is obtained, we can draw a box around it and consider the abstracted
    problem~(\cref{fig:gmcdp_chassis_plus_motor-1}).


    \begin{figure}[h!]
        \begin{center}
            \includesag{520_dp_chassis_plus_motor}
            \caption{\label{fig:gmcdp_chassis_plus_motor-1}}
        \end{center}
    \end{figure}

%\captionsideleft{\label{fig:gmcdp_chassis_plus_motor-1}}{\includegraphics[scale=0.33]{gmcdp_chassis_plus_motor2.pdf}}

    \label{exa:finish}Let us finish assembling our robot. A motor needs
    a motor control board. The functional requirements are the (peak)
    \F{output current} and the \F{output voltage range}~(\cref{fig:mcb}).

    \begin{figure}[h]
        \centering
        \includesag{520_dp_board}
        \caption{\label{fig:mcb}}
    \end{figure}

%\captionsideleft{\label{fig:mcb}}{\includegraphics[scale=0.33]{gmcdp_mcb.pdf}}

    \noindent The functionality for a power supply could be parameterized
    by the \F{output current}, the \F{output voltages}, and the \F{capacity}.
    The resources could include \R{cost} and \R{mass} (\cref{fig:example-ba}).

    \begin{figure}[h]
        \centering
        \includesag{520_dp_power_supply}
        \caption{\label{fig:example-ba}}
    \end{figure}


%\captionsideleft{\label{fig:example-ba}}{\includegraphics[scale=0.33]{gmcdp_battery.pdf}}

    Relations such as ${\colF\mbox{current}}\times{\colF\mbox{voltage}}\posleq{\colR\mbox{power required}}$
    and ${\colF\mbox{power}}\times{\colF\mbox{endurance}}\posleq{\colR\mbox{energy required}}$
    can be modeled by a trivial ``multiplication'' DPI (\cref{fig:current_times_voltage}).

    \begin{figure}[h]
        \centering
        \includesag{520_dp_current_times_voltage}
        \caption{\label{fig:current_times_voltage}}
    \end{figure}

%\captionsideleft{\label{fig:current_times_voltage}}{\includegraphics[scale=0.33]{gmcdp_voltage_current.pdf}}

    We can connect these DPs to obtain a co-design problem with
    functionality \F{voltage}, \F{current}, \F{endurance} and resources
    \R{mass} and \R{cost}~(\cref{fig:connect}).

    \begin{figure}[h]
        \centering
        \includegraphics[scale=0.29]{gmcdp_MCB_PSU_2}
        \caption{\label{fig:connect}}
    \end{figure}

%\captionsideleft{\label{fig:connect}}{\includegraphics[scale=0.29]{gmcdp_MCB_PSU_2.pdf}}

    Draw a box around the diagram, and call it ``MCB+PSU'';
    then interconnect it with the ``chassis+motor'' diagram in~\cref{fig:another}.


    \begin{figure}[h]
        \begin{centering}
            \includegraphics[scale=0.33]{gmcdp_mobility_power}
        \end{centering}
        \caption{\label{fig:another}}
    \end{figure}

    We can further abstract away the diagram in~\cref{fig:another} as
    a ``mobility+power'' CDPI, as in \cref{fig:shipping}. The formalism
    allows to consider \R{mass} and \R{cost} as independent resources,
    meaning that we wish to obtain the Pareto frontier for the minimal
    resources. Of course, one can always reduce everything to a scalar
    objective. For example, a conversion from mass to cost exists and
    it is called ``shipping''. Depending on the destination, the conversion
    factor is between~$\$0.5/\mbox{lbs}$, using USPS, to~$\$10\mbox{k}/\mbox{lbs}$
    for sending your robot to low Earth orbit.


    \begin{figure}[h]
        \centering{}\includegraphics[scale=0.33]{gmcdp_shipping}\caption{\label{fig:shipping}}
    \end{figure}

\end{example}

