% !TEX root = chapter-standalone.tex


\section{Queries}
\label{sec:design-problems-querying}

A DPI is a model to which we can associate a family of optimization problems.

\todo{Connect with discussion at beginning of the part.}

One query is ``Given a lower bound on the functionality~\fun, what are the implementations that have minimal resources usage?''~(\cref{fig:setup-1}).

\begin{problem}[\FixFunMinReq]
  \label{prob:FixFunMinReq}
  \label{prob:problem1}
  Given~$\fun\in\funsp$, find the implementations in~\impsp that realize the functionality~\fun (or higher) with minimal resources, or provide a proof that there are none:
  \begin{equation}
    \begin{cases}
      \with & \imp\in\impsp,\\
      \Min_{\resleq} & \res,\\
      \subto & \res=\req(\imp),\\
      & \fun\funleq\prov(\imp).
    \end{cases}\label{eq:objective}
  \end{equation}
\end{problem}
\todo{re-do figure}
%\captionsideleft{\label{fig:setup-1}}{\includegraphics[scale=0.33]{gmcdp_setup_query_f}}

\captionsideleft{\label{fig:setup-1}}{\includesag{funresuplow_1}}


\begin{remark}[Minimal \emph{vs} least solutions]
  Note the use of~``$\Min_{\resleq}$'' in~\cref{eq:objective},
  which indicates the set of minimal (non-dominated) elements according
  to~$\resleq$, rather than~``$\min_{\resleq}$'', which would
  presume the existence of a least element. In all problems in this
  paper, the goal is to find the optimal trade-off of resources (``Pareto
  front''). So, for each~\fun, we expect to find an antichain~${\colR R}\in\Aressp$.
  We will see that this formalization allows an elegant way to treat
  multi-objective optimization. The algorithm to be developed will directly
  solve for the set~${\colR R}$, without resorting to techniques such
  as \emph{scalarization}, and therefore is able to work with arbitrary
  posets, possibly discrete.
\end{remark}


In an entirely symmetric fashion, we could fix an upper bound on
the resources usage, and then maximize the functionality provided~(\cref{fig:setup_max_f}).
The formulation is entirely dual, in the sense that it is obtained
from \cref{eq:objective} by swapping~$\Min$ with~$\Max$, \funsp~with~\ressp,
and $\prov$~with~$\req$.

\begin{problem}[\FixResMinFun]
  \label{prob:FixResMinFun}
  Given~$\res\in\ressp$, find the implementations
  in~\impsp that requires~\res (or lower)
  and provide the maximal functionality, or provide a proof that there are none:
  \begin{equation}
    \begin{cases}
      \with & \imp\in\impsp,\\
      \Max_{\funleq} & \fun,\\
      \subto & \fun=\prov(\imp),\\
      & \res\posgeq_{\ressp}\req(\imp).
    \end{cases}\label{eq:objective-1}
  \end{equation}
\end{problem}

\captionsideleft{\label{fig:setup_max_f}}{\includegraphics[scale=0.4]{gmcdp_setup_query_r}}

\vspace{1cm}
\captionsideleft{\label{fig:funresuplow_2}}{\includesag{funresuplow_2}}


Another type of query is
``Given a lower bound on the functionality~\fun
and an upper bound on the costs~\fun, what are the feasible implementations?


\begin{problem}[\FeasibleImp]
  \label{prob:FeasibleImp}
  Given~$\fun\in\funsp$ and $\res\in\ressp$, find the implementations
  in~\impsp that requires~\res (or lower)
  and provide~\fun (or higher)
  \begin{equation}
    \begin{cases}
      \with & \imp\in\impsp,\\
      \subto & \fun \funleq \prov(\imp),\\
      \subto &  \prov(\imp) \resleq \req(\imp) ,\\
    \end{cases}
  \end{equation}
\end{problem}

Another variation is to find only whether there are feasible solutions or not.

\begin{problem}[\Feasibility]
  \label{prob:Feasibility}
  Given~$\fun\in\funsp$ and $\res\in\ressp$, find if \cref{prob:FeasibleImp} is feasible.
\end{problem}
