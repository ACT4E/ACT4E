% !TEX root = chapter-standalone.tex

\section[Series composition]{Series composition of  design problems}

We will define several ways to connect \SY{design problems} together.
The first and most basic way is series composition, or just ``composition''.

\linkvideo{spring2021-functorial-comp-b:solving-queries:solving-series} % Series composition

\begin{definition}[Series composition]
    \label{def:dp-series}
    Let~$\adpa \colon \funposA \profto \resposB$ and~$\adpb \colon \funposB \profto \resposC$ be \SY{design problems}.
    We define their \emph{series composition}~$\adpab \colon \funposA \profto \resposC$ as:
    \begin{equation}
        \label{eq:composition2}
        \defmapperiod{
            \adpab
        }{
            \funposA\posop \Ptimes \resposC
        }{
            \toinPos
        }{
            \Bool
        }{
            \tup{\funposAopel, \resposCel}
        }{
            \bigvee_{\posBel \setin \posBset}
            \adpa(\funposAopel,\resposBel)
            \booland
            \adpb(\funposBopel,\resposCel)
        }
    \end{equation}

\end{definition}
The series composition~$\adpab$ judges a pair~$\tup{\funposAopel,\resposCel}$ as feasible if and only if there exists a~$\posBel \setin \posBset$ such that~$\adpa(\funposAopel,\resposBel)$ and~$\adpb(\funposBopel,\resposCel)$ are feasible.

Given a set~\setA and a map~$\stylemaps{s}\colon \setA\sto\boolset$, we can define the boolean~$\bigvee_{\setAel\setin \setA}\stylemaps{s}(\setAel)$ by
\begin{equation}
    \bigvee_{\setAel\setin \setA}\stylemaps{s}(\setAel)\definedas
    \begin{cases}
        \true  & \text{if there exists }\setAel\setin \setA\text{ for which }\stylemaps{s}(\setAel)=\true,           \\
        \false & \text{if there exists \emph{no} }\setAel\setin \setA\text{ for which }\stylemaps{s}(\setAel)=\true.
    \end{cases}
\end{equation}

In~\cref{eq:composition2} we could have written ``$\exists_{\posBel\setin \posB}$'' instead of ``$\bigvee_{\posBel\setin \posB}$'':
\begin{equation}
    % \defmapperiod{
    %     \adpab
    % }{
    %     \funposA\op \Ctimes \resposC
    % }{
    %     \toinPos
    % }{
    %     \Bool
    % }{
    %     \tup{\funposAopel, \resposCel}
    % }{
    \exists_{\posBel \setin \posBset}\ \adpa(\funposAopel,\resposBel) \booland \adpb(\funposBopel,\resposCel).
    % }
    %                \adpab
    %                \colon \funposA\op \cartprod \resposC & \toinPos  \Bool, \\
    %                \tup{\funposAopel, \resposCel}        & \mapsto \bigvee_{\posBel \setin \posB} \adpa(\funposAopel,\resposBel) \booland \adpb(\funposBopel,\resposCel).
\end{equation}
Using $\bigvee$ form highlights the connection with an integration operation $\int_\posBel$.

We use the same diagrammatic notation for DPs as for DPIs.
We represent series composition as
%
\equationsag{50_series_diag}{fig:compositiondiagram}
%
One can notice the ``co-design constraint'' $\posleq$, which can be interpreted as follows.
The \textR{resource} required by a component is limited by the \textF{functionality} produced by another component.

When viewing compositions (and larger diagrams) formed from these boxes, it is tempting to interpret the boxes as input-output processes.
However, that would be misleading.
The arrows do not represent information flow, materials flow, or energy flow.
Design problems do not represent input-output processes but rather a static calculus of requirements--a requirements flow.

When the \SY{posets} involved are finite, the series composition of \SY{design problems} can be calculated visually, using the kind of representation discussed in \cref{exa:visualize-dp}.

To explain how this works, consider the design problem
\begin{equation}
    \adpa \colon \F{\posgenA} \posop \Ptimes \R{\posgenB} \toinPos \Bool,
\end{equation}
from \cref{exa:visualize-dp}, visualized again for convenience in the first row of \cref{fig:example_dp_graph_xyz}.

And consider another \SY{design problem} of the type
\begin{equation}
    \adpb \colon \F{\posgenB} \posop \Ptimes \R{\posgenC} \toinPos \Bool,
\end{equation}
as given by the visualization in the first row of \cref{fig:example_dp_graph_xyz}.

We can calculate the series composition~$\adpa\dpthen \adpb$ by tracing paths in the ``composite'' visualization given in the second row of \cref{fig:example_dp_graph_xyz}.
Namely, a pair~$\tup{\setAel, \setCel}$ is in the feasibility set of~$\adp \dpthen \adpb$ if and only if we can trace a path from~$\setAel$ to~$\setCel$ by only moving upwards in the \SY{posets}~\posA, \posB, and~\posC, or crossing from one \SY{poset} to another following dashed arrows in the direction they are pointing.
Thus, the visualization of the composite~$\adpa\dpthen \adpb$ is as in the third row of \cref{fig:example_dp_graph_xyz}.

\begin{figure}[h!]
    \centering
    \begin{tabular}{cc}
        \includesag{example_dp_composition_xy}\includesag{example_dp_composition_yz} \\
    \end{tabular}
    \begin{tabular}{c}
        \includesag{example_dp_composition_xyz} \\
        \includesag{example_dp_composition_xz}
    \end{tabular}
    \caption{ }
    \label{fig:example_dp_graph_xyz}
\end{figure}

\todotextjira{359}{\bernina: @Andrea: Implement material explaining and illustrating the perspective on composing \SY{design problems} in terms of matrices and matrix multiplication.
}

% \begin{example}
%     After the Bucket of Boom X100 blew upon re-entry, Jeb's Spaceship Parts is building the X101.
%     This time, they are making sure to take into account other aspects of the rocket design, such as the choice of propellant and nozzle:
%     \equationsag{50_X101}{fig:examplecomposition}
% \end{example}

Let us check that, given \SY{design problems}~$\adpa$ and~$\adpb$, their series composition~$\adpab$ is in fact a design problem.
\begin{lemma}
    Series composition as in~\cref{eq:composition2} is monotone in~$\posAel$ and~$\posCel$.
\end{lemma}
\begin{proof}
    We need to show that~$\adpab(\funposAopel,\resposCel)$ is monotone in~$\funposAopel$ and~$\resposCel$.
    Because~$\adpa$ represents a design problem,~$\adpa(\funposAopel,\resposBel)$ is monotone in~$\funposAopel$, and similarly~$\adpb(\funposBopel,\resposCel)$ is monotone in~$\resposCel$.
    The conjunction ``$\booland$'' is monotone in both variables, and likewise the ``$\boolor$'' operation.
\end{proof}

We can show two important properties for the ``$\dpthen$'' operation: associativity and unitality.
\begin{lemma}
    The series composition operation as in~\cref{eq:composition2} is associative:
    \begin{equation}
        (\adpa\dpthen \adpb)
        \dpthen \adpc = \adpa\dpthen (\adpb\dpthen \adpc).
    \end{equation}
\end{lemma}

\begin{proof}
    Consider~$\adpa\colon \funposA\profto \resposB$,~$\adpb\colon \funposB \profto \resposC$,~$\adpc\colon \funposC\profto \resposD$.
    To show that the operation is \SY{associative}, we can use distributivity and commutativity in~\Bool:
    %
    \begin{equation}
        \label{eq:dp_associativity}
        \begin{aligned}
            \pars{\adpab \dpthen \adpc} (\funposAopel,\resposDel)
             & = \bigvee_{\posCel \setin \posC} \pars{\ \bigvee_{\posBel\setin \posB} \adpa(\funposAopel,\resposBel) \booland \adpb(\funposBopel,\resposCel) } \booland \adpc (\funposCopel, \resposDel) \\
             & = \bigvee_{\posCel \setin \posC} \pars{\ \bigvee_{\posBel\setin \posB} \adpa(\funposAopel,\resposBel)
                \booland \adpb(\funposBopel,\resposCel) \booland \adpc (\funposCopel, \resposDel)
            } \\
             & = \bigvee_{\posBel \setin \posB } \adpa(\funposAopel,\resposBel) \booland \pars{ \bigvee_{\posCel\setin \posC} \adpb(\funposBopel,\resposCel) \booland \adpc (\funposCopel, \resposDel) } \\
             & = \pars{\adpa\dpthen (\adpb\dpthen \adpc )} (\funposAopel, \resposDel).
        \end{aligned}
    \end{equation}
\end{proof}

Because of associativity, we can write~$\adpa\dpthen\adpb\dpthen\adpc$ without ambiguity.
Associativity of composition is represented as:
\equationsag{50_assoc_1_2_3}{fig:compositionassociativity}

\section{Identity for DP}
There exists an identity for the ``$\dpthen$'' operation.
We define the identity~$\dpidat\posA \colon \funposA \profto \resposA$ as follows.

\begin{definition}[Identity design problem]
    \label{def:dp-identity}
    For any poset~\posA, the \maindef{identity design problem}~$\dpidat\posA \colon \funposA \profto \resposA$ is a monotone map
    \begin{equation}
        \label{eq:identity}
        \defmapperiod{
            \dpidat\posA
        }{
            \funposA\posop \Ptimes \resposA
        }{
            \toinPos
        }{
            \Bool
        }{
            \tupp{\funposAel_\F{1}^\F{*},\resposAel_\R{2}}
        }{
            \funposAel_\F{1} \posleqof{\posA} \resposAel_\R{2}
        }
    \end{equation}
\end{definition}
\begin{remark}[Monotonicity of the identity]
    Let's consider~$\funposAel_\F{1}\F{'} \posleqof{\posA} \funposAel_\F{1}$.
    If it holds~$\funposAel_\F{1}\posleqof{\posA}\resposAel_\R{2}$, then it also holds~$\funposAel_\F{1}\F{'}\posleqof{\posA}\resposAel_\R{2}$.
    Similarly, now consider~$\resposAel_\R{2}\posleqof{\posA}\resposAel_\R{2}\R{'}$.
    If it holds~$\funposAel_\F{1} \posleqof{\posA} \resposAel_\R{2}$, then it also holds~$\funposAel_\F{1} \posleqof{\posA} \resposAel_\R{2}\R{'}$.
\end{remark}
In the diagrammatic notation, we represent~$\dpidat\posA$ as:
%
\equationsag{110_identity}{fig:identitydp}

\begin{lemma}
    \label{lem:compositionunital}
    The series composition operation as in~\cref{eq:composition2} satisfies the left and right unit laws (\cref{fig:compositionunital}).
    \equationsag{50_composition_unitality}{fig:compositionunital}
\end{lemma}

\begin{proof}
    Given~$\adpa\colon \funposA\profto \resposB$, we need to show:
    \begin{equation}
        \dpidat\posA \dpthen \adpa = \adpa = \adpa \dpthen \dpidat\posB.
    \end{equation}
    In the following, we prove~$\dpidat\posA \dpthen \adpa = \adpa$.
    Proving~$\adpa\dpthen \dpidat\posB=\adpa$ is similar.
    Consider the poset~\Bool.
    Since for~$\ela,\elb\setin \boolset$,
    \begin{equation}
        \prfcomma{
            \ela\cong \elb
        }{
            \ela=\elb
        }
    \end{equation}
    (also referred to as skeletality~\cite{fong2019}), we just need to show that~$\adpa\posleq \dpidat\posA\dpthen \adpa$ and~$\dpidat\posA\dpthen \adpa\posleq \adpa$.
    Here,~$\adpa\posleq \adpb$ means~$\adpa(\funposAopel,\resposBel)\posleqof{\Bool}\adpb(\funposAopel,\resposBel)$ for all~$\funposAel\setin \funposA$,~$\resposBel\setin \resposB$.
    We have
    \begin{equation}
        \label{eq:dp_unit_1}
        \begin{aligned}
            \adpa(\funposAopel,\resposBel) & =\true \booland \adpa(\funposAopel,\resposBel) \\
                                           & = \dpidat\posA(\funposAopel,\resposAel) \booland \adpa(\funposAopel,\resposBel) \\
                                           & \posleq \bigvee_{\posAel\elprime \setin \posAset}\dpidat\posA(\funposAopel,\R{\posgenAel'})\booland \adpa(\funposAel\F{'}\F{\opel},\resposBel) \\
                                           & =(\dpidat\posA\dpthen \adpa)(\funposAopel,\resposBel).
        \end{aligned}
    \end{equation}
    For the other direction, we need to show that~$\dpidat\posA\dpthen \adpa\posleq \adpa$:
    \begin{equation}
        \label{eq:dp_unit_2}
        \bigvee_{\posAel\elprime\setin \posAset}\dpidat\posA(\funposAopel,\R{\posgenAel'})\booland \adpa(\F{\posgenAel'^*},\resposBel) \posleq \adpa(\funposAopel,\resposBel).
    \end{equation}
    This holds if and only if~$\dpidat\posA(\funposAopel,\R{\posgenAel'})\booland \adpa(\F{\posgenAel'^*},\resposBel) \posleq \adpa(\funposAopel,\resposBel)$ for some~$\posAel\elprime\setin \posAset$.
    If there is no such~$\posAel\elprime$, then the inequality holds ($\false \posleq \false$ and~$\false \posleq \true$).
    If there is such an element~$\posAel\elprime$, it means that~$\dpidat\posA(\funposAopel,\R{\posgenAel'})=\true$ and~$\adpa(\F{\posgenAel'^*},\resposBel)=\true$.
    We know that
    \begin{equation}
        \prfdouble{
            \dpidat\posA(\funposAopel,\R{\posgenAel'})=\true
        }{
            \funposAel \posleq \R{\posAel'}
        }
    \end{equation}
    and hence~$\adpa(\funposAopel,\resposBel)=\true$.
\end{proof}

\begin{gradedexercise}[\exname{ComposingDesignProblems}]
    \label{ex:ComposingDesignProblems}

    Consider the following posets, given in terms of Hasse diagrams:
    \begin{itemize}
        \item $\speed$:
              \begin{center}
                  \includesag{exam_pos_speed}
              \end{center}
        \item $\size$:
              \begin{center}
                  \includesag{exam_pos_size}
              \end{center}
        \item $\money$:
              \begin{center}
                  \includesag{exam_pos_money}
              \end{center}

        \item $\ttime$:
              \begin{center}
                  \includesag{exam_pos_time}
              \end{center}
    \end{itemize}
    Furthermore let the poset~$\speed \Ptimes \size$ be equipped with the standard product partial ordering.

    Consider the design problem given by the monotone function
    \begin{equation}
        \mora\colon (\F{\speed \times \size})\op \Ptimes \R{\money} \mto \Bool
    \end{equation}
    \begin{center}
        \includesag{exam_dp_veh}
    \end{center}
    with
    \todographics{Extend makeset for multiline}
    \begin{align*}
        \mora^{-1}(\true) = & \left\{\tup{\tup{\F{\slow}, \F{\ssmall}}, \R{\cheap}}, \tup{\tup{\F{\slow}, \F{\ssmall}}, \R{\midprice}}, \tup{\tup{\F{\slow}, \F{\ssmall}}, \R{\expensive}},\right.
        \\
                            & \tup{\tup{\F{\slow}, \F{\llarge}}, \R{\midprice}}, \tup{\tup{\F{\slow}, \F{\llarge}}, \R{\expensive}} \\
                            & \left. \tup{\tup{\F{\fast}, \F{\ssmall}}, \R{\expensive}} \right\}
    \end{align*}
    along with the design problem given by the monotone function
    \begin{equation}
        \morb\colon \F{\money}\op \Ptimes \R{\ttime} \mto \Bool
    \end{equation}
    \begin{center}
        \includesag{exam_dp_saving}
    \end{center}
    with
    \begin{equation*}
        \mapb^{-1}(\true) = \makeset{\tup{\F{\cheap}, \R{\sshort}}, \tup{\F{\cheap}, \R{\llong}}, \tup{\F{\midprice}, \R{\llong}}}.
    \end{equation*}

    \begin{enumerate}
        \item Compute the series composition~$\mapa \mthenof\DP \mapb$ in the category of design problems.
              The result should be a design problem described in terms of a monotone function~$(\F{\speed} \Ptimes \F{\size})\op \Ptimes \R{\ttime} \mto \Bool$.
        \item We interpret elements of~$\speed \Ptimes \size$ as properties of cars that Alice considers buying.
              $\money$ represents the amounts of money that she would need to buy one of said cars, and $\ttime$ represents the amounts of time Alice could spend saving money.
              The feasibility relations $\mapa$ and~$\mapb$ describe what is possible for Alice.
              According to~$\mapa \mthenof\DP \mapb$, is it feasible for Alice to buy a fast small car?
              If yes, will she have to work, at minimum, a long or a short amount of time in order to save enough money to buy it?
    \end{enumerate}
\end{gradedexercise}

\solutionof{ComposingDesignProblems}
\begin{gradedexercise}[\exname{DPComposition}]
    \label{ex:DPComposition}

    Consider the following posets, given in terms of Hasse diagrams:
    \begin{itemize}
        \item $\ttime$:
              \begin{center}
                  \includesag{exam_pos_time}
              \end{center}

        \item $\money$:
              \begin{center}
                  \includesag{exam_pos_money_bis}
              \end{center}

        \item $\textbf{Tech complexity}$:
              \begin{center}
                  \includesag{exam_pos_tech}
              \end{center}

        \item $\textbf{Task complexity}$:
              \begin{center}
                  \includesag{exam_pos_task}
              \end{center}
    \end{itemize}
    Let the poset~$\ttime \Ptimes \money$ be equipped with the standard product partial ordering.

    You are in charge of an engineering team that should develop a robotic system.
    It is not yet clear what strategy you wish follow for realizing the project, and you wish to do a preliminary feasibility study.
    You will potentially need to make some tradeoffs between the \emph{time} that your team works on the project and the \emph{money} that you invest.
    Furthermore, you will need to decide between implementing either a cutting edge \emph{fancy} robotic system, or a \emph{simple} one.
    This will have an impact on the \emph{complexity of the tasks} that the system will be able to handle.

    Design problems capture feasibility relations.
    In your case, the feasibility relation between \textbf{Task complexity} and \textbf{Tech complexity} is given by the design problem
    \begin{equation}
        \mora\colon \F{\textbf{Task complexity}}\op \Ptimes \R{\textbf{Tech complexity}} \mto \Bool
    \end{equation}
    \begin{center}
        \includesag{exam_dp_tasktech}
    \end{center}
    with
    \begin{equation*}
        \mora^{-1}(\true) = \makeset{ \ \tup{\F{\text{high} }, \R{ \text{fancy} }}, \tup{\F{\text{medium} }, \R{ \text{fancy} }}, \tup{\F{\text{low} }, \R{ \text{fancy} }}, \tup{\F{\text{low} }, \R{ \text{simple} }}}
    \end{equation*}
    and the feasibility relation between \textbf{Tech complexity} and $\ttime \Ptimes \money$ is given by the design problem
    \begin{equation}
        \morb \colon \F{\textbf{Tech complexity}}\op \Ptimes (\R{\ttime \times \money}) \mto \Bool
    \end{equation}
    \begin{center}
        \includesag{exam_dp_tasktechmoney}
    \end{center}
    with
    \begin{align*}
        \morb^{-1}(\true) = & \{ \tup{\F{\text{fancy} }, \R{ \tup{\text{long}, \text{100K}} }}, \tup{\F{\text{simple} }, \R{ \tup{\text{long}, \text{100K}} }}, \tup{\F{\text{simple} }, \R{ \tup{\text{long}, \text{20K}} }}, \\
                            & \tup{\F{\text{simple} }, \R{ \tup{\text{short}, \text{100K}} }} \}.
    \end{align*}

    \

    Your tasks in this exercise:
    \begin{enumerate}
        \item Compute the series composition~$\mapa \mthenof\DP \mapb$ in the category of design problems.
        \item Based on the previous calculation, would it be feasible to build a robotic system that can handle medium-complexity tasks if your team works only a short time on the project but invests 100K?
    \end{enumerate}
\end{gradedexercise}

\solutionof{DPComposition}
