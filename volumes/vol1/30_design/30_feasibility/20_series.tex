% !TEX root = chapter-standalone.tex

\section[Series composition]{Series composition of design problems}

We will define several ways to connect design problems together.
The first and most basic way is series composition, or just ``composition''.

\linkvideo{spring2021-functorial-comp-b:solving-queries:solving-series} % Series composition

\begin{definition}[Series composition]
    \label{def:dp-series}
    Let~$\adpa \colon  \funposA \profto \resposB$ and~$\adpb \colon \funposB \profto \resposC$ be design problems.
    We define their \emph{series composition}~$\adpab \colon  \funposA \profto \resposC$ as:
    \begin{equation}
        \label{eq:composition2}
        \defmapperiod{
            \adpab
        }{
            \funposA\op \Ctimes \resposC
        }{
            \toinPos
        }{
            \Bool
        }{
            \tup{\funposAopel, \resposCel}
        }{
            \bigvee_{\posBel \setin \posB} \adpa(\funposAopel,\resposBel) \booland \adpb(\funposBopel,\resposCel)
        }.
    \end{equation}

\end{definition}
The series composition~$\adpab$ judges a pair~$\tup{\funposAopel,\resposCel}$ as feasible if and only if there exists a~$\posBel \setin \posB$ such that~$\adpa(\funposAopel,\resposBel)$ and~$\adpb(\funposBopel,\resposCel)$ are feasible.

Given a set~$\setA$ and a map~$\stylemaps{s}\colon \setA\to\boolset$, we can define the boolean~$\bigvee_{\setAel\setin \setA}\stylemaps{s}(\setAel)$ by
\begin{equation}
    \bigvee_{\setAel\setin \setA}\stylemaps{s}(\setAel)\definedas
    \begin{cases}
        \true  & \text{if there exists }\setAel\setin \setA\text{ for which }\stylemaps{s}(\setAel)=\true,           \\
        \false & \text{if there exists \emph{no} }\setAel\setin \setA\text{ for which }\stylemaps{s}(\setAel)=\true.
    \end{cases}
\end{equation}

In~\cref{eq:composition2} we could have written ``$\exists_{\posBel\setin \posB}$'' instead of ``$\bigvee_{\posBel\setin \posB}$'':
\begin{equation}
    % \defmapperiod{
    %     \adpab
    % }{
    %     \funposA\op \Ctimes \resposC
    % }{
    %     \toinPos
    % }{
    %     \Bool
    % }{
    %     \tup{\funposAopel, \resposCel}
    % }{
    \exists_{\posBel \setin \posB}\  \adpa(\funposAopel,\resposBel) \booland \adpb(\funposBopel,\resposCel)
    % }
    %                \adpab
    %                \colon \funposA\op \cartprod \resposC & \toinPos  \Bool, \\
    %                \tup{\funposAopel, \resposCel}        & \mapsto \bigvee_{\posBel \setin \posB} \adpa(\funposAopel,\resposBel) \booland \adpb(\funposBopel,\resposCel).
\end{equation}
Using $\bigvee$ form highlights the connection with an integration operation $\int_\posBel$.

We use the same diagrammatic notation for DPs as for DPIs.
We represent series composition as
%
\equationsag{50_series_diag}{fig:compositiondiagram}
%
One can notice the ``co-design constraint'' $\posleq$, which can be interpreted as follows.
The \textR{resource} required by a component is limited by the \textF{functionality} produced by another component.

When viewing compositions (and larger diagrams) formed from these boxes, it is tempting to interpret the boxes as input-output processes.
However, that would be misleading.
The arrows do not represent information flow, materials flow, or energy flow.
Design problems do not represent input-output processes but rather a static calculus of requirements--a requirements flow.

When the posets involved are finite, the series composition of design problems can be calculated visually, using the kind of representation discussed in \cref{exa:visualize-dp}.

To explain how this works, consider the design problem
\begin{equation}
    \adpa \colon \F{\posgenA} \op \Ctimes  \R{\posgenB} \toinPos \Bool,
\end{equation}
from \cref{exa:visualize-dp}, visualized again for convenience in \cref{fig:example_dp_graph_xy_again}.

And consider another design problem of the type
\begin{equation}
    \adpb \colon \F{\posgenB} \op \Ctimes  \R{\posgenC} \toinPos \Bool,
\end{equation}
as given by the visualization in \cref{fig:example_dp_graph_yz}.

We can calculate the series composition~$\adpa\dpthen \adpb$ by tracing paths in the ``composite'' visualization given in \cref{fig:example_dp_graph_xyz}.
Namely, a pair~$\tup{\setAel, \setCel}$ is in the feasibility set of~$\adp \fthen \adpb$ if and only if we can trace a path from~$\setAel$ to~$\setCel$ by only moving upwards in the posets~$\posA$,~$\posB$, and~$\posC$, or crossing from one poset to another following dashed arrows in the direction they are pointing.
Thus, the visualization of the composite~$\adpa\dpthen \adpb$ is as in \cref{fig:example_dp_graph_xz}.

\todographics{@Gioele: put these 4 figures together.
    First row: example-dp-composition-xy, example-dp-composition-yz.
    Second row: example-dp-composition-xyz.
    Third row: example-dp-composition-xz }
\begin{figure}[h!]
    \centering
    \includesag{example_dp_composition_xy}
    \caption{ }
    \label{fig:example_dp_graph_xy_again}
    \todographics{@Gioele: add name of posets and of DPs on the arrow}
\end{figure}[h!]
\begin{figure}
    \centering
    \includesag{example_dp_composition_yz}
    \caption{ }
    \label{fig:example_dp_graph_yz}
    \todographics{@Gioele: add name of posets and of DPs on the arrow}
\end{figure}
\begin{figure}[h!]
    \centering
    \includesag{example_dp_composition_xyz}
    \caption{}
    \label{fig:example_dp_graph_xyz}
    \todographics{@Gioele: add name of posets and of DPs on the arrow}
\end{figure}
\begin{figure}[h!]
    \centering
    \includesag{example_dp_composition_xz}
    \caption{}
    \label{fig:example_dp_graph_xz}
    \todographics{@Gioele: add name of posets and of DPs on the arrow}
\end{figure}

\todotextjira{359}{@Andrea: Implement material explaining and illustrating the perspective on composing design problems in terms of matrices and matrix multiplication.
}

% \begin{example}
%     After the Bucket of Boom X100 blew upon re-entry, Jeb's Spaceship Parts is building the X101.
%     This time, they are making sure to take into account other aspects of the rocket design, such as the choice of propellant and nozzle:
%     \equationsag{50_X101}{fig:examplecomposition}
% \end{example}

Let us check that, given design problems~$\adpa$ and~$\adpb$, their series composition~$\adpab$ is in fact a design problem.
\begin{lemma}
    Series composition as in~\cref{eq:composition2} is monotone in~$\posAel$ and~$\posCel$.
\end{lemma}
\begin{proof}
    We need to show that~$\adpab(\funposAopel,\resposCel)$ is monotone in~$\funposAopel$ and~$\resposCel$.
    Because~$\adpa$ represents a design problem,~$\adpa(\funposAopel,\resposBel)$ is monotone in~$\funposAopel$, and similarly~$\adpb(\funposBopel,\resposCel)$ is monotone in~$\resposCel$.
    The conjunction ``$\booland$'' is monotone in both variables, and likewise the ``$\boolor$'' operation.
\end{proof}

We can show two important properties for the ``$\fthen$'' operation: associativity and unitality.
\begin{lemma}
    The series composition operation as in~\cref{eq:composition2} is associative:
    \begin{equation}
        (\adpa\dpthen \adpb)
        \fthen \adpc = \adpa\dpthen (\adpb\dpthen \adpc).
    \end{equation}
\end{lemma}

\begin{proof}
    Consider~$\adpa\colon \funposA\profto \resposB$,~$\adpb\colon \funposB \profto \resposC$,~$\adpc\colon \funposC\profto \resposD$.
    To show that the operation is associative, we can use distributivity and commutativity in~\Bool:
    %
    \begin{equation}
        \label{eq:dp_associativity}
        \begin{aligned}
            \left(\adpab \fthen \adpc\right) (\funposAopel,\resposDel)
             & = \bigvee_{\posCel \setin \posC} \left (\ \bigvee_{\posBel\setin \posB} \adpa(\funposAopel,\resposBel) \booland \adpb(\funposBopel,\resposCel) \right )  \booland  \adpc (\funposCopel, \resposDel) \\
             & = \bigvee_{\posCel \setin \posC} \left (\ \bigvee_{\posBel\setin \posB} \adpa(\funposAopel,\resposBel)
            \booland \adpb(\funposBopel,\resposCel) \booland \adpc (\funposCopel, \resposDel)
            \right ) \\
             & = \bigvee_{\posBel \setin \posB } \adpa(\funposAopel,\resposBel) \booland \left ( \bigvee_{\posCel\setin \posC} \adpb(\funposBopel,\resposCel) \booland \adpc (\funposCopel, \resposDel) \right ) \\
             & = \left(\adpa\dpthen (\adpb\dpthen \adpc )\right) (\funposAopel, \resposDel).
        \end{aligned}
    \end{equation}
\end{proof}

Because of associativity, we can write~$\adpa\dpthen\adpb\dpthen\adpc$ without ambiguity.
Associativity of composition is represented as:
\equationsag{50_assoc_1_2_3}{fig:compositionassociativity}

\section{Identity for DP}
There exists an identity for the ``$\fthen$'' operation.
We define the identity~$\dpid_\posA \colon \funposA \profto \resposA$ as follows.

\begin{definition}[Identity design problem]
    \label{def:dp-identity}
    For any poset~$\posA$, the \emph{\iindex{identity design problem}}~$\dpid_\posA \colon \funposA \profto \resposA$ is a monotone map
    \begin{equation}
        \label{eq:identity}
        \defmapperiod{\dpid_\posA}{\funposA\op \Ctimes \resposA}{\toinPos}{\Bool}{\tupp{\funposAel_\F{1}^\F{*},\resposAel_\R{2}}}{\funposAel_\F{1} \posleqof{\posA} \resposAel_\R{2}}
    \end{equation}
    %    \begin{equation}
    %        \begin{aligned}
    %            \dpid_\posA \colon \funposA\op \cartprod \resposA & \toinPos   \Bool, \label{eq:identity} \\
    %            \tupp{\funposAel_\F{1}^\F{*},\resposAel_\R{2}}                  & \mapsto \funposAel_\F{1} \posleqof{\posA} \resposAel_\R{2}.
    %        \end{aligned}
    %    \end{equation}
\end{definition}
\todotext{Describe why this function is monotone.
    not obvious.}
In the diagrammatic notation, we represent~$\dpid_\posA$ as:
%
\equationsag{110_identity}{fig:identitydp}

\begin{lemma}
    \label{lem:compositionunital}
    The series composition operation as in~\cref{eq:composition2} satisfies the left and right unit laws (\cref{fig:compositionunital}).
    \equationsag{50_composition_unitality}{fig:compositionunital}
\end{lemma}

\begin{proof}
    Given~$\adpa\colon \funposA\profto \resposB$, we need to show:
    \begin{equation}
        \dpid_\posA \fthen \adpa = \adpa = \adpa \dpthen \dpid_\posB.
    \end{equation}
    In the following, we prove~$\dpid_\posA \fthen \adpa = \adpa$.
    Proving~$\adpa\dpthen \dpid_\posB=\adpa$ is similar.
    Consider the poset~\Bool.
    Since for~$\ela,\elb\setin \Bool$,
    \begin{equation}
        \prfcomma{
            \ela\cong \elb
        }{
            \ela=\elb
        }
    \end{equation}
    (also referred to as skeletality~\cite{fong2019}), we just need to show that~$\adpa\posleq \dpid_\posA\fthen \adpa$ and~$\dpid_\posA\fthen \adpa\posleq \adpa$.
    \todotext{@Gioele: likely this $\posleq$ is on \Bool, to be understood ``for all p, q...''? clarify}
    We have
    \begin{equation}
        \label{eq:dp_unit_1}
        \begin{aligned}
            \adpa(\funposAopel,\resposBel) & =\true \booland \adpa(\funposAopel,\resposBel) \\
                                           & = \dpid_\posA(\funposAopel,\R{\posAel'}) \booland \adpa(\funposAopel,\resposBel) \\
                                           & \posleq \bigvee_{\posAel\elprime \setin \posA}\dpid_\posA(\funposAopel,\R{\posgenAel'})\booland \adpa(\F{\posAel'^*},\resposBel) \\
                                           & =(\dpid_\posA\fthen \adpa)(\funposAopel,\resposBel).
        \end{aligned}
    \end{equation}
    \todotext{@Gioele: unclear the second and third steps - what is $\posAel'$? etc. }
    For the other direction, we need to show that~$\dpid_\posA\fthen \adpa\posleq \adpa$:
    \begin{equation}
        \label{eq:dp_unit_2}
        \bigvee_{\posAel\elprime\setin \posA}\dpid_\posA(\funposAopel,\R{\posgenAel'})\booland \adpa(\F{\posgenAel'^*},\resposBel) \posleq \adpa(\funposAopel,\resposBel).
    \end{equation}
    This holds if and only if~$\dpid_\posA(\funposAopel,\R{\posgenAel'})\booland \adpa(\F{\posgenAel'^*},\resposBel) \posleq \adpa(\funposAopel,\resposBel)$ for some~$\posAel\elprime\setin \posA$.
    If there is no such~$\posAel\elprime$, then the inequality holds ($\false \posleq \false$ and~$\false \posleq \true$).
    If there is such an element~$\posAel\elprime$, it means that~$\dpid_\posA(\funposAopel,\R{\posgenAel'})=\true$ and~$\adpa(\F{\posgenAel'^*},\resposBel)=\true$.
    We know that
    \begin{equation}
        \prfdouble{
            \dpid_\posA(\funposAopel,\R{\posgenAel'})=\true
        }{
            \funposAel \posleq \R{\posAel'}
        }
    \end{equation}
    and hence~$\adpa(\funposAopel,\resposBel)=\true$.
\end{proof}
