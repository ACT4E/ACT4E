% !TEX root = chapter-standalone.tex
\section{Series composition of design problems}
\todotext{create macro also for the neutral elements of neutral posets}
We will define several ways to connect and compose design problems. The first and most basic way is series composition, or just `composition'.

\begin{definition}[Series composition]
  \label{def:dp-series}
  Let~$\adpa \colon  \funposA \profto \resposB$ and~$\adpb \colon \funposB \profto \resposC$ be design problems. We define their \emph{series composition}~$\adpab \colon  \funposA \profto \resposC$ as:
  \begin{equation}
    \label{eq:composition2}
    \begin{aligned}
    \adpab
      \colon \ \funposA\op \times \resposC & \toinPos  \Bool, \\
      \tup{\funposAel^*, \resposCel} &\mapsto \bigvee_{b\in \posgenB} f(\funposAel^*,\resposBel) \wedge g(\funposBel^*,\resposCel).
    \end{aligned}
  \end{equation}
  Alternatively:
  \begin{equation}
    \begin{aligned}
      \label{eq:composition}
      \adpab\  \colon \ \funposA\op \times \resposC & \toinPos  \Bool,  \\
      \tup{\funposAel^*, \resposCel} &\mapsto \bigvee_{\resposBel_1\posleq\funposBel_2, \resposBel_1,\funposBel_2\in \posgenB} \adpa(\funposAel^*,\resposBel_1) \wedge \adpb(\funposBel_2^*,\resposCel).
    \end{aligned}
  \end{equation}
\end{definition}
We represent series in the diagrammatic notation reported in~\cref{fig:compositiondiagram}.

\begin{figure}[h!]
  \begin{center}
    \includesag{50_series_diag}
  \end{center}
  \caption{Diagrammatic representation of the series composition of design problems. \label{fig:compositiondiagram}}
\end{figure}

One can notice the ``co-design constraint'' $\ordleq$, which can be interpreted as follows.
The \textR{resource} required by a component is limited by the \textF{functionality} produced by another component.

\begin{remark}
  \label{lem:composition_equivalency}
  The series composition operations given in \cref{eq:composition2,eq:composition} are equivalent.

  First consider the direction \cref{eq:composition}~$\implies$ \cref{eq:composition2}. In order for%
  \begin{equation*}
    \bigvee_{\resposBel_1\ordleq\funposBel_2, \resposBel_1,\funposBel_2\in \posgenB} f(\funposAel^*,\resposBel_1) \wedge g(\funposBel_2^*,\resposCel)%
  \end{equation*}
  to be true, there should exist some~$\resposBel_1\ordleq \funposBel_2$ for which~$f(\funposAel^*,\resposBel_1) \wedge g(\funposBel_2^*,\resposCel)$ is true. However, due to the monotonicity of~$f$,~$f(\funposAel^*,\R{b_2}) \wedge g(\funposBel_2^*,\resposCel)$ and \cref{eq:composition2} must be true as well. On the other hand, if%
  \begin{equation*}
    \bigvee_{\resposBel_1\ordleq\funposBel_2, \resposBel_1,\funposBel_2\in \posgenB} f(\funposAel^*,\resposBel_1) \wedge g(\funposBel_2^*,\resposCel)%
  \end{equation*}
  is false, then due to the equality, it is false for any~$\resposBel_1=\funposBel_2$, and therefore all inner terms of \cref{eq:composition2} must be false as well.

  The other direction, \cref{eq:composition2}~$\implies$ \cref{eq:composition}, can be shown in a similar way. If \cref{eq:composition2} is true, then there must exist a~$b'$ such that~$f(\funposAel^*, \R{b'})=\true$ and~$g(\F{b'}^*, \resposCel)=\true$. Then, the inner term in \cref{eq:composition} is true for~$\resposBel_1=\funposBel_2=b'$. If \cref{eq:composition2} is false, then there is no such~$b'$ for which both~$f(\funposAel^*, \R{b'})$ and~$g(\F{b'}^*, \resposCel)$ are true, but then due to the monotonicity of~$f$ and~$g$ they also cannot be true for any~$\resposBel_1 \ordleq \funposBel_2 = b'$ or~$b'=\resposBel_1 \ordleq \funposBel_2$. Hence, \cref{eq:composition} must also be false.
\end{remark}
\begin{remark}
  At first sight, \cref{eq:composition} might seem like a more verbose version of \cref{eq:composition2}. However, assume that we have the means to obtain the minimal antichain of the feasible set of resources that provide~$\funposA$ for the first term:
  \begin{equation*}
    \resposB_f = \Min \{b_1\in \posgenB \mid f(\funposAel^*,\resposBel_1)=\true\} \in \antichains \posgenB.
  \end{equation*}
  This represents the minimal resources with which~$f$ can provide~$\funposA$. Assume further that we similarly have the means to obtain the maximal antichain of the feasible set of functionalities that~$\resposCel$ provides for the second term
  \begin{equation*}
    \funposB_g = \Max \{b_2\in \posgenB \mid g(\funposBel_2^*,\resposCel)=\true\} \in \antichains \posgenB,
  \end{equation*}
  which represents the maximal functionality that~$g$ can provide given~$\resposCel$. Then, \cref{eq:composition} implies that it suffices to only evaluate
  \begin{equation*}
    \bigvee_{\substack{\resposBel_1\ordleq\funposBel_2 \\ \resposBel_1\in \resposB_f,~ \funposBel_2\in\funposB_g }} f(\funposAel^*,\resposBel_1) \wedge g(\funposBel_2^*,\resposCel),
  \end{equation*}
  which can be much more efficient than iterating over all~$b\in \posgenB$.
\end{remark}

\paragraph{Intended semantics}
The series composition~$f\then g$ judges a pair~$\tup{\funposAel^*,\resposCel}$ as feasible if and only if there exists a~$b \in B$ such that~$f(\funposAel^*,\resposBel)$ and~$g(\funposBel^*,\resposCel)$ are feasible.\footnote{In~\cref{eq:composition2} we could have written ``$\exists_{b\in B}$'' instead of ``$\bigvee_{b\in B}$''; the latter form highlights the connection with operations on matrices. Given a set~$I$ and a map~$s\colon I\to\Bool$, we can define the boolean~$\bigvee_{i\in I}s(i)$ by
  \begin{equation*}
    \bigvee_{i\in I}s(i)\definedas
    \begin{cases}
      \true&\text{ if there exists }i\in I\text{ for which }s(i)=\true,\\
      \false&\text{ if there exists \emph{no} }i\in I\text{ for which }s(i)=\true.
    \end{cases}
  \end{equation*}
  For any~$I$, if we have~$i_0\in I$ then~$s(i)\leq\bigvee_{i\in I}s(i)$. One can also check that for any~$b\in\Bool$ or, more generally, any set of booleans~$t\colon J\to\Bool$, we have
  \begin{equation*}
    \bigvee_{i\in I}(b\wedge s(i))=b\wedge\left(\bigvee_{i\in I}s(i)\right)
    \quad\text{and}\quad
    \bigvee_{(i,j)\in I\times  J}\big(s(i)\wedge t(j)\big)=\left(\bigvee_{i\in I}s(i)\right)\wedge\left(\bigvee_{j\in J} t(j)\right).
  \end{equation*}
}

\begin{example}
  After the Bucket of Boom X100 blew upon re-entry, Jeb's Spaceship Parts is building the X101. This time, they are making sure to take into account other aspects of the rocket design, such as the choice of propellant and nozzle (\cref{fig:examplecomposition}).
  \begin{figure}[h!]
    \begin{center}
      \includesag{50_X101}
    \end{center}
    \caption{Example of composition. \label{fig:examplecomposition}}
  \end{figure}
\end{example}

\begin{remark}
  When viewing compositions (and larger diagrams) formed from these boxes, it is tempting to interpret the boxes as input-output processes. However, that would be misleading. The arrows do not represent information flow, materials flow, or energy flow. Design problems do not represent input-output processes but rather a static calculus of requirements--a requirements flow.
\end{remark}

Let us check that, given design problems~$f$ and~$g$, their series composition~$f\then g$ is in fact a design problem.
\begin{lemma}
  Series composition as in~\cref{eq:composition2} is monotone in~$a$ and~$c$.
\end{lemma}
\begin{proof}
  We need to show that~$[f\then g](\funposAel^*,\resposCel)$ is monotone in~$\funposAel^*$ and~$\resposCel$. Because~$f$ represents a design problem,~$f(\funposAel^*,\resposBel)$ is monotone in~$\funposAel^*$, and similarly~$g(\funposBel^*,\resposCel)$ is monotone in~$\resposCel$. The conjunction ``$\wedge$'' is monotone in both variables, and likewise the ``$\vee$'' operation.
\end{proof}

We can show two important properties for the ``$\then$'' operation: associativity and unitality.
\begin{lemma}
  The series composition operation as in~\cref{eq:composition2} is associative:
  \begin{equation}
  (f\then g)
    \then h = f\then (g\then h).
  \end{equation}
\end{lemma}

\devel{
\begin{forslides}
  \begin{equation*}
    \label{eq:dp_assoc_1}
    f\colon \funposA\profto \resposB, \quad g\colon \funposB \profto \resposC, \quad h\colon \funposC\profto \resposD
\end{equation*}
  \begin{equation*}
    \label{eq:dp_unit_3}
    (f\posleq \id_\posgenA\then f) \wedge (\id_\posgenA\then f\posleq f)
\end{equation*}
\end{forslides}
}

\begin{proof}
  Consider~$f\colon \funposA\profto \resposB$,~$g\colon \funposB \profto \resposC$,~$h\colon \funposC\profto \resposD$.
  To show that the operation is associative, we can use distributivity and commutativity in~\Bool:
%
  \begin{equation}
    \label{eq:dp_associativity}
    \begin{aligned}
      \left((f\then g) \then h\right) (\funposAel^*,\resposDel)
      &= \bigvee_{c \in \posgenC} \left (\ \bigvee_{b\in \posgenB} f(\funposAel^*,\resposBel) \wedge g(\funposBel^*,\resposCel) \right )  \wedge  h (\funposCel^*, \resposDel) \\
      &= \bigvee_{c \in \posgenC} \left (\ \bigvee_{b\in B } f(\funposAel^*,\resposBel)
      \wedge g(\funposBel^*,\resposCel) \wedge h (\funposCel^*, \resposDel)
      \right ) \\
      &= \bigvee_{b \in \posgenB } f(\funposAel^*,\resposBel) \wedge \left ( \bigvee_{c\in \posgenC} g(\funposBel^*,\resposCel) \wedge h (\funposCel^*, \resposDel) \right ) \\
      &= \left(f \then (g \then h )\right) (\funposAel^*, \resposDel).
    \end{aligned}
  \end{equation}
%
\end{proof}

Because of associativity, we can write~$f\then g\then h$ without ambiguity.
Associativity of composition is represented as in~\cref{fig:compositionassociativity}.

\begin{figure}[h!]
  \begin{center}
    \includesag{50_assoc_1_2_3}
  \end{center}
  \caption{Series composition is associative. \label{fig:compositionassociativity}}
\end{figure}

There exists an identity for the ``$\then$'' operation.
We define the identity~$\id_\posgenA \colon \funposA \profto \resposA$ as follows.

\begin{definition}[Identity design problem]
  \label{def:dp-identity}
  For any poset~$\posgenA$, the \emph{\iindex{identity design problem}}~$\id_\posgenA \colon \funposA \profto \resposA$ is a monotone map
  \begin{equation}
    \begin{aligned}
      \id_\posgenA \colon \funposA\op \times \resposA & \toinPos   \Bool, \label{eq:identity}\\
      \tup{\funposAel_1^*, \resposAel_2} & \mapsto \funposAel_1 \posleq_{\posgenA} \resposAel_2.
    \end{aligned}
  \end{equation}
\end{definition}
In the diagrammatic notation, we represent~$\id_\posgenA$ as in~\cref{fig:identitydp}.

\begin{figure}[h!]
  \begin{center}
    \includesag{110_identity}
  \end{center}
  \caption{Diagrammatic representation of the identity design problem. \label{fig:identitydp}}
\end{figure}

\begin{lemma}\label{lem:compositionunital}
  The series composition operation as in~\cref{eq:composition2} satisfies the left and right unit laws (\cref{fig:compositionunital}).
\end{lemma}
  \begin{figure}[h!]
    \begin{center}
      \includesag{50_composition_unitality}
    \end{center}
    \caption{Composition satisfies left and right unit laws. \label{fig:compositionunital}}
  \end{figure}
\begin{proof}
  Given~$f\colon \funposA\profto \resposB$, we need to show:
  \begin{equation*}
    \id_\posgenA \then f= f = f\then \id_\posgenB.
  \end{equation*}
  In the following, we prove~$\id_\posgenA \then f = f$. Proving~$f\then \id_\posgenB=f$ is similar.
  Consider the poset~\Bool. Since for~$x,y\in \Bool$,~$x\cong y \Imp x=y$ (also referred to as skeletality~\cite{fong2019}), we just need to show that~$f\posleq \id_\posgenA\then f$ and~$\id_\posgenA\then f\posleq f$.
  We have
  \begin{equation*}
    \label{eq:dp_unit_1}
    \begin{aligned}
      f(\funposAel^*,\resposBel)&=\true \wedge f(\funposAel^*,\resposBel)\\
      &\posleq \id_A(\funposAel^*,\R{a}) \wedge f(\funposAel^*,\resposBel)\\
      &\posleq \bigvee_{a'\in \posgenA}\id_A(\funposAel^*,\R{a'})\wedge f(\F{a'}^*,\resposBel)\\
      &=(\id_\posgenA\then f)(\funposAel^*,\resposBel).
    \end{aligned}
  \end{equation*}
  For the other direction, we need to show that~$\id_\posgenA\then f\ordleq f$:
  \begin{equation*}
    \label{eq:dp_unit_2}
    \bigvee_{a'\in \posgenA}\id_\posgenA(\funposAel^*,\R{a'})\wedge f(\F{a'}^*,\resposBel) \posleq f(\funposAel^*,\resposBel).
  \end{equation*}
  This holds if and only if~$\id_\posgenA(\funposAel^*,\R{a'})\wedge f(\F{a'}^*,\resposBel) \posleq f(\funposAel^*,\resposBel)$ for some~$a'\in \posgenA$.
  If there is no such~$a'$, then the inequality holds ($\false \posleq \false$ and $\false \posleq \true$).
  If there is such an~$a'$, it means that~$\id_\posgenA(\funposAel^*,\R{a'})=\true$ and~$f(\F{a'}^*,\resposBel)=\true$.
  We know that~$\id_\posgenA(\funposAel^*,\R{a'})=\true \Leftrightarrow \funposA\posleq \R{a'}$, and hence~$f(\funposAel^*,\resposBel)=\true$.
\end{proof}
