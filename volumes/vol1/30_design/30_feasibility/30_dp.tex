% !TEX root = chapter-standalone.tex
\section{The category of design problems}

We will show that the class of all design problems forms a category, which we call~\iindex{\DP}.

\begin{definition}[Category of design problems]
  \label{def:DP}
  The \emph{\iindex{category of design problems},~\DP}, consists of the following constituents:
  \begin{compactenum}
    \item \emph{Objects}: The objects of~\DP are posets.
    \item \emph{Morphisms}: The morphisms of~\DP are design problems (\cref{def:design-problem}).
    \item \emph{Identity morphism}: The identity morphism~$\catid_\posgenA \colon \funposA \profto \resposA$ is given by \cref{def:dp-identity}.
    \item \emph{Composition operation}: Given two morphisms~$\mora \colon  \funposA \profto \resposB$ and~$\morb \colon \funposB \profto \resposC$, their
    composition~$\mora \mthen \morb\colon \funposA\profto \resposC$ is given by \cref{def:dp-series}.
  \end{compactenum}
\end{definition}

We have already shown that the composition operator ``$\mthen$'' is associative and unital, and that the composition of two design problems is a design problem (closure).
Therefore, \DP is a category.

\begin{remark}
  \DP is called \feas or~$\Cat{Prof}_\Bool$ in~\cite{fong2019}.
\end{remark}
