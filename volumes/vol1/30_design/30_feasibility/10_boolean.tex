% !TEX root = chapter-standalone.tex


\section{Design problems as monotone maps}
\label{sec:dpdefinition}

A DPI (\cref{def:DPI}) describes a relation between three sets:~\funsp, \ressp, \impsp.
If we are not interested in the implementations, but just in the relation between \funsp and \ressp, then we can describe a DPI more compactly as a ``DP''\@.

Recall how the problem \Feasibility was defined in \cref{prob:Feasibility}.
Given a particular functionality~$\fun$ and resource~$\res$, we would like to know if they are feasible.

This is a function from~$\funsp \times \ressp$ to~$\Bool$:
\begin{equation}
  f: \funsp \cartprod \ressp \sto \Bool.
\end{equation}
The value~$f(\fun^\res)$ is the answer to the question ``is the functionality~\fun feasible with resources~\res?''. Due to how the problem is defined, we know that%
\begin{compactenum}
  \item If~\fun is feasible with~\res, then any~$\fun_2 \funleq \fun$ is feasible with~\res.
  \item If~\fun is feasible with~\res, then~\fun is feasible with any~$\res_2 \resgeq \res$.
\end{compactenum}
Therefore, we can conclude that~$f$ is monotone in the second argument~\res, and antitone in the first argument~$\fun$.

It is going to be convenient to have functions that are monotone, and not mixed
monotone/antitone.
Instead of considering a map from~$\funsp \cartprod \ressp$ to~$\Bool$, we can turn things around and look at a map~$\adp$ from~$\funsp\op \cartprod \ressp$ to~$\Bool$, defined as~$\adp(x,\res) = f(x\op,\res)$.
Because we use~$\funsp\op$ greather than \funsp, the map~$\adp$ is monotone.

The feasibility map~$\adp$ has now forgotten everything about the implementations; however, it does contain all the information we need to solve co-design feasibility problems.

Therefore, we can define, DPs (design problems), now without implementation.


\begin{definition}[Design Problem]
  \label{def:design-problem}
  A \iindex{design problem} (DP) is a tuple~$\tup{\funsp, \ressp, \adp}$, where~$\funsp, \ressp$ are posets and~$\adp$ is a monotone map of the form%
  \begin{equation*}
    \label{eq:design_prob_as_monfun}
    \adp \colon  \funsp\op \Ctimes \ressp \toinPos \Bool.
  \end{equation*}
  We represent it by an arrow~$\adp \colon \funsp \profto \ressp$.
\end{definition}

\devel{
\begin{forslides}
  \begin{equation*}
    \label{eq:dpi_to_dp_1}
    \tup{\fun^*,\res}\mapsto \exists \imp \in \impsp \colon (\fun \posleq_\funsp \prov(\imp)) \wedge (\req(\imp)\posleq_\ressp \res)
\end{equation*}

    \begin{equation*}
    \label{eq:dpi_to_dp_2}
    \tup{\fun^*,\res}\mapsto \{ \imp \in \impsp \colon (\fun \posleq_\funsp \prov(\imp)) \wedge (\req(\imp)\posleq_\ressp \res)\}
\end{equation*}
\end{forslides}
}

%\paragraph{Intended semantics}
%When we consider a design problem~$\adp\colon \funsp \profto \ressp$, we imagine the poset~\funsp to represent the \textbf{f}unctionality to be provided, while the poset~$R$ represents the \textbf{r}esources required.

% The object~$\adp \colon \funsp \profto \ressp$ is a relation that describes which combinations of functionality and resources are feasible: for each~$\fun^* \in \funsp\op$ and~$\res \in \ressp$,~$\adp(\fun^*,\res)$ is a truth value,~$\true$ or~$\false$, which we call the \emph{feasibility of~\fun given~\res}. The value~$\adp(\fun^*,\res)$ is the answer to the question ``is the functionality~\fun feasible with resources~\res?''.

% This is the basic fact of life in engineering: there is a price to pay for everything, and there are trade-offs to be made.

% The monotonicity of~$\adp$ represents the two following assumptions.

\todotext{not sure we need example here anyway}
\begin{example}
  Imagine a truck to be driving at constant speed on a straight street.
  If it can cover \unit[100]{km} with \unit[10]{L} of gasoline, it can also cover \unit[80]{km} with it.
  Furthermore, it will be able to cover the \unit[100]{km} also with \unit[10]{L} of gasoline.
\end{example}
%
%A design problem~$\adp$ will satisfy these conditions if and only if it is represented by a monotone function.

\todotext{not sure if it is used later}
\begin{definition}[Feasible set of a design problem]
  \label{def:dp-feasible-set}
  We define the \emph{feasible set}~$K_\adp$ of a design problem~$\adp\colon \funsp\op \times \ressp\toinPos \Bool$ as the subset of~$\funsp \op\times \ressp$ for which~$\adp$ is the \emph{indicator function}, that is%
  \begin{equation*}
    K_\adp = \{ \tup{\F{f^*},\res} \in \funsp \op\times \ressp  \mid %
    \adp(\fun^*, \res) = \true %
    \}.
  \end{equation*}
\end{definition}
\begin{remark}
  The set~$K_\adp$ is always an upper set (\cref{def:upperset}).
  In fact, another way to define a design problem is to declare it as ``an upper set in~$\funsp\op \times \ressp$''. This is simpler than declaring it as ``a monotone map to~\Bool''.
  However, the definition as a monotone map will lend very easily to further generalization.
\end{remark}
The Boolean-valued design problems we are considering here do not distinguish between particular implementations: they only tell us if \emph{any} implementation or solution exists for given functionality and resources. We will define~\Set-enriched design problems in \cref{sec:enriched}, which directly generalize Boolean-valued design problems and do distinguish between particular implementations.
\todotext{The enriched section should not be cited here if it is in vol 2}

\paragraph{Diagrammatic notation} We represent design problems using a diagrammatic notation. One design problem~$\adp \colon \funsp \profto \ressp$ is represented as a box with functionality~\funsp on the \emph{left} and resources~\ressp on the \emph{right} (\cref{fig:diagrammaticdp}).
\begin{figure}[h!]
  \begin{center}
    \includesag{50_diagrammatic}
  \end{center}
  \caption{Diagrammatic representation of a design problem. \label{fig:diagrammaticdp}}
\end{figure}
We will connect these diagrams.
\begin{example}
  An aerospace company, Jeb's Spaceship Parts, is designing a new rocket engine, the Bucket of Boom X100. The engine requires fuel and provides thrust, and so it can be modeled as a design problem where \R{fuel} and \F{thrust} are two totally-ordered sets representing their respective resources. The corresponding diagram is reported in~\cref{fig:enginedp}.

  \begin{marginfigure}
    \begin{center}
      \includesag{50_engine}
    \end{center}
    \caption{Diagram of the engine design problem. \label{fig:enginedp}}
  \end{marginfigure}

  Concretely, ``engine'' is represented as a monotone map%
  \begin{equation}
    \text{engine} \colon \Ftext{thrust}\op \times \Rtext{fuel} \toinPos \Bool.
  \end{equation}

  Assuming that the posets \R{fuel}, \F{thrust}$\op$ are finite, we can think of the ``engine'' design problem as a matrix, where each~$(i,j)$-th entry is the answer to the question, ``is the amount of thrust~$\F{f_i}$ feasible with the amount of fuel~$\R{r_j}$?'':

  \begin{equation}
    \begin{blockarray}{ccccccc}
      &&&& \Rtext{Fuel} \\
      && \R{r_1} = 0  & \R{r_2} & \R{r_3} & \hdots & \R{r_m} \\
      \begin{block}{cc(ccccc)}
        & \F{f_n}^* = 0 & 0 & 0 & 0 & & 0 \\
        & \F{f_{n-1}}^* & 0 & 0 & 0 & & 1 \\
        \Ftext{Thrust}\op & \F{f_{n-2}}^* & 0 & 1 & 1 & & 1 \\
        & \vdots &  &  &  & \ddots & \\
        & \F{f_1}^* & 1 & 1 & 1 & & 1 \\
      \end{block}
    \end{blockarray}
  \end{equation}

  Suppose we have tested or are given the performance data of a few different engines, as possible solutions to the ``engine'' design problem, each with a fixed optimal fuel-thrust value. To illustrate the monotonicity assumption, we can render the data of ``engine'' as a graph, as depicted in~\cref{fig:solenginedp}.
  \begin{figure}[h!]
    \begin{center}
      \includesag{50_engine_graph}
    \end{center}
    \caption{Graphical representation of the possible solutions of the engine design problem. \label{fig:solenginedp}}
  \end{figure}

  Note that the shaded regions cover the feasible solution set. This feasible solution set is always an \emph{upper set} (\cref{def:upperset}) in~$\Ftext{thrust}\op \times \Rtext{fuel}$, which is another way of characterizing the monotonicity of the design problem. The optimal solutions, indicated by dots, form an \emph{antichain} of solutions. We will come back to antichains in \cref{sec:computation}, when discussing how to compute optimal solutions of design problems.
\end{example}


%\JT{Do we need 2 large examples, or can we move the battery example to the notes?}
%
%\newcommand{\cCapacity}{\text{Capacity}}
%\newcommand{\cMass}{\text{Mass}}
%
%\begin{example}[Battery]
%   A battery is a store of elecrical energy.
%   We are interested in the relation between the capacity of the battery, measured
%   in joules, and the mass of the battery, measured in grams.
%   We will model the battery as a design problem
%   \[
%       \text{battery} : \cCapacity \profto \cMass,
%   \]
%   with the two posets $\cCapacity$ and $\cMass$ defined as
%   $\cCapacity \definedas \langle\mathbb{R}_+^{\text{J}}, \leq\rangle$
%   and $\cMass \definedas \langle\mathbb{R}_+^{\text{g}}, \leq\rangle$, where ``$\leq$''' is the usual
%   order on $\mathbb{R}_+$.
%
%This is the corresponding diagram:
%   \[
%   \centering
%   \begin{tikzpicture}[oriented WD, bb min width =1.5cm, bby=2ex, bbx=.7cm,bb port length=3pt]
%       \node[bb port sep=0.8, bb={1}{1}, bb name={battery}] (dp) {};
%       \node [black, left = 0.2 of dp] {$\cCapacity$};
%       \node [black, right = 0.2 of dp] {$\cMass$};
%   \end{tikzpicture}
%   \]
%   The concrete representation of the design problem is
%   \[
%       \text{battery} \colon\cCapacity\op\times \cMass \toinPos \Bool,
%   \]
%%
%   In this case, the feasibility question is
%   \[
%           \text{battery}(c^*, m) = \true \quad \Leftrightarrow  \quad
%            \text{a battery of mass $m$ is sufficient to provide the capacity $c^*$.}
%   \]
%   It is easy to convince oneself that this ``$\text{battery}$'' is monotone from basic physics consideration.
%   Monotonicity is equivalent to the following two assertions:
%%
%   \newcommand{\qsmall}[1]{{\color{blue}#1}}
%   \newcommand{\qlarge}[1]{{\color{red}#1}}
%%
%\begin{enumerate}
%\item We can provide a smaller capacity with the same mass:
%   \begin{eqnarray}
%       \text{For all}\  \qlarge{c_2}  \geq_{\cCapacity} \qsmall{c_1},\quad
%       \text{a battery of mass $m$ is sufficient to provide the capacity $\qlarge{c_2}$} \\
%       \Rightarrow
%       \text{a battery of mass $m$ is sufficient to provide the capacity $\qsmall{c_1}$.}
%   \end{eqnarray}
%%
%\item A battery of larger mass can provide the same capacity:
%   \begin{eqnarray}
%       \text{For all}\ \qlarge{m_2} \geq_{\cMass} \qsmall{m_1},\quad
%       \text{a battery of mass $\qsmall{m_1}$ is sufficient to provide the capacity $c^*$} \\
%       \Rightarrow
%       \text{a battery of mass $\qlarge{m_2}$ is sufficient to provide the capacity $c^*$.}
%   \end{eqnarray}
%\end{enumerate}
%
%   Assume that there is a linear relation between mass and capacity,
%   and such relation is described by the energy density~$\rho$ [Wh/kg].
%   Then the minimum mass to provide~$c^*$ is~$m_\text{min} = m_0 + c^* / \rho.$
%   So we have
%   \[
%       \text{battery}(c^*, m) = \true \quad \Leftrightarrow \quad  m \geq m_0 + c^* / \rho.
%   \]
%   A visualization of the feasible set $\feasibleset{\text{battery}}$ is in \cref{fig:battery-1}.
%   Monotonicity means that if we fix one point $\tup{c^*,m}$,
%   if we increase $c^*$, we can only see at most one transition, from feasible to unfeasible.
%   Similarly, if we increase $m$, there is at most one transition from unfeasible to feasible.
%
%\end{example}
%
%\begin{figure}[h!]
%    \todo{figure battery-1}
%    \caption{\label{fig:battery-1}}
%\end{figure}
