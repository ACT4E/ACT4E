% !TEX root = chapter-standalone.tex


\section[DPs as monotone maps]{Design problems as monotone maps}
\label{sec:dpdefinition}


\linkvideo{spring2021-design:design:from-dpi-to-dp} % From DPI to DP

A DPI (\cref{def:DPI}) describes a relation between three sets:~\funsp, \ressp, \impsp.
If we are not interested in the implementations, but just in the relation between \funsp and \ressp, then we can describe a DPI more compactly as a ``DP''\@.

Recall how the problem \Feasibility was defined in \cref{prob:Feasibility}.
Given a particular functionality~$\fun$ and resource~$\res$, we would like to know if they are feasible.

This is a function from~$\funsp \cartprod \ressp$ to~$\Bool$:

\begin{equation}
    \mapb\colon \funsp \cartprod \ressp \sto \Bool.
\end{equation}
%
The value~$\mapb(\fun, \res)$ is the answer to the question ``is the functionality~\fun feasible with resources~\res?''.
Due to how the problem is defined, we know that%
\begin{compactenum}
    \item If~\fun is feasible with~\res, then any~$\fun_2 \funleq \fun$ is feasible with~\res.
    \item If~\fun is feasible with~\res, then~\fun is feasible with any~$\res_2 \resgeq \res$.
\end{compactenum}
Therefore, we can conclude that~$\mapb$ is monotone in the second argument~\res, and antitone in the first argument~$\fun$.

It is going to be convenient to have functions that are monotone, and not mixed monotone/antitone.
Instead of considering a map from~$\funsp \cartprod \ressp$ to~$\Bool$, we can turn things around and look at a map~$\adp$ from~$\funsp\op \cartprod \ressp$ to~$\Bool$, defined as~$\adp(\F{x},\res) = \mapb(\F{x}^*,\res)$.
Because we use~$\funsp\op$ rather than \funsp, the map~$\adp$ is monotone.

\begin{remark}[Spoilers]
    The map $\adp$ has the color of functors, because it will be shown to be a particular type of functors - a profunctor~(\cref{def:profunctor}).
\end{remark}


The feasibility map~$\adp$ has now forgotten everything about the implementations; however, it does contain all the information we need to solve co-design feasibility problems.


\linkvideo{spring2021-design:design:bool-prof} % Boolean profunctors
\begin{definition}[Design Problem]
    \label{def:design-problem}
    A \iindex{design problem} (DP) is a tuple~$\tup{\funsp, \ressp, \adp}$, where~$\funsp, \ressp$ are posets and~$\adp$ is a monotone map of the form%
    \begin{equation*}
        \label{eq:design_prob_as_monfun}
        \adp \colon  \funsp\op \Ctimes \ressp \toinPos \Bool.
    \end{equation*}
    We will also use the notation~$\adp \colon \funsp \profto \ressp$ for design problems, in order to emphasize how we think of them as morphisms.
    This will be explained below.
\end{definition}

\begin{remark}
    Given a DPI~$\tup{\funsp, \ressp, \impsp, \prov,\req}$ it is always possible to obtain from it the following DP
    \begin{equation}
        \label{eq:dpi_to_dp_1}
        \begin{aligned}
            \adp\colon \funsp\op \cartprod \ressp &\toinPos \Bool\\
            \tup{\fun^*,\res}&\mapsto \exists \imp \in \impsp \colon (\fun \posleq_\funsp \prov(\imp)) \wedge (\req(\imp)\posleq_\ressp \res).
        \end{aligned}
    \end{equation}
    Evaluating this DP is the same as asking whether the set
    \begin{equation}
        \label{eq:dpi_to_dp_2}
        \{ \imp \in \impsp \colon (\fun \posleq_\funsp \prov(\imp)) \wedge (\req(\imp)\posleq_\ressp \res)\}
    \end{equation}
    is empty or not.
\end{remark}

\begin{example}
    Recall \cref{exa:dpi_elmotor}, with the catalogue of electric motors in \cref{tab:electric_motors_3}.
    
    \begin{table*}[h]
        \centering
        \adjustbox{max width=\textwidth}{%
            \begin{tabular}{c|c|c|c|c|c}
                Motor ID & Company & $\unit[\text{Torque}]{[kg\cdot cm]}$ & \unit[Weight]{[g]} & \unit[Max Power]{[W]}
                & \unit[Cost]{[USD]}
                \\
                \hline
                \textsf{1204} & \textsf{SOYO}        & 0.18 & 60.0  & 2.34 & 19.95  \\
                \textsf{1206} & \textsf{SOYO}        & 0.95 & 140.0 & 3.00 & 19.95  \\
                \textsf{1207} & \textsf{SOYO}        & 0.65 & 130.0 & 2.07 & 12.95  \\
                \textsf{2267} & \textsf{SOYO}        & 3.7  & 285.0 & 4.76 & 16.95  \\
                \textsf{2279} & \textsf{Sanyo Denki} & 1.9  & 165.0 & 5.40 & 164.95 \\
                \textsf{1478} & \textsf{SOYO}        & 19.0 & 1,000 & 8.96 & 49.95  \\
                \textsf{2299} & \textsf{Sanyo Denki} & 2.2  & 150.0 & 5.90 & 59.95
            \end{tabular}%
        }
        \caption{A simplified catalogue of motors.}
        \label{tab:electric_motors_3}
    \end{table*}
    
    The catalogue induces a design problem~$\adp_\mathrm{EM}$ with diagrammatic form as in \cref{fig:dp_em_2}.
    In particular, we can query the design problem for combinations of \textF{functionalities} and \textR{resources}.
    For instance:
    \begin{equation}
        \adp\left(\unit[0.2]{kg\cdot cm}, \tup{\unit[50.0]{g},\unit[2.0]{W},\unit[15.0]{USD}} \right)=\false,
    \end{equation}
    since no listed model can provide~$\unit[0.2]{kg\cdot cm}$ \textF{torque} by requiring the set of \textR{resources}~$\tup{\unit[50.0]{g},\unit[2.0]{W},\unit[15.0]{USD}}$ or less.
    
    \begin{figure}[tbh]
        \centering
        \includesag{520_dp_electric_motor}
        \caption{Electric motor design problem.}
        \label{fig:dp_em_2}
    \end{figure}
\end{example}


%\paragraph{Intended semantics}
%When we consider a design problem~$\adp\colon \funsp \profto \ressp$, we imagine the poset~\funsp to represent the \textbf{f}unctionality to be provided, while the poset~$R$ represents the \textbf{r}esources required.

% The object~$\adp \colon \funsp \profto \ressp$ is a relation that describes which combinations of functionality and resources are feasible: for each~$\fun^* \in \funsp\op$ and~$\res \in \ressp$,~$\adp(\fun^*,\res)$ is a truth value,~$\true$ or~$\false$, which we call the \emph{feasibility of~\fun given~\res}. The value~$\adp(\fun^*,\res)$ is the answer to the question ``is the functionality~\fun feasible with resources~\res?''.

% This is the basic fact of life in engineering: there is a price to pay for everything, and there are trade-offs to be made.

% The monotonicity of~$\adp$ represents the two following assumptions.
%
%\begin{comment}
%    \todotext{not sure we need example here anyway}
%    \begin{example}
%        Imagine a truck to be driving at constant speed on a straight street.
%        If it can cover \unit[100]{km} with \unit[10]{L} of gasoline, it can also cover \unit[80]{km} with it.
%        Furthermore, it will be able to cover the \unit[100]{km} also with \unit[10]{L} of gasoline.
%    \end{example}
%
%%
%%A design problem~$\adp$ will satisfy these conditions if and only if it is represented by a monotone function.
%\end{comment}

\todotext{@Gioele: here $K$ does not have a macro}

\begin{definition}[Feasible set of a design problem]
    \label{def:dp-feasible-set}
    We define the \emph{feasible set}~$K_\adp$ of a design problem~
    \begin{equation}
        \label{eq:generic-design-prob}
        \adp\colon \funsp\op \cartprod \ressp \toinPos \Bool
    \end{equation}
    as the subset of~$\funsp \op\cartprod \ressp$ for which~$\adp$ is the \emph{indicator function}, that is%
    \begin{equation}
        \label{eq:design-prob-as-upper-set}
        K_\adp = \{ \tup{\F{f^*},\res} \in \funsp \op\cartprod \ressp  \mid %
        \adp(\fun^*, \res) = \true %
        \}.
    \end{equation}
\end{definition}

Note that the feasibility set~$K_\adp$ of a design problem~$\adp \colon \funsp \op \cartprod \ressp\toinPos \Bool$ is a binary relation $K_\adp \subseteq \funsp\op \cartprod \ressp$.
We saw in \cref{re:rel-three-fun-descriptions} that there is a one-to-one correspondence between functions~$\mapa \colon \setA \cartprod \setB \to \Bool$ and binary relations~$\relA \subseteq \setA \cartprod \setB$.

An analogous correspondence holds in the context of design problems:

\begin{exercise}
    \label{ex:adp-uppersets}
    Prove that there is a one-to-one correspondence between design problems~$\adp \colon \funsp \op \cartprod \ressp\toinPos \Bool$ and upper sets~$K \subseteq \funsp\op \cartprod \ressp$.
\end{exercise}
\begin{solution}
    \todotext{@J: Write solution to exercise \cref{ex:adp-uppersets}}
\end{solution}

Recall that when working with a relation~$\relA \subseteq \setA \cartprod \setB$ between sets, if the sets in question were finite, then we could conveniently draw the relation~$\relA$ using arrows to connect elements of~$\setA$ to those elements of~$\setB$ to which they are related via~$\relA$.
Given a design problem~$\adp \colon \funsp \op \cartprod \ressp\toinPos \Bool$ involving finite posets, we can visualize it in a similar fashion.
We use Hasse diagrams to visualize the posets involved, and we use dashed arrows to connect those elements which are related via the feasibility set $K_\adp$.


\begin{marginfigure}
    \centering
    \includesag{example_dp_composition_xy}
    \caption{}
    \label{fig:example_dp_graph_xy}
\end{marginfigure}

\begin{example}
    \label{exa:visualize-dp}
    In \cref{fig:example_dp_graph_xy} we have illustrated this kind of visualization in the case of a design problem of the type
    \begin{equation*}
        \adp \colon \F{\posgenA} \op \cartprod  \R{\posgenB} \toinPos \Bool,
    \end{equation*}
    where~$\F{\posgenA} = \tup{\posAset, \posleq}$ and~$ \R{\posgenB} = \tup{\posBset, \posleq}$ are finite posets, with
    \begin{equation}
        \posAset = \{ \setAel_1, \setAel_2, \setAel_3 \}
        \quad \text{and} \quad
        \posBset =  \{ \setBel_1, \setBel_2\}
    \end{equation}
    and ordered as shown in the figure.
    
    The feasibility set is marked with the dashed arrows; it is the set
    \begin{equation*}
        K_\adp = \{\tup{\setAel_1, \setBel_1}, \tup{\setAel_1, \setBel_2}, \tup{\setAel_2, \setBel_2}, \tup{\setAel_3, \setBel_2} \}.
    \end{equation*}
\end{example}


\begin{remark}
    \label{re:dp-from-monotone}
    Given any monotone map~$\mapb\colon \funsp \toinPos \ressp$, we can turn it into a design problem
    \begin{equation*}
        \adp_\mapb \colon \funsp\op \cartprod \ressp \toinPos \Bool
    \end{equation*}
    via the following recipe.
    Set
    \begin{equation*}
        \adp_\mapb (\fun^*, \res) = \true \quad \text{if and only if} \quad \mapb(\fun) \posleq \res.
    \end{equation*}
\end{remark}

\begin{exercise}
    \label{ex:adp-monotone}
    Prove that $\adp_\mapb$ as defined in \cref{re:dp-from-monotone} is indeed a design problem when $\mapb$ is a monotone map.
\end{exercise}
\begin{solution}
    \todotext{@J: Write solution to exercise \cref{ex:adp-monotone}}
\end{solution}

The Boolean-valued design problems we are considering in this section do not distinguish between particular implementations: they only tell us if \emph{any} implementation or solution exists for given functionality and resources.
We will define~\Set-enriched design problems, which directly generalize Boolean-valued design problems and do distinguish between particular implementations.

\todotext{JL: move the above paragraph somewhere else? E.g. to the place where we introduce Set-enriched design problems.}

\paragraph{Diagrammatic notation} We represent design problems using a diagrammatic notation.
One design problem~$\adp \colon \funsp \profto \ressp$ is represented as a box with functionality~\funsp on the \emph{left} and resources~\ressp on the \emph{right} (\cref{fig:diagrammaticdp}).
\begin{figure}[h!]
    \centering
    \includesag{50_diagrammatic}
    \caption{Diagrammatic representation of a design problem. }
    \label{fig:diagrammaticdp}
\end{figure}
As we did for DPIs, we will connect these diagrams.
\begin{example}
    An aerospace company, Jeb's Spaceship Parts, is designing a new rocket engine, the Bucket of Boom X100.
    The engine requires fuel and provides thrust, and so it can be modeled as a design problem where \R{fuel} and \F{thrust} are two totally-ordered sets representing their respective resources.
    
    The corresponding diagram is reported in~\cref{fig:enginedp}.
    
    \begin{marginfigure}
        \centering
        \includesag{50_engine}
        \caption{Diagram of the engine design problem.}
        \label{fig:enginedp}
    \end{marginfigure}
    
    Concretely, ``engine'' is represented as a monotone map%
    %
    \begin{equation}
        \text{engine} \colon \Ftext{thrust}\op \cartprod \Rtext{fuel} \toinPos \Bool.
    \end{equation}
    %
    Assuming that the posets \R{fuel}, \F{thrust}$\op$ are finite, we can think of the ``engine'' design problem as a matrix, where each~$(i,j)$-th entry is the answer to the question, ``is the amount of thrust~$\F{f_i}$ feasible with the amount of fuel~$\R{r_j}$?'':
    
    \begin{equation}
        \begin{blockarray}{ccccccc}
            &&&& \Rtext{Fuel} \\
            && \R{r_1} = 0  & \R{r_2} & \R{r_3} & \hdots & \R{r_m} \\
            \begin{block}{cc(ccccc)}
                & \F{f_n}^* = 0 & 0 & 0 & 0 & & 0 \\
                & \F{f_{n-1}}^* & 0 & 0 & 0 & & 1 \\
                \Ftext{Thrust}\op & \F{f_{n-2}}^* & 0 & 1 & 1 & & 1 \\
                & \vdots &  &  &  & \ddots & \\
                & \F{f_1}^* & 1 & 1 & 1 & & 1 \\
            \end{block}
        \end{blockarray}
    \end{equation}
    %
    Suppose we have tested or are given the performance data of a few different engines, as possible solutions to the ``engine'' design problem, each with a fixed optimal fuel-thrust value.
    To illustrate the monotonicity assumption, we can render the data of ``engine'' as a graph, as depicted in~\cref{fig:solenginedp}.
    \begin{figure}[h!]
        \centering
        \includesag{50_engine_graph}
        \caption{Graphical representation of the possible solutions of the engine design problem. }
        \label{fig:solenginedp}
    \end{figure}
    
    Note that the shaded regions cover the feasible solution set.
    This feasible solution set is always an \emph{upper set} (\cref{def:upperset}) in~$\Ftext{thrust}\op \cartprod \Rtext{fuel}$, which is another way of characterizing the monotonicity of the design problem.
    The optimal solutions, indicated by dots, form an \emph{antichain} of solutions.
    We will come back to antichains when discussing how to compute optimal solutions of design problems.

\end{example}


%\JT{Do we need 2 large examples, or can we move the battery example to the notes?}
%
%\newcommand{\cCapacity}{\text{Capacity}}
%\newcommand{\cMass}{\text{Mass}}
%
%\begin{example}[Battery]
%   A battery is a store of elecrical energy.
%   We are interested in the relation between the capacity of the battery, measured
%   in joules, and the mass of the battery, measured in grams.
%   We will model the battery as a design problem
%   \[
%       \text{battery} : \cCapacity \profto \cMass,
%   \]
%   with the two posets $\cCapacity$ and $\cMass$ defined as
%   $\cCapacity \definedas \langle\mathbb{R}_+^{\text{J}}, \leq\rangle$
%   and $\cMass \definedas \langle\mathbb{R}_+^{\text{g}}, \leq\rangle$, where ``$\leq$''' is the usual
%   order on $\mathbb{R}_+$.
%
%This is the corresponding diagram:
%   \[
%   \centering
%   \begin{tikzpicture}[oriented WD, bb min width =1.5cm, bby=2ex, bbx=.7cm,bb port length=3pt]
%       \node[bb port sep=0.8, bb={1}{1}, bb name={battery}] (dp) {};
%       \node [black, left = 0.2 of dp] {$\cCapacity$};
%       \node [black, right = 0.2 of dp] {$\cMass$};
%   \end{tikzpicture}
%   \]
%   The concrete representation of the design problem is
%   \[
%       \text{battery} \colon\cCapacity\op\times \cMass \toinPos \Bool,
%   \]
%%
%   In this case, the feasibility question is
%   \[
%           \text{battery}(c^*, m) = \true \quad \Leftrightarrow  \quad
%            \text{a battery of mass $m$ is sufficient to provide the capacity $c^*$.}
%   \]
%   It is easy to convince oneself that this ``$\text{battery}$'' is monotone from basic physics consideration.
%   Monotonicity is equivalent to the following two assertions:
%%
%   \newcommand{\qsmall}[1]{{\color{blue}#1}}
%   \newcommand{\qlarge}[1]{{\color{red}#1}}
%%
%\begin{enumerate}
%\item We can provide a smaller capacity with the same mass:
%   \begin{eqnarray}
%       \text{For all}\  \qlarge{c_2}  \geq_{\cCapacity} \qsmall{c_1},\quad
%       \text{a battery of mass $m$ is sufficient to provide the capacity $\qlarge{c_2}$} \\
%       \Rightarrow
%       \text{a battery of mass $m$ is sufficient to provide the capacity $\qsmall{c_1}$.}
%   \end{eqnarray}
%%
%\item A battery of larger mass can provide the same capacity:
%   \begin{eqnarray}
%       \text{For all}\ \qlarge{m_2} \geq_{\cMass} \qsmall{m_1},\quad
%       \text{a battery of mass $\qsmall{m_1}$ is sufficient to provide the capacity $c^*$} \\
%       \Rightarrow
%       \text{a battery of mass $\qlarge{m_2}$ is sufficient to provide the capacity $c^*$.}
%   \end{eqnarray}
%\end{enumerate}
%
%   Assume that there is a linear relation between mass and capacity,
%   and such relation is described by the energy density~$\rho$ [Wh/kg].
%   Then the minimum mass to provide~$c^*$ is~$m_\text{min} = m_0 + c^* / \rho.$
%   So we have
%   \[
%       \text{battery}(c^*, m) = \true \quad \Leftrightarrow \quad  m \geq m_0 + c^* / \rho.
%   \]
%   A visualization of the feasible set $\feasibleset{\text{battery}}$ is in \cref{fig:battery-1}.
%   Monotonicity means that if we fix one point $\tup{c^*,m}$,
%   if we increase $c^*$, we can only see at most one transition, from feasible to unfeasible.
%   Similarly, if we increase $m$, there is at most one transition from unfeasible to feasible.
%
%\end{example}
%
%\begin{figure}[h!]
%    \todo{figure battery-1}
%    \caption{}
% \label{fig:battery-1}
%\end{figure}
