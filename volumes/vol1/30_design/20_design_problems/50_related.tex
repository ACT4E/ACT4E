% !TEX root = chapter-standalone.tex

\section{Discussion of related work}
\label{sec:design-problems-related}

\subsection{Theory of design}

Modern engineering has long since recognized the two ideas of modularity and hierarchical decomposition, yet there exists no general quantitative theory of design that is applicable to different domains.
Most of the works in the theory of design literature study abstractions that are meant to be useful for a human designer, rather than an automated system.
For example, a \emph{function structure} diagram~\cite[p.
    32]{pahl07} decomposes the function of a system in subsystems that exchange energy,
materials, and signals, but it is not a formal representation.
From the point of view of the theory of design, the contribution of this work is that the \emph{design problem} abstraction developed, where one takes functionality and resources as the interfaces for the subsystems, is at the same time (1) mathematically precise; (2)~intuitive to understand; and (3)~leads to tractable optimization problems.

This work also provides a clear answer to one long-standing issue in the theory of design: the inter-dependence between subsystems, (in other words, cycles in the co-design graph).
Consider, as an example, Suh's theory of \emph{axiomatic design~}\cite{suh01}, in which the first ``axiom'' is to keep the design requirements orthogonal (in other words, do not introduce cycles).
This work shows that it is possible to deal elegantly with recursive constraints.

\subsection{Partial Order Programming}

In ``Partial Order Programming''~\cite{parkerjr89partial} Parker studies a hierarchy of optimization problems that can be represented as a set of partial order constraints.
The main interest is to study partial order constraints as the means to define the semantics of programming languages and for declarative approaches to knowledge representation.

In Parker's hierarchy, CDPIs are most related to the class of problems called \emph{continuous monotone partial order program} (CMPOP).
CMPOPs are the least specific class of problems studied by Parker for which it is possible to obtain existence results and a systematic solution procedure.
CDPIs subsume CMPOPs.
A CMPOP is an CDPIs where: 1)~All functionality and resources belong to the same poset~\posA ($\funsp_{v}=\ressp_{v}=\posA$);
2)~Each functionality/resource relation is a simple map, rather than a multivalued relation; 3)~There are no dangling functionality edges in the co-design diagram ($\funsp=\One$).

In a CDPI, each DP is described by a \SY{Scott continuous} map~$\ftor:\funsp\rightarrow\Aressp$ which maps one functionality to a minimal set of resources.
By contrast, in a CMPOP an operator corresponds to a \SY{Scott continuous} map~$\ftor:\funsp\rightarrow\ressp$.
The consequence is that a CMPOP has a unique solution~\cite[Theorem 8]{parkerjr89partial},
while an CDPI can have multiple minimal solutions (or none at all).
\todotext{@AC: note here we use the other notation with \SY{antichains} }

\subsection{Abstract interpretation}

The methods used from order theory are the same used in the field of \emph{abstract interpretation}~\cite{cousot14abstract}.
In that field, the \SY{least fixed point} semantics arises from problems such as computing the sets of reachable states.
Given a starting state,
one is interested to find a subset of states that is closed under the dynamics (in other words, a fixed point), and that is the smallest that contains the given initial state (in other words, a \emph{least} fixed point).
Reachability and other properties lead to considering systems of equation of the form
\begin{align}
    x_{i} & =\varphi_{i}(x_{1},\,\dots\,,x_{i},\,\dots\,,x_{n}),\quad i=1,\dots,n,\label{eq:ai}
\end{align}
where each value of the index $i$ is for a control point of the program, and~$\varphi_{i}$ are \SY{Scott continuous} functions on the abstract \SY{lattice} that represents the properties of the program.
In the simplest case, each~$x_{i}$ could represent intervals that a variable could assume at step~$i$.
By applying the iterations, one finds which properties can be inferred to be valid at each step.

We can repeat the same considerations we did for Parker's CMPOPs \vs CDPIs.
In particular, in a CDP we deal with multivalued maps, and there is more than one solution.

In the field of abstract interpretation much work has been done towards optimizing the rate of convergence.
The order of evaluation in~\cref{eq:ai} does not matter.
Asynchronous and ``chaotic'' iterations were proposed early~\cite{cousot77asynchronous} and are still object of investigation~\cite{bourdoncleefficient}.
To speed up convergence, the so-called ``widening'' and ``narrowing'' operators are used~\cite{cortesi11widening}.
The ideas of chaotic iteration, widening, narrowing, are not immediately applicable to CDPs, but it is a promising research direction.

