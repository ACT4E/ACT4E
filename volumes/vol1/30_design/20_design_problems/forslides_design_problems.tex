\showslides{

    \section{Lecture slides materials}

    \begin{forslides}
        \subsection{Section: Design problems}

        \begin{equation}
            \label{eq:fun_poset}
            \tup{\funsp, {{\preceq_{\funsp}}}}
        \end{equation}
        \begin{equation}
            \label{eq:res_poset}
            \tup{\ressp,{{ \preceq_{\ressp}}}}
        \end{equation}
        \begin{equation}
            \label{eq:decl_dp}
            \adp \colon \funsp \profto \ressp
        \end{equation}
        \begin{equation}
            \label{eq:decl_dp_1}
            \adpa \colon \funspa \profto \resspb
        \end{equation}
        \begin{equation}
            \label{eq:decl_dp_std}
            \adp \colon \funspop \Ptimes \ressp \toinPos \Bool
        \end{equation}
        \begin{equation}
            \label{eq:decl_dp_2}
            \adpb \colon \funspb \profto \resspc
        \end{equation}
        \begin{equation}
            \label{eq:decl_dp_3}
            (\adpa\dpithen \adpb) \colon \funspa \profto \resspc
        \end{equation}
        \begin{equation}
            \label{eq:opt_1}
            \res_k\setin \tup{\ressp_k,\posleqof{\ressp_k}}
        \end{equation}
        \begin{equation}
            \label{eq:opt_2}
            \fun_k\setin \tup{\funsp_k,\posleqof{\funsp_k}}
        \end{equation}
        \begin{equation}
            \label{eq:opt_3}
            \adp_k\colon \funsp_k \profto \ressp_k
        \end{equation}
        \begin{equation}
            \label{eq:opt_4}
            \adp_k(\fun_k^*,\res_k)=\true
        \end{equation}
        \begin{equation}
            \label{eq:opt_5}
            \res_i\posleq \fun_j
        \end{equation}
        \begin{equation}
            \label{eq:opt_6}
            \Min_{\posleq}\bar{\res}
        \end{equation}
        \begin{equation}
            \label{eq:genfun_1}
            \ressp
        \end{equation}
        \begin{equation}
            \label{eq:genfun_2}
            \funsp
        \end{equation}
        \begin{equation}
            \label{eq:genfun_3}
            \ressp_1
        \end{equation}
        \begin{equation}
            \label{eq:genfun_3_1}
            \ressp_2
        \end{equation}
        \begin{equation}
            \label{eq:genfun_3_2}
            \ressp_1\Ptimes \ressp_2
        \end{equation}
        \begin{equation}
            \label{eq:genfun_4}
            \funsp_1
        \end{equation}
        \begin{equation}
            \label{eq:genfun_5}
            \funsp_2
        \end{equation}
        \begin{equation}
            \label{eq:genfun_6}
            \fun_1
        \end{equation}
        \begin{equation}
            \label{eq:genfun_7}
            \fun_2
        \end{equation}
        \begin{equation}
            \label{eq:genfun_8}
            \res
        \end{equation}
        \begin{equation}
            \label{eq:genfun_9}
            \begin{aligned}
                \ftor_\mathrm{loop}\colon \funsp_1 & \mto \antichains(\ressp) \\
                \fun_1                             & \mapsto \text{least-fixed-point}(\Phi_{\fun_1})
            \end{aligned}
        \end{equation}
        \begin{equation}
            \label{eq:genfun_10}
            \begin{aligned}
                \Phi_{\fun_1}\colon \antichains(\ressp) & \mto \antichains(\ressp) \\
                \R{S}                                   & \mapsto \Min_{\posleqof\ressp} \bigsetunion_{\res \setin \R{S}}\ftor(\fun_1,\res)\setintersection \upit \res
            \end{aligned}
        \end{equation}
        \begin{equation}
            \label{eq:genfun_11}
            \R{S}\subset \antichains(\ressp)
        \end{equation}
        \begin{equation}
            \label{eq:genfun_12}
            \R{S_0}=\makeset{ \posbot_\ressp}
        \end{equation}
        \begin{equation}
            \label{eq:genfun_13}
            \R{S_{k+1}}=\Phi_{\fun_1}(\R{S_k})
        \end{equation}
        \begin{equation}
            \label{eq:genfun_14}
            \posbot_{\ressp}
        \end{equation}
        \begin{equation}
            \label{eq:imp_comp}
            \impsp \setsubseteq \cObj{ \impsp_{1}, \impsp_{2}}
        \end{equation}
        \begin{equation}
            \label{eq:dp_mob_4}
            \ftor_k\colon \funsp_k \mto \antichains \ressp_k
        \end{equation}
        \begin{equation}
            \label{eq:dp_mob_5}
            \ftor\colon \funsp \mto \antichains \ressp
        \end{equation}
        \begin{equation}
            \label{eq:dp_mob_6}
            \adpa\colon \funspa \mto \uppersets \resspb
        \end{equation}
        \begin{equation}
            \label{eq:dp_mob_7}
            \adpb\colon \funspb \mto \uppersets \resspc
        \end{equation}
        \begin{equation}
            \label{eq:dp_mob_8}
            (\adpa\dpithen \adpb) \colon \funspa \mto \uppersets \resspc
        \end{equation}
        \begin{equation}
            \label{eq:cont_dyn}
            \begin{aligned}
                \text{d} \mat{x}_t & =\mat{A}\mat{x}_t\text{d} t+\mat{B}\mat{u}_t \text{d} t+\mat{E} \text{d} \mat{w}_t \\
                \text{d} \mat{y}_t & = \mat{C}\mat{x}_t \text{d} t+\mat{G}\text{d} \mat{v}_t,
            \end{aligned}
        \end{equation}

        \begin{equation}
            \label{eq:general_continuous_cost}
            J=\lim_{T\to \infty} \frac{1}{T}\Evalue\makeset{\int_{0}^{T} \pars{\pars{ \mat{x}_t\mattransp \mat{Q} \mat{x}_t}+\pars{\mat{u}_t\mattransp \mat{R}\mat{u}_t}} \text{d} t},
        \end{equation}

        \begin{lemma}
            \label{lem:lqgstandard}
            The optimal control law for the LQG problem is~$\mat{u}_t^\star =-\mat{K}\hat{\mat{x}}_t=-\mat{R}^{-1}\mat{B}^*\bar{\mat{S}}\hat{\mat{x}}_t$, where $\hat{\mat{x}}_t$ is the unbiased minimum-variance estimate of $\mat{x}_t$ given previous measurements and $\bar{\mat{S}}\setin \posdef$ solves the Riccati equation
            \begin{equation}
                \label{eq:cont_ric_1}
                \mat{SA}+\mat{A}^*\mat{S}-\mat{SBR}^{-1}\mat{B}^*\mat{S}+\mat{Q}=\mat{0}.
            \end{equation}
            The minimum cost~$J^\star$ achieved by the optimal control is:
            %\footnote{Note that~\cite{davis1977} contains a typo at p.188 (one extra $\mat{V}^{-1}$ factor). Instead,~\cite{kwakernaak1972} has a cleaner derivation and exposition.}
            \begin{equation}
                \label{eq:contmincost}
                \begin{aligned}
                    J^\star & =\mattrace{(\bar{\mat{S}}\bar{\mat{\Sigma}}\mat{C}^*\mat{V}^{-1}\mat{C}\bar{\mat{\Sigma}}+\bar{\mat{\Sigma}}\mat{Q})} \\
                            & =\mattrace{(\bar{\mat{\Sigma}}\bar{\mat{S}}\mat{BR}^{-1}\mat{B}^*\bar{\mat{S}}+\bar{\mat{S}}\mat{W})},
                \end{aligned}
            \end{equation}
            where~$\bar{\mat{\Sigma}}\setin \posdef$ is the solution of the Riccati equation
            \begin{equation}
                \label{eq:cont_ric_2}
                \mat{A\Sigma} + \mat{\Sigma A}^*-\mat{\Sigma C}^*\mat{V}^{-1}\mat{C\Sigma}+\mat{W}=\mat{0}.
            \end{equation}
        \end{lemma}

        \begin{equation}
            \label{eq:ptrack}
            \track=\lim_{t\to \infty}\Evalue\makeset{\mat{x}_t\mattransp \mat{Q}\mat{x}_t}.
        \end{equation}

        \begin{equation}
            \label{eq:peffort}
            \effort=\lim_{t\to \infty}\Evalue\makeset{\mat{u}_t\mattransp \mat{R} \mat{u}_t }
        \end{equation}

        \begin{lemma}
            \label{lem:precision}
            The metrics $\track$ and $\effort$ can be written as
            \begin{equation}
                \begin{aligned}
                    \lim_{t\to \infty}\Evalue\makeset{ \mat{x}_t\mattransp \mat{Q}_0\mat{x}_t} & =\mattrace(\mat{Q}_0\pars{ \mat{\Sigma} +\mat{F}}), \\
                    \lim_{t\to \infty}\Evalue\makeset{ \mat{u}_t\mattransp \mat{R}_0\mat{u}_t} & =\mattrace(\mat{S}\mat{B}^*\mat{R}^{-1}\mat{R}_0\mat{R}^{-1}\mat{B}\mat{S}\mat{F}),
                \end{aligned}
            \end{equation}
            where
            $\mat{\Sigma}$ solves the Riccati equation for estimation,~$\mat{F}$ solves the Lyapunov equation
            \begin{equation}
                \label{eq:lyapct}
                \pars{ \mat{A}-\mat{B}\mat{K}}\mat{F}+\mat{F}\pars{\mat{A}-\mat{B}\mat{K}}^*+\mat{L}\mat{V}\mat{L}^*=\mat{0},
            \end{equation}
            $\mat{S}$ solves the Riccati equation for control, and~$\mat{L}=\mat{\Sigma}\mat{C}^*\mat{V}^{-1}$ is the Kalman gain.
        \end{lemma}

        \begin{lemma}
            \label{lem:codesigncnt_1}
            Let~$\mat{Q}(\alpha)=\alpha \mat{Q}_0$ and~$\mat{R}(\alpha)=\frac{1}{\alpha}\mat{R}_0$, $\alpha\setin \mathbb{R}_+$.
            Let~$\mat{u}^\star(\alpha)$ be the solution of the LQG problem with $\mat{Q}(\alpha)$ and $\mat{R}(\alpha)$.
            Then, under optimal control we have:
            \begin{itemize}
                \item $P_\mathrm{track}(\alpha)$ is decreasing with $\alpha$
                      increasing.
                \item $P_\mathrm{effort}(\alpha)$ is increasing with $\alpha$ increasing.
            \end{itemize}
        \end{lemma}

        \begin{lemma}
            \label{lem:cont_V_W}
            The solution of \cref{eq:cont_ric_2} is monotonic in $\mat{V}$ and $\mat{W}$, \ie, $\tup{\mat{V},\mat{W}}\preceq \tup{\mat{V}',\mat{W}'} \Rightarrow \mat{\Sigma}(\mat{V},\mat{W})\preceq \mat{\Sigma}(\mat{V}',\mat{W}')$.
        \end{lemma}

        \begin{lemma}
            \label{lem:codesigncnt_2}
            Consider the situation of \cref{lem:codesigncnt_1}:
            \begin{itemize}
                \item Fix $P_\mathrm{track}$.
                      $P_\mathrm{effort}$ is monotonic in $\mat{W}$ and in $\mat{V}$.
                \item Fix $P_\mathrm{effort}$. $P_\mathrm{track}$ is monotonic in $\mat{W}$ and in $\mat{V}$.
            \end{itemize}
        \end{lemma}

        \begin{equation}
            \label{eq:lqg_simple}
            \begin{tikzpicture}[DP, dp port sep =3]
                \node(ref) at (0,0){};
                \node[dp={2}{2},below=0.5cm of ref] (cnt) {LQG};
                \draw[runconn, runame={tracking error $\track$}, relres=above, posres=2.6] (cnt_res1){};
                \draw[runconn, runame={control effort $\effort$},relres=above,posres=2.6] (cnt_res2){};
                \draw[funconn, funame={observation noise $\mat{V}$},relfun=above, posfun=2.3] (cnt_fun1){};
                \draw[funconn, funame={system noise $\mat{W}$},relfun=above, posfun=2.3] (cnt_fun2){};
            \end{tikzpicture}
        \end{equation}

        \begin{equation}
            \label{eq:brown_1}
            \mat{v}_t
        \end{equation}

        \begin{equation}
            \label{eq:brown_2}
            \mat{w}_t
        \end{equation}

        \begin{equation}
            \label{eq:cov_1}
            \mat{W}=\mat{EE^*}
        \end{equation}
        \begin{equation}
            \label{eq:cov_2}
            \mat{V}=\mat{GG^*}
        \end{equation}

        \begin{equation}
            \label{eq:optimal_control}
            \mat{u}_t^\star =-\mat{K}\hat{\mat{x}}_t=-\mat{R}^{-1}\mat{B}^*\bar{\mat{S}}\hat{\mat{x}}_t
        \end{equation}

        \begin{equation}
            \label{eq:x_estimate}
            \hat{\mat{x}}_t
        \end{equation}
        \begin{equation}
            \label{eq:x_real}
            \mat{x}_t
        \end{equation}

        \begin{equation}
            \label{eq:ric_1}
            \bar{\mat{S}}
        \end{equation}
        \begin{equation}
            \label{eq:ric_2}
            \bar{\mat{\Sigma}}
        \end{equation}

        \begin{equation}
            \label{eq:mon_est}
            \tup{\mat{V},\mat{W}}\preceq \tup{\mat{V}',\mat{W}'} \Rightarrow \mat{\Sigma}(\mat{V},\mat{W})\preceq \mat{\Sigma}(\mat{V}',\mat{W}')
        \end{equation}
        %
        % \begin{definition}
        %     \label{def:poset_cont}
        %     A \emph{poset} is a tuple $\tup{P,\preceq_P}$, where $P$ is a set and $\preceq_P$ is a partial order, defined as a \SY{reflexive}, \SY{transitive}, and antisymmetric relation.
        % \end{definition}

        \begin{equation}
            \label{eq:posreals}
            \tup{\nonNegReals,{{\Rleq}}}
        \end{equation}
        \begin{equation}
            \label{eq:nats}
            \natswithleq
        \end{equation}

        \begin{equation}
            \label{eq:matrix_poset}
            \mat{A}=\begin{bmatrix}
                1 & 0 \\0& 1
            \end{bmatrix}, \quad \mat{B}=\begin{bmatrix}
                2 & 0 \\0& 1
            \end{bmatrix},\quad \mat{C}=\begin{bmatrix}
                2 & 0 \\0& 0.5
            \end{bmatrix}
        \end{equation}

        \begin{equation}
            \label{eq:lqg_delay}
            \begin{tikzpicture}[DP, dp port sep =3]
                \node[dp={3}{2}] (cnt) {LQG};
                \draw[runconn, runame={tracking error $\track$}, relres=above, posres=2.6] (cnt_res1){};
                \draw[runconn, runame={control effort $\effort$},relres=above,posres=2.6] (cnt_res2){};
                \draw[funconn, funame={observation noise $\mat{V}$},relfun=above, posfun=2.3] (cnt_fun1){};
                \draw[funconn, funame={system noise $\mat{W}$},relfun=above, posfun=2.3] (cnt_fun2){};
                \draw[funconn, funame={delay},relfun=above, posfun=1] (cnt_fun3){};
            \end{tikzpicture}
        \end{equation}

        \begin{equation}
            \label{eq:lqg_disc}
            \begin{tikzpicture}[DP, dp port sep = 1.5]
                \node[dp={4}{2}] (cnt) {LQG Control};
                \draw[runconn, runame={tracking error $\track$}, relres=right] (cnt_res1){};
                \draw[runconn, runame={control effort $\effort$},relres=right] (cnt_res2){};
                \draw[funconn, funame={observation noise $\mat{V}$},relfun=left] (cnt_fun1){};
                \draw[funconn, funame={system noise $\mat{W}$},relfun=left] (cnt_fun2){};
                \draw[funconn, funame={delay $d$},relfun=left] (cnt_fun3){};
                %\draw[funconn, funame={sampling period $\delta$},relfun=left] (cnt_fun4){};
                \draw[funconn, funame={dropping probability $p$},relfun=left] (cnt_fun4){};
            \end{tikzpicture}
        \end{equation}

        \begin{equation}
            \label{eq:poset_order_mat}
            \mat{A}\posleq \mat{B} \Leftrightarrow (\mat{B}-\mat{A})\setin \mathcal{P}^n, \quad \mat{A},\mat{B}\setin \mathcal{P}^n
        \end{equation}

        \subsection{Section: DPI}

        \begin{equation}
            \label{eq:adpa}
            \adpa
        \end{equation}
        \begin{equation}
            \label{eq:adpb}
            \adpb
        \end{equation}
        \begin{equation}
            \label{eq:adp1}
            \adpa: \FB{A} \profto  \RB{B}
        \end{equation}
        %
        \begin{equation}
            \label{eq:adp1pf}
            \adpa: \FB{A}\op \cartprod  \RB{B}  \toinPos \Bool
        \end{equation}
        %
        \begin{equation}
            \label{eq:adp1us}
            \adpa: \FB{A} \to \uppersets \RB{B}
        \end{equation}
        %
        \begin{equation}
            \label{eq:adp2pf}
            \adpb: \FB{B}\op \cartprod  \RB{C}  \toinPos  \Bool
        \end{equation}
        %
        \begin{equation}
            \label{eq:adp2us}
            \adpb: \FB{B} \to \uppersets \RB{C}
        \end{equation}
        %
        \begin{equation}
            \label{eq:adp3us}
            \adpb: \FB{C} \to \uppersets \RB{D}
        \end{equation}
        %
        \begin{equation}
            \label{eq:adp12pf}
            (\adpa\dpithen\adpb): \FB{A}\op \cartprod  \RB{C}  \toinPos  \Bool
        \end{equation}
        %
        \begin{equation}
            \label{eq:adp12us}
            (\adpa\dpithen\adpb): \FB{A}  \to\uppersets \RB{C}
        \end{equation}
        %
        \begin{equation}
            \label{eq:adpM12us}
            (\adpa\dpithen\adpb):(\FB{A} \cartprod \FB{C})  \to\uppersets (\RB{C} \cartprod \RB{D})
        \end{equation}
        %
        \begin{equation}
            \label{eq:adp12pfexpr}
            \langle\F{a}^*, \R{c}\rangle \mapsto \bigvee_{b_1\leq b_2} \adpa(\F{a}\op, b_1) \wedge \adpb(b_2\op, \R{c})
        \end{equation}
        %
        \begin{equation}
            \label{eq:adp2}
            \adpb: \FB{B} \profto  \RB{C}
        \end{equation}
        %
        \begin{equation}
            \label{eq:adp1adp2}
            \adpa\dpithen\adpb: \FB{A} \profto  \RB{C}
        \end{equation}
        %
        \begin{equation}
            \label{eq:FA}
            \FB{A}
        \end{equation}
        \begin{equation}
            \label{eq:FB}
            \FB{B}
        \end{equation}
        \begin{equation}
            \label{eq:FC}
            \FB{C}
        \end{equation}
        \begin{equation}
            \label{eq:FD}
            \FB{D}
        \end{equation}
        \begin{equation}
            \label{eq:RA}
            \RB{A}
        \end{equation}
        \begin{equation}
            \label{eq:RB}
            \RB{B}
        \end{equation}
        \begin{equation}
            \label{eq:RC}
            \RB{C}
        \end{equation}
        \begin{equation}
            \label{eq:RD}
            \RB{D}
        \end{equation}
        \begin{equation}
            \label{eq:impsp1}
            \impsp_1
        \end{equation}
        \begin{equation}
            \label{eq:impsp2}
            \impsp_2
        \end{equation}
        \begin{equation}
            \label{eq:impsp12}
            \impsp_{1} \cartprod \impsp_{2}
        \end{equation}
        \begin{equation}
            \label{eq:dproba}
            \adp \colon \funsp \profto \ressp
        \end{equation}
        %
        \begin{equation}
            \label{eq:dprob2}
            \ftor \colon \funsp \profto \Uressp
        \end{equation}

    \end{forslides}

}