% !TEX root = chapter-standalone.tex

\section{Queries}
\label{sec:design-problems-querying}

\linkvideo{spring2021-design:design:queries} % Design queries
A DPI is a model to which we can associate a family of optimization problems.
While in previous examples we covered the problem ``feasibility'', we still miss \FixFunMinRes, \FixResMaxFun, and \FeasibleImp.

The first can be translated to ``Given a lower bound on the functionality~\fun, what are the implementations that have minimal resource usage?''~(\cref{fig:setup-1}).

\begin{problem}[\FixFunMinRes]
\label{prob:FixFunMinRes}
Given~$\fun\setin\funsp$, find the implementations in~\impsp that realize the functionality~\fun (or higher) with minimal resources, or provide a proof that there are none:
\begin{equation}
    \label{eq:objective}
    \begin{cases}
        \with          & \imp\setin\impsp,       \\
        \Min_{\resleq} & \res,                   \\
        \subto         & \res=\req(\imp),        \\
                       & \fun\funleq\prov(\imp).
    \end{cases}
\end{equation}
\end{problem}

%\captionsideleft{\label{fig:setup-1}}{\includegraphics[scale=0.33]{gmcdp_setup_query_f}}
\begin{figure*}
    \centering
    \includesag{funresuplow_1}
    \caption{}
    \label{fig:setup-1}
\end{figure*}

\begin{remark}[Minimal \emph{vs} least solutions]
    Note the use of~$\Min_{\resleq}$ in~\cref{eq:objective},
    which indicates the set of minimal (non-dominated) elements according to~$\resleq$, rather than~$\min_{\resleq}$, which would presume the existence of the least element.
    In all problems in this paper, the goal is to find the optimal trade-off of resources (``Pareto front'').
    So, for each~\fun, we expect to find an \SY{antichain}~${\colR R}\setin\Aressp$.
    We will see that this formalization allows an elegant way to treat multi-objective optimization problems.
    The algorithm to be developed will directly solve for the set~${\colR R}$, without resorting to techniques such as \emph{scalarization}, and therefore is able to work with arbitrary \SY{posets}, possibly discrete.
\end{remark}

In an entirely symmetric fashion, we could fix an upper bound on the resource usage, and then maximize the functionality provided~(\cref{fig:funresuplow_2}).
The formulation is entirely dual, in the sense that it is obtained from \cref{eq:objective} by swapping~$\Min$ with~$\Max$, \funsp~with~\ressp, and $\prov$~with~$\req$.

\begin{problem}[\FixResMaxFun]
\label{prob:FixResMaxFun}
Given~$\res\setin\ressp$, find the implementations in~\impsp that requires~\res (or lower) and provide the maximal functionality, or provide a proof that there are none:
\begin{equation}
    \begin{cases}
        \with          & \imp\setin\impsp,               \\
        \Max_{\funleq} & \fun,                           \\
        \subto         & \fun=\prov(\imp),               \\
                       & \res\posgeq_{\ressp}\req(\imp).
    \end{cases}\label{eq:objective-1}
\end{equation}
\end{problem}

%\captionsideleft{\label{fig:setup_max_f}}{\includegraphics[scale=0.4]{gmcdp_setup_query_r}}

\begin{figure*}
    \centering
    \includesag{funresuplow_2}
    \caption{}
    \label{fig:funresuplow_2}
\end{figure*}

Another type of query is: ``Given a lower bound on the functionality~\fun and an upper bound on the costs~\fun, what are the feasible implementations?

\begin{problem}[\FeasibleImp]
\label{prob:FeasibleImp}
Given~$\fun\setin\funsp$ and $\res\setin\ressp$, find the implementations in~\impsp that requires~\res (or lower) and provide~\fun (or higher)
\begin{equation}
    \label{eq:FeasibleImp}
    \begin{cases}
        \with  & \imp\setin\impsp,               \\
        \subto & \fun \funleq \prov(\imp),       \\
        \subto & \prov(\imp) \resleq \req(\imp), \\
    \end{cases}
\end{equation}
\end{problem}

Another variation is to find only whether there are feasible solutions or not.

\begin{problem}[\Feasibility]
\label{prob:Feasibility}
Given~$\fun\setin\funsp$ and $\res\setin\ressp$, find if~\cref{eq:FeasibleImp} is feasible.
\end{problem}
