% !TEX root = chapter-standalone.tex

\section[Formal engineering design]{Basic concepts of formal engineering design}

\linkvideo{spring2021-design:design:abstract-view-dp} % An abstract view of \SY{design problems} We will informally introduce some basic nomenclature of engineering design~\cite{antonsson2005formal,deweck2011}.
Later, all these concepts will find a formal definition in the language of category theory.

\paragraph{Functionality and functional requirements}
You are an engineer in front of an empty whiteboard, ready to start designing the next product.
The first question to ask is: What is the \emph{purpose} of the product to be designed?
The purpose of the product is expressed by the \emph{functional requirements}, sometimes called
\emph{functional specifications}, \emph{desired behavior}, \emph{objectives}, or simply \emph{function}.

Unfortunately, the word ``function'' conflicts with the mathematical concept.
Therefore, we will talk about \emph{functionality}.
Moreover, we will never use the word ``function'', and instead use \emph{map} to denote the mathematical concept.

\begin{example}
    These are a few examples of functional requirements:
    \begin{itemize}
        \item A car must be able to transport at least~$n \geq 4$ passengers.
        \item A battery must store at least~$\unit[100]{kJ}$ of energy.
        \item An autonomous vehicle should reach at least~$\unit[20]{mph}$ while guaranteeing safety.
    \end{itemize}
\end{example}

\paragraph{Resources and resource constraints}

We call \emph{resources} what we need to pay to realize the given functionality.
In some contexts, these are better called \emph{costs}, or \emph{dependencies}.

\begin{example}
    These are a few examples of resource constraints:
    \begin{itemize}
        \item A car should not cost more than \unit[15,000]{USD}.
        \item A battery should not weigh more than \unit[1]{kg}.
        \item A process should not take more than \unit[10]{s}.
    \end{itemize}
\end{example}

\paragraph{Duality of functionality and resources}

There is an interesting duality between functionality and resources.
When designing systems, one is given functional requirements, as a \emph{lower bound} on the functionality to provide,
and one is given resource constraints, which are an \emph{upper bound} on the resources to use.

As far as design objectives go, most can be understood as either \emph{minimize resource usage}
or \emph{maximize functionality provided}.

This duality between functionality and resources will be at the center of our formalization.

\paragraph{Non-functional requirements}

Functionality and resources do not cover all the requirements-- there is, for example, a large class of \emph{non-functional requirements}~\cite{deweck11} such as the extensibility and the maintainability of the system.
Nevertheless, functionality and resources can express most of the requirements which can be quantitatively evaluated, at least prior to designing, assembling, and testing the entire system.

\paragraph{Implementation space}

The \emph{implementation space} or \textit{design space} is the set of all possible design choices that could be chosen; by \textit{implementation}, or the word ``design'', used as a noun, we mean one particular set of choices.
The implementation space~\impsp is the set over which we are optimizing; an implementation~$\imp\setin\impsp$ is a particular point in that set~(\cref{fig:impspace}).

\begin{figure}[h!]
    \centering
    \includesag{15_imp_space}
    \caption{An \emph{implementation}~\imp is a particular point in the implementation space~\impsp.}
    \label{fig:impspace}
\end{figure}

The interconnection between functionality, resources, and implementation spaces is as follows.
We will assume that, given one implementation, we can evaluate it to know the functionality and the resources spaces~(\cref{fig:FIR}).

\begin{figure}[h!]
    \centering
    \includesag{funimpres_bis}
    \caption{Evaluation of specific implementations to get functionality and resources spaces.\label{fig:FIR}}
\end{figure}

\paragraph{Functional Interfaces and interconnection}
Components are \emph{interconnected} to create a system.
This implies that we have defined the \emph{interfaces} of components, which have the dual function of delimiting when one component ends and another begins, and also to describe exactly what is the nature of their interaction.

We will develop a formalism in which the functionality and resources are the interfaces used for interconnection: two components are connected if the resources required by the first correspond to the functionality provided by the second.

\paragraph{Abstraction}
By \emph{abstraction}, we mean that it is possible to ``zoom out'', in the sense that a system of components can be seen as a component itself, which can be part of larger systems.

\paragraph{Compositionality}
A \emph{compositional} property is a property that is preserved by interconnection and abstraction; assuming each component in a system satisfies that property, also the system as a whole satisfies the property.

\begin{example}
    One can compose two electronic circuits by joining their terminals to obtain another electronic circuit.
    We would say that the property of being an electronic circuit is compositional.
\end{example}

\todojira{150}{\bernina: Would be nice to also give an example of something that is *not* compositional}

\todojira{151}{\bernina: @Gioele: Add watch example?}
