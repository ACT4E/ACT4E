% !TEX root = chapter-standalone.tex


\section{Monoidal DPs}

If the underlying posets of functionality and resources are monoidal (\cref{def:monoidal-poset}), then we can define some canonical DPs.

\begin{definition}[Sum of resources]
    \label{def:sum-resources}
    If the poset~$\posgenA$ is monoidal with monoidal product $\mtimescat$, then the ``\emph{sum}'' of~$n$ copies of $\posgenA$ is a design problem given by
    \begin{equation}
        \definemappos{
                {\Sigma}^n
        }{
            (\posgenA^n)\op \times \posgenA
        }{
            \Bool
        }{
            \tup{ \tupp{ \F{x_1}, \dots, \F{x_n}}^*, \R{\poselb}}
        }{
            (\F{x_1} \mtimescat \dots \mtimescat \F{x_n} \posleq_\posgenA \R{\poselb}).
        }
    \end{equation}
    Clearly $\Sigma^n$ is monotone. Diagrammatically:
    \begin{center}
        \includesag{60_dependent}
    \end{center}
    \todographics{Change the symbol, to read $\Sigma^n$}
\end{definition}

We can do the symmetric construction.

\begin{definition}[Sum of functionalities for monoidal posets]
    \label{def:sum-functionality}
    If the poset~$\posgenA$ is monoidal with monoidal product $\mtimescat$, then the ``\emph{sum}'' of~$m$ copies of $\posgenA$ is a design problem given by
    \begin{equation}
        \definemappos{
                {\Sigma}_m
        }{
            \posgenA \times (\posgenA^m)\op
        }{
            \Bool
        }{
            \tup{ \F{x}^*, \tupp{ \R{y_1}, \dots, \R {y_m}}}
        }{
            (\F{x}   \posleq_\posgenA \R{y_1} \mtimescat \dots \mtimescat \R{y_m}).
        }
    \end{equation}
    Diagrammatically:
    \begin{center}
        \includesag{60_dependent}
    \end{center}
    \todographicsjira{208}{Do corresponding graphics}
\end{definition}


We can now put these in series to obtain the generic DP ${\Sigma}^n_m$ with $n$ functionalities
and $m$ resources.

\begin{center}
    \includesag{60_dependent}
\end{center}

\todographicsjira{208}{Do corresponding graphics - put the two from before in series,
trace dotted line around, show equivalent DP.}


\todojira{209}{Now show that it is the composition of conjoint and companion of "+".}
