% !TEX root = standalone.tex


\section{Companions and conjoint}
We round out our discussion of~\DP by introducing two formulae for transforming monotone maps in~\Pos into design problems in~\DP. Each monotone map~$f$ can be transformed into two design problems, called its \emph{companion}~$\comp{f}$ and \emph{conjoint}~$\conj{f}$. Many of the design problems that we have introduced can be realized as companions and conjoints of appropriate monotone maps.

\begin{definition}[Companion and conjoint]
  \label{def:comp_conj}
  Let~$\posA$ and~$\posB$ be posets, and suppose that~$\mapa\colon\posA \toinPos \posB$ is a monotone map. We define its \emph{companion} in~\DP, denoted~$\comp{\mapa}\colon \F{\posA} \profto \R{\posB}$,
  and its \emph{conjoint}, denoted~$\conj{\mapa}\colon \F{\posB} \profto \R{\posA}$ as
  \begin{equation}
    \comp{\mapa}(\F{\posAel}^*,\R{\posBel})\definedas \mapa(\F{\posAel}) \posleq_\posB \R{\posBel}
    \qquad\text{and}\qquad
    \conj{\mapa}(\F{\posBel}^*,\R{\posAel})\definedas \F{\posBel} \posleq_\posA \mapa(\R{\posAel}).
  \end{equation}
\end{definition}

\begin{lemma}
  \label{prop:comp_conj}
  Both the companion and conjoint constructions from \cref{def:comp_conj} are functorial from~\Pos to~\DP: they preserve identities and composition.
\end{lemma}
\begin{proof}
  We will show that the companion and conjoint are functors of the following forms:
  \begin{equation}
    \comp{(\cdot)}\colon\Pos\to\DP
    \qquad\text{and}\qquad
    \conj{(\cdot)}\colon\Pos\to\DP\op.
  \end{equation}
  First, we see that they send the identity monotone map~$\mapa(\posAel)=\posAel$ to the unit~$\Unit{\posA }$ for any poset~$\posA $, because
  \begin{equation}
    \begin{aligned}
      \comp{\id}(\F{\posAel_1}^*,\R{\posAel_2})&= (\F{\posAel_1} \posleq_{\posA} \R{\posAel_2})\\
      &=\conj{\id}(\F{\posAel_1}^*,\R{\posAel_2}).
    \end{aligned}
  \end{equation}
  Now suppose that~$\mapa\colon  \posA \toinPos \posB $ and~$g\colon \posB \toinPos \posC$ are given. We first show that $\conj{\mapb}\then\conj{\mapa}=\conj{\mapa\then \mapb}$.
  For any~$\posAel\in \posA$ and $\posCel\in \posC$, one has
  \begin{equation}
    \begin{aligned}
      \left(\conj{\mapb}\then \conj{\mapa}\right)(\F{\posAel}^*,\R{\posCel})
      &=\bigvee_{\posBel\in \posB} \conj{\mapb}(\F{\posCel}^*,\R{\posBel})\wedge\conj{\mapa}(\F{\posBel}^*,\R{p})\\
      &=\bigvee_{q\in Q} (\F{r}\ordleq_R g(\R{q})) \wedge (\F{q}\ordleq_Q f(\R{p})) \\
      &= \F{r}\ordleq_R g(f(\R{p}))\\
      &=\left(\conj{f\then g}\right)(\F{r}^*,\R{p}).
    \end{aligned}
  \end{equation}
  Similarly, we can prove that~$\comp{f}\then \comp{g}=\comp{f\then g}$:
  \begin{equation}
    \begin{aligned}
      \left(\comp{f}\then \comp{g}\right)(\F{p}^*,\R{r})&=\bigvee_{q\in Q} \comp{f}(\F{p}^*,\R{q})\wedge\comp{g}(\F{q}^*,\R{r})\\
      &=\bigvee_{q\in Q} (f(\F{p})\ordleq_Q \R{q})\wedge (g(\F{q})\ordleq_R \R{r})\\
      &=g(f(\F{p}))\ordleq_R \R{r}\\
      &=\left(\comp{f\then g}\right)(\F{p}^*,\R{r}).
    \end{aligned}
  \end{equation}
\end{proof}

\begin{example}
  The identity design problem~$\id_A\colon \F{A} \profto \R{A}$ is the companion (and the conjoint) of the identity map~$\id_A'\colon A \toinPos A$. This is easy to check, as
  \begin{equation}
    \begin{aligned}
      \comp{\id}_A'(\F{a_1}^*,\R{a_2})&=\id_A'(\F{a_1})\ordleq \R{a_2}\\
      &=\F{a_1}\ordleq \R{a_2}\\
      &=\id_A(\F{a_1}^*,\R{a_2}).
    \end{aligned}
  \end{equation}
\end{example}

\begin{example}
  The coproduct injections~$\iota_A, \iota_B$ for design problems are the companions of the coproduct injections for the disjoint union.
\end{example}

\begin{example}
  The product projections~$\pi_A, \pi_B$ for design problems are the conjoints of the coproduct injections for the disjoint union.
\end{example}

\paragraph{Interesting implications}
Consider a poset $A$, which can be thought of as a map~$f\colon 1\to A$. By taking the companion of~$f$ one gets
\begin{equation}
  \begin{aligned}
    \comp{f}\colon \F{1}&\profto \R{A}\\
    \tup{\F{1},\R{a}}&\mapsto f(1)\ordleq \R{a}.
  \end{aligned}
\end{equation}
By taking the conjoint, one gets
\begin{equation}
  \begin{aligned}
    \conj{f}\colon \F{A}&\profto \R{1}\\
    \tup{\F{a}^*,\R{1}}&\mapsto \F{a}\ordleq f(\R{1}).
  \end{aligned}
\end{equation}
These two cases represent design problems with either \emph{constant} resources or constant, functionalities, respectively.
