% !TEX root = chapter-standalone.tex


\section{Companions and conjoint}
We round out our discussion of~\DP by introducing two formulae for transforming monotone maps in~\Pos into design problems in~\DP. Each monotone map~$f$ can be transformed into two design problems, called its \emph{companion}~$\comp{f}$ and \emph{conjoint}~$\conj{f}$. Many of the design problems that we have introduced can be realized as companions and conjoints of appropriate monotone maps.

\begin{definition}[Companion and conjoint]
    \label{def:comp_conj}
    Let~$\posA$ and~$\posB$ be posets, and suppose that~$\mapa\colon\posA \toinPos \posB$ is a monotone map. We define its \emph{companion} in~\DP, denoted~$\comp{\mapa}\colon \F{\posA} \profto \R{\posB}$,
    and its \emph{conjoint}, denoted~$\conj{\mapa}\colon \F{\posB} \profto \R{\posA}$ as
    \begin{equation}
        \comp{\mapa}(\F{\posAel}^*,\R{\posBel})\definedas \mapa(\F{\posAel}) \posleq_\posB \R{\posBel}
        \qquad\text{and}\qquad
        \conj{\mapa}(\F{\posBel}^*,\R{\posAel})\definedas \F{\posBel} \posleq_\posA \mapa(\R{\posAel}).
    \end{equation}
\end{definition}

\begin{lemma}
    \label{lem:comp_conj}
    Both the companion and conjoint constructions from \cref{def:comp_conj} are functorial from~\Pos to~\DP: they preserve identities and composition.
\end{lemma}
\begin{proof}
    We will show that the companion and conjoint are functors of the following forms:
    \begin{equation}
        \comp{(\cdot)}\colon\Pos\to\DP
        \qquad\text{and}\qquad
        \conj{(\cdot)}\colon\Pos\to\DP\op.
    \end{equation}
    First, we see that they send the identity monotone map~$\mapa(\posAel)=\posAel$ to the unit~$\Unit{\posA }$ for any poset~$\posA $, because
    \begin{equation}
        \begin{aligned}
            \comp{\id}(\F{\posAel_1}^*,\R{\posAel_2})&= (\F{\posAel_1} \posleq_{\posA} \R{\posAel_2})\\
            &=\conj{\id}(\F{\posAel_1}^*,\R{\posAel_2}).
        \end{aligned}
    \end{equation}
    Now suppose that~$\mapa\colon  \posA \toinPos \posB $ and~$g\colon \posB \toinPos \posC$ are given. We first show that $\conj{\mapb}\then\conj{\mapa}=\conj{\mapa\then \mapb}$.
    For any~$\posAel\in \posA$ and $\posCel\in \posC$, one has
    \begin{equation}
        \begin{aligned}
            \left(\conj{\mapb}\then \conj{\mapa}\right)(\F{\posAel}^*,\R{\posCel})
            &=\bigvee_{\posBel\in \posB} \conj{\mapb}(\F{\posCel}^*,\R{\posBel})\wedge\conj{\mapa}(\F{\posBel}^*,\R{\posAel})\\
            &=\bigvee_{\posBel\in \posB} (\F{\posCel}\posleq_\posC \mapb(\R{\posBel})) \wedge (\F{\posBel}\posleq_\posB \mapa(\R{\posAel})) \\
            &= \F{\posCel}\posleq_\posC \mapb(\mapa(\R{\posAel}))\\
            &=\left(\conj{\mapa\then \mapb}\right)(\F{\posCel}^*,\R{\posAel}).
        \end{aligned}
    \end{equation}
    Similarly, we can prove that~$\comp{\mapa}\then \comp{\mapb}=\comp{\mapa\then \mapb}$:
    \begin{equation}
        \begin{aligned}
            \left(\comp{\mapa}\then \comp{\mapb}\right)(\F{\posAel}^*,\R{\posCel})&=\bigvee_{\posBel\in \posB} \comp{\mapa}(\F{\posAel}^*,\R{\posBel})\wedge\comp{\mapb}(\F{\posBel}^*,\R{\posCel})\\%...
            &=\bigvee_{\posBel\in \posB} (\mapb(\F{\posAel})\posleq_\posB \R{\posBel})\wedge (\mapb(\F{\posBel})\posleq_\posC \R{\posCel})\\
            &=\mapb(\mapa(\F{\posAel}))\posleq_\posC \R{\posCel}\\
            &=\left(\comp{\mapa\then \mapb}\right)(\F{\posAel}^*,\R{\posCel}).
        \end{aligned}
    \end{equation}
\end{proof}

\begin{example}
    The identity design problem~$\id_\posgenA\colon \F{\posgenA} \profto \R{\posgenA}$ is the companion (and the conjoint) of the identity map~$\id_\posgenA'\colon \posgenA \toinPos \posgenA$.
    This is easy to check, as
    \begin{equation}
        \begin{aligned}
            \comp{\id}_\posgenA'(\F{a_1}^*,\R{a_2})&=\id_\posgenA'(\F{a_1})\posleq \R{a_2}\\
            &=\F{a_1}\posleq \R{a_2}\\
            &=\id_\posgenA(\F{a_1}^*,\R{a_2}).
        \end{aligned}
    \end{equation}
\end{example}

\begin{example}
    The coproduct injections~$\iota_\posgenA, \iota_\posgenB$ for design problems are the companions of the coproduct injections for the disjoint union.
\end{example}

\begin{example}
    The product projections~$\pi_\posgenA, \pi_\posgenB$ for design problems are the conjoints of the coproduct injections for the disjoint union.
\end{example}

\paragraph{Interesting implications}
Consider a poset~$\posgenA$, which can be thought of as a map~$\mapa \colon 1\to \posgenA$. By taking the companion of~$\mapa$ one gets
\begin{equation}
    \begin{aligned}
        \comp{\mapa}\colon \F{1}&\profto \R{\posgenA}\\
        \tup{\F{1},\R{a}}&\mapsto \mapa(1)\ordleq \R{a}.
    \end{aligned}
\end{equation}
By taking the conjoint, one gets
\begin{equation}
    \begin{aligned}
        \conj{f}\colon \F{\posgenA}&\profto \R{1}\\
        \tup{\F{a}^*,\R{1}}&\mapsto \F{a}\ordleq f(\R{1}).
    \end{aligned}
\end{equation}
These two cases represent design problems with either \emph{constant} resources or constant, functionalities, respectively.
