% !TEX root = chapter-standalone.tex


\section{DPIs are spans}
\label{sec:spans}


\begin{ctdefinition}[Span]
  \label{def:span}
  Given a category~\CatC, a \emph{\iindex{span}} from an object~$\Obja$ to an object~$\Objb$ is a diagram of the form
  \begin{center}
    \includesag{51_span}
  \end{center}
  where~$\Objc$ is some other object of~\CatC.
\end{ctdefinition}

\begin{example}
  Consider the category~\Berg, introduced in \cref{sec:trekking}. An example of span in this category is reported in~\cref{fig:exmountains}.
  \begin{figure}[h!]
    \begin{center}
      \includesag{130_mountains}
    \end{center}
    \caption{Swiss peaks can be thought of as a span in~\Berg. \label{fig:exmountains}}
  \end{figure}
  Recall that \textsf{Matterhorn Peak}, \textsf{Jungfrau Peak}, and \textsf{Pilatus Peak} are objects of~\Berg, and the arrows are morphisms in~\Berg (paths from one location to the other).
\end{example}

\devel{
\todo{change, dpis are spans}
\begin{definition}[Catalogue]
  \label{def:catalogue}
  A \emph{catalogue} is a span in~\Pos.
  It thus consists of 3 posets~\impsp,~\funsp,~\ressp.
  We call them implementation space, functionality space, and requirements space, respectively. We need to define two maps~$\prov \colon \impsp \to \funsp$ (an implementation \textbf{prov}ides a functionality) and~$\req\colon \impsp \to \ressp$ (an implementation \textbf{req}uires resources):
  \begin{center}
    \includesag{130_catalogue}
  \end{center}
\end{definition}}


%\begin{definition}[Design problem induced by a catalogue]
%  Every catalogue~$\tup{\impsp,\prov,\req}$ \emph{induces} a design problem of the form~$d\colon \funsp \profto \ressp$, with
%  \begin{equation*}
%    \begin{aligned}
%      d\colon \funsp \op \times \ressp &\toinPos \Bool\\
%      \tup{\fun^*,\res}&\mapsto \bigvee_{\imp\in \impsp}\left(\prov(\imp)\funleq \fun \right)\wedge \left( \req(\imp)\resleq \res \right)
%    \end{aligned}
%  \end{equation*}
%\end{definition}
