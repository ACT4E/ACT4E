% !TEX root = chapter-standalone.tex

\section{Union and intersection with companion and conjoints}

\todotext{? cite the section here}

We can also re-define the sum~$\join$ and intersection~$\meet$ using companions and conjoints, which allows us to introduce some useful constructions.

\begin{definition}[Diagonal function]
    \label{def:diagonal-function}
    Define the \emph{diagonal function}~$\diag_\posgenA\colon \posgenA \to \posgenA \cartprod \posgenA$:
    \begin{equation}
        \begin{aligned}
            \diag_\posgenA \colon \posgenA & \to \posgenA \cartprod \posgenA,      \\
            \posgenAel                     & \mapsto \tup{\posgenAel, \posgenAel}.
        \end{aligned}
    \end{equation}
\end{definition}

\begin{definition}[Codiagonal function]
    \label{def:codiagonal-function}
    Define the \emph{codiagonal function}~$\codiag_\posgenA\colon \posgenA+\posgenA \to \posgenA $:
    \begin{equation}
        \begin{aligned}
            \codiag_\posgenA \colon \posgenA + \posgenA & \to \posgenA,       \\
            \disunionA{\posgenAel}                      & \mapsto \posgenAel, \\
            \disunionB{\posgenAel}                      & \mapsto \posgenAel.
        \end{aligned}
    \end{equation}
\end{definition}

Using the diagonal function, \XXX can be rewritten as the following lemma.
\todotext{The dead reference above is in chapters, 155-relations-facts}
\begin{lemma}
    Given~$\adpa, \adpb\colon \funposA \profto \resposB$, we have:
    \begin{equation}
        \adpa \join \adpb =  \conj{\diag}_\posgenA \then (\adpa + \adpb)\then \comp{\diag}_\posgenB.
    \end{equation}
\end{lemma}

\begin{proof}
    First of all, note that
    \begin{equation}
        \begin{aligned}
            \conj{\diag}_\posgenA\colon \funposA          & \profto \resposA+\resposA                \\
            \tup{\funposAel_1^*,\disunionA{\resposAel_2}} & \mapsto \funposAel_1\posleq \resposAel_2 \\
            \tup{\funposAel_1^*,\disunionA{\resposAel_3}} & \mapsto \funposAel_1\posleq \resposAel_3
        \end{aligned}
    \end{equation}
    and
    \begin{equation}
        \begin{aligned}
            \comp{\diag}_\posgenB\colon \funposB+\funposB & \profto \resposB                           \\
            \tup{\disunionA{\funposBel_1}^*,\resposBel_3} & \mapsto \funposBel_1\posleq \resposBel_3   \\
            \tup{\disunionB{\funposBel_2}^*,\resposBel_3} & \mapsto \funposBel_2 \posleq \resposBel_3.
        \end{aligned}
    \end{equation}
    We start by looking at~$\underbrace{\conj{\diag}_\posgenA\then (\adpa+\adpb)}_{\star}\colon \funposA \profto \resposB+\resposB$.
    \begin{equation}
        \begin{aligned}
            \star (\tup{\funposAel^*,\resposBel}) & =\bigvee_{\posAel'\in \posgenA+\posgenA} \conj{\diag}_\posgenA(\tupp{\funposAel^*,\resposAel'})\booland (\adpa+\adpb)(\tupp{\funposAel'^*,\resposBel})                                                                                                                                              \\
                                                  & =\left( \bigvee_{\disunionA{\posgenAel'}\in \posgenA+\posgenA} (\funposAel\posleq \resposAel')\booland \adpa(\funposAel'^*,\resposBel) \right)\boolor \left( \bigvee_{\disunionB{\posgenAel'}\in \posgenA+\posgenA} (\funposAel\posleq \resposAel')\booland \adpb(\funposAel'^*,\resposBel) \right) \\
                                                  & =\adpa(\funposAel^*,\resposBel) \boolor \adpb(\funposAel^*,\resposBel).
        \end{aligned}
    \end{equation}
    %
    Let's now look at~$\star \then \comp{\diag}_\posgenB\colon \funposA \profto \resposB$:
    \begin{equation}
        \begin{aligned}
             & (\star \then \comp{\diag}_\posgenB)(\funposAel^*,\resposBel')                                                                                       \\
             & =\bigvee_{\posgenBel\in \posgenB+\posgenB} \star(\funposAel^*,\resposBel)\booland \comp{\diag}_\posgenB(\funposBel^*,\resposBel')                   \\
             & =\left(\bigvee_{\disunionA{\posgenBel}\in \posgenB+\posgenB} \adpa(\funposAel^*,\resposBel) \booland (\funposBel\posleq \resposBel')\right) \boolor
            \left(\bigvee_{\disunionB{\posgenBel}\in \posgenB+\posgenB} \adpb(\funposAel*,\resposBel) \booland (\funposBel\posleq \resposBel')\right)              \\
             & =\adpa(\funposAel^*,\resposBel')\boolor \adpb(\funposAel^*,\resposBel').
        \end{aligned}
    \end{equation}
\end{proof}

Similarly, using the codiagonal function, one can prove the following.
\begin{lemma}
    Given~$\adpa, \adpb\colon \funposA \profto \resposB$, we have:
    \begin{equation}
        \adpa \meet \adpb = \comp{\codiag}_\posgenA \then(\adpa + \adpb) \then \conj{\codiag}_\posgenB.
    \end{equation}
\end{lemma}
\begin{proof}
    First, note that
    \begin{equation}
        \begin{aligned}
            \comp{\codiag}_\posgenA \colon \funposA        & \profto \resposA\cartprod \resposA                                                \\
            \tup{\funposAel_1^*,\resposAel_2,\resposAel_3} & \mapsto \codiag_\posgenA(\funposAel_1)\posleq \tup{\resposAel_2,\resposAel_3}     \\
                                                           & = \tup{\funposAel_1,\funposAel_1}\posleq \tup{\resposAel_2,\resposAel_3}          \\
                                                           & = (\funposAel_1\posleq \resposAel_2) \booland (\funposAel_1\posleq \resposAel_3).
        \end{aligned}
    \end{equation}
    and
    \begin{equation}
        \begin{aligned}
            \conj{\codiag}_\posgenB \colon \funposB\cartprod \funposB & \profto \resposB                                                                  \\
            \tup{\tup{\funposBel_1,\funposBel_2}^*,\resposBel_3}      & \mapsto \tup{\funposBel_1,\funposBel_2}\posleq \codiag_\posgenB(\resposBel_3)     \\
                                                                      & = (\funposBel_1\posleq \resposBel_3) \booland (\funposBel_2\posleq \resposBel_3).
        \end{aligned}
    \end{equation}
    We start by looking at~$\comp{\codiag}_\posgenA \then (\adpa+\adpb) \colon \funposA\profto \resposB+\resposB$:
    \begin{equation}
        \begin{aligned}
            \left(\comp{\codiag}_\posgenA\then (\adpa+\adpb)\right)\left(\tup{\funposAel^*,\resposBel}\right) & =\bigvee_{} \XXX
        \end{aligned}
    \end{equation}
    \todotext{Adjust signatures, have to find a good way to write it down}
\end{proof}
Unlike~$\conj{\diag} = \mathsf{split}$ and $\comp{\diag} = \mathsf{fuse}$,~$\comp{\codiag}$ and $\conj{\codiag}$ do not have an intuitive diagrammatic representation.
