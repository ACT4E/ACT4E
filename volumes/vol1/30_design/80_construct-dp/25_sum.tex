% !TEX root = chapter-standalone.tex
\section{Sum and intersection with companion and conjoints}

We can also re-define the sum~$\vee$ and intersection~$\wedge$ using companions and conjoints, which allows us to introduce some useful constructions.

\begin{definition}[Diagonal function]
  Define the \emph{diagonal function}~$\Delta_P\colon P \to P \times P$:
  \begin{equation}
    \begin{aligned}
      \Delta_P \colon P & \to P \times P, \\
      p & \mapsto \tup{p, p}.
    \end{aligned}
  \end{equation}
\end{definition}

\begin{definition}[Codiagonal function]
  Define the \emph{codiagonal function}~$\Diamond_P\colon P+P \to P $:
  \begin{equation}
    \begin{aligned}
      \Diamond_P \colon P + P & \to P,  \\
      \disunionA{p} & \mapsto p, \\
      \disunionB{p} & \mapsto p.
    \end{aligned}
  \end{equation}
\end{definition}

\noindent Using the diagonal function, \cref{lem:intersection} can be rewritten as the following lemma.

\begin{lemma}
  Given~$f, g\colon \F{A} \profto \R{B}$, we have:
  \begin{equation}
    f \vee g =  \conj{\Diamond}_A \then (f + g)\then \comp{\Diamond}_B.
  \end{equation}
\end{lemma}

\begin{proof}
  First of all, note that
  \begin{equation}
    \begin{aligned}
      \conj{\Diamond}_A\colon \F{A}&\profto \R{A}+\R{A}\\
      \tup{\F{a_1}^*,\disunionA{\R{a_2}}}&\mapsto \F{a_1}\ordleq \R{a_2}\\
      \tup{\F{a_1}^*,\disunionA{\R{a_3}}}&\mapsto \F{a_1}\ordleq \R{a_3}
    \end{aligned}
  \end{equation}
  and
  \begin{equation}
    \begin{aligned}
      \comp{\Diamond}_B\colon \F{B}+\F{B}&\profto \R{B}\\
      \tup{\disunionA{\F{b_1}}^*,\R{b_3}}&\mapsto \F{b_1}\ordleq \R{b_3}\\
      \tup{\disunionB{\F{b_2}}^*,\R{b_3}}&\mapsto \F{b_2}\ordleq \R{b_3}
    \end{aligned}
  \end{equation}
  We start by looking at~$\underbrace{\conj{\Diamond}_A\then (f+g)}_{\star}\colon \F{A} \profto \R{B}+\R{B}$.
  \begin{equation}
    \begin{aligned}
      \star (\tup{\F{a}^*,\R{b}})&=\bigvee_{a'\in A+A} \conj{\Diamond}_A(\tup{\F{a}^*,\R{a'}})\wedge (f+g)(\tup{\F{a'}^*,\R{b}})\\
      &=\left( \bigvee_{\disunionA{a'}\in A+A} (\F{a}\ordleq \R{a'})\wedge f(\F{a'}^*,\R{b}) \right)\vee \left( \bigvee_{\disunionB{a'}\in A+A} (\F{a}\ordleq \R{a'})\wedge g(\F{a'}^*,\R{b}) \right)\\
      &=f(\F{a}^*,\R{b}) \vee g(\F{a}^*,\R{b}).
    \end{aligned}
  \end{equation}

  Let's now look at~$\star \then \comp{\Diamond}_B\colon \F{A} \profto \R{B}$:
  \begin{equation}
    \begin{aligned}
      &(\star \then \comp{\Diamond}_B)(\F{a}^*,\R{b'})\\
      &=\bigvee_{b\in B+B} \star(\F{a}^*,\R{b})\wedge \comp{\Diamond}_B(\F{b}^*,\R{b'}) \\
      &=\left(\bigvee_{\disunionA{b}\in B+B} f(\F{a}^*,\R{b}) \wedge (\F{b}\ordleq \R{b'})\right) \vee
      \left(\bigvee_{\disunionB{b}\in B+B} g(\F{a}^*,\R{b}) \wedge (\F{b}\ordleq \R{b'})\right)\\
      &=f(\F{a}^*,\R{b'})\vee g(\F{a}^*,\R{b'}).
    \end{aligned}
  \end{equation}
\end{proof}

Similarly, using the codiagonal function, one can prove the following.
\begin{lemma}
  Given~$f, g\colon \F{A} \profto \R{B}$, we have:
  \begin{equation}
    f \wedge g = \comp{\Delta}_A \then(f + g) \then \conj{\Delta}_B.
  \end{equation}
\end{lemma}
\begin{proof}
  First, note that
  \begin{equation}
    \begin{aligned}
      \comp{\Delta}_A \colon \F{A}&\profto \R{A}\times \R{A}\\
      \tup{\F{a_1}^*,\tup{\R{a_2},\R{a_3}}}&\mapsto \Delta_A(\F{a_1})\leq \tup{\R{a_2},\R{a_3}}\\
      &= \tup{\F{a_1},\F{a_1}}\leq \tup{\R{a_2},\R{a_3}}\\
      &= (\F{a_1}\leq \R{a_2}) \wedge (\F{a_1}\leq \R{a_3}).
    \end{aligned}
  \end{equation}
  and
  \begin{equation}
    \begin{aligned}
      \conj{\Delta}_B \colon \F{B}\times \F{B}&\profto \R{B}\\
      \tup{\tup{\F{b_1},\F{b_2}}^*,\R{b_3}}&\mapsto \tup{\F{b_1},\F{b_2}}\leq \Delta_B(\R{b_3})\\
      &= (\F{b_1}\leq \R{b_3}) \wedge (\F{b_2}\leq \R{b_3}).
    \end{aligned}
  \end{equation}
  We start by looking at $\comp{\Delta}_A \then (f+g) \colon \F{A}\profto \R{B}+\R{B}$:
  \begin{equation}
    \begin{aligned}
      \left(\comp{\Delta}_A\then (f+g)\right)\left(\tup{\F{a}^*,\R{b}}\right)&=\bigvee_{}
    \end{aligned}
  \end{equation}
  \todo{Adjust signatures, have to find a good way to write it down}
\end{proof}
Unlike $\conj{\Diamond} = \mathsf{split}$ and $\comp{\Diamond} = \mathsf{fuse}$, $\comp{\Delta}$ and $\conj{\Delta}$ do not have an intuitive diagrammatic representation.
