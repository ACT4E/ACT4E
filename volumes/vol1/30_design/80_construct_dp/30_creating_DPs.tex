% !TEX root = chapter-standalone.tex

\section{Monoidal DPs}

If the underlying \SY{posets} of functionality and resources are monoidal (\cref{def:monoidal-poset}), then we can define some canonical DPs.

\begin{marginfigure}
    \centering
    \vspace{1cm}
    \includesag{60_dependent}
\end{marginfigure}

\begin{definition}[Sum of resources for monoidal posets]
    \label{def:sum-resources}
    If the poset~$\posgenA$ is monoidal with monoidal product $\monpostimes$, then the ``\emph{sum}'' of~$n$ copies of \posA is a \SY{design problem} given by
    \begin{equation}
        \defmapperiodpos{
            \dpsigma{n}{}
        }{
            (\posA^n)\posop \Ptimes \posA
        }{
            \Bool
        }{
            \tup{\tupp{ \F{\posgenAel_1}, \dots, \F{\posgenAel_n}}\Fop, \RposgenBel}
        }{
            (\F{\posgenAel_1} \monpostimes \dots \monpostimes \F{\posgenAel_n} \posleqof\posA \RposgenBel)
        }
    \end{equation}
    % Clearly $\dpsigma{n}{}$ is monotone.
    % Diagrammatically:
    % \equationsag{60_dependent}{eq:60_dependent}%
\end{definition}

We can do the symmetric construction.

\begin{marginfigure}
    \centering
    \vspace{1cm}
    \includesag{60_dependent_bis}
\end{marginfigure}

\begin{definition}[Sum of functionalities for monoidal posets]
    \label{def:sum-functionality}
    If the poset~\posA is monoidal with monoidal product~$\monpostimes$, then the ``\emph{sum}'' of~$m$ copies of \posA is a \SY{design problem} given by
    \begin{equation}
        \defmapperiodpos{
            \dpsigma{}{m}
        }{
            \posA\posop \Ptimes (\posA^m)
        }{
            \Bool
        }{
            \tup{\FposgenAel\Fop, \tupp{ \R{\posgenBel_1}, \dots, \R {\posgenBel_m}}}
        }{
            \ \FposgenAel \posleqof\posA (\R{\posgenBel_1} \monpostimes \dots \monpostimes \R{\posgenBel_m})
        }
    \end{equation}
    % Diagrammatically:
    % %
    % \equationsag{60_dependent_bis}{eq:60_dependent_bis}
\end{definition}

We can now put these in series to obtain the generic DP~$\dpsigma{n}{m}$ with $n$ functionalities and~$m$ resources:
\begin{equation}
    \dpsigma{n}{m} = \dpsigma{n}{} \dpthen \dpsigma{}{m}.
\end{equation}
%
\equationsag{60_dependent_series}{eq:60_dependent_series}

Note that this works with addition, but also with other \SY{associative} operations, such as multiplication, $\max$, \etc
