\section{Lattice homomorphisms}
In this section, we want to abstract the concept of \SY{lattice} and describe a category in which the objects are \SY{lattices} themselves, and the morphisms are \SY{lattice homomorphisms}.
We call this category~\Lat.

\begin{ctdefinition}[Lattice homomorphism]
    \label{def:lattice-homomorphism}
    Given two \SY{lattices}~$\posA, \posB$, a \maindef{lattice homomorphism} is a map~$\mora\colon \posA\mto \posB$ which preserves meets and joins:
    \begin{equation}
        \begin{aligned}
            \mora(\posAel \meetof\posA \posBel) & = \mora(\posAel) \meetof\posB \mora(\posBel), \\
            \mora(\posAel \joinof\posA \posBel) & = \mora(\posAel) \joinof\posB \mora(\posBel).
        \end{aligned}
    \end{equation}
\end{ctdefinition}

\begin{example}
    We consider the \SY{lattices}~$\posA=\tupp{\powerset \makeset{\sbretzel,\sfondue}, \setintersection, \setunion}$, and~$\posB=\tupp{\makeset{\stea\com \swine} \com \max \com \min}$, where~$\min$ ($\max$) refer to the minimum (maximum) alcoholic content of the beverage (assuming Swiss beers, which have alcohol content lower than wine).
    Furthermore, consider
    \begin{equation}
        \label{eq:example_lat_hom_a}
        \defmapperiod{
            \mora
        }{
            \powerset \makeset{\sbretzel,\sfondue}
        }{
            \mto
        }{
            \makeset{\swine, \stea}
        }{
            \setA
        }{
            \begin{cases}
                \stea,  & \sbretzel \setin \setA, \\
                \swine, & \text{ otherwise}
            \end{cases}
        }
    \end{equation}

    The explicit evaluations of~$\mora$ are reported in \cref{tab:example_lattice_hom}.
    \begin{margintable}
        \begin{center}
            \begin{tabular}{c|c}
                \setA                            & $\mora(\setA)$ \\
                \midrule
                $\Emptyset$                      & \swine \\
                $\makeset{\sbretzel}$            & \stea \\
                $\makeset{\sfondue}$             & \swine \\
                $\makeset{ \sbretzel, \sfondue}$ & \stea
            \end{tabular}
        \end{center}
        \caption{\label{tab:example_lattice_hom}}
    \end{margintable}

    Is~$\mora$ a \SY{lattice homomorphism}?
    Yes.
    We can check it explicitly.
    Consider~$\setA,\setB\setsubseteq \makeset{\sbretzel, \sfondue}$.
    We need to show that
    \begin{equation}
        \label{eq:hom_lat_cond_a}
        \mora(\setA\setintersection \setB)=\max\makeset{\mora(\setA),\mora(\setB)}\\
    \end{equation}
    and
    \begin{equation}
        \label{eq:hom_lat_cond_b}
        \mora(\setA\setunion \setB)=\min\makeset{\mora(\setA),\mora(\setB)}.
    \end{equation}

    Technically, we can check every possible pair of~$\setA,\setB$ (only 16 for this case), but that's not efficient.
    First, consider~$\mora(\setA\setintersection \setB)=\swine$.
    Following~\cref{eq:example_lat_hom_a}, this means~$\sbretzel \notsetin \setA\setintersection \setB$ (in other words, either~$\sbretzel \notsetin \setA$,~$\sbretzel \notsetin \setB$, or both).
    At least one of~$\mora(\setA)$ and~$\mora(\setB)$ is~$\swine$, because
    \begin{equation}
        \vmiddle{
            \prftree{
                \sbretzel \notsetin \setA
            }{
                \mora(\setA) = \swine
            }
        }
        \qqand
        \vmiddle{
            \prfperiod{
                \sbretzel \notsetin \setB
            }{
                \mora(\setB) = \swine
            }
        }
    \end{equation}
    This implies~$\max\makeset{\mora(\setA),\mora(\setB)}=\swine$, which verifies \cref{eq:hom_lat_cond_a}.

    If instead, we have~$\mora(\setA\setintersection \setB)=\stea$, then~$\sbretzel \setin \setA\setintersection \setB$, meaning that~$\sbretzel \setin \setA$ and~$\sbretzel \setin \setB$.
    Therefore,~$\max\makeset{ \mora(\setA),\mora(\setB)}=\stea$, which verifies \cref{eq:hom_lat_cond_a}.

    Condition \cref{eq:hom_lat_cond_b} can be verified analogously.
\end{example}

The notion of \SY{lattice homomorphism} can be extended to \SY{bounded lattices}.

\begin{ctdefinition}[Bounded lattice homomorphism]
    \label{def:bounded-lattice-homomorphism}
    Given two \SY{bounded lattices}~$\posA,\posB$, a \maindef{bounded lattice homomorphism} is a \SY{lattice homomorphism}~$\mora\colon \posA\mto \posB$ which also preserves top and bottom:
    \begin{equation}
        \begin{aligned}
            \mora(\posbot_\posA) & =\posbot_\posB, \\
            \mora(\postop_\posA) & =\postop_\posB.
        \end{aligned}
    \end{equation}
\end{ctdefinition}

Note that (bounded) \SY{lattice homomorphisms} are necessarily monotone.

% We are now ready to introduce~\Lat and~$\BoundedLat$.
