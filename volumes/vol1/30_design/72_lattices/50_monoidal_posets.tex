% !TEX root = chapter-standalone.tex

\section{Monoidal posets}
\label{sec:monoidal-posets}
\linkvideo{spring2021-par-feedback:mon-cat:mon-pos} % Monoidal posets

A \SY{monoidal poset} is a \SY{poset} that is also a \SY{monoid}, and in which the monoidal product is a \SY{monotone map} that is compatible with the order.

% In order to ``warm up'', we first consider the definition of a monoidal structure for a poset.
% Posets are a simpler special case of categories, and the following definition is a special case of the general definition of a monoidal category.
\begin{ctdefinition}[Monoidal poset]
    \label{def:monoidal-poset}
    \SYNDEF{monoidal poset}
    A \emph{monoidal structure} on a poset~$\posAdefinition$ is specified by:

    \constit
    \begin{enumerate}
        \item A \SY{monotone map}~$\monpostimes \colon \posA \Ptimes \posA \toinPos \posA$, called the \emph{monoidal product}.
        \item An element~$\idmoncat \setin \posAset$, called the \emph{monoidal unit}.
    \end{enumerate}

    \condit
    \begin{enumerate}
        \item Associativity: for all~$\posela, \poselb, \poselc \setin \posAset$:
              \begin{equation}
                  (\posela \monpostimes \poselb)
                  \monpostimes \poselc =  \posela  \monpostimes ( \poselb \monpostimes \poselc).
              \end{equation}
        \item Left and right unitality: for all~$\posela \setin \posAset$:
              \begin{equation}
                  \idmoncat \monpostimes \posela = \posela
                  \qqand
                  \posela \monpostimes \idmoncat = \posela.
              \end{equation}
    \end{enumerate}

    \noindent A \SY{poset} equipped with a monoidal structure is called a \emph{monoidal poset}.
\end{ctdefinition}

Note that here we are implicitly assuming~$\posA \Ptimes \posA$ as having the \SY{product order} (\cref{def:poset-product}).
In detail, \SY{monotonicity} means that, for all~$\poselna{1}, \poselna{2}, \poselnb{1}, \poselnb{2} \setin \posAset$:
\begin{equation}
    \prfperiod{
        \poselna{1} \posAleq \poselnb{1}}{\poselna{2} \posAleq \poselnb{2}}{ (\poselna{1} \monpostimes \poselna{2}) \posAleq (\poselnb{1} \monpostimes \poselnb{2})}
\end{equation}

\begin{ctdefinition}[Symmetric monoidal poset]
    \label{def:sym-monoidal-poset}
    A \maindef{symmetric monoidal poset} is a \SY{monoidal poset} $\posA=\tup{\posAset, \posAleq, {\monpostimes}, \idmoncat}$ such that, for all $\posela, \poselb \setin \posAset$,
    \begin{equation}
        \posela \monpostimes \poselb = \poselb \monpostimes \posela.
    \end{equation}
\end{ctdefinition}

\begin{example}[Reals with addition]
    \label{ex:monoidal-pos-reals}
    Consider the real numbers~\reals with the \SY{poset} structure given the usual ordering.
    Consider 0 as the monoidal unit and the operation~$+\colon \reals\cartprod \reals\sto \reals$ as monoidal product.
    It is easy to see that the conditions of~\cref{def:monoidal-poset} are satisfied:
    \begin{enumerate}[(a)]
        \item Given~$\posAnel{1},\posAnel{2},\posBnel{1},\posBnel{2}\setin \reals$, we know:
              \begin{equation}
                  \prfperiod{
                      \posAnel{1}\Rleq \posAnel{2}
                  }{\quad}{
                      \posBnel{1}\Rleq \posBnel{2}
                  }{
                      (\posAnel{1}+\posAnel{2}) \Rleq (\posBnel{1}+\posBnel{1})
                  }
              \end{equation}
        \item $0+\posAnel{}=\posAnel{}+0=0$,~$\forall \posAnel{}\setin \reals$.
        \item $(\posAnel{}+\posBel)+\posCel=\posAel+(\posBel+\posCel)$,~$\forall \posAel,\posBel,\posCel\setin \reals$.
    \end{enumerate}
\end{example}

\begin{counterexample}
    Someone proposes now to substitute the monoidal unit in \cref{ex:monoidal-pos-reals} with 1 and the monoidal product with multiplication ``$\cdot$''.
    This does not form a \SY{monoidal poset} anymore.
    To see a simple counterexample, consider the fact that~$\shortminus 5\Rleq 0$ and~$\shortminus 4\Rleq 3$.
    However,~$(\shortminus 5)\cdot (\shortminus 4) \not{\Rleq} 0 \cdot 3$.
\end{counterexample}

\begin{margintable}
    \begin{tabular}{c|cc}
        $\booland$ & $\false$ & $\true$ \\
        \hline
        $\false$   & $\false$ & $\false$ \\
        $\true$    & $\false$ & $\true$
    \end{tabular}
\end{margintable}
\begin{example}[Boolean monoid]
    The booleans form a \SY{monoidal poset} $\tupp{\Bool,\posleqof{\Bool},\true,\booland}$
    with the unit being~$\true$ and the product being $\booland$.
    The action of the monoidal product ``$\booland$'' can be summarized in the table on the side.
    From this table, it is clear that given~$\ela_1\posleqof{\Bool}\elb_1$ and~$\ela_2\posleqof{\Bool} \elb_2$, we have~$\ela_1\booland x_2\posleqof{\Bool} \elb_1\booland \elb_2$ (if you do not believe it, try all possible combinations).
    Furthermore,~$\ela\booland \true=\ela=\true \booland \ela$.
\end{example}

\showslides{
    \begin{forslides}
        \begin{equation}
            \label{eq:monpos_1}
            \prftree{p_1\leq p_2}{q_1\leq q_2}{p_1+p_2\leq q_1+q_2}
        \end{equation}
        \begin{equation}
            \label{eq:monpos_2}
            0+p=p+0=0
        \end{equation}
        \begin{equation}
            \label{eq:monpos_3}
            (p+q)+r=p+(q+r)
        \end{equation}

    \end{forslides}
}

\begin{gradedexercise}[\exname{HwkMonoidalPosets}]
\label{ex:HwkMonoidalPosetsi}

Prove or disprove that the following are monoidal posets: 
\begin{enumerate}
\item The set $\reals$ equipped with the usual ordering, addition as monoidal product, and $0 \setin \reals$ as monoidal unit. 
\item The set $\reals$ equipped with the usual ordering, multiplication as monoidal product, and $1 \setin \reals$ as monoidal unit. 
\end{enumerate}
\end{gradedexercise}

\solutionof{HwkMonoidalPosets}

\begin{gradedexercise}[\exname{HwkInternalHomCancelling}]
\label{ex:HwkInternalHomCancelling}
Let $\posgenA = \tup{\posAset, \posleq, \mtimescat, \idmoncat, \leftinthom{\ }{\ }}$ be a closed monoidal poset. Prove that for any $\ela, \elb, \elc \setin \posAset$, we have 
\begin{equation}
(\leftinthom{\ela }{\elb }) \mtimescat (\leftinthom{\elb }{\elc }) \posleq \leftinthom{\ela }{\elc }. 
\end{equation}

\end{gradedexercise}

\solutionof{HwkInternalHomCancelling}

\begin{gradedexercise}[\exname{HwkInternalLeastUpperBounds}]
\label{ex:HwkInternalLeastUpperBounds}

\begin{enumerate}
\item
Let $\posgenA = \tup{\posAset, \posleq}$ be a poset, and consider a subset $\subA \subseteq \posAset$. A \emph{least upper bound}, or \emph{join}, for $\subA$ is an element $\ela \setin \posAset$ which satisfies the conditions
\begin{enumerate}
\item $\elb \posleq \ela \  \forall \elb \setin \subA$;
\item if $\ela' \setin \posAset$ is such that $\elb \posleq \ela' \  \forall \elb \setin \subA$, then $\ela \posleq \ela'$ must hold. 
\end{enumerate}

Prove that, if a least upper bound of a subset $\subA$ exists, then it is unique. In this case we use the notation $\bigvee \subA$ to denote it. 

\

\item
Suppose that we have a Galois connection 
\begin{equation}
\middlesag{globular_galois_connection}
\end{equation}
between posets $\posgenA = $ and $\posgenB$. Furthermore, assume that $\posgenA$ has all joins, meaning that for any subset $\subA \subseteq \posAset$, the least upper bound $\bigvee \subA$ exists and is an element of $\posAset$. 

Prove that for any subset $\subA \subseteq \posAset$ it holds that 
\begin{equation}
\mora (\bigvee \subA) = \bigvee \makeset{\mora(\ela) \mid \ela \setin \subA}. 
\end{equation}

\end{enumerate}
\end{gradedexercise}

\solutionof{HwkInternalLeastUpperBounds}