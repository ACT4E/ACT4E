% !TEX root = chapter-standalone.tex


\section{Ordering DPs}\label{sec:ordering-order}
We claimed that category theory is an efficient language for talking about \emph{structure}, and showed how the category~\DP could accommodate all the basic operations required by a theory of formal engineering design. Here, we illustrate some of the applications and advantages of~\DP in reasoning about and solving design problems, starting with the fact that that~\DP is compact closed, which allows us to compose and reason about ``design problems of design problems''.

\begin{definition}[Order on~\DP]
    \label{def:DP_loc_pos}
    Suppose that~$\F{\posgenA}$ and~$\R{\posgenB}$ are posets, and that~$\adpa,\adpb \colon \F{\posgenA} \profto \R{\posgenB}$ are design problems.
    We define the order as follows.
    \begin{equation*}
        \prftree[double]{\adpa \posDPleq \adpb}{\adpa(\F{a}^*,\R{b}) \posleq_\Bool \adpb(\F{a}^*,\R{b}), \quad \text{for all }\F{a} \in \F{\posgenA},\R{b} \in \R{\posgenB}}
    \end{equation*}
\end{definition}

We diagrammatically represent the relation~$\adpa \posDPleq \adpb$ as in~\cref{fig:dpimplies}.

\devel{\AC{I would draw
the homset as a bubble and put two points in there. Also note that this figure was
from before when we used the implication arrow instead of $\leq$.}}

\begin{figure}[h!]
    \begin{center}
        \todo{change pic}
    \end{center}
    \caption{The design problem~$\adpa$ implies the design problem $\adpb$. \label{fig:dpimplies}}
\end{figure}

\begin{remark}
    For any functionality-resource pair~$\F{\posgenA},\R{\posgenB}$, we denote by~$\postop_{\F{\posgenA},\R{\posgenB}}$ the design problem which is always feasible.
    We denote by~$\posbot_{\F{\posgenA},\R{\posgenB}}$ the design problem which is never feasible, for any functionality-resource pair~$\F{\posgenA},\R{\posgenB}$.
\end{remark}
\begin{lemma}
    \label{lem:dpboundedlattice}
    $\HomSet{\DP}{\F{\posgenA}}{\R{\posgenB}}$ is a bounded lattice with union~$\vee$ as meet, intersection~$\wedge$ as join, least upper bound~$\top_{\F{\posgenA},\R{\posgenB}}$ and greatest lower bound~$\bot_{\F{\posgenA},\R{\posgenB}}$.
\end{lemma}

\begin{proof}
    First of all, we need to prove that~$\HomSet{\DP}{\F{\posgenA}}{\R{\posgenB}}$ is a poset. To prove this, we check the following:

    \begin{compactitem}
        \item \emph{Reflexivity}: Given~$\adpa\in \HomSet{\DP}{\F{A}}{\R{B}}$,~$\adpa\posDPleq \adpa$ is always true.
        \item \emph{Antisymmetry}: Given~$\adpa,\adpb\in \HomSet{\DP}{\F{\posgenA}}{\R{\posgenB}}$, if~$\adpa\posDPleq \adpb$ and~$\adpb\posDPleq \adpa$, then~$\adpa=\adpb$.
        \item \emph{Transitivity}: Given~$\adpa,\adpb,\adpc\in \HomSet{\DP}{\F{\posgenA}}{\R{\posgenB}}$,~$\adpa\posDPleq \adpb$, and~$\adpb\posDPleq \adpc$, then~$\adpa\posDPleq \adpc$.
    \end{compactitem}
    Therefore,~$\Hom_\DP$ is a poset.
    Furthermore, consider two design problems~$\adpa,\adpb\in \HomSet{\DP}{\F{\posgenA}}{\R{\posgenB}}$.
    Their least upper bound (join) is~$\adpa\wedge \adpb$, since it is the least design problem implying both~$\adpa$ and~$\adpb$.
    Their greatest lower bound (meet), instead, is~$\adpa\vee \adpb$, since it is the greatest design problem implied by both~$\adpa$ and~$\adpb$.
    This proves that~$\Hom_\DP$ is a lattice.
    To prove that it is bounded, we identify the top element as~$\top_{\F{\posgenA},\R{\posgenB}}$ (it implies all other design problems) and the bottom element as~$\top_{\F{\posgenA},\R{\posgenB}}$ (it is implied by all the other design problems).
\end{proof}

\showslides{

    \begin{forslides}
        \includesag{52_un}
        \includesag{52_int}
        \begin{equation*}
            \label{eq:dpord_1}
            \bot_{\funspa,\resspb}
        \end{equation*}
        \begin{equation*}
            \label{eq:dpord_2}
            \adpa\wedge \adpb
        \end{equation*}
        \begin{equation*}
            \label{eq:dpord_3}
            \adpa\vee \adpb
        \end{equation*}
        \begin{equation*}
            \label{eq:dpord_4}
            \top_{\funspa,\resspb}
        \end{equation*}
        \begin{equation*}
            \label{eq:dpord_5}
            \HomSet{\DP}{\funspa}{\resspb}
        \end{equation*}
        \begin{equation*}
            \label{eq:dpord_6}
            \wedge
        \end{equation*}
        \begin{equation*}
            \label{eq:dpord_7}
            \vee
        \end{equation*}
    \end{forslides}

}
