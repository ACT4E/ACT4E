% !TEX root = chapter-standalone.tex

\section{Ordering DPs}
\label{sec:ordering-order}

We claimed that category theory is an efficient language for talking about \emph{structure}, and showed how the category~\DP could accommodate all the basic operations required by a theory of formal engineering design.
Here, we illustrate some applications and advantages of~\DP in reasoning about and solving design problems, starting with the fact that that~\DP is compact closed, which allows us to compose and reason about ``design problems of design problems''.

\begin{definition}[Order on~\DP]
    \label{def:DP_loc_pos}
    Suppose that~$\F{\posgenA}$ and~$\R{\posgenB}$ are posets, and that~$\adpa,\adpb \colon \F{\posgenA} \profto \R{\posgenB}$ are design problems.
    We define the order as follows:
    \begin{equation}
        \prfdoublecomma{
            \adpa \posDPleq \adpb
        }{
            \adpa(\FposgenAelop,\RposgenBel) \posleqof\Bool \adpb(\FposgenAelop,\RposgenBel)
            \text{~for all~} \FposgenAel \setin \F{\posgenA}, \RposgenBel \setin \R{\posgenB}.
        }
    \end{equation}
\end{definition}

We diagrammatically represent the relation~$\adpa \posDPleq \adpb$ as in~\cref{fig:dpimplies}.

\begin{marginfigure}
    \centering
    \begin{tikzcd}[catcd]
        \bullet \adpb \\
        \bullet \adpa
    \end{tikzcd}
    \caption{The design problem~$\adpa$ implies the design problem~$\adpb$.}
    \label{fig:dpimplies}
    \todographics{@Gioele: fixme}
\end{marginfigure}

\showslides{
    \begin{forslides}
        \includesag{52_un}
        \includesag{52_int}
        %
        \begin{equation*}
            \label{eq:dpord_1}
            \bot_{\funspa,\resspb}
        \end{equation*}
        %
        \begin{equation*}
            \label{eq:dpord_2}
            \adpa\wedge \adpb
        \end{equation*}
        %
        \begin{equation*}
            \label{eq:dpord_3}
            \adpa\vee \adpb
        \end{equation*}
        %
        \begin{equation*}
            \label{eq:dpord_4}
            \top_{\funspa,\resspb}
        \end{equation*}
        %
        \begin{equation*}
            \label{eq:dpord_5}
            \HomSet{\DP}{\funspa}{\resspb}
        \end{equation*}
        %
        \begin{equation*}
            \label{eq:dpord_6}
            \wedge
        \end{equation*}
        %
        \begin{equation*}
            \label{eq:dpord_7}
            \vee
        \end{equation*}
    \end{forslides}
}
