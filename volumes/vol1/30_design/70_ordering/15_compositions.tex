% !TEX root = chapter-standalone.tex


\section{Interaction with composition}
In the previous section, we introduced the concept of order in~\DP, and proved that hom-sets of~\DP form a bounded lattice.
In this section, we show that composition (\cref{def:dp-series}), monoidal product (\cref{def:monoidalproduct}), and trace (\cref{def:dp-trace}) of design problems are order-preserving operations.

\todographics{@Gioele: (polish) These relations are very well explained with diagrams.}
\begin{lemma}
    \label{lem:series_pres_order}
    Given~$\adpa,\adpab \in \HomSet{\DP}{\F{\posgenA}}{\R{\posgenB}}$ and~$\adpc,\adpd \in \HomSet{\DP}{\F{\posgenB}}{\R{\posgenC}}$ one has:
    \begin{equation}
        \prftree[r]{.}{\adpa\posDPleq \adpb}{\adpc\posDPleq \adpd}{\adpa\mthen \adpc \posDPleq \adpb\mthen \adpd}
    \end{equation}
    In other words, series composition is order-preserving on \DP.
\end{lemma}

\begin{proof}
    We have
    \begin{equation}
        \label{eq:series_pres_ord}
        \begin{aligned}
            \left( \adpa\mthen \adpc\right)(\F{\posgenAel}^*,\R{\posgenCel})&=
            \bigvee_{\posgenBel\in \posgenB}\adpa(\F{\posgenAel}^*,\R{\posgenBel})\booland \adpc(\F{\posgenBel}^*,\R{\posgenCel})\\
            &\posDPleq \bigvee_{\posgenBel\in \posgenB} \adpb(\F{\posgenAel}^*,\R{\posgenBel})\booland \adpd(\F{\posgenBel}^*,\R{\posgenCel})\\
            &=\left( \adpb\mthen \adpd\right)(\F{\posgenAel}^*,\R{\posgenCel}).
        \end{aligned}
    \end{equation}
    Therefore~$\mthen$ is order-preserving on~\DP.
\end{proof}

\begin{lemma}
    \label{lem:tens_pres_order}
    Given~$\adpa,\adpb \in \HomSet{\DP}{\F{\posgenA}}{\R{\posgenB}}$ and~$\adpc,\adpd \in \HomSet{\DP}{\F{\posgenC}}{\R{\posgenD}}$, one has:
    
    \begin{equation}
        \prftree[r]{.}{\adpa\posDPleq \adpb}{\adpc\posDPleq \adpd}{\adpa\mtimescat \adpc \posDPleq \adpb\mtimescat \adpd}
    \end{equation}
    In other words, the monoidal product preserves order on~\DP.
\end{lemma}

\begin{proof}
    We have
    \begin{equation}
        \label{eq:tens_pres_ord}
        \begin{aligned}
            \left( \adpa\mtimescat \adpc\right) (\tup{\F{\posgenAel},\F{\posgenCel}}^*,\tup{\R{\posgenBel},\R{\posgenDel}})&=
            \adpa(\F{\posgenAel}^*,\R{\posgenBel})\booland \adpc(\F{\posgenCel}^*,\R{\posgenDel})\\
            &\posDPleq \adpb(\F{\posgenAel}^*,\R{\posgenBel})\booland \adpd(\F{\posgenCel}^*,\R{\posgenDel})\\
            &=\left( \adpb\mtimescat \adpc\right) (\tup{\F{\posgenAel},\F{\posgenCel}}^*,\tup{\R{\posgenBel},\R{\posgenDel}}).
        \end{aligned}
    \end{equation}
    Therefore,~$\mtimescat$ is order-preserving on \DP.
\end{proof}

\begin{lemma}
    \label{lem:trace_pres_order}
    Given~$\adpa_1,\adpa_2\in \HomSet{\DP}{\F{\posgenC}\times \F{\posgenA}}{\R{\posgenC}\times \R{\posgenB}}$, one has:
    
    \begin{equation}
        \prftree[r]{.}{\adpa\posDPleq \adpb}{\Tr_{\F{\posgenA},\R{\posgenB}}^\posgenC(\adpa)\posDPleq \Tr_{\F{\posgenA},\R{\posgenB}}^\posgenC(\adpb)}
    \end{equation}
    In other words, trace preserves order on~\DP.
\end{lemma}

\begin{proof}
    One has
    \begin{equation}
        \label{eq:trace_pres_ord}
        \begin{aligned}
            \Tr_{\F{\posgenA},\R{\posgenB}}^\posgenC(\adpa)(\F{\posgenAel}^*,\R{\posgenBel})&=
            \bigvee_{\posgenCel\in \posgenC}\adpa(\tup{\F{\posgenCel},\F{\posgenAel}}^*,\tup{\R{\posgenCel},\R{\posgenBel}})\\
            &\posDPleq \bigvee_{\posgenCel\in \posgenC}\adpb(\tup{\F{\posgenCel},\F{\posgenAel}}^*,\tup{\R{\posgenCel},\R{\posgenBel}})\\
            &=\Tr_{\F{\posgenA},\R{\posgenB}}^\posgenC(\adpb)(\F{\posgenAel}^*,\R{\posgenBel}).
        \end{aligned}
    \end{equation}
    Therefore,~$\Tr$ is order-preserving on~\DP.
\end{proof}

\begin{lemma}
    \label{lem:coprod_mon}
    Given~$\adpa,\adpb,\adpc,\adpd \in \HomSet{\DP}{\F{\posgenA}}{\R{\posgenB}}$ one has:
    \begin{equation*}
        \prftree[r]{.}{\adpa\posleq_\DP \adpb\ }{\ \adpc\posleq_\DP\adpd}{(\adpa\join\adpc)\posleq_\DP (\adpb\join \adpd)}
    \end{equation*}
\end{lemma}
\begin{proof}
    For any~$\F{\posgenAel}\in \F{\posgenA}$,~$\R{\posgenBel}\in \R{\posgenB}$:
    \begin{equation*}
        \begin{aligned}
        (\adpa \join \adpc)(\F{\posgenAel}^*,\R{\posgenBel})
            &=\adpa(\F{\posgenAel}^*,\R{\posgenBel})\boolor \adpc(\F{\posgenAel}^*,\R{\posgenBel})\\
            &\posleq_\DP \adpb(\F{\posgenAel}^*,\R{\posgenBel})\boolor \adpc(\F{\posgenAel}^*,\R{\posgenBel})\\
            &\posleq_\DP \adpb(\F{\posgenAel}^*,\R{\posgenBel})\boolor \adpd(\F{\posgenAel}^*,\R{\posgenBel}).
        \end{aligned}
    \end{equation*}
\end{proof}



\begin{lemma}
    \label{lem:intersection_mon}
    Given~$\adpa,\adpb,\adpc,\adpd \in \HomSet{\DP}{\F{\posgenA}}{\R{\posgenB}}$ one has:
    \begin{equation*}
        \prftree[r]{.}{\adpa\posleq_\DP \adpb\ }{\ \adpc\posleq_\DP \adpd}{(\adpa\join \adpc)\posleq_\DP (\adpb\join \adpd)}
    \end{equation*}
\end{lemma}
\begin{proof}
    For any~$\F{\posgenAel}\in \F{\posgenA}$,~$\R{\posgenBel}\in \R{\posgenB}$:
    \begin{equation*}
        \begin{aligned}
        (\adpa \join \adpc)(\F{\posgenAel}^*,\R{\posgenBel})
            &=\adpa(\F{\posgenAel}^*,\R{\posgenBel})\booland \adpc(\F{\posgenAel}^*,\R{\posgenBel})\\
            &\posleq_\DP \adpb(\F{\posgenAel}^*,\R{\posgenBel})\booland \adpc(\F{\posgenAel}^*,\R{\posgenBel})\\
            &\posleq_\DP \adpb(\F{\posgenAel}^*,\R{\posgenBel})\booland \adpd(\F{\posgenAel}^*,\R{\posgenBel}).
        \end{aligned}
    \end{equation*}
\end{proof}