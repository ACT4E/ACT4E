% !TEX root = chapter-standalone.tex

\section{Interaction with series composition}
In the previous section, we introduced the concept of order in~\DP, and proved that hom-sets of~\DP form a bounded lattice.
In this section, we show that composition (\cref{def:dp-series}) of design problems is an order-preserving operation.

\todographicsjira{432}{@Gioele: (polish) These relations are very well explained with diagrams.}
\begin{lemma}
    \label{lem:series_pres_order}
    Given~$\adpa,\adpab \in \HomSet{\DP}{\F{\posgenA}}{\R{\posgenB}}$ and~$\adpc,\adpd \in \HomSet{\DP}{\F{\posgenB}}{\R{\posgenC}}$ one has:
    \begin{equation}
        \prfperiod{
            \adpa\posDPleq \adpb
        }{
            \adpc\posDPleq \adpd
        }{
            (\adpa\fthen \adpc) \posDPleq (\adpb\fthen \adpd)
        }
    \end{equation}
    In other words, series composition is order-preserving on \DP.
\end{lemma}

\begin{proof}
    We have
    \begin{equation}
        \label{eq:series_pres_ord}
        \begin{aligned}
            ~ & \left( \adpa\fthen \adpc\right)(\FposgenAelop,\R{\posgenCel}) \\
              & = \bigvee_{\posgenBel\in \posgenB}\adpa(\FposgenAelop,\RposgenBel)\booland \adpc(\FposgenBelop,\R{\posgenCel}) \\
              & \posDPleq \bigvee_{\posBel\in \posB} \adpb(\FposgenAelop,\RposgenBel)\booland \adpd(\FposgenBelop,\R{\posgenCel}) \\
              & =\left( \adpb\fthen \adpd\right)(\FposgenAelop,\R{\posgenCel}).
        \end{aligned}
    \end{equation}
    Therefore~$\fthen$ is order-preserving on~\DP.
\end{proof}

