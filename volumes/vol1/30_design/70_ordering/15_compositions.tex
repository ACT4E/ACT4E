% !TEX root = chapter-standalone.tex

\section{Interaction with composition}
In the previous section, we introduced the concept of order in~\DP, and proved that hom-sets of~\DP form a bounded lattice.
In this section, we show that composition (\cref{def:dp-series}), monoidal product (\cref{def:monoidalproduct}), and trace (\cref{def:dp-trace}) of design problems are order-preserving operations.

\todographicsjira{432}{@Gioele: (polish) These relations are very well explained with diagrams.}
\begin{lemma}
    \label{lem:series_pres_order}
    Given~$\adpa,\adpab \in \HomSet{\DP}{\F{\posgenA}}{\R{\posgenB}}$ and~$\adpc,\adpd \in \HomSet{\DP}{\F{\posgenB}}{\R{\posgenC}}$ one has:
    \begin{equation}
        \prfperiod{
            \adpa\posDPleq \adpb
        }{
            \adpc\posDPleq \adpd
        }{
            (\adpa\fthen \adpc) \posDPleq (\adpb\fthen \adpd)
        }
    \end{equation}
    In other words, series composition is order-preserving on \DP.
\end{lemma}

\begin{proof}
    We have
    \begin{equation}
        \label{eq:series_pres_ord}
        \begin{aligned}
            ~ & \left( \adpa\fthen \adpc\right)(\FposgenAelop,\R{\posgenCel})                                                     \\
              & = \bigvee_{\posgenBel\in \posgenB}\adpa(\FposgenAelop,\RposgenBel)\booland \adpc(\FposgenBelop,\R{\posgenCel})    \\
              & \posDPleq \bigvee_{\posBel\in \posB} \adpb(\FposgenAelop,\RposgenBel)\booland \adpd(\FposgenBelop,\R{\posgenCel}) \\
              & =\left( \adpb\fthen \adpd\right)(\FposgenAelop,\R{\posgenCel}).
        \end{aligned}
    \end{equation}
    Therefore~$\fthen$ is order-preserving on~\DP.
\end{proof}

\begin{lemma}
    \label{lem:tens_pres_order}
    Given~$\adpa,\adpb \in \HomSet{\DP}{\F{\posgenA}}{\R{\posgenB}}$ and~$\adpc,\adpd \in \HomSet{\DP}{\F{\posgenC}}{\R{\posgenD}}$, one has:

    \begin{equation}
        \prfperiod{
            \adpa\posDPleq \adpb
        }{
            \adpc\posDPleq \adpd
        }{
            (\adpa\mtimescat \adpc )\posDPleq (\adpb\mtimescat \adpd)
        }
    \end{equation}
    In other words, the monoidal product preserves order on~\DP.
\end{lemma}

\begin{proof}
    We have
    \begin{equation}
        \label{eq:tens_pres_ord}
        \begin{aligned}
            ~ & \left( \adpa\mtimescat \adpc\right) (\tup{\FposgenAel,\F{\posgenCel}}\Fop,\tup{\RposgenBel,\R{\posgenDel}})   \\
              & = \adpa(\FposgenAelop,\RposgenBel)\booland \adpc(\F{\posgenCel^*},\R{\posgenDel})                             \\
              & \posDPleq \adpb(\FposgenAelop,\RposgenBel)\booland \adpd(\F{\posgenCel^*},\R{\posgenDel})                     \\
              & =\left( \adpb\mtimescat \adpc\right) (\tup{\FposgenAel,\F{\posgenCel}}\Fop,\tup{\RposgenBel,\R{\posgenDel}}).
        \end{aligned}
    \end{equation}
    Therefore,~$\mtimescat$ is order-preserving on \DP.
\end{proof}

\begin{lemma}
    \label{lem:trace_pres_order}
    Given~$\adpa,\adpb\in \HomSet{\DP}{\F{\posgenC}\cartprod \F{\posgenA}}{\R{\posgenC}\cartprod \R{\posgenB}}$, one has:

    \begin{equation}
        \prfperiod{\adpa\posDPleq \adpb}{\Tr_{\F{\posgenA},\R{\posgenB}}^\posC(\adpa)\posDPleq \Tr_{\F{\posgenA},\R{\posgenB}}^\posC(\adpb)}
    \end{equation}
    In other words, trace preserves order on~\DP.
\end{lemma}

\begin{proof}
    One has
    \begin{equation}
        \label{eq:trace_pres_ord}
        \begin{aligned}
            ~ & \Tr_{\F{\posgenA},\R{\posgenB}}^\posC(\adpa)(\FposgenAelop,\RposgenBel)                                          \\
              & = \bigvee_{\posCel\in \posC}\adpa(\tup{\F{\posgenCel},\FposgenAel}\Fop,\tup{\R{\posgenCel},\RposgenBel})         \\
              & \posDPleq \bigvee_{\posCel\in \posC}\adpb(\tup{\F{\posgenCel},\FposgenAel}\Fop,\tup{\R{\posgenCel},\RposgenBel}) \\
              & =\Tr_{\F{\posgenA},\R{\posgenB}}^\posC(\adpb)(\FposgenAelop,\RposgenBel).
        \end{aligned}
    \end{equation}
    Therefore,~$\Tr$ is order-preserving on~\DP.
\end{proof}

\begin{lemma}
    \label{lem:coprod_mon}
    Given~$\adpa,\adpb,\adpc,\adpd \in \HomSet{\DP}{\F{\posgenA}}{\R{\posgenB}}$ one has:
    \begin{equation*}
        \prfperiod{\adpa\posleqof\DP \adpb\ }{\ \adpc\posleqof\DP\adpd}{(\adpa\join\adpc)\posleqof\DP (\adpb\join \adpd)}
    \end{equation*}
\end{lemma}
\begin{proof}
    For any~$\FposgenAel\in \F{\posgenA}$,~$\RposgenBel\in \R{\posgenB}$:
    \begin{equation*}
        \begin{aligned}
            ~ & (\adpa \join \adpc)(\FposgenAelop,\RposgenBel)                                         \\
              & =\adpa(\FposgenAelop,\RposgenBel)\boolor \adpc(\FposgenAelop,\RposgenBel)              \\
              & \posleqof\DP \adpb(\FposgenAelop,\RposgenBel)\boolor \adpc(\FposgenAelop,\RposgenBel)  \\
              & \posleqof\DP \adpb(\FposgenAelop,\RposgenBel)\boolor \adpd(\FposgenAelop,\RposgenBel).
        \end{aligned}
    \end{equation*}
\end{proof}

\begin{lemma}
    \label{lem:intersection_mon}
    Given~$\adpa,\adpb,\adpc,\adpd \in \HomSet{\DP}{\F{\posgenA}}{\R{\posgenB}}$ one has:
    \begin{equation*}
        \prfperiod{\adpa\posleqof\DP \adpb\ }{\ \adpc\posleqof\DP \adpd}{(\adpa\meet \adpc)\posleqof\DP (\adpb\meet \adpd)}
    \end{equation*}
\end{lemma}
\begin{proof}
    For any~$\FposgenAel\in \F{\posgenA}$,~$\RposgenBel\in \R{\posgenB}$:
    \begin{equation*}
        \begin{aligned}
            ~ & (\adpa \meet \adpc)(\FposgenAel^*,\RposgenBel)                                          \\
              & =\adpa(\FposgenAelop,\RposgenBel)\booland \adpc(\FposgenAelop,\RposgenBel)              \\
              & \posleqof\DP \adpb(\FposgenAelop,\RposgenBel)\booland \adpc(\FposgenAelop,\RposgenBel)  \\
              & \posleqof\DP \adpb(\FposgenAelop,\RposgenBel)\booland \adpd(\FposgenAelop,\RposgenBel).
        \end{aligned}
    \end{equation*}
\end{proof}
