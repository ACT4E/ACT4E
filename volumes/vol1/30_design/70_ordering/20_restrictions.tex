% !TEX root = standalone.tex


\section{Restrictions and alternatives}

\subsection{Union of Design Problems}
Let~$f\colon \F{A}\profto \R{B}$ and~$g\colon \F{A}\profto \R{B}$ be design problems. We define the \emph{union}~$f \vee g$ to be the design problem which is feasible whenever \emph{either}~$f$ or~$g$ is feasible.
This models~$f$ and~$g$ as interchangeable technologies: either one can replace the other.

\begin{definition}[Union of design problems]
  Given two design problems~$f \colon \F{A} \profto \R{B}$ and~$g \colon \F{A} \profto \R{B}$, their \emph{union}~$f \vee g \colon \F{A} \profto \R{B}$ is defined by
  \begin{equation}
    \begin{aligned}
    (f \vee g)
      \colon \F{A}\op \times \R{B} & \toinPos \Bool \\
      \tup{\F{a}^*, \R{b}} & \mapsto f(\F{a}^*, \R{b}) \vee g(\F{a}^*, \R{b}),
    \end{aligned}
  \end{equation}
  and represented as in~\cref{fig:uniondp}.
\end{definition}

\begin{figure}[h!]
  \begin{center}
    \includesag{52_union}
  \end{center}
  \caption{Diagrammatic representation of the union of design problems. }
  \label{fig:uniondp}
\end{figure}

\begin{example}
  Jeb's Spaceship Parts is locked in a deadly rivalry with Starshow Bob to supply engines for the new X103 space orbiter. Neither knows the exact operational scenario that the X103 will encounter, but have provided a range of performance benchmarks for their engines (\cref{fig:exunion_1}).
  \begin{figure}[h!]
    \begin{center}
      \includesag{50_rival_jeb_bob}
    \end{center}
    \caption{Example of two engine producers. }
    \label{fig:exunion_1}
  \end{figure}
  Back at NASA headquarters, Beau has uploaded Jeb and Bob's data in order to construct the design problem reported in~\cref{fig:exunion_2}.
  \begin{figure}[h!]
    \begin{center}
      \includesag{50_rival_beau}
    \end{center}
    \caption{Example of the union of the engine design problems. }
    \label{fig:exunion_2}
  \end{figure}
\end{example}

\subsection{In DPI}\label{subsec:dpi-union}

\todo{Finish}

The union of two design problems is a design problem with the implementation
space~$\impsp=\impsp_{1}\sqcup\impsp_{2}$, and it represents the
exclusive choice between two possible alternative families of designs.
\begin{definition}[Coproduct]
  \label{def:parallel-1}Given two DPIs with same functionality and
  resources~$\dprob_{1}=\left\langle \funsp,\ressp,\impsp_{1},\prov_{1},\req_{1}\right\rangle$
  and~$\dprob_{2}=\left\langle \funsp,\ressp,,\impsp_{2},\prov_{2},\req_{2}\right\rangle$,
  define their co-product as
  \begin{equation}
    \dprob_{1}\sqcup\dprob_{2}\definedas\left\langle \funsp,\ressp,\impsp_{1}\sqcup\impsp_{2},\prov,\req\right\rangle ,
  \end{equation}
  where
  \begin{eqnarray}
    \prov & : & \imp\mapsto\begin{cases}
                             \prov_{1}(\imp), & \text{if }\imp\in\impsp_{1},\\
                             \prov_{2}(\imp), & \text{if }\imp\in\impsp_{2},
    \end{cases}\label{eq:dppar-exec-1}\\
    \req & : & \imp\mapsto\begin{cases}
                            \req_{1}(\imp), & \text{if }\imp\in\impsp_{1},\\
                            \req_{2}(\imp), & \text{if }\imp\in\impsp_{2}.
    \end{cases}\nonumber
  \end{eqnarray}
\end{definition}

%
%\captionsideleft{\label{fig:dpcoproduct}}{
%  \includegraphics[scale=0.33]{gmcdp_coproduct}
%}

\todographics{to Andrea: we need to converge on the picture style. The old gmcdpcoproduct is not updated, do we want the colored dashed lines? \AC{no, but we need the dashed line separating the two DPs. See slides}}

\begin{figure}[h!]
  \centering
  \includesag{20_dpi_coproduct}
  \caption{\label{fig:dpcoproduct}}
\end{figure}

\subsection{Intersection of Design Problems}

Given two design problems~$f, g \colon \F{A} \profto \R{B}$, we can define a design problem~$f \wedge g$ that is feasible if only if~$f$ and~$g$ are both feasible. We call~$f \wedge g$ the \emph{intersection} of~$f$ and~$g$. One interpretation of~$f \wedge g$ is that~$f$ and~$g$ are two slightly different models of the same process, and we want to make sure that the design is conservatively feasible for both models.

\begin{definition}[Intersection of design problems]
  \label{def:dp-intersection}
  Given design problems~$f\colon \F{A} \profto \R{B}$ and~$g\colon \F{A} \profto \R{B}$,
  their \emph{intersection} is denoted~$(f \wedge g)\colon \F{A} \profto \R{B}$, defined by:
  \begin{equation}
    \begin{aligned}
    (f \wedge g)
      \colon \F{A}\op \times \R{B} & \toinPos \Bool \\
      \tup{\F{a}^*, \R{b}} & \mapsto f(\F{a}^*, \R{b}) \wedge  g(\F{a}^*, \R{b}),
    \end{aligned}
  \end{equation}
  and represented as in~\cref{fig:intersectiondp}.
\end{definition}

\begin{figure}[h!]
  \begin{center}
    \includesag{52_intersection}
  \end{center}
  \caption{Diagrammatic representation of the intersection of design problems. }
  \label{fig:intersectiondp}
\end{figure}

We can directly generalize the intersection~$f \wedge g$ by allowing~$f$ and~$g$ to have different domain and codomains,~$f \colon \F{A} \profto \R{B}$ and~$g \colon \F{C} \profto \R{D}$. We call this putting two design problems ``in parallel''.

\subsection{In DPI}\label{subsec:dpi-intersection}

\todo{Finish, do the equivalent to \cref{subsec:dpi-union}}
