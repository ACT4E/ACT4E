% !TEX root = chapter-standalone.tex


\section{Restrictions and alternatives}

\linkvideo{spring2021-functorial-comp-b:solving-queries:or-and-and} % Join and Meet in DP

\subsection{Union of Design Problems}
Let~$\adpa\colon \F{\posgenA}\profto \R{\posgenB}$ and~$\adpb\colon \F{\posgenA}\profto \R{\posgenB}$ be design problems.
We define the \emph{union}~$\adpa \vee \adpb$ to be the design problem which is feasible whenever \emph{either}~$\adpa$ or~$\adpb$ is feasible.
This models~$\adpa$ and~$\adpb$ as interchangeable technologies: either one can replace the other.

\begin{definition}[Union of design problems]
    \label{def:union_dp}
    Given two design problems~$\adpa \colon \F{\posgenA} \profto \R{\posgenB}$ and~$\adpb\colon \F{\posgenA} \profto \R{\posgenB}$, their \emph{union}~$\adpa \vee \adpb\colon \F{\posgenA} \profto \R{\posgenB}$ is defined by
    \begin{equation}
        \begin{aligned}
        (\adpa \vee \adpb)
            \colon \F{\posgenA}\op \times \R{\posgenB} & \toinPos \Bool \\
            \tup{\F{a}^*, \R{b}} & \mapsto \adpa(\F{a}^*, \R{b}) \vee \adpb(\F{a}^*, \R{b}).
        \end{aligned}
    \end{equation}
\end{definition}

The union of design problems is represented as in~\cref{fig:uniondp}.

\begin{figure}[h!]
    \begin{center}
        \includesag{52_union}
    \end{center}
    \caption{Diagrammatic representation of the union of design problems. }
    \label{fig:uniondp}
\end{figure}

\begin{example}
    Jeb's Spaceship Parts is locked in a deadly rivalry with Starshow Bob to supply engines for the new X103 space orbiter. Neither knows the exact operational scenario that the X103 will encounter, but have provided a range of performance benchmarks for their engines (\cref{fig:exunion_1}).
    \begin{figure}[h!]
        \begin{center}
            \includesag{50_rival_jeb_bob}
        \end{center}
        \caption{Example of two engine producers. }
        \label{fig:exunion_1}
    \end{figure}
    Back at NASA headquarters, Beau has uploaded Jeb and Bob's data in order to construct the design problem reported in~\cref{fig:exunion_2}.
    \begin{figure}[h!]
        \begin{center}
            \includesag{50_rival_beau}
        \end{center}
        \caption{Example of the union of the engine design problems. }
        \label{fig:exunion_2}
    \end{figure}
\end{example}

\subsection{In DPI}\label{subsec:dpi-union}

\todojira{204}{Finish}

The union of two design problems with implementation is a design problem with the implementation
space~$\impsp=\impsp_{1}\sqcup\impsp_{2}$, and it represents the
exclusive choice between two possible alternative families of designs.
\begin{definition}[Coproduct of DPIs]
    \label{def:parallel-1}Given two DPIs with same functionality and
    resources~$\adpa=\tup{\funsp,\ressp,\impsp_{1},\prov_{1},\req_{1}}$
    and~$\adpb=\tup{\funsp,\ressp,\impsp_{2},\prov_{2},\req_{2}}$,
    define their co-product as
    \begin{equation}
        \adpa\sqcup\adpb\definedas\tup{\funsp,\ressp,\impsp_{1}\sqcup\impsp_{2},\prov,\req} ,
    \end{equation}
    where
    \begin{eqnarray}
        \prov & \colon & \imp\mapsto\begin{cases}
                                        \prov_{1}(\imp), & \text{if }\imp\in\impsp_{1},\\
                                        \prov_{2}(\imp), & \text{if }\imp\in\impsp_{2},
        \end{cases}\label{eq:dppar-exec-1}\\
        \req & \colon & \imp\mapsto\begin{cases}
                                       \req_{1}(\imp), & \text{if }\imp\in\impsp_{1},\\
                                       \req_{2}(\imp), & \text{if }\imp\in\impsp_{2}.
        \end{cases}\nonumber
    \end{eqnarray}
\end{definition}

%
%\captionsideleft{\label{fig:dpcoproduct}}{
%  \includegraphics[scale=0.33]{gmcdp_coproduct}
%}

\begin{figure}[h!]
    \centering
    \includesag{20_dpi_coproduct}
    \caption{\label{fig:dpcoproduct}}
\end{figure}

\subsection{Intersection of Design Problems}

Given two design problems~$\adpa, \adpb \colon \F{\posgenA} \profto \R{\posgenB}$, we can define a design problem~$\adpa \wedge \adpb$ that is feasible if only if~$\adpa$ and~$\adpb$ are both feasible.
We call~$\adpa \wedge \adpb$ the \emph{intersection} of~$\adpa$ and~$\adpb$.
One interpretation of~$\adpa \wedge \adpb$ is that~$\adpa$ and~$\adpb$ are two slightly different models of the same process, and we want to make sure that the design is conservatively feasible for both models.

\begin{definition}[Intersection of design problems]
    \label{def:intersection_dp}
    \label{def:dp-intersection}
    Given design problems~$\adpa\colon \F{\posgenA} \profto \R{\posgenB}$ and~$\adpb\colon \F{\posgenA} \profto \R{\posgenB}$,
    their \emph{intersection} is denoted~$(\adpa \wedge \adpb)\colon \F{\posgenA} \profto \R{\posgenB}$, defined by:
    \begin{equation}
        \begin{aligned}
        (\adpa \wedge \adpb)
            \colon \F{\posgenA}\op \times \R{\posgenB} & \toinPos \Bool \\
            \tup{\F{a}^*, \R{b}} & \mapsto \adpa(\F{a}^*, \R{b}) \wedge  \adpb(\F{a}^*, \R{b}).
        \end{aligned}
    \end{equation}
\end{definition}
The intersection of design problems is represented as in~\cref{fig:intersectiondp}.

\begin{figure}[h!]
    \begin{center}
        \includesag{52_intersection}
    \end{center}
    \caption{Diagrammatic representation of the intersection of design problems. }
    \label{fig:intersectiondp}
\end{figure}

We can directly generalize the intersection~$\adpa \wedge \adpb$ by allowing~$\adpa$ and~$\adpb$ to have different domain and codomains,~$\adpa \colon \F{\posgenA} \profto \R{\posgenB}$ and~$\adpb \colon \F{\posgenC} \profto \R{\posgenD}$.
We call this putting two design problems ``in parallel''.

\subsection{In DPI}\label{subsec:dpi-intersection}

\todotextjira{205}{not sure the definition below is what we want}
\devel{
\begin{definition}[Intersection of DPIs]
    \label{def:intersection-1}
    Given two DPIs with same functionality and
    resources~$\adpa=\tup{\funsp,\ressp,\impsp_{1},\prov_{1},\req_{1}}$
    and~$\adpb=\tup{\funsp,\ressp,\impsp_{2},\prov_{2},\req_{2}}$,
    define their intersection as
    \begin{equation}
        \adpa\sqcap\adpb\definedas\tup{\funsp,\ressp,\impsp_{1}\cap\impsp_{2},\prov,\req} ,
    \end{equation}
    where
    \begin{eqnarray}
        \prov & \colon & \imp\mapsto\begin{cases}
                                        \prov_{1}(\imp), & \text{if }\imp\in\impsp_{1}\cap \impsp_{2} \text{ and }\prov_1(\imp)\posleq \prov_2(\imp)\\
                                        \prov_{2}(\imp), & \text{if }\imp\in\impsp_{1}\cap \impsp_{2} \text{ and }\prov_2(\imp)\posleq \prov_1(\imp)\\
                                        \bot_\funsp,& \text{else}.
        \end{cases}\label{eq:dppar-exec-2}\\
        \req & \colon & \imp\mapsto\begin{cases}
                                       \req_{1}(\imp), & \text{if }\imp\in\impsp_{1}\cap \impsp_{2} \text{ and }\req_1(\imp)\posgeq \req_2(\imp)\\
                                        \req_{2}(\imp), & \text{if }\imp\in\impsp_{1}\cap \impsp_{2} \text{ and }\req_2(\imp)\posgeq \req_1(\imp)\\
                                        \top_\ressp,& \text{else}.
        \end{cases}\nonumber
    \end{eqnarray}
\end{definition}}

\begin{figure}[h!]
    \centering
    \includesag{dpi_intersection}
    \caption{\label{fig:intersection}}
\end{figure}

\section{Lattice structure of DP homsets}

Given the definitions of $\vee$ and $\wedge$ in the previous sections, we can prove that 
every \DP homsets have a lattice structure. 

This lattice is bounded by a "true" and a "false" DP.
For any functionality-resource pair~$\F{\posgenA},\R{\posgenB}$, we denote by~$\postop_{\F{\posgenA},\R{\posgenB}}$ the design problem which is always feasible.
We denote by~$\posbot_{\F{\posgenA},\R{\posgenB}}$ the design problem which is never feasible, for any functionality-resource pair~$\F{\posgenA},\R{\posgenB}$.

\begin{lemma}
    \label{lem:dpboundedlattice}
    $\HomSet{\DP}{\F{\posgenA}}{\R{\posgenB}}$ is a bounded lattice with union~$\vee$ as meet, intersection~$\wedge$ as join, least upper bound~$\top_{\F{\posgenA},\R{\posgenB}}$ and greatest lower bound~$\bot_{\F{\posgenA},\R{\posgenB}}$.
\end{lemma}

\begin{proof}
    First, we need to prove that~$\HomSet{\DP}{\F{\posgenA}}{\R{\posgenB}}$ is a poset. To prove this, we check the following:
% 
    \begin{compactitem}
        \item \emph{Reflexivity}: Given~$\adpa\in \HomSet{\DP}{\F{\posgenA}}{\R{\posgenB}}$,~$\adpa\posDPleq \adpa$ is always true.
        \item \emph{Antisymmetry}: Given~$\adpa,\adpb\in \HomSet{\DP}{\F{\posgenA}}{\R{\posgenB}}$, if~$\adpa\posDPleq \adpb$ and~$\adpb\posDPleq \adpa$, then~$\adpa=\adpb$.
        \item \emph{Transitivity}: Given~$\adpa,\adpb,\adpc\in \HomSet{\DP}{\F{\posgenA}}{\R{\posgenB}}$,~$\adpa\posDPleq \adpb$, and~$\adpb\posDPleq \adpc$, then~$\adpa\posDPleq \adpc$.
    \end{compactitem}
    Therefore,~$\HomSet{\DP}{\F{\posgenA}}{\R{\posgenB}}$ is a poset.
    Furthermore, consider two design problems~$\adpa,\adpb\in \HomSet{\DP}{\F{\posgenA}}{\R{\posgenB}}$.
    Their least upper bound (join) is~$\adpa\wedge \adpb$, since it is the least design problem implying both~$\adpa$ and~$\adpb$.
    Their greatest lower bound (meet), instead, is~$\adpa\vee \adpb$, since it is the greatest design problem implied by both~$\adpa$ and~$\adpb$.
    This proves that~$\Hom_\DP$ is a lattice.
    To prove that it is bounded, we identify the top element as~$\top_{\F{\posgenA},\R{\posgenB}}$ (it implies all other design problems) and the bottom element as~$\top_{\F{\posgenA},\R{\posgenB}}$ (it is implied by all the other design problems).
\end{proof}

\devel{

We show that a \DP Homset is a \emph{complete lattice}.

\begin{definition}[Complete Lattice]\label{def:complete-lattice}
    A poset~$\tup{\posA,\posleq}$ is a \emph{\iindex{complete lattice}} if every subset~$\posB$ of~$\posA$ has both a \emph{greatest lower bound} (often referred to as the \emph{infimum, meet}) and a \emph{least upper bound} (often referred to as the \emph{supremum, join}) in~$\tup{\posA, \posleq}$.
\end{definition}

\begin{example}
    Consider the power set of any given set, ordered by inclusion. The supremum of any two subsets is given by their union.
    The infimum of any two subsets is given by their intersection.
\end{example}


\begin{lemma}[\DP homsets are complete lattices]\label{def:DP-homsets-complete-lattice}
    \todotextjira{274}{Show DP homsets are complete lattices.}
\end{lemma}

\subsection{Interaction with composition}

Furthermore, we show that all composition operations preserve meet and joins.

\begin{lemma}
    \label{lem:series_vee}
    Consider~$\adpa,\adpb\in \HomSet{\DP}{\F{\posgenA}}{\R{\posgenB}}$ and~$\adpc\in \HomSet{\DP}{\F{\posgenB}}{\R{\posgenC}}$. One has
    \begin{equation*}
        (\adpa \vee \adpb)\mthen \adpc=(\adpa \mthen \adpc) \vee (\adpb\mthen \adpc).
    \end{equation*}
\end{lemma}
\begin{proof}
    One has:
    \begin{equation*}
        \begin{aligned}
            ((\adpa \vee \adpb)\mthen \adpc)(\F{a}^*,\R{c})&=\bigvee_{b\in \posgenB} (\adpa \vee \adpb)(\F{a}^*,\R{b})\wedge \adpc(\F{b}^*,\R{c})\\
            &=\bigvee_{b\in \posgenB} (\adpa(\F{a}^*,\R{b}) \vee \adpb(\F{a}^*,\R{b}))\wedge \adpc(\F{b}^*,\R{c})\\
            &=\bigvee_{b\in \posgenB} (\adpa(\F{a}^*,\R{b}) \wedge  \adpc(\F{b}^*,\R{c})) \vee (\adpb(\F{a}^*,\R{b})\wedge \adpc(\F{b}^*,\R{c}))\\
            &=((\adpa \mthen \adpc) \vee (\adpb\mthen \adpc))(\F{a}^*,\R{c}).
        \end{aligned}
    \end{equation*}
\end{proof}

\begin{remark}
    Consider~$\adpa,\adpb\in \HomSet{\DP}{\F{\posgenA}}{\R{\posgenB}}$ and~$\adpc,\adpd\in \HomSet{\DP}{\F{\posgenB}}{\R{\posgenC}}$.
    In general, one has:
    \begin{equation*}
        (\adpa\vee \adpb)\mthen (\adpc\vee \adpd) \neq (\adpa \mthen \adpc)\vee (\adpb \mthen \adpd).
    \end{equation*}
    \todojira{275}{Write down counterexample (after derivation).}
\end{remark}

\begin{lemma}
    \label{lem:series_wedge}
    Consider~$\adpa,\adpb\in \HomSet{\DP}{\F{\posgenA}}{\R{\posgenB}}$ and~$\adpc\in \HomSet{\DP}{\F{\posgenB}}{\R{\posgenC}}$. One has
    \begin{equation*}
        (\adpa \wedge \adpb)\mthen \adpc=(\adpa \mthen \adpc) \wedge (\adpb\mthen \adpc).
    \end{equation*}
\end{lemma}
\begin{proof}
    One has:
    \begin{equation*}
        \begin{aligned}
            ((\adpa \wedge \adpb)\mthen \adpc)(\F{a}^*,\R{c})&=\bigvee_{b\in \posgenB} (\adpa \wedge \adpb)(\F{a}^*,\R{b})\wedge \adpc(\F{b}^*,\R{c})\\
            &=\bigvee_{b\in \posgenB} (\adpa(\F{a}^*,\R{b}) \wedge \adpb(\F{a}^*,\R{b}))\wedge \adpc(\F{b}^*,\R{c})\\
            &=\bigvee_{b\in \posgenB} (\adpa(\F{a}^*,\R{b}) \wedge  \adpc(\F{b}^*,\R{c})) \wedge (\adpb(\F{a}^*,\R{b})\wedge \adpc(\F{b}^*,\R{c}))\\
            &=((\adpa \mthen \adpc) \wedge (\adpb\mthen \adpc))(\F{a}^*,\R{c}).
        \end{aligned}
    \end{equation*}
\end{proof}

\begin{lemma}
    \label{lem:times_vee}
    Consider~$\adpa,\adpb\in \HomSet{\DP}{\F{\posgenA}}{\R{\posgenB}}$ and~$\adpc\in \HomSet{\DP}{\F{\posgenC}}{\R{\posgenD}}$. One has
    \begin{equation*}
        (\adpa \vee \adpb)\mtimescat \adpc=(\adpa \mtimescat \adpc) \vee (\adpb\mtimescat \adpc).
    \end{equation*}
\end{lemma}
\begin{proof}
    One has:
    \begin{equation*}
        \begin{aligned}
            ((\adpa \vee \adpb)\mtimescat \adpc)(\tup{\F{a},\F{c}}^*,\tup{\R{b},\R{d}})&=
            (\adpa \vee \adpb)(\F{a}^*,\R{b})\wedge \adpc(\F{c}^*,\R{d})\\
            &=(\adpa(\F{a}^*,\R{b}) \vee \adpb(\F{a}^*,\R{b}))\wedge \adpc(\F{c}^*,\R{d})\\
            &=(\adpa(\F{a}^*,\R{b}) \wedge  \adpc(\F{c}^*,\R{d})) \vee (\adpb(\F{a}^*,\R{b})\wedge \adpc(\F{c}^*,\R{d}))\\
            &=((\adpa \mtimescat \adpc) \vee (\adpb\mtimescat \adpc))(\tup{\F{a},\F{c}}^*,\tup{\R{b},\R{d}}).
        \end{aligned}
    \end{equation*}
\end{proof}

\begin{lemma}
    \label{lem:times_wedge}
    Consider~$\adpa,\adpb\in \HomSet{\DP}{\F{\posgenA}}{\R{\posgenB}}$ and~$\adpc\in \HomSet{\DP}{\F{\posgenC}}{\R{\posgenD}}$. One has
    \begin{equation*}
        (\adpa \wedge \adpb)\mtimescat \adpc=(\adpa \mtimescat \adpc) \wedge (\adpb\mtimescat \adpc).
    \end{equation*}
\end{lemma}
\begin{proof}
    One has:
    \begin{equation*}
        \begin{aligned}
            ((\adpa \wedge \adpb)\mtimescat \adpc)(\tup{\F{a},\F{c}}^*,\tup{\R{b},\R{d}})&=
            (\adpa \wedge \adpb)(\F{a}^*,\R{b})\wedge \adpc(\F{c}^*,\R{d})\\
            &=(\adpa(\F{a}^*,\R{b}) \wedge \adpb(\F{a}^*,\R{b}))\wedge \adpc(\F{c}^*,\R{d})\\
            &=(\adpa(\F{a}^*,\R{b}) \wedge  \adpc(\F{c}^*,\R{d})) \wedge (\adpb(\F{a}^*,\R{b})\wedge \adpc(\F{c}^*,\R{d}))\\
            &=((\adpa \mtimescat \adpc) \wedge (\adpb\mtimescat \adpc))(\tup{\F{a},\F{c}}^*,\tup{\R{b},\R{d}}).
        \end{aligned}
    \end{equation*}
\end{proof}


\begin{lemma}
    \label{lem:vee_vee}
    Consider~$\adpa,\adpb,\adpc\in \HomSet{\DP}{\F{\posgenA}}{\R{\posgenB}}$. One has
    \begin{equation*}
        (\adpa \vee \adpb)\vee \adpc=(\adpa \vee \adpc) \vee (\adpb\vee \adpc).
    \end{equation*}
\end{lemma}
\begin{proof}
    One has:
    \begin{equation*}
        \begin{aligned}
            ((\adpa \vee \adpb)\vee \adpc)(\F{a}^*,\R{b})&=
            (\adpa \vee \adpb)(\F{a}^*,\R{b})\vee \adpc(\F{a}^*,\R{b})\\
            &=(\adpa(\F{a}^*,\R{b}) \vee \adpb(\F{a}^*,\R{b}))\vee \adpc(\F{a}^*,\R{b})\\
            &=(\adpa(\F{a}^*,\R{b}) \vee  \adpc(\F{a}^*,\R{b})) \vee (\adpb(\F{a}^*,\R{b})\vee \adpc(\F{a}^*,\R{b}))\\
            &=((\adpa \vee \adpc) \vee (\adpb\vee \adpc))(\F{a}^*,\R{b}).
        \end{aligned}
    \end{equation*}
\end{proof}

\begin{lemma}
    \label{lem:vee_wedge}
    Consider~$\adpa,\adpb,\adpc\in \HomSet{\DP}{\F{\posgenA}}{\R{\posgenB}}$. One has
    \begin{equation*}
        (\adpa \wedge \adpb)\vee \adpc=(\adpa \vee \adpc) \wedge (\adpb\vee \adpc).
    \end{equation*}
\end{lemma}
\begin{proof}
    One has:
    \begin{equation*}
        \begin{aligned}
            ((\adpa \wedge \adpb)\vee \adpc)(\F{a}^*,\R{b})&=
            (\adpa \wedge \adpb)(\F{a}^*,\R{b})\vee \adpc(\F{a}^*,\R{b})\\
            &=(\adpa(\F{a}^*,\R{b}) \wedge \adpb(\F{a}^*,\R{b}))\vee \adpc(\F{a}^*,\R{b})\\
            &=(\adpa(\F{a}^*,\R{b}) \vee  \adpc(\F{a}^*,\R{b})) \wedge (\adpb(\F{a}^*,\R{b})\vee \adpc(\F{a}^*,\R{b}))\\
            &=((\adpa \vee \adpc) \wedge (\adpb\vee \adpc))(\F{a}^*,\R{b}).
        \end{aligned}
    \end{equation*}
\end{proof}


\begin{lemma}
    \label{lem:wedge_vee}
    Consider~$\adpa,\adpb,\adpc\in \HomSet{\DP}{\F{\posgenA}}{\R{\posgenB}}$. One has
    \begin{equation*}
        (\adpa \vee \adpb)\wedge \adpc=(\adpa \wedge \adpc) \vee (\adpb\wedge \adpc).
    \end{equation*}
\end{lemma}
\begin{proof}
    One has:
    \begin{equation*}
        \begin{aligned}
            ((\adpa \vee \adpb)\wedge \adpc)(\F{a}^*,\R{b})&=
            (\adpa \vee \adpb)(\F{a}^*,\R{b})\wedge \adpc(\F{a}^*,\R{b})\\
            &=(\adpa(\F{a}^*,\R{b}) \vee \adpb(\F{a}^*,\R{b}))\vee \adpc(\F{a}^*,\R{b})\\
            &=(\adpa(\F{a}^*,\R{b}) \wedge  \adpc(\F{a}^*,\R{b})) \vee (\adpb(\F{a}^*,\R{b})\wedge \adpc(\F{a}^*,\R{b}))\\
            &=((\adpa \wedge \adpc) \vee (\adpb\wedge \adpc))(\F{a}^*,\R{b}).
        \end{aligned}
    \end{equation*}
\end{proof}

\begin{lemma}
    \label{lem:wedge_wedge}
    Consider~$\adpa,\adpb,\adpc\in \HomSet{\DP}{\F{\posgenA}}{\R{\posgenB}}$. One has
    \begin{equation*}
        (\adpa \wedge \adpb)\wedge \adpc=(\adpa \wedge \adpc) \wedge (\adpb\wedge \adpc).
    \end{equation*}
\end{lemma}
\begin{proof}
    One has:
    \begin{equation*}
        \begin{aligned}
            ((\adpa \wedge \adpb)\wedge \adpc)(\F{a}^*,\R{b})&=
            (\adpa \wedge \adpb)(\F{a}^*,\R{b})\wedge \adpc(\F{a}^*,\R{b})\\
            &=(\adpa(\F{a}^*,\R{b}) \wedge \adpb(\F{a}^*,\R{b}))\wedge \adpc(\F{a}^*,\R{b})\\
            &=(\adpa(\F{a}^*,\R{b}) \wedge  \adpc(\F{a}^*,\R{b})) \wedge (\adpb(\F{a}^*,\R{b})\wedge \adpc(\F{a}^*,\R{b}))\\
            &=((\adpa \wedge \adpc) \wedge (\adpb\wedge \adpc))(\F{a}^*,\R{b}).
        \end{aligned}
    \end{equation*}
\end{proof}

\todotextjira{275}{Do traces}

\todotextjira{275}{Prove that the composition operations preserve join and meet}


}

