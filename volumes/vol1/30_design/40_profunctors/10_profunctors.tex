% !TEX root = chapter-standalone.tex


\section{Profunctors}

\showslides{
\begin{forslides}
\begin{ctdefinition}[$\DP$]
  \label{def:dp-again}
  Given two posets $\funsp$ and $\ressp$, a \emph{design problem} with functionality $\funsp$ and resources~$\ressp$ is a monotone function of the form
  \begin{equation}\label{eq:dp-again-1}
    \adp \colon \funsp\op \Ctimes \ressp \fto_{\Pos} \Bool.
  \end{equation}
  This is also written as
  \begin{equation}\label{eq:dp-again-2}
    \adp\colon \funsp \profto \ressp.
  \end{equation}
\end{ctdefinition}

\begin{equation}\label{eq:dp-again-3}
  \tupp{\fun^{\op}, \res} \quad \mapsto \quad \exists \imp \in \impsp :  (\fun \funleq \prov(\imp)) \wedge (\req(\imp) \resleq \res)
\end{equation}


\begin{equation}\label{eq:dp-again-set}
  \tupp{\fun^{\op}, \res} \quad \mapsto \quad \{ \imp \in \impsp :  (\fun \funleq \prov(\imp)) \wedge (\req(\imp) \resleq \res) \}
\end{equation}
\begin{equation}\label{eq:dp-again-prof}
  \adp \colon \funsp\op \Ctimes \ressp \fto_{\Pos} \powerset \impsp
\end{equation}

\begin{equation}\label{eq:profunctors-rela}
  \relA \colon \setA \times  \setB \to \Bool.
\end{equation}


\end{forslides}
}

We recall the definition of boolean profunctors:

\begin{ctdefinition}[Boolean profunctors]
  \label{def:boolean-profunctor-again}
  Given two posets~$\posAn$ and~$\posBn$, a \emph{boolean profunctor} from~$\posAn$ to~$\posBn$ is a monotone function of the form
  \begin{equation}
    \label{eq:boolean-profunctor-funa}
    \funa\colon \posAn\op \Ctimes \posBn \fto_{\Pos} \Bool.
  \end{equation}
  This is also written as
  \begin{equation}
    \funa\colon \posAn \profto \posBn.
  \end{equation}
\end{ctdefinition}

We are going to extend this to general profunctors:
\begin{compactitem}
  \item Instead of posets, we will have arbitrary categories.
  \item Instead of \Bool, we will have \Set.
  \item Instead of monotone functions, we will have functors.
\end{compactitem}

\begin{ctdefinition}[Profunctors]
  \label{def:profunctor}
  Given two categories \CatC and \CatD, a \emph{\iindex{profunctor}} from \CatC to \CatD is a functor of the form
  \begin{equation}
    \label{eq:profunctors-def-1}
    \funa\colon \CatC\op \Ctimes \CatD \fto_{\Category} \Set.
  \end{equation}
  This is also written as
  \begin{equation}
    \label{eq:profunctors-def-2}
    \funa\colon \CatC \profto \CatD.
  \end{equation}
\end{ctdefinition}


\todotext{This needs to be re-visited - it doesn't say
what is the equivalence relation}

\begin{widepar}
\begin{ctdefinition}[Profunctor composition]
  \label{def:profunctor-composition}
  Given two profunctors~$\funa\colon \CatC \profto \CatD$ and~$\funb\colon \CatD \profto \CatE$
  we can define their composition~$\funab\colon \CatC \profto \CatE$ as follows:
  \begin{equation}
    \begin{aligned}
    \funabob \colon \Ob(\CatC\op \Ctimes \CatE) & \fto  \Ob\, \Set, \\
    \tup{\prcop, \pre} & \mapsto \productop_{\prd\in\Ob \CatD} \funaob(\prcop, \prd) \times \funbob(\prdop, \pre) / \sim
    \end{aligned}
  \end{equation}
  \begin{equation}
    \begin{aligned}
      \funabmor  \colon \Hom_{(\CatC\op \Ctimes \CatE)}( \tupp{\prcop_1, \pre_1}; \tupp{\prcop_2, \pre_2}) \to \Hom_{\Set}( \funabob(\prcop_1, \pre_1); \funabob(\prcop_2, \pre_2)) \\
      \tup{\pralpha^\ast, \prbeta} \mapsto  \begin{cases}
        \begin{aligned}
          \funabob(\prcop_1, \pre_1) & \to   \funabob(\prcop_2, \pre_2)) \\
          \tup{\prs, \prt} & \mapsto  \tup{
              \funamor(\tup{\pralpha, \catid_{\prd}})(\prs),
              \funbmor(\tup{\catid_{\prdop}, \prbeta})(\prt)}
        \end{aligned}
      \end{cases}
    \end{aligned}
  \end{equation}
  In the formulas:
\begin{equation}
  \pralpha\colon \prc_2 \mto \prc_1, \qquad
  \prbeta\colon \pre_1 \mto \pre_2,
\end{equation}
and~$\tup{\prs, \prt}$ is a pair of elements for which there exists a~$\prd\in\Ob \CatD$ such that
  \begin{equation}
    \prs \in \funaob(\prcop_1, \prd), \qquad\prt \in \funbob(\prdop, \pre).
\end{equation}
\end{ctdefinition}
\end{widepar}

Unfortunately the composition is not associative, hence profunctors do not form a category.

\devel{
\begin{exercise}
  \todotext{Dummy exercise to check code}
\end{exercise}
\begin{solution}
  \todotext{dummy exercise}
\end{solution}
}

\showslides{
\begin{forslides}
  \begin{definition}[$\Cat{V}$-profunctor]
    \label{def:enriched-profunctor}
  Let \CatA and \CatB be categories enriched in~$\Cat{V}$. A profunctor enriched in~$\Cat{V}$ from \CatA to \CatB, denoted~$\funa\colon \CatA \profto \CatB$, is a functor enriched in~$\Cat{V}$:
    \begin{equation}
      \funa\colon \CatA\op \times \CatB \fto \Cat{V}.
  \end{equation}
\end{definition}

\end{forslides}
}
