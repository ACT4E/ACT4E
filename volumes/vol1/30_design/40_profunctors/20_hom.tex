% !TEX root = chapter-standalone.tex


\section{Hom Profunctor}
\linkvideo{spring2021-profunctors:hom-prof} % The Hom profunctor

\todographicsjira{178}{Create the diagrams in the slides}

In this section, we will see that given any category \CatC, its set of morphisms~$\Hom$ can be seen as a profunctor of the form:
\begin{equation}
    \label{eq:hom-signature}
    \Hom_\CatC\colon \CatC \op \times \CatC \toinCat \Set
\end{equation}
First, we need to specify how this profunctor acts on objects and on morphisms.
It maps any two objects~$\Obja,\Objb\in \CatC$ to their~$\Hom$-sets:
\begin{equation}
    \label{eq:hom-action}
    \Hom_\CatC(\tup{\Obja^*,\Objb})=\HomSet{\CatC}{\Obja}{\Objb}
\end{equation}
The action on the morphisms is more involved. First of all, a morphism in~$\CatC\op \times \CatC$ has signature~
\begin{equation}
    \label{eq:hom0}
    \mora \colon \tup{\Obja^*,\Objb}\mto \tup{\Objc^*, \Objd}.
\end{equation}

In particular,~$\mora$ can be written as~$\tup{\mora_1\op,\mora_2}$ with~$\mora_1\op\colon \Obja^*\mto \Objc^*$ and~$\mora_2\colon \Objb\mto \Objd$.
Now, the profunctor maps each morphism~$\mora \colon \tup{\Obja^*,\Objb}\mto \tup{\Objc^*, \Objd}$ to a morphism
\begin{equation}
    \label{eq:hom2}
    \Hom_\CatC(\mora)\colon \HomSet{\CatC}{\Obja}{\Objb}\mto \HomSet{\CatC}{\Objc}{\Objd}.
\end{equation}
More specifically, one has:
\begin{equation}
    \label{eq:hom3}
    \prftree{\mora_1\colon \Objc \mto \Obja}{\mora_2\colon \Objb\mto \Objd}{\morb\colon \Obja\mto \Objb}{(\mora_1\mthen \morb \mthen \mora_2)\colon \Objc\mto \Objd}
\end{equation}
Therefore, the functor maps each morphism~$\mora=\tup{\mora_1\op,\mora_2}\in \HomSet{\CatC}{\Obja}{\Objb}$ to the morphism in \Set given by:
\begin{equation}
    \label{eq:hom4}
    \begin{aligned}
        \Hom_\CatC(\mora)\colon \HomSet{\CatC}{\Obja}{\Objb}&\mto \HomSet{\CatC}{\Objc}{\Objd}\\
        \morb &\mapsto (\mora_1\mthen \morb \mthen \mora_2).
    \end{aligned}
\end{equation}

\linkvideo{spring2021-profunctors:hom-prof-check} % Checking that Hom is a profunctor

We now need to check that this indeed forms a profunctor.
In other words, we want to check that this is a functor of the specific kind described.
First of all, we want to check that
\begin{equation}
    \label{eq:hom5}
    \Hom_\CatC(\mora \mthen \morb)=\Hom_\CatC(\mora)\mthen \Hom_\CatC(\morb).
\end{equation}
Let's consider any~$\mora=\tup{\mora_1\op, \mora_2}$ and~$\morb=\tup{\morb_1\op,\morb_2}$ in~$\CatC\op \times \CatC$.
We know that
\begin{equation}
    \label{eq:hom6}
    \begin{aligned}
        \mora\mthen \morb&=\tup{\mora_1\op \mthen \morb_1\op, \mora_2\mthen \morb_2}\\
        &=\tup{(\morb_1\mthen \mora_1)\op, \mora_2\mthen \morb_2}.
    \end{aligned}
\end{equation}
Therefore, one can write
\begin{equation}
    \label{eq:hom7}
    \begin{aligned}
        \Hom_\CatC(\mora \mthen \morb)(z)&=(\morb_1\mthen \mora_1)\mthen z \mthen (\mora_2\mthen \morb_2)\\
        &=\morb_1\mthen (\mora_1\mthen z \mthen \mora_2) \mthen \morb_2\\
        &=(\mora_1\mthen z \mthen \mora_2)\mthen \Hom_\CatC(\morb)\\
        &=(\Hom_\CatC(\mora)\mthen \Hom_\CatC(\morb))(z),
    \end{aligned}
\end{equation}
proving the composition property.
We now want to check that
\begin{equation}
    \label{eq:hom_identity}
    \Hom_\CatC(\catid_{\tup{\Obja^*,\Objb}})=\catid_{\Hom_\CatC(\tup{\Obja^*,\Objb})}.
\end{equation}

One has
\begin{equation*}
\begin{aligned}
    \Hom_\CatC(\catid_{\tup{\Obja^*,\Objb}})\colon \HomSet{\CatC}{\Obja}{\Objb}&\mto \HomSet{\CatC}{\Obja}{\Objb}\\
    \morb&\mapsto (\catid_\Obja \mthen \morb \mthen \catid_\Objb)=\morb.
\end{aligned}
\end{equation*}
Analogously:

\begin{equation*}
\begin{aligned}
    \catid_{\Hom_\CatC(\tup{\Obja^*,\Objb})}=\catid_{\HomSet{\CatC}{\Obja^*}{\Objb}} \colon \HomSet{\CatC}{\Obja}{\Objb}&\mto \HomSet{\CatC}{\Obja}{\Objb}\\
    \morb&\mapsto \morb.
\end{aligned}
\end{equation*}


\devel{\includepdf[scale=0.8,pages={22-26},nup=1x3,frame,pagecommand={}]{ACT4E-09-design.pdf}}

\showslides{
    \begin{forslides}
        \begin{equation}
            \label{eq:hom1}
            \mora = \tup{\mora_1\op,\mora_2}
        \end{equation}

        \begin{equation}
            \label{eq:hom1b}
            \morb = \tup{\morb_1\op,\morb_2}
        \end{equation}

        \begin{equation}
            \label{eq:hom2b}
            \CatC\op \Ctimes\CatC
        \end{equation}

        \begin{equation}
            \label{eq:CatC}
            \CatC
        \end{equation}
        \begin{equation}
            \label{eq:hom-X}
            \Obja
        \end{equation}

        \begin{equation}
            \label{eq:hom-X-op}
            \Obja^\ast
        \end{equation}

        \begin{equation}
            \label{eq:hom-Y}
            \Objb
        \end{equation}
        \begin{equation}
            \label{eq:hom-Y-op}
            \Objb^\ast
        \end{equation}

        \begin{equation}
            \label{eq:hom-Z}
            \Objc
        \end{equation}

        \begin{equation}
            \label{eq:hom-Z-op}
            \Objc^\ast
        \end{equation}

        \begin{equation}
            \label{eq:hom-W}
            \Objd
        \end{equation}

        \begin{equation}
            \label{eq:hom-W-op}
            \Objd^\ast
        \end{equation}

        \begin{equation}
            \label{eq:f1}
            \mora_1
        \end{equation}
        \begin{equation}
            \label{eq:f2}
            \mora_2
        \end{equation}

        \begin{equation}
            \label{eq:g1}
            \morb_1
        \end{equation}
        \begin{equation}
            \label{eq:g2}
            \morb_2
        \end{equation}


        \begin{equation}
            \label{eq:g1f1}
            \morb_1\mthen \mora_1
        \end{equation}
        \begin{equation}
            \label{eq:f2g2}
            \mora_2\mthen \morb_2
        \end{equation}
    \end{forslides}
}
