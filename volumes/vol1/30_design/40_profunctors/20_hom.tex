% !TEX root = chapter-standalone.tex
\section{Hom Profunctor}
In this section, we will see that given any category \CatC, its set of morphisms~$\Hom$ can be seen as a profunctor of the form:
\begin{equation*}
    \Hom_\CatC\colon \CatC \op \times \CatC \toinCat \Set
\end{equation*}
First, we need to specify how this profunctor acts on objects and on morphisms.
It maps any two objects~$\Obja,\Objb\in \CatC$ to their~$\Hom$-sets:
\begin{equation*}
    \Hom_\CatC(\tup{\Obja^*,\Objb})=\HomSet{\CatC}{\Obja}{\Objb}
\end{equation*}
The action on the morphisms is more involved. First of all, a morphism in~$\CatC\op \times \CatC$ has signature~$\mora \colon \tup{\Obja^*,\Objb}\mto \tup{\Objc^*, \Objd}$.
In particular,~$\mora$ can be written as~$\tup{\mora_1\op,\mora_2}$ with~$\mora_1\op\colon \Obja^*\mto \Objc^*$ and~$\mora_2\colon \Objb\to \Objd$.
Now, the profunctor maps each morphism~$\mora \colon \tup{\Obja^*,\Objb}\mto \tup{\Objc^*, \Objd}$ to a morphism
\begin{equation*}
\Hom_\CatC(\mora)\colon \HomSet{\CatC}{\Obja}{\Objb}\mto \HomSet{\CatC}{\Objc}{\Objd}.
\end{equation*}

\todotext{finish}
\devel{\includepdf[scale=0.8,pages={22-26},nup=1x3,frame,pagecommand={}]{ACT4E-09-design.pdf}}


