\section{Upper/lower bounds}
\linkvideo{spring2021-tradeoffs:tradeoffs:orders:up-low-bounds} % Upper and lower bounds
\begin{ctdefinition}[Upper bounds in a poset]
    \label{def:least-upper-bound}
    \label{def:upper-bounds}
    The \maindef{upper bounds} of a subset~$\subA$ of a poset~\posA are, if they exist, the elements of~\posA which dominate all elements in~$\subA$.
    In other words, the upper bounds of~$\subA$ are the elements of the set
    \begin{equation}
        \upperb \subA\definedas \makeset{ \elb \setin \posAset \mid \forall \ela \setin \subA \colon \ela \posAleq \elb }.
    \end{equation}
\end{ctdefinition}

\begin{ctdefinition}[Least upper bound / \SY{join} / supremum]\label{def:supremum}
    A \maindef{least upper bound} of~$\subA \setsubseteq \posAset$, if it exists, is the least element among the upper bounds of~$\subA$.
    It is denoted~$\join \subA$ or~$\Sup\subA$, and also called the \maindef{join} or \maindef{supremum} of~$\subA$.
\end{ctdefinition}

So, given~$\subA \setsubseteq \posAset$ and~$\elb \setin \posAset$,~$\elb = \join \subA$ if and only if
\begin{enumerate}
    \item $\ela \posAleq \elb, \ \forall \ela \setin \subA$, and
    \item $\ela \posAleq \elc, \ \forall \ela \setin \subA \Imp \elb \posAleq \elc.
          $
\end{enumerate}

If a least upper bound of a subset~$\subA \setsubseteq \posAset$ exists, it is unique (can you prove this?), so we speak of ``the'' least upper bound.

\begin{exercise}
    Let~\posA be a \SY{poset} and~$\subA \setsubseteq \posAset$ a subset of the underlying set of~\posA.
    Show that if~$\join \subA$ exists, then it is unique.
    For this, assume that~$\elb$ and~$\elc$ are both least upper bounds of~$\subA$, and then show that this assumption implies that in fact~$\elb = \elc$.
\end{exercise}
\begin{solution}
    Assume that~$\elb$ and~$\elc$ are both least upper bounds of~$\subA \setsubseteq \posA$.
    In other words, one knows~$\ela \posAleq \elb$ and~$\ela\posAleq \elc$ for all~$\ela\setin \subA$.
    However, one also has~$\elb\posAleq \elc$ and~$\elc\posAleq \elb$ (from~$\elb,\elc$ assumed to be both least upper bounds).
    Because of antisymmetry, this implies~$\elb=\elc$ and proves the uniqueness of least upper bounds in a poset.
\end{solution}

\begin{example}
    Consider the poset~\posA and its subset~$\subA$ depicted in \cref{fig:upper_bound_example_bis}.
    The \textcolor{red}{red} markers~$\textcolor{red}{\marker}$ represent the upper bound of~$\subA$.
    For this specific case, there is \emph{a single} least upper bound.
\end{example}

\begin{figure}[h!]
    \middlesag{upper_bound_2}
    \caption{Example of upper bounds and least upper bound for~$\subA$.  \label{fig:upper_bound_example_bis}}
\end{figure}
\begin{example}
    Least upper bounds need not necessarily exist even in total orders.
    For instance, the subset
    \begin{equation}
        \posReals = \makeset{x\setin \reals \colon x>0}
    \end{equation}
    of the poset~\reals (with the usual ordering) does not have a least upper bound.
\end{example}

Analogously to the case of (least) upper bounds, we can define lower bounds and greatest lower bounds.

\begin{ctdefinition}[Lower bounds in a poset]
    \label{def:greatest-lower-bound}
    \label{def:lower-bounds}
    The \maindef{lower bounds} of a subset~$\subA$ of a poset~\posA are, if they exist, the elements which are dominated by all elements in~$\subA$.
    In other words, the lower bounds of~$\subA$ are the elements of the set
    \begin{equation}
        \lowerb \subA\definedas \makeset{ \elb \setin \posAset \mid \forall \ela \setin \subA \colon \elb \posAleq \ela }.
    \end{equation}
\end{ctdefinition}

\begin{ctdefinition}[Greatest lower bound / \SY{meet} / infimum]\label{def:infimum}
    The \maindef{greatest lower bound}, if it exists, is the greatest among the lower bounds of~$\subA$.
    This is denoted~$\meet \subA$ or~$\Inf \subA$ and also called the \maindef{meet} or \maindef{infimum} of~$\subA$.
\end{ctdefinition}

\begin{exercise}
    Come up with an example of a subset~$\subA$ of a poset~\posA which has lower bounds but no greatest lower bound.
    Then, modify it to have a greatest lower bound.
\end{exercise}

\begin{solution}
    \begin{marginfigure}
        \centering
        \includesag{lower_bound_1}
        \caption{
            Example of lower bounds of~$\subA$.
        }
        \label{fig:lower_bound_1}
    \end{marginfigure}

    \begin{marginfigure}
        \centering
        \includesag{lower_bound_2}
        \caption{
            Example of lower bounds and greatest lower bounds of~$\subA$.
        }
        \label{fig:lower_bound_2}
    \end{marginfigure}

    In \cref{fig:lower_bound_1} you find an example of a subset~$\subA$ of a poset~\posA which has incomparable lower bounds.
    In \cref{fig:lower_bound_2} instead, there is a greatest lower bound.
\end{solution}
\linkvideo{spring2021-tradeoffs:tradeoffs:orders:top-bottom} % Top and bottom

\begin{definition}[Top and bottom]
    \label{def:top}
    \label{def:bot}
    If there is a least upper bound for the entire \SY{lattice}~\posA, it is called the \emph{top} ($\postop$).
    If a greatest lower bound exists, it is called the \emph{bottom} ($\posbot$).
\end{definition}
