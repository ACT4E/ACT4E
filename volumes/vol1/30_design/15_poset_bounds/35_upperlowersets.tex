\section{Upper and lower sets}
\label{sec:UpperLowerSets}

\linkvideo{spring2021-design:up-low-sets} % Upper and lower sets

\begin{definition}[Upper set]
    \label{def:upperset}
    An \maindef{upper set}~$\stylesets{U}$ is a subset of a poset~\posA such that, if~$\ela \setin \stylesets{U}$, then all elements of~\posA that are above~$\ela$ are also in~$\stylesets{U}$.
    In other words:
    \begin{equation}
        \prfperiod{\styleelements{x}\setin \stylesets{U}}{\styleelements{x}\posAleq \styleelements{y}}{\styleelements{y}\setin \stylesets{U}}
    \end{equation}
    %In formulas:
    %\begin{equation}
    %  \text{$\stylesets{U}$ is an upperset of $\posA$} \equiv \forall \styleelements{x}\setin \stylesets{S}, \forall \styleelements{y}\setin \posA \colon \styleelements{x}\posAleq \styleelements{y} \Imp \styleelements{y}\setin \stylesets{S}.
    %\end{equation}
\end{definition}
We call~$\setOfUppersets \posA$ the set of \SY{upper sets} of~\posA.

\begin{definition}[Lower set]
    \label{def:lowerset}
    A \maindef{lower set}~$\stylesets{L}$ is a subset of a poset~\posA such that, if~$\ela \setin \stylesets{L}$, then all elements of~\posA that are below~$\ela$ are also in~$\stylesets{L}$.
    In other words:
    \begin{equation}
        \prfperiod{\styleelements{x}\setin \stylesets{L}}{\styleelements{y}\posAleq \styleelements{x}}{\styleelements{y}\setin \stylesets{L}}
    \end{equation}
    %In formulas:
    %\begin{equation}
    %  \text{$S$ is a lower set of $\posA$} \equiv \forall x\setin S, \forall y\setin \posA \colon y\posAleq x \Imp y\setin S.
    %\end{equation}
\end{definition}
We call~$\setOfLowersets\posA$ the set of \SY{lower sets} of~\posA.
%
%\begin{remark}
%  Note that if~$A$ is an antichain of a poset~$\posA$, then the set
%  \begin{equation}
%    I(A)=\makeset{x\colon x\posAleq y, y\setin A}
%  \end{equation}
%  is a lower set of~$\posA$.
%\end{remark}

\begin{marginfigure}
    \subfloat[
        \label{fig:upperset}
    ]{
        \includesag{70_upper_lower_set}
    }
    \\
    \subfloat[
        \label{fig:upperset_b}
    ]{
        \includesag{70_upper_lower_set_b}
    }
    \caption{}
\end{marginfigure}
%

Given the battery choices~$\makeset{\tupp{\unit[10]{\standardcurrency},\unit[500]{g}}, \tupp{\unit[20]{\standardcurrency},\unit[250]{g}}}$, we can represent an \SY{upper set} as in~\cref{fig:upperset}.
The \SY{upper set} can be interpreted as all the potential battery choices which are dominated by at least one of the two choices we have (in case we want to minimize mass and cost).
Similarly, the \SY{lower set} in \cref{fig:upperset_b} can be interpreted as all the potential battery choices which dominate at least one of the choices we have.
Here when considering ``the choices we have'' in \cref{fig:upperset_b}, we not only consider the two choices directly presented to us, but also any convex combination of them.

%\begin{example}[Upper and lower sets in~\Bool]
%  The booleans~$\makeset{\false, \true }$ form a \SY{poset} with~$\false \leq \true\colon(\Bool,\posleq)$ . The subset~$\makeset{\false} \setsubseteq \Bool$ is not an upper set, since~$\false \leq \true$ and~$\true \notsetin \makeset{\false }$.
%\end{example}
\showslides{
    \begin{forslides}
        \begin{equation*}
            \label{eq:upperposet-def}
            \upperposet{\posA}
        \end{equation*}
        \begin{equation*}
            \label{eq:lowerposet-def}
            \lowerposet{\posA}
        \end{equation*}

        \begin{equation*}
            \label{eq:designlecture-topa}
            \top=\Emptyset
        \end{equation*}
        \begin{equation*}
            \label{eq:designlecture-topb}
            \top=\posA
        \end{equation*}
        \begin{equation*}
            \label{eq:designlecture-bota}
            \bot=\posA
        \end{equation*}
        \begin{equation*}
            \label{eq:designlecture-botb}
            \bot=\Emptyset
        \end{equation*}
        \begin{equation*}
            \label{eq:designlecture-upa}
            \setA=\makeset{\setAel,\setBel,\setCel}
        \end{equation*}
        \begin{equation*}
            \label{eq:designlecture-upb}
            \upit \setA
        \end{equation*}
        \begin{equation*}
            \label{eq:designlecture-upc}
            \downit \setA
        \end{equation*}
        \begin{equation*}
            \label{eq:designlecture-upd}
            \upit \setA \definedas \bigsetunion_{\setAel \setin \setA}\upit \setAel
        \end{equation*}
        \begin{equation*}
            \label{eq:designlecture-upe}
            \upit \setAel \definedas \makeset{\setBel \setin \posAset \colon \setAel \posAleq \setBel}
        \end{equation*}
        \begin{equation*}
            \label{eq:designlecture-upf}
            \tup{\posAset,\posAleq}
        \end{equation*}
        \begin{equation*}
            \label{eq:designlecture-upg}
            \prftree{\posAel\posAleq \posBel}{\upit \posAel \setsupseteq \upit \posBel}
        \end{equation*}
    \end{forslides}
}

\begin{gradedexercise}[\exname{UpperSetsOfPreferences}]
    \label{ex:UpperSetsOfPreferences}

    Consider the poset $\posA$ described by the following Hasse diagram:
    \begin{center}
        \begin{tikzcd}[column sep=small]
                                                 & \text{fast and cheap} \arrow[d,dash] & \\
                                                 & \text{normal service}                & \\
            \text{slow and cheap}\arrow[ur,dash] &                                      & \text{fast and expensive}\arrow[ul,dash]
        \end{tikzcd}
    \end{center}
    Your task in this exercise is to compute all upper sets of $\posA$.
\end{gradedexercise}

\solutionof{UpperSetsOfPreferences}