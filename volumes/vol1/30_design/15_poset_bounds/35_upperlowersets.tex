\section{Upper and lower sets}
\label{sec:UpperLowerSets}

\linkvideo{spring2021-design:up-low-sets} % Upper and lower sets

\begin{definition}[Upper set]
    \label{def:upperset}
    An \emph{upper set}~$\stylesets{U}$ is a subset of a poset~$\posA$ such that, if~$\ela \in \stylesets{U}$, then all elements of~$\posA$ that are above~$\ela$ are also in~$\stylesets{U}$.
    In other words:
    \begin{equation}
        \prfperiod{\styleelements{x}\in \stylesets{U}}{\styleelements{x}\posAleq \styleelements{y}}{\styleelements{y}\in \stylesets{U}}
    \end{equation}
    %In formulas:
    %\begin{equation}
    %  \text{$\stylesets{U}$ is an upperset of $\posA$} \equiv \forall \styleelements{x}\in \stylesets{S}, \forall \styleelements{y}\in \posA \colon \styleelements{x}\posAleq \styleelements{y} \Imp \styleelements{y}\in \stylesets{S}.
    %\end{equation}
\end{definition}
We call~$\setOfUppersets \posA$ the set of upper sets of~$\posA$.

\begin{definition}[Lower set]
    \label{def:lowerset}
    A \emph{lower set}~$\stylesets{L}$ is a subset of a poset~$\posA$ such that, if~$\ela \in \stylesets{L}$, then all elements of~$\posA$ that are below~$\ela$ are also in~$\stylesets{L}$.
    In other words:
    \begin{equation}
        \prfperiod{\styleelements{x}\in \stylesets{L}}{\styleelements{y}\posAleq \styleelements{x}}{\styleelements{y}\in \stylesets{L}}
    \end{equation}
    %In formulas:
    %\begin{equation}
    %  \text{$S$ is a lower set of $\posA$} \equiv \forall x\in S, \forall y\in \posA \colon y\posAleq x \Imp y\in S.
    %\end{equation}
\end{definition}
We call~$\setOfLowersets\posA$ the set of lower sets of~$\posA$.
%
%\begin{remark}
%  Note that if~$A$ is an antichain of a poset~$\posA$, then the set
%  \begin{equation}
%    I(A)=\{x\colon x\posAleq y, y\in A\}
%  \end{equation}
%  is a lower set of~$\posA$.
%\end{remark}

Given the battery choices~$\{\tupp{\unit[10]{\standardcurrency},\unit[500]{g}},\tupp{\unit[20]{\standardcurrency},\unit[250]{g}}\}$, one can represent an upper set as in~\cref{fig:upperset}.
The upper set can be interpreted as all the potential battery choices which are dominated (in case we want to minimize mass and cost).
Similarly, the lower set can be interpreted as all the potential battery choices which would be better than the ones on the curve (\cref{fig:upperset_b}).

\begin{figure}
    \subfloat[
        \label{fig:upperset}
    ]{
        \includesag{70_upper_lower_set}
    }
    \subfloat[
        \label{fig:upperset_b}
    ]{
        \includesag{70_upper_lower_set_b}
    }
    \caption{}
\end{figure}
%
%\begin{example}[Upper and lower sets in~\Bool]
%  The booleans~$\{\false, \true \}$ form a poset with~$\false \leq \true\colon(\Bool,\posleq)$ . The subset~$\{\false\} \subseteq \Bool$ is not an upper set, since~$\false \leq \true$ and~$\true \notin \{\false \}$.
%\end{example}
