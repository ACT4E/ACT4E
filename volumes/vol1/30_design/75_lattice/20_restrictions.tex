% !TEX root = chapter-standalone.tex


\section{Restrictions and alternatives}

\linkvideo{spring2021-functorial-comp-b:solving-queries:or-and-and} % Join and Meet in DP

\subsection{Union of Design Problems}
Let~$\adpa\colon \F{\posgenA}\profto \R{\posgenB}$ and~$\adpb\colon \F{\posgenA}\profto \R{\posgenB}$ be design problems.
We define the \emph{union}~$\adpa \vee \adpb$ to be the design problem which is feasible whenever \emph{either}~$\adpa$ or~$\adpb$ is feasible.
This models~$\adpa$ and~$\adpb$ as interchangeable technologies: either one can replace the other.

\begin{definition}[Union of design problems]
    \label{def:union_dp}
    Given two design problems~$\adpa \colon \F{\posgenA} \profto \R{\posgenB}$ and~$\adpb\colon \F{\posgenA} \profto \R{\posgenB}$, their \emph{union}~$\adpa \vee \adpb\colon \F{\posgenA} \profto \R{\posgenB}$ is defined by
    \begin{equation}
        \begin{aligned}
        (\adpa \vee \adpb)
            \colon \F{\posgenA}\op \times \R{\posgenB} & \toinPos \Bool \\
            \tup{\F{\posgenAel}^*, \R{\posgenBel}} & \mapsto \adpa(\F{\posgenAel}^*, \R{\posgenBel}) \vee \adpb(\F{\posgenAel}^*, \R{\posgenBel}).
        \end{aligned}
    \end{equation}
\end{definition}

The union of design problems is represented as in~\cref{fig:uniondp}.

\begin{figure}[h!]
    \centering
    \includesag{52_union}
    \caption{Diagrammatic representation of the union of design problems. }
    \label{fig:uniondp}
\end{figure}

\begin{example}
    Jeb's Spaceship Parts is locked in a deadly rivalry with Starshow Bob to supply engines for the new X103 space orbiter. Neither knows the exact operational scenario that the X103 will encounter, but have provided a range of performance benchmarks for their engines (\cref{fig:exunion_1}).
    \begin{figure}[h!]
        \centering
        \includesag{50_rival_jeb_bob}
        \caption{Example of two engine producers. }
        \label{fig:exunion_1}
    \end{figure}
    Back at NASA headquarters, Beau has uploaded Jeb and Bob's data in order to construct the design problem reported in~\cref{fig:exunion_2}.
    \begin{figure}[h!]
        \centering
        \includesag{50_rival_beau}
        \caption{Example of the union of the engine design problems. }
        \label{fig:exunion_2}
    \end{figure}
\end{example}

\subsection{In DPI}

\label{subsec:dpi-union}

\todojira{204}{Finish}

The union of two design problems with implementation is a design problem with the implementation
space~$\impsp=\impsp_{1}\sqcup\impsp_{2}$, and it represents the
exclusive choice between two possible alternative families of designs.
\begin{definition}[Coproduct of DPIs]
    \label{def:parallel-1}Given two DPIs with same functionality and
    resources~$\adpa=\tup{\funsp,\ressp,\impsp_{1},\prov_{1},\req_{1}}$
    and~$\adpb=\tup{\funsp,\ressp,\impsp_{2},\prov_{2},\req_{2}}$,
    define their co-product as
    \begin{equation}
        \adpa\sqcup\adpb\definedas\tup{\funsp,\ressp,\impsp_{1}\sqcup\impsp_{2},\prov,\req} ,
    \end{equation}
    where
    \begin{eqnarray}
        \prov & \colon & \imp\mapsto\begin{cases}
                                        \prov_{1}(\imp), & \text{if }\imp\in\impsp_{1},\\
                                        \prov_{2}(\imp), & \text{if }\imp\in\impsp_{2},
        \end{cases}\label{eq:dppar-exec-1}\\
        \req & \colon & \imp\mapsto\begin{cases}
                                       \req_{1}(\imp), & \text{if }\imp\in\impsp_{1},\\
                                       \req_{2}(\imp), & \text{if }\imp\in\impsp_{2}.
        \end{cases}\nonumber
    \end{eqnarray}
\end{definition}

%
%\captionsideleft{\label{fig:dpcoproduct}}{
%  \includegraphics[scale=0.33]{gmcdp_coproduct}
%}

\begin{figure}[h!]
    \centering
    \includesag{20_dpi_coproduct}
    \caption{}
    \label{fig:dpcoproduct}
\end{figure}

\subsection{Intersection of Design Problems}

Given two design problems~$\adpa, \adpb \colon \F{\posgenA} \profto \R{\posgenB}$, we can define a design problem~$\adpa \wedge \adpb$ that is feasible if only if~$\adpa$ and~$\adpb$ are both feasible.
We call~$\adpa \wedge \adpb$ the \emph{intersection} of~$\adpa$ and~$\adpb$.
One interpretation of~$\adpa \wedge \adpb$ is that~$\adpa$ and~$\adpb$ are two slightly different models of the same process, and we want to make sure that the design is conservatively feasible for both models.

\begin{definition}[Intersection of design problems]
    \label{def:intersection_dp}
    \label{def:dp-intersection}
    Given design problems~$\adpa\colon \F{\posgenA} \profto \R{\posgenB}$ and~$\adpb\colon \F{\posgenA} \profto \R{\posgenB}$,
    their \emph{intersection} is denoted~$(\adpa \wedge \adpb)\colon \F{\posgenA} \profto \R{\posgenB}$, defined by:
    \begin{equation}
        \begin{aligned}
        (\adpa \wedge \adpb)
            \colon \F{\posgenA}\op \times \R{\posgenB} & \toinPos \Bool \\
            \tup{\F{\posgenAel}^*, \R{\posgenBel}} & \mapsto \adpa(\F{\posgenAel}^*, \R{\posgenBel}) \wedge  \adpb(\F{\posgenAel}^*, \R{\posgenBel}).
        \end{aligned}
    \end{equation}
\end{definition}
The intersection of design problems is represented as in~\cref{fig:intersectiondp}.

\begin{figure}[h!]
    \centering
    \includesag{52_intersection}
    \caption{Diagrammatic representation of the intersection of design problems. }
    \label{fig:intersectiondp}
\end{figure}

We can directly generalize the intersection~$\adpa \wedge \adpb$ by allowing~$\adpa$ and~$\adpb$ to have different domain and codomains,~$\adpa \colon \F{\posgenA} \profto \R{\posgenB}$ and~$\adpb \colon \F{\posgenC} \profto \R{\posgenD}$.
We call this putting two design problems ``in parallel''.

\subsection{In DPI}\label{subsec:dpi-intersection}

\todotextjira{205}{not sure the definition below is what we want}
\devel{
    \begin{definition}[Intersection of DPIs]
        \label{def:intersection-1}
        Given two DPIs with same functionality and
        resources~$\adpa=\tup{\funsp,\ressp,\impsp_{1},\prov_{1},\req_{1}}$
        and~$\adpb=\tup{\funsp,\ressp,\impsp_{2},\prov_{2},\req_{2}}$,
        define their intersection as
        \begin{equation}
            \adpa\sqcap\adpb\definedas\tup{\funsp,\ressp,\impsp_{1}\cap\impsp_{2},\prov,\req} ,
        \end{equation}
        where
        \begin{eqnarray}
            \prov & \colon & \imp\mapsto\begin{cases}
                                            \prov_{1}(\imp), & \text{if }\imp\in\impsp_{1}\cap \impsp_{2} \text{ and }\prov_1(\imp)\posleq \prov_2(\imp)\\
                                            \prov_{2}(\imp), & \text{if }\imp\in\impsp_{1}\cap \impsp_{2} \text{ and }\prov_2(\imp)\posleq \prov_1(\imp)\\
                                            \bot_\funsp,& \text{else}.
            \end{cases}\label{eq:dppar-exec-2}\\
            \req & \colon & \imp\mapsto\begin{cases}
                                           \req_{1}(\imp), & \text{if }\imp\in\impsp_{1}\cap \impsp_{2} \text{ and }\req_1(\imp)\posgeq \req_2(\imp)\\
                                           \req_{2}(\imp), & \text{if }\imp\in\impsp_{1}\cap \impsp_{2} \text{ and }\req_2(\imp)\posgeq \req_1(\imp)\\
                                           \top_\ressp,& \text{else}.
            \end{cases}\nonumber
        \end{eqnarray}
    \end{definition}}

\begin{figure}[h!]
    \centering
    \includesag{dpi_intersection}
    \caption{}
    \label{fig:intersection}
\end{figure}

