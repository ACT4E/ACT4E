% !TEX root = chapter-standalone.tex

\section{Monotone maps}\label{sec:monotonicity-monotone-maps}
\todojira{80}{@Andrea: Add some motivation from design.
    \eg generalization of "non-decreasing" and "non-increasing" from real numbers to partial orders}

%\linkvideo{spring2021-functors:semi-and-fun:mon-functions} % Monotone functions
\linkvideo{spring2021-functors:semi-and-fun:mon-functions:fun-req-mon} % Functionalities and requirements
\linkvideo{spring2021-functors:semi-and-fun:mon-functions:mon-on-pos} % Monotone functions on posets

\begin{definition}[Monotone map]
    \label{def:monotone}
    A \emph{\iindex{monotone map}} between two posets~$\posA=\tup{\posAset, \posAleq}$ and~$\posB=\tup{\posBset, \posBleq}$ is a map~$\mapa$ that preserves the ordering, in the sense that
    \begin{equation}
        \prfperiod{\posAel_1 \posAleq \posAel_2 }{\mapa(\posAel_1) \posBleq \mapa(\posAel_2)}
        %\posAel_1 \posAleq \posAel_2 \quad \Imp \quad \mapa(\posAel_1) \posBleq \mapa(\posAel_2).
    \end{equation}
\end{definition}

\begin{remark}
    Given a poset~$\posA$, the map~$\catid_\posA\colon \posA \mto \posA$ is monotone, since if $\posAel_1\posAleq \posAel_2$, then $\catid_\posA(\posAel_1)=\posAel_1 \posAleq  \posAel_2 =  \catid_\posA(\posAel_2 )$.

\end{remark}

\begin{definition}[Antitone map]
    \label{def:antitone}
    An \emph{\iindex{antitone map}} between two posets~$\posA=\tup{\posAset, \posAleq}$ and~$\posB=\tup{\posBset, \posBleq}$ is a map~$\mapa$ that reverses the ordering, in the sense that
    \begin{equation}
        \prfperiod{\posAel_1 \posAleq \posAel_2}{\mapa(\posAel_1) \posgeq_\posB \mapa(\posAel_2)}
        %\posAel_1 \posAleq \posAel_2 \quad \Imp \quad \mapa(\posAel_1) \posBleq \mapa(\posAel_2).
    \end{equation}
\end{definition}

\begin{marginfigure}
    \includesag{unit_manufacturing}
    \caption{Unit cost vs.
        number of widgets.}
    \label{fig:unit_manufacturing}
\end{marginfigure}

\begin{marginfigure}
    \includesag{total_manufacturing}
    \caption{Total cost vs. number of widgets.}
    \label{fig:total_manufacturing}
\end{marginfigure}

\begin{example}[Unit cost, total cost]
    Assume that you want to produce some widgets, and that the manifacturing cost depends on the number of widgets.
    The function describing the total cost~$\stylemaps{t}\colon \natnumbers\to \nonNegReals$ is a map between the ordered sets~$\natnumbers$ and~$\nonNegReals$, and maps each quantity of widgets to a total manufacturing cost (\cref{fig:total_manufacturing}).
    Clearly,~$\stylemaps{t}$ is a monotone function.
    Conversely, the unit cost function~$\stylemaps{u}\colon \natnumbers\to \nonNegReals$ is antitone (\cref{fig:unit_manufacturing}).
\end{example}

\begin{example}[Rounding functions]
    \label{ex:rounding-functions}
    In this example we look at three rounding functions: \funceil, \funfloor, and \rtntte.
    Both the maps
    \begin{equation*}
        \begin{aligned}
            \funceil\colon \tup{\reals,\leq} & \to \tup{\natnumbers,\leq}                    \\
            x                                & \mapsto i \in \natnumbers \colon i-1<x\leq i,
        \end{aligned}
    \end{equation*}
    and
    \begin{equation*}
        \begin{aligned}
            \funfloor\colon \tup{\reals,\leq} & \to \tup{\natnumbers,\leq}                     \\
            x                                 & \mapsto i \in \natnumbers \colon i\leq x< i+1.
        \end{aligned}
    \end{equation*}
    are monotone, since~$x\leq y$ implies both~$\funceil(x)\leq \funceil(y)$ and~$\funfloor(x)\leq\funfloor(y)$.
    \todojira{83}{@Andrea: Define IEEE754 formally}
    \todographicsjira{84}{@Gioele: Add picture with the 3 functions, ceil, floor, rtntee}
\end{example}

\begin{example}[Cardinality map]
    In \cref{ex:hasseinclusion} we presented the poset arising from the power set of a set~$\setA=\{a,b,c\}$ and ordered via subset inclusion.

    The map~$\vert \cdot \vert \colon \powerset{\setA} \to \natnumbers$ (cardinality), is a monotone map (\cref{fig:cardinality}).
    \begin{figure}[h!]
        \centering
        \includesag{40_dpcatfig_exmonotone}
        \caption{The cardinality map is a monotone map. }
        \label{fig:cardinality}
    \end{figure}
\end{example}

\begin{lemma}
    Consider a discrete poset~$\posA$ and a poset~$\posB$.
    Any map~$\mapa \colon \posA\to \posB$ is monotone.
\end{lemma}
\newcommand{\samewidth}[1]{\makebox[3cm]{$#1$}}
\begin{proof}
    Since~$\posA$ is a discrete poset, one has
    \begin{equation}
        \prfdoubleperiod{\posAel_1\posAleq \posAel_2}{\posAel_1=\posAel_2}
        %\posAel_1\posAleq \posAel_2 \iff \posAel_1=\posAel_2.
    \end{equation}
    Therefore, one has
    \begin{equation}
        \prfperiod{
            \prftree{
                \prftree[r]{\quad($P$ discrete)}{
                    \samewidth{\posAel_1\posAleq \posAel_2}
                }{
                    \samewidth{\posAel_1=\posAel_2}
                }
            }{
                \samewidth{\mapa(\posAel_1)=\mapa(\posAel_2)}
            }
        }{
            \samewidth{\mapa(\posAel_1)\posBleq \mapa(\posAel_2)}
        }
    \end{equation}
\end{proof}
Unless indicated otherwise, in this book all maps between posets are assumed to be monotone or will turn out to be monotone.
In a similar way, one can define antitone maps.

\linkvideo{spring2021-tradeoffs:tradeoffs:orders:set-based-filtering} % Set-based filtering
\begin{example}
    We now look at an example of \textbf{set-based filtering}, where filtering refers to online inference.
    Suppose that we want to track the value of a quantity~$x\in [0,100]$, without having a priory information about~$x$.
    We are equipped with sensors, which periodically measure the quantity~$x$ with some variable precision.
    At time~$t\in \nonNegReals $ they produce an \emph{observation}~$y_t\colon x_t\in [l_t,u_t]$.
    Also, note that the quantity fluctuates randomly, and we bound its ``velocity'' to be~$\dot{x}_t\in [-1,1]$ (except at boundaries).
    At the beginning, our information state~$\bar{i}_0$ could be that~$x\in [0,100]$.
    At time 0, we get an observation~$y_0$, that says~$x\in [21,24]$.
    The new information state can be obtained by ``fusing'' the two inputs we have received about~$x$.
    This corresponds to the intersection
    \begin{equation*}
        x\in \left( [0,100] \cap [21,24]\right)\equiv x\in [21,24].
    \end{equation*}
    Let's now say we get an observation~$y_1$ which says~$x\in [19,22]$.
    We now need to take into account the evolution/dynamics of the quantity we are tracking.
    From the interval~$[21,24]$ we know that the variable could have evolved in~$[20,25]$ (dynamics are bounded with a unit increase/decrease).
    Therefore, the new information state is given by
    \begin{equation*}
        x\in \left( [20,25] \cap [21,24]\right)\equiv x\in [21,24].
    \end{equation*}
    One of the structures which could sustain this kind of inference, is the of \emph{posets of intervals} (\cref{def:poset_intervals}).
    The Hasse diagram representing a situation related to this diagram could be as reported in \cref{fig:hasse_filtering}.
    \begin{figure}[h!]
        \centering
        \includesag{080_hasse_filtering}
        \caption{}
        \label{fig:hasse_filtering}
    \end{figure}
\end{example}

\devel{
    A monotone map is an \emph{order isomorphism} if the other direction of the implication holds as well:
    \begin{equation*}
        \prfdouble{\posAel \posleq_\posA \posBel}{\mapa(\posAel) \posleq_\posB \mapa(\posBel)}
        \posAel \posleq_\posA \posBel \quad \Leftrightarrow \quad \mapa(\posAel) \posleq_\posB \mapa(\posBel).
    \end{equation*}
}
