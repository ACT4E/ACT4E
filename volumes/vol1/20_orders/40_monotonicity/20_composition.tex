% !TEX root = chapter-standalone.tex


\section{Compositionality of monotonicity}
Note that monotonicity is a compositional property.
\begin{lemma}
    Given posets~$\posA,\posB,\posC$ and monotone maps~$\mapa\colon \posA \to \posB$ and~$\mapb\colon \posB \to \posC$, the composite map~$\mapa\then \mapb\colon  \posA \to \posC$ is
    monotone as well.
\end{lemma}
\begin{proof}
    Consider~$\posAel_1,\posAel_2 \in \posA$,~$\posBel_1,\posBel_2\in \posB$.
    We have, by definition,
    \begin{equation}
        \begin{aligned}
            \posAel_1\posAleq \posAel_2 &\Imp \mapa(\posAel_1)\posAleq \mapa(\posAel_2)\\
            \posBel_1\posBleq \posBel_2 &\Imp \mapb(\posBel_1)\posCleq \mapb(\posBel_2).
        \end{aligned}
    \end{equation}
    By substituting the above in the map composition formula, one has
    \begin{equation}
        \posAel_1\posAleq \posAel_2 \Imp (\mapa\then \mapb)(\posAel_1) \posCleq (\mapa\then \mapb)(\posAel_2),
    \end{equation}
    which is the monotonicity condition for the composite map~$(\mapa\then \mapb)$.
\end{proof}
