% !TEX root = chapter-standalone.tex


\section{Compositionality of monotonicity}
Note that monotonicity is a compositional property.
\begin{lemma}
  Given posets~$\posA,\posB,\posC$ and monotone maps~$f\colon \posA \to \posB$ and~$g\colon \posB \to \posC$, the composite map~$f\then g\colon  \posA \to \posC$ is
  monotone as well.
\end{lemma}
\begin{proof}
  Consider~$\posAel_1,\posAel_2 \in \posA$,~$\posBel_1,\posBel_2\in \posB$. We have, by definition,
  \begin{equation}
    \begin{aligned}
      \posAel_1\posAleq \posAel_2 &\Imp f(\posAel_1)\posAleq f(\posAel_2)\\
      \posBel_1\posBleq \posBel_2 &\Imp g(\posBel_1)\posCleq g(\posBel_2).
    \end{aligned}
  \end{equation}
  By substituting the above in the map composition formula, one has
  \begin{equation}
    \posAel_1\posAleq \posAel_2 \Imp (f\then g)(\posAel_1) \posCleq (f\then g)(\posAel_2),
  \end{equation}
  which is the monotonicity condition for the composite map~$(f\then g)$.
\end{proof}

