\section[The category \Lat]{The category~\Lat}
In this section, we want to abstract the concept of lattice and describe a category in which the objects are lattices themselves, and the morphisms are lattice homomorphisms.
We call this category~\Lat.

\begin{ctdefinition}[Lattice homomorphism]
    \label{def:lattice_homomorphism}
    Given two lattices~$\posA, \posB$, a \emph{\iindex{lattice homomorphism}} is a map~$\mora\colon \posA\to \posB$ which preserves meets and joins:
    \begin{equation}
        \begin{aligned}
            \mora(\posAel \meet_\posA \posBel) & = \mora(\posAel) \meet_\posB \mora(\posBel), \\
            \mora(\posAel \join_\posA \posBel) & = \mora(\posAel) \join_\posB \mora(\posBel).
        \end{aligned}
    \end{equation}
\end{ctdefinition}

\begin{example}
    We consider the lattices~$\posA=\tupp{\powerset \{\sbretzel,\sfondue\}, \setintersection, \setunion}$, and~$\posB=\tupp{\{\stea\com \swine\} \com \max \com \min}$, where~$\min$ ($\max$) refer to the minmum (maximum) alcoholic content of the beverage (assuming Swiss beers, which have alcohol content lower than wine).
    Furthermore, consider
    \begin{equation}
        \label{eq:example_lat_hom_a}
        \begin{aligned}
            \mora\colon \powerset \{\sbretzel,\sfondue\} & \mto \{\swine, \stea\}                \\
            \setA                                        & \mapsto \begin{cases}
                                                                       \stea,  & \sbretzel \in \setA, \\
                                                                       \swine, & \text{ otherwise}.
                                                                   \end{cases}
        \end{aligned}
    \end{equation}
    The explicit evaluations of~$\mora$ are reported in \cref{tab:example_lattice_hom}.
    \begin{margintable}
        \begin{center}
            \begin{tabular}{c|c}
                $\setA$                    & $\mora(\setA)$ \\
                \midrule
                $\emptyset$                & \swine         \\
                $\{\sbretzel\}$            & \stea          \\
                $\{\sfondue\}$             & \swine         \\
                $\{ \sbretzel, \sfondue\}$ & \stea
            \end{tabular}
        \end{center}
        \caption{\label{tab:example_lattice_hom}}
    \end{margintable}

    Is~$\mora$ a lattice homomorphism?
    Yes.
    Let's check it explicitly.
    Consider~$\setA,\setB\subseteq \{\sbretzel, \sfondue\}$.
    We need to show that
    \begin{equation}
        \label{eq:hom_lat_cond_a}
        \mora(\setA\setintersection \setB)=\max\{\mora(\setA),\mora(\setB)\}\\
    \end{equation}
    and
    \begin{equation}
        \label{eq:hom_lat_cond_b}
        \mora(\setA\setunion \setB)=\min\{\mora(\setA),\mora(\setB)\}.
    \end{equation}

    Technically, one can check every possible pair of~$\setA,\setB$ (only 16 for this case), but that's not efficient.
    First, consider~$\mora(\setA\setintersection \setB)=\swine$.
    Following~\cref{eq:example_lat_hom_a}, this means~$\sbretzel \notin \setA\setintersection \setB$ (in other words, either~$\sbretzel \notin \setA$,~$\sbretzel \notin \setB$, or both).
    At least one of~$\mora(\setA)$ and~$\mora(\setB)$ is~$\swine$, because
    \begin{equation*}
        \prfcomma{\sbretzel \notin \setA}{\mora(\setA)=\swine} \qquad
        \prfperiod{\sbretzel \notin \setB}{\mora(\setB)=\swine}
    \end{equation*}
    This implies~$\max\{\mora(\setA),\mora(\setB)\}=\swine$, which verifies \cref{eq:hom_lat_cond_a}.

    If instead, one has~$\mora(\setA\setintersection \setB)=\stea$, then~$\sbretzel \in \setA\setintersection \setB$, meaning that~$\sbretzel \in \setA$ and~$\sbretzel \in \setB$.
    Therefore,~$\max\{ \mora(\setA),\mora(\setB)\}=\stea$, which verifies \cref{eq:hom_lat_cond_a}.

    Condition \cref{eq:hom_lat_cond_b} can be verified analogously.
\end{example}

The notion of lattice homomorphism can be extended to bounded lattices.

\begin{ctdefinition}[Bounded lattice homomorphism]
    \label{def:bounded_lat_homomorphism}
    Given two bounded lattices~$\posA,\posB$, a \emph{\iindex{bounded lattice homomorphism}} is a lattice homomorphism~$\mora\colon \posA\to \posB$ which also preserves top and bottom:
    \begin{equation}
        \begin{aligned}
            \mora(\posbot_\posA) & =\posbot_\posB \\
            \mora(\postop_\posA) & =\postop_\posB
        \end{aligned}
    \end{equation}
\end{ctdefinition}

Note that (bounded) lattice homomorphisms are necessarily monotone.

We are now ready to introduce~\Lat and~$\BoundedLat$.

\begin{ctdefinition}[Category \Lat]
    \label{def:Lat}
    The category~\iindex{\Lat} is defined by:
    \begin{enumerate}
        \item \emph{Objects}: The objects of this category are all lattices.
        \item \emph{Morphisms}: The morphisms from a lattice~$\Obja$ to a lattice~$\Objb$ are the lattice homomorphisms maps from~$\Obja$ to~$\Objb$.
        \item \emph{Identity morphism}: The identity morphism for the lattice~$\Obja$
              is the identity map~$\catid_\Obja$.
        \item \emph{Composition operation}: The composition operation is composition of maps.
    \end{enumerate}
\end{ctdefinition}

\begin{ctdefinition}[Category \BoundedLat]
    \label{def:BoundedLat}
    The category~\iindex{\BoundedLat} is defined by:
    \begin{enumerate}
        \item \emph{Objects}: The objects of this category are all bounded lattices.
        \item \emph{Morphisms}: The morphisms from a lattice~$\Obja$ to a lattice~$\Objb$ are the bounded lattice homomorphisms maps from~$\Obja$ to~$\Objb$.
        \item \emph{Identity morphism}: The identity morphism for the bounded lattice~$\Obja$
              is the identity map~$\catid_\Obja$.
        \item \emph{Composition operation}: The composition operation is composition of maps.
    \end{enumerate}
\end{ctdefinition}

\begin{exercise}
    \label{ex:lat_is_cat}
    Show that~$\Lat$ is a category.
\end{exercise}
\begin{solution}
    Clearly, given any lattice~$\Obja$, the identity map~$\catid_\Obja$ is a lattice homomorphism, since
    \begin{equation*}
        \begin{aligned}
            \catid_\Obja(\Objaela \meet_\Obja \Objaelb) & =\Objaela \meet_\Obja \Objaelb  \\
            \catid_\Obja(\Objaela \join_\Obja \Objaelb) & =\Objaela \join_\Obja \Objaelb.
        \end{aligned}
    \end{equation*}
    This said, the identity map satisfies unitality.
    Now, given lattice homomorphisms
    \begin{equation}
        \mora\colon \Obja \mto \Objb \qqand \morb\colon \Objb\mto \Objc,
    \end{equation}
    their composition is a lattice homomorphism, since
    \begin{equation*}
        \begin{aligned}
            (\mora \mthen \morb)(\Objaela \meet_\Obja \Objaelb)
             & = \mora(\Objaela \meet_\Obja \Objaelb) \mthen \morb                         \\
             & = \morb(\mora(\Objaela) \meet_\Objb \mora(\Objaelb))                        \\
             & =(\mora \mthen \morb)(\Objaela) \meet_\Objc (\mora \mthen \morb)(\Objaelb),
        \end{aligned}
    \end{equation*}
    and
    \begin{equation*}
        \begin{aligned}
            (\mora \mthen \morb)(\Objaela \join_\Obja \Objaelb)
             & = \mora(\Objaela \join_\Obja \Objaelb) \mthen \morb                         \\
             & = \morb(\mora(\Objaela) \join_\Objb \mora(\Objaelb))                        \\
             & =(\mora \mthen \morb)(\Objaela) \join_\Objc (\mora \mthen \morb)(\Objaelb),
        \end{aligned}
    \end{equation*}
    We have already checked in the past the map composition is associative (\eg when checking that \Set and \Pos are categories).
\end{solution}

\begin{exercise}
    \label{ex:boundlat_is_cat}
    Show that~$\BoundedLat$ is a category.
\end{exercise}

\begin{solution}
    Consider the solution of \cref{ex:lat_is_cat} as a starting point.
    Clearly, given any lattice~$\Obja$, the identity map~$\catid_\Obja$ is also a bounded lattice homomorphism, since
    \begin{equation*}
        \begin{aligned}
            \catid_\Obja(\bot_\Obja) & =\bot_\Obja  \\
            \catid_\Obja(\top_\Obja) & =\top_\Obja.
        \end{aligned}
    \end{equation*}
    This said, the identity map satisfies unitality.
    Now,  Now, given lattice homomorphisms~$\mora\colon \Obja \mto \Objb, \morb\colon \Objb\mto \Objc$, their composition is also a bounded lattice homomorphism, since
    \begin{equation*}
        \begin{aligned}
            (\mora \mthen \morb)(\bot_\Obja)
             & = \morb(\bot_\Objb) \\
             & = \bot_\Objc,
        \end{aligned}
    \end{equation*}
    and
    \begin{equation*}
        \begin{aligned}
            (\mora \mthen \morb)(\top_\Obja)
             & = \morb(\top_\Objb) \\
             & = \top_\Objc.
        \end{aligned}
    \end{equation*}
    We have already checked in the past the map composition is associative (\eg when checking that \Set and \Pos are categories).
\end{solution}
