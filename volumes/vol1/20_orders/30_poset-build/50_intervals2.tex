% !TEX root = chapter-standalone.tex
\section{A different poset of intervals}
\begin{definition}[Another poset of intervals]
\label{def:second_interval_poset}
Given a partially ordered set~$\posA$, an interval is an ordered pair of elements~$\tup{l,u}$ of~$\posA$, such that~$l\posAleq u$.
One can define a \emph{poset of intervals} on~$\posA$, denoted~$\posintbis{P}$. Intervals can be ordered using the following rule:
\begin{equation}
    \prftree[double]{\tup{\posAel_1,\posAel_2}\posleq_{\posintbis{\posA}}\tup{\posBel_1,\posBel_2}}{(\posAel_1\posleq_\posA \posBel_1) \wedge (\posAel_2\posleq_\posA \posBel_2)}
    %\tup{\posAel_1,\posAel_2}\posleq_{\posintbis{P}}\tup{\posBel_1,\posBel_2} \Leftrightarrow (\posAel_1\posleq_\posA \posBel_1) \wedge (\posAel_2\posleq_\posA \posBel_2).
\end{equation}
\end{definition}

This partially ordered set will be instrumental when we define uncertainty in design problems.

\begin{exercise}
    Check that the relation defined in \cref{def:second_interval_poset} is indeed a poset.
\end{exercise}