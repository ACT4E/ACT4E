% !TEX root = chapter-standalone.tex

\section{A different poset of intervals}
\begin{definition}[Another poset of intervals]
    \label{def:second_interval_poset}
    Given a partially ordered set~$\posA$, an interval is an ordered pair of elements~$\tupp{l,u}$ of~$\posA$, such that~$l\posAleq u$.
    One can define a \emph{poset of intervals} on~$\posA$, denoted~$\posintbis\posA$.
    Intervals can be ordered using the following rule:
    \begin{equation}
        \prfdoubleperiod{\tupp{\posAel_1,\posAel_2}\posleqof{\posintbis{\posA}}\tupp{\posBel_1,\posBel_2}}{(\posAel_1\posleqof\posA \posBel_1) \wedge (\posAel_2\posleqof\posA \posBel_2)}
        %\tupp{\posAel_1,\posAel_2}\posleqof{\posintbis{P}}\tupp{\posBel_1,\posBel_2} \Leftrightarrow (\posAel_1\posleqof\posA \posBel_1) \wedge (\posAel_2\posleqof\posA \posBel_2).
    \end{equation}
\end{definition}

This partially ordered set will be instrumental when we define uncertainty in design problems.

\begin{exercise}
    Check that the relation defined in \cref{def:second_interval_poset} is indeed a poset.
\end{exercise}
\begin{solution}
    We check the three conditions.
    \begin{itemize}
        \item First, we know that~$\tupp{\posAel_1,\posBel_1}\posleqof{\posintbis{\posA}}\tupp{\posAel_1,\posBel_1}$, since~$\posAel_1\posAleq \posAel_1$ and~$\posBel_1\posAleq \posBel_1$.
        \item Second,~$\tupp{\posAel_1,\posBel_1}\posleqof{\posintbis{\posA}}\tupp{\posAel_2,\posBel_2}$ and~$\tupp{\posAel_2,\posBel_2}\posleqof{\posintbis{\posA}}\tupp{\posAel_3,\posBel_3}$ imply $$\tupp{\posAel_1,\posBel_1}\posleqof{\posintbis{\posA}}\tupp{\posAel_3,\posBel_3}.$$
        \item Third, if~$\tupp{\posAel_1,\posBel_1}\posleqof{\posintbis{\posA}}\tupp{\posAel_2,\posBel_2}$ and~$\tupp{\posAel_2,\posBel_2}\posleqof{\posintbis{\posA}}\tupp{\posAel_1,\posBel_1}$, then~$\posAel_1=\posAel_2$ and~$\posBel_1=\posBel_2$.
    \end{itemize}
\end{solution}


\todotext{Missing a lot of stuff here... not even a figure about these intervals?}