% !TEX root = chapter-standalone.tex

\section{Product of posets}
%\linkvideo{spring2021-tradeoffs:tradeoffs:orders:composing-posets} % Composing posets
\linkvideo{spring2021-tradeoffs:tradeoffs:orders:composing-posets:product-poset} % Product of posets
We can think of the product of posets.

\begin{definition}[Product of posets]
    \label{def:productposet}
    Given two posets~$\posA=\tupp{\posAset, \posAleq}$
    and~$\posB=\tupp{\posBset, \posBleq}$, the \emph{product poset} is~$\posA\times \posB=\tup{\posAset \times \posBset, \ordleq_{\posA\times \posB}}$, where~$\posAset \times \posBset$ is the Cartesian product of the sets~$\posAset$ and $\posBset$ (\cref{def:cartesian-product}), and the order~$\posleqof{\posA \times \posB}$ is given by:
    \begin{equation}
        \prfdoubleperiod{\tup{\posAel_1, \posBel_1}
            \posleqof{\posA\times \posB}
            \tup{\posAel_2, \posBel_2}}{(\posAel_1 \posAleq \posAel_2) \wedge
            (\posBel_1 \posBleq \posBel_2)}
        %\tup{\posAel_1, \posBel_1}
        %\posleqof{\posA\times \posB}
        %\tup{\posAel_2, \posBel_2}
        %\quad
        %\Leftrightarrow
        %\quad
        %(\posAel_1 \posAleq \posAel_2) \wedge
        %(\posBel_1 \posBleq \posBel_2).
    \end{equation}
\end{definition}
Recalling the pizza recipes example, we have the two posets representing time and money.
Given that we want to minimize both time and costs, by considering the money poset containing elements \unit[10]{\CHF}, \unit[20]{\CHF}, and \unit[30]{\CHF}, and the time poset containing elements \unit[1]{hours}, and \unit[2]{hours}, one can represent the product as in~\cref{fig:productpizza}.

\todotext{We should highlight the quantitative-but-not-numeric angle.
    Change the example
    to have ``cheap'', ``midrange'', ``expensive'' and ``fast'', ``slow''.
}

\begin{figure*}[h!]
    \centering
    \includesag{70_hasse_pizza_product}
    \caption{Product poset of time and cost for pizza recipes.}
    \label{fig:productpizza}
\end{figure*}

\begin{example}
    Consider now two posets and their product, given in~\cref{fig:composing_posets_1}.
    \begin{figure*}[h!]
        \centering
        \includesag{40_exposet_1_1}
        \caption{Product of two posets.}
        \label{fig:composing_posets_1}
    \end{figure*}
\end{example}

\todotext{These are not good as exercises.
    Put in text.
}

\begin{exercise}
    \label{ex:width}
    If you know the width of the posets~$\posA$ and~$\posB$, can you compute the width of~$\posA\cartprod\posB$?
\end{exercise}
\begin{solution}
    From~\cite{bezrukovantichains}, we know
    \begin{equation*}
        \posetwidth(\posA)\cartprod \posetwidth(\posB)\leq \posetwidth(\posA\cartprod \posB)\leq \min \{ \vert \posA\vert \cdot \posetwidth(\posB),\vert \posB \vert \cdot \posetwidth(\posA)\}.
    \end{equation*}
\end{solution}

\begin{exercise}
    \label{ex:height}
    If you know the height of the posets~$\posA$ and~$\posB$, can you compute the height of~$\posA\cartprod\posB$?
\end{exercise}
%
\begin{solution}
    Yes.
    First of all, one can construct the longest chain in~$\posA$:
    \begin{equation*}
        A=\{\posAel_1,\ldots, \posAel_{\posetheight(\posA)}\}.
    \end{equation*}
    Furthermore, one can construct the longest chain in~$\posB$:
    \begin{equation*}
        B=\{\posBel_1,\ldots, \posBel_{\posetheight(\posB)}\}.
    \end{equation*}
    Out of them, one can construct the chain
    \begin{equation*}
        C=\{ \tup{\posAel_1,\posBel_1},\tup{\posAel_2,\posBel_1},\ldots, \tup{\posAel_{\posetheight(\posA)}, \posBel_1}, \tup{\posAel_{\posetheight(\posA)}, \posBel_2},\ldots\},
    \end{equation*}
    which has height~$\posetheight(\posA)+\posetheight(\posB)-1$.
    So we know that at least~$\posetheight(\posA\cartprod \posB)\geq \posetheight(\posA)+\posetheight(\posB)-1$.
    Now, consider a chain~$\{\tup{\posAel_1,\posBel_1},\ldots, \tup{\posAel_n,\posBel_n}\}$ in~$\posA\cartprod \posB$.
    In general, this means that at least a coordinate of~$\tup{\posAel_{i},\posBel_{i}}$ must increase in$\tup{\posAel_{i+1},\posBel_{i+1}}$.
    The first coordinate can only increase~$\posetheight(\posA)-1$ times, and the second one~$\posetheight(\posB)-1$ times.
    Summing up, the total number of elements in the chain is \emph{at most}~$\posetheight(\posA)+\posetheight(\posB)-1$.
    Note that this result holds only assuming that~$\posA$ and~$\posB$ are not empty (for that case,~$\posetheight(\posA\cartprod \posB)=0$).
\end{solution}
