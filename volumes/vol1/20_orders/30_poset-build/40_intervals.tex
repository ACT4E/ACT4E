% !TEX root = chapter-standalone.tex

\section{Poset of intervals}
\linkvideo{spring2021-tradeoffs:tradeoffs:orders:interval-poset} % Poset of intervals
\begin{definition}[Poset of intervals]
    \label{def:poset_intervals}
    An interval is an ordered pair of elements~$\tup{\posAel,\posBel}$ of~$\posA$, such that~$\posAel \posAleq \posBel$.
    Given a poset~$\posA$, one can define a \emph{poset of intervals} on~$\posA$.
    Intervals can be ordered by inclusion:
    \begin{equation}
        \prfdoubleperiod{\tup{\posAel_1,\posBel_1}\posleq_{\posint{\posA}}\tup{\posAel_2,\posBel_2} }{ (\posAel_1\posAleq \posAel_2)\wedge (\posBel_2\posAleq \posBel_1)}
        %\tup{\posAel_1,\posBel_1}\ordleq_{\posint{\posA}}\tup{\posAel_2,\posBel_2} \Leftrightarrow (\posAel_1\posAleq \posAel_2)\wedge (\posBel_2\posAleq \posBel_1).
    \end{equation}
\end{definition}

\begin{exercise}
    Check that the relation defined in \cref{def:poset_intervals} is indeed a poset.
\end{exercise}
\begin{solution}
    We prove the three conditions.
    \begin{itemize}
        \item First, we know that~$\tup{\posAel_1,\posBel_1}\posleq_{\posint{\posA}}\tup{\posAel_1,\posBel_1}$, since~$\posAel_1\posAleq \posAel_1$ and~$\posBel_1\posAleq \posBel_1$.
        \item Second,~$\tup{\posAel_1,\posBel_1}\posleq_{\posint{\posA}}\tup{\posAel_2,\posBel_2}$ and~$\tup{\posAel_2,\posBel_2}\posleq_{\posint{\posA}}\tup{\posAel_3,\posBel_3}$ imply~$\tup{\posAel_1,\posBel_1}\posleq_{\posint{\posA}}\tup{\posAel_3,\posBel_3}$.
        \item Third, if~$\tup{\posAel_1,\posBel_1}\posleq_{\posint{\posA}}\tup{\posAel_2,\posBel_2}$ and~$\tup{\posAel_2,\posBel_2}\posleq_{\posint{\posA}}\tup{\posAel_1,\posBel_1}$, then~$\posAel_1=\posAel_2$ and~$\posBel_1=\posBel_2$.
    \end{itemize}
\end{solution}
