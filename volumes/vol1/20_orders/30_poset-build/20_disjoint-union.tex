% !TEX root = standalone.tex

\section{Disjoint union of posets}

Similarly to what we have done for sets in \cref{sec:coproductset}, we can think of alternatives in the poset case through their disjoint union.

\begin{definition}[Disjoint union of posets]
  Given posets~$\tup{\posA, \posAleq}$ and~$\tup{\posB, \posBleq}$, we can define their \emph{disjoint union}~$\tup{\posA + \posB, \posleq_{\posA + \posB}}$, where~$\posA + \posB$
  is the disjoint union of the sets~$\posA$ and~$\posB$ (\cref{def:disjoint-union}), and the
  order~$\posleq_{\posA + \posB}$ is given by:
  \begin{equation}
    \posAel \posleq_{\posA + \posB} \posBel \quad\equiv\quad
    \begin{cases}
      \posAel \posAleq \posBel, & \posAel,\posBel \in \posA, \\
      \posAel \posBleq \posBel, & \posAel,\posBel \in \posB,
      %\false,  & \text{otherwise}.
    \end{cases}
  \end{equation}
  with
  \begin{equation}
    \begin{aligned}
      \posleq_{\posA+\posB}\colon (\posA+\posB)\times (\posA+\posB)&\to \Bool\\
      \disunionA{\posAel_1},\disunionA{\obja_2}&\mapsto (\obja_1\ordleq_{\Obja} \obja_2)\\
      \disunionB{\objb},\disunionA{\obja}&\mapsto \false\\
      \disunionA{\obja},\disunionB{\objb}&\mapsto \true\\
      \disunionB{\objb_1},\disunionB{\objb_2}&\mapsto (\objb_1\ordleq_\Objb \objb_2).
    \end{aligned}
  \end{equation}
\end{definition}


\begin{example}
  Consider the posets~$\Obja=\tup{\diamond, \star}$ with~$\diamond \ordleq_\Obja \star$, and~$\Objb=\tup{\dagger,\ast}$, with~$\ast \ordleq_\Objb \dagger$. Their disjoint union can be represented as in \cref{fig:poset-coproduct}.

  \begin{figure}[h!]
    \centering
    \includesag{40_disjoint_union}
    \caption{Disjoint union of posets. \todographics{lacks set shadow and labels}}
    \label{fig:poset-coproduct}
  \end{figure}
\end{example}
