% !TEX root = chapter-standalone.tex

\section{Opposite of a poset}\label{sec:opposite-of-a-poset}

\begin{definition}
    \label{def:poset-opposite}
    The \emph{opposite} of a poset~$\posA=\tup{\posAset, \posAleq}$ is the poset denoted~$\posA\op=\tup{\posAset, \posAleq\op}$.
    It has the same elements as~$\posA$, but is equipped with the reverse ordering (\cref{fig:poset-opposite}).
    For a given~$\posAel \in \posA$, we will sometimes write~$\posAel^*$ do denote its corresponding copy in~$\posA\op$, in order to emphasize that~$\posAel$ and~$\posAel^*$ belong to distinct posets.
    However, often we will not be so pedantic with our notation.
    Reversing the order means that, for all $\posAel,\posBel\in \posA$,
    \begin{equation}
        \prfperiod{\posAel \posAleq \posBel}{\posBel^* \posAleq\op \posAel^*}
    \end{equation}
\end{definition}

\begin{figure}[tbh]
    \centering
    \includesag{40_dpcatfig_opposite}
    \caption{Opposite of a poset.}
    \todographics{@Gioele: Add labels $posA$, $posA^*$}
    
    \label{fig:poset-opposite}
\end{figure}

\begin{example}[Credit and debt]
    Let us define the set
    \begin{equation*}
        \posAset=\reals \times \{\stdcurr\}=\{0.00,0.01,0.02,\dots\}
    \end{equation*}
    of all \stdcurr \ monetary quantities approximated to the cent.
    \todotext{@Gioele: This doesn't work anymore with bitcoin. Maybe just use CHF as the standard currency?}
    From this set we can define two posets:~$\posA^{+} = \tup{\posAset, \posAleq}$ and~$\posA^{-} = \tup{\posAset, \posAgeq}$, that are the opposite of each other.
    If the context is that, given two quantities~\unit[1]{\stdcurr} and \unit[2]{\stdcurr}, we prefer \unit[1]{\stdcurr} to \unit[2]{\stdcurr} (for example because it is a cost to pay to acquire a component), then we are working in~$\posA^{+}$, otherwise we are working in~$\posA^{-}$ (for example because it represents the price at which we are selling our product).
    Traditionally, in double-entry ledger systems, the numbers were not written with negative signs, but rather in color: red and black.
    From this convention we get the idioms ``being in the black'' and ``being in the red''.
\end{example}
