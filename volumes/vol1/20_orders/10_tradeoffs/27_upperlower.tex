
\section{Upper and lower bounds}
\linkvideo{spring2021-tradeoffs:tradeoffs:orders:up-low-bounds} % Upper and lower bounds
\begin{ctdefinition}[Upper bounds in a poset]
    \label{def:least-upper-bound}
    The \emph{upper bounds} of a subset~$\setA$ of a poset~$\posA$ are, if they exist, the elements of~$\posA$ which dominate all elements in~$\setA$.
    In other words, the upper bounds of~$\setA$ are the elements of the set
    \begin{equation*}
        \{ \elb \in \posA \mid \forall \ela \in \setA  \colon \ela \leq \elb \}.
    \end{equation*}
    A \emph{least upper bound} of~$\setA \subseteq \posA$, if it exists, is a least element among the upper bounds of~$\setA$.
    It is denoted~$\join \setA$ or~$\Sup\setA$, and also called the \emph{join} or \emph{supremum} of~$\setA$.
    So, given~$\setA \subseteq \posA$ and~$\elb \in \posA$,~$\elb =  \join \setA$ if and only if
    \begin{enumerate}
        \item $\ela \leq \elb, \ \forall \ela \in \setA$, and
        \item $\ela \leq \elc, \ \forall \ela \in \setA \Rightarrow \elb \leq \elc.
              $
    \end{enumerate}
    If a least upper bound of a subset~$\setA \subseteq \posA$ exists, it is unique (can you prove this?
    ), so we speak of ``the'' least upper bound.
\end{ctdefinition}

\begin{exercise}
    Let~$\posA$ be a poset and~$\setA \subseteq \posA$ a subset of the underlying set of~$\posA$.
    Show that if~$\join \setA \in \posA$ exists, then it is unique.
    For this, assume that~$\elb$ and~$\elb'$ are both least upper bounds of~$\setA$, and then show that this assumption implies that in fact~$\elb = \elb'$.
\end{exercise}
\begin{solution}
    \missingsolution
    \todotextjira{291}{@Gioele: Write solution}
\end{solution}

\begin{marginfigure}
    \centering
    \includesag{upper_bound_1}
    \caption{Example of upper bounds. }
    \label{fig:upper_bound_example}
\end{marginfigure}

\begin{marginfigure}
    \centering

    \includesag{upper_bound_2}
    \caption{Example of upper bounds and least upper bound. }
    \label{fig:upper_bound_example_bis}
\end{marginfigure}

\begin{example}
    Consider the poset~$\posA$ and its subset~$\setA$ depicted in \cref{fig:upper_bound_example}.
    The \textcolor{red}{red} markers~$\textcolor{red}{\marker}$ represent the upper bound of~$\setA$.
    For this specific case, there is \emph{no} least upper bound, since the upper bounds do not dominate each other.
    Similarly, the poset~$\posB$ and its subset~$\setA$ depicted in \cref{fig:upper_bound_example_bis} possess two upper bounds.
    In this case, however, we can relate the two upper bounds, and find a least upper bound.

\end{example}

\begin{example}
    Least upper bounds need not necessarily exist even in total orders.
    For instance, the subset
    \begin{equation*}
        \posReals = \{x\in \reals \colon x>0\}
    \end{equation*}
    of the poset~$\reals$ (with the usual ordering) does not have a least upper bound.
\end{example}

Analogously to the case of (least) upper bounds, one can define lower bounds and greatest lower bounds.

\begin{ctdefinition}[Lower bounds in a poset]
    \label{def:greatest-lower-bound}
    The \emph{lower bounds} of a subset~$\setA$ of a poset~$\posA$ are, if they exist, the elements which are dominated by all elements in~$\setA$.
    The \emph{greatest lower bound}, if it exists, is the greatest among the lower bounds of~$\setA$.
    This is denoted~$\meet \setA$ or~$\Inf \setA$ and also called the \emph{meet} or \emph{infimum} of~$\setA$.
\end{ctdefinition}

\begin{exercise}
    Come up with an example of a subset~$\setA$ of a poset~$\posA$ which has lower bounds but no greatest lower bound.
    Then, modify it to have a greatest lower bound.
\end{exercise}

\begin{solution}
    \begin{marginfigure}
        \centering
        \includesag{lower_bound_1}
        \caption{Example of lower bounds. \label{fig:lower_bound_1}}
    \end{marginfigure}
    \begin{marginfigure}
        \centering
        \includesag{lower_bound_2}
        \caption{Example of lower bounds and greatest lower bounds. \label{fig:lower_bound_2}}
    \end{marginfigure}

    In \cref{fig:lower_bound_1} you find an example of a subset~$\setA$ of a poset~$\posA$ which has uncomparable lower bounds.
    In \cref{fig:lower_bound_2} instead, one also haves a greatest lower bound.

\end{solution}
