% !TEX root = chapter-standalone.tex

\section{Chains and Antichains}
\label{sec:chains-antichains}

\begin{definition}[Chain in a poset]
    \label{def:chain}
    Given a poset~$\posA = \tup{\posAset, \posAleq}$, a \emph{chain} is a subset~$\stylesets{S} \subseteq \posAset$ such that any two elements of~$\stylesets{S}$ are comparable.
    If~$\ela, \elb \in \stylesets{S}$, then~$\ela \posAleq \elb$ or~$\elb \posAleq \ela$.
    %a sequence of elements~$
    %{\posAel_i}$ in~$\posA$ where two successive elements are comparable:
    %    \begin{equation}
    %        \prftree{i \Nleq j }{\posAel_i \posAleq \posAel_j}
    %        %i \Nleq j \Rightarrow s_i \posAleq s_j.
    %    \end{equation}
\end{definition}

\begin{definition}[Antichain in a poset]
    \label{def:antichain}
    An \emph{antichain} is a subset~$\stylesets{S}$ of a poset where no elements are comparable.
    If~$\setAel,\setBel \in \stylesets{S}$, then:
    \begin{equation}
        \prfperiod{\setAel \posAleq \setBel}{\setAel=\setBel}
    \end{equation}
\end{definition}

We denote the set of antichains of a poset~$\posA$ by~$\antichains\posA$.

\begin{remark}
    Note that given a poset~$\posA=\tup{\posAset,\posAleq}$, the empty set~$\emptyset$ is both a chain and an antichain.
\end{remark}

\begin{marginfigure}
    \centering
    \includesag{70_antichain}
    \caption{Example of discrete antichains.}
    \label{fig:antichain}
\end{marginfigure}

In the context of battery choices, consider the diagram reported in~\cref{fig:antichain}.
The black markers represent an antichain of choices~$\{\tupp{\unit[10]{USD},\unit[500]{g}},\tupp{\unit[20]{USD},\unit[250]{g}}\}$.
It is a set of antichains because they do not dominate each other: one is cheaper, but takes longer, and the other is more expensive, but quicker, making them uncomparable.
The red marker~$\textcolor{red}{\marker}$ represents a recipe which cannot be part of the antichain, since it is dominated by~$\tupp{\unit[20]{USD},\unit[250]{g}}$.
Similarly, one could think of a continuous law which relates battery cost and mass.
%
\begin{marginfigure}
    \centering
    \includesag{70_antichain_2}
    \caption{Example of continuous antichains.}
    \label{fig:antichain_2}
\end{marginfigure}
%
For instance, consider the antichain given by~$\text{mass}=500-25\cdot \text{cost}$, with maximum possible cost~$\unit[20]{USD}$ (\cref{fig:antichain_2}).

We will now have a look at couple more examples.
\begin{example}
    Let's consider the poset~$\posA=\tup{\posAset,\posAleq}$ where~$\posAel \posAleq \posBel$ if~$\posAel$ is a divisor of~$\posBel$ and~$\posAset=\{1,5,10,11,13,15\}$.
    A chain of~$\posA$ is~$\{1,5,10,15\}$.
    An antichain of~$\posA$ is~$\{10,11,13\}$.
\end{example}

\begin{example}
    Consider \cref{ex:hasseinclusion}.
    Examples of chains are
    \begin{equation}
        \{\varnothing,\{\setAel\},\{\setAel,\setBel\},\{\setAel,\setBel,\setCel\}\}, \quad  \{\varnothing,\{\setBel\},\{\setBel,\setCel\},\{\setAel,\setBel,\setCel\}\}.
    \end{equation}
    Examples of antichains are
    \begin{equation}
        \{\{\setAel\},\{\setBel\},\{\setCel\}\}, \quad \{ \{\setAel,\setBel\},\{\setAel,\setCel\}, \{\setBel,\setCel\}\}.
    \end{equation}
\end{example}

\section{Upper and lower sets}
\label{sec:UpperLowerSets}

\linkvideo{spring2021-design:up-low-sets} % Upper and lower sets

\begin{definition}[Upper set]
    \label{def:upperset}
    An \emph{upper set}~$\stylesets{U}$ is a subset of a poset~$\posA$ such that, if~$\ela \in \stylesets{U}$, then all elements of~$\posA$ that are above~$\ela$ are also in~$\stylesets{U}$.
    In other words:
    \begin{equation}
        \prfperiod{\styleelements{x}\in \stylesets{U}}{\styleelements{x}\posAleq \styleelements{y}}{\styleelements{y}\in \stylesets{U}}
    \end{equation}
    %In formulas:
    %\begin{equation}
    %  \text{$\stylesets{U}$ is an upperset of $\posA$} \equiv \forall \styleelements{x}\in \stylesets{S}, \forall \styleelements{y}\in \posA \colon \styleelements{x}\posAleq \styleelements{y} \Imp \styleelements{y}\in \stylesets{S}.
    %\end{equation}
\end{definition}
We denote by~$\uppersets \posA$ the set of upper sets of~$\posA$.

\begin{definition}[Lower set]
    \label{def:lowerset}
    A \emph{lower set}~$\stylesets{L}$ is a subset of a poset~$\posA$ such that, if~$\ela \in \stylesets{L}$, then all elements of~$\posA$ that are below~$\ela$ are also in~$\stylesets{L}$.
    In other words:
    \begin{equation}
        \prfperiod{\styleelements{x}\in \stylesets{L}}{\styleelements{y}\posAleq \styleelements{x}}{\styleelements{y}\in \stylesets{L}}
    \end{equation}
    %In formulas:
    %\begin{equation}
    %  \text{$S$ is a lower set of $\posA$} \equiv \forall x\in S, \forall y\in \posA \colon y\posAleq x \Imp y\in S.
    %\end{equation}
\end{definition}
We denote by~$\lowersets \posA$ the set of lower sets of~$\posA$.
%
%\begin{remark}
%  Note that if~$A$ is an antichain of a poset~$\posA$, then the set
%  \begin{equation}
%    I(A)=\{x\colon x\posAleq y, y\in A\}
%  \end{equation}
%  is a lower set of~$\posA$.
%\end{remark}

Given the battery choices~$\{\tupp{\unit[10]{USD},\unit[500]{g}},\tupp{\unit[20]{USD},\unit[250]{g}}\}$, one can represent an upper set as in~\cref{fig:upperset} on the left.
For the case of the curve~$\text{mass}=750-25\cdot \text{cost}$, one can represent an upper set as in \cref{fig:upperset} on the right.
The upper set can be interpreted as all the potential battery choices which are dominated (in case we want to minimize mass and cost).
Similarly, the lower set can be interpreted as all the potential battery choices which would be better than the ones on the curve.

\begin{figure}[h!]
    \centering
    \includesag{70_upper_lower_set}
    \caption{Example of upper and lower sets of a poset of pizza recipes.}
    \label{fig:upperset}
\end{figure}

%
%\begin{example}[Upper and lower sets in~\Bool]
%  The booleans~$\{\false, \true \}$ form a poset with~$\false \leq \true\colon(\Bool,\posleq)$ . The subset~$\{\false\} \subseteq \Bool$ is not an upper set, since~$\false \leq \true$ and~$\true \notin \{\false \}$.
%\end{example}
