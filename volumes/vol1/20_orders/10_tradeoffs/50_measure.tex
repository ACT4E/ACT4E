
\section{Measuring posets}
\begin{definition}[Width of a poset]
    \label{def:poset-width}
    The \emph{width} of a poset, denoted~$\posetwidth(\posA)$, is the maximum cardinality of an antichain in~$\posA$.
\end{definition}

\begin{definition}[Height of a poset]
    \label{def:poset-height}
    The \emph{height} of a poset, denoted~$\posetheight(\posA)$, is the maximum cardinality of a chain in~$\posA$.
\end{definition}

\begin{marginfigure}
    \centering
    \includesag{height_width_posets}
    \caption{Example for height and width of a poset.}
    \label{fig:poset-height-width}
\end{marginfigure}

\begin{example}
    Consider the poset~$\posA$ reported in \cref{fig:poset-height-width}.
    The longest antichains of~$\posA$ are~$\makeset{\sapple, \schoco, \scheese}$,~$\makeset{\sbretzel, \schoco, \scheese}$,~$\makeset{\sapple, \schoco, \sgrapes}$,~$\makeset{\sbretzel, \schoco, \sgrapes}$, and~$\makeset{\sbretzel, \sburger, \scheese}$.
    Therefore,~$\posetwidth(\posA)=3$.
    The longest chain in the poset is given by~$\makeset{\scarrot,\schoco,\sburger,\sfondue}$, and therefore~$\posetheight(\posA)=4$.
\end{example}

\begin{exercise}
    \label{ex:width}
    If you know the width of the posets~$\posA$ and~$\posB$, can you compute the width of~$\posA\cartprod\posB$?
\end{exercise}
\begin{solution}
    From~\cite{bezrukovantichains}, we know
    \begin{equation*}
        \posetwidth(\posA)\cartprod \posetwidth(\posB)\leq \posetwidth(\posA\cartprod \posB)\leq \min \{ \vert \posA\vert \cdot \posetwidth(\posB),\vert \posB \vert \cdot \posetwidth(\posA)\}.
    \end{equation*}
\end{solution}

\begin{exercise}
    \label{ex:height}
    If you know the height of the posets~$\posA$ and~$\posB$, can you compute the height of~$\posA\cartprod\posB$?
\end{exercise}
%
\begin{solution}
    Yes.
    First of all, one can construct the longest chain in~$\posA$:
    \begin{equation*}
        A=\{\posAel_1,\ldots, \posAel_{\posetheight(\posA)}\}.
    \end{equation*}
    Furthermore, one can construct the longest chain in~$\posB$:
    \begin{equation*}
        B=\{\posBel_1,\ldots, \posBel_{\posetheight(\posB)}\}.
    \end{equation*}
    Out of them, one can construct the chain
    \begin{equation*}
        C=\{ \tup{\posAel_1,\posBel_1},\tup{\posAel_2,\posBel_1},\ldots, \tup{\posAel_{\posetheight(\posA)}, \posBel_1}, \tup{\posAel_{\posetheight(\posA)}, \posBel_2},\ldots\},
    \end{equation*}
    which has height~$\posetheight(\posA)+\posetheight(\posB)-1$.
    So we know that at least~$\posetheight(\posA\cartprod \posB)\geq \posetheight(\posA)+\posetheight(\posB)-1$.
    Now, consider a chain~$\{\tup{\posAel_1,\posBel_1},\ldots, \tup{\posAel_n,\posBel_n}\}$ in~$\posA\cartprod \posB$.
    In general, this means that at least a coordinate of~$\tup{\posAel_{i},\posBel_{i}}$ must increase in$\tup{\posAel_{i+1},\posBel_{i+1}}$.
    The first coordinate can only increase~$\posetheight(\posA)-1$ times, and the second one~$\posetheight(\posB)-1$ times.
    Summing up, the total number of elements in the chain is \emph{at most}~$\posetheight(\posA)+\posetheight(\posB)-1$.
    Note that this result holds only assuming that~$\posA$ and~$\posB$ are not empty (for that case,~$\posetheight(\posA\cartprod \posB)=0$).
\end{solution}
