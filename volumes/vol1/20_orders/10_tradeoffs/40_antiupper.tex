% !TEX root = chapter-standalone.tex

\section{From antichains to uppersets, and viceversa}

\linkvideo{spring2021-design:up-low-closure} % Upper and lower closure
\begin{definition}[Upper closure operator]
  \label{def:upperclosure}
  The \emph{upper closure operator}~$\upit $ maps a subset to the smallest upper set that includes it:
  \begin{equation}
    \begin{aligned}
      \upit  \colon \posPA&\to \posUA \\
      S&\mapsto \{\styleelements{y}\in \posA \mid \exists \styleelements{x}\in \stylesets{S} \colon \styleelements{x}\posAleq \styleelements{y}\}.
    \end{aligned}
  \end{equation}
\end{definition}
\begin{remark}
  Note that, by definition, an upper set is closed to upper closure.
\end{remark}
\begin{lemma}
  For any~$\stylesets{S}\in \posPA$,~$\upit  \stylesets{S}$ is in fact an upper set.
\end{lemma}
\begin{proof}
  Suppose~$\styleelements{y}\in \upit  \stylesets{S}$ and~$\styleelements{z}\in \posA$, and suppose $\styleelements{y}\posAleq \styleelements{z}$.
  By definition there exists a~$\styleelements{x}$ s.t.~$\styleelements{x}\posAleq \styleelements{y}$, meaning that~$\styleelements{x}\posAleq \styleelements{z}$.
  Thus,~$\styleelements{z}\in \upit  \stylesets{S}$, as was to be shown.
\end{proof}


\begin{lemma}
  The upper closure operator~$\upit$ is a monotone map.
\end{lemma}
\begin{proof}
  Consider the posets~$\tup{\posPA,\subseteq}$ and~$\tup{\posUA ,\supseteq}$, and~$\stylesets{S_1},\stylesets{S_2}\in \posPA$.
  It is clear that given~$\stylesets{S_1}\subseteq \stylesets{S_2}$, one has
  \begin{equation*}
    \{\styleelements{y}\in \posA\mid \exists \styleelements{x}\in \stylesets{S_1}\colon \styleelements{x}\posAleq \styleelements{y}\} \supseteq \{\styleelements{y}\in \posA\mid \exists \styleelements{x}\in \stylesets{S_2}\colon \styleelements{x}\posAleq \styleelements{y}\}.
  \end{equation*}
  Therefore,~$\upit  \stylesets{S_1}\supseteq \ \upit  \stylesets{S_2}$, satisfing the monotonicity property for~$\upit $.
\end{proof}

\begin{lemma}
  \label{lem:up-cl-inj-antichains}
  Let~$\setA$ and~$\setB$ be subsets of~$\posA$ that are antichains. Then
  \begin{equation*}
    \prftree{\upit  \setA = \ \upit  \setB}{\setA = \setB}
    %\upit  \setA = \ \upit  \setB \quad \Imp \quad \setA = \setB.
  \end{equation*}
\end{lemma}

\begin{proof}
  First, let's fix an~$\setAel \in \setA$.
  From~$\upit  \setA = \ \upit  \setB$ we know that in particular$\setA \subseteq \ \upit  \setB$.
  This means that for our fixed~$\setAel \in \setA$ there exists~$\setBel \in \setB$ such that~$\setBel \posleq \setAel$.
  From~$\upit \setA = \ \upit  \setB$ it also follows that~$\setB \subseteq \ \upit  \setA$, so to the~$\setBel \in \setB$ given above, there exists~$\setAel' \in \setA$ such that~$\setAel' \posleq \setBel$.
  In total, we have~$\setAel' \posleq \setBel \posleq \setAel$, and since $\setA$ is an antichain, we must have~$\setAel' = \setAel$.
  This implies that~$\setAel' = \setBel = \setAel$. In particular, we have~$\setAel \in \setB$.

  The above shows that~$\setA \subseteq \setB$.
  To show~$\setB \subseteq \setA$, we can fix any~$\setBel \in \setB$ and repeat the above argumentation, now with the roles of~$\setA$ and~$\setB$ exchanged.
\end{proof}

In the example of battery choices, first, consider the upper closure of a single element of the poset, for instance~$\posAel_1=\tup{\unit[10]{USD},\unit[500]{g}}$ (\cref{fig:upperclosure_1}, left).
Second, we can look at the upper closure when we add the choice~$\posAel_2=\tup{\unit[20]{USD},\unit[250]{g}}$ (\cref{fig:upperclosure_2}, center).
Note that the upper set of the subset formed by the two elements is the union of the upper sets of the single elements.
Finally, we can also define the set~$\setA=\{\tup{\text{cost},\text{mass}}\mid \text{mass}=750-25\cdot \text{cost}, \forall \text{cost} \in [0,20]\}$, and find its upper closure (\cref{fig:upperclosure_1}, right).
\begin{figure*}[h!]
  \begin{center}
    \includesag{70_upper_closure_1}
  \end{center}
  \caption{Example of uppler closure for different sets of battery choices. }
  \label{fig:upperclosure_1}
\end{figure*}

\begin{definition}[Lower closure operator]
  \label{def:lowerclosure}
  The \emph{lower closure operator}~$\downit$ maps a subset to the smallest lower set that includes it:
  \begin{equation*}
    \begin{aligned}
      \downit \colon \posPA&\to \posLA\\
      \stylesets{S}&\mapsto \{ \styleelements{y}\in \posA \mid \exists \styleelements{x}\in \stylesets{S} \colon \styleelements{y}\posAleq \styleelements{x}\}.
    \end{aligned}
  \end{equation*}
\end{definition}

\begin{lemma}
  \label{lem:lower_closure_monotone}
  The lower closure operator~$\downit$ is a monotone map.
\end{lemma}

\begin{exercise}
  Prove \cref{lem:lower_closure_monotone}.
\end{exercise}
\begin{solution}
  Consider the posets~$\tup{\posPA,\subseteq}$ and~$\tup{\posLA,\subseteq}$, and let~$\stylesets{S_1},\stylesets{S_2}\in \posPA$.
  It is clear that given~$\stylesets{S_1}\subseteq \stylesets{S_2}$, one has
  \begin{equation}
    \{\styleelements{y}\in \posA\mid \exists \styleelements{x}\in \stylesets{S_1}\colon \styleelements{y}\posAleq \styleelements{x}\} \subseteq \{\styleelements{y}\in \posA\mid \exists \styleelements{x}\in \stylesets{S_2}\colon \styleelements{y}\posAleq \styleelements{x}\}.
  \end{equation}
  Therefore,~$\downit \stylesets{S_1}\subseteq \ \downit \stylesets{S_2}$, satisfing the monotonicity property for~$\downit$.
\end{solution}


  Consider the battery example, and the antichain given by the battery models~$\posAel_1=\tup{\unit[10]{USD},\unit[1000]{g}}$,~$\posAel_2=\tup{\unit[20]{USD},\unit[500]{g}}$, and~$\posAel_3=\tup{\unit[30]{USD},\unit[250]{g}}$ (\cref{fig:examplebatt}, left).
  The lower closure operator~$\downit\{\posAel_1,\posAel_2,\posAel_3\}$ represents all the battery models which, if existing, would dominate~$\{\posAel_1,\posAel_2,\posAel_3\}$.
  One could instead consider linear maps between the points getting a poset~$\posA$, and obtain the lower closure depicted in \cref{fig:examplebatt} on the right.

\begin{figure*}[h!]
  \begin{center}
    \includesag{70_battery_1}
  \end{center}
  \caption{Example of lower closures.}
  \label{fig:examplebatt}
\end{figure*}


\begin{definition}[Min]
  \label{def:Min}
  $\Min \colon \posPA \to \posAA$ is the map that sends a subset~$S$ of a poset to the minimal elements of that subset (those elements~$a \in S$ such that~$a \posAleq b$ for all~$b \in S$).
  In formulas:
  \begin{equation*}
    \begin{aligned}
      \Min \colon \posPA &\to \posAA\\
      S&\mapsto \{ x\in S\colon (y\in S)\wedge(y\posAleq x)\Imp (x=y)\}.
    \end{aligned}
  \end{equation*}
  Note that~$\Min(S)$ could be empty.
\end{definition}

\begin{definition}[Max]
  \label{def:Max}
  $\Max \colon \posPA \to \posAA$ is the map that sends a subset~$S$ of a poset to the maximal elements of that subset, \ie , those elements~$a \in S$ such that~$a \ordgeq b$ for all~$b \in S$. In formulas:
  \begin{equation*}
    \begin{aligned}
      \Max \colon \posPA &\to \posAA\\
      S&\mapsto \{ x\in S\colon (y\in S)\wedge(y\ordgeq x)\Imp (x=y)\}.
    \end{aligned}
  \end{equation*}
  Note that~$\Max(S)$ could be empty.
\end{definition}

\begin{comment}
\todo{This is a remnant of older times. To remove. }

\begin{lemma}\label{lem:orderantichain}
  Given a poset~$\tupp{\posA,\posAleq}$,~$\tupp{\posAA,\posAAleq}$ is a poset with
  \begin{equation}
    \label{eq:orderantichain}
    A\posAAleq B \text{ if and only if } \upit  A \supseteq \ \upit  B.
  \end{equation}
  Furthermore, it is bounded by the top~$\postop_{\posAA}=\emptyset$ and the bottom~$\posbot_{\posAA}=\{\posbot_{\posA}\}$.
\end{lemma}

\begin{proof}
  We need to show the poset properties (\cref{def:poset}).
  We can prove the following:
%
  \begin{compactitem}
    \item \emph{Reflexivity}: From~$\tupp{\posA,\posAleq}$ being a poset we know that
    \begin{equation}
      \begin{aligned}
        \{y\in \posA \mid \exists x\in A \colon x\posAleq y\} &\supseteq \{y\in \posA \mid \exists x\in A \colon x\posAleq y\},\\
        \upit  A =\ \upit  A
      \end{aligned}
    \end{equation}
    and hence~$A\posAAleq A$.
%
    \item \emph{Antisymmetry}: One has
    \begin{equation}
      \begin{aligned}
        \left(A\posAAleq B\right) \wedge \left(B\posAAleq A\right)
        &\Leftrightarrow \left(\upit  A \supseteq \ \upit  \ B\right) \wedge \left( \upit   B\supseteq \ \upit  \ A\right)\\
        &\Leftrightarrow \ \upit  A= \ \upit  B \\
        & \Imp A = B.
      \end{aligned}
    \end{equation}
    The last implication is by  \cref{lem:up-cl-inj-antichains}.

    \item \emph{Transitivity}: One has
    \begin{equation}
      \begin{aligned}
        \left(A\posAAleq B\right) \wedge \left(B\posAAleq C\right)&\Leftrightarrow  \left(\upit  A \supseteq \ \upit  \ B\right) \wedge \left( \upit   B\supseteq \ \upit  C\right)\\
        &\Imp \ \upit  A\supseteq \ \upit  C\\
        &\Imp A\posAAleq C.
      \end{aligned}
    \end{equation}
    In order to find the top, we need to find the smallest set~$\postop_{\posAA}$ such that~$A\posAAleq \postop_{\posAA}$ for all~$A\in \posAA$.
    In other words, such that~$\upit  A\supseteq \ \upit  \postop_{\posAA}$ for all~$A\in \posAA$. This is clearly~$\emptyset$, since~$\upit  \emptyset = \emptyset$.
    Similarly, in order to find the bottom, we need to find the set~$\posbot_{\posAA}$ such that~$\posbot_{\posAA} \posAAleq A$ for all~$A\in \posAA$.
    In other words, such that~$\upit  \posbot_{\posAA} \supseteq \ \upit  A$ for all~$A\in \posAA$.
    We obtain a bottom if we set~$\posbot_{\posAA} \definedas \postop_{\posA}$, since~$\postop_{\posA} \supseteq A$ for all~$A \subseteq P$, and hence, by monotonicity of~$\upit $, we have in particular~$\upit  \postop_{\posA} \supseteq \upit  A$ for all antichains~$A$.
  \end{compactitem}
\end{proof}
\end{comment}

\begin{definition}[Downward closed set]
  \label{def:downward-closed-upperset}
  An upper set~$S$ is \emph{downward-closed} in a poset~$\posA$ if
  \begin{equation}
    S =\, \upit  \Min S.
  \end{equation}
\end{definition}

The set of downward-closed upper sets of~$\posA$ is denoted~$\dcuppersets \posA$.

\subsection{Measuring posets}
\begin{definition}[Width of a poset]
  \label{def:poset-width}
  The \emph{width} of a poset, denoted~$\posetwidth(\posA)$, is the maximum cardinality of an antichain in~$\posA$.
\end{definition}

\begin{definition}[Height of a poset]
  \label{def:poset-height}
  The \emph{height} of a poset, denoted~$\posetheight(\posA)$, is the maximum cardinality of a chain in~$\posA$.
\end{definition}

\begin{exercise}\label{ex:width}
  If you know the width of the posets~$\posA$ and~$\posB$, can you compute the width of ~$\posA\times\posB$?
\end{exercise}
\begin{solution}
  From~\cite{bezrukovantichains}, we know
  \begin{equation*}
    \posetwidth(\posA)\times \posetwidth(\posB)\leq \posetwidth(\posA\times \posB)\leq \min \{ \vert \posA\vert \cdot \posetwidth(\posB),\vert \posB \vert \cdot \posetwidth(\posA)\}.
  \end{equation*}
\end{solution}

\begin{exercise}\label{ex:height}
  If you know the height of the posets~$\posA$ and~$\posB$, can you compute the height of~$\posA\times\posB$?
\end{exercise}

\begin{solution}
  Yes. First of all, one can construct the longest chain in~$\posA$:
\begin{equation*}
  A=\{\posAel_1,\ldots, \posAel_{\posetheight(\posA)}\}.
\end{equation*}
  Furthermore, one can construct the longest chain in~$\posB$:
  \begin{equation*}
  B=\{\posBel_1,\ldots, \posBel_{\posetheight(\posB)}\}.
\end{equation*}
  Out of them, one can construct the chain
  \begin{equation*}
    C=\{ \tup{\posAel_1,\posBel_1},\tup{\posAel_2,\posBel_1},\ldots, \tup{\posAel_{\posetheight(\posA)}, \posBel_1}, \tup{\posAel_{\posetheight(\posA)}, \posBel_2},\ldots\},
  \end{equation*}
  which has height~$\posetheight(\posA)+\posetheight(\posB)-1$. So we know that at least~$\posetheight(\posA\times \posB)\geq \posetheight(\posA)+\posetheight(\posB)-1$.
  Now, consider a chain~$\{\tup{\posAel_1,\posBel_1},\ldots, \tup{\posAel_n,\posBel_n}\}$ in~$\posA\times \posB$.
  In general, this means that at least a coordinate of~$\tup{\posAel_{i},\posBel_{i}}$ must increase in$\tup{\posAel_{i+1},\posBel_{i+1}}$.
  The first coordinate can only increase~$\posetheight(\posA)-1$ times, and the second one~$\posetheight(\posB)-1$ times. Summing up, the total number of elements in the chain is \emph{at most}~$\posetheight(\posA)+\posetheight(\posB)-1$.
  Note that this result holds only assuming that~$\posA$ and~$\posB$ are not empty (for that case,~$\posetheight(\posA\times \posB)=0$).
\end{solution}

\todographicsjira{73}{pictures and examples for width/height of posets}