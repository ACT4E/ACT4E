% !TEX root = chapter-standalone.tex

\section[Antichains]{Upper and lower closure}

\linkvideo{spring2021-design:up-low-closure} % Upper and lower closure
\begin{definition}[Upper closure operator]
    \label{def:upperclosure}
    The \emph{upper closure operator}~$\upit $ maps a subset to the smallest upper set that includes it:
    \begin{equation}
        \begin{aligned}
            \upit  \colon \posPA & \to \setOfUppersets\posA,                                                                                                                                \\
            S                    & \mapsto \{\styleelements{y}\in \posA \mid \exists \styleelements{x}\in \stylesets{S} \colon \styleelements{x}\posAleq \styleelements{y}\}.
        \end{aligned}
    \end{equation}
\end{definition}
\begin{remark}
    Note that, by definition, an upper set is closed to upper closure.
\end{remark}
\begin{lemma}
    For any~$\stylesets{S}\in \posPA$,~$\upit  \stylesets{S}$ is in fact an upper set.
\end{lemma}
\begin{proof}
    Suppose~$\styleelements{y}\in \upit  \stylesets{S}$ and~$\styleelements{z}\in \posA$, and suppose $\styleelements{y}\posAleq \styleelements{z}$.
    By definition there exists a~$\styleelements{x}$ s.t.~$\styleelements{x}\posAleq \styleelements{y}$, meaning that~$\styleelements{x}\posAleq \styleelements{z}$.
    Thus,~$\styleelements{z}\in \upit  \stylesets{S}$, as was to be shown.
\end{proof}


In the example of battery choices, \todotext{Which example?} first, consider the upper closure of a single element of the poset, for instance~$\posAel_1=\tupp{\unit[10]{USD},\unit[500]{g}}$ (\cref{fig:upperclosure_1}, left).
Second, we can look at the upper closure when we add the choice~$\posAel_2=\tupp{\unit[20]{USD},\unit[250]{g}}$ (\cref{fig:upperclosure_1}, center).

% \begin{figure*}
%     \begin{center}
%         \includesag{70_upper_closure_2}
%         \hspace{0.1cm}
%         \includesag{70_upper_closure_2_bis}
%     \end{center}
%     \caption{The upper closure of a set of pizza recipes.}
%     \todotext{@Gioele: still pizza?}
%     \label{fig:upperclosure_2}
% \end{figure*}

Note that the upper set of the subset formed by the two elements is the union of the upper sets of the single elements.
%
Finally, we can also define the set
%
\begin{equation}
    \setA=\{
    \tupp{\text{cost},\text{mass}}\mid \text{mass}=750-25\cdot \text{cost},
    \forall \text{cost} \in [0,20]
    \},
\end{equation}
%
and find its upper closure (\cref{fig:upperclosure_1}, right).
%
\begin{figure*}[h!]
    \centering
    \includesag{70_uc_1_1} \hspace{0.1cm}
    \includesag{70_uc_1_2} \hspace{0.1cm}
    \includesag{70_uc_1_3}
    \caption{Example of uppler closure for different sets of battery choices. }
    \label{fig:upperclosure_1}
    \todographics{In the panel right, highlight the set $\setA$ (actually use $\subA$ as symbol)}
\end{figure*}

\begin{definition}[Lower closure operator]
    \label{def:lowerclosure}
    The \emph{lower closure operator}~$\downit$ maps a subset to the smallest lower set that includes it:
    \begin{equation*}
        \begin{aligned}
            \downit \colon \posPA & \to \setOfLowersets   \posA                                                                                                                               \\
            \stylesets{S}         & \mapsto \{ \styleelements{y}\in \posA \mid \exists \styleelements{x}\in \stylesets{S} \colon \styleelements{y}\posAleq \styleelements{x}\}.
        \end{aligned}
    \end{equation*}
\end{definition}

Consider the battery example, and the antichain given by the battery models~$\posAel_1=\tupp{\unit[10]{USD},\unit[1000]{g}}$, $\posAel_2=\tupp{\unit[20]{USD},\unit[500]{g}}$, and $\posAel_3=\tupp{\unit[30]{USD},\unit[250]{g}}$ (\cref{fig:examplebatt}, left).
The lower closure operator~$\downit\makeset{\posAel_1,\posAel_2,\posAel_3}$ represents all the battery models which, if existing, would dominate~$\makeset{\posAel_1,\posAel_2,\posAel_3}$.
One could instead consider linear maps between the points getting a poset~$\posA$, and obtain the lower closure depicted in \cref{fig:examplebatt} on the right.

\begin{figure*}[h!]
    \centering 
    \includesag{70_battery_1}
    \hspace{0.1cm}
    \includesag{70_battery_1_bis}
    \todographics{@Gioele: Make figure with 3 panels - same a upper closure}
    \todographics{@Gioele: In the panel right, highlight the set $\setA$ (actually use $\subA$ as symbol)}
    \caption{Example of lower closures.}
    
    \label{fig:examplebatt}
\end{figure*}

\subsection{Antichains and upper sets}

\begin{lemma}
    \label{lem:up-cl-inj-antichains}
    Let~$\setA$ and~$\setB$ be subsets of~$\posA$ that are antichains.
    Then
    \begin{equation*}
        \prfperiod{\upit  \setA = \ \upit  \setB}{\setA = \setB}
        %\upit  \setA = \ \upit  \setB \quad \Imp \quad \setA = \setB.
    \end{equation*}
\end{lemma}

\begin{proof}
    First, let's fix an~$\setAel \in \setA$.
    From~$\upit  \setA = \ \upit  \setB$ we know that in particular $\setA \subseteq \ \upit  \setB$.
    This means that for our fixed~$\setAel \in \setA$ there exists~$\setBel \in \setB$ such that~$\setBel \posleq \setAel$.
    From~$\upit \setA = \ \upit  \setB$ it also follows that~$\setB \subseteq \ \upit  \setA$, so to the~$\setBel \in \setB$ given above, there exists~$\setAel' \in \setA$ such that~$\setAel' \posleq \setBel$.
    In total, we have~$\setAel' \posleq \setBel \posleq \setAel$, and since $\setA$ is an antichain, we must have~$\setAel' = \setAel$.
    This implies that~$\setAel' = \setBel = \setAel$.
    In particular, we have~$\setAel \in \setB$.

    The above shows that~$\setA \subseteq \setB$.
    To show~$\setB \subseteq \setA$, we can fix any~$\setBel \in \setB$ and repeat the above argumentation, now with the roles of~$\setA$ and~$\setB$ exchanged.
\end{proof}

\begin{definition}[Downward closed set]
    \label{def:downward-closed-upperset}
    An upper set~$\setA$ is \emph{downward-closed} in a poset~$\posA$ if
    \begin{equation}
        \setA =\, \upit  \Min \setA.
    \end{equation}
    The set of downward-closed upper sets of~$\posA$ is denoted~$\setOfDCUppersets \posA$.

\end{definition}

\begin{definition}[Upward closed set]
    \label{def:upward-closed-lowerset}
    A lower set~$\setA$ is \emph{upward-closed} in a poset~$\posA$ if
    \begin{equation}
        \setA =\, \downit  \Max \setA.
    \end{equation}
    The set of upward-closed lower sets of~$\posA$ is denoted~$\setOfUCLowersets \posA$.

\end{definition}
