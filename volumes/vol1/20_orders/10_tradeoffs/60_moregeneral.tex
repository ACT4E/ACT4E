\section{More general preferences}

\linkvideo{spring2021-tradeoffs:tradeoffs:orders:general-pref} % More general preference structures
\begin{remark}
    There are more general ways to describe preferences.
    One has:
    \begin{itemize}
        \item \emph{Quasitransitive relations}, which are relations~$\posAleq$ over a set~$\posAset$ for which
              \begin{equation}
                  \prfperiod{(\posAel \posAleq \posBel)}{\neg(\posBel \posAleq \posAel)}{(\posBel \posAleq \posCel)}{\neg(\posCel \posAleq \posBel)}{(\posAel \posAleq \posCel) \wedge \neg(\posCel \posAleq \posAel)}
              \end{equation}
        \item \emph{Semiorders}, which are relations~$\posAleq$ over a set~$\posAset$ such that:
              \begin{itemize}
                  \item \emph{Asymmetry} (not antisymmetry!
                        ) holds:
                        \begin{equation}
                            \prfperiod{\posAel \posAleq \posBel}{\neg(\posAel \posAleq \posAel)}
                        \end{equation}
                  \item Let's denote two elements~$\posBel, \posCel \in \posA$ which are \emph{uncomparable} by~$\posBel \sim \posCel$.
                        One has:
                        \begin{equation}
                            \prfperiod{\posAel \posAleq \posBel}{\posBel \sim \posCel}{\posCel \posAleq \posDel}{\posAel \posAleq \posDel}
                        \end{equation}
                  \item One has:
                        \begin{equation}
                            \prfperiod{\posAel \posAleq \posBel}{\posBel \posAleq \posCel}{(\posAel \posAleq \posDel) \vee (\posDel \posAleq \posCel)}
                        \end{equation}
              \end{itemize}
              An example of semiorder is the following.
              Say that you have to express your preference over numbers in~$\posA=\{10,11,12\}$.
              You are indifferent between 10 and 11, and between 11 and 12.
              Though, you prefer 10 to 12.
    \end{itemize}
\end{remark}

\begin{publictodo}
    Missing part about properties and examples.
\end{publictodo}

\todotextjira{471}{@Andrea: properties/examples of semi-orders, quasitransitive relations}
