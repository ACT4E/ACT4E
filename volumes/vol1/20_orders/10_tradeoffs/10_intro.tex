% !TEX root = chapter-standalone.tex
\section{Trade-offs}

\todo{Add the examples from session 6:}
Trade-offs characterize all engineering disciplines, and can be literally found everywhere. A typical trade-off is the one reported in \cref{fig:fast_good_cheap}. When designing a product, you want it to be \emph{good}, \emph{fast}, and \emph{cheap}, and typically you can just choose between two of these qualities.

\begin{figure}[h!]
  \begin{center}
    \includesag{070_fast_good_cheap}
  \end{center}
  \caption{}
  \label{fig:fast_good_cheap}
\end{figure}

\todo{finish example}

\section{The 3 diagrams}

\todo{Functionality, requirements, 3 types of achievable accuracy plots}

\devel{\includepdf[scale=0.8,pages={10},nup=1x3,frame,pagecommand={}]{ACT4E-06-posets.pdf}} % diagram shapes}

In this sections, we introduce concepts which will be important throughout the book, when talking about theories of design.
We distinguish sematically between \textF{functionalities} and \textR{requirements/costs}.
In general, you prefer \textF{functionalities} to be ``large'' (\cref{fig:fun_large}) and \textR{requirements/costs} to be ``small'' (\cref{fig:res_small}).

\begin{marginfigure}
  \begin{center}
  \includesag{res_small}
\end{center}
   \caption{\label{fig:res_small}}

\end{marginfigure}

\begin{marginfigure}
  \begin{center}
  \includesag{fun_large}
  \end{center}
  \caption{\label{fig:fun_large}}
\end{marginfigure}

We think of three achievable accuracy plots (\cref{fig:accu_res_fun}).

First, we can plot trade-offs in costs and add a ``feasibility'' curve.
Everything above this curve is feasible and will cost more than what is \emph{on} the curve.
Second, we can plot trade-offs in functionalities and add a ``feasibility'' curve.
Everything below the curve is feasible, but is below the ``standards'' required by the curve.
Finally, we can plot functionality and resource together, representing the trade-offs between ``how good a product is" and ``how much one needs to pay for it.
Feasibly pairs are represented via the feasibility curve. Everything above the curve will be feasible (by paying more).


\begin{figure}[h]
    \begin{center}
  \begin{tabular}{ccc}
    \includesag{accu_1}&
    \includesag{accu_2}&
    \includesag{accu_3}
\end{tabular}
\end{center}
    \caption{\label{fig:accu_res_fun}}
  \end{figure}


\subsection{example: Trade-offs for the human body}

\devel{\includepdf[scale=0.8,pages={11,12},nup=1x3,frame,pagecommand={}]{ACT4E-06-posets.pdf}} % human body}

\subsection{example: Masks}

\devel{\includepdf[scale=0.8,pages={13-16},nup=1x3,frame,pagecommand={}]{ACT4E-06-posets.pdf} }% masks}


\subsection{example: Hats and headphones}

\devel{\includepdf[scale=0.8,pages={17-21},nup=1x3,frame,pagecommand={}]{ACT4E-06-posets.pdf}} % producs
