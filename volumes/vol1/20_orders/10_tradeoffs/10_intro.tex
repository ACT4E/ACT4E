% !TEX root = chapter-standalone.tex
\section{Trade-offs}

\todo{Add the examples from session 6:}
Trade-offs characterize all engineering disciplines, and can be literally found everywhere. A typical trade-off is the one reported in \cref{fig:fast_good_cheap}. When designing a product, you want it to be \emph{good}, \emph{fast}, and \emph{cheap}, and typically you can just choose between two of these qualities.

\begin{figure}[h!]
  \begin{center}
    \includesag{070_fast_good_cheap}
  \end{center}
  \caption{}
  \label{fig:fast_good_cheap}
\end{figure}

\todo{finish example}

\section{The 3 diagrams}

\todo{Functionality, requirements, 3 types of achievable accuracy plots}

\includepdf[scale=0.8,pages={10},nup=1x3,frame,pagecommand={}]{ACT4E-06-posets.pdf} % diagram shapes


\subsection{example: Trade-offs for the human body}

\includepdf[scale=0.8,pages={11,12},nup=1x3,frame,pagecommand={}]{ACT4E-06-posets.pdf} % human body

\subsection{example: Masks}

\includepdf[scale=0.8,pages={13-16},nup=1x3,frame,pagecommand={}]{ACT4E-06-posets.pdf} % masks


\subsection{example: Hats and headphones}

\includepdf[scale=0.8,pages={17-21},nup=1x3,frame,pagecommand={}]{ACT4E-06-posets.pdf} % producs
