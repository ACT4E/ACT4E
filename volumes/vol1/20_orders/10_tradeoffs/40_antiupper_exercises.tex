\section{Code exercise}
\subsection{Properties of subsets}


\begin{gradedexercise}
  Check if a subset is a chain.
  \methodsource{FinitePosetSubsetProperties}{is_chain}{}
\end{gradedexercise}

\begin{gradedexercise}
  Check if a subset is an antichain.
  \methodsource{FinitePosetSubsetProperties}{is_antichain}{}
\end{gradedexercise}

\subsection{Operations on subsets of posets}

\begin{gradedexercise}
  Compute the upper closure of a subset.
  \methodsource{FinitePosetSubsetOperations}{upperclosure}{}

\end{gradedexercise}


\begin{gradedexercise}
  Compute the lower closure of a subset, reducing to
  using the function \funcname{upperclosure} that you
  already defined.
  \methodsource{FinitePosetSubsetOperations}{lowerclosure}{}

\end{gradedexercise}



\begin{gradedexercise}
  Compute the minimal elements of a subset.

  \methodsource{FinitePosetSubsetOperations}{minimal}{}

\end{gradedexercise}


\begin{gradedexercise}
  Compute the maximal elements of a subset, reducing
  to use the function \funcname{minimal} already defined.

  \methodsource{FinitePosetSubsetOperations}{maximal}{}

\end{gradedexercise}


\begin{gradedexercise}
  Compute the infimum/supremum/meet/join of a subset.
  In case one does not exist, return None.

  \methodsource{FinitePosetSubsetOperations}{infimum}{}
  \methodsource{FinitePosetSubsetOperations}{supremum}{}
  \methodsource{FinitePosetSubsetOperations}{meet}{}
  \methodsource{FinitePosetSubsetOperations}{join}{}

\end{gradedexercise}

\subsection{Lower/upper sets}
\begin{gradedexercise}
  Given a set, return the power set, a poset ordered by inclusion.

  \methodsource{FinitePosetConstructors}{powerset}{}
\end{gradedexercise}



\begin{gradedexercise}
  Given a poset, compute the interval poset.

  \methodsource{FinitePosetConstructors}{intervals}{}
\end{gradedexercise}


\begin{gradedexercise}
  Given a poset, compute the other interval poset.

  \methodsource{FinitePosetConstructors}{intervals2}{}
\end{gradedexercise}


\begin{gradedexercise}
  Given a poset, define the poset of upper sets.

  \methodsource{FinitePosetConstructors}{uppersets}{}
\end{gradedexercise}

\begin{gradedexercise}
  Given a poset, define the poset of lower sets.
  \methodsource{FinitePosetConstructors}{lowersets}{}
\end{gradedexercise}


\begin{gradedexercise}
  Given a set, construct the discrete poset.

  \methodsource{FinitePosetConstructors}{discrete}{}
\end{gradedexercise}


\begin{gradedexercise}
  Given two posets, return their product poset.
  \methodsource{FinitePosetOperations}{product}{}
\end{gradedexercise}
\begin{gradedexercise}
  Given two posets, return their disjoint union.

  \methodsource{FinitePosetOperations}{disjoint_union}{}
\end{gradedexercise}
