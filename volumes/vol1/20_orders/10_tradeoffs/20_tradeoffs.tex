% !TEX root = chapter-standalone.tex

\section{Ordered sets}

\label{sec:tradeoffs-ordered-sets}
\linkvideo{spring2021-tradeoffs:tradeoffs:orders:pre-pos-tot} % Pre-orders, partial orders, and total orders

%\linkvideo{spring2021-tradeoffs:tradeoffs:orders} % Orders

\showslides{
    \begin{forslides}
        \begin{equation*}
            \label{eq:tradeoff_map}
            \mapa \colon \stylesets{X} \sto \stylesets{Y}
        \end{equation*}
        \begin{equation*}
            \label{eq:tradeoff_mor}
            \mora \colon \Obja \mto \Objb
        \end{equation*}
        \begin{equation*}
            \label{eq:tradeoff_typw}
            \mora \colon (\Obja \mto \Objb)
        \end{equation*}
        \begin{equation*}
            \label{eq:tradeoff_cap}
            \begin{aligned}
                \setA & \cap \setB      \\
                \setA & \cup \setB      \\
                \setA & \subseteq \setB
            \end{aligned}
        \end{equation*}
        \begin{equation*}
            \label{eq:tradeoff_wedge}
            \begin{aligned}
                a & \wedge b \\
                a & \vee b   \\
            \end{aligned}
        \end{equation*}
        \begin{equation*}
            \label{eq:tradeoff_leq}
            \setA \leq \setB
        \end{equation*}
        \begin{equation*}
            \label{eq:abc}
            \prflineextra=0.6em
            \prftree
            {\prftree
                {a}
                {b}}
            {c}
        \end{equation*}
        \begin{equation*}
            \label{eq:abc_2}
            \begin{aligned}
                           & a \\
                \Imp \quad & b \\
                \Imp \quad & c
            \end{aligned}
        \end{equation*}
        \begin{equation*}
            \label{eq:abc_3}
            \prftree{a}{b}{c}
        \end{equation*}
        \begin{equation*}
            \label{eq:abc_4}
            (a\wedge b)\Imp c
        \end{equation*}
        \begin{equation*}
            \label{eq:abc_5}
            \prflineextra=0.6em
            \prftree[d]
            {\prftree[d]
                {a}
                {b}}
            {c}
        \end{equation*}
        \begin{equation*}
            \label{eq:abc_6}
            a\Leftrightarrow b \Leftrightarrow c
        \end{equation*}
        \begin{equation*}
            \label{eq:abc_7}
            \prflineextra=0.6em
            \prftree{a}{b}
        \end{equation*}
        \begin{equation*}
            \label{eq:abc_8}
            \prflineextra=0.6em
            \prftree{b}{c}
        \end{equation*}
        \begin{equation*}
            \label{eq:abc_9}
            \prflineextra=0.6em
            \prftree{a}{c}
        \end{equation*}
        \begin{equation*}
            \label{eq:mapf}
            \mapa \colon
        \end{equation*}
        \begin{equation*}
            \label{eq:mapg}
            \mapb \colon
        \end{equation*}
        \begin{equation*}
            \label{eq:mapcomp}
            \mapa \then \mapb \colon
        \end{equation*}
        \begin{equation*}
            \label{eq:top1}
            \prflineextra=0.6em
            \prftree{\true}{a}
        \end{equation*}
        \begin{equation*}
            \label{eq:bot1}
            \prflineextra=0.6em
            \prftree{a}{\false}
        \end{equation*}
        \begin{equation*}
            \label{eq:rel_refl}
            \prflineextra=0.6em
            \prftree{\true}{a\relA a}
        \end{equation*}
        \begin{equation*}
            \label{eq:rel_tot}
            \prflineextra=0.6em
            \prftree{\true}{a\relA b \quad \vee \quad b\relA a}
        \end{equation*}
        \begin{equation*}
            \label{eq:rel_tran}
            \prflineextra=0.6em
            \prftree{a\relA b}{b\relA c}{a\relA c}
        \end{equation*}
        \begin{equation*}
            \label{eq:rel_sym}
            \prflineextra=0.6em
            \prftree[d]{a \relA b}{b\relA a}
        \end{equation*}
        \begin{equation*}
            \label{eq:rel_irr}
            \prflineextra=0.6em
            \prftree{a \relA a}{\false}
        \end{equation*}
        \begin{equation*}
            \label{eq:rel_asy}
            \prflineextra=0.6em
            \prftree{a \relA b}{b\relA a}{\false}
        \end{equation*}
        \begin{equation*}
            \label{eq:rel_antisy}
            \prflineextra=0.6em
            \prftree{a \relA b}{b\relA a}{a=b}
        \end{equation*}
        \begin{equation*}
            \label{eq:ex}
            a\relA b \Leftrightarrow (a-b) \text{ mod }3=0
        \end{equation*}
        \begin{equation*}
            \label{eq:fun_cost}
            (\F{\text{keeps warm }}\cartprod \F{\text{ music fidelity}})\cartprod(\R{\text{ price }} \cartprod \R{\text{ frequency of charging }}\cartprod \R{\text{ wires hassle}})
        \end{equation*}
        \begin{equation*}
            \label{eq:id_ex}
            \prftree{\true}{\catid_\Obja\colon \Obja \mto \Obja}
        \end{equation*}
        \begin{equation*}
            \label{eq:comp_cat}
            \prftree{\mora \colon \Obja \mto \Objb}{\morb\colon \Objb \mto \Objc}{\mora\mthen \morb \colon \Obja \mto \Objc}
        \end{equation*}
        \begin{equation*}
            \label{eq:mor_pre}
            \prftree[d]{\ast\colon \posAel \mto \posBel}{\posAel \posleq \posBel}
        \end{equation*}
        \begin{equation*}
            \label{eq:hom_pre}
            \HomSet{}{\posAel}{\posBel}=\begin{cases}
                \{\ast\}  & \text{iff } \posAel \posleq \posBel \\
                \emptyset & \text{otherwise}
            \end{cases}
        \end{equation*}
        \begin{equation*}
            \label{eq:comp_pre}
            \ast \mthen \ast = \ast
        \end{equation*}
        \begin{equation*}
            \label{eq:ex_pre}
            \begin{aligned}
                 & \posAel,\posBel,\posCel \\
                 & \posAel \posleq \posBel \\
                 & \posCel \posleq \posBel
            \end{aligned}
        \end{equation*}
        \begin{equation*}
            \label{eq:pre_p}
            \posAel
        \end{equation*}
        \begin{equation*}
            \label{eq:pre_q}
            \posBel
        \end{equation*}
        \begin{equation*}
            \label{eq:pre_r}
            \posCel
        \end{equation*}
        \begin{equation*}
            \label{eq:eq_rel}
            \prftree[d]{\posAel\sim \posBel}{(\posAel \posleq \posBel)\wedge (\posBel \posleq \posAel)}
        \end{equation*}
        \begin{equation*}
            \label{eq:pos_int}
            \prflinepadbefore=0.6em
            \prflinepadafter=0.6em
            \prftree[d]{\tup{l_1,u_1}\posleq_{\styleelements{\text{Int}}\posA} \tup{l_2,u_2}}{(l_1\posAleq l_2)\wedge (u_2\posAleq u_1)}
        \end{equation*}
        \begin{equation*}
            \label{eq:leq}
            \posleq
        \end{equation*}
        \begin{equation*}
            \label{eq:pos_prod}
            \prflinepadbefore=0.6em
            \prflinepadafter=0.6em
            \prftree[d]{\tup{\posAel_1,\posBel_1}\posleq_{\posA\cartprod \posB} \tup{\posAel_2,\posBel_2}}{(\posAel_1\posAleq \posAel_2)\wedge(\posBel_1\posBleq \posBel_2)}
        \end{equation*}
        \begin{equation*}
            \label{eq:pos_P}
            \posA
        \end{equation*}
        \begin{equation*}
            \label{eq:pos_Q}
            \posB
        \end{equation*}
        \begin{equation*}
            \label{eq:pos_PQ}
            \posA\cartprod \posB
        \end{equation*}
        \begin{equation*}
            \label{eq:set_disj}
            \posA+\posB=\{ \tup{1,\posAel} \mid \posAel\in \posA \}\cup \{\tup{2,\posBel}\mid \posBel \in \posB\}
        \end{equation*}
        \begin{equation*}
            \label{eq:latt_1}
            \posAel \wedge \posBel
        \end{equation*}
        \begin{equation*}
            \label{eq:latt_2}
            \posAel \vee \posBel
        \end{equation*}
        \begin{equation*}
            \label{eq:latt_3}
            (\posAel \wedge \posBel)\posleq \posAel \posleq (\posAel \vee \posBel)
        \end{equation*}
        \begin{equation*}
            \label{eq:latt_4}
            \posAel
        \end{equation*}
        \begin{equation*}
            \label{eq:latt_5}
            \posBel
        \end{equation*}
        \begin{equation*}
            \label{eq:prop_1}
            \prftree[d]{\posAel \posleq \posBel}{\posAel \Imp \posBel}
        \end{equation*}
        \begin{equation*}
            \label{eq:prop_2}
            \prftree[d]{\posAel \Leftrightarrow \posBel}{\posAel = \posBel}
        \end{equation*}
        \begin{equation*}
            \label{eq:prop_3}
            \prftree{\true}{\posAel \Imp \posAel}
        \end{equation*}
        \begin{equation*}
            \label{eq:prop_4}
            \prftree{(\posAel \Imp \posBel)\wedge (\posBel \Imp \posCel)}{\posAel \Imp \posCel}
        \end{equation*}
        \includesag{060_propositions_lat}
        \begin{equation*}
            \label{eq:pow_1}
            \setA \posleq \setB \definedas \setA \subseteq \setB
        \end{equation*}
        \begin{equation*}
            \label{eq:pow_2}
            \setA \vee \setB \definedas \setA \cup \setB
        \end{equation*}
        \begin{equation*}
            \label{eq:pow_3}
            \setA \wedge \setB \definedas \setA \cap \setB
        \end{equation*}
        \begin{equation*}
            \label{eq:pow_4}
            \postop \definedas \stylesets{S}
        \end{equation*}
        \begin{equation*}
            \label{eq:pow_5}
            \posbot \definedas \stylesets{\emptyset}
        \end{equation*}
        \begin{equation*}
            \label{eq:pow_6}
            \powerset{\stylesets{S}}
        \end{equation*}
        \begin{equation*}
            \label{eq:pow_7}
            \stylesets{S}
        \end{equation*}
        \begin{equation*}
            \label{eq:pre_ex1}
            \{\tup{a,a},\tup{a,b},\tup{b,a},\tup{b,b}\}
        \end{equation*}
        \begin{equation*}
            \label{eq:pre_ex2}
            \{\tup{a,b},\tup{b,c},\tup{c,a}\}
        \end{equation*}
        \begin{equation*}
            \label{eq:pre_ex3}
            \{\tup{a,a},\tup{b,b},\tup{c,c},\tup{d,d}\}
        \end{equation*}
        \begin{definition}
            \label{def:posdef}

            A symmetric matrix~$\mat{M}\in \reals^{n\times n}$ is \emph{positive semi-definite} if~$x^\intercal \mat{M}x\geq 0$ for all non-zero~$x\in \reals^n$.
            We call the set of all such matrices~$\mathcal{P}^n$.
        \end{definition}
        \begin{equation*}
            \label{eq:posdef_1}
            \mat{A}\posleq \mat{B} \Leftrightarrow (\mat{A}-\mat{B})\in \mathcal{P}^n, \quad \mat{A},\mat{B}\in \mathcal{P}^n
        \end{equation*}
    \end{forslides}
}

So far, the discussion has been purely qualitative.
While we discussed how categories can describe the way in which one resource can be turned into another, this kind of modelling did not allow for quantitative statements.
For example, it is good to know that we can obtain motion from electric power, but, how fast can we go with a certain amount of power?

To achieve a quantitative theory, we need to specify various degrees of resources and functionality.
One way of doing this, is through the idea of orders.

Such orderings arise naturally in engineering as criteria for judging whether one design is better or worse than another.
As an example, suppose you need to buy a battery.
In this simple example, you can think of having two resources: mass and money.
A lighter battery might be more expensive, and a heavier one could be more affordable.
How to choose among batteries, if you do not prefer one resource over the other?
How to model this?
In this section, we will assume that functionality and resources are \emph{ordered sets}, and will introduce pre-orders, partial orders, and total orders.

Davey and Priestley~\cite{davey02} and Roman~\cite{roman08} are possible reference texts.

We introduce these concepts by adding levels of specificity.

\begin{definition}[Pre-ordered set]
    \label{def:preorder}
    A \emph{pre-ordered} set is a tuple~$\posA = \tup{\posAset,\posAleq}$, where~$\posAset$ is a set (also called the \emph{carrier set} or \emph{underlying set}), together with a relation~$\posAleq$ that is:
    \begin{itemize}
        \item \emph{Reflexive:}
              \begin{equation}
                  \prfperiod{\true}{\posAel \posAleq \posAel}
              \end{equation}
        \item \emph{Transitive:}
              \begin{equation}
                  \prfperiod{\posAel \posAleq \posBel}{\posBel \posAleq \posCel}{\posAel \posAleq \posCel}
              \end{equation}
    \end{itemize}
\end{definition}
By adding \emph{antisymmetry} (\cref{def:antisymmetry}), one obtains a partially ordered set.
\begin{ctdefinition}[Partially ordered set]
    \label{def:poset}
    A pre-ordered set~$\posA = \tupp{\posAset,\posAleq}$ is a \emph{partially-ordered set (\iindex{poset})} if the relation~$\posAleq$ is \emph{antisymmetric}.
    In other words, if:
    \begin{equation}
        \prfperiod{\posAel \posAleq \posBel}{\posBel \posAleq \posAel}{\posAel = \posBel}
    \end{equation}
\end{ctdefinition}

By obtaining \emph{totality} (\cref{def:endo_total}), one obtains a total order.

\begin{ctdefinition}[Totally ordered set]
    \label{def:total_order}
    A partially ordered set~$\posA = \tupp{\posAset,\posAleq}$ is a \emph{totally ordered set} if the relation~$\posAleq$ is \emph{total}.
    In other words, if:
    \begin{equation}
        \prfperiod{\true}{(\posAel \posAleq \posBel) \vee (\posBel \posAleq \posAel)}
    \end{equation}
\end{ctdefinition}


\linkvideo{spring2021-tradeoffs:tradeoffs:orders:hasse} % Hasse diagrams

A \emph{\iindex{Hasse diagram}} is an economical (in terms of arrows) way to visualize a poset.
In a Hasse diagram elements are points, and if~$\posAel \posAleq \posBel$ then~$\posAel$ is drawn lower than~$\posBel$ and with an ege connected to it, if no other point is in between.
Hasse diagrams are directed graphs.

\begin{marginfigure}
    \centering
    \includesag{70_hasse_pizza}
    \caption{The cost of a battery can be represented as a poset.}
    \label{fig:hassebattery}
\end{marginfigure}

In the example of the battery choice, both mass and money can be thought of as partially ordered sets~$\tupp{\nonNegReals,\Rleq}$ (actually, both are totally ordered sets).
Imagine that you have batteries costing \unit[10]{USD}, \unit[20]{USD}, and \unit[30]{USD}.
This can be represented as in~\cref{fig:hassebattery}.

\begin{marginfigure}
    \centering
    \includesag{40_dpcatfig_hasse}
    \caption{Example of Hasse diagram of~$\posA$.}
    \label{fig:hasse}
\end{marginfigure}

\begin{example}
    Consider a poset~$\posA$ representing your food preference over the set~$\posAset=\{\sbretzel,\sfondue,\schoco,\sburger,\sapple\}$ with~$\sbretzel\posAleq \sfondue$,~$\sbretzel\posAleq \schoco$,~$\sburger\posAleq \schoco$, and~$\sburger\posAleq \sapple$.
    This can be represented with a Hasse diagram as in~\cref{fig:hasse}.
\end{example}

\begin{marginfigure}
    \centering
    \includesag{40_dpcatfig_boolean}
    \caption{}
    \label{fig:boolean}
\end{marginfigure}

\begin{margintable}
    \centering
    \begin{tabular}{cc|ccc}
        $a$      & $b$      & $a \posleq  b$ & $a \wedge b$ & $a \vee b$ \\ \hline
        $\true$  & $\true$  & $\true$        & $\true$      & $\true$    \\
        $\true$  & $\false$ & $\false$       & $\false$     & $\true$    \\
        $\false$ & $\true$  & $\true$        & $\false$     & $\true$    \\
        $\false$ & $\false$ & $\true$        & $\false$     & $\false$
    \end{tabular}
    \caption{Properties of the \Bool poset.
        Note that $\posleq \equiv \Imp$.
    }
    \label{tab:boolposet}
\end{margintable}

\begin{example}[Booleans]
    \label{ex:bool}
    The booleans \index{\Bool} is a poset with carrier set~$\{\true,\false\}$ and the order relation given by~$b_1 \posleq_\Bool b_2$ iff~$b_1 \Imp b_2$, that is,~$\false \posleq_\Bool \true$ (\cref{fig:boolean}).

    This relation should be familiar from~\cref{tab:boolposet}.

    In addition to the operation
    \begin{equation*}
        \Imp\colon\Bool\cartprod \Bool\to\Bool,
    \end{equation*}
    called \emph{Imp}, there are also the familiar \emph{and} ($\wedge$) and \emph{or} ($\vee$) operations.
    Note that~$\wedge$ and~$\vee$ are commutative ($b\wedge c = c\wedge b$,~$b\vee c = c\vee b$ ), whereas~$\Imp$ is not.
\end{example}

\begin{example}[Reals]
    The real numbers \reals form a poset with carrier \reals and order relation given by the usual ordering~$r_1 \Rleq r_2$.
\end{example}

\begin{marginfigure}
    \centering
    \includesag{40_discrete}
    \caption{Example of a discrete poset.}
    \label{fig:discretepos}
\end{marginfigure}

\begin{example}[Discrete partially ordered sets]
    \label{ex:discreteposet}
    Every set~$\posAset$ can be considered as a \emph{discrete poset}~$\posA = \tup{\posAset,=}$.
    Discrete posets are represented as collection of points (\cref{fig:discretepos}).
\end{example}

\begin{marginfigure}
    \centering
    \includesag{40_dpcatfig_power}
    \caption{Power set as a category.}
    \label{fig:powersetcat}
\end{marginfigure}
\newcommand{\fitinmargin}[1]{%
    \maxsizebox{\marginparwidth}{!}{#1}%
}

\newcommand{\fitinline}[1]{%
    \maxsizebox{\textwidth}{!}{#1}%
}

\begin{marginfigure}
    \centering
    \subfloat[
        Example of ellipses representing positive semi-definite matrices.
        \label{fig:posdef_draw}
    ]{
        \centering
        \fitinmargin{%
        \includesag{20_ellipses_mat}
        }
    }

    \subfloat[
        Example of order between positive semi-definite matrices.
        \label{fig:posdef_hasse}
    ] {
        \centering
        \includesag{20_mat_order}
    }
    \caption{}
    \todographics{@Gioele: label the ellipses with $\langle x, Ax\rangle=1$.}
\end{marginfigure}

\begin{example}
    \begin{definition}[Positive semi-definite matrix]
        A symmetric matrix~$\mat{M}\in \reals^{n\times n}$ is \emph{positive semi-definite} if~$x^\intercal \mat{M}x\geq 0$ for all non-zero~$x\in \reals^n$.
        We call the set of all such matrices~$\mathcal{P}^n$.
    \end{definition}
    Positive semi-deminite matrices have real, semi-positive eigenvalues, which can be interpreted as axes lenghts of ellipsoids.
    Any matrix~$\mat{A}\in \mathcal{P}^n$ describes an ellipsoid, descriptive equation of which can be written as a quadratic form:
    \begin{equation*}
        x^\intercal \mat{A}x=1,\quad x\in \reals^n.
    \end{equation*}
    We can define a partial order on~$\mathcal{P}^n$ as
    \begin{equation*}
        \mat{A}\posleq \mat{B} \Leftrightarrow (\mat{B}-\mat{A})\in \mathcal{P}^n, \quad \mat{A},\mat{B}\in \mathcal{P}^n.
    \end{equation*}
    The order can be interpreted as an inclusion of ellipsoids.
    Take for instance the matrices
    \begin{equation*}
        \mat{A}=\begin{pmatrix}
            1 & 0 \\0& 1
        \end{pmatrix}, \quad \mat{B}=\begin{pmatrix}
            2 & 0 \\0& 1
        \end{pmatrix},\quad \mat{C}=\begin{pmatrix}
            2 & 0 \\0& 0.5
        \end{pmatrix}.
    \end{equation*}
    The order on the set~$\{\mat{A},\mat{B},\mat{C}\}$ is reported in \cref{fig:posdef_hasse}, and it is easily explained via \cref{fig:posdef_draw}.
    The ellipse representing~$\mat{A}$ (in red) is included by the one representing matrix~$\mat{B}$ (in blue), but not by the one representing matrix~$\mat{C}$ (in green).
    Furthermore, the one representing~$\mat{B}$ includes the one representing~$\mat{C}$.
\end{example}

\begin{example}
    \label{ex:hasseinclusion}
    Given a set~$\setA=\{\setAel,\setBel,\setCel\}$, consider its power set~$\powerset(\setA)$.
    Define sets as the objects of this new category and define the morphisms to be inclusions (\cref{fig:powersetcat}).

    The identity morphism of each set is the inclusion with itself (every set is a subset of itself).
    Composition is given by composition of inclusions, meaning that if~$\setA\subseteq \setB \subseteq \setC$, then~$\setA\subseteq \setC$.
\end{example}

\vfill
%

\begin{gradedexercise}[\exname{PolynomialDivisibility}]
    \label{ex:PolynomialDivisibility}
    Let~$\setA$ be the set of all polynomials with coefficients in $\reals$.
    Recall that a polynomial~$p$ \emph{divides} a polynomial~$q$ if there exists a polynomial~$m$ such that~$p \cdot m = q$.
    If~$p$ divides~$q$ we denote this by~$p \vert q$.
    Divisibility defines an endorelation on~$\setA$ by saying~$p$ is related to~$q$ iff~$p \vert q$.
    Does this define a preorder structure on~$\setA$?
    Does this define a poset structure on~$\setA$?
    Justify your answer.
\end{gradedexercise}

\solutionof{PolynomialDivisibility}

\devel{To move somewhere else
\paragraph{A note on preorders}
The theory of design problems can be easily generalized to preorders.
This means that there could be two elements~$\posAel$ and~$\posBel$ such that~$\posAel\posAleq \posBel$ and~$\posAel \posAgeq \posBel$ but~$\posAel \neq \posBel$.

This is actually common in practice.
For example, if the order relation comes from human judgement, such as customer preference, all bets are off regarding the consistency of the relation.
We will only refer to posets for two reasons:
\begin{enumerate}
    \item The exposition is smoother.
    \item Given a pre-order, computation will always involve passing to the poset representation.
\end{enumerate}
This means that, given a pre-order, we can consider the poset of its isomorphism classes, by means of the following equivalence relation:
\begin{equation}
    \posAel \simeq \posBel \quad \equiv \quad (\posAel \posAleq \posBel) \wedge (\posBel \posAleq \posAel).
\end{equation}
}

\section{Counting orders}
\linkvideo{spring2021-tradeoffs:tradeoffs:orders:counting-orders} % Counting orders

For one element, one has only the singleton poset (\cref{fig:singleton}).

On 2-elements sets, one has, up to isomorphism, the 2 posets reported in~\cref{fig:twoelementspos}.

On 3-elements sets, one has 5 posets reported in~\cref{fig:threeelementspos}.

On 4-elements sets, one has 16 posets reported in~\cref{fig:fourelementspos}.

\begin{marginfigure}
    \centering
    \includesag{40_dpcatfig_singleton}
    \caption{The singleton poset.}
    \label{fig:singleton}
\end{marginfigure}

\begin{example}[Singleton poset]
    \label{ex:singleton}
    If a set has only one element, say~$\singletonel$, then there is a unique order relation on it (\cref{fig:singleton}).
    We denote the resulting poset again by~$\singleton$.
\end{example}

\begin{figure*}[p]
    \centering
    \setlength{\tabcolsep}{20pt}
    \begin{tabular}{cc}
        \middlesag{70_pos_2_1} & \middlesag{70_pos_2_2}
    \end{tabular}
    \caption{All posets on 2-elements sets, up to isomorphisms.}
    \label{fig:twoelementspos}
\end{figure*}

\begin{figure*}[p]
    \centering
    \setlength{\tabcolsep}{20pt}
    \begin{tabular}{ccccc}
        \middlesag{70_pos_3_1} &
        \middlesag{70_pos_3_2} &
        \middlesag{70_pos_3_3} &
        \middlesag{70_pos_3_4} &
        \middlesag{70_pos_3_5}
    \end{tabular}
    \caption{All posets on 3-elements sets, up to isomorphisms. }
    \label{fig:threeelementspos}
\end{figure*}


\begin{figure*}[p]
    \centering
    \setlength{\tabcolsep}{15pt}
    \begin{tabular}{cccc}
        \middlesag{70_pos_1}  & \middlesag{70_pos_2}  & \middlesag{70_pos_3}  & \middlesag{70_pos_4}  \\[+40pt]
        \middlesag{70_pos_5}  & \middlesag{70_pos_6}  & \middlesag{70_pos_7}  & \middlesag{70_pos_8}  \\[+40pt]
        \middlesag{70_pos_9}  & \middlesag{70_pos_10} & \middlesag{70_pos_11} & \middlesag{70_pos_12} \\[+40pt]
        \middlesag{70_pos_13} & \middlesag{70_pos_14} & \middlesag{70_pos_15} & \middlesag{70_pos_16}
    \end{tabular}
    \caption{All posets on a 4-element sets, up to isomorphism. }
    \label{fig:fourelementspos}
\end{figure*}


\section{Upper and lower bounds}
\linkvideo{spring2021-tradeoffs:tradeoffs:orders:up-low-bounds} % Upper and lower bounds
\begin{ctdefinition}[Upper bounds in a poset]
    \label{def:least-upper-bound}
    The \emph{upper bounds} of a subset~$\setA$ of a poset~$\posA$ are, if they exist, the elements of~$\posA$ which dominate all elements in~$\setA$.
    In other words, the upper bounds of~$\setA$ are the elements of the set
    \begin{equation*}
        \{ \elb \in \posA \mid \forall \ela \in \setA  \colon \ela \leq \elb \}.
    \end{equation*}
    A \emph{least upper bound} of~$\setA \subseteq \posA$, if it exists, is a least element among the upper bounds of~$\setA$.
    It is denoted~$\join \setA$ or~$\Sup\setA$, and also called the \emph{join} or \emph{supremum} of~$\setA$.
    So, given~$\setA \subseteq \posA$ and~$\elb \in \posA$,~$\elb =  \join \setA$ if and only if
    \begin{enumerate}
        \item $\ela \leq \elb \ \forall \ela \in \setA$, and
        \item $\ela \leq \elb' \ \forall \ela \in \setA \Rightarrow \elb \leq \elb'.
              $
    \end{enumerate}
    If a least upper bound of a subset~$\setA \subseteq \posA$ exists, it is unique (can you prove this?
    ), so we speak of ``the'' least upper bound.
\end{ctdefinition}

\begin{exercise}
    Let~$\posA$ be a poset and~$\setA \subseteq \posA$ a subset of the underlying set of~$\posA$.
    Show that if~$\join \setA \in \posA$ exists, then it is unique.
    For this, assume that~$\elb$ and~$\elb'$ are both least upper bounds of~$\setA$, and then show that this assumption implies that in fact~$\elb = \elb'$.
\end{exercise}
\begin{solution}
    \missingsolution
    \todotextjira{291}{Write solution}
\end{solution}



\begin{marginfigure}
    \centering
    \includesag{upper_bound_1}
    \caption{Example of upper bounds. }
    \label{fig:upper_bound_example}
\end{marginfigure}

\begin{marginfigure}
    \centering

    \includesag{upper_bound_2}
    \caption{Example of upper bounds and least upper bound. }
    \label{fig:upper_bound_example_bis}
\end{marginfigure}

\begin{example}
    Consider the poset~$\posA$ and its subset~$\setA$ depicted in \cref{fig:upper_bound_example}.
    The \textcolor{red}{red} markers~$\textcolor{red}{\marker}$ represent the upper bound of~$\setA$.
    For this specific case, there is \emph{no} least upper bound, since the upper bounds do not dominate each other.
    Similarly, the poset~$\posB$ and its subset~$\setA$ depicted in \cref{fig:upper_bound_example_bis} possess two upper bounds.
    In this case, however, we can relate the two upper bounds, and find a least upper bound.

\end{example}

\begin{example}
    Least upper bounds need not necessarily exist even in total orders.
    For instance, the subset
    \begin{equation*}
        \posReals = \{x\in \reals \colon x>0\}
    \end{equation*}
    of the poset~$\reals$ (with the usual ordering) does not have a least upper bound.
\end{example}

Analogously to the case of (least) upper bounds, one can define lower bounds and greatest lower bounds.

\begin{ctdefinition}[Lower bounds in a poset]
    \label{def:greatest-lower-bound}
    The \emph{lower bounds} of a subset~$\setA$ of a poset~$\posA$ are, if they exist, the elements which are dominated by all elements in~$\setA$.
    The \emph{greatest lower bound}, if it exists, is the greatest among the lower bounds of~$\setA$.
    This is denoted~$\meet \setA$ or~$\Inf \setA$ and also called the \emph{meet} or \emph{infimum} of~$\setA$.
\end{ctdefinition}

\begin{exercise}
    Come up with an example of a subset~$\setA$ of a poset~$\posA$ which has lower bounds but no greatest lower bound.
    Then, modify it to have a greatest lower bound.
\end{exercise}

\begin{solution}
    \begin{marginfigure}
        \centering
        \includesag{lower_bound_1}
        \caption{Example of lower bounds. \label{fig:lower_bound_1}}
    \end{marginfigure}
    \begin{marginfigure}
        \centering
        \includesag{lower_bound_2}
        \caption{Example of lower bounds and greatest lower bounds. \label{fig:lower_bound_2}}
    \end{marginfigure}

    In \cref{fig:lower_bound_1} you find an example of a subset~$\setA$ of a poset~$\posA$ which has uncomparable lower bounds.
    In \cref{fig:lower_bound_2} instead, one also haves a greatest lower bound.

\end{solution}
