% !TEX root = chapter-standalone.tex


\section{Ordered sets} \label{sec:tradeoffs-ordered-sets}
\linkvideo{spring2021-tradeoffs:tradeoffs:orders:pre-pos-tot} % Pre-orders, partial orders, and total orders

\linkvideo{spring2021-tradeoffs:tradeoffs:orders} % Orders

\devel{
\begin{forslides}
\begin{equation*}
    \label{eq:tradeoff_map}
    \mapa \colon \stylesets{X} \sto \stylesets{Y}
\end{equation*}
\begin{equation*}
    \label{eq:tradeoff_mor}
    \mora \colon \Obja \mto \Objb
\end{equation*}
\begin{equation*}
    \label{eq:tradeoff_typw}
    \mora \colon (\Obja \mto \Objb)
\end{equation*}
    \begin{equation*}
    \label{eq:tradeoff_cap}
    \begin{aligned}
        \setA &\cap \setB\\
        \setA &\cup \setB\\
        \setA & \subseteq \setB
    \end{aligned}
\end{equation*}
     \begin{equation*}
    \label{eq:tradeoff_wedge}
    \begin{aligned}
        a &\wedge b\\
        a &\vee b\\
    \end{aligned}
\end{equation*}
\begin{equation*}
    \label{eq:tradeoff_leq}
    \setA \leq \setB
\end{equation*}
\begin{equation*}
    \label{eq:abc}
    \prflineextra=0.6em
    \prftree
    {\prftree
    {a}
      {b}}
    {c}
\end{equation*}
\begin{equation*}
    \label{eq:abc_2}
    \begin{aligned}
        &a\\
        \Imp \quad &b\\
        \Imp \quad &c
\end{aligned}
\end{equation*}
    \begin{equation*}
        \label{eq:abc_3}
        \prftree{a}{b}{c}
\end{equation*}
        \begin{equation*}
        \label{eq:abc_4}
        (a\wedge b)\Imp c
\end{equation*}
\begin{equation*}
    \label{eq:abc_5}
    \prflineextra=0.6em
    \prftree[d]
    {\prftree[d]
    {a}
      {b}}
    {c}
\end{equation*}
\begin{equation*}
    \label{eq:abc_6}
    a\Leftrightarrow b \Leftrightarrow c
\end{equation*}
    \begin{equation*}
    \label{eq:abc_7}
    \prflineextra=0.6em
    \prftree{a}{b}
\end{equation*}
        \begin{equation*}
    \label{eq:abc_8}
    \prflineextra=0.6em
    \prftree{b}{c}
\end{equation*}
            \begin{equation*}
    \label{eq:abc_9}
    \prflineextra=0.6em
    \prftree{a}{c}
\end{equation*}
\begin{equation*}
    \label{eq:mapf}
    \mapa \colon
\end{equation*}
\begin{equation*}
    \label{eq:mapg}
    \mapb \colon
\end{equation*}
\begin{equation*}
    \label{eq:mapcomp}
    \mapa \then \mapb \colon
\end{equation*}
\begin{equation*}
    \label{eq:top1}
    \prflineextra=0.6em
    \prftree{\top}{a}
\end{equation*}
\begin{equation*}
    \label{eq:bot1}
    \prflineextra=0.6em
    \prftree{a}{\bot}
\end{equation*}
\begin{equation*}
    \label{eq:rel_refl}
    \prflineextra=0.6em
    \prftree{\top}{a\relA a}
\end{equation*}
\begin{equation*}
    \label{eq:rel_tot}
    \prflineextra=0.6em
    \prftree{\top}{a\relA b \quad \vee \quad b\relA a}
\end{equation*}
\begin{equation*}
    \label{eq:rel_tran}
    \prflineextra=0.6em
    \prftree{a\relA b}{b\relA c}{a\relA c}
\end{equation*}
\begin{equation*}
    \label{eq:rel_sym}
    \prflineextra=0.6em
    \prftree[d]{a \relA b}{b\relA a}
\end{equation*}
\begin{equation*}
    \label{eq:rel_irr}
    \prflineextra=0.6em
    \prftree{a \relA a}{\bot}
\end{equation*}
\begin{equation*}
    \label{eq:rel_asy}
    \prflineextra=0.6em
    \prftree{a \relA b}{b\relA a}{\bot}
\end{equation*}
\begin{equation*}
    \label{eq:rel_antisy}
    \prflineextra=0.6em
    \prftree{a \relA b}{b\relA a}{a=b}
\end{equation*}
    \begin{equation*}
    \label{eq:ex}
    a\relA b \Leftrightarrow (a-b) \text{ mod }3=0
\end{equation*}
    \begin{equation*}
        \label{eq:fun_cost}
        (\F{\text{keeps warm }}\times \F{\text{ music fidelity}})\times(\R{\text{ price }} \times \R{\text{ frequency of charging }}\times \R{\text{ wires hassle}})
\end{equation*}
    \begin{equation*}
        \label{eq:id_ex}
        \prftree{\top}{\catid_\Obja\colon \Obja \mto \Obja}
\end{equation*}
\begin{equation*}
\label{eq:comp_cat}
\prftree{\mora \colon \Obja \mto \Objb}{\morb\colon \Objb \mto \Objc}{\mora\mthen \morb \colon \Obja \mto \Objc}
\end{equation*}
\begin{equation*}
\label{eq:mor_pre}
\prftree[d]{\ast\colon \posAel \mto \posBel}{\posAel \posleq \posBel}
\end{equation*}
\begin{equation*}
\label{eq:hom_pre}
\HomSet{}{\posAel}{\posBel}=\begin{cases}\{\ast\} &\text{iff } \posAel \posleq \posBel\\
\emptyset &\text{otherwise}\end{cases}
\end{equation*}
 \begin{equation*}
\label{eq:comp_pre}
\ast \mthen \ast = \ast
\end{equation*}
 \begin{equation*}
\label{eq:ex_pre}
\begin{aligned}
    &\posAel,\posBel,\posCel\\
    &\posAel \posleq \posBel\\
    &\posCel \posleq \posBel
\end{aligned}
\end{equation*}
 \begin{equation*}
\label{eq:pre_p}
\posAel
\end{equation*}
 \begin{equation*}
\label{eq:pre_q}
\posBel
\end{equation*}
 \begin{equation*}
\label{eq:pre_r}
\posCel
\end{equation*}
\begin{equation*}
    \label{eq:eq_rel}
    \prftree[d]{\posAel\sim \posBel}{(\posAel \posleq \posBel)\wedge (\posBel \posleq \posAel)}
\end{equation*}
\begin{equation*}
    \label{eq:pos_int}
    \prflinepadbefore=0.6em
    \prflinepadafter=0.6em
    \prftree[d]{\tup{l_1,u_1}\posleq_{\styleelements{\text{Int}}\posA} \tup{l_2,u_2}}{(l_1\posAleq l_2)\wedge (u_2\posAleq u_1)}
\end{equation*}
 \begin{equation*}
    \label{eq:leq}
     \posleq
\end{equation*}
\begin{equation*}
    \label{eq:pos_prod}
    \prflinepadbefore=0.6em
    \prflinepadafter=0.6em
    \prftree[d]{\tup{\posAel_1,\posBel_1}\posleq_{\posA\times \posB} \tup{\posAel_2,\posBel_2}}{(\posAel_1\posAleq \posAel_2)\wedge(\posBel_1\posBleq \posBel_2)}
\end{equation*}
 \begin{equation*}
    \label{eq:pos_P}
     \posA
\end{equation*}
     \begin{equation*}
    \label{eq:pos_Q}
     \posB
\end{equation*}
     \begin{equation*}
    \label{eq:pos_PQ}
     \posA\times \posB
\end{equation*}
    \begin{equation*}
        \label{eq:set_disj}
       \posA+\posB=\{ \tup{1,\posAel} \mid \posAel\in \posA \}\cup \{\tup{2,\posBel}\mid \posBel \in \posB\}
\end{equation*}
    \begin{equation*}
        \label{eq:latt_1}
        \posAel \wedge \posBel
\end{equation*}
        \begin{equation*}
        \label{eq:latt_2}
        \posAel \vee \posBel
\end{equation*}
            \begin{equation*}
        \label{eq:latt_3}
        (\posAel \wedge \posBel)\posleq \posAel \posleq (\posAel \vee \posBel)
\end{equation*}
        \begin{equation*}
        \label{eq:latt_4}
        \posAel
\end{equation*}
        \begin{equation*}
        \label{eq:latt_5}
        \posBel
\end{equation*}
    \begin{equation*}
        \label{eq:prop_1}
        \prftree[d]{\posAel \posleq \posBel}{\posAel \Imp \posBel}
\end{equation*}
        \begin{equation*}
        \label{eq:prop_2}
        \prftree[d]{\posAel \Leftrightarrow \posBel}{\posAel = \posBel}
\end{equation*}
        \begin{equation*}
        \label{eq:prop_3}
        \prftree{\top}{\posAel \Imp \posAel}
\end{equation*}
        \begin{equation*}
        \label{eq:prop_4}
        \prftree{(\posAel \Imp \posBel)\wedge (\posBel \Imp \posCel)}{\posAel \Imp \posCel}
\end{equation*}
    \includesag{060_propositions_lat}
    \begin{equation*}
        \label{eq:pow_1}
        \setA \posleq \setB \definedas \setA \subseteq \setB
\end{equation*}
        \begin{equation*}
        \label{eq:pow_2}
        \setA \vee \setB \definedas \setA \cup \setB
\end{equation*}
            \begin{equation*}
        \label{eq:pow_3}
        \setA \wedge \setB \definedas \setA \cap \setB
\end{equation*}
            \begin{equation*}
        \label{eq:pow_4}
        \top \definedas \stylesets{S}
\end{equation*}
                \begin{equation*}
        \label{eq:pow_5}
        \bot \definedas \stylesets{\emptyset}
\end{equation*}
\begin{equation*}
        \label{eq:pow_6}
        \powerset{\stylesets{S}}
\end{equation*}
    \begin{equation*}
        \label{eq:pow_7}
        \stylesets{S}
\end{equation*}
    \begin{equation*}
        \label{eq:pre_ex1}
        \{\tup{a,a},\tup{a,b},\tup{b,a},\tup{b,b}\}
\end{equation*}
        \begin{equation*}
        \label{eq:pre_ex2}
        \{\tup{a,b},\tup{b,c},\tup{c,a}\}
\end{equation*}
            \begin{equation*}
        \label{eq:pre_ex3}
        \{\tup{a,a},\tup{b,b},\tup{c,c},\tup{d,d}\}
\end{equation*}
    \begin{definition}
        \label{def:posdef}
        A symmetric matrix~$\mat{M}\in \reals^{n\times n}$ is \emph{positive semi-definite} if~$x^\intercal \mat{M}x\geq 0$ for all non-zero~$x\in \reals^n$. We call the set of all such matrices~$\mathcal{P}^n$.
\end{definition}
    \begin{equation*}
        \label{eq:posdef_1}
        \mat{A}\posleq \mat{B} \Leftrightarrow (\mat{A}-\mat{B})\in \mathcal{P}^n, \quad \mat{A},\mat{B}\in \mathcal{P}^n
\end{equation*}
\end{forslides}}


So far, the discussion has been purely qualitative. While we discussed how
categories can describe the way in which one resource can be turned into another,
this kind of modelling did not allow for quantitative statements. For example, it
is good to know that we can obtain motion from electric power, but, how fast can
we go with a certain amount of power?

To achieve a quantitative theory, we need to specify various degrees of resources and functionality.
One way of doing this, is through the idea of orders.


Such orderings arise naturally in engineering as criteria for judging whether one design is better or worse than another.
As an example, suppose you need to prepare some pizza: you have to buy specific ingredients and cook them, using a recipe you decide to follow.
In this simple example, you can think of having two resources: time and money.
A quicker recipe might include more expensive ingredients, and a slower recipe could feature more affordable ones.
How to choose among recipes, if you do not prefer one resource over the other? How to model this? In this section, we will assume that functionality and resources
are \emph{ordered sets}, and will introduce pre-orders, partial orders, and total orders.

Davey and Priestley~\cite{davey02}
and Roman~\cite{roman08} are possible reference texts.

We introduce these concepts by adding levels of specificity.

\begin{definition}[Pre-ordered set]
\label{def:preorder}
    A \emph{pre-ordered} set is a tuple~$\tup{\posA,\posAleq}$, where~$\posA$ is a set (also called the \emph{carrier set}), together with a relation~$\posAleq$ that is:
    \begin{compactitem}
        \item \emph{Reflexive:}
        \begin{equation}
            \prftree{\top}{\posAel \posAleq \posAel}
        \end{equation}
        \item \emph{Transitive:}
        \begin{equation}
            \prftree{\posAel \posAleq \posBel}{\posBel \posAleq \posCel}{\posAel \posAleq \posCel}
        \end{equation}
    \end{compactitem}
\end{definition}
By adding a property called \emph{antisymmetry}, one obtains a partially ordered set.
\begin{definition}[Partially ordered set]
  \label{def:poset}
  A pre-ordered set~$\tupp{\posA,\posAleq}$ is a \emph{partially-ordered set (\iindex{poset})} if the relation~$\posAleq$ is \emph{antisymmetric}. In other words, if:
      \begin{equation}
          \prftree{\posAel \posAleq \posBel}{\posBel \posAleq \posAel}{\posAel = \posBel}
    \end{equation}
\end{definition}

\begin{definition}[Totally ordered set]
    \label{def:total_order}
    A partially ordered set~$\tupp{\posA,\posAleq}$ is a \emph{totally ordered set} if the relation~$\posAleq$ is \emph{total}. In other words, if:
    \begin{equation}
        \prftree{\top}{(\posAel \posAleq \posBel) \vee (\posBel \posAleq \posAel)}
    \end{equation}
\end{definition}

\linkvideo{spring2021-tradeoffs:tradeoffs:orders:general-pref} % More general preference structures
\begin{remark}
    There are more general ways to describe preferences. One has:
    \begin{compactitem}
    \item \emph{Quasitransitive relations}, which are relations~$\posAleq$ over a set~$\posA$ for which
    \begin{equation}
        \prftree{(\posAel \posAleq \posBel)}{\neg(\posBel \posAleq \posAel)}{(\posBel \posAleq \posCel)}{\neg(\posCel \posAleq \posBel)}{(\posAel \posAleq \posCel) \wedge \neg(\posCel \posAleq \posAel)}
    \end{equation}
    \item \emph{Semiorders}, which are relations~$\posAleq$ over a set~$\posA$ such that:
    \begin{compactitem}
        \item \emph{Asymmetry} (not antisymmetry!) holds:
        \begin{equation}
            \prftree{\posAel \posAleq \posBel}{\neg(\posAel \posAleq \posAel)}
        \end{equation}
        \item Let's denote two elements~$\posBel, \posCel \in \posA$ which are \emph{incomparable} by~$\posBel \sim \posCel$. One has:
        \begin{equation}
            \prftree{\posAel \posAleq \posBel}{\posBel \sim \posCel}{\posCel \posAleq \posDel}{\posAel \posAleq \posDel}
        \end{equation}
        \item One has:
        \begin{equation}
            \prftree{\posAel \posAleq \posBel}{\posBel \posAleq \posCel}{(\posAel \posAleq \posDel) \vee (\posDel \posAleq \posAel)}
        \end{equation}
    \end{compactitem}
        An example of semiorder is the following. Say that you have to express your preference over numbers in~$\posA=\{10,11,12\}$. You are indifferent between 10 and 11, and between 11 and 12. Though, you prefer 10 to 12.
    \end{compactitem}
\end{remark}


\linkvideo{spring2021-tradeoffs:tradeoffs:orders:hasse} % Hasse diagrams
A \emph{\iindex{Hasse diagram}} is an economical (in terms of arrows) way to visualize a poset.
In a Hasse diagram elements are points, and if~$\posAel \posAleq \posBel$ then~$\posAel$ is drawn lower than~$\posBel$ and with an ege connected to it, if no other point is in between.
Hasse diagrams are directed graphs.


\begin{marginfigure}
  \begin{center}
    \includesag{70_hasse_pizza}
    \caption{The cost of pizza ingredients can be represented as a poset.}
    \label{fig:hassepizza}
  \end{center}
\end{marginfigure}

In the example of the pizza recipes, both time and money can be thought of as partially ordered sets~$\tup{\nonNegReals,\Rleq}$ (actually, both are particularly totally ordered sets).
Imagine that you have recipes costing \unit[1]{\stdcurr}, \unit[2]{\stdcurr}, and \unit[3]{\stdcurr}.
This can be represented as in~\cref{fig:hassepizza}.


\begin{marginfigure}
  \centering
  \includesag{40_dpcatfig_hasse}
  \caption{Example of Hasse diagram of~$\posA$.}
  \label{fig:hasse}
\end{marginfigure}


\begin{example}
  Consider a poset~$\posA=\{a,b,c,d,e\}$ with~$a\posAleq b$,~$a\posAleq c$,~$d\posAleq c$, and~$d\posAleq e$.
  This can be represented with a Hasse diagram as in~\cref{fig:hasse}.
\end{example}

\begin{marginfigure}
  \centering
  \includesag{40_dpcatfig_singleton}
  \caption{The singleton poset.}
  \label{fig:singleton}
\end{marginfigure}


\begin{example}[Singleton poset]
  \label{ex:singleton}
  If a set has only one element, say~$\singletonel$, then there is a unique order relation on it (\cref{fig:singleton}).
  We denote the resulting poset again by~$\singleton$.
\end{example}

 \begin{figure}[h!]
    \begin{center}
      \setlength{\tabcolsep}{20pt}
      \begin{tabular}{ccc}
        \includesag{70_pos_3_1}& \includesag{70_pos_3_2}& \includesag{70_pos_3_3}
      \end{tabular}\\
      \begin{tabular}{cc}
        \includesag{70_pos_3_4}& \includesag{70_pos_3_5}
      \end{tabular}
    \end{center}
    \caption{All posets on 3-elements sets, up to isomorphisms. }
    \label{fig:threeelementspos}
  \end{figure}

\linkvideo{spring2021-tradeoffs:tradeoffs:orders:counting-orders} % Counting orders
\begin{example}
  In this example, we represent all posets up to isomorphisms on up to 4 elements.
  For one element, one has only the singleton poset (\cref{fig:singleton}).
  On 2-elements sets, one has the posets reported in~\cref{fig:twoelementspos}.
  \begin{figure}[tbh]
    \begin{center}
      \setlength{\tabcolsep}{20pt}
      \begin{tabular}{cc}
        \includesag{70_pos_2_1}& \includesag{70_pos_2_2}
      \end{tabular}
    \end{center}
    \caption{All posets on 2-elements sets, up to isomorphisms.}
    \label{fig:twoelementspos}
  \end{figure}
  On 3-elements sets, one has the posets reported in~\cref{fig:threeelementspos}.

  On 4-elements sets, one has the posets reported in~\cref{fig:fourelementspos}.

\end{example}

\begin{figure*}[p]
    \centering
%      \adjustbox{max width=\textwidth}{
        \setlength{\tabcolsep}{15pt}
        \begin{tabular}{cccc}
          \includesag{70_pos_1}  & \includesag{70_pos_2}  & \includesag{70_pos_3}  & \includesag{70_pos_4}  \\[+30pt]
          \includesag{70_pos_5}  & \includesag{70_pos_6}  & \includesag{70_pos_7}  & \includesag{70_pos_8}\\[+30pt]
          \includesag{70_pos_9}  & \includesag{70_pos_10} & \includesag{70_pos_11} & \includesag{70_pos_12}\\[+30pt]
          \includesag{70_pos_13} & \includesag{70_pos_14} & \includesag{70_pos_15} & \includesag{70_pos_16}
        \end{tabular}
%      }
    \caption{All posets on 4-elements sets, up to isomorphisms. }
    \label{fig:fourelementspos}
  \end{figure*}


\begin{marginfigure}
\centering
\includesag{40_dpcatfig_boolean}
\caption{}
\label{fig:boolean}
\end{marginfigure}

\begin{example}[Booleans]
  \label{ex:bool}
  The booleans \index{\Bool} is a poset with carrier set~$\{\true,\false\}$ and the order relation given by~$b_1 \posleq_\Bool b_2$ iff~$b_1 \Imp b_2$, that is,~$\false \posleq_\Bool \true$ (\cref{fig:boolean}).


  This relation should be familiar from~\cref{tab:boolposet}.

  \begin{margintable}
    \centering
      \begin{tabular}{cc|ccc}
        $a$      & $b$      & $a \posleq  b$ & $a \wedge b$ & $a \vee b$ \\ \hline
        $\true$  & $\true$  & $\true$             & $\true$      & $\true$    \\
        $\true$  & $\false$ & $\false$            & $\false$     & $\true$    \\
        $\false$ & $\true$  & $\true$             & $\false$     & $\true$    \\
        $\false$ & $\false$ & $\true$             & $\false$     & $\false$
      \end{tabular}
    \caption{Properties of the \Bool poset. Note that $\posleq \equiv \Imp$.}
     \label{tab:boolposet}
  \end{margintable}

  In addition to the operation
  \begin{equation*}
    \Imp\colon\Bool\times \Bool\to\Bool,
  \end{equation*}
  called \emph{Imp}, there are also the familiar \emph{and} ($\wedge$) and \emph{or} ($\vee$) operations. Note that~$\wedge$ and~$\vee$ are commutative ($b\wedge c = c\wedge b$,~$b\vee c = c\vee b$ ), whereas~$\Imp$ is not.
\end{example}

\begin{example}[Reals]
  The real numbers \reals form a poset with carrier \reals and order relation given by the usual ordering~$r_1 \Rleq r_2$.
\end{example}

 \begin{marginfigure}
    \centering
    \includesag{40_discrete}
    \caption{Example of a discrete poset.}
    \label{fig:discretepos}
  \end{marginfigure}

\begin{example}[Discrete partially ordered sets]
  \label{ex:discreteposet}
  Every set~$\setA$ can be considered as a \emph{discrete poset}~$\tup{\setA,=}$.
  Discrete posets are represented as collection of points (\cref{fig:discretepos}).
\end{example}

 \begin{marginfigure}
      \begin{center}
        \includesag{40_dpcatfig_power}
      \end{center}
      \caption{Power set as a category.}
      \label{fig:powersetcat}
  \end{marginfigure}

\begin{example}
A symmetric matrix~$\mat{M}\in \reals^{n\times n}$ is \emph{positive semi-definite} if~$x^\intercal \mat{M}x\geq 0$ for all non-zero~$x\in \reals^n$.
We call the set of all such matrices~$\mathcal{P}^n$. Such matrices have real, semi-positive eigenvalues, which can be interpreted as axes lenghts of ellipsoids.
Any matrix~$\mat{A}\in \mathcal{P}^n$ describes an ellipsoid, descriptive equation of which can be written as a quadratic form:
\begin{equation*}
    x^\intercal \mat{A}x=1,\quad x\in \reals^n.
\end{equation*}
We can define a partial order on~$\mathcal{P}^n$ as:
\begin{equation*}
\mat{A}\posleq \mat{B} \Leftrightarrow (\mat{A}-\mat{B})\in \mathcal{P}^n, \quad \mat{A},\mat{B}\in \mathcal{P}^n
\end{equation*}
The order can be interpreted as an inclusion of ellipsoids. Take for instance the matrices
\begin{equation*}
    \mat{A}=\begin{pmatrix}1&0\\0&1\end{pmatrix}, \quad \mat{B}=\begin{pmatrix}2&0\\0&1\end{pmatrix},\quad \mat{A}=\begin{pmatrix}2&0\\0&0.5\end{pmatrix}.
\end{equation*}
The order on the set~$\{\mat{A},\mat{B},\mat{C}\}$ is reported in \cref{fig:posdef_hasse}, and it is easily explained via \cref{fig:posdef_draw}.
The ellipse representing~$\mat{A}$ (in red) is included by the one representing matrix~$\mat{B}$ (in blue), but not by the one representing matrix~$\mat{C}$ (in green).
Furthermore, the one representing~$\mat{B}$ includes the one representing~$\mat{C}$.
\end{example}

\begin{marginfigure}
    \begin{center}
\includesag{20_ellipses_mat}
    \end{center}
    \caption{Example of ellipses representing positive semi-definite matrices.}
    \label{fig:posdef_draw}
\end{marginfigure}

\begin{marginfigure}
    \begin{center}
\includesag{20_mat_order}
\end{center}
    \caption{Example of order between positive semi-definite matrices.}
    \label{fig:posdef_hasse}
\end{marginfigure}

  \begin{example}
    \label{ex:hasseinclusion}
    Given a set~$\setA=\{\setAel,\setBel,\setCel\}$, consider its power set~$\powerset(\setA)$.
    Define sets as the objects of this new category and define the morphisms to be inclusions (\cref{fig:powersetcat}).

    The identity morphism of each set is the inclusion with itself (every set is a subset of itself).
    Composition is given by composition of inclusions, meaning that if~$\setA\subseteq \setB \subseteq \setC$, then~$\setA\subseteq \setC$.
  \end{example}


\paragraph{A note on preorders}
The theory of design problems can be easily generalized to preorders.
This means that there could be two elements~$\posAel$ and~$\posBel$ such that~$\posAel\posAleq \posBel$ and~$\posAel \posAgeq \posBel$ but~$\posAel \neq \posBel$.

This is actually common in practice.
For example, if the order relation comes from human judgement, such as customer preference, all bets are off regarding the consistency of the relation.
We will only refer to posets for two reasons:
\begin{enumerate}
  \item The exposition is smoother.
  \item Given a pre-order, computation will always involve passing to the poset representation.
\end{enumerate}
This means that, given a pre-order, we can consider the poset of its isomorphism classes, by means of the following equivalence relation:
\begin{equation}
  \posAel \simeq \posBel \quad \equiv \quad (\posAel \posAleq \posBel) \wedge (\posBel \posAleq \posAel).
\end{equation}

\begin{gradedexercise}[\exname{PolynomialDivisibility}]\label{ex:PolynomialDivisibility}
Let~$\setA$ be the set of all polynomials with coefficients in $\reals$.
Recall that a polynomial~$p$ \emph{divides} a polynomial~$q$ if there exists a polynomial~$m$ such that~$p \cdot m = q$.
If~$p$ divides~$q$ we denote this by~$p \vert q$.
Divisibility defines an endorelation on~$\setA$ by saying~$p$ is related to~$q$ iff~$p \vert q$.
Does this define a preorder structure on~$\setA$? Does this define a poset structure on~$\setA$? Justify your answer.
\end{gradedexercise}

\solutionof{PolynomialDivisibility}

\section{Upper and lower bounds}
\todo{add}
\linkvideo{spring2021-tradeoffs:tradeoffs:orders:up-low-bounds} % Upper and lower bounds