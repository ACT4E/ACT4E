% !TEX root = chapter-standalone.tex


\section{Lattices}

\linkvideo{spring2021-tradeoffs:tradeoffs:orders:top-bottom} % Top and bottom
\linkvideo{spring2021-tradeoffs:tradeoffs:orders:lattices} % Lattices


\begin{definition}[Lattice]
    \label{def:lattice}
    A \emph{lattice} is a poset~$\posA=\tupp{\posAset,\posAleq}$ with the additional property that, for any two-element subset~$\{\ela, \elb \} \subseteq \posA$, both the join~$\join \{\ela, \elb \}$ and the meet~$\meet \{\ela, \elb \}$ exist.
    Usually these are written using infix notation as~$\ela \join \elb$ and~$\ela \meet \elb$, respectively.
\end{definition}

\todographicsjira{74}{@Gioele: Add the wooden lattice figure here.}

\todojira{75}{@Gioele: These can be defined for preorders as well... move before.}
\begin{remark}[Bounded lattices]
    \label{rem:bounded-lattices}
    \label{def:top}
    \label{def:bot}
    If there is a least upper bound for the entire lattice~$\posA$, it is called
    the \emph{top} ($\postop$). If a greatest lower bound exists it is called the \emph{bottom} ($\posbot$).
    If both a top and a bottom exist, we call the lattice \emph{bounded}, and denote it by~$\tup{\posAset,\ordleq,\join,\meet,\posbot,\postop}$.
\end{remark}

\begin{example}
    In \cref{ex:hasseinclusion} we presented the poset arising from the power set~$\powerset \setA$ of a set~$\setA$ and ordered via subset inclusion.
    This is a lattice, bounded by~$\setA$ and by the empty set~$\emptyset$.
    Note that this lattice possesses two (dual) monoidal structures~$\tup{\powerset \setA,\subseteq,\emptyset,\cup}$ and~$\tup{\powerset \setA,\subseteq,\setA,\cap}$.
\end{example}
\begin{marginfigure}

    \subfloat[]{
        \includesag{40_dpcatfig_exlattice}}

    \subfloat[]{
        \includesag{40_dpcatfig_exlattice_bis}
    }
    \caption{Examples of a lattice and a non-lattice. }
    \label{fig:exlattice}
\end{marginfigure}

\begin{example}
    Consider the set~$\{1,2,3,6\}$ ordered by divisibility.
    For instance, since 2 divides 6, we have~$2\posleq 6$.
    This is a lattice.
    However, the set~$\{1,2,3\}$ ordered by divisibility is not, since 2 and 3 lack a meet (\cref{fig:exlattice}).
\end{example}


\begin{gradedexercise}[\exname{UpperLowerBounds}]
    \label{ex:UpperLowerBounds}
    Let~$\setA = \{\setAel, \setBel, \setCel, \setDel, \setEel \}$.
    Give examples of the following situations using Hasse diagrams.
    In each case, provide a poset structure on~$\setA$ and a subset~$\setB \subseteq \setA$ such that:
    \begin{enumerate}
        \item $\setB$ has a least upper bound;
        \item $\setB$ has a greatest lower bound;
        \item $\setB$ has no least upper bound;
        \item $\setB$ has no greatest lower bound.
    \end{enumerate}
\end{gradedexercise}

\solutionof{UpperLowerBounds}

\begin{lemma}
    \label{lem:u_bounded_lat}
    $\Up \posA$ is a bounded lattice (\cref{def:lattice}) with
    \begin{equation}
        \begin{aligned}
            \posleq_{\Up \posA}&\definedas \ \supseteq\\
            \posbot_{\Up \posA}&\definedas \posA\\
            \postop_{\Up \posA}&\definedas \emptyset\\
            \vee_{\Up \posA}&\definedas \cap\\
            \wedge_{\Up \posA}&\definedas \cup.
        \end{aligned}
    \end{equation}
\end{lemma}
\begin{proof}
    Consider the poset~$\uppersets \posA=\tup{\setOfUppersets \posA,\supseteq}$ and~$\setA,\setB\in \setOfUppersets \posA$.
    \begin{compactitem}
        \item First, we need to show that~$\setA\cap \setB\in \setOfUppersets \posA$.
        One has~$\setA \subseteq \setOfUppersets \posA$ and $\setB\subseteq \setOfUppersets \posA$, meaning that by definition, if~$\setAel\in \setA\cap \setB$, we have~$\setAel \in \setA \wedge \setAel \in \setB$.
        It follows that~$\setAel \in \setOfUppersets \posA$ for all~$\setAel \in \setA\cap \setB$.
        Furthermore, we need to show that~$\setA\cap \setB$ is the least upper bound of~$\setA$ and~$\setB$.
        Assume this is not true, meaning that there exists a~$\setC\in \setOfUppersets \posA$,~$\setC\neq \setA\cap \setB$, such that~$\setA\supseteq \setC\supseteq \setA\cap \setB$ and~$\setB\supseteq \setC\supseteq \setA\cap \setB$.
        Using the fact that intersection preserves inclusions, one has
        \begin{equation}
            \begin{aligned}
                \setA\cap \setB &\supseteq \setC\cap \setC \supseteq \setA\cap \setB\\
                \setA\cap \setB &\supseteq \setC \supseteq \setA\cap \setB\\
                \setC&= \setA\cap \setB,
            \end{aligned}
        \end{equation}
        which contradicts the assumption. Therefore,~$\setA\cap \setB$ is the least upper bound of~$\setA$ and~$\setB$.
        \item Second, we need to show that~$\setA\cup \setB\in \setOfUppersets \posA$.
        One has~$\setA\subseteq \setOfUppersets \posA$ and~$\setB\subseteq \setOfUppersets \posA$, meaning that by definition, if~$\setAel\in \setA\cup \setB$, we have either~$\setAel\in \setA$ or~$\setAel\in \setB$.
        If~$\setAel\in \setA$, then~$\setAel\in \setOfUppersets \posA$.
        If~$\setAel\in \setB$, then~$\setAel \in \setOfUppersets \posA$.
        It follows that~$\setAel \in \setOfUppersets \posA$ for all~$\setAel\in \setA\cup \setB$.
        Furthermore, we need to show that~$\setA\cup \setB$ is the greatest lower bound of~$\setA$ and~$\setB$.
        Assume this is not true, meaning that there exists a~$\setC\in \setOfUppersets \posA$,~$\setC\neq \setA\cup \setB$, such that~$\setA\cup \setB\supseteq \setC\supseteq \setA$ and~$\setA\cup \setB\supseteq \setC\supseteq \setB$.
        Using the fact that union preserves inclusions, one has
        \begin{equation}
            \begin{aligned}
            (\setA\cup \setB)
                \cup (\setA\cup \setB) &\supseteq \setC \cup \setC \supseteq \setA\cup \setB\\
                \setA\cup \setB &\supseteq \setC\supseteq \setA\cup \setB\\
                \setC&=\setA\cup \setB,
            \end{aligned}
        \end{equation}
        which contradicts the assumption. Therefore,~$\setA\cup \setB$ is the greatest lower bound of~$\setA$ and~$\setB$.
    \end{compactitem}
    We have therefore proved that~$\uppersets \posA=\tup{\setOfUppersets \posA,\supseteq}$ is a lattice.
    To show that it is bounded, we notice that~$\emptyset \subseteq \setC$ for any~$\setC\in \setOfUppersets \posA$, meaning that~$\emptyset$ is the top.
    Furthermore, we notice that~$\setC\subseteq \posA$ for any~$\setC\in \setOfUppersets \posA$, meaning that~$\posA$ is a bottom.
    Therefore, the lattice is bounded.
\end{proof}

\begin{lemma}
    $\posLA$ is a bounded lattice (\cref{def:lattice}) with:
    \begin{equation}
        \begin{aligned}
            \posleq_{\posLA}&\definedas \ \subseteq\\
            \posbot_{\posLA}&\definedas \emptyset\\
            \postop_{\posLA}&\definedas \posA\\
            \vee_{\posLA}&\definedas \cup\\
            \wedge_{\posLA}&\definedas \cap.
        \end{aligned}
    \end{equation}
\end{lemma}
\begin{proof}
    Consider the poset~$\posLA=\tup{\setOfLowersets \posA,\subseteq}$ and~$\setA,\setB\in \setOfLowersets \posA$.
    \begin{compactitem}
        \item First, we need to show that~$\setA\cup \setB\in \setOfLowersets \posA$.
        One has~$\setA \subseteq \setOfLowersets \posA$ and~$\setB\subseteq \setOfLowersets \posA$, meaning that by definition, if~$\setAel\in \setA\cup \setB$, either~$\setAel\in \setA$ or~$\setAel\in \setB$.
        If~$\setAel\in \setA$, then~$\setAel\in \setOfLowersets \posA$.
        If~$\setAel\in \setB$, then~$\setAel\in \setOfLowersets \posA$.
        It follows that~$\setAel\in \setOfLowersets \posA$ for all~$\setAel\in \setA\cup \setB$.
        Furthermore, we need to show that~$\setA\cup \setB$ is the least upper bound of~$\setA$ and~$\setB$.
        Assume this is not true, meaning that there exists a~$\setC\in \setOfLowersets \posA$,~$\setC\neq \setA\cup \setB$, such that~$\setA\subseteq \setC\subseteq \setA\cup \setB$ and~$\setB\subseteq \setC\subseteq \setA\cup \setB$.
        Using the fact that union preserves inclusions, one has
        \begin{equation}
            \begin{aligned}
                \setA\cup \setB &\subseteq \setC\cup \setC \subseteq \setA\cup \setB\\
                \setA\cup \setB &\subseteq \setC \subseteq \setA\cup \setB\\
                \setC&= \setA\cup \setB,
            \end{aligned}
        \end{equation}
        which contradicts the assumption. Therefore,~$\setA\cup \setB$ is the least upper bound of~$\setA$ and~$\setB$.
        \item Second, we need to show that~$\setA\cap \setB\in \setOfLowersets \posA$.
        One has~$\setA\subseteq \setOfLowersets \posA$ and~$\setB\subseteq \setOfLowersets \posA$, meaning that by definition, if~$\setAel\in \setA\cap \setB$, we have~$\setAel \in \setA\wedge \setAel\in \setB$ ($\setAel\in \setOfLowersets \posA$, for all~$\setAel\in \setA\cap \setB$).
        Furthermore, we need to show that~$\setA\cap \setB$ is the greatest lower bound of~$\setA$ and~$\setB$.
        Assume this is not true, meaning there exists a~$\setC\in \setOfLowersets \posA$,~$\setC\neq \setA\cap \setB$, such that~$\setA\cap \setB\subseteq \setC\subseteq \setA$ and~$\setA\cap \setB\subseteq \setC\subseteq \setB$.
        Using the fact that intersection preserves inclusions, one has
        \begin{equation}
            \begin{aligned}
            (\setA\cap \setB)
                \cap (\setA\cap \setB) &\subseteq \setC \cap \setC \subseteq \setA\cap \setB\\
                \setA\cap \setB &\subseteq \setC\subseteq \setA\cap \setB\\
                \setC&=\setA\cap \setB,
            \end{aligned}
        \end{equation}
        which contradicts the assumption.
        Therefore,~$\setA\cap \setB$ is the greatest lower bound of~$\setA$ and~$\setB$.
    \end{compactitem}
    We have therefore proved that~$\posLA=\tup{\setOfLowersets \posA,\subseteq}$ is a lattice.
    To show that it is bounded, we notice that~$\emptyset \subseteq \setC$ for any~$\setC\in \setOfLowersets \posA$, meaning that~$\emptyset$ is the bottom.
    Furthermore, we notice that~$\setC\subseteq \posA$ for any~$\setC\in \setOfLowersets \posA$, meaning that~$\posA$ is a top.
    Therefore, the lattice is bounded.
\end{proof}
