% !TEX root = chapter-standalone.tex


\section{Lattices}

\begin{definition}[Lattice]
  \label{def:lattice}
  A \emph{lattice} is a poset~$\tupp{\posA,\posAleq}$ with some additional properties:
  \begin{compactenum}
    \item Given two points~$\ela, \elb \in \posA$, it is always possible to define their least upper bound, called \emph{join}, and indicated as~$\ela \join \elb$.
    \item Given two points~$\ela, \elb \in \posA$, it is always possible to define their greatest lower bound, called \emph{meet}, and indicated as~$\ela \meet \elb$.
  \end{compactenum}
\end{definition}

\todo{Add the wooden lattice figure here.}

\todo{These can be defined for preorders as well... move before.}
\begin{remark}[Bounded lattices]
  \label{def:top}
  \label{def:bot}
  If there is a least upper bound for the entire lattice~$\posA$, it is called
  the \emph{top} ($\postop$). If a greatest lower bound exists it is called the \emph{bottom} ($\posbot$). If both a top and a bottom exist, we call the lattice \emph{bounded}, and denote it by~$\tup{\posA,\ordleq,\join,\meet,\posbot,\postop}$.
\end{remark}

\begin{example}
  In \cref{ex:hasseinclusion} we presented the poset arising from the power set~$\powerset \setA$ of a set~$\setA$ and ordered via subset inclusion. This is a lattice, bounded by~$\setA$ and by the empty set~$\emptyset$. Note that this lattice possesses two (dual) monoidal structures~$\tup{\powerset \setA,\subseteq,\emptyset,\cup}$ and~$\tup{\powerset \setA,\subseteq,\setA,\cap}$.
\end{example}
\begin{marginfigure}

\subfloat[]{
  \includesag{40_dpcatfig_exlattice}}

  \subfloat[]{
  \includesag{40_dpcatfig_exlattice_bis}
  }
\caption{Examples of a lattice and a non-lattice. }
\label{fig:exlattice}
\end{marginfigure}

\begin{example}
  Consider the set~$\{1,2,3,6\}$ ordered by divisibility. For instance, since 2 divides 6, we have~$2\posleq 6$.
  This is a lattice. However, the set~$\{1,2,3\}$ ordered by divisibility is not, since 2 and 3 lack a meet (\cref{fig:exlattice}).
\end{example}


\begin{gradedexercise}[\exname{UpperLowerBounds}] \label{ex:UpperLowerBounds}
Let~$\setA = \{a, b, c, d, e \}$. Give examples of the following situations using Hasse diagrams.
In each case, provide a poset structure on~$\setA$ and a subset~$\setB \subseteq \setA$ such that:
\begin{enumerate}
\item $\setB$ has a least upper bound;
\item $\setB$ has a greatest lower bound; 
\item $\setB$ has no least upper bound;
\item $\setB$ has no greatest lower bound. 
\end{enumerate}
\end{gradedexercise}

\solutionof{UpperLowerBounds}


\begin{lemma}
  \label{lem:u_bounded_lat}
  $\Up \posA$ is a bounded lattice (\cref{def:lattice}) with
  \begin{equation}
    \begin{aligned}
      \posleq_{\Up \posA}&\definedas \ \supseteq\\
      \posbot_{\Up \posA}&\definedas \posA\\
      \postop_{\Up \posA}&\definedas \emptyset\\
      \vee_{\Up \posA}&\definedas \cap\\
      \wedge_{\Up \posA}&\definedas \cup.
    \end{aligned}
\end{equation}
\end{lemma}
\begin{proof}
  Consider the poset~$\tup{\Up \posA,\supseteq}$ and~$P,Q\in \Up \posA$.
  \begin{compactitem}
    \item First, we need to show that~$P\cap Q\in \Up \posA$. One has~$P \subseteq \Up \posA$ and $Q\subseteq \Up \posA$, meaning that by definition, if~$x\in P\cap Q$, we have~$x\in P \wedge x\in Q$.
    It follows that~$x\in \Up \posA$ for all~$x\in P\cap Q$. Furthermore, we need to show that~$P\cap Q$ is the least upper bound of $P,Q$.
    Assume this is not true, meaning that  there exists a~$T\in \Up \posA$,~$T\neq P\cap Q$, such that~$P\supseteq T\supseteq P\cap Q$ and~$Q\supseteq T\supseteq P\cap Q$.
    Using the fact that intersection preserves inclusions, one has
    \begin{equation}
      \begin{aligned}
        P\cap Q &\supseteq T\cap T \supseteq P\cap Q\\
        P\cap Q &\supseteq T \supseteq P\cap Q\\
        T&= P\cap Q,
      \end{aligned}
    \end{equation}
    which contradicts the assumption. Therefore,~$P\cap Q$ is the least upper bound of~$P,Q$.
    \item Second, we need to show that~$P\cup Q\in \Up \posA$. One has~$P\subseteq \Up \posA$ and~$Q\subseteq \Up \posA$, meaning that by definition, if~$x\in P\cup Q$, we have either~$x\in P$ or~$x\in Q$.
    If~$x\in P$, then~$x\in \Up \posA$. If~$x\in Q$, then~$x\in \Up \posA$. It follows that~$x\in \Up \posA$ for all~$x\in P\cup Q$.
    Furthermore, we need to show that~$P\cup Q$ is the greatest lower bound of~$P,Q$.
    Assume this is not true, meaning that  there exists a~$T\in \Up \posA$,~$T\neq P\cup Q$, such that~$P\cup Q\supseteq T\supseteq P$ and~$P\cup Q\supseteq T\supseteq Q$.
    Using the fact that union preserves inclusions, one has
    \begin{equation}
      \begin{aligned}
      (P\cup Q)
        \cup (P\cup Q) &\supseteq T \cup T \supseteq P\cup Q\\
        P\cup Q &\supseteq T\supseteq P\cup Q\\
        T&=P\cup Q,
      \end{aligned}
    \end{equation}
    which contradicts the assumption. Therefore,~$P\cup Q$ is the greatest lower bound of~$P,Q$.
  \end{compactitem}
  We have therefore proved that~$\tup{\Up \posA,\supseteq}$ is a lattice.
  To show that it is bounded, we notice that~$\emptyset \subseteq T$ for any~$T\in \Up \posA$, meaning that~$\emptyset$ is the top.
  Furthermore, we notice that~$T\subseteq \posA$ for any~$T\in \Up \posA$, meaning that~$\posA$ is a bottom. Therefore, the lattice is bounded.
\end{proof}

\begin{lemma}
  $\posLA$ is a bounded lattice (\cref{def:lattice}) with:
  \begin{equation}
  \begin{aligned}
    \posleq_{\posLA}&\definedas \ \subseteq\\
    \posbot_{\posLA}&\definedas \emptyset\\
    \postop_{\posLA}&\definedas \posA\\
    \vee_{\posLA}&\definedas \cup\\
    \wedge_{\posLA}&\definedas \cap.
  \end{aligned}
  \end{equation}
\end{lemma}
\begin{proof}
  Consider the poset~$\tup{\posLA,\subseteq}$ and~$P,Q\in \posLA$.
  \begin{compactitem}
    \item First, we need to show that~$P\cup Q\in \posLA$. One has~$P \subseteq \posLA$ and~$Q\subseteq \posLA$, meaning that by definition, if~$x\in P\cup Q$, either~$x\in P$ or~$x\in Q$.
    If~$x\in P$, then~$x\in \posLA$. If~$x\in Q$, then~$x\in \posLA$. It follows that~$x\in \posLA$ for all~$x\in P\cup Q$.
    Furthermore, we need to show that~$P\cup Q$ is the least upper bound of~$P,Q$.
    Assume this is not true, meaning that  there exists a~$T\in \posLA$,~$T\neq P\cup Q$, such that~$P\subseteq T\subseteq P\cup Q$ and~$Q\subseteq T\subseteq P\cup Q$.
    Using the fact that union preserves inclusions, one has
    \begin{equation}
      \begin{aligned}
        P\cup Q &\subseteq T\cup T \subseteq P\cup Q\\
        P\cup Q &\subseteq T \subseteq P\cup Q\\
        T&= P\cup Q,
      \end{aligned}
    \end{equation}
    which contradicts the assumption. Therefore,~$P\cup Q$ is the least upper bound of~$P,Q$.
    \item Second, we need to show that~$P\cap Q\in \posLA$.
    One has~$P\subseteq \posLA$ and~$Q\subseteq \posLA$, meaning that by definition, if~$x\in P\cap Q$, we have~$x\in P\wedge x\in Q$ ($x\in \posLA$, for all~$x\in P\cap Q$).
    Furthermore, we need to show that~$P\cap Q$ is the greatest lower bound of~$P,Q$.
    Assume this is not true, meaning there exists a~$T\in \posLA$,~$T\neq P\cap Q$, such that~$P\cap Q\subseteq T\subseteq P$ and~$P\cap Q\subseteq T\subseteq Q$.
    Using the fact that intersection preserves inclusions, one has
    \begin{equation}
      \begin{aligned}
      (P\cap Q)
        \cap (P\cap Q) &\subseteq T \cap T \subseteq P\cap Q\\
        P\cap Q &\subseteq T\subseteq P\cap Q\\
        T&=P\cap Q,
      \end{aligned}
    \end{equation}
    which contradicts the assumption. Therefore,~$P\cap Q$ is the greatest lower bound of~$P,Q$.
  \end{compactitem}
  We have therefore proved that~$\tup{\posLA,\subseteq}$ is a lattice.
  To show that it is bounded, we notice that~$\emptyset \subseteq T$ for any~$T\in \posLA$, meaning that~$\emptyset$ is the bottom.
  Furthermore, we notice that~$T\subseteq \posA$ for any~$T\in \posLA$, meaning that~$\posA$ is a top. Therefore, the lattice is bounded.
\end{proof}