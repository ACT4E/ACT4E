% !TEX root = chapter-standalone.tex

\section{Lattices}

\linkvideo{spring2021-tradeoffs:tradeoffs:orders:top-bottom} % Top and bottom
\linkvideo{spring2021-tradeoffs:tradeoffs:orders:lattices} % Lattices

\begin{definition}[Lattice]
	\label{def:lattice}
	A \emph{lattice} is a poset~$\posA=\tupp{\posAset,\posAleq}$ with the additional property that, for any two-element subset~$\{\ela, \elb \} \subseteq \posA$, both the join~$\join \{\ela, \elb \}$ and the meet~$\meet \{\ela, \elb \}$ exist.
	Usually these are written using infix notation as~$\ela \join \elb$ and~$\ela \meet \elb$, respectively.
\end{definition}

\todographicsjira{74}{@Gioele: Add the wooden lattice figure here.}

\begin{remark}[Bounded lattices]
	\label{rem:bounded-lattices}
	\label{def:top}
	\label{def:bot}
	If there is a least upper bound for the entire lattice~$\posA$, it is called the \emph{top} ($\postop$).
	If a greatest lower bound exists it is called the \emph{bottom} ($\posbot$).
	If both a top and a bottom exist, we call the lattice \emph{bounded}, and denote it by~$\posA = \tup{\posAset,\ordleq,\join,\meet,\posbot,\postop}$.
\end{remark}

\begin{example}
	In \cref{ex:hasseinclusion} we presented the poset arising from the power set~$\powerset \setA$ of a set~$\setA$ and ordered via subset inclusion.
	This is a lattice, bounded by~$\setA$ and by the empty set~$\emptyset$.
	Note that this lattice possesses two (dual) monoidal structures~$\tup{\powerset \setA,\subseteq,\emptyset,\setunion}$ and~$\tup{\powerset \setA,\subseteq,\setA,\setintersection}$.
\end{example}

\begin{marginfigure}
	\centering
	\subfloat[]{
		\centering
		\includesag{40_dpcatfig_exlattice}}

	\subfloat[]{
		\centering
		\includesag{40_dpcatfig_exlattice_bis}
	}
	\caption{Examples of a lattice and a non-lattice. }
	\label{fig:exlattice}
\end{marginfigure}

\begin{example}
	Consider the set~$\{1,2,3,6\}$ ordered by divisibility.
	For instance, since 2 divides 6, we have~$2\posleq 6$.
	This is a lattice.
	However, the set~$\{1,2,3\}$ ordered by divisibility is not, since 2 and 3 lack a meet (\cref{fig:exlattice}).
\end{example}

\begin{lemma}
	\label{lem:u_bounded_lat}
	$\Up \posA$ is a bounded lattice (\cref{def:lattice}) with
	\begin{equation}
		\begin{aligned}
			\posleqof{\Up \posA} & \definedas \ \supseteq, \\
			\posbot_{\Up \posA}  & \definedas \posA,       \\
			\postop_{\Up \posA}  & \definedas \emptyset,   \\
			\meet_{\Up \posA}     & \definedas \setintersection,        \\
			\join_{\Up \posA}   & \definedas \setunion.
		\end{aligned}
	\end{equation}
\end{lemma}
\begin{proof}
	Consider the poset~$\uppersets \posA=\tup{\setOfUppersets \posA,\supseteq}$ and~$\setA,\setB\in \setOfUppersets \posA$.

	First, we need to show that~$\setA\setintersection \setB\in \setOfUppersets \posA$.
	One has~$\setA \subseteq \setOfUppersets \posA$ and $\setB\subseteq \setOfUppersets \posA$, meaning that by definition, if~$\setAel\in \setA\setintersection \setB$, we have~$\setAel \in \setA \booland \setAel \in \setB$.
	It follows that~$\setAel \in \setOfUppersets \posA$ for all~$\setAel \in \setA\setintersection \setB$.
	Furthermore, we need to show that~$\setA\setintersection \setB$ is the least upper bound of~$\setA$ and~$\setB$.
	Assume this is not true, meaning that there exists a~$\setC\in \setOfUppersets \posA$,~$\setC\neq \setA\setintersection \setB$, such that~$\setA\supseteq \setC\supseteq \setA\setintersection \setB$ and~$\setB\supseteq \setC\supseteq \setA\setintersection \setB$.
	\todotext{@Gioele: Add parentheses in formulas }
	Using the fact that intersection preserves inclusions, one has
	\begin{equation}
		\begin{aligned}
			\setA\setintersection \setB & \supseteq \setC\setintersection \setC \supseteq \setA\setintersection \setB \\
			\setA\setintersection \setB & \supseteq \setC \supseteq \setA\setintersection \setB           \\
			\setC           & = \setA\setintersection \setB,
		\end{aligned}
	\end{equation}
	which contradicts the assumption.
	Therefore,~$\setA\setintersection \setB$ is the least upper bound of~$\setA$ and~$\setB$.

	Second, we need to show that~$\setA\setunion \setB\in \setOfUppersets \posA$.
	One has~$\setA\subseteq \setOfUppersets \posA$ and~$\setB \subseteq \setOfUppersets \posA$, meaning that by definition, if~$\setAel\in \setA\setunion \setB$, we have either~$\setAel\in \setA$ or~$\setAel\in \setB$.
	If~$\setAel\in \setA$, then~$\setAel\in \setOfUppersets \posA$.
	If~$\setAel\in \setB$, then~$\setAel \in \setOfUppersets \posA$.
	It follows that~$\setAel \in \setOfUppersets \posA$ for all~$\setAel\in \setA\setunion \setB$.
	Furthermore, we need to show that~$\setA\setunion \setB$ is the greatest lower bound of~$\setA$ and~$\setB$.
	Assume this is not true, meaning that there exists a~$\setC\in \setOfUppersets \posA$,~$\setC\neq \setA\setunion \setB$, such that~$\setA\setunion \setB\supseteq \setC\supseteq \setA$ and~$\setA\setunion \setB\supseteq \setC\supseteq \setB$.
	Using the fact that union preserves inclusions, one has
	\begin{equation}
		\begin{aligned}
			(\setA\setunion \setB)
			\setunion (\setA\setunion \setB) & \supseteq \setC \setunion \setC \supseteq \setA\setunion \setB \\
			\setA\setunion \setB        & \supseteq \setC\supseteq \setA\setunion \setB             \\
			\setC                  & =\setA\setunion \setB,
		\end{aligned}
	\end{equation}
	which contradicts the assumption.
	Therefore,~$\setA\setunion \setB$ is the greatest lower bound of~$\setA$ and~$\setB$.

	We have therefore proved that~$\uppersets \posA=\tup{\setOfUppersets \posA,\supseteq}$ is a lattice.
	To show that it is bounded, we notice that~$\emptyset \subseteq \setC$ for any~$\setC\in \setOfUppersets \posA$, meaning that~$\emptyset$ is the top.
	Furthermore, we notice that~$\setC\subseteq \posA$ for any~$\setC\in \setOfUppersets \posA$, meaning that~$\posA$ is a bottom.
	Therefore, the lattice is bounded.
\end{proof}

\begin{lemma}
	$\posLA$ is a bounded lattice (\cref{def:lattice}) with:
	\begin{equation}
		\begin{aligned}
			\posleqof{\posLA} & \definedas \ \subseteq, \\
			\posbot_{\posLA}  & \definedas \emptyset,   \\
			\postop_{\posLA}  & \definedas \posA,       \\
			\meet_{\posLA}     & \definedas \setunion,        \\
			\join_{\posLA}   & \definedas \setintersection.
		\end{aligned}
	\end{equation}
\end{lemma}
\begin{proof}
	Consider the poset~$\posLA=\tup{\setOfLowersets \posA,\subseteq}$ and~$\setA,\setB\in \setOfLowersets \posA$.

	First, we need to show that~$\setA\setunion \setB\in \setOfLowersets \posA$.
	One has~$\setA \subseteq \setOfLowersets \posA$ and~$\setB\subseteq \setOfLowersets \posA$, meaning that by definition, if~$\setAel\in \setA\setunion \setB$, either~$\setAel\in \setA$ or~$\setAel\in \setB$.
	If~$\setAel\in \setA$, then~$\setAel\in \setOfLowersets \posA$.
	If~$\setAel\in \setB$, then~$\setAel\in \setOfLowersets \posA$.
	It follows that~$\setAel\in \setOfLowersets \posA$ for all~$\setAel\in \setA\setunion \setB$.
	Furthermore, we need to show that~$\setA\setunion \setB$ is the least upper bound of~$\setA$ and~$\setB$.
	Assume this is not true, meaning that there exists a~$\setC\in \setOfLowersets \posA$,~$\setC\neq \setA\setunion \setB$, such that~$\setA\subseteq \setC\subseteq \setA\setunion \setB$ and~$\setB\subseteq \setC\subseteq \setA\setunion \setB$.
	Using the fact that union preserves inclusions, one has
	\begin{equation}
		\begin{aligned}
			\setA\setunion \setB & \subseteq \setC\setunion \setC \subseteq \setA\setunion \setB \\
			\setA\setunion \setB & \subseteq \setC \subseteq \setA\setunion \setB           \\
			\setC           & = \setA\setunion \setB,
		\end{aligned}
	\end{equation}
	which contradicts the assumption.
	Therefore,~$\setA\setunion \setB$ is the least upper bound of~$\setA$ and~$\setB$.

	Second, we need to show that~$\setA \setintersection \setB\in \setOfLowersets \posA$.
	One has~$\setA \subseteq \setOfLowersets \posA$ and~$\setB\subseteq \setOfLowersets \posA$, meaning that by definition, if~$\setAel\in \setA\setintersection \setB$, we have~$\setAel \in \setA\booland \setAel\in \setB$ ($\setAel\in \setOfLowersets \posA$, for all~$\setAel\in \setA \setintersection \setB$).
	Assume this is not true, meaning there exists a~$\setC\in \setOfLowersets \posA$,~$\setC\neq \setA\setintersection \setB$, such that~$\setA\setintersection \setB\subseteq \setC\subseteq \setA$ and~$\setA\setintersection \setB\subseteq \setC\subseteq \setB$.
	Using the fact that intersection preserves inclusions, one has
	\begin{equation}
		\begin{aligned}
			(\setA\setintersection \setB)
			\setintersection (\setA\setintersection \setB) & \subseteq \setC \setintersection \setC \subseteq \setA\setintersection \setB \\
			\setA\setintersection \setB        & \subseteq \setC\subseteq \setA\setintersection \setB             \\
			\setC                  & =\setA\setintersection \setB,
		\end{aligned}
	\end{equation}
	which contradicts the assumption.
	Therefore,~$\setA\setintersection \setB$ is the greatest lower bound of~$\setA$ and~$\setB$.

	We have therefore proved that~$\posLA=\tup{\setOfLowersets \posA,\subseteq}$ is a lattice.
	To show that it is bounded, we notice that~$\emptyset \subseteq \setC$ for any~$\setC\in \setOfLowersets \posA$, meaning that~$\emptyset$ is the bottom.
	Furthermore, we notice that~$\setC\subseteq \posA$ for any~$\setC\in \setOfLowersets \posA$, meaning that~$\posA$ is a top.
	Therefore, the lattice is bounded.
\end{proof}

\vfill

\begin{gradedexercise}[\exname{UpperLowerBounds}]
	\label{ex:UpperLowerBounds}
	Let~$\setA = \{\setAel, \setBel, \setCel, \setDel, \setEel \}$.
	Give examples of the following situations using Hasse diagrams.
	In each case, provide a poset structure on~$\setA$ and a subset~$\setB \subseteq \setA$ such that:
	\begin{enumerate}
		\item $\setB$ has a least upper bound;
		\item $\setB$ has a greatest lower bound;
		\item $\setB$ has no least upper bound;
		\item $\setB$ has no greatest lower bound.
	\end{enumerate}
\end{gradedexercise}

\solutionof{UpperLowerBounds}

\todojira{465}{G: Category lat, introduce lattice homomorphisms etc.}
