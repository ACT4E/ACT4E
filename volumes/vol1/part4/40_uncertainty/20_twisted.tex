\section{Intervals as categories}
\subsection{Twisted arrow category}
\begin{ctdefinition}[Twisted arrow category]
  \label{def:twisted-arrow}
  Given a category \CatC, we denote its \emph{twisted arrow category} by $\twisted{\CatC}$. This is a category which is composed of:
  \begin{compactenum}
    \item \emph{Objects:} Arrows (morphisms) in \CatC.
    \item \emph{Morphisms:} A morphism between two arrows $\mora\colon \Obja \cto \Objb $, $\morb\colon \Objc \cto \Objd$ is given by the pair of arrows $\tup{\morc,\mord}$ such that the following diagram commutes:
    \begin{center}
      \includesag{180_twistedarrow}
    \end{center}
  \end{compactenum}
\end{ctdefinition}



\begin{example}[Intervals]
  Consider a poset~$\posA$. The twisted arrow category~$\twisted{\posA}$ is isomorphic to the set of nonempty \emph{intervals}~$[a,b]=\{p \in \posA \mid a\posAleq p \posAleq b\}$. Note that~$\twisted{\posA}$ is a poset as well, ordered by inclusion.
\end{example}
\begin{remark}
  Recall \cref{sec:posetsarecats} and note that the map which sends a poset (a category) to its twisted arrow category is a functor, which sends objects of the poset
\end{remark}

\subsection{Arrow category}

\todotext{to write}
