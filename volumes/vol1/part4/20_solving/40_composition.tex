% !TEX root = standalone.tex


\section{Composition operators for design problems}

This section defines a handful of composition operators for design
problems. Later, \cref{sec:Decomposing2} will prove that any co-design
problem can be described in terms of a subset of these operators.

\label{sec:threeoperators}

\begin{definition}[$\dppar$]
  \label{def:parallel}
  The parallel composition of two DPIs $\dprob_{1}=\left\langle \funsp_{1},\ressp_{1},\impsp_{1},\prov_{1},\req_{1}\right\rangle $
  and $\dprob_{2}=\langle\funsp_{2},\ressp_{2},\impsp_{2},\prov_{2},$
  $\req_{2}\rangle$ is
  \[
    \dppar(\dprob_{1},\dprob_{2})\doteq\left\langle \funsp_{1}\times\funsp_{2},\ressp_{1}\times\ressp_{2},\impsp_{1}\times\impsp_{2},\prov,\req\right\rangle ,
  \]
  where:
  \begin{eqnarray}
    \prov & : & \left\langle \imp_{1},\imp_{2}\right\rangle \mapsto\left\langle \prov_{1}(\imp_{1}),\prov_{2}(\imp_{2})\right\rangle ,\label{eq:dppar-exec}\\
    \req & : & \left\langle \imp_{1},\imp_{2}\right\rangle \mapsto\left\langle \req_{1}(\imp_{1}),\req_{2}(\imp_{2})\right\rangle .\nonumber
  \end{eqnarray}

\end{definition}
\captionsideleft{\label{fig:dppar}}{
  \includegraphics[scale=0.33]{gmcdp_parallel2}
}


\begin{definition}[$\dploop$]
  \label{def:dp_loop}Suppose~$\dprob$ is a DPI with factored functionality
  space~$\funsp_{1}\times\ressp$:
  \[
    \dprob=\left\langle \funsp_{1}\times\ressp,\ressp,\impsp,\left\langle \prov_{1},\prov_{2}\right\rangle ,\req\right\rangle.
  \]
  Then we can define the DPI~$\dploop(\dprob)$ as
  \[
    \dploop(\dprob)\doteq\left\langle \funsp_{1},\ressp,\impsp',\prov_{1},\req\right\rangle ,
  \]
  where~$\impsp'\subseteq\impsp$ limits the implementations to those
  that respect the additional constraint~$\req(\imp)\posleq\prov_{2}(\imp)$:
  \[
    \impsp'=\{\imp\in\impsp:\req(\imp)\posleq\prov_{2}(\imp)\}.
  \]
  This is equivalent to ``closing a loop'' around~$\dprob$ with
  the constraint~$\fun_{2}\posgeq\res$~(\cref{fig:sloop}).
\end{definition}

\captionsideleft{\label{fig:sloop}\label{fig:sloop2}}{
  \includegraphics[scale=0.33]{gmcdp_sloop2}
}

The operator~$\dploop$ is asymmetric because it acts on a design
problem with 2 functionalities and 1 resources. We can define a symmetric
feedback operator $\dploopb$ as in \cref{fig:loop_general}, which
can be rewritten in terms of $\dploop$, using the construction in~\cref{fig:loop_general2}\emph{.}

\begin{figure}[h]
  \hspace*{\fill}
  \subfloat[\label{fig:loop_general}]{
    \includegraphics[scale=0.33]{gmcdp_loop_general}

  }
  \hspace*{\fill}
  \subfloat[\label{fig:loop_general2}]{
    \includegraphics[scale=0.33]{gmcdp_loop_general2}
  }
  \hspace*{\fill}
  \caption{A symmetric operator $\dploopb$ can be defined in terms of $\dploop$.}
\end{figure}
