Both sample exams are thought to last 90 minutes and to be open book (all notes allowed).
\section*{Example 1: Uncertain Machines}
Consider the category of Moore machines acting on signals as described in the course.
Assuming the input/output sets are also ordered sets (posets), construct the Kleisli category corresponding to the interval monad~$\Umon$.
That is, the signals are closed intervals of values.
Call this category $\Cat{UMoore}$.
\begin{enumerate}
    \item (30\%) Is $\Cat{UMoore}$ a monoidal category?
    \item (20\%) Is $\Cat{UMoore}$ a traced monoidal category?
    \item (50\%) Would the answers be the same if asked about the More category?
\end{enumerate}

\section*{Example 2: Machines with resources consumption}
Consider the category of Moore machines.
We want to be more precise about resource consumption and want to define an extension of Moore machines in which each machine also has associated a certain time~$T_1\geq 0$ to run the dynamic function~$\prdyn$ and a certain time~$T_2\geq 0$ for running the readout function~$\prreadout$.
We call these resource-Moore machines ($\Cat{RMoore}$).

\begin{enumerate}
    \item (30\%) Formalize the $\Cat{RMoore}$ category giving formulas for identities, compositions, and proof of associativity.
    \item (70\%) Suppose that you want to design the software for a robot.
    You are given the wiring diagram of the architecture, in which you have to plug in specific $\Cat{RMoore}$ machines to implement the algorithmic functionality.
    Assume that the wiring diagram does not contain any loop - only series and parallel composition, and that there is only one input (robot observations) and one output (robot commands).
    For each hole in the diagram, you are given a set of 1 or more $\Cat{RMoore}$ machines that can implement the functionality.
    Assume that the computer on which to run everything has~$N\geq 1$ processors.
Think of each processor as a ``lane'' in which the operations of each machine are cars that must run sequentially.
    
    Formalize the design problem in the category \DPI that corresponds to choosing the best combination of machines and the best assignment to processors.
    Include as a resource the number of processors and as functionality the throughput of the system (how many commands are generated per second).
\end{enumerate}

