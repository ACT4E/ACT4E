\section{Queries are functors from problem statements to solutions}

In this and the following chapters we are going to build towards the solution of co-design problems.
We will consider an arbitrary graph of \SY{design problems}, in which nodes are \SY{design problems} and edges are arbitrary interconnections between functionality and resources, obtained through the operations of a \SY{traced monoidal category} (series, parallel, feedback) plus the \SY{lattice} structure (and, or) of \SY{design problems}.
On this structure we want to solve the query \FixFunMinRes (\cref{prob:FixFunMinRes}) or, symmetrically, \FixResMaxFun (\cref{prob:FixResMaxFun})

We look at this from a compositional point of view.
We will assume that we know the solution to \FixFunMinRes for each of the components.
We think of the components as primitive blocks, because they are given in a catalogue format as a DPI, or they are special cases ($+$, $\mtimescat$, \etc) which we will solve as special cases.
Given the solution for the primitive blocks, we want to know what is the solution for \FixFunMinRes for the entire diagram.

What is the form of the solution that we expect?
Given a DP~$\adp\colon \funsp \profto \ressp$ we expect the solution to \FixFunMinRes to be a function that, given a fixed functionality~$\fun\setin\funsp$, returns the minimal resources, which form an \SY{upper set}.
We call this function~$\ftoR_\adp$.

\begin{definition}
    \label{def:ftoR-dp}
    Given a DP~$\adp\colon\funsp\profto\ressp$ we denote by~$\ftoR_{\adp}\colon \funsp\toinPos \uppersets{\ressp}$ the map that associates to each functionality~\fun the set of minimal resources sufficient to realize~\fun:
    \begin{equation}
        \begin{aligned}
            \ftoR_{\adp}\colon \funsp & \toinPos \uppersets{\ressp}, \\
            \fun                      & \mapsto \makeset{ \res \setin \ressp \colon \adp(\fun\Fop, \res)}.
        \end{aligned}
    \end{equation}
    If a certain functionality~\fun is infeasible, then~$\ftoR(\fun)=\Emptyset$.
\end{definition}

\begin{remark}[Monotonicity]
    Consider a DP~$\adp\colon\funsp\profto\ressp$ and~$\fun\posleq \fun'$.
    We know
    \begin{equation}
        \begin{aligned}
            \ftoR_\adpa(\fun) & =\makeset{ \res \setin \ressp \colon \adp(\fun\Fop, \res)} \\
                              & \setsupseteq \makeset{ \res \setin \ressp \colon \adp(\fun\F{'^*}, \res)} \\
                              & =\ftoR_\adpa(\fun\F{'}),
        \end{aligned}
    \end{equation}
    showing monotonicity.
\end{remark}

Symmetrically, the solution to \FixResMaxFun is given by a function that we call~$\rtoF_\adp$.

\begin{definition}
    \label{def:rtoF-dp}
    Given a DP~$\tup{\funsp,\ressp,\impsp,\prov,\req}$, define the map~$\rtoF_{\adp}\colon \ressp\toinPos \lowersets{\funsp}$ that associates to each resource~\res the set of functionalities which can be realized with~$\res$:
    \begin{equation}
        \begin{aligned}
            \rtoF_{\adp}\colon \ressp & \toinPos \lowersets{\funsp}, \\
            \res                      & \mapsto \makeset{ \fun \setin \funsp \colon \adp(\fun\Fop, \res)}.
        \end{aligned}
    \end{equation}
    If a certain resource~\res only leads to infeasible functionalities, then~$\rtoF(\res)=\Emptyset$.
\end{definition}

\begin{remark}[Monotonicity]
    Consider a DP~$\adp\colon\funsp\profto\ressp$ and~$\res\posleq \res\R{'}$.
    We know
    \begin{equation}
        \begin{aligned}
            \rtoF_\adpa(\res) & =\makeset{ \fun \setin \funsp \colon \adp(\fun\Fop, \res)} \\
                              & \setsubseteq \makeset{ \fun \setin \funsp \colon \adp(\fun\Fop, \res\R{'})} \\
                              & =\rtoF_\adpa(\res\R{'}),
        \end{aligned}
    \end{equation}
    showing monotonicity.
\end{remark}

A question that arises naturally is whether the map~$\ftoR_\adp$ is sufficient to reconstruct the original DP.
The answer is yes.
We will prove that~$\ftoR_\adp$ defines a morphism in a category called~\UPos, and that this category is equivalent (\cref{def:cat-equivalence}) to~\DP, therefore being traced monoidal, with a \SY{lattice} structure.
In fact, \FixFunMinRes can be seen as a \SY{functor} from~\DP to~\UPos.
Symmetrically,~$\rtoF_\adp$ is a morphism in a category \LPos equivalent to \DP and \FixResMaxFun can be seen as the \SY{functor} from \DP to \LPos.

This situation is represented in \cref{fig:upos_lpos_dp}.

\begin{figure}[tbh]
    \centering
    \includesag{upos_lpos_dp}
    \caption{In this chapter, we show that the queries \FixResMaxFun and \FixFunMinRes
        can be seen as \SY{functors} from \DP to two new categories, \UPos and \LPos.
        We show that \DP is equivalent to these categories: a \DP is univocally
        defined by the answers to the two queries.
    }
    \label{fig:upos_lpos_dp}
\end{figure}

In the course of this chapter, by defining the two \SY{functors}\FixFunMinRes and\linebreak[0] \FixResMaxFun, we effectively have solved the problem of optimization for DPs in the ``mathematical'' way.
However, this is only the first step, because it does not say anything about whether the \SY{functor} is actually computable.
In the next chapter (\cref{chap:solving}) we will look at finite approximations of DPs and the computational complexity of the solution.
Then, we will introduce the theory of \SY{monads}, and based on that, we will be able to show how to construct bounded finite approximations of any DPs.
