% !TEX root = chapter-standalone.tex

\section{The $\UPos$ and~$\LPos$ categories}

\begin{definition}[Category \UPos]
    \label{def:upos_cat}
    The category \UPos consists of:
    \begin{enumerate}
        \item \emph{Objects}: objects are posets;
        \item \emph{Morphisms}: given objects~$\Obja,\Objb\setin \Obof{\UPos}$, morphisms from~$\mora\colon \Obja \mto \Objb$ are \SY{monotone maps} of the form~$\ulposmap{\mapa} \colon \Obja \toinPos \uppersets{\Objb}$.
        \item \emph{Composition of morphisms}: Given morphisms~$\mora\colon \Obja \mto \Objb$,~$\morb\colon \Objb\mto \Objc$, their composition~$\morab\colon \Obja\mto \Objc$ is given by
              \begin{equation}
                  \begin{aligned}
                      \ulposmap{(\mapa \mthen \mapb)} \colon \Obja & \toinPos \uppersets{\Objc} \\
                      \Objael                                      & \mapsto \bigsetunion_{\Objbel\setin \ulposmap{\mapa}(\Objael)}\ulposmap{\mapb}(\Objbel);
                  \end{aligned}
              \end{equation}
        \item \emph{Identity morphism}: given an object~$\Obja\setin \Obof{\UPos}$, the \SY{identity morphism} $\catidat\Obja \colon \Obja \mto \Obja$ is given by the application of the \SY{upper closure} operator:
              \begin{equation}
                  \ulposmap{\catid}_\Obja(\Objael)\definedas \upit\makeset{\Objael}.
              \end{equation}
    \end{enumerate}
\end{definition}

\begin{remark}
    Note that the composition of morphisms in this category corresponds to the generalization of the series operator for boolean \SY{profunctors}.
\end{remark}

Analogously, we can define the \LPos category.
\begin{definition}[Category \LPos]
    \label{def:lpos_cat}
    The category \LPos consists of:
    \begin{enumerate}
        \item \emph{Objects}: objects are posets;
        \item \emph{Morphisms}: given objects~$\Obja,\Objb\setin \Obof{\LPos}$, morphisms~$\mora\colon \Obja\mto \Objb$ are \SY{monotone maps} of the form~$\ulposmap{\mora} \colon \Obja \toinPos \lowersets{\Objb}$.
        \item \emph{Composition of morphisms}: Given morphisms~$\mora \colon \Obja \mto \Objb$,~$\morb \colon \Objb\mto \Objc$, their composition~$\morab\colon \Obja\mto \Objc$ is given by
              \begin{equation}
                  \begin{aligned}
                      \ulposmap{(\mapa \mthen \mapb)} \colon \Obja & \toinPos \lowersets{\Objc} \\
                      \Objael                                      & \mapsto \bigsetunion_{\Objbel\setin \ulposmap{\mora} (\Objael)}\ulposmap{\morb}(\Objbel);
                  \end{aligned}
              \end{equation}
        \item \emph{Identity morphism}: given an object $\Obja\setin \Obof{\LPos}$, the \SY{identity morphism} $\catidat\Obja \colon \Obja \mto \Obja$ is given by the application of the \SY{lower closure} operator:
              \begin{equation}
                  \ulposmap{\catid}_\Obja(\Objael)\definedas \downit\makeset{\Objael}.
              \end{equation}
    \end{enumerate}
\end{definition}

We now show that \UPos and \LPos are indeed categories.

\begin{lemma}
    \label{lem:upos_lpos_cats}
    \UPos and \LPos are categories.
\end{lemma}

\begin{proof}
    We prove that \UPos is a category.
    The proof for \LPos is analogous.
    In the following, we show unitality and associativity.

    \textbf{Unitality}:
    Given~$\mora\colon \Obja \mto \Objb$, we have:
    \begin{equation}
        \begin{aligned}
            \ulposmap{\pars{ \mora \mthen \catidat\Objb}}(\Objael) & =\bigsetunion_{\Objbel\setin \ulposmap{\mora}(\Objael)}\ulposmap{\catidat\Objb}(\Objbel) \\
                                                                   & =\bigsetunion_{\Objbel\setin \ulposmap{\mora}(\Objael)}\upit\makeset{\Objbel} \\
                                                                   & =\bigsetunion_{\Objbel\setin \ulposmap{\mora}(\Objael)}\makeset{\Objbel'\setin \Objb \colon \Objbel\posleqof\Objb \Objbel' }.
        \end{aligned}
    \end{equation}
    We know that~$\ulposmap{\mora}(\Objael)$ is an \SY{upper set}:
    \begin{equation}
        \begin{aligned}
            \ulposmap{\mora}(\Objael) & =\bigsetunion_{\Objbel\setin \ulposmap{\mora}(\Objael)}\makeset{\Objbel} \\
                                      & =\bigsetunion_{\Objbel\setin \ulposmap{\mora}(\Objael)}\makeset{ \Objbel'\setin \Objb \colon \Objbel\posleqof\Objb \Objbel'}.
        \end{aligned}
    \end{equation}
    Therefore,~$\ulposmap{\pars{ \mora \mthen \catidat\Objb}}(\Objael)=\ulposmap{\mora}(\Objael)$ for all~$\Objael\setin \Obja$.
    Similarly, we have:
    \begin{equation}
        \begin{aligned}
            \ulposmap{(\catidat\Obja \mthen \mora)}(\Objael) & =\bigsetunion_{\Objael'\setin \ulposmap{\catidat\Obja}(\Objael)}\ulposmap{\mora}(\Objael') \\
                                                             & =\bigsetunion_{\Objael'\setin \upit \makeset{\Objael}}\ulposmap{\mora}(\Objael') \\
                                                             & =\ulposmap{\mora}(\Objael),
        \end{aligned}
    \end{equation}
    where the last equality holds since~$\ulposmap{\mora}$ is a \SY{monotone function} and~$\ulposmap{\mora}(\Objael')\setsubseteq \ulposmap{\mora}(\Objael)$ for all~$\Objael'\setin \upit \makeset{\Objael}$.

    \textbf{Associativity}:
    Consider three morphisms~$\mora\colon \Obja \mto \Objb$, $\morb\colon \Objb\mto \Objc$, and $\morc \colon \Objc\mto \Objd$.
    We have:
    \begin{equation}
        \begin{aligned}
            \ulposmap{\pars{ (\morab) \mthen \morc}}(\Objael) & =
            \bigsetunion_{\Objcel \setin \pars{ \bigsetunion_{\Objbel \setin \ulposmap{\mora}(\Objael)}\ulposmap{\morb}(\Objbel)}}\ulposmap{\morc}(\Objcel) \\
                                                              & =\bigsetunion_{\Objbel\setin \ulposmap{\mora}(\Objael)}\bigsetunion_{\Objcel\setin \ulposmap{\morb}(\Objbel)}\ulposmap{\morc}(\Objcel) \\
                                                              & =\ulposmap{\pars{ \mora \mthen \pars{ \morb\mthen \morc}}}(\Objael).
        \end{aligned}
    \end{equation}
    Therefore, \UPos is a category.
\end{proof}

We can show that \UPos and \LPos are equivalent categories (\cref{def:cat-equivalence}).

\begin{lemma}
    \label{lem:ulposequiv}
    \UPos and \LPos are isomorphic: there exists a pair of \SY{functors} \begin{equation}
        \begin{aligned}
            \funeqa\colon \UPos & \fto \LPos, \\
            \funeqb\colon \LPos & \fto \UPos,
        \end{aligned}
    \end{equation}
    such that~$\funeqa \fthen \funeqb=\funid_{\UPos}$ and~$\funeqb \fthen \funeqa=\funid_{\LPos}$, where~$\funid_{\UPos}$ and~$\funid_{\LPos}$ are the \SY{identity functors} on \UPos and~\LPos, respectively.
\end{lemma}

\begin{proof}
    To prove this, we need to define the needed \SY{functors} and to show that they satisfy the listed properties.
    We choose the \SY{functors} to be the ones that map a poset~\posA in a category to its opposite version~$\posAop$ in another category.
    Given a morphism~$\mora\colon \Obja \mto \Objb$ in \UPos, we have:
    \begin{equation}
        \begin{aligned}
            \ulposmap{(\funeqa(\mora))}\colon \Obja\op & \toinPos \lowersets{\Objb}\op \\
            \Objael                                    & \mapsto \ulposmap{\mora}(\Objael).
        \end{aligned}
    \end{equation}
    Given a morphism~$\morb\colon \Obja \mto \Objb$ in \LPos, we have:
    \begin{equation}
        \begin{aligned}
            \ulposmap{(\funeqb(\morb))}\colon \Obja\op & \toinPos \uppersets{\Objb}\op \\
            \Objael                                    & \mapsto \ulposmap{\morb}(\Objael).
        \end{aligned}
    \end{equation}

    \textbf{$\funeqa$ and~$\funeqb$ are \SY{functors}}:
    \begin{itemize}
        \item \emph{Preservation of identities}: Given~$\Obja\setin \Obof{\UPos}$, we have:
              \begin{equation}
                  \begin{aligned}
                      \ulposmap{(\funeqa(\catidat\Obja))} & =\upit_\Obja\makeset{\Objael} \\
                                                          & =\downit_{\Obja\op}\makeset{\Objael} \\
                                                          & =\ulposmap{\catidat{\Obja\op}},
                  \end{aligned}
              \end{equation}
              where~$\catidat\Obja$ is an \SY{identity morphism} in \UPos, and~$\catidat{\Obja\op}$ is an \SY{identity morphism} in \LPos.
              Similarly, given~$\Obja\setin \Obof{\LPos}$ we have:
              \begin{equation}
                  \begin{aligned}
                      \ulposmap{(\funeqb(\catidat\Obja))} & =\downit_\Obja\makeset{\Objael} \\
                                                          & =\upit_{\Obja\op}\makeset{\Objael} \\
                                                          & =\ulposmap{\catid}_{\Obja\op}.
                  \end{aligned}
              \end{equation}
        \item \emph{Preservation of composition}: This can be easily seen as follows.
              Given any~$\mora\setin \HomSet{\UPos}{\Obja}{\Objb}$,~$\morb\setin \HomSet{\UPos}{\Objb}{\Objc}$:
              \begin{equation}
                  \begin{aligned}
                      \ulposmap{(\funeqa(\morab))} & =\ulposmap{(\morab)} \\
                                                   & =\ulposmap{(\funeqa(\mora)\mthen \funeqa(\morb))}.
                  \end{aligned}
              \end{equation}
              Similarly, given any~$\mora\setin \HomSet{\LPos}{\Obja}{\Objb}$,~$\morb\setin \HomSet{\LPos}{\Objb}{\Objc}$:
              \begin{equation}
                  \begin{aligned}
                      \ulposmap{(\funeqb(\morab))} & =\ulposmap{(\morab)} \\
                                                   & =\ulposmap{(\funeqb(\mora)\mthen \funeqb(\morb))}.
                  \end{aligned}
              \end{equation}
    \end{itemize}

    \textbf{Compositions return \SY{identity functors}}:
    We want to show that by composing the two \SY{functors} we obtain the \SY{identity functors} in \UPos and \LPos, respectively.
    Clearly, composing the two \SY{functors} returns the identity on the objects, since for any poset~\posA, we have~$(\posAop)\op=\posA$.
    The \SY{functors} act on morphisms by ``flipping the context'', and ``flipping'' twice is the ``same'' as not flipping.
\end{proof}

We can show that~\UPos and~\LPos are \SY{monoidal categories}.

\begin{lemma}
    \label{lem:upos_moncat}
    \UPos is a \SY{monoidal category} with the following additional structure:
    \begin{enumerate}
        \item \emph{Tensor product~$\mtimescat$}: On objects, the tensor product corresponds to the product of \SY{posets}.
              Given two morphisms~$\mora\colon \Obja\mto \Objb$ and~$\morb\colon \Objc\mto \Objd$, we have~$\mora \mtimescatmor \morb\colon \Obja \cartprod \Objc \mto \Objb\cartprod \Objd$, with
              \begin{equation}
                  \begin{aligned}
                      \ulposmap{(\mora\mtimescatmor \morb)}\colon \Obja \cartprod \Objc & \toinPos \uppersets{(\Objb\cartprod \Objd)} \\
                      \tup{\Objael,\Objcel}                                             & \mapsto \ulposmap{\mora}(\Objael)\cartprod \ulposmap{\morb}(\Objcel).
                  \end{aligned}
              \end{equation}
              Note that the \SY{cartesian product} of \SY{upper sets} is an \SY{upper set}.

        \item \emph{Unit}: The unit is the identity poset: the \SY{poset} with a singleton carrier set and only the identity relation.
              We denote this by~$\singleton$.
        \item \emph{Left unitor}: The left unitor is given by the pair of morphisms~$\leftunitor_\Obja\colon \singletonobj\cartprod \Obja \mto \Obja$ and~$\leftunitor_\Obja^{-1}\colon \Obja \mto \singletonobj\cartprod \Obja$, with
              \begin{equation}
                  \begin{aligned}
                      \ulposmap{\leftunitor}_\Obja\colon \singletonobj\cartprod \Obja & \toinPos \Up{\Obja} \\
                      \tup{\styleobj{\singletonel},\Objael}                           & \mapsto \upit\makeset{\Objael},
                  \end{aligned}
              \end{equation}
              and
              \begin{equation}
                  \begin{aligned}
                      \ulposmap{\leftunitor_{\Obja}^{-1}}\colon \Obja & \toinPos \Up(\singletonobj\cartprod \Obja) \\
                      \Objael                                         & \mapsto \singletonobj \cartprod \upit \makeset{\Objael},
                  \end{aligned}
              \end{equation}
              respectively.
        \item \emph{Right unitor}: The right unitor is given by the pair of morphisms~$\rightunitor_\Obja\colon \Obja\cartprod \singletonobj  \mto \Obja$ and~$\rightunitor_\Obja^{-1}\colon \Obja \mto \Obja \cartprod \singletonobj$, with
              \begin{equation}
                  \begin{aligned}
                      \ulposmap{\rightunitor}_\Obja\colon \Obja\cartprod \singletonobj & \toinPos \Up{\Obja} \\
                      \tup{\Objael,\styleobj{\singletonel}}                            & \mapsto \upit\makeset{\Objael},
                  \end{aligned}
              \end{equation}
              and
              \begin{equation}
                  \begin{aligned}
                      \ulposmap{\rightunitor_{\Obja}^{-1}}\colon \Obja & \toinPos \Up( \Obja \cartprod \singletonobj) \\
                      \Objael                                          & \mapsto \upit \makeset{\Objael} \cartprod \singletonobj,
                  \end{aligned}
              \end{equation}
              respectively.
        \item \emph{Associator}: The associator is given by the pair of morphisms~$\associator_{\Obja\Objb,\Objc}\colon (\Obja\cartprod \Objb)\cartprod \Objc \mto \Obja\cartprod ( \Objb \cartprod \Objc)$ and~$\associator_{\Obja,\Objb\Objc}\colon \Obja\cartprod (\Objb\cartprod \Objc) \mto (\Obja\cartprod \Objb) \cartprod \Objc$, given by
              \begin{equation}
                  \begin{aligned}
                      \ulposmap{\associator_{\Obja\Objb,\Objc}}\colon (\Obja\cartprod \Objb)\cartprod \Objc & \toinPos \Up \Obja\cartprod (\Up \Objb \cartprod \Up \Objc) \\
                      \tup{\tup{\Objael,\Objbel},\Objcel}                                                   & \mapsto \upit\makeset{\Objael}\cartprod (\upit\makeset{\Objbel}\cartprod \upit\makeset{\Objcel}),
                  \end{aligned}
              \end{equation}
              and
              \begin{equation}
                  \begin{aligned}
                      \ulposmap{\associator_{\Obja,\Objb\Objc}}\colon \Obja\cartprod (\Objb\cartprod \Objc) & \toinPos (\Up \Obja\cartprod \Up \Objb) \cartprod \Up \Objc \\
                      \tup{\Objael, \tup{\Objbel,\Objcel}}                                                  & \mapsto (\upit\makeset{\Objael}\cartprod \upit\makeset{\Objbel})\cartprod \upit\makeset{\Objcel}.
                  \end{aligned}
              \end{equation}
    \end{enumerate}
\end{lemma}
\todojira{238}{type proof}

We now want to show that \UPos can be equipped to become a \SY{symmetric monoidal category}.
To do so, we first need the following two facts.

\begin{lemma}
    \label{lem:unpack_u_functor}
    Given \SY{posets}~$\posA,\posB$, a \SY{monotone maps}~$\mora \colon \posA \to \posB$, and a family of singleton sets~$\makeset{S_i}_{i\setin I}$, with~$S_i=\makeset{s_i}$,~$s_i\setin \posA$, the following equality holds:
    \begin{equation}
        \label{eq:lemma_unpack}
        \upit \pars{ \bigsetunion_{\posAel \setin \upit \bigsetunion_{i\setin I}
                S_i}\makeset{\mora(\posAel)}}= \upit \pars{ \bigsetunion_{i\setin I} \makeset{\mora(s_i)}}.
    \end{equation}
\end{lemma}

\todotextjira{241}{the phrasing here seems unnecessarily complicated\dots maybe we phrase the lemma as something like ``$\upit  f( \upit S) = \upit f(S) $'' for any set $S$ and any \SY{monotone map} $f$ ?}

\begin{proof}
    We first want to show that:
    \begin{equation}
        \label{eq:unpack_1}
        \underbrace{\upit \pars{\bigsetunion_{\posAel \setin \upit \bigsetunion_{i\setin I}
                    S_i}\makeset{\mora(\posAel)} }}_{\star}\setsubseteq \upit \underbrace{\pars{ \bigsetunion_{i\setin I}\makeset{\mora(s_i)}}}_{\diamond}.
    \end{equation}
    Take a
    \begin{equation}
        \posBel \setin \upit\pars{ \bigsetunion_{\posAel \setin \upit \bigsetunion_{i\setin I}
                S_i}\makeset{\mora(\posAel)}}.
    \end{equation}
    If we have such a~$\posBel$, it means that there exists a
    \begin{equation}
        \posBel'\setin \bigsetunion_{\posAel\setin \upit\bigsetunion_{i\setin I}
            S_i}\makeset{\mora(\posAel)}
    \end{equation}
    such that~$\posBel'\posleqof\posB \posBel$, and hence there is a~$\posAel'\setin \upit \bigsetunion_{i\setin I} S_i$ such that~$\posBel'=\mora(\posAel')$.
    Consequently, there must exist an~$i'\setin I$ such that~$s_{i'}\posleqof\posA \posAel'$.
    The monotonicity of~$\mapa$ implies:
    \begin{equation}
        \mora(s_{i'})\posleqof\posA \mora(\posAel')=\posBel'\posleqof\posB \posBel.
    \end{equation}
    We know that~$s_{i'}\setin \diamond$ and any~$\posBel^*\setin \posB$ satisfying~$\mora(s_{i'})\posleqof\posB \posBel^*$ belongs to~$\upit \diamond$.
    Therefore,~$\star\setsubseteq \upit \diamond$, which proves the validity of \cref{eq:unpack_1}.

    We now want to show that:
    \begin{equation}
        \label{eq:unpack_2}
        \upit \pars{\bigsetunion_{\posAel \setin \upit \bigsetunion_{i\setin I}
                S_i}\makeset{\mora(\posAel)} }\setsupseteq \upit \pars{ \bigsetunion_{i\setin I}\makeset{\mora(s_i)}}.
    \end{equation}
    By now taking a
    \begin{equation}
        \posBel\setin \upit \pars{ \bigsetunion_{i\setin I}\makeset{\mora(s_i)}},
    \end{equation}
    we know that there is an~$i'\setin I$ such that~$\mora(s_{i'})\posleqof\posB \posBel$.
    Furthermore, we know that~$\mora(s_{i'})\setin \diamond$.
    Therefore, any~$\posBel^*\posleqof\posB \mora(s_{i'})$ must be in~$\upit \diamond$, meaning that~$\posBel\setin \star$, and proving the validity of \cref{eq:unpack_2}.

    The validity of \cref{eq:unpack_1} and \cref{eq:unpack_2} implies \cref{eq:lemma_unpack}.
\end{proof}

\begin{remark}
    \label{rem:unpack_u_functor_bis}
    Given \SY{posets}~$\posA,\posB$ and a \SY{monotone maps}~$\mora\colon \posA\to \posB$, we have:
    \begin{equation}
        \upit \pars{ \bigsetunion_{\posAel'\setin \upit \makeset{\posAel}} \makeset{\mora(\posAel')}}=\upit \makeset{\mora(\posAel)}.
    \end{equation}
    This follows from \cref{lem:unpack_u_functor}, by considering a family of singleton sets consisting solely of the set~$\makeset{\posAel}$.
\end{remark}

\begin{lemma}\label{lem:usets-product}
    The \SY{cartesian product} of \SY{upper sets} is an \SY{upper set}.
    The \SY{cartesian product} of \SY{lower sets} is a \SY{lower set}.
\end{lemma}
\begin{proof}
    Consider two \SY{posets}~$\posA,\posB$ and two respective \SY{upper sets}~$\setA,\setB$.
    We have
    \begin{equation}
        \prfcomma{\setAel \setin \setA}{\setAel \posleqof\posA \setAel'}{\setAel'\setin \setA}
    \end{equation}
    and
    \begin{equation}
        \prfperiod{\setBel \setin \setB}{\setBel \posleqof\posB \setBel'}{\setBel'\setin \setB}
    \end{equation}
    Therefore:
    \begin{equation}
        \prfcomma{
            \tup{\setAel,\setBel} \setin \setA\cartprod \setB
        }{
            \tup{\setAel,\setBel} \posleqof{\posA \Ptimes \posB} \tup{\setAel',\setBel'}
        }{
            \tup{\setAel',\setBel'}\setin \setA \cartprod \setB
        }
    \end{equation}
    which proves that~$\setA\cartprod \setB$ is an \SY{upper set}.
    The proof for the product of \SY{lower sets} is analogous.
\end{proof}
\begin{lemma}
    \label{lem:UPos-is-sym-mon}
    $\tup{\UPos, \mtimescat,\idmoncat}$ from \cref{lem:upos_moncat} equipped with the braiding isomorphism~$\braiding_{\Obja,\Objb}\colon \Obja \cartprod \Objb \mtoiso \Objb \cartprod \Obja$, given by
    \begin{equation}
        \label{eq:Upos-braiding}
        \begin{aligned}
            \ulposmap{\braiding}_{\Obja,\Objb}\colon \Obja \cartprod \Objb & \toinPos \uppersets{(\Objb \cartprod \Obja)} \\
            \tup{\Objael,\Objbel}                                          & \mapsto \upit\makeset{\Objbel} \cartprod \upit\makeset{\Objael},
        \end{aligned}
    \end{equation}
    defined for all~$\Obja,\Objb\setin \Obof{\UPos}$, forms a symmetric mo\-noid\-al category.
\end{lemma}
\begin{proof}
    We first show that the braiding defines an isomorphism.
    In other words, we want to show
    \begin{equation}
        \ulposmap{(\braiding_{\Obja,\Objb}\mthen \braiding_{\Objb,\Obja})}=\ulposmap{\catidat{\Obja \cartprod \Objb}}.
    \end{equation}
    We have
    \begin{equation}
        \begin{aligned}
            ~ & \ulposmap{(\braiding_{\Obja,\Objb}\mthen \braiding_{\Objb,\Obja})}(\Objael,\Objbel) \\
              & =\bigsetunion_{\tup{\Objael',\Objbel'}\setin \ulposmap{\braiding_{\Obja,\Objb}}(\Objael,\Objbel)}\ulposmap{\braiding_{\Objb,\Obja}}(\Objael',\Objbel') \\
              & =\bigsetunion_{\tup{\Objbel',\Objael'}\setin \upit\makeset{\Objbel}\cartprod \upit\makeset{\Objael}}\upit\makeset{ \Objael'}\cartprod \upit\makeset{ \Objbel'} \\
              & =\upit\makeset{\Objael} \cartprod \upit\makeset{\Objbel} \\
              & =\ulposmap{\catidat{\Obja\cartprod \Objb}}(\Objael,\Objbel).
        \end{aligned}
    \end{equation}
    Note that this comes from the fact that~$\braiding$ is an involution.
    We now show naturality.
    Consider~$\mora \colon \Obja \mto \Objb$,~$\morb\colon \Objc \mto \Objd$.
    We have
    %
    \begin{equation}
        \label{eq:braid-upos-natural-a}
        \begin{aligned}
            ~ & \ulposmap{\pars{\pars{ \mora \mtimescatmor \morb}\mthen \braiding_{ \Objb, \Objd}}}(\Objael,\Objcel) \\
              & = \tup{\ulposmap{\mora}(\Objael),\ulposmap{\morb}(\Objcel)}\mthen \braiding_{ \Objb, \Objd} \\
              & =\tup{\bigsetunion_{\Objcel'\setin \ulposmap{\morb}(\Objcel)}\upit \Objcel',\bigsetunion_{\Objael'\setin \ulposmap{\mora}(\Objael)}\upit \Objael'}.
        \end{aligned}
    \end{equation}
    On the other hand:
    \begin{equation}
        \label{eq:braid-upos-natural-b}
        \begin{aligned}
            ~ & \ulposmap{\pars{\braiding_{ \Objd, \Objb}\mthen \pars{ \mora \mtimescatmor \morb}}}(\Objael,\Objcel) \\
              & =
            \tup{\upit\makeset{\Objcel},\upit\makeset{\Objael}}\mthen \ulposmap{\pars{ \mora \mtimescatmor \morb}} \\
              & =\tup{\bigsetunion_{\Objcel'\setin \upit\makeset{\Objcel}} \ulposmap{\morb}(\Objcel'),\bigsetunion_{\Objael'\setin \upit\makeset{\Objael}}\ulposmap{\mora}(\Objael')}.
        \end{aligned}
    \end{equation}
    Clearly, from \cref{lem:unpack_u_functor} and \cref{rem:unpack_u_functor_bis} we know that \cref{eq:braid-upos-natural-a} and \cref{eq:braid-upos-natural-b} are equivalent, proving naturality.
    We now just need to show hexagon identities.
    First, we want to show that
    \begin{equation}
        \label{eq:upos_hexagon_1}
        (\braiding_{\Obja,\Objb}\mtimescatmor \catidat\Objc) \mthen \associator_{\Objb,\Obja,\Objc}\mthen (\catidat\Objb\mtimescatmor \braiding_{\Obja,\Objc})=
        \associator_{\Obja,\Objb,\Objc} \mthen \braiding_{\Obja,\Objb\mtimescatob \Objc}\mthen \associator_{\Objb,\Objc,\Obja}
    \end{equation}
    To do so, we first look at the left-hand side of \cref{eq:upos_hexagon_1}.
    We have
    \begin{equation}
        \begin{aligned}
             & \ulposmap{((\braiding_{\Obja,\Objb}\mtimescatmor \catidat\Objc) \mthen \associator_{\Objb,\Obja,\Objc})}(\tup{\Objael,\Objbel},\Objcel) \\
             & =\bigsetunion_{\tup{\tup{\Objbel',\Objael'},\Objcel'}\setin \ulposmap{(\braiding_{\Obja,\Objb}\mtimescatmor \catidat\Objc)}(\tup{\Objael,\Objbel},\Objcel)}\ulposmap{\associator}_{\Objb,\Obja,\Objc}(\Objbel',\Objael',\Objcel') \\
             & =\bigsetunion_{\tup{\tup{\Objbel',\Objael'},\Objcel'}\setin(\upit \makeset{\Objbel} \cartprod \upit\makeset{ \Objael})\cartprod \upit \makeset{\Objcel}}\upit\makeset{\Objbel'} \cartprod (\upit \makeset{\Objael'}\cartprod \upit\makeset{ \Objcel'}) \\
             & =\upit\makeset{\Objbel} \cartprod (\upit \makeset{\Objael}\cartprod \upit\makeset{ \Objcel}).
        \end{aligned}
    \end{equation}
    Furthermore, we have
    \begin{equation}
        \label{eq:upos_hexagon_2}
        \begin{aligned}
             & \ulposmap{((\braiding_{\Obja,\Objb}\mtimescatmor \catidat\Objc) \mthen \associator_{\Objb,\Obja,\Objc}\mthen (\catidat\Objb\mtimescatmor \braiding_{\Obja,\Objc}))}(\tup{\Objael,\Objbel},\Objcel) \\
             & =\bigsetunion_{\tup{\Objbel', \tup{\Objael',\Objcel'}}\setin \upit\makeset{\Objbel} \cartprod (\upit \makeset{\Objael}\cartprod \upit\makeset{ \Objcel})} \ulposmap{(\catidat\Objb\mtimescatmor \braiding_{\Obja,\Objc})}(\Objbel', \tup{\Objael',\Objcel'}) \\
             & =\bigsetunion_{\tup{\Objbel', \tup{\Objael',\Objcel'}}\setin \upit\makeset{\Objbel} \cartprod (\upit \makeset{\Objael}\cartprod \upit\makeset{ \Objcel})} \upit\makeset{\Objbel'}\cartprod (\upit\makeset{\Objcel'}\cartprod \upit\makeset{\Objael'}) \\
             & =\upit\makeset{\Objbel} \cartprod (\upit \makeset{\Objcel}\cartprod \upit\makeset{ \Objael}).
        \end{aligned}
    \end{equation}
    We now look at the right-hand side of \cref{eq:upos_hexagon_1}.
    We have
    \begin{equation}
        \begin{aligned}
             & \ulposmap{\associator_{\Obja,\Objb,\Objc} \mthen \braiding_{\Obja,\Objb\mtimescatob \Objc}}(\tup{\Objael,\Objbel},\Objcel) \\
             & =\bigsetunion_{\tup{\Objael', \tup{\Objbel',\Objcel'}}\setin \ulposmap{\associator}_{\Obja,\Objb,\Objc}(\tup{\Objael,\Objbel},\Objcel)}\ulposmap{\braiding}_{\Obja,\Objb\mtimescatob\Objc}(\Objael', \tup{\Objbel',\Objcel'}) \\
             & =\bigsetunion_{\tup{\Objael', \tup{\Objbel',\Objcel'}}\setin \upit\makeset{\Objael} \cartprod (\upit \makeset{ \Objbel}\cartprod \upit\makeset{ \Objcel})} (\upit\makeset{\Objbel'} \cartprod \upit \makeset{ \Objcel'})\cartprod \upit\makeset{ \Objael'} \\
             & =(\upit\makeset{\Objbel} \cartprod \upit \makeset{ \Objcel})\cartprod \upit\makeset{ \Objael}.
        \end{aligned}
    \end{equation}
    Furthermore, we have
    \begin{equation}
        \label{eq:upos_hexagon_3}
        \begin{aligned}
             & \ulposmap{(\associator_{\Obja,\Objb,\Objc} \mthen \braiding_{\Obja,\Objb\mtimescatob \Objc}\mthen \associator_{\Objb,\Objc,\Obja})}(\tup{\Objael,\Objbel},\Objcel) \\
             & =\bigsetunion_{\tup{\tup{\Objbel',\Objcel'},\Objael'}\setin (\upit\makeset{\Objbel} \cartprod \upit \makeset{ \Objcel})\cartprod \upit\makeset{ \Objael}}\ulposmap{\associator}_{\Objb,\Objc,\Obja}(\tup{\Objbel',\Objcel'},\Objael') \\
             & =\bigsetunion_{\tup{\tup{\Objbel',\Objcel'},\Objael'}\setin (\upit\makeset{\Objbel} \cartprod \upit \makeset{ \Objcel})\cartprod \upit\makeset{ \Objael}}\upit\makeset{\Objbel'}\cartprod (\upit\makeset{\Objcel'}\cartprod \upit\makeset{\Objael'}) \\
             & =\upit\makeset{\Objbel} \cartprod (\upit \makeset{\Objcel}\cartprod \upit\makeset{ \Objael}).
        \end{aligned}
    \end{equation}
    Clearly, since \cref{eq:upos_hexagon_2} and \cref{eq:upos_hexagon_3} are equal, the first hexagon identity is checked.
    The second hexagon identity can be checked analogously.
\end{proof}

\begin{definition}[Trace in \UPos]
    \label{def:trace-UPos}
    Given a morphism~$\mora\colon \Obja \cartprod \Objc \mto \Objb\cartprod \Objc$ in \UPos, its trace in is defined as a morphism~$\Tr_{\Obja,\Objb}^{\Objc}(\mora)\colon \Obja \mto \Objb$, given by
    \begin{equation}
        \begin{aligned}
            \ulposmap{\Tr_{\Obja,\Objb}^{\Objc}(\mora)}\colon \Obja & \mto \uppersets \Objb \\
            \Objael                                                 & \mapsto \makesett{\Objbel \setin \Objb \mid \nlbigvee_{\Objcel \setin \Objc}\tup{\Objbel, \Objcel}\setin \ulposmap{\mora}(\Objael,\Objcel)}.
        \end{aligned}
    \end{equation}
\end{definition}
\begin{lemma}
    \label{lem:UPos-is-traced}
    $\tup{\UPos, \mtimescat, \idmoncat,\braiding}$ equipped with the trace operation defined in \cref{def:trace-UPos} is a \SY{traced monoidal category}.
\end{lemma}
\begin{proof}
    We have already checked that~$\tup{\UPos, \mtimescat, \idmoncat,\braiding}$ forms a \SY{symmetric monoidal category}.
    First, we check that the trace indeed returns a valid morphism in \UPos.
    Given any~$\Obja,\Objb,\Objc\setin \Obof{\UPos}$ and~$\mora\colon \Obja\cartprod \Objc \mto \Objb\cartprod \Objc$,
    and any~$\Objael\posleq \Objael'\setin \Obja$, we need to prove that
    \begin{equation}
        \prfdouble{ \Tr_{\Obja,\Objb}^{\Objc}(\mora)(\Objael) \posleqof{\UPos} \Tr_{\Obja,\Objb}^{\Objc}(\mora)(\Objael')}{\ulposmap{\Tr_{\Obja,\Objb}^{\Objc}(\mora)}(\Objael) \setsupseteq \ulposmap{\Tr_{\Obja,\Objb}^{\Objc}(\mora)}(\Objael')}
    \end{equation}
    We know that~$\ulposmap{\mora}$ is a \SY{monotone map}, meaning that
    \begin{equation}
        \prftree{\tup{\Objbel,\Objcel} \setin \ulposmap{\mora}(\Objael',\Objcel)}{\tup{\Objbel,\Objcel}\setin \ulposmap{\mora}(\Objael,\Objcel)}
    \end{equation}
    Therefore:
    \begin{equation}
        \prftree{\Objbel \setin \ulposmap{\Tr_{\Obja,\Objb}^{\Objc}(\mora)}(\Objael')}{\Objbel \setin \ulposmap{\Tr_{\Obja,\Objb}^{\Objc}}(\mora)(\Objael)}
    \end{equation}
    proving that~$\ulposmap{\Tr_{\Obja,\Objb}^{\Objc}(\mora)}$ is a \SY{monotone function}.
    Furthermore, due to the monotonicity of~$\ulposmap{\mora}$, for any~$\Objbel\posleq \Objbel'\setin \Objb$,~$\Objael\setin \Obja$,~$\Objcel \setin \Objc$, we have:
    \begin{equation}
        \prftree{\tup{\Objbel,\Objcel}\setin \ulposmap{\mora}(\Objael,\Objcel)}{\tup{\Objbel',\Objcel}\setin \ulposmap{\mora}(\Objael,\Objcel)}
    \end{equation}
    proving that~$\ulposmap{\Tr_{\Obja,\Objb}^{\Objc}(\mora)}(\Objael)$ is an \SY{upper set} for all~$\Objael\setin \Obja$.
    We now check the trace axioms one by one.

    \textbf{Naturality I}:
    Given any object~$\Obja,\Obja',\Objb,\Objc \setin \Obof{\UPos}$, a morphism~$\mora\colon \Obja\cartprod \Objc \mto \Objb\cartprod \Objc$, and a morphism~$\morb\colon \Obja'\mto \Obja$, we have:
    \begin{equation}
        \label{eq:natu-1-upos-a}
        \begin{aligned}
             & \ulposmap{\Tr_{\Obja',\Objb}^{\Objc}((\morb\mtimescatmor \catidat\Objc)\mthen \mora)}(\Objael') \\
             & = \makesett{ \Objbel \setin \Objb \mid \nlbigvee\nolimits_{\Objcel\setin \Objc} \tup{\Objbel,\Objcel}\setin (\ulposmap{(\morb \mtimescatmor \catidat\Objc)\mthen \mora)}(\Objael',\Objcel)} \\
             & =\makesett{ \Objbel \setin \Objb \mid \nlbigvee_{\Objcel\setin \Objc} \tup{\Objbel,\Objcel}\setin \bigsetunion_{\tup{\Objael,\Objcel'}\setin \ulposmap{\morb}(\Objael')\cartprod\upit\makeset{\Objcel}}\ulposmap{\mora}(\Objael,\Objcel')} \\
             & =\makesett{\Objbel \setin \Objb\mid \nlbigvee_{\Objcel\setin \Objc}\tup{\Objbel,\Objcel}\setin \bigsetunion_{\Objael\setin \ulposmap{\morb}(\Objael')} \ulposmap{\mora}(\Objael,\Objcel)}.
        \end{aligned}
    \end{equation}
    On the other hand, we have
    \begin{equation}
        \label{eq:natu-1-upos-b}
        \begin{aligned}
            \ulposmap{(\morb\mthen \Tr_{\Obja,\Objc}^{\Objc}(\mora))}(\Objael') & =
            \bigsetunion_{\Objael\setin \ulposmap{\morb}(\Objael')}\ulposmap{\Tr_{\Obja,\Objc}^{\Objc}(\mora)}(\Objael) \\
                                                                                & =\bigsetunion_{\Objael\setin \ulposmap{\morb}(\Objael')}\makesett{\Objbel \setin \Objb\mid \nlbigvee_{\Objcel\setin \Objc}\tup{\Objbel,\Objcel}\setin \ulposmap{\mora}(\Objael,\Objcel)} \\
                                                                                & =\makesett{\Objbel \setin \Objb\mid \nlbigvee_{\Objcel\setin \Objc}\tup{\Objbel,\Objcel}\setin \bigsetunion_{\Objael\setin \ulposmap{\morb}(\Objael')} \ulposmap{\mora}(\Objael,\Objcel)}.
        \end{aligned}
    \end{equation}
    Clearly \cref{eq:natu-1-upos-a} and \cref{eq:natu-1-upos-b} are equivalent, proving the first naturality condition.

    \textbf{Naturality II}:
    Given any~$\Obja,\Objb,\Objb',\Objc \setin \Obof{\UPos}$, $\mora\colon \Obja\cartprod \Objc \mto \Objb\cartprod \Objc$, and $\morb\colon \Objb\mto \Objb'$, we have:
    \begin{equation}
        \begin{aligned}
            ~ & \ulposmap{\Tr_{\Obja,\Objb'}^{\Objc}(\mora \mthen (\morb \mtimescatmor \catidat\Objc))}(\Objael) \\
              & =
            \makesett{\Objbel'\setin \Objb' \mid \nlbigvee_{\Objcel\setin \Objc} \tup{\Objbel',\Objcel}\setin \ulposmap{(\mora \mthen (\morb \mtimescatmor \catidat\Objc))}(\Objael,\Objcel)} \\
              & =\makesett{ \Objbel'\setin \Objb'\mid \nlbigvee_{\Objcel\setin \Objc} \tup{\Objbel',\Objcel}\setin \bigsetunion_{\tup{\Objbel,\Objcel}\setin \ulposmap{\mapa}(\Objael,\Objcel)} \ulposmap{\morb}(\Objbel) \cartprod \upit\makeset{ \Objcel}}
        \end{aligned}
    \end{equation}
    On the other hand
    \begin{equation}
        \begin{aligned}
            \ulposmap{(\Tr_{\Obja,\Objb}^{\Objc}(\mora)\mthen \morb)}(\Objael) & =
            \bigsetunion_{\Objbel \setin \makeset{ \Objbel \setin \Objb \mid \nlbigvee_{\Objcel \setin \Objc}\tup{\Objbel,\Objcel}\setin \ulposmap{\mora}(\Objael,\Objcel)}}\ulposmap{\morb}(\Objbel)
        \end{aligned}
    \end{equation}
    \todotext{Not clear yet how to massage to show equivalence.
        Talk with J}
    \textbf{Vanishing}:
    Given any~$\Obja,\Objb\setin \Obof{\UPos}$ and~$\mora\colon \Obja \mto \Objb$ in \UPos, we have
    \begin{equation}
        \begin{aligned}
            ~ & \ulposmap{\Tr_{\Obja,\Objb}^{\singleton}(\mora)}(\Objael) \\
              & =\makeset{\Objbel \setin \Objb \mid \tup{\Objbel,\singletonel}\setin \ulposmap{(\mora\mtimescatmor \catidat{\singleton})}(\Objael,\singletonel)} \\
              & =\ulposmap{\mora}(\Objael).
        \end{aligned}
    \end{equation}
    Furthermore, given any~$\Obja,\Objb,\Objc,\Objd \setin \Obof{\UPos}$ and~$\mora\colon \Obja \cartprod \Objc \cartprod \Objd\mto \Objb \cartprod \Objc \cartprod \Objd$, we have
    \begin{equation}
        \label{eq:vanish-upos-a}
        \begin{aligned}
            ~ & \ulposmap{\Tr_{\Obja,\Objb}^{\Objc\cartprod \Objd}(\mora)}(\Objael) \\
              & =\makesett{\Objbel \setin \Objb \mid \nlbigvee_{\tup{\Objcel,\Objdel}\setin \Objc\cartprod \Objd} \tup{\Objbel,\Objcel,\Objdel}\setin \ulposmap{\mora}(\Objael,\Objcel,\Objdel) }
        \end{aligned}
    \end{equation}
    To check the second vanishing axiom, we also write:
    \begin{equation}
        \begin{aligned}
            ~ & \ulposmap{\Tr_{\Obja\cartprod \Objc,\Objb\cartprod \Objc}^{\Objd}(\mora)}(\Objael,\Objcel) \\
              & =\makesett{\tup{\Objbel,\Objcel}\setin \Objb\cartprod \Objc \mid \nlbigvee_{\Objdel\setin \Objd} \tup{\Objbel,\Objcel,\Objdel} \setin \ulposmap{\mora}(\Objael,\Objcel,\Objdel) }.
        \end{aligned}
    \end{equation}
    Therefore, we can write:
    \begin{equation}
        \label{eq:vanish-upos-b}
        \begin{aligned}
            ~ &
            \ulposmap{\pars{\Tr_{\Obja,\Objb}^{\Objc}\pars{\Tr_{\Obja\cartprod \Objc,\Objb\cartprod \Objc}^{\Objd}(\mora)}}}(\Objael) \\
              & =
            \makesett{\Objbel \setin \Objb\mid \nlbigvee_{\Objcel \setin \Objc}\tup{\Objbel, \Objcel}\setin \ulposmap{\Tr_{\Obja\cartprod \Objc,\Objb\cartprod \Objc}^{\Objd}(\mora)}(\Objael,\Objcel)} \\
              & =\makesett{ \Objbel \setin \Objb\mid \nlbigvee_{\Objcel \setin \Objc}\tup{\Objbel, \Objcel}\setin \makesett{ \tup{\Objbel',\Objcel'}\setin \Objb\cartprod \Objc \mid \nlbigvee_{\Objdel\setin \Objd}\tup{\Objbel',\Objcel',\Objd}\setin \ulposmap{\mora}(\Objael,\Objcel',\Objdel)}} \\
              & =\makesett{ \Objbel \setin \Objb\mid \nlbigvee{\Objcel\setin \Objc}\pars{\nlbigvee_{\Objdel \setin \Objd} \tup{\Objbel,\Objcel,\Objdel}\setin \ulposmap{\mora}(\Objael,\Objcel,\Objdel) }} \\
              & =\makesett{\Objbel \setin \Objb \mid \nlbigvee_{\tup{\Objcel,\Objdel}\setin \Objc\cartprod \Objd} \tup{\Objbel,\Objcel,\Objdel}\setin \ulposmap{\mora}(\Objael,\Objcel,\Objdel) }.
        \end{aligned}
    \end{equation}
    Clearly, \cref{eq:vanish-upos-a} and \cref{eq:vanish-upos-b} are equivalent, proving the second vanishing axiom.
    \textbf{Superposing}:
    Given any~$\Obja,\Objb,\Objc\setin \Obof{\UPos}$ and~$\mora\colon \Obja\cartprod \Objc \mto \Objb \cartprod \Objc$, we have:
    \begin{equation}
        \label{eq:superposing-upos-a}
        \begin{aligned}
             & \ulposmap{\Tr_{\Objd\cartprod \Obja, \Objd\cartprod \Objb}^{\Objc}(\catidat\Objd \mtimescatmor \mora)}(\Objdel,\Objael) \\
             & =\makesett{ \tup{\Objdel, \Objbel}\setin \Objd \cartprod \Objb \mid \nlbigvee_{\Objcel \setin \Objc}\tup{\Objdel,\Objbel, \Objcel}\setin \ulposmap{(\catidat\Objd \mtimescatmor \mora)}(\Objdel,\Objael, \Objcel)} \\
             & =\makesett{ \tup{\Objdel, \Objbel}\setin \Objd \cartprod \Objb \mid \nlbigvee_{\Objcel \setin \Objc} (\Objdel \setin \ulposmap{\catidat\Objd}(\Objdel))\booland (\tup{\Objbel, \Objcel}\setin \ulposmap{\mora}(\Objael, \Objcel))} \\
             & =\makesett{ \tup{\Objdel, \Objbel}\setin \Objd \cartprod \Objb \mid \nlbigvee_{\Objcel \setin \Objc} (\Objdel \setin \upit \makeset{\Objdel})\booland (\tup{\Objbel, \Objcel}\setin \ulposmap{\mora}(\Objael, \Objcel))} \\
             & =\makesett{ \tup{\Objdel, \Objbel}\setin \upit\makeset{\Objdel} \cartprod \Objb \mid \nlbigvee_{\Objcel \setin \Objc}  \tup{\Objbel, \Objcel}\setin \ulposmap{\mora}(\Objael, \Objcel)} \\
             & =\upit\makeset{\Objdel}\cartprod \makesett{\Objbel \setin \Objb \mid \nlbigvee_{\Objcel \setin \Objc}  \tup{\Objbel, \Objcel}\setin \ulposmap{\mora}(\Objael, \Objcel)} \\
        \end{aligned}
    \end{equation}
    On the other hand, we have:
    \begin{equation}
        \label{eq:superposing-upos-b}
        \begin{aligned}
            \ulposmap{(\catidat\Objd \mtimescat \Tr_{\Obja,\Objb}^{\Objc}(\mora))}(\Objdel,\Objael) & =
            \upit\makeset{\Objdel}\cartprod\makesett{ \Objbel \setin \Objb \mid \nlbigvee_{\Objcel\setin \Objc}\tup{\Objbel,\Objcel}\setin \ulposmap{\mora}(\Objael,\Objcel)}.
        \end{aligned}
    \end{equation}
    Clearly, \cref{eq:superposing-upos-a} and \cref{eq:superposing-upos-b} are equivalent, proving the superposing axiom.

    \textbf{Yanking}:
    Consider~$\Obja\setin \Obof{\UPos}$.
    We have
    \begin{equation}
        \begin{aligned}
            ~ & \ulposmap{\Tr_{\Obja,\Obja}^{\Obja}(\braiding_{\Obja,\Obja})}(\Objael) \\
              & =\makeset{ \Objael'\setin \Obja \mid \nlbigvee_{\Objael''\setin \Obja}\tup{\Objael',\Objael''}\setin \ulposmap{\braiding}_{\Obja,\Obja}(\Objael,\Objael'')} \\
              & =\makeset{ \Objael'\setin \Obja \mid \nlbigvee_{\Objael''\setin \Obja}\tup{\Objael',\Objael''}\setin \upit\makeset{\Objael''}\cartprod \upit\makeset{ \Objael}} \\
              & =\makeset{ \Objael'\setin \Obja \mid \nlbigvee_{\Objael''\setin \Obja} (\Objael'\setin \upit\makeset{\Objael''}) \booland(\Objael'' \setin \upit\makeset{ \Objael})} \\
              & =\makeset{ \Objael'\setin \Obja \mid \Objael'\setin \upit\makeset{\Objael}} \\
              & =\upit \makeset{\Objael} \\
              & =\ulposmap{\catid}_\Obja(\Objael),
        \end{aligned}
    \end{equation}
    proving the yanking axiom.
    \todotextjira{339}{@Gioele: S how UPos traced}
\end{proof}

\begin{definition}
    [Order on morphisms in~\UPos]
    \label{def:upos_order}
    Given any two morphisms~$\mora,\morb\colon \Obja \mto \Objb$ in~\UPos, we define an order between them as
    \begin{equation}
        \prfperiod{\mora \posleqof\UPos \morb}{\ulposmap{\mora}(\Objael)\posleqof{\uppersets \Objb} \ulposmap{\morb}(\Objael), \quad \forall \Objael \setin \Obja}
    \end{equation}
\end{definition}

\begin{definition}
    [Order on morphisms in~\LPos]
    \label{def:lpos_order}
    Given any two morphisms~$\mora,\morb\colon \Obja \mto \Objb$ in~\LPos, we define an order between them as
    \begin{equation}
        \prfperiod{\mora \posleqof\LPos \morb}{\ulposmap{\mora}(\Objael)\posleqof{\lowersets \Objb} \ulposmap{\morb}(\Objael), \quad \forall \Objael \setin \Obja}
    \end{equation}
\end{definition}

\begin{definition}
    [Intersection of morphisms in~\LPos]\label{def:Lpos-morphism-intersection}
    Given two morphisms~$\mora,\morb\colon \Obja \mto \Objb$ in \LPos, their \emph{intersection} (meet) is a morphism~$\mora \meet \morb \colon \Obja \mto \Objb$, given by
    \begin{equation}
        \begin{aligned}
            \ulposmap{(\mora \meet \morb)}\colon \Obja & \mto \lowersets \Objb \\
            \Objael                                    & \mapsto \ulposmap{\mora}(\Objael) \setintersection \ulposmap{\morb}(\Objael).
        \end{aligned}
    \end{equation}
\end{definition}

\begin{definition}[Union of morphisms in~\UPos]
    \label{def:union-morph-Upos}
    Given two morphisms~$\mora,\morb\colon \Obja \mto \Objb$ in \UPos, their \emph{union} (join) is a morphism~$\mora \join \morb \colon \Obja \mto \Objb$, given by
    \begin{equation}
        \begin{aligned}
            \ulposmap{(\mora \join \morb)}\colon \Obja & \mto \uppersets \Objb \\
            \Objael                                    & \mapsto \ulposmap{\mora}(\Objael) \setunion \ulposmap{\morb}(\Objael).
        \end{aligned}
    \end{equation}
\end{definition}

\begin{definition} [Union of morphisms in~\LPos]\label{def:union-morph-Lpos}
    Given two morphisms~$\mora,\morb\colon \Obja \mto \Objb$ in \LPos, their \emph{union} (join) is a morphism~$\mora \join \morb \colon \Obja \mto \Objb$, given by
    \begin{equation}
        \begin{aligned}
            \ulposmap{(\mora \join \morb)}\colon \Obja & \mto \lowersets \Objb \\
            \Objael                                    & \mapsto \ulposmap{\mora}(\Objael) \setunion \ulposmap{\morb}(\Objael).
        \end{aligned}
    \end{equation}
\end{definition}

\begin{lemma}
    \label{lem:UPos-is-bounded-lattice}
    Given any~$\Obja,\Objb\setin \Obof\UPos$,~$\HomSet{\UPos}{\Obja}{\Objb}$ is a \SY{bounded lattice} with union~$\join$ of morphisms in~\UPos as \SY{join}, intersection~$\meet$ of morphisms in \UPos as \SY{meet}, least upper bound~$\postop_{\HomSet{\UPos}{\Obja}{\Objb}}\colon \Obja \mto \Objb$ given by
    \begin{equation}
        \begin{aligned}
            \ulposmap{\postop_{\HomSet{\UPos}{\Obja}{\Objb}}}\colon \Obja & \mto \uppersets \Objb \\
            \Objael                                                       & \mapsto \Emptyset,
        \end{aligned}
    \end{equation}
    and greatest lower bound~$\posbot_{\HomSet{\UPos}{\Obja}{\Objb}}\colon \Obja \mto \Objb$ given by
    \begin{equation}
        \begin{aligned}
            \ulposmap{\posbot_{\HomSet{\UPos}{\Obja}{\Objb}}}\colon \Obja & \mto \uppersets \Objb \\
            \Objael                                                       & \mapsto \Objb.
        \end{aligned}
    \end{equation}
\end{lemma}
\begin{proof}
    First, we need to prove that~$\HomSet{\UPos}{\Obja}{\Objb}$ forms a poset.
    To prove this, we check the following, using the order defined in \cref{def:upos_order}
    \begin{itemize}
        \item \emph{Reflexivity:} Given~$\mora \setin \HomSet{\UPos}{\Obja}{\Objb}$, we can write
              \begin{equation}
                  \ulposmap{\mora}(\Objael) \setsupseteq \ulposmap{\mora}(\Objael), \quad \forall \Objael \setin \Obja,
              \end{equation}
              which implies~$\mora \posleqof{\UPos} \mora$.
        \item \emph{Antisymmetry:}
              Consider
              \begin{equation}
                  \mora,\morb \setin \HomSet{\UPos}{\Obja}{\Objb}
              \end{equation} with~$\mora \posleqof\UPos \morb$ and~$\morb \posleqof\UPos \mora$.
              We know
              \begin{equation}
                  (\mora \posleqof\UPos \morb)
                  \Imp \ulposmap{\mora}(\Objael) \setsupseteq \ulposmap{\morb}(\Objael), \quad \forall \Objael \setin \Obja,
              \end{equation}
              but also
              \begin{equation}
                  (\morb \posleqof\UPos \mora)
                  \Imp \ulposmap{\morb}(\Objael) \setsupseteq \ulposmap{\mora}(\Objael), \quad \forall \Objael \setin \Obja,
              \end{equation}
              implying~$\mora=\morb$.
        \item \emph{Transitivity:}
              Consider
              \begin{equation}
                  \mora,\morb,\morc \setin \HomSet{\UPos}{\Obja}{\Objb}
              \end{equation}
              with~$\mora \posleqof\UPos \morb$ and~$\morb \posleqof\UPos \morc$.
              We have, for all~$\Objael\setin \Obja$,
              \begin{equation}
                  \begin{aligned}
                      (\ulposmap{\mora}(\Objael) \setsupseteq \ulposmap{\morb}(\Objael))
                      \wedge ( \ulposmap{\morb}(\Objael) \setsupseteq \ulposmap{\morc}(\Objael))
                       & \Imp \ulposmap{\mora}(\Objael) \setsupseteq \ulposmap{\morc}(\Objael) \\
                       & \Imp \mora \posleqof\UPos \morc.
                  \end{aligned}
              \end{equation}
    \end{itemize}
    \todotext{Here, \SY{meet} and \SY{join} are swapped (wrt to symbols), because of the inverse order.
        Should we define them oppositely for UPos?
    }
    Consider now~$\mora,\morb \setin \HomSet{\UPos}{\Obja}{\Objb}$.
    Their least upper bound (join) is~$\mora \meet \morb$, since it is the least morphism such that~$\mora \posleqof\UPos (\mora \meet \morb)$ and~$\morb \posleqof\UPos (\mora \meet \morb)$.
    Their greatest lower bound (meet) is~$\mora \join \morb$, since it is the greatest morphism such that~$(\mora \join \morb)\posleqof\UPos \mora $ and~$(\mora \join \morb)\posleqof\UPos \morb$.
    \todotext{Discuss with J if we actually want to show explicitly ``greatest'' and ``lowest''}
    Furthermore, for any~$\mora \setin \HomSet{\UPos}{\Obja}{\Objb}$, one will have, for all~$\Objael \setin \Obja$
    \begin{equation}
        \ulposmap{\mora}(\Objael) \setsupseteq \Emptyset=\ulposmap{\postop_{\HomSet{\UPos}{\Obja}{\Objb}}}(\Objael),
    \end{equation}
    implying that for all~$\mora \setin \HomSet{\UPos}{\Obja}{\Objb}$ we have~$\mora \posleqof\UPos \postop_{\HomSet{\UPos}{\Obja}{\Objb}}$.
    Finally, for any~$\mora \setin \HomSet{\UPos}{\Obja}{\Objb}$, one will have, for all~$\Objael \setin \Obja$
    \begin{equation}
        \ulposmap{\posbot_{\HomSet{\UPos}{\Obja}{\Objb}}}(\Objael)=\Objb \setsupseteq \ulposmap{\mora}(\Objael)
    \end{equation}
    implying that for all~$\mora \setin \HomSet{\UPos}{\Obja}{\Objb}$ we have~$\posbot_{\HomSet{\UPos}{\Obja}{\Objb}}\posleqof\UPos \mora$.
\end{proof}

\begin{lemma}
    \label{lem:LPos-is-bounded-lattice}
    Given any~$\Obja,\Objb\setin \Obof\UPos$,~$\HomSet{\LPos}{\Obja}{\Objb}$ is a \SY{bounded lattice} with intersection~$\wedge$ of morphisms in~\LPos as \SY{meet}, union~$\vee$ of morphisms in \LPos as \SY{join}, least upper bound~$\postop_{\HomSet{\LPos}{\Obja}{\Objb}}\colon \Obja \mto \Objb$ given by
    \begin{equation}
        \begin{aligned}
            \ulposmap{\postop_{\HomSet{\LPos}{\Obja}{\Objb}}}\colon \Obja & \mto \lowersets \Objb \\
            \Objael                                                       & \mapsto \Objb,
        \end{aligned}
    \end{equation}
    and greatest lower bound~$\posbot_{\HomSet{\UPos}{\Obja}{\Objb}}\colon \Obja \mto \Objb$ given by
    \begin{equation}
        \begin{aligned}
            \ulposmap{\posbot_{\HomSet{\LPos}{\Obja}{\Objb}}}\colon \Obja & \mto \lowersets \Objb \\
            \Objael                                                       & \mapsto \Emptyset.
        \end{aligned}
    \end{equation}
\end{lemma}
\begin{proof}
    The proof is analogous to the one of \cref{lem:UPos-is-bounded-lattice}.
    Note that meets/joins and top/bottom are switched in meaning, because of the difference in order between~$\uppersets \Obja$ and~$\lowersets \Obja$.
\end{proof}

