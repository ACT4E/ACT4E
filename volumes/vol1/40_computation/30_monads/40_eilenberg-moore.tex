% !TEX root = chapter-standalone.tex


\section{Algebras of a monad}
\label{sec:Eilenberg-Moore}

In the context of Kleisli morphisms, we developed the intuition that monads can be used to encode ``generalized objects'' and ``generalized morphisms''. In this section we will introduce a different intuition for monads: that they can be used to provide coherent way to encode ``formal expressions'', together with a way to ``evaluate'' or ``compute'' such expressions. 

Let us explain what we mean using an example. Given a set $\setA$, say
\begin{equation}
\setA = \{ \sapple, \sbanana, \scarrot \},
\end{equation}
let's define a certain type of ``formal expressions'', using elements of $\setA$ and using a ``formal composition symbol'', which we choose to be `` $*$ ''. Now, the ``formal expressions'' we'll consider are all finite expressions which have a form such as 
\begin{equation}\label{eq:formal-composition-1}
 \sbanana * \sbanana * \sapple,
\end{equation}
or
\begin{equation}\label{eq:formal-composition-2}
\sapple * \sbanana * \sapple * \sapple * \sbanana * \sapple,
\end{equation}
and so on. 
These expressions are ``formal'' (or ``purely symbolic'') in the sense that, a priori, the symbol `` $*$ '' does not have any \emph{meaning} beyond simply being a symbol, a ``marking''. After all, so far, $\setA$ is just a set, and we did not assume that it comes equipped with any sort of ``multiplication operation'', for instance. In the following we will discuss a way of giving such formal expressions a ``computational meaning'' by specifying a way to evaluate them. 


Before we come to this however, let us introduce a notation to explicitly distinguish when we are thinking about $\sapple$ as an element of $\setA$, or $\sapple$ as a ``formal expression''. For the latter situation we'll write `` $[\sapple]$ ''.
In other words, the square brackets indicate that $[\sapple]$ is a formal expression. And we'll say that formal expressions can be combined, using $*$, to larger formal expressions. So, following this convention, $[\sapple] * [\sbanana]$ is also a formal expression. And in particular, instead of \cref{eq:formal-composition-2}, we'll write
\begin{equation}\label{eq:formal-composition-bracketed}
[\sapple] * [\sbanana] * [\sapple] * [\sapple] * [\sbanana] * [\sapple].
\end{equation}

In an expression such as \cref{eq:formal-composition-bracketed} we'll think of the components $[\sapple]$ and $[\sbanana]$ as if they are ``letters'' (but we won't count ``$*$'' as a letter) and we'll think of the whole expression \cref{eq:formal-composition-bracketed} like a ``word''. For any such word, we'll say its \emph{length} is the number of letters it is built from. So, for instance, the word in \cref{eq:formal-composition-bracketed} has a length of $6$. In our notion of formal expression, we'll choose to include a unique special word of length $0$, which we call the ``empty formal expression'' and denote by ``[ ]''.  


 
Using square brackets we can also build ``formal expressions of formal expressions'' or ``second-order formal expressions''. For example, given the formal expression $[\sapple] * [\sbanana]$, we can turn it into a second-order formal expression by putting brackets around it: 
\begin{equation}\label{eq:double-formal-expression}
[[\sapple] * [\sbanana]].
\end{equation}
And given another second-order formal expression, say $[[\sbanana] * [\sbanana] * [\sapple]]$, we can ``compose'' it with the one in \cref{eq:double-formal-expression} like so: 
\begin{equation}\label{eq:double-formal-expression-comp}
[[\sapple] * [\sbanana]] * [[\sbanana] * [\sbanana] * [\sapple]]. 
\end{equation}
The notion of length will also apply to second-order expressions. For instance, the second-order expression \cref{eq:double-formal-expression-comp} has length 2, and is composed of one first-order expression of length 2 and one first-order expression of length 3. 

This whole game can continue ad infinitum: we define third-order formal expressions to be those with three-layers of square brackets, fourth-order formal expressions have four layers of brackets, and so on. In the following, we'll probably only ever consider up to three layers. 

We started our story just with the set $\setA = \{ \sapple, \sbanana, \scarrot \}$. However we can do the same construction -- using ``$*$'' and building formal expressions of any order -- with any set. In fact, we can define a functor $\funa: \Set \to \Set$ which maps any set $\setA$ to the set $\funa \setA$ whose elements are all finite first-order formal expressions built from elements of $\setA$. We also include this to mean the empty formal expression ``$[ ]$''. What might this functor do on the level of morphisms? For concreteness, let $\setA = \{ \sapple, \sbanana, \scarrot \} $ and $\setB = \{ \swater, \stea \}$. Given a function $\mora: \setA \to \setB$, we define
\begin{equation}
\funa(\mora) : \funa(\setA) \to \funa(\setA)
\end{equation}
to act on (first-order) expressions like so
\begin{equation}
\funa(\mora)([\sapple]*[\sbanana]) = [\mora(\sapple)]*[\mora(\sbanana)].
\end{equation}
It turns out that this functor $\funa: \Set \to \Set$ can be made into a monad! 

Let us explain how the unit and multiplication for this monad are defined. 
For any set $\setA$, the component of the unit at $\setA$ is
\begin{equation}
\monunit_\setA : \setA \to \funa \setA, \ela \mapsto [\ela]. 
\end{equation}
The multiplication is a bit more of a mouthful. It's component at $\setA$, 
\begin{equation}
\moncomp_\setA : (\funa \then \funa)\setA \to \funa \setA,
\end{equation}
is the function which takes a second-order formal exression
\begin{equation}
[[\ela_{11}]* [\ela_{12}] * \cdots * [\ela_{1k_1}]] * \cdots * [[\ela_{n1}]* [\ela_{n2}] * \cdots * [\ela_{nk_n}]] 
\end{equation}
and ``collapses'' it to the first-order expression
\begin{equation}
[\ela_{11}]* [\ela_{12}] * \cdots * [\ela_{1k_1}] * \cdots * [\ela_{n1}]* [\ela_{n2}] * \cdots * [\ela_{nk_n}].
\end{equation}
In other words, this operation simply ``removes the outer brackets'' from a second-order formal expression. 

\begin{gradedexercise}[\exname{ListMonad}]\label{ex:ListMonad}
Let $\funa : \Set \fto \Set$ be the functor above that sends any set $\setA$ to the set of first-order formal expressions of the form
\begin{equation*}
[\ela_1]* [\ela_2] * \cdots * [\ela_n] \quad \quad \ela_i \in \setA, n \in \wnumbers_{\geq 0},
\end{equation*}
and let $\moncomp_\setA$ and $\monunit_\setA$ be defined as above. 

Show that: 
\begin{enumerate}
\item the components $\moncomp_\setA$ define a natural transformation $\moncomp : \funa \then \funa \nto \funa$; 
\item the components $\monunit_\setA$ define a natural transformation $\monunit : \stylefunctors{\text{Id}}_{\Set} \fto \funa$; 
\item $\moncomp$ and $\monunit$ satisfy the conditions for $\tup{\funa, \moncomp, \monunit}$ to be a monad. 
\end{enumerate}

\end{gradedexercise}



Now let's finally talk about giving formal expressions a computational meaning: a way to \emph{evaluate} them. The way we will do this is to define a notion of ``evaluation map'' $\stylemorph{a}: \funa \setA \to \setA$ which we will interpret as a way of specifying how any formal expression -- an element of $\funa \setA$ -- should be evaluated (or: computed) to an element of $\setA$. We will require such evaluation maps to additionaly satisfy two coherence conditions, and the resulting mathematical structure will be what is called an \emph{algebra} of the monad $\funa$. 

\

\todotext{explain further by giving a concrete example of an algebra for this monad}

\begin{ctdefinition}[Algebra of a monad]
    \label{def:algebramon}
Let~$\tup{\monA, \moncomp, \monunit}$ be a monad on a category~$\CatC$. An algebra of~$\monA$ (also called an $\monA$-algebra) is specified by: \

\constit
\begin{compactenum}
\item an object $\Obja$ of $\CatC$;
\item a morphism $\monact: \monA(\Obja) \mto \Obja$ of $\CatC$.
\end{compactenum}
\condit
\begin{compactenum}
\item \emph{Composition}: the diagram
\begin{center}
    \includesag{monalg-composition}
  \end{center}
commutes.
\item \emph{Unit}: the diagram 
\begin{center}
    \includesag{monalg-unit}
  \end{center}
commutes.
\end{compactenum}
\end{ctdefinition}


\begin{ctdefinition}[$\monA$-algebra morphism]
    \label{def:algebramorphism}
Let~$\tup{\monA, \monunit, \moncomp}$ be a monad on a category~$\CatC$, and let~$\tup{\Obja_1, \monact_1}$ and~$\tup{\Obja_2, \monact_2}$ be algebras of $\monA$. A morphism $\tup{\Obja_1, \monact_1} \mto \tup{\Obja_2, \monact_2}$ of $\monA$-algebras is specified by: \

\constit
\begin{compactenum}
\item A morphism $\mora : \Obja_1 \mto \Obja_2$ in $\CatC$.
\end{compactenum}
\condit
\begin{compactenum}
\item The diagram
\begin{center}
    \includesag{monalg-morph-compat}
  \end{center}
commutes. 
\end{compactenum}
\end{ctdefinition}


\begin{ctdefinition}[Category of $\monA$-algebras]
    \label{def:catofmonadalgebras}
Let $\tup{\monA, \monunit, \moncomp}$ be a monad on a category $\CatC$. The \emph{category of $\monA$-algebras} $\CatC^\monA$ of the monad $\monA$ is specified by:
\begin{compactenum}
\item \emph{Objects}: $\monA$-algebras;
\item \emph{Morphisms:} $\monA$-algebra morphisms;
\item \emph{Identities}: given an $\monA$-algebra $\tup{\Obja, \monact}$, its identity morphism is $\catid_\Obja$;
\item \emph{Composition}: is induced by the composition of morphisms in $\CatC$.
\end{compactenum}
\end{ctdefinition}


