% !TEX root = chapter-standalone.tex


\section{Algebras for a monad}
\label{sec:Eilenberg-Moore}

In the context of Kleisli morphisms, we developed the intuition that monads can be used to encode ``generalized objects'' and ``generalized morphisms''. In this section we will introduce a different intuition for monads: that we can be used to provide coherent way to encode ``formal expressions'' of a given kind, together with a way to ``evaluate'' or ``compute'' such expressions. 

Let us explain what we mean using an example. Given a set $\setA$ and elements $\ela_1, ..., \ela_n$, a ``formal expression'' could be a symbolic expression such as 
\begin{equation}\label{eq:formal-sum}
\ela_1 + \ela_2 + ... + \ela_n.
\end{equation}
This expression is ``formal'' in the sense that, a priori, the symbol ``+'' does not have any \emph{meaning} beyond simply being a symbol here: $\setA$ is, so far, just a set, and we did not assume that it comes equipped with any sort of ``addition operation''. Of course, the symbol ``+'' that we have chosen is suggestive of the fact that we intend to talk about something that behaves like addition. But so far, we could have just as well writing something like
\begin{equation}\label{eq:formal-zodiac-sum}
\ela_1 \  \alphabetasymba \ \ela_2 \ \alphabetasymba \ ...  \ \alphabetasymba \  \ela_n
\end{equation}
or 
\begin{equation}\label{eq:formal-apple-sum}
\ela_1 \ \sapple \ \ela_2 \ \sapple \ ... \ \sapple  \ \ela_n.
\end{equation}
instead. 
