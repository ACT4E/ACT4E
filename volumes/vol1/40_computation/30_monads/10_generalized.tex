% !TEX root = chapter-standalone.tex

The definition of monads is very powerful because it is very abstract and can fit many possible scenarios.
Before getting to the formal definition, it is useful to build up some intuition using several examples.

\section{Generalized objects and operations}
\linkvideo{spring2021-monads-a:intro-monad} % Introduction to monads

Monads are a type of algebraic structure that is well-suited to represent generalized objects and operations.
We begin by giving several examples.

\subsection{Modeling nondeterministic uncertainty}
\linkvideo{spring2021-monads-a:modeling-nondet-unc} % Modeling nondeterministic uncertainty

For the engineer, one intuitive scenario where generalization is necessary is handling uncertainty.
We have seen the category  \Set of sets and functions between sets.
A function
\begin{equation}
    \label{eq:monad-simple}
    \mora\colon\Obja\to\Objb
\end{equation}
between sets is ``deterministic'', in the sense that, given as input an element of~$\Obja$, it always produces the same output in~$\Objb$.

Suppose now we want to deal with nondeterministic functions: functions that return, for each input, a \emph{set} of possible values.
We can model nondeterministic functions from $\Obja$ to $\Objb$ as functions of the type
\begin{equation}
    \label{eq:monad-generalized}
    \mora\colon\Obja\to\powerset\Objb.
\end{equation}
%
Note that this is a generalization, in the sense that any deterministic function is a special nondeterministic function.
For example, the function
%
\begin{equation}
    \label{eq:monad-alpha-example1}
    \definemap{\alpha}{\natnumbers}{\natnumbers,}%
    {x}{x^2,}
\end{equation}
%
can be rewritten as the function
%
\begin{equation}
    \label{eq:monad-alpha-example2}
    \definemap{\alpha'}{\natnumbers}{\powerset\natnumbers,}%
    {x}{\{x^2\},}
\end{equation}
%
which maps each element to a singleton set.

Need: mapping a regular morphism into a generalized morphism.

Once we have these generalized functions, we really want them to form a category, so that they can compose.
To do this, we need extra information: additional structure.
So far we know the rules of composition for functions:
%
\begin{equation}
    \label{eq:monad-function-composition}
    \prftree{\mora\colon \Obja \to \Objb}{\quad}{\morb\colon \Objb \to \Objc}
    {\definemap{(\mora\mthen\morb)}{\Obja}{\Objc}%
        {x}{\morb(\mora(x)),}}
\end{equation}
%
but this does not help for the generalized functions.
How can we compose nondeterministic function?
What to fill in the space below?
%
\begin{equation}
    \label{eq:monad-gen-function-composition}
    \prftree{\mora\colon \Obja \to \powerset\Objb}{\quad}{\morb\colon \Objb \to \powerset \Objc}
    {\definemap{(\mora\mthen\morb)}{\Obja}{\powerset\Objc}%
        {x}{\text{\color{red}?}}}
\end{equation}
%
There is one natural way to define this operation.
Did you notice the following?

\begin{lemma}
    \label{lem:powersets-relations}
    Functions from~$\Obja$ to~$\powerset{\Objb}$ are in one-to-one correspondence with relations from~$\Obja$ to~$\Objb$.
\end{lemma}

\begin{proof}
    \todotextjira{226}{@J: We showed this already in a previous chapter.
        We could include a cross reference and/or also explain how this works again.
    }
\end{proof}

Therefore, we expect that nondeterministic functions can be composed like relations.
If we have~$\mora\colon \Obja \to \powerset\Objb$ and~$\morb\colon \Objb \to \powerset \Objc$ then we define the composite non-deterministic function to be
%
\begin{equation}
    \label{eq:fish-nondet}
    \definemap{(\mora\mthen\morb)}{\Obja}{\powerset\Objc}%
    {x}{ \displaystyle \bigcup_{y\in \mora(x)}\morb(y)}
\end{equation}
%
One can convince oneself that this is correct by looking at~\cref{fig:mapping-nondeterministic}.

\begin{figure}[h]
    \todographicsjira{227}{@Gioele: Put here the same figure for composition of relation; however now making it clear that it's not x in relation with many ys, but rather x maps to a subset of ys.}
    \caption{}
    \label{fig:mapping-nondeterministic}
\end{figure}

We should also check that this composition is associative; however, this comes automatically from the fact that we already know that \Rel is a category.

To summarize:
\begin{itemize}
    \item We wanted to extend \Set from functions $\Obja \to \Objb$ to nondeterministic functions of type $\Obja \to \powerset\Objb$.
    \item To do this, we needed three things:
          \begin{enumerate}
              \item The particular choice of $\powerset$ as what maps a set to another set.
              \item A way to lift a function of type $\Obja \to \Objb$ to a function of type $\Obja \to \powerset\Objb$, so that we can say that nondeterministic functions are a generalization of deterministic functions.
                    This lifting operation is a family of functions of type
                    \begin{equation}
                        \label{eq:monad-lift}
                        \lift_{\Obja, \Objb}: (\Obja \to \Objb) \to (\Obja \to \powerset\Objb).
                    \end{equation}
              \item A way to define composition, through a map, traditionally called ``fish'', of type
                    \begin{equation}
                        \label{eq:monad-fish}
                        \fish_{\Obja, \Objb, \Objc}: (\Obja \to \powerset\Objb) \cartprod (\Objb \to \powerset\Objc)
                        \to (\Obja \to \powerset \Objc)
                    \end{equation}
          \end{enumerate}
    \item And we needed these pieces to satisfy the conditions:
          \begin{enumerate}
              \item $\fish$ is associative.
                    This ensures the generalized functions form a semi-category.
              \item The composition of the lifted functions are the lifting of the composition:
                    \begin{equation}
                        \label{eq:monad-fish-associative}
                        \prftree{\mora\colon \Obja \to \powerset\Objb}{\quad}{\morb\colon \Objb \to \powerset \Objc}{
                            \lift_{\Obja,\Objb}(\mora)\mthen\lift_{\Objb,\Objc}(\morb) =  \lift_{\Obja,\Objc}(\mora\mthen\morb)}
                    \end{equation}
                    This ensures that inside the generalized functions, the composition of regular functions continues to work as it should.
          \end{enumerate}
\end{itemize}

\subsection{Modeling interval uncertainty}
\linkvideo{spring2021-monads-a:intervals} % Intervals

We continue to build intuition considering another type of uncertainty.

We have seen the category \Pos of posets and monotone functions between posets~(\cref{def:Pos}).

In this category it is easy to propagate uncertainty if the uncertain sets are represented by intervals.

Recall that for a poset $\posgenA$, we can define the poset of intervals~$\Arrow\,\posgenA$
and~$\TwistedArrow\,\posgenA$.

Analogously to the previous case, here we want to generalize from monotone functions to nondeterministic monotone functions, where the uncertainty is represented by intervals.

We use the following notation for intervals:
\begin{itemize}
    \item The interval
          \begin{equation}
              \label{eq:interv-1}
              \{x\colon a \posleq x \posleq b\}
          \end{equation}
          is denoted
          \begin{equation}
              \label{eq:interv-2}
              \interv{a}{b}
          \end{equation}
    \item $\udpL\cdot$ and $\udpU\cdot$ extract the lower and upper bound from the interval, so that we have
          \begin{equation}
              \label{eq:interv-L}
              \udpL \interv{a}{b} = a
          \end{equation}
          \begin{equation}
              \label{eq:interv-U}
              \udpU \interv{a}{b} = b
          \end{equation}
\end{itemize}
%
\begin{equation}
    \label{eq:interv-L-def}
    \definemappos{\udpL_{\posgenA}}{\Arrow\,\posgenA}{\posgenA}%
    {\interv{a}{b}}{ a }
\end{equation}
%
\begin{equation}
    \label{eq:interv-U-def}
    \definemappos{\udpU_{\posgenA}}{\Arrow\,\posgenA^{\op}}{\posgenA}%
    {\interv{a}{b}}{ b }
\end{equation}
%
For example, one such function is
%
\begin{equation}
    \label{eq:interval-func-example}
    \definemap{\mora}{\reals}{\Arrow\,\reals,}%
    {x}{ \interv{x-1}{x+1}.
    }
\end{equation}
Because an interval is defined by two values, a function that returns an interval is a pair of functions, whose results are constrained to be ordered.
In the example above, we always have $x-1\leq x+1$.

We now retrace the steps of the previous example.

First, we need to ensure that we can see regular monotone functions as special cases of interval functions.
For example, we want a function
%
\begin{equation}
    \label{eq:monotone-simple}
    \definemap{\alpha}{\tup{\natnumbers,\leq}}{\tup{\natnumbers,\leq}}%
    {x}{x^2,}
\end{equation}
%
can be rewritten as the function
%
\begin{equation}
    \label{eq:monotone-simple-as-interval}
    \definemap{\alpha'}{\tup{\natnumbers,\leq}}{\Arrow\,\tup{\natnumbers,\leq},}%
    {x}{ \interv{x^2}{x^2}}
\end{equation}
%
Generically, this is the definition of the $\lift$ function
\begin{equation}
    \label{eq:monotone-lift}
    \definemap%
    {\lift_{\Obja, \Objb}}%
    {(\Obja \to_{\Pos} \Objb)}%
    {(\Obja \to_{\Pos} \Arrow\,\Objb)}%
    {\mora}{%
        \begin{cases}
            \definemap{\lift_{\Obja, \Objb}\mora}%
            {\Obja}{\Arrow\,\Objb}
            {x}{\interv{\mora(x)}{\mora(x)}}
        \end{cases}
    }
\end{equation}
%
What is the "fish" function?
Note that an interval-valued monotone map is also a special relation.
Therefore, we want that composition continues to work in the same manner.

Therefore, we obtain for the $\fish$ operation:
\begin{equation}
    \label{eq:monotone-fish}
    \prftree%[l]{\fish\colon}%
    {\mora: \Obja \to \Arrow\,\Objb}%
    {\quad}{\morb: \Objb \to \Arrow\,\Objc}%
    {\definemap{(\mora\mthen\morb)}{\Obja}{\Arrow\,\Objc}%
        {x}{ \interv{\udpL\morb(\udpL\mora(x))}
            {\udpU\morb(\udpU\mora(x))}}
    }
\end{equation}
%

\subsection{Upper sets}

We have seen that to solve DPs we encountered functions of type
%
\begin{equation}
    \label{eq:monad-up-example1}
    f: \funsp \mtoin\Pos \uppersets \ressp
\end{equation}
%
We can see this as another example of generalization from a functions
\begin{equation}
    \label{eq:monad-up-gen0}
    f: \Obja \mtoin\Pos \Objb
\end{equation}
%
to functions
%
\begin{equation}
    \label{eq:monad-up-gen1}
    f: \Obja \mtoin\Pos \uppersets \Objb
\end{equation}
%
Once again we see these as particular types of relations.

The $\lift$ operation is
%
\begin{equation}
    \label{eq:monad-up-lift}
    \definemap%
    {\lift_{\Obja, \Objb}}%
    {(\Obja \to_{\Pos} \Objb)}%
    {(\Obja \to_{\Pos} \uppersets\Objb)}%
    {\mora}{%
        \begin{cases}
            \definemap{\lift_{\Obja, \Objb}\mora}%
            {\Obja}{\uppersets\Objb}
            {x}{\upit \mora(x)}
        \end{cases}
    }
\end{equation}
%
The $\fish$ operation is
%
\begin{equation}
    \label{eq:monad-up-fish}
    \prftree%[l]{\fish\colon}%
    {\mora: \Obja \to \uppersets\Objb}%
    {\quad}{\morb: \Objb \to \uppersets\Objc}%
    {\definemap{(\mora\mthen\morb)}{\Obja}{\uppersets\Objc}%
        {x}{    \bigcup_{y\in\mora(x)}\morb(y) }}
\end{equation}
%
Note that the expression is the same as in \cref{eq:fish-nondet} - only we are guaranteed to obtain upper sets.

\subsection{Keeping track of resource usage}

As another example, we consider the case where we want to attach additional information to a category.

Suppose we want to consider not functions, but \emph{procedures} which have resource consumption.
A function is a mathematical entity - a procedure is a program that \emph{implements} a function.
Suppose we want to model execution time - and, that execution time might depend on the input of the procedure.

In this case, we would like to extend a function
\begin{equation}
    \label{eq:monad-resource-0}
    \mora: \Obja \to \Objb
\end{equation}
into a procedure
\begin{equation}
    \label{eq:monad-resource-1}
    \mora: \Obja \to (\Objb\cartprod \nonNegReals)
\end{equation}
which gives both the result as well as the execution time.

We consider the ideal functions to be procedures that have zero execution time.

We can define the $\lift$ map as follows:
\begin{equation}
    \label{eq:monad-resource-lift}
    \definemap%
    {\lift_{\Obja, \Objb}}%
    {(\Obja \to \Objb)}%
    {(\Obja \to \Objb \cartprod \nonNegReals)}%
    {\mora}{%
        \begin{cases}
            \definemap{\lift_{\Obja, \Objb}\mora}%
            {\Obja}{\Objb\cartprod\nonNegReals}
            {x}{\tup{x, 0}}
        \end{cases}
    }
\end{equation}
%

% As for the fish function we have

% \begin{equation}
%     \prftree%[l]{\fish\colon}%
%     {\mora: \Obja \to (\Objb \times \nonNegReals)}%
%     {\quad}{\morb: \Objb \to (\Objc \times \nonNegReals)}%
%     {\definemap{(\mora\mthen\morb)}{\Obja}{(\Objc\times \nonNegReals)}%
%     {x}{ \tup{\morb_1(\mora_1(x)), \morb_2(\mora_1(x)) + \mora_2(x) } }}
%   \end{equation}

As for the fish function we have
%
\begin{equation}
    \label{eq:monad-resource-fish}
    \prftree%[l]{\fish\colon}%
    {\mora: \Obja \to (\Objb \cartprod \nonNegReals)}%
    {\quad}{\morb: \Objb \to (\Objc \cartprod \nonNegReals)}%
    {\definemap{(\mora\mthen\morb)}{\Obja}{(\Objc\cartprod \nonNegReals)}%
        {x}{ \tup{\morb_1(\mora_1(x))\ ,\  \morb_2(\mora_1(x)) + \mora_2(x) } }}
\end{equation}
%
Note that we can recover the naked function by taking the first component of the tuple.

\subsubsection{Generalization with monoid}
\linkvideo{spring2021-monads-a:how-to-gen-monads} % How to generalize

This can be further generalized from the real numbers to an arbitrary monoid
\begin{equation}
    \label{eq:monad-resource-monoid}
    \monoidA = \tup{\monoidAset, \mtimesof\monoidA, \idmon_\monoidA}
\end{equation} structure.
%
\begin{equation}
    \label{eq:monad-resource-monoid-lift}
    \definemap%
    {\lift_{\Obja, \Objb}}%
    {(\Obja \to \Objb)}%
    {(\Obja \to \Objb \cartprod \monoidAset)}%
    {\mora}{%
        \begin{cases}
            \definemap{\lift_{\Obja, \Objb}\mora}%
            {\Obja}{\Objb\cartprod\monoidAset}
            {x}{\tup{x, \idmon_\monoidA}}
        \end{cases}
    }
\end{equation}
%
As for the fish function we have
%
\begin{equation}
    \label{eq:monad-resource-monoid-fish}
    \prftree%[l]{\fish\colon}%
    {\mora: \Obja \to (\Objb \cartprod \monoidAset)}%
    {\quad}{\morb: \Objb \to (\Objc \cartprod \monoidAset)}%
    {\definemap{(\mora\mthen\morb)}{\Obja}{(\Objc\cartprod \monoidAset)}%
        {x}{ \tup{\morb_1(\mora_1(x))\ ,\  \morb_2(\mora_1(x)) \mtimesof\monoidA \mora_2(x) } }}
\end{equation}
