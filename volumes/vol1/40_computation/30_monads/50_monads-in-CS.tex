% !TEX root = chapter-standalone.tex


\section{Monads, computer science definition}
\label{sec:monads-CS}


\begin{definition}[Monad (computer science)]\label{def:monad-computer-science}
  A monad \begin{equation}
    \tup{ \monA,\return,\mjoin,\fmap,\bind,\fish,\lift} 
  \end{equation}
  is a set of operations with the following signature:
  
  \begin{align*}
  \return & :A\to\monA A\\
  \mjoin & :\monA{\monA A}\to\monA A\\
  \fmap & :(A\to B)\to(\monA A\to\monA B)\\
  \lift & :(A\to B)\to( A\to\monA B)\\
  \bind & :\monA A\to(A\to\monA B)\to\monA B\\
  \fish & :(A\to\monA B)\to(B\to\monA C)\to(A\to\monA C)
  \end{align*}
  $\return$ is also known as ``return''
  
  These maps satisfy the equivalent axioms of unitality and associativity:
  \begin{itemize}
  \item $\return$ is a left identity for $\bind$:
  
  \todo[inline]{...}
  \item $\return$ is a right identity for $\bind$:
  
  \todo[inline]{...}
  \item $\bind$ is associative
  
  \todo[inline]{...}
  \end{itemize}
\end{definition}