% !TEX root = chapter-standalone.tex

\section{Monads, computer science definition}
\label{sec:monads-CS}
\linkvideo{spring2021-monads-a:monad-cs} % Computer science definition

\begin{definition}[Monad in functional programming]
    \label{def:monad-computer-science}
    A monad \begin{equation}
        \tup{\return,\mjoin,\fmap,\bind,\fish,\lift}
    \end{equation}
    is a set of operations with the following signature:
    %
    \begin{align*}
        \return & :\Obja\to\monA \Obja                                                   \\
        \lift   & :(\Obja\to \Objb)\to( \Obja\to\monA \Objb)                             \\
        \fish   & :(\Obja\to\monA \Objb)\to(\Objb\to\monA \Objc)\to(\Obja\to\monA \Objc) \\
        \mjoin  & :\monA{\monA \Obja}\to\monA \Obja                                      \\
        \fmap   & :(\Obja\to \Objb)\to(\monA \Obja\to\monA \Objb)                        \\
        \bind   & :\monA \Obja\to(\Obja\to\monA \Objb)\to\monA \Objb                     
    \end{align*}
    These maps satisfy the equivalent axioms of unitality and associativity:
    \begin{itemize}
        \item $\return$ is a left identity for $\bind$;
              % \todo[inline]{...}
        \item $\return$ is a right identity for $\bind$;
              % \todo[inline]{...}
        \item $\bind$ is associative.
              % \todo[inline]{...}
    \end{itemize}
\end{definition}

\todotextjira{234}{Spell out conditions in the definitions}
