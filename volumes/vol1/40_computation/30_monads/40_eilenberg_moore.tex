% !TEX root = chapter-standalone.tex

\section{Algebras of a monad}
\label{sec:Eilenberg-Moore}
\todotext{@JL: make the arbitrary operation a macro}

\linkvideo{spring2021-monads-b:another-persp-mon} % Another perspective on monads
\linkvideo{spring2021-monads-b:formal-expressions} % Formal expressions
\linkvideo{spring2021-monads-b:eval-formal-expr} % Evaluate formal expressions

\linkvideo{spring2021-monads-b:monad-formal-expr} % Monad for formal expressions

In the context of Kleisli morphisms, we developed the intuition that \SY{monads} can be used to encode ``generalized objects'' and ``generalized morphisms''.
In this section we will introduce a different intuition for \SY{monads}: that they can be used to provide coherent way to encode ``formal expressions'', together with a way to ``evaluate'' or ``compute'' such expressions.

Let us explain what we mean using an example.
Given a set \Obja, say
\begin{equation}
    \Obja = \makeset{ \sapple, \sbanana, \scarrot },
\end{equation}
let's define a certain type of ``formal expressions'', using elements of \Obja and using a ``formal composition symbol'', which we choose to be ``$*$''.
Now, the ``formal expressions'' we consider are all finite expressions which have a form such as
\begin{equation}
    \label{eq:formal-composition-1}
    \sbanana * \sbanana * \sapple,
\end{equation}
or
\begin{equation}
    \label{eq:formal-composition-2}
    \sapple * \sbanana * \sapple * \sapple * \sbanana * \sapple,
\end{equation}
and so on.
These expressions are ``formal'' (or ``purely symbolic'') in the sense that, a priori, the symbol ``$*$'' does not have any \emph{meaning} beyond simply being a symbol, a ``marking''.
After all, so far, \Obja is just a set, and we did not assume that it comes equipped with any sort of ``multiplication operation'', for instance.
In the following we will discuss a way of giving such formal expressions a ``computational meaning'' by specifying a way to evaluate them.

Before we come to this however, let us introduce a notation to explicitly distinguish when we are thinking about $\sapple$ as an element of \Obja, or $\sapple$ as a ``formal expression''.
\todotext{\alphubel: use a macro for the formal expression - also do not use the list brackets.}
For the latter situation we write ``$[\sapple]$''.
In other words, the square brackets indicate that $[\sapple]$ is a formal expression.
And we'll say that formal expressions can be combined, using $*$, to larger formal expressions.
So, following this convention, $[\sapple] * [\sbanana]$ is also a formal expression.
And in particular, instead of \cref{eq:formal-composition-2}, we'll write
\begin{equation}
    \label{eq:formal-composition-bracketed}
    [\sapple] * [\sbanana] * [\sapple] * [\sapple] * [\sbanana] * [\sapple].
\end{equation}
%
In an expression such as \cref{eq:formal-composition-bracketed} we'll think of the components $[\sapple]$ and $[\sbanana]$ as if they are ``letters'' (but we won't count ``$*$'' as a letter) and we'll think of the whole expression \cref{eq:formal-composition-bracketed} like a ``word''.
For any such word, we'll say its \emph{length} is the number of letters it is built from.
So, for instance, the word in \cref{eq:formal-composition-bracketed} has a length of $6$.
In our notion of formal expression, we'll choose to include a unique special word of length $0$, which we call the ``empty formal expression'' and denote by ``[ ]''.

Using square brackets we can also build ``formal expressions of formal expressions'' or ``second-order formal expressions''.
For example, given the formal expression $[\sapple] * [\sbanana]$, we can turn it into a second-order formal expression by putting brackets around it:
\begin{equation}
    \label{eq:double-formal-expression}
    [[\sapple] * [\sbanana]].
\end{equation}
And given another second-order formal expression, say $[[\sbanana] * [\sbanana] * [\sapple]]$, we can ``compose'' it with the one in \cref{eq:double-formal-expression} like so:
\begin{equation}
    \label{eq:double-formal-expression-comp}
    [[\sapple] * [\sbanana]] * [[\sbanana] * [\sbanana] * [\sapple]].
\end{equation}
The notion of length will also apply to second-order expressions.
For instance, the second-order expression \cref{eq:double-formal-expression-comp} has length 2, and is composed of one first-order expression of length 2 and one first-order expression of length 3.

This whole game can continue ad infinitum: we define third-order formal expressions to be those with three-layers of square brackets, fourth-order formal expressions have four layers of brackets, and so on.
In the following, we'll probably only ever consider up to three layers.

We started our story just with the set $\Obja = \makeset{ \sapple, \sbanana, \scarrot }$.
However, we can do the same construction -- using ``$*$'' and building formal expressions of any order -- with any set.
In fact, we can define a \SY{functor} $\funa: \Set \fto \Set$ which maps any set \Obja to the set $\funa \Obja$ whose elements are all finite first-order formal expressions built from elements of \Obja.
We also include this to mean the empty formal expression ``$[ ]$''.
What might this \SY{functor} do on the level of morphisms?
For concreteness, let $\Obja = \makeset{ \sapple, \sbanana, \scarrot } $ and $\setB = \makeset{ \swater, \stea }$.
Given a function $\mora \colon \Obja \sto \setB$, we define
\begin{equation}
    \funa(\mora) : \funa(\Obja) \to \funa(\Obja)
\end{equation}
to act on (first-order) expressions like so
\begin{equation}
    \funa(\mora)([\sapple]*[\sbanana]) = [\mora(\sapple)]*[\mora(\sbanana)].
\end{equation}
It turns out that this \SY{functor} $\funa: \Set \fto \Set$ can be made into a monad!

Let us explain how the unit and multiplication for this monad are defined.
For any set \Obja, the component of the unit at \Obja is
\todotext{\alphubel: use defmap}
\begin{equation}
    \monunit_\Obja : \Obja \sto \funa \Obja, \ela \mapsto [\ela].
\end{equation}
The multiplication is a bit more of a mouthful.
Its component at \Obja,
\begin{equation}
    \moncomp_\Obja : (\funa \fthen \funa)\Obja \sto \funa \Obja,
\end{equation}
is the function which takes a second-order formal expression
\begin{equation}\label{eq:monad-formatl-mult-input}
    \mF{
        \MF{ \MFb{\ela_{11}} , \MFb{\ela_{12}} , \mathcolor{level2}{\cdots} , \MFb{\ela_{1k_1}}}
        , \cdots ,
        \MF{\MFb{\ela_{n1}}, \MFb{\ela_{n2}}, \mathcolor{level2}{\cdots} , \MFb{\ela_{nk_n}}}
    }
\end{equation}
and ``collapses'' it to the first-order expression
\begin{equation}\label{eq:monad-formatl-mult-output}
    \mF { \MF{\ela_{11}}
        , \MF{\ela_{12}} , \cdots , \MF{\ela_{1k_1}}, \cdots, \MF{\ela_{n1}} , \MF{\ela_{n2}} * \cdots * \MF{\ela_{nk_n}}.
    }
\end{equation}
In other words, this operation simply ``removes the outer brackets'' from a second-order formal expression.

\begin{gradedexercise}[\exname{ListMonad}]
    \label{ex:ListMonad}
    Let $\funa : \Set \fto \Set$ be the \SY{functor} above that sends any set \Obja to the set of first-order formal expressions of the form
    \begin{equation}
        [\ela_1]
        * [\ela_2] * \cdots * [\ela_n] \quad \quad \ela_i \setin \Obja, n \setin \wnumbers_{\geq 0},
    \end{equation}
    and let $\moncomp_\Obja$ and $\monunit_\Obja$ be defined as above.

    Show that:
    \begin{enumerate}
        \item the components $\moncomp_\Obja$ define a \SY{natural transformation} $\moncomp : \funa \fthen \funa \nto \funa$;
        \item the components $\monunit_\Obja$ define a \SY{natural transformation} $\monunit : \stylefunctors{\text{Id}}_{\Set} \fto \funa$;
        \item $\moncomp$ and $\monunit$ satisfy the conditions for $\tup{\funa, \moncomp, \monunit}$ to be a monad.
    \end{enumerate}
\end{gradedexercise}

\solutionof{ListMonad}

Now let's finally talk about giving formal expressions a computational meaning: a way to \emph{evaluate} them.
The way we will do this is to define a notion of ``evaluation map'' $\stylemorph{a}: \funa \Obja \sto \Obja$ which we will interpret as a way of specifying how any formal expression -- an element of $\funa \Obja$ -- should be evaluated (or: computed) to an element of \Obja.
We will require such evaluation maps to additionally satisfy two coherence conditions, and the resulting mathematical structure will be what is called an \emph{algebra} of the monad $\funa$.

\todotextjira{233}{explain further by giving a concrete example of an algebra for this monad}

\linkvideo{spring2021-monads-b:algebra-monad} % Algebras for a monad
\begin{ctdefinition}[Algebra of a monad]
    \SYNDEF{algebra of a monad}
    \label{def:monad-algebra}
    Let~$\tup{\monA, \moncomp, \monunit}$ be a monad on a category~\CatC.
    An algebra of~$\monA$ (also called an $\monA$-algebra) is specified by: \

    \constit
    \begin{enumerate}
        \item an object $\Obja$ of~\CatC;
        \item a morphism $\monact\colon \monA(\Obja) \mto \Obja$ of \CatC.
    \end{enumerate}
    \condit
    \begin{enumerate}
        \item \emph{Composition}: the following diagram commutes:
              \equationsag{monalg-composition}{eq:monalg-composition}
        \item \emph{Unit}: the following diagram commutes:
              \equationsag{monalg-unit}{eq:monalg-unit}

    \end{enumerate}
\end{ctdefinition}

\linkvideo{spring2021-monads-b:actions-as-algebras} % Actions as M-algebras

\todotextjira{421}{@JL: Write some examples of Eilenberg-Moore algebras here.}

\linkvideo{spring2021-monads-b:morph-algebras} % Morphisms of M-algebras
\begin{ctdefinition}[$\monA$-algebra morphism]
    \label{def:algebramorphism}
    Let~$\tup{\monA, \monunit, \moncomp}$ be a monad on a category~\CatC, and let~$\tup{\Obja_1, \monact_1}$ and~$\tup{\Obja_2, \monact_2}$ be algebras of $\monA$.
    A morphism $\tup{\Obja_1, \monact_1} \mto \tup{\Obja_2, \monact_2}$ of $\monA$-algebras is specified by: \

    \constit
    \begin{enumerate}
        \item A morphism $\mora \colon \Obja_1 \mto \Obja_2$ in \CatC.
    \end{enumerate}
    \condit
    \begin{enumerate}
        \item The following diagram commutes:
              \equationsag{monalg-morph-compat}{eq:monalg-morph-compat}

    \end{enumerate}
\end{ctdefinition}

\linkvideo{spring2021-monads-b:cat-algebras} % Category of M-algebras
\begin{ctdefinition}
    [Category of $\monA$-algebras]
    \label{def:catofmonadalgebras}
    Let $\tup{\monA, \monunit, \moncomp}$ be a monad on a category \CatC.
    The \emph{category of $\monA$-algebras} $\CatC^\monA$ of the monad $\monA$ is specified by:
    \begin{enumerate}
        \item \emph{Objects}: $\monA$-algebras;
        \item \emph{Morphisms:} $\monA$-algebra morphisms;
        \item \emph{Identities}: given an $\monA$-algebra $\tup{\Obja, \monact}$, its \SY{identity morphism} is $\catidat\Obja$;
        \item \emph{Composition}: is induced by the composition of morphisms in \CatC.
    \end{enumerate}
\end{ctdefinition}

\devel{
    \subsection{Comparing the two perspectives}
    \linkvideo{spring2021-monads-b:comparing-perspectives} % Comparing the two perspectives
    \todotext{@JL: missing section}
}

\showslides{
    \begin{forslides}

        % 1 2 3
        % 6   4
        % 7 8 5   

        \SquareDiagramMorphism{1}{2}{3}{4}{5}{6}{7}{8}

        \SquareDiagramMapsto{1}{2}{3}{4}{5}{6}{7}{8}

        % 1 2 3
        %  \  4
        % 6   5   

        \TriangleDiagramMorphism{1}{2}{3}{4}{5}{6}

        \begin{tikzcd}[every arrow/.append style={arrowmorphismstyle}]
            \Obja \arrow[r, "\monunit_\Obja"] \arrow[dr, "\catid"'] & \monA(\Obja)  \arrow[d, "\monact"] \\
                                                                    & \Obja
        \end{tikzcd}

        \TriangleDiagramMorphism{\Obja}{\monunit_\Obja}{\monA(\Obja)}{\monact}{\Obja}{\catid}

        \begin{equation}\label{eq:MX-is-formal-product}
            \monA(\Obja) \definedas \text{``Formal products of elements of $\Obja$''}
        \end{equation}

        \begin{equation}\label{eq:a-is-operation-for-evaluating}
            \monAa  \stackrel{\monact}{\mto} \Obja \definedas \text{``operation for evaluating formal products in $\Obja$''}
        \end{equation}

        \begin{equation}\label{eq:monact2}
            \monAa  \stackrel{\monact}{\mto} \Obja
        \end{equation}

        \begin{equation}\label{eq:unit-of-the-monad}
            \defmapset{
                \monunit_\Obja
            }{
                \Obja
            }{
                \monAa
            }{
                x
            }{
                \MF{x}
            }
        \end{equation}
        \begin{equation}\label{eq:multiplication-of-the-monad}
            \moncomp_\Obja : \monA\monA\Obja \sto \monA \Obja
        \end{equation}

        \begin{equation}\label{eq:monad-xy}
            \SquareDiagramMorphism{\Obja}{\mora}{\Objb}{\monunit_\Objb}{\monA\Objb}%
            {\monunit_\Obja}{\monAa}{\monA\mora}
        \end{equation}
        \begin{equation}\label{eq:monad-xy-component}
            \SquareDiagramMapsto{x}{}{\mora x}{}{\MF{\mora x}}%
            {}{\MF{x}}{}
        \end{equation}
        \begin{equation}\label{eq:monad-multiplication1}
            \mF{\MF{\MFb{x_1}, \MFb{x_2}},\MF{\MFb{x_3}}} \mapsto \mF{\MF{x_1}, \MF{x_2}, \MF{x_3}}
        \end{equation}
        \begin{equation}\label{eq:monad-multiplication-component}
            \SquareDiagramMapsto{
                \mF{
                    \MF{
                        \MFb{\MFc{x_1},\MFc{x_2}}, \MFb{\MFc{x_4}}
                    },
                    \MF{\MFb{\MFc{x_4}}}}
            }{}{
                \mF{ \MF{\MFb{x_1}, \MFb{x_2}},  \MF{\MFb{x_3}, \MFb{x_4}}}
            }{}{
                \mF{\MF{x_1},\MF{x_2},\MF{x_3},\MF{x_4}}
            }{}{
                \mF{\MF{\MFb{x_1}, \MFb{x_2}, \MFb{x_3}}, \MF{\MFb{x_4}}}
            }{}
        \end{equation}

        \begin{equation}\label{eq:monad-ex1-set-1}
            \Obja = \makeset{\sbanana, \sapple, \scarrot, \dots}
        \end{equation}
        \begin{equation}\label{eq:monad-ex1-set-2}
            \monAa = \makeset{ \MF{}, \MF{}, \MF{2}, \cdots, \MF{\MFb{0}, \MFb{1}}, \cdots, \MF{\MFb{\MFc{12}}}, \cdots}
        \end{equation}
        \begin{equation}\label{eq:monad-ex1-set-2-bis}
            \monAa = \makeset{ \MF{}, \MF{\sbanana}, \MF{\scarrot}, \cdots, \mF{\MF{\sbanana}, \MF{\scarrot}}, \cdots, \mF{\MF{\sbanana}, \MF{\sbanana}, \MF{\scarrot}}, \cdots}
        \end{equation}

        \begin{equation}\label{eq:monad-ex1-functor-1}
            \mora\colon\Obja\mto\Objb
        \end{equation}
        \begin{equation}\label{eq:monad-ex1-functor-2}
            \monA\mora\colon\monA\Obja\mto\monA\Objb
        \end{equation}
        \begin{equation}\label{eq:monad-ex1-functor-3}
            \begin{aligned}
                \MF{}               & \mapsto \MF{} \\
                \mF{\MF{a}}         & \mapsto \mF{\MF{\mora(a)}} \\
                \mF{\MF{a}, \MF{b}} & \mapsto \mF{\MF{\mora(a)}, \MF{\mora(b)}}
            \end{aligned}
        \end{equation}
        \begin{equation}\label{eq:monad-ex1-functor-4}
            \mF{\MF{a}} \mapsto \mF{\MF{\mora(a)}}
        \end{equation}
        \begin{equation}\label{eq:monad-ex1-functor-5}
            \mF{\MF{a}, \MF{b}} \mapsto \mF{\MF{\mora(a)}, \MF{\mora(b)}}
        \end{equation}

        \begin{equation}\label{eq:monad-ex1-set-3}
            \monact \colon \MF{\MFb{1}, \MFb{1}} \mapsto 2  \qquad (= \text{``$1 + 1$''})
        \end{equation}

        \begin{equation}\label{eq:monad-ex1-set-4}
            \monact \colon \MF{\MFb{1}, \MFb{2}, \MFb{3}} \mapsto 6  \qquad (= \text{``$ 1 + 2 + 3$''})
        \end{equation}

        \begin{equation}
            \monAa =  \makeset{\makeset{\sapple},\makeset{\sbanana}}
        \end{equation}

        \begin{equation}
            \monAa =  \MF{\MF{}, \MF{\sapple}, \MF{\sbanana}}
        \end{equation}

        \begin{equation}\label{eq:monact}
            \monact\colon \monAa \mto \Obja
        \end{equation}
        \begin{equation}\label{eq:monact-without}
            \monAa \mto \Obja
        \end{equation}
        \begin{equation}\label{eq:monact-without2}
            \mora \colon \Obja \mto \monAa
        \end{equation}

        \begin{equation}\label{eq:addition-algebra-condition}
            \TriangleDiagramMapsto{x}{\monunit_\Obja}{\MF{x}}{\monact}{x}{\catid}
        \end{equation}

        \begin{equation}\label{eq:addition-algebra-condition-sq}
            \SquareDiagramMapsto{
                \mF{\MF{\MFb{1},\MFb{2}},\MF{\MFb{3}}}
            }{
                \monA \monact
            }{
                \mF{ \MF{\monact(\mFb{\MFb{1},\MFb{2}})},\MF{\monact(\MFb{3})}}
            }{
                \monact
            }{
                6
            }{
                \moncomp_\Obja
            }{
                \mF{\MF{1}, \MF{2}, \MF{3}}
            }{
                \monact
            }
        \end{equation}

        \begin{equation}\label{eq:writer-1}
            \writerM \Obja = \Obja \times \monoidAset
        \end{equation}

        \begin{equation}\label{eq:writer-2}
            \defmapset{
                \writerM \mora
            }{
                \Obja \times \monoidAset
            }{
                \Objb \times \monoidAset
            }{
                \tup{x, m}
            }{
                \tup{\mora(x), m}
            }
        \end{equation}
        \begin{equation}\label{eq:writer-monoid}
            \monoidA = \tupp{\monoidAset, \mtimes_{\monoidA}, \idmon_{\monoidA}}
        \end{equation}
        \begin{equation}\label{eq:writer-comp}
            \defmapset{
                \monunit_\Obja
            }{
                \Obja
            }{
                \Obja \times \monoidAset
            }{
                x
            }{
                \tup{x, \idmon_{\monoidAset}}
            }
        \end{equation}

        \begin{equation}\label{eq:writer-comp2}
            \defmapset{
                \moncomp_\Obja
            }{
                (\Obja \times \monoidAset) \times \monoidAset
            }{
                \Obja \times \monoidAset
            }{
                \tupp{\tupp{x,m_1 }, m_2}
            }{
                \tupp{x, m_1 \mtimes m_2}
            }
        \end{equation}

        % \begin{equation}\label{eq:writer-unit}
        %     \monunit_\Obja \colon  \Obja \sto  \Obja \times \monoidAset
        % \end{equation}

        % \begin{equation}\label{eq:writer-unit-diagram}
        % \SquareDiagramMorphism{\Obja}{\mora}{\Objb}{\monunit_\Objb}{\monA\Objb}%
        % {\monunit_\Obja}{\monAa}{\monA\mora}
        % \end{equation}

        \begin{equation}\label{eq:writer-algebra-condition-triangle-domain}
            \TriangleDiagramMorphism{\Obja}{\monunit_\Obja}{\Obja\times\monoidAset}{\monact}{\Obja}{\catid}
        \end{equation}
        \begin{equation}\label{eq:writer-algebra-condition-triangle}
            \TriangleDiagramMapsto{x}{\monunit_\Obja}{\tup{x, \idmon_{\monoidAset}}}{\monact}{x}{\catid}
        \end{equation}

        \begin{equation}\label{eq:writer-algebra-condition-square-domain}
            \SquareDiagramMorphism{
                (\Obja \times \monoidAset) \times \monoidAset
            }{
                \writerM \monact
            }{
                \Obja \times \monoidAset
            }{
                \monact
            }{
                \Obja
            }{
                \moncomp_\Obja
            }{
                \Obja \times \monoidAset
            }{
                \monact
            }
        \end{equation}

        \begin{equation}\label{eq:writer-algebra-condition-square}
            \SquareDiagramMapsto{
                \tup{\tupp{x, m_1}, m_2}
            }{
                \writerM \monact
            }{
                \tup{\monact(\tupp{x, m_1}), m_2}
            }{
                \monact
            }{
                \monact(\tup{x, m_1 \mtimes m_2}) = \monact(\tup{\monact(\tupp{x, m_1}), m_2})
            }{
                \moncomp_\Obja
            }{
                \tup{x, m_1 \mtimes m_2}
            }{
                \monact
            }
        \end{equation}

        \begin{equation}\label{eq:writer-is-monoid-action}
            \monact(\tup{x, m_1 \mtimes m_2}) = \monact(\tup{\monact(\tupp{x, m_1}), m_2})
        \end{equation}

        \begin{equation}\label{eq:writer-is-monoid-action-unit}
            \monact(\tup{x, \idmon_{\monoidAset}}) = x
        \end{equation}

        \begin{equation}\label{eq:writer-algebras}
            \monact \colon \Obja \times \monoidAset \mto \Obja
        \end{equation}

        \begin{equation}\label{eq:writer-algebra-morphism-square-domain}
            \SquareDiagramMorphism{
                \Obja_1 \times \monoidAset
            }{
                \writerM \mora
            }{
                \Obja_2 \times \monoidAset
            }{
                \monact_2
            }{
                \Obja_2
            }{
                \monact_1
            }{
                \Obja_1
            }{
                \mora
            }
        \end{equation}

        \begin{equation}\label{eq:formal-algebra-morphism-square-domain}
            \SquareDiagramMorphism{
                \monA \Obja_1
            }{
                \monA \mora
            }{
                \monA \Obja_1
            }{
                \monact_2
            }{
                \Obja_2
            }{
                \monact_1
            }{
                \Obja_1
            }{
                \mora
            }
        \end{equation}

        \begin{equation}\label{eq:formal-algebra-morphism-square-1-x}
            \Obja_1 = \posReals
        \end{equation}
        \begin{equation}\label{eq:formal-algebra-morphism-square-1-a}
            \begin{aligned}
                \monact_1 \colon \monA \posReals & \mto \posReals \\
                \MF{}                            & \mapsto 1 \\
                \MF{x}                           & \mapsto x \\
                \mF{\MF{x}, \MF{y}}              & \mapsto x \cdot y
            \end{aligned}
        \end{equation}

        \begin{equation}\label{eq:formal-algebra-morphism-square-2-x}
            \Obja_2 = \reals
        \end{equation}
        \begin{equation}\label{eq:formal-algebra-morphism-square-2-a}
            \begin{aligned}
                \monact_2 \colon \monA \reals & \mto \reals \\
                \MF{}                         & \mapsto 0 \\
                \MF{x}                        & \mapsto x \\
                \mF{\MF{x}, \MF{y}}           & \mapsto x + y
            \end{aligned}
        \end{equation}

        \begin{equation}\label{eq:formal-algebra-morphism-add-square-domain}
            \SquareDiagramMorphism{
                \monA \posReals
            }{
                \monA \mora
            }{
                \monA \reals
            }{
                \monact_2
            }{
                \reals
            }{
                \monact_1
            }{
                \posReals
            }{
                \mora
            }
        \end{equation}
        \begin{equation}\label{eq:formal-algebra-morphism-add-square}
            \SquareDiagramMapsto{
                \mF{\MF{x}, \MF{y}}
            }{
                \monA \mora
            }{
                \mF{\MF{\mora{(x)}}, \MF{\mora{(y)}}}
            }{
                \monact_2
            }{
                \mora(x + y)
                = \mora(x) \cdot \mora(y)\qquad
            }{
                \monact_1
            }{
                x + y
            }{
                \mora
            }
        \end{equation}

        \begin{equation}
            \label{eq:monad-fish-associative-2}
            \prftree{
                \mora\colon \Obja \mto \mathcal{\stylefunctors{P}} \Objb
            }{
                \quad
            }{
                \morb\colon \Objb \mto \mathcal{\stylefunctors{P}} \Objc
            }{
                \lift_{\Obja,\Objb}(\mora)\mthen\lift_{\Objb,\Objc}(\morb) =  \lift_{\Obja,\Objc}(\morab)
            }.
        \end{equation}

        \begin{equation}\label{eq:formal-algebra-morphism-square-result}
            \mora(x + y) = \mora(x) \cdot \mora(y) \qquad \Rightarrow \qquad \mora(x) = \exp(x)
        \end{equation}

    \end{forslides}
}

\begin{proposition}\label{prop:monad-monoid}
    There is an $\monA$-algebra structure on $\Obja$ if and only if there
    is a monoidal structure on $\Obja$.
    The correspondence is as follows:
    \begin{itemize}
        \item The neutral element of $\Obja$ corresponds to $\monact(\MF{})$.
        \item The composition in $\Obja$ corresponds to the action of $\monact$:
              \begin{equation}
                  \mF{\MF{x_1}, \MF{x_2}} \stackrel{\monact}{\mto} x_1 \mthen x_2.
              \end{equation}
        \item The associativity and unitality of monoid composition are encoded by the $\monA$-algebra axioms.
    \end{itemize}
\end{proposition}
\begin{equation}\label{eq:addition-algebra-1}
    \Obja = \makeset{0, 1, 2, 3, \dots }
\end{equation}
\begin{equation}\label{eq:addition-algebra-2}
    \begin{aligned}
        \monact \colon \monAa & \mto \Obja \\
        \MF{}                 & \mapsto 0 \\
        \MF{x}                & \mapsto x \\
        \mF{\MF{x}, \MF{y}}   & \mapsto x + y
    \end{aligned}
\end{equation}

\begin{gradedexercise}[\exname{HwkFreeAlgebras}]
    \label{ex:HwkFreeAlgebras}
    Let $\tup{\monA, \moncomp, \monunit}$ be a monad on a category $\CatC$, and let $\Obja$ be an object of $\CatC$.
    Your task is to prove that the object $\monA \Obja$, together with the morphism
    \begin{equation}
        \moncomp_\Obja \colon \monA \monA \Obja \mto \monA \Obja
    \end{equation}
    defines an algebra for the monad $\monA$.
    \emph{Hint}: use the axioms/conditions in the definition of a monad.
\end{gradedexercise}
\solutionof{HwkFreeAlgebras}
