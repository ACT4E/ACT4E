% !TEX root = chapter-standalone.tex

\section{The Kleisli construction}
\label{sec:Kleisli}

We return now to the discussion from the opening section of this chapter, in order to spell out further the relationship to \SY{monads}.

\linkvideo{spring2021-monads-a:generalized-morphisms} % Generalized morphisms
\begin{ctdefinition}[Kleisli morphisms]
    \label{def:kleisli-morphism}
    Let~$\tup{\monA, \moncomp, \monunit}$ be a monad on a category~\CatC, and let~$\Obja, \Objb \setin \ObC$.
    A \maindef{Kleisli morphism}~$\Obja \mto \Objb$ is a morphism of~\CatC of the form~$\Obja \mto \monA \Objb$.
\end{ctdefinition}

\begin{ctdefinition}[Kleisli composition]
    \label{def:kleisli-composition}
    Let~$\tup{\monA, \moncomp, \monunit}$ be a monad on a category~\CatC, let~$\Obja, \Objb, \Objc \setin \ObC$, and let~$\mora \colon \Obja \mto \monA \Objb$ and $\morb \colon \Objb \mto \monA \Objc$ be morphisms in \CatC (so, they are Kleisli morphisms).
    Their \maindef{Kleisli composition} is the morphism in \CatC given by the composition
    \begin{equation}
        \Obja \overset{\mora}{\mto} \monA(\Objb) \overset{\monA \morb}{\mto} (\monAA)(\Objc) \overset{\moncomp_\Objc}{\mto} \monA(\Objc).
    \end{equation}
\end{ctdefinition}

\begin{ctdefinition}[Kleisli category]
    \label{def:kleisli-category}
    Let~$\tup{\monA, \moncomp, \monunit}$ be a monad on a category \CatC.
    The \maindef{Kleisli category} $\CatC_\monA$ of the monad $\monA$ is specified by:
    \begin{enumerate}
        \item \emph{Objects}:~$\Ob(\CatC_\monA) \definedas \Ob(\CatC)$;
        \item \emph{Morphisms:}~$\Hom_{\CatC_\monA}(\Obja, \Objb) \definedas \Hom_{\CatC}(\Obja, \monA(\Objb))$;
        \item \emph{Identities}:~$\catidat\Obja \definedas \monunit_\Obja$;
        \item \emph{Composition}: \SY{Kleisli composition}.
    \end{enumerate}
\end{ctdefinition}

\begin{gradedexercise}[\exname{HwkRelKleisli}]
    \label{ex:HwkRelKleisli}

    Let \Set denote the category of sets and $\powerset : \Set \to \Set$ the powerset functor which assigns, to any given set \setA, the set of subsets of \setA.
    The endofunctor~$\powerset$ may be equipped with the structure of a monad~$\tup{\powerset, \moncomp, \monunit}$, where the components of the natural transformations~$\moncomp$ and~$\monunit$ are given, respectively, by the functions
    \begin{equation}
        \defmap{\moncomp_{\setA}}{\powerset \powerset \setA}{\sto}{\powerset \setA}{\setC}{\bigsetunion_{\setB \setin \setC} \setB}
    \end{equation}
    and
    \begin{equation}
        \defmapperiod{\monunit_{\setA}}{\setA}{\sto}{\powerset \setA}{\ela}{\makeset{\ela}}
    \end{equation}

    Let~$\Set_{\powerset}$ denote the Kleisli category of the monad~$\tup{\powerset, \moncomp, \monunit}$.
    The aim of this exercise is to show that the category~$\Set_{\powerset}$ and the category \Rel of sets and relations are isomorphic as categories.

    We define functors~$\funa\colon \Set_{\powerset} \fto \Rel$ and~$\funb\colon \Rel \fto \Set_{\powerset}$ as follows.
    On objects~$\funa$ is the identity function, and given a function~$\mora \colon \setA \mtoin{\Set} \powerset \setB$ (in other words, a Kleisli morphism~$\mora \colon \setA \mtoin{\Set_{\powerset}} \setB$) we let~$\funa(\mora)$ be the relation
    \begin{equation}
        \funa(\mora) = \makeset{\tup{\ela, \elb} \setin \setA \cartprod \setB \mid \elb \setin \mora(\ela)}.
    \end{equation}
    The functor $\funb$ is also defined to be the identity function on objects, and given a relation $\relA \colon \setA \mto \setB$, we let $\funb(\relA)$ be the Kleisli morphisms represented by the following function:
    \begin{equation}
        \funb(\relA)\colon \setA \mtoin{\Set} \powerset \setB, \ela \mapsto \makeset{\elb \setin\setB \mid \tup{\ela, \elb} \setin \relA}.
    \end{equation}

    Your tasks:
    \begin{enumerate}
        \item Prove that $\funa: \Set_{\powerset} \to \Rel$ and $\funb: \Rel \to \Set_{\powerset}$ are in fact functors.
        \item Prove that $\funa$ is an isomorphism of categories, with inverse $\funb$.
    \end{enumerate}
\end{gradedexercise}

\solutionof{HwkRelKleisli}