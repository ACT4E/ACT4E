% !TEX root = chapter-standalone.tex

\section{The Kleisli construction}
\label{sec:Kleisli}

We return now to the discussion from the opening section of this chapter, in order to spell out further the relationship to \SY{monads}.

\linkvideo{spring2021-monads-a:generalized-morphisms} % Generalized morphisms
\begin{ctdefinition}[Kleisli morphisms]
    \label{def:kleisli-morphism}
    Let~$\tup{\monA, \moncomp, \monunit}$ be a monad on a category~\CatC, and let~$\Obja, \Objb \setin \ObC$.
    A \maindef{Kleisli morphism}~$\Obja \mto \Objb$ is a morphism of~\CatC of the form~$\Obja \mto \monA \Objb$.
\end{ctdefinition}

\begin{ctdefinition}[Kleisli composition]
    \label{def:kleisli-composition}
    Let~$\tup{\monA, \moncomp, \monunit}$ be a monad on a category~\CatC, let~$\Obja, \Objb, \Objc \setin \ObC$, and let~$\mora \colon \Obja \mto \monA \Objb$ and $\morb \colon \Objb \mto \monA \Objc$ be morphisms in \CatC (so, they are Kleisli morphisms).
    Their \maindef{Kleisli composition} is the morphism in \CatC given by the composition
    \begin{equation}
        \Obja \overset{\mora}{\mto} \monA(\Objb) \overset{\monA \morb}{\mto} (\monAA)(\Objc) \overset{\moncomp_\Objc}{\mto} \monA(\Objc).
    \end{equation}
\end{ctdefinition}

\begin{ctdefinition}[Kleisli category]
    \label{def:kleisli-category}
    Let~$\tup{\monA, \moncomp, \monunit}$ be a monad on a category \CatC.
    The \maindef{Kleisli category} $\CatC_\monA$ of the monad $\monA$ is specified by:
    \begin{enumerate}
        \item \emph{Objects}:~$\Ob(\CatC_\monA) \definedas \Ob(\CatC)$;
        \item \emph{Morphisms:}~$\Hom_{\CatC_\monA}(\Obja, \Objb) \definedas \Hom_{\CatC}(\Obja, \monA(\Objb))$;
        \item \emph{Identities}:~$\catidat\Obja \definedas \monunit_\Obja$;
        \item \emph{Composition}: \SY{Kleisli composition}.
    \end{enumerate}
\end{ctdefinition}

