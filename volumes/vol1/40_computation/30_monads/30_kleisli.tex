% !TEX root = chapter-standalone.tex


\section{The Kleisli construction}
\label{sec:Kleisli}

We return now to the discussion from the opening section of this chapter, in order to spell out further the relationship to monads. 

\todotext{Place the definition of kleisli morphisms and the kleisli category of a monad here?}

\begin{ctdefinition}[Kleisli morphisms]\label{def:kleisli-morphisms}
Let $\tup{\monA, \monunit, \moncomp}$ be a monad on a category $\CatC$, and let $\Obja, \Objb \in \Ob_\CatC$. A 
\emph{Kleisli morphism} $\Obja \rightsquigarrow \Objb$ is morphism of $\CatC$ of the form $\Obja \mto \monA \Objb$. 
\end{ctdefinition}


\begin{ctdefinition}[Kleisli composition]\label{def:kleisli-composition}
Let $\tup{\monA, \monunit, \moncomp}$ be a monad on a category $\CatC$, let $\Obja, \Objb, \Objc \in \Ob_\CatC$,  and let $\mora : \Obja \mto \monA \Objb$ and $\morb : \Objb \mto \monA \Objc$ be morphisms in $\CatC$ (so, they are Kleisli morphisms). Their \emph{Kleisli composition} is the morphism in $\CatC$ given by the composition
\begin{equation}
\Obja \overset{\mora}{\mto} \monA \Objb \overset{\monA \morb}{\mto} \monA \monA \Objc \overset{\moncomp_\Objc}{\mto} \monA \Objc.
\end{equation}
\end{ctdefinition}


