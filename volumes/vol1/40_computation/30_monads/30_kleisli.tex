% !TEX root = chapter-standalone.tex


\section{The Kleisli construction}
\label{sec:Kleisli}

We return now to the discussion from the opening section of this chapter, in order to spell out further the relationship to monads. 

 \linkvideo{spring2021-monads-a:generalized-morphisms} % Generalized morphisms
\begin{ctdefinition}[Kleisli morphisms]
    \label{def:kleislimor}
Let~$\tup{\monA, \moncomp, \monunit}$ be a monad on a category~$\CatC$, and let~$\Obja, \Objb \in \Ob_\CatC$. A
\emph{Kleisli morphism}~$\Obja \mto \Objb$ is morphism of~$\CatC$ of the form~$\Obja \mto \monA \Objb$.
\end{ctdefinition}


\begin{ctdefinition}[Kleisli composition]
    \label{def:kleislicomp}
Let~$\tup{\monA, \moncomp, \monunit}$ be a monad on a category~$\CatC$, let~$\Obja, \Objb, \Objc \in \Ob_\CatC$,  and let~$\mora : \Obja \mto \monA \Objb$ and $\morb : \Objb \mto \monA \Objc$ be morphisms in $\CatC$ (so, they are Kleisli morphisms). Their \emph{Kleisli composition} is the morphism in $\CatC$ given by the composition
\begin{equation}
\Obja \overset{\mora}{\mto} \monA(\Objb) \overset{\monA \morb}{\mto} (\monAA)(\Objc) \overset{\moncomp_\Objc}{\mto} \monA(\Objc).
\end{equation}
\end{ctdefinition}

\begin{ctdefinition}[Kleisli category]
    \label{def:kleislicat}
Let $\tup{\monA, \moncomp, \monunit}$ be a monad on a category $\CatC$. The \emph{Kleisli category} $\CatC_\monA$ of the monad $\monA$ is specified by:
\begin{compactenum}
\item \emph{Objects}:~$\Ob(\CatC_\monA) \coloneqq \Ob(\CatC)$;
\item \emph{Morphisms:}~$\Hom_{\CatC_\monA}(\Obja, \Objb) \coloneqq \Hom_{\CatC}(\Obja, \monA(\Objb))$;
\item \emph{Identities}:~$\catid_\Obja \coloneqq \monunit_\Obja$;
\item \emph{Composition}: Kleisli composition. 
\end{compactenum}
\end{ctdefinition}


