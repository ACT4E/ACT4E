% !TEX root = chapter-standalone.tex

\section{Slides}
\label{sec:monads-slides}


\showslides{
    \begin{forslides}
        \begin{equation}
            \label{eq:set-i-want-1}
            \Obja
        \end{equation}
        %
        \begin{equation}
            \label{eq:set-i-want-2}
            \Objb
        \end{equation}
        %
        \begin{equation}
            \label{eq:set-i-want-3}
            \Objc
        \end{equation}
        %
        \begin{equation}
            \label{eq:morph-i-want-1}
            \mora \colon \Obja \mto \Objb
        \end{equation}
        %
        \begin{equation}
            \label{eq:morph-i-want-2}
            \morb \colon \Objb \mto \Objc
        \end{equation}
        %
        \begin{equation}
            \label{eq:set-apple-banana}
            \Obja = \makeset{ \sapple, \sbanana, \scarrot }
        \end{equation}
        %
        \begin{equation}
            \label{eq:fruit-sum-0}
            \sapple \sbanana \scarrot
        \end{equation}
        %
        \begin{equation}
            \label{eq:fruit-sum-1}
            \mF{ \MF {\sapple} ,   \MF{\sbanana} , \MF  {\scarrot} }
        \end{equation}
        %
        \begin{equation}
            \label{eq:fruit-sum-2}
            \mF{\MF{\sapple} , \MF{\sbanana } }
        \end{equation}
        %
        \begin{equation}
            \label{eq:fruit-sum-3}
            \mF{ \MF{\sbanana},  \MF{\scarrot} }
        \end{equation}
        %
        \begin{equation}
            \label{eq:fruit-sum-4}
            \MF{\MFb{\sapple}   , \MFb{\sbanana} , \MFb{\scarrot}}
        \end{equation}
        %
        \begin{equation}
            \label{eq:fruit-sum-5}
            \MF{\MFb{\sapple}}
        \end{equation}
        %
        \begin{equation}
            \label{eq:fruit-sum-6}
            \mF {\MF{\sapple}  ,  \MF{\sbanana}}
        \end{equation}
        \begin{equation}
            \label{eq:fruit-sum-6a}
            \MF{\sapple}
        \end{equation}
        %
        \begin{equation}
            \label{eq:fruit-sum-6b}
            \MF{\MFb{\sapple}}
        \end{equation}
        \begin{equation}
            \label{eq:fruit-sum-6-bis}
            \MF{\MFb{\sapple}  , \MFb{\sbanana}}
        \end{equation}
        %
        \begin{equation}
            \label{eq:fruit-sum-7}
            \mF {\MF{\MFb{\sapple}} , \MF{\MFb{\sbanana}}  , \MF{\MFb{\scarrot}}}
        \end{equation}
        %
        \begin{equation}
            \label{eq:fruit-sum-7-bis}
            \mF{[\MFb{\sapple}  , \MFb{\sbanana}] ,  [\MFb{\scarrot}]}
        \end{equation}
        %
        \begin{equation}
            \label{eq:fruit-sum-8}
            \mF{\MFb{\sapple}   , [\MFb{\sbanana}  * \MFb{\scarrot}]}
        \end{equation}
        %
        \begin{equation}
            \label{eq:fruit-sum-9}
            \mF{\MF{\MFb{\sapple}}  , \MF{\MFb{\sbanana}} ,  \MF{\MFb{\scarrot}}}
        \end{equation}
        %
        \begin{equation}
            \label{eq:fruit-sum-10}
            \mF{\MF{\sapple}   ,  \MF{\sapple}  ,  \MF{\sapple}}
        \end{equation}
        %
        \begin{equation}
            \label{eq:fruit-sum-11}
            \mF{\MF{\sapple}   , \MF{\sapple}  ,\MF{\scarrot}}
        \end{equation}
        %
        \begin{equation}
            \label{eq:fruit-sum-12}
            \mF{\MF{\sbanana} , \MF{\sapple}}
        \end{equation}
        %
        \begin{equation}
            \label{eq:fruit-sum-13}
            \mF{\MF{\sapple}   ,  \MF{\sbanana}  ,  \MF{\scarrot}}
        \end{equation}
        %
        \begin{equation}
            \label{eq:fruit-sum-14}
            \MF{\MFb{\sapple}   ,  \MFb{\sbanana}  ,  \MFb{\scarrot}}
        \end{equation}
        %
        \begin{equation}
            \label{eq:fruit-sum-15}
            \MF{\MFb{\sapple}  ,  \MFb{\sbanana}  ,  \MFb{\scarrot}}
        \end{equation}
        %
        \begin{equation}
            \label{eq:fruit-sum-16}
            \MF{\sapple}
        \end{equation}
        %
        \begin{equation}
            \label{eq:fruit-sum-17}
            \MF{\sbanana}
        \end{equation}
        %
        \begin{equation}
            \label{eq:fruit-sum-18}
            \MF{\scarrot}
        \end{equation}
        %
        \begin{equation}
            \label{eq:fruit-sum-20}
            \mF{
                \MF{\MFb{\sapple}  , \MFb{\sbanana}},
                \MF{\MFb{\sapple}   , \MFb{\MFc{\sapple}, \MFc{\sbanana}} ,  \MFb{\scarrot}},
                \MF{\MFb{\MFc{\sapple}}}
            }
        \end{equation}
        \begin{equation}
            \label{eq:fruit-sum-21-empty}
            \MF{}
        \end{equation}
        \begin{equation}
            \label{eq:fruit-sum-22-empty}
            \MF{\MFb{\MFc{}}}
        \end{equation}
        \begin{equation}
            \label{eq:monad-power-functor}
            \defmapset{
                \powerset \mora
            }{
                \powerset \Obja
            }{
                \powerset \Objb
            }{
                S
            }{
                \bigsetunion_{x\setin S} \mora(x)
            }
        \end{equation}
        %
        \begin{equation}
            \label{eq:monad-power-monunit}
            \defmapset{
                \monunit_\Obja
            }{
                \Obja
            }{
                \powerset\Obja
            }{
                x
            }{
                \makeset{ x }
            }
        \end{equation}

        \begin{equation}
            \label{eq:monad-power-monunit-power}
            \defmapset{
                \monunit_{\powerset \Obja}
            }{
                \powerset \Obja
            }{
                \powerset (\powerset\Obja)
            }{
                \makeset{a, b, c}
            }{
                \makeset{ \makeset{a, b, c } }
            }
        \end{equation}

        \begin{equation}
            \label{eq:monad-power-power-monunit}
            \defmapset{
                \powerset(\monunit_\Obja)
            }{
                \powerset \Obja
            }{
                \powerset (\powerset\Obja)
            }{
                \makeset{a, b, c}
            }{
                \makeset{ \makeset{a}, \makeset{ b}, \makeset{c} }
            }
        \end{equation}

        \begin{equation}
            \label{eq:monad-power-moncomp}
            \defmapset{
                \moncomp_{\Obja}
            }{
                \powerset (\powerset\Obja)
            }{
                \powerset \Obja
            }{
                \makeset{S_1, S_2, \dots }
            }{
                \bigsetunion_i S_i
            }
        \end{equation}

        \begin{equation}
            \label{eq:monad-power-power-moncomp}
            \defmapset{
                \powerset \moncomp_{\Obja}
            }{
                \powerset (\powerset (\powerset\Obja))
            }{
                \powerset (\powerset \Obja)
            }{
                \makeset{  \makeset{   \makeset{a}, \makeset{b} }, \makeset{ \makeset{c}, \makeset{d} }}
            }{
                \makeset{  \makeset{ a, b }, \makeset{ c, d } }
            }
        \end{equation}

        \begin{equation}
            \label{eq:monad-power-moncomp-power}
            \defmapset{
                \moncomp_{\powerset \Obja}
            }{
                \powerset (\powerset (\powerset\Obja))
            }{
                \powerset (\powerset \Obja)
            }{
                \makeset{  \makeset{   \makeset{a}, \makeset{b} }, \makeset{ \makeset{c}, \makeset{d} }}
            }{
                \makeset{    \makeset{a}, \makeset{b}  ,  \makeset{c}, \makeset{d}  }
            }
        \end{equation}

        \begin{equation}\label{eq:MX-is-formal-product}
            \monA(\Obja) \definedas \text{``Formal products of elements of $\Obja$''}
        \end{equation}

        \begin{equation}\label{eq:a-is-operation-for-evaluating}
            \monAa  \stackrel{\monact}{\mto} \Obja \definedas \text{``operation for evaluating formal products in $\Obja$''}
        \end{equation}

        \begin{equation}\label{eq:monact2}
            \monAa  \stackrel{\monact}{\mto} \Obja
        \end{equation}

        \begin{equation}\label{eq:unit-of-the-monad}
            \defmapset{
                \monunit_\Obja
            }{
                \Obja
            }{
                \monAa
            }{
                x
            }{
                \MF{x}
            }
        \end{equation}
        \begin{equation}\label{eq:multiplication-of-the-monad}
            \moncomp_\Obja : \monA\monA\Obja \sto \monA \Obja
        \end{equation}

        \begin{equation}\label{eq:monad-xy}
            \SquareDiagramMorphism{\Obja}{\mora}{\Objb}{\monunit_\Objb}{\monA\Objb}%
            {\monunit_\Obja}{\monAa}{\monA\mora}
        \end{equation}
        \begin{equation}\label{eq:monad-xy-component}
            \SquareDiagramMapsto{x}{}{\mora x}{}{\MF{\mora x}}%
            {}{\MF{x}}{}
        \end{equation}
        \begin{equation}\label{eq:monad-multiplication1}
            \mF{\MF{\MFb{x_1}, \MFb{x_2}},\MF{\MFb{x_3}}} \mapsto \mF{\MF{x_1}, \MF{x_2}, \MF{x_3}}
        \end{equation}
        \begin{equation}\label{eq:monad-multiplication-component}
            \SquareDiagramMapsto{
                \mF{
                    \MF{
                        \MFb{\MFc{x_1},\MFc{x_2}}, \MFb{\MFc{x_4}}
                    },
                    \MF{\MFb{\MFc{x_4}}}}
            }{}{
                \mF{ \MF{\MFb{x_1}, \MFb{x_2}},  \MF{\MFb{x_3}, \MFb{x_4}}}
            }{}{
                \mF{\MF{x_1},\MF{x_2},\MF{x_3},\MF{x_4}}
            }{}{
                \mF{\MF{\MFb{x_1}, \MFb{x_2}, \MFb{x_3}}, \MF{\MFb{x_4}}}
            }{}
        \end{equation}

        \begin{equation}\label{eq:monad-ex1-set-1}
            \Obja = \makeset{\sbanana, \sapple, \scarrot, \dots}
        \end{equation}
        \begin{equation}\label{eq:monad-ex1-set-2}
            \monAa = \makeset{ \MF{}, \MF{}, \MF{2}, \cdots, \MF{\MFb{0}, \MFb{1}}, \cdots, \MF{\MFb{\MFc{12}}}, \cdots}
        \end{equation}
        \begin{equation}\label{eq:monad-ex1-set-2-bis}
            \monAa = \makeset{ \MF{}, \MF{\sbanana}, \MF{\scarrot}, \cdots, \mF{\MF{\sbanana}, \MF{\scarrot}}, \cdots, \mF{\MF{\sbanana}, \MF{\sbanana}, \MF{\scarrot}}, \cdots}
        \end{equation}

        \begin{equation}\label{eq:monad-ex1-functor-1}
            \mora\colon\Obja\mto\Objb
        \end{equation}
        \begin{equation}\label{eq:monad-ex1-functor-2}
            \monA\mora\colon\monA\Obja\mto\monA\Objb
        \end{equation}
        \begin{equation}\label{eq:monad-ex1-functor-3}
            \begin{aligned}
                \MF{}               & \mapsto \MF{} \\
                \mF{\MF{a}}         & \mapsto \mF{\MF{\mora(a)}} \\
                \mF{\MF{a}, \MF{b}} & \mapsto \mF{\MF{\mora(a)}, \MF{\mora(b)}}
            \end{aligned}
        \end{equation}
        \begin{equation}\label{eq:monad-ex1-functor-4}
            \mF{\MF{a}} \mapsto \mF{\MF{\mora(a)}}
        \end{equation}
        \begin{equation}\label{eq:monad-ex1-functor-5}
            \mF{\MF{a}, \MF{b}} \mapsto \mF{\MF{\mora(a)}, \MF{\mora(b)}}
        \end{equation}

        \begin{equation}\label{eq:monad-ex1-set-3}
            \monact \colon \MF{\MFb{1}, \MFb{1}} \mapsto 2  \qquad (= \text{``$1 + 1$''})
        \end{equation}

        \begin{equation}\label{eq:monad-ex1-set-4}
            \monact \colon \MF{\MFb{1}, \MFb{2}, \MFb{3}} \mapsto 6  \qquad (= \text{``$ 1 + 2 + 3$''})
        \end{equation}

        \begin{equation}
            \monAa =  \makeset{\makeset{\sapple},\makeset{\sbanana}}
        \end{equation}

        \begin{equation}
            \monAa =  \MF{\MF{}, \MF{\sapple}, \MF{\sbanana}}
        \end{equation}

        \begin{equation}\label{eq:monact}
            \monact\colon \monAa \mto \Obja
        \end{equation}
        \begin{equation}\label{eq:monact-without}
            \monAa \mto \Obja
        \end{equation}
        \begin{equation}\label{eq:monact-without2}
            \mora \colon \Obja \mto \monAa
        \end{equation}

        \begin{equation}\label{eq:addition-algebra-condition}
            \TriangleDiagramMapsto{x}{\monunit_\Obja}{\MF{x}}{\monact}{x}{\catid}
        \end{equation}

        \begin{equation}\label{eq:addition-algebra-condition-sq}
            \SquareDiagramMapsto{
                \mF{\MF{\MFb{1},\MFb{2}},\MF{\MFb{3}}}
            }{
                \monA \monact
            }{
                \mF{ \MF{\monact(\mFb{\MFb{1},\MFb{2}})},\MF{\monact(\MFb{3})}}
            }{
                \monact
            }{
                6
            }{
                \moncomp_\Obja
            }{
                \mF{\MF{1}, \MF{2}, \MF{3}}
            }{
                \monact
            }
        \end{equation}

        \begin{equation}\label{eq:writer-1}
            \writerM \Obja = \Obja \times \monoidAset
        \end{equation}

        \begin{equation}\label{eq:writer-2}
            \defmapset{
                \writerM \mora
            }{
                \Obja \times \monoidAset
            }{
                \Objb \times \monoidAset
            }{
                \tup{x, m}
            }{
                \tup{\mora(x), m}
            }
        \end{equation}
        \begin{equation}\label{eq:writer-monoid}
            \monoidA = \tupp{\monoidAset, \mtimes_{\monoidA}, \idmon_{\monoidA}}
        \end{equation}
        \begin{equation}\label{eq:writer-comp}
            \defmapset{
                \monunit_\Obja
            }{
                \Obja
            }{
                \Obja \times \monoidAset
            }{
                x
            }{
                \tup{x, \idmon_{\monoidAset}}
            }
        \end{equation}

        \begin{equation}\label{eq:writer-comp2}
            \defmapset{
                \moncomp_\Obja
            }{
                (\Obja \times \monoidAset) \times \monoidAset
            }{
                \Obja \times \monoidAset
            }{
                \tupp{\tupp{x,m_1 }, m_2}
            }{
                \tupp{x, m_1 \mtimes m_2}
            }
        \end{equation}

        % \begin{equation}\label{eq:writer-unit}
        %     \monunit_\Obja \colon  \Obja \sto  \Obja \times \monoidAset
        % \end{equation}

        % \begin{equation}\label{eq:writer-unit-diagram}
        % \SquareDiagramMorphism{\Obja}{\mora}{\Objb}{\monunit_\Objb}{\monA\Objb}%
        % {\monunit_\Obja}{\monAa}{\monA\mora}
        % \end{equation}

        \begin{equation}\label{eq:writer-algebra-condition-triangle-domain}
            \TriangleDiagramMorphism{\Obja}{\monunit_\Obja}{\Obja\times\monoidAset}{\monact}{\Obja}{\catid}
        \end{equation}
        \begin{equation}\label{eq:writer-algebra-condition-triangle}
            \TriangleDiagramMapsto{x}{\monunit_\Obja}{\tup{x, \idmon_{\monoidAset}}}{\monact}{x}{\catid}
        \end{equation}

        \begin{equation}\label{eq:writer-algebra-condition-square-domain}
            \SquareDiagramMorphism{
                (\Obja \times \monoidAset) \times \monoidAset
            }{
                \writerM \monact
            }{
                \Obja \times \monoidAset
            }{
                \monact
            }{
                \Obja
            }{
                \moncomp_\Obja
            }{
                \Obja \times \monoidAset
            }{
                \monact
            }
        \end{equation}

        \begin{equation}\label{eq:writer-algebra-condition-square}
            \SquareDiagramMapsto{
                \tup{\tupp{x, m_1}, m_2}
            }{
                \writerM \monact
            }{
                \tup{\monact(\tupp{x, m_1}), m_2}
            }{
                \monact
            }{
                \monact(\tup{x, m_1 \mtimes m_2}) = \monact(\tup{\monact(\tupp{x, m_1}), m_2})
            }{
                \moncomp_\Obja
            }{
                \tup{x, m_1 \mtimes m_2}
            }{
                \monact
            }
        \end{equation}

        \begin{equation}\label{eq:writer-is-monoid-action}
            \monact(\tup{x, m_1 \mtimes m_2}) = \monact(\tup{\monact(\tupp{x, m_1}), m_2})
        \end{equation}

        \begin{equation}\label{eq:writer-is-monoid-action-unit}
            \monact(\tup{x, \idmon_{\monoidAset}}) = x
        \end{equation}

        \begin{equation}\label{eq:writer-algebras}
            \monact \colon \Obja \times \monoidAset \mto \Obja
        \end{equation}

        \begin{equation}\label{eq:writer-algebra-morphism-square-domain}
            \SquareDiagramMorphism{
                \Obja_1 \times \monoidAset
            }{
                \writerM \mora
            }{
                \Obja_2 \times \monoidAset
            }{
                \monact_2
            }{
                \Obja_2
            }{
                \monact_1
            }{
                \Obja_1
            }{
                \mora
            }
        \end{equation}

        \begin{equation}\label{eq:formal-algebra-morphism-square-domain}
            \SquareDiagramMorphism{
                \monA \Obja_1
            }{
                \monA \mora
            }{
                \monA \Obja_1
            }{
                \monact_2
            }{
                \Obja_2
            }{
                \monact_1
            }{
                \Obja_1
            }{
                \mora
            }
        \end{equation}

        \begin{equation}\label{eq:formal-algebra-morphism-square-1-x}
            \Obja_1 = \posReals
        \end{equation}
        \begin{equation}\label{eq:formal-algebra-morphism-square-1-a}
            \begin{aligned}
                \monact_1 \colon \monA \posReals & \mto \posReals \\
                \MF{}                            & \mapsto 1 \\
                \MF{x}                           & \mapsto x \\
                \mF{\MF{x}, \MF{y}}              & \mapsto x \cdot y
            \end{aligned}
        \end{equation}

        \begin{equation}\label{eq:formal-algebra-morphism-square-2-x}
            \Obja_2 = \reals
        \end{equation}
        \begin{equation}\label{eq:formal-algebra-morphism-square-2-a}
            \begin{aligned}
                \monact_2 \colon \monA \reals & \mto \reals \\
                \MF{}                         & \mapsto 0 \\
                \MF{x}                        & \mapsto x \\
                \mF{\MF{x}, \MF{y}}           & \mapsto x + y
            \end{aligned}
        \end{equation}

        \begin{equation}\label{eq:formal-algebra-morphism-add-square-domain}
            \SquareDiagramMorphism{
                \monA \posReals
            }{
                \monA \mora
            }{
                \monA \reals
            }{
                \monact_2
            }{
                \reals
            }{
                \monact_1
            }{
                \posReals
            }{
                \mora
            }
        \end{equation}
        \begin{equation}\label{eq:formal-algebra-morphism-add-square}
            \SquareDiagramMapsto{
                \mF{\MF{x}, \MF{y}}
            }{
                \monA \mora
            }{
                \mF{\MF{\mora{(x)}}, \MF{\mora{(y)}}}
            }{
                \monact_2
            }{
                \mora(x + y)
                = \mora(x) \cdot \mora(y)\qquad
            }{
                \monact_1
            }{
                x + y
            }{
                \mora
            }
        \end{equation}

        \begin{equation}
            \label{eq:monad-fish-associative-2}
            \prftree{
                \mora\colon \Obja \mto \mathcal{\stylefunctors{P}} \Objb
            }{
                \quad
            }{
                \morb\colon \Objb \mto \mathcal{\stylefunctors{P}} \Objc
            }{
                \lift_{\Obja,\Objb}(\mora)\mthen\lift_{\Objb,\Objc}(\morb) =  \lift_{\Obja,\Objc}(\morab)
            }.
        \end{equation}

        \begin{equation}\label{eq:formal-algebra-morphism-square-result}
            \mora(x + y) = \mora(x) \cdot \mora(y) \qquad \Rightarrow \qquad \mora(x) = \exp(x)
        \end{equation}

    \end{forslides}
}

