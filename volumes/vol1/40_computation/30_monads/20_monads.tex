% !TEX root = chapter-standalone.tex


\section{Monads}
\label{sec:monads}

In the previous section we considered the operation of taking the powerset of a set. This was used to think about generalized sets and generalized functions: given a set $\setA$, we thought of its powerset $\powerset \setA$ as a set whose elements -- subsets of $\setA$ -- are ``generalized elements of $\setA$''. Furthermore we saw that it was possible to generalize a function $\mora : \setA \mto \setB$ to a function $\widehat \mora:  \setA \mto \powerset \setB$, and that such generalized functions could be composed. In this section we will see that this construction is possible because the operation of ``taking the powerset'' has a special structure to it: it is an example of a \emph{monad}. 

Before introducing the the general definition, let us approach this concept by listing some of the properties that make the ``powerset operation''  a monad. 

Firstly, note that the ``powerset operation'' can be viewed as functor $\Set \mto \Set$. For each object -- that is, for each set $\setA$ -- it produces a new set, $\powerset \setA$. For each function $\mora: \setA \mto \setB$ we define $\powerset \mora : \powerset \setA \mto \powerset \setB$ by
\begin{equation}\label{eq:powerset-functor}
 \subA \longmapsto \bigcup_{\ela \in \subA} \mora(\ela).
\end{equation}
In other words, $\powerset \mora$ maps a subset $\subA \subseteq \setA$ to its image under $\mora$, which is a subset of $\setB$. 


[INSERT EXERCISE: check that this is a functor]


Second, as noted in the previous section, each element $\ela \in \setA$ can be viewed as a ``generalized element'' $\{ \ela \} \in \powerset \setA$. So we have, for each set, a function $\monunit_\setA : \setA \to \powerset \setA, \ela \mapsto \{ \ela \}$. 
Furthermore, these maps are ``coherent'' with morphisms in $\Set$ in the sense that ...  In other words, the diagram ... commutes. This means precisely that the maps ... are the components of a natural transformation .... . 

Third, consider the question of how to compose generalized morphisms -- that is, how to define the mape \emph{fish}... We saw that... 

\begin{ctdefinition}[Monad]
  \label{def:monad}
  Let \CatC be a category. A \emph{monad} on \CatC is specified by:
  \constit
  \begin{compactenum}
    \item A functor~$ \monA \colon \CatC \fto \CatC$;
    \item A natural transformation~$\monunit \colon \funid_\CatC \nto \monA$, called the \emph{unit};
    \item A natural transformation~$\moncomp \colon \monAA \nto \monA$, called the \emph{composition} or \emph{multiplication}.
  \end{compactenum}
  \condit
  \begin{compactenum}
 \item \emph{Left and right unitality}: the diagrams
  \begin{center}
    \includesag{55_monad_1}
  \end{center}
  must commute. 
  \item \emph{Associativity}: the diagram
  \begin{center}
    \includesag{55_monad_2}
  \end{center}
  must commute.
  \end{compactenum}
\end{ctdefinition}
\todographics{In the diagram above, can we have Rightarrow without the white filling?}

\begin{remark}
In terms of components, the unitality conditions state that for every object $\Obja \in \ObC$, the diagrams 

[INSERT DIAGRAMS]

commute. And the associativity condition states that for every object $\Obja \in \ObC$, 

[INSERT DIAGRAM]

commutes. 
\end{remark}


\todotext{Let's add also the diagrams above in components - more verbose but clearer. }
\todotext{I'm not sure if we every introduced ``whiskering'' and the notation for it (which is used in the diagrams in the definition of a monad). We should check and if needed introduce this in the section on natural transformations.}

\

\todotext{explain how the three conditions in the definition of a monad spell out in the case of the powerset monad (as Paolo does in his notes)}




\devel{

\begin{forslides}
...

\end{forslides}

}



\begin{exercise}
  \todotext{Dummy exercise to check code}
\end{exercise}
\begin{solution}
  \todotext{to write}
\end{solution}


\section{Monads, computer sceince definition}

\begin{definition}[Monad (computer science)]\label{def:monad-computer-science}
  A monad \begin{equation}
    \tup{ \monA,\return,\mjoin,\fmap,\bind,\fish,\lift} 
  \end{equation}
  is a set of operations with the following signature:
  
  \begin{align*}
  \return & :A\to\monA A\\
  \mjoin & :\monA{\monA A}\to\monA A\\
  \fmap & :(A\to B)\to(\monA A\to\monA B)\\
  \lift & :(A\to B)\to( A\to\monA B)\\
  \bind & :\monA A\to(A\to\monA B)\to\monA B\\
  \fish & :(A\to\monA B)\to(B\to\monA C)\to(A\to\monA C)
  \end{align*}
  $\return$ is also known as ``return''
  
  These maps satisfy the equivalent axioms of unitality and associativity:
  \begin{itemize}
  \item $\return$ is a left identity for $\bind$:
  
  \todo[inline]{...}
  \item $\return$ is a right identity for $\bind$:
  
  \todo[inline]{...}
  \item $\bind$ is associative
  
  \todo[inline]{...}
  \end{itemize}
\end{definition}