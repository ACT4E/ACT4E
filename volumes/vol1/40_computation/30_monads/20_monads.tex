% !TEX root = chapter-standalone.tex


\section{Monads}
\label{sec:monads}

\begin{ctdefinition}[Monad]
  \label{def:monad}
  Let \CatC be a category. A \emph{monad} on \CatC is specified by:
  \constit
  \begin{compactenum}
    \item A functor~$ \monA \colon \CatC \fto \CatC$;
    \item A natural transformation~$\monunit \colon \funid_\CatC \nto \monA$, called the \emph{unit};
    \item A natural transformation~$\moncomp \colon \monAA \nto \monA$, called the \emph{composition} or \emph{multiplication}.
  \end{compactenum}
  \condit
  \begin{compactenum}
 \item \emph{Left and right unitality}: the diagrams
  \begin{center}
    \includesag{55_monad_1}
  \end{center}
  must commute. 
  \item \emph{Associativity}: the diagram
  \begin{center}
    \includesag{55_monad_2}
  \end{center}
  must commute.
  \end{compactenum}
\end{ctdefinition}
\todographics{In the diagram above, can we have Rightarrow without the white filling?}
\todotext{Let's add also the diagrams above in components - more verbose but clearer. }
\todotext{I'm not sure if we every introduced ``whiskering'' and the notation for it (which is used in the diagrams in the definition of a monad). We should check and if needed introduce this in the section on natural transformations.}



\begin{exercise}
  \todotext{Dummy exercise to check code}
\end{exercise}
\begin{solution}
  \todotext{to write}
\end{solution}
