% !TEX root = chapter-standalone.tex

%
%\section{Using monads to understand uncertainty}
%
%\AC{is this a thing?? }
%Take the $\mathsf{Unc}$ functor $\mathsf{Unc}\colon \DP\to \DP$ which
%\begin{compactenum}
%  \item Maps an object $P$ in \DP (poset) to its twisted arrow category $\twisted{P}$, representing a poset interval.
%  \item Maps a morphism in \DP $d\colon \funsp \profto \ressp$ to $\tup{\low d,\upp d}$, where
%  \begin{equation}
%    \begin{aligned}
%      \low d\colon \F{F_{\low}}&\profto \R{R_{\low}},\\
%      \upp d\colon \F{F_{\upp}}&\profto \R{R_{\upp}},
%    \end{aligned}
%  \end{equation}
%  and $\tup{\low d,\upp d}$ is a boolean profunctor of the form
%  \begin{equation}
%    \begin{aligned}
%      &\tup{\low d,\upp d}\colon \left(\F{F_{\low}}\times \F{F_{\upp}} \right)\op \times \left(\R{R_{\low}}\times \R{R_{\upp}} \right)\toinPos \Bool\\
%      &\tup{\tup{\F{f_{\low}},\F{f_{\upp}}}^*,\tup{\R{r_{\low}},\R{r_{\upp}}}}\mapsto \low d(\F{f_{\low}}^*,\R{r_{\low}})\wedge \upp d(\F{f_{\upp}}^*,\R{r_{\upp}})
%    \end{aligned}
%  \end{equation}
%\end{compactenum}
%
%\todo{write down better}
%Is this a functor?
%
%\begin{proof}
%  Consider two design problems $f\colon \F{A}\profto \R{B}$ and $g\colon \F{B}\profto \R{C}$.
%
%  We know that
%  \begin{equation}
%    \begin{aligned}
%      \unc(f)&\colon \left( \F{A_{\low}}\times \F{A_{\upp}}\right)\op \times \left( \R{B_{\low}}\times \R{B_{\upp}}\right)\toinPos \Bool\\
%      \unc(g)&\colon \left( \F{B_{\low}}\times \F{B_{\upp}}\right)\op \times \left( \R{C_{\low}}\times \R{C_{\upp}}\right)\toinPos \Bool.
%    \end{aligned}
%  \end{equation}
%
%  We have
%  \begin{equation}
%    \begin{aligned}
%      &\left(\unc(f)\then \unc(g)\right) (\tup{\ubar{a},\bar{a}}^*,\tup{\ubar{c},\bar{c}})\\
%      &=\bigvee_{\tup{\ubar{b},\bar{b}}\in B_{\low}\times B_{\upp}} \unc(f)(\tup{\ubar{a},\bar{a}}^*,\tup{\ubar{b},\bar{b}})\wedge \unc(g)(\tup{\ubar{b},\bar{b}}^*,\tup{\ubar{c},\bar{c}})\\
%      &= \bigvee_{\tup{\ubar{b},\bar{b}}\in B_{\low}\times B_{\upp}} \low f(\ubar{a}^*,\ubar{b})\wedge \upp f(\bar{a}^*,\bar{b})\wedge \low g(\ubar{b}^*,\ubar{c})\wedge \upp g(\bar{b}^*,\bar{c})\\
%      &=(\low f\then \low g)(\ubar{a}^*,\ubar{c})\wedge (\upp f\then \upp g)(\bar{a}^*,\bar{c})\\
%      &=\low f\then g (\ubar{a}^*,\ubar{c})\wedge \upp f\then g(\bar{a}^*,\bar{c})\\
%      &=\unc(f\then g)(\tup{\ubar{a},\bar{a}}^*,\tup{\ubar{c},\bar{c}}).
%    \end{aligned}
%  \end{equation}
%\end{proof}
%

\section{$\UPos$ and $\LPos$ categories}

\begin{definition}[Category \UPos]
\label{def:upos_cat}
The category \UPos consists of:
\begin{compactenum}
    \item \emph{Objects}: objects are posets;
    \item \emph{Morphisms}: given objects~$\Obja,\Objb\in \Ob_\UPos$, morphisms from~$\Obja$ to~$\Objb$ are monotone maps of the form~$\mora \colon \Obja \to \Up{\Objb}$.
    \item \emph{Composition of morphisms}: Given morphisms $\mora\colon \Obja \to \Up{\Objb}$~ $\morb\colon \Objb\to \Up{\Objc}$, their composition is given as
    \begin{equation}
    \begin{aligned}
        \mora \mthen \morb \colon \Obja&\to \Up{\Objc}\\
        \Objael&\mapsto \bigcup_{\Objbel\in \mora(\Objael)}\morb(\Objbel);
    \end{aligned}
    \end{equation}
    \item \emph{Identity morphism}: given an object~$\Obja\in \Ob_\UPos$, the identity morphism is given by the application of the upper closure operator:~$\catid_\Obja(\Objael)\definedas \upit\{\Objael\}$.
\end{compactenum}
\end{definition}
\begin{remark}
Note that the composition of morphisms in this category corresponds to the generalization of the series operator for boolean profunctors.
\end{remark}

Analogously, we can define the \LPos category.
\begin{definition}[Category \LPos]
\label{def:lpos_cat}
The category \LPos consists of:
\begin{compactenum}
    \item \emph{Objects}: objects are posets;
    \item \emph{Morphisms}: given objects~$\Obja,\Objb\in \Ob_\LPos$, morphisms from~$\Obja$ to~$\Objb$ are monotone maps of the form~$\mora \colon \Obja \to \Lo{\Objb}$.
    \item \emph{Composition of morphisms}: Given morphisms $\mora \colon \Obja \mto \Lo{\Objb}$, $\morb \colon \Objb\mto \Lo{\Objc}$, their composition is given by
    \begin{equation}
    \begin{aligned}
        \mora \mthen \morb \colon \Obja &\mto \Up{\Objc}\\
        \Objael&\mapsto \bigcup_{\Objbel\in \mora (\Objael)}\morb(\Objbel);
    \end{aligned}
    \end{equation}
    \item \emph{Identity morphism}: given an object $\Obja\in \Ob_\LPos$, the identity morphism is given by the application of the lower closure operator:~$\catid_\Obja(\Objael)\definedas \downit\{\Objael\}$.
\end{compactenum}
\end{definition}
We now show that \UPos and \LPos are indeed categories.
\begin{lemma}
\label{lem:upos_lpos_cats}
\UPos and \LPos are categories.
\end{lemma}

\begin{proof}
We prove that \UPos is a category. The proof for \LPos is analogous. In the following, we show unitality and associativity.
\paragraph*{Unitality} Given~$\mora\colon \Obja \mto \Up{\Objb}$, we have:
\begin{equation*}
    \begin{aligned}
    \left( \mora \mthen \catid_\Objb\right)(\Objael)&=\bigcup_{\Objbel\in \mora(\Objael)}\catid_\Objb(\Objbel)\\
    &=\bigcup_{\Objbel\in \mora(\Objael)}\upit\{\Objbel\}\\
    &=\bigcup_{\Objbel\in \mora(\Objael)}\{\Objbel'\in \Objb \colon \Objbel\posleq_\Objb \Objbel' \}
    \end{aligned}
\end{equation*}
We know that~$\mora(\Objael)$ is an upperset:
\begin{equation*}
    \begin{aligned}
    \mora(\Objael)&=\bigcup_{\Objbel\in \mora(\Objael)}\{\Objbel\}\\
    &=\bigcup_{\Objbel\in \mora(\Objael)}\{ \Objbel'\in \Objb \colon \Objbel\posleq_\Objb \Objbel'\}.
    \end{aligned}
\end{equation*}
Therefore,~$\left( \mora \mthen \catid_\Objb\right)(\Objael)=\mora(\Objael)$ for all~$\Objael\in \Obja$. Similarly, we have:
\begin{equation*}
    \begin{aligned}
    (\catid_\Obja \mthen \mora)(\Objael)&=\bigcup_{\Objael'\in \catid_\Obja(\Objael)}\mora(\Objael')\\
    &=\bigcup_{\Objael'\in \upit \{\Objael\}}\mora(\Objael')\\
    &=\mora(\Objael),
    \end{aligned}
\end{equation*}
where the last equality holds since~$\mora$ is a monotone function and~$\mora(\Objael')\subseteq \mora(\Objael)$ for all~$\Objael'\in \upit \{\Objael\}$.
\paragraph*{Associativity} Let's consider three morphisms~$\mora\colon \Obja \mto \Up{\Objb}$,~$\morb\colon \Objb\mto \Up{\Objc}$, and~$\morc\colon \Objc\mto \Up{\Objd}$. We have:
\begin{equation*}
    \begin{aligned}
    \left( \left( \mora \mthen \morb\right)\mthen\morc\right)(\Objael)&=\bigcup_{\Objcel \in \left( \bigcup_{\Objbel \in \mora(\Objael)}\morb(\Objbel)\right)}\morc(\Objcel)\\
    &=\bigcup_{\Objbel\in \mora(\Objael)}\bigcup_{\Objcel\in \morb(\Objbel)}\morc(\Objcel)\\
    &=\left( \mora \mthen \left( \morb\then \morc\right)\right)(\Objael).
    \end{aligned}
\end{equation*}
Therefore, \UPos is a category.
\end{proof}

\begin{definition}[Equivalence of categories]
\label{def:equivalence_cat}
An \emph{equivalence} between two categories \CatC and \CatD is given by:
\begin{compactenum}
\item A pair of functors~$\funa,\funb$ of the form:
    \begin{equation}
    \includesag{cat_equiv}
    \end{equation}
\item natural isomorphisms~$\funa\then \funb\cong \catid_\CatC$ and~$\funb\then \funa\cong \catid_\CatD$.
\end{compactenum}
\end{definition}
We can show that \UPos and \LPos are equivalent.

\begin{lemma}
\label{lem:ulposequiv}
\UPos and \LPos are equivalent: there exists a pair of functors
\begin{equation}
    \begin{aligned}
    \funeqa\colon \UPos&\fto \LPos,\\
    \funeqb\colon \LPos&\fto \UPos,
    \end{aligned}
\end{equation}
such that~$\funeqa \then \funeqb=\funid_{\UPos}$ and~$\funeqb \then \funeqa=\funid_{\LPos}$, where~$\funid_{\UPos}$ and~$\funid_{\LPos}$ are the identity functors on $\UPos$ and~$\LPos$, respectively.
\end{lemma}

\begin{proof}
To prove this, we need to define the needed functors and to show that they satisfy the listed properties.
We choose the functors to be the ones that map a poset~$\posA$ in a category to its opposite version~$\posA\op$ in another category.
Given a morphism~$\mora\colon \Obja \mto \Up{\Objb}$ in \UPos, we have:
\begin{equation*}
    \begin{aligned}
    \funeqa(\mora)\colon \Obja\op &\mto \Lo(\Objb \op)\\
    \Objael&\mapsto \mora(\Objael).
    \end{aligned}
\end{equation*}
Given a morphism~$\morb\colon \Obja \mto \Lo{\Objb}$ in \LPos, we have:
\begin{equation*}
    \begin{aligned}
    \funeqb(\morb)\colon \Obja\op &\mto \Up(\Objb \op)\\
    \Objael&\mapsto \morb(\Objael).
    \end{aligned}
\end{equation*}
\paragraph*{$\funeqa$ and~$\funeqb$ are functors}
\begin{itemize}
    \item \emph{Preservation of identites}: Given~$\Obja\in \Ob_\UPos$, we have:
    \begin{equation*}
        \begin{aligned}
            \funeqa(\catid_\Obja)&=\upit_\Obja\{\Objael\}\\
            &=\downit_{\Obja\op}\{\Objael\}\\
            &=\catid_{\Obja\op},
        \end{aligned}
    \end{equation*}
where~$\catid_\Obja$ is an identity morphism in \UPos, and~$\catid_{\Obja\op}$ is an identity morphism in \LPos. Similarly, given~$\Obja\in \Ob_\LPos$ we have:
\begin{equation*}
    \begin{aligned}
    \funeqb(\catid_\Obja)&=\downit_\Obja\{\Objael\}\\
    &=\upit_{\Obja\op}\{\Objael\}\\
    &=\catid_{\Obja\op}.
    \end{aligned}
\end{equation*}
\item \emph{Preservation of composition}: This can be easily seen as follows. Given any~$\mora\in \HomSet{\UPos}{\Obja}{\Objb}$,~$\morb\in \HomSet{\UPos}{\Objb}{\Objc}$:
\begin{equation*}
    \begin{aligned}
    \funeqa(\mora \mthen \morb)&=\mora \mthen \morb\\
    &=\funeqa(\mora)\mthen \funeqa(\morb).
    \end{aligned}
\end{equation*}
Similarly, given any~$\mora\in \HomSet{\LPos}{\Obja}{\Objb}$,~$\morb\in \HomSet{\LPos}{\Objb}{\Objc}$:
\begin{equation*}
    \begin{aligned}
    \funeqb(\mora \mthen \morb)&=\mora \mthen \morb\\
    &=\funeqb(\mora)\mthen \funeqb(\morb).
    \end{aligned}
\end{equation*}
\end{itemize}
\paragraph*{Compositions return identity functors}
We want to show that by composing the two functors we obtain the identity functors in \UPos and \LPos, respectively. Clearly, comosing the two functors returns the identity on the objects, since for any poset~$\posA$, one has~$(\posA\op)\op=\posA$. The functors act on morphisms by ``flipping the context'', and ``flipping'' twice is the ``same'' as not flipping.
\end{proof}

We can show that~$\UPos$ and~$\LPos$ are monoidal categories.


\begin{lemma}
\label{lem:upos_moncat}
\UPos is a monoidal category with the following additional structure:
\begin{enumerate}
    \item \emph{Tensor product $\mtimes$}: On objects, the tensor product corresponds to the product of posets. Given two morphisms~$\mora\colon \Obja\mto \Up{\Objb}$ and~$\morb\colon \Objc\mto \Up{\Objd}$, we have:
    \begin{equation}
    \begin{aligned}
        \mora \mtimes \morb\colon \Obja \times \Objc &\mto \Up(\Objb\times \Objd)\\
        \tup{\Objael,\Objcel}&\mapsto \mora(\Objael)\times \morb(\Objcel).
    \end{aligned}
    \end{equation}
    Note that the Cartesian product of upper sets is an upper set.
    \item \emph{Unit}: The unit is the identity poset.
    \item \emph{Left unitor}: The left unitor is given by the pair of morphisms
    \begin{equation}
        \begin{aligned}
            \leftunitor_\posA\colon \styleobj{\singleton}\mtimes \Obja &\mto \Up{\Obja}\\
            \tup{\styleobj{\singletonel},\Objael}&\mapsto \upit\{\Objael\},
        \end{aligned}
    \end{equation}
    and
    \begin{equation}
        \begin{aligned}
            \leftunitor_\posA^{-1}\colon \Obja &\mto \Up(\styleobj{\singleton}\mtimes \Obja) \\
            \Objael&\mapsto \{\styleobj{\singletonel}\} \times \upit \{\Objael\}.
        \end{aligned}
    \end{equation}
    \item \emph{Right unitor}: The right unitor is given by the pair of morphisms
    \begin{equation}
        \begin{aligned}
            \rightunitor_\posA\colon \Obja\mtimes \styleobj{\singleton}  &\mto \Up{\Obja}\\
            \tup{\Objael,\styleobj{\singletonel}}&\mapsto \upit\{\Objael\},
        \end{aligned}
    \end{equation}
    and
    \begin{equation}
        \begin{aligned}
            \rightunitor_\posA^{-1}\colon \Obja &\mto \Up( \Obja \mtimes \styleobj{\singleton}) \\
            \Objael&\mapsto \upit \{\Objael\} \times \{\styleobj{\singletonel}\}.
        \end{aligned}
    \end{equation}
    \item \emph{Associator}: The associator is given by the pair of morphisms:
    \begin{equation}
        \begin{aligned}
            \associator_{\Obja\Objb,\Objc}\colon (\Obja\mtimes \Objb)\mtimes \Objc &\mto \Up \Obja\times (\Up \Objb \times \Up \Objc)\\
            \tup{\tup{\Objael,\Objbel},\Objcel}&\mapsto \upit\{\Objael\}\times (\upit\{\Objbel\}\times \upit\{\Objcel\}),
        \end{aligned}
    \end{equation}
    and
    \begin{equation}
        \begin{aligned}
            \associator_{\Obja,\Objb\Objc}\colon \Obja\mtimes (\Objb\mtimes \Objc) &\mto (\Up \Obja\times \Up \Objb) \times \Up \Objc\\
            \tup{\Objael,\tup{\Objbel,\Objcel}}&\mapsto (\upit\{\Objael\}\times \upit\{\Objbel\})\times \upit\{\Objcel\}.
        \end{aligned}
    \end{equation}
\end{enumerate}
\end{lemma}
\begin{proof}
    \todojira{238}{type proof}
\end{proof}

\section{From~$\DP$ to~$\UPos$ and~$\LPos$}
\begin{lemma}
\label{lem:covfunctor}
There is a \emph{functor}~$\cofun\colon \DP \fto \UPos$ which maps:
\begin{compactenum}
\item An object (poset) in \DP to the same object (poset) in \UPos.
\item A morphism~$\stylemorph{d}\in \HomSet{\DP}{\funsp}{\ressp}$ to the morphism~$\stylemorph{h_d}\in \HomSet{\UPos}{\funsp}{\ressp}$, where:
\begin{equation}
\begin{aligned}
    \stylemorph{h_d}\colon \funsp\op &\toinPos \tup{\Up{\ressp},\subseteq}\\
    \fun^*&\mapsto \{\res \in \ressp\mid d(\fun^*,\res)=\true\}.
\end{aligned}
\end{equation}
\end{compactenum}
\end{lemma}

\begin{lemma}
\label{lem:confunctor}
There is a \emph{contravariant functor} $\confun\colon \DP \fto \LPos$ which maps:
\begin{compactenum}
\item An object (poset) of \DP to the same object (poset) in \LPos.
\item A morphism $\dprob \in \HomSet{\DP}{\funsp}{\ressp}$ to the morphism~$\morb\in \HomSet{\LPos}{\ressp}{\funsp}$, where:
\begin{equation*}
    \begin{aligned}
    \morb\colon \ressp &\to \Lo{\funsp}\\
    \res&\mapsto \{ \fun \in \funsp \mid \dprob(\fun,\res)=\true\}.
    \end{aligned}
\end{equation*}
\end{compactenum}
\end{lemma}

\todojira{240}{I have proofs for this and for many other things handwritten. Will type them once we decide we want this to be in the book. There is a nice interpretation with an endofunctor Dual, which commutes with the aforementioned ones, meaning that you can solve and then convert to lowersets, or dualize , solve, and then convert to upper sets.}

\section{$\Lmon$ and~$\Umon$ monads}
In this section we propose another example of monads related to posets and upper/lower sets. We start by defining the \emph{$\Uendo$ endofunctor}.
\begin{definition}[$\Uendo$ endofunctor]
  \label{def:Uendo}
  The \emph{$\Uendo$ endofunctor} has the form~$\Uendo \colon \Pos \to \Pos$ and acts on objects and morphisms as follows:
  \begin{compactenum}
    \item \emph{On objects}: Given a poset~$\posA \in \Ob_\Pos$,~$\Uendo$ maps~$\posA$ to its upper set\footnote{Recall that in \cref{lem:u_bounded_lat} we proved that the upper set is itself an object of \Pos.}.
    \item \emph{On morphisms}: Given posets~$\posA,\posB$, and a monotone map~$\mora \colon \posA\to \posB$, the~$\Uendo$ endofunctor acts as:
    \begin{equation}
      \begin{aligned}
        \Uendo(\mora)\colon \Up \posA&\to \Up \posB\\
        \posA'&\mapsto \upit \left( \bigcup_{\posAel\in \posA'} \{\mapa(\posAel)\}\right).
      \end{aligned}
    \end{equation}
  \end{compactenum}
\end{definition}
We now want to prove that the~$\Uendo$ endofunctor is an endofunctor, and the proof requires the following two facts.

\begin{lemma}
  \label{lem:unpack_u_functor}
  Given posets~$\posA,\posB$, a monotone map~$\mora \colon \posA \to \posB$, and a family of singleton sets~$\{S_i\}_{i\in I}$, with~$S_i=\{s_i\}$,~$s_i\in \posA$, the following equality holds:
  \begin{equation}
    \label{eq:lemma_unpack}
    \upit \left( \bigcup_{\posAel \in \upit \bigcup_{i\in I}S_i}\{\mora(\posAel)\}\right)= \upit \left( \bigcup_{i\in I} \{\mora(s_i)\}\right).
  \end{equation}
\end{lemma}

\todotextjira{241}{the phrasing here seems unnecessarily complicated... maybe we phrase the lemma as something like ``$\upit  f( \upit S) = \upit f(S) $" for any set $S$ and any monotone map $f$ ??}

\begin{proof}
  We first want to show that:
  \begin{equation}
    \label{eq:unpack_1}
    \underbrace{\upit \left(\bigcup_{\posAel \in \upit \bigcup_{i\in I}S_i}\{\mora(\posAel)\} \right)}_{\star}\subseteq \upit \underbrace{\left( \bigcup_{i\in I}\{\mora(s_i)\}\right)}_{\diamond}.
  \end{equation}
  Let's take a
  \begin{equation}
    \posBel \in \upit\left( \bigcup_{\posAel \in \upit \bigcup_{i\in I}S_i}\{\mora(\posAel)\}\right).
  \end{equation}
  If we have such a~$\posBel$, it means that there exists a
  \begin{equation}
    \posBel'\in \bigcup_{\posAel\in \upit\bigcup_{i\in I}S_i}\{\mora(\posAel)\}
  \end{equation}
  such that~$\posBel'\posleq_\posB \posBel$, and hence there is a~$\posAel'\in \upit \bigcup_{i\in I} S_i$ such that~$\posBel'=\mora(\posAel')$.
  Consequently, there must exist an~$i'\in I$ such that~$s_{i'}\posleq_\posA \posAel'$. The monotonicity of~$\mapa$ implies:
  \begin{equation}
    \mora(s_{i'})\posleq_\posA \mora(\posAel')=\posBel'\posleq_\posB \posBel.
  \end{equation}
  We know that~$s_{i'}\in \diamond$ and any~$\posBel^*\in \posB$ satisfying~$\mora(s_{i'})\posleq_\posB \posBel^*$ belongs to~$\upit \diamond$.
  Therefore,~$\star\subseteq \upit \diamond$, which proves the validity of \cref{eq:unpack_1}.

  We now want to show that:
  \begin{equation}
    \label{eq:unpack_2}
    \upit \left(\bigcup_{\posAel \in \upit \bigcup_{i\in I}S_i}\{\mora(\posAel)\} \right)\supseteq \upit \left( \bigcup_{i\in I}\{\mora(s_i)\}\right).
  \end{equation}
  By now taking a
  \begin{equation}
    \posBel\in \upit \left( \bigcup_{i\in I}\{\mora(s_i)\}\right),
  \end{equation}
  we know that there is a~$i'\in I$ such that~$\mora(s_{i'})\posleq_\posB \posBel$.
  Furthermore, we know that~$\mora(s_{i'})\in \diamond$.
  Therefore, any~$\posBel^*\posleq_\posB \mora(s_{i'})$ must be in~$\upit \diamond$, meaning that~$\posBel\in \star$, and proving the validity of \cref{eq:unpack_2}.

  The validity of \cref{eq:unpack_1} and \cref{eq:unpack_2} implies \cref{eq:lemma_unpack}.
\end{proof}

\begin{lemma}
  \label{lem:unpack_part_2}
  Given posets~$\posA,\posB$ and a monotone map~$\mora\colon \posA\to \posB$, we have:
  \begin{equation}
    \upit \left( \bigcup_{\posAel'\in \upit \{\posAel\}} \{\mora(\posAel')\}\right)=\upit \{\mora(\posAel)\}.
  \end{equation}
\end{lemma}
\begin{proof}
  The proof follows from \cref{lem:unpack_u_functor}, by considering a family of singleton sets consisting solely of the set~$\{\posAel\}$.
\end{proof}
\todotextjira{243}{I wouldn't make the above its own lemma, that seems a bit silly/overkill.}
We can now show that the~$\Uendo$ endofunctor is indeed a functor.

\begin{lemma}
  The~$\Uendo$ endofunctor is indeed a functor.
\end{lemma}

\begin{proof}
  $\Uendo$ has a valid form and given a poset~$\posA$, maps~$\catid_\posA$ to~$\catid_{\Up \posA}$.
  \todotextjira{244}{.I think the compatibility with identities is non-trivial and should be shown here. I think there is a mistake: the source and target should be UP and UR respectively, not ``Pos''. Not clear to me what ``valid form'' means}
  We now need to show that~$\Uendo$ is compatible with morphism composition. Consider maps~$\mora\colon \posA \to \posB$ and~$\morb\colon \posB \to \posC$. We have:
  \begin{equation}
    \label{eq:ufunctor_1}
    \begin{aligned}
      \Uendo(\mora\mthen \morb)\colon \Pos &\to \Pos\\
      \posA'&\mapsto \upit \left( \bigcup_{\posAel\in \posA'} \{\morb(\mora(\posAel))\}\right).
    \end{aligned}
  \end{equation}

  On the other hand, we have:
  \begin{equation}
    \begin{aligned}
      \Uendo(\mora)\colon \Pos &\to \Pos\\
      \posA'&\mapsto \upit \left( \bigcup_{\posAel\in \posA'}\{\mora(\posAel)\}\right),
    \end{aligned}
  \end{equation}
  and
  \begin{equation}
    \begin{aligned}
      \Uendo(\morb)\colon \Pos &\to \Pos\\
      \posB'&\mapsto \upit \left( \bigcup_{\posBel\in \posB'}\{\morb(\posBel)\}\right),
    \end{aligned}
  \end{equation}
  leading to
  \begin{equation}
    \label{eq:ufunctor_2}
    \begin{aligned}
      \Uendo(\mora)\mthen \Uendo(\morb)\colon \Pos &\to \Pos\\
      \posB'&\mapsto \upit\left( \bigcup_{\posBel\in \upit \bigcup_{\posAel\in \posB'}\{\mora(\posAel)\}}\{\morb(\posBel)\}\right)\\
      &\mapsto \upit \left(\bigcup_{\posAel\in \posA'} \{ \morb(\mora(\posAel))\}\right). \qquad \qquad (\cref{lem:unpack_u_functor})
    \end{aligned}
  \end{equation}
  Since \cref{eq:ufunctor_1} and \cref{eq:ufunctor_2} are equivalent, $\Uendo$ is a functor.
\end{proof}
Having proven that~$\Uendo$ is a valid functor, we are now ready to define the~$\Umon$ monad.
\begin{definition}[$\Umon$ monad]
  \label{def:Umon}
  The \emph{$\Umon$ monad} on \Pos consists of:
  \begin{compactenum}
    \item The~$\Uendo$ endofunctor (\cref{def:Uendo}).
    \item The unit natural transformation~$\monunit_{\Umon} \colon \funid_{\Pos}\nto \Uendo$, which associates to every object~$\posA \in \Ob_\Pos$ a morphisms in \Pos given by:
    \begin{equation}
      \begin{aligned}
        \monunit_{\Umon}^\posA\colon \posA &\to \Up \posA\\
        \posAel&\mapsto \upit \{\posAel\}.
      \end{aligned}
    \end{equation}
    \item The compositional natural transformation~$\moncomp_{\Umon}\colon \Uendo\then \Uendo\nto \Uendo$, which associates to every~$\posA\in \Ob_\Pos$ the morphism in \Pos given by:
    \begin{equation}
      \begin{aligned}
        \moncomp_{\Umon}^\posA\colon \Up{(\Up{\posA})}&\to \Up \posA\\
        \posA''&\mapsto \bigcup_{\posA'\in \posA''}\posA'.
      \end{aligned}
    \end{equation}
  \end{compactenum}
\end{definition}

Before showing that this indeed defines a monad, we list some results which will be instrumental later.
\begin{lemma}
  \label{lem:upperunionupper}
  Given a poset~$\posA$ and a family of upper sets of~$\posA$, denoted~$\{\posA_i\}_{i\in \stylesets{I}}$ (with index set~$\stylesets{I}$), their union is also an upper set.
\end{lemma}

\begin{lemma}
  \label{lem:setunionset}
  Given a set~$\setA$ and a family of subsets of~$\setA$, denoted~$\{\setA_i\}_{i\in \stylesets{I}}$ (with index set~$\stylesets{I}$), one has:
  \begin{equation}
    \bigcup \bigcup_{i\in \stylesets{I}}\{\setA_i\} = \bigcup_{i\in \stylesets{I}}\{\setA_i\}.
  \end{equation}
\end{lemma}


\begin{lemma}
  The~$\Umon$ monad is indeed a monad.
\end{lemma}
\begin{proof}
  To show that~$\Umon$ is indeed a monad, we need to show the following:
  \begin{compactenum}
    \item $\monunit_{\Umon}$ is a natural transformation;
    \item $\moncomp_{\Umon}$ is a natural transformation;
    \item left unitality holds;
    \item right unitality holds;
    \item associativity holds;
  \end{compactenum}
  We prove them in order.

  \emph{1)~$\monunit_{\Umon}$ is a natural transformation}: We need to show that for any~$\mora\in \HomSet{\Pos}{\posA}{\posB}$, we have:
  \begin{equation}
    \funid_{\Pos}(\mora)\mthen \monunit_{\Umon}^\posB=\monunit_{\Umon}^\posA\mthen \Uendo(\mora).
  \end{equation}
  By expanding the left-hand side, we obtain:
  \begin{equation}
    \left[\funid_{\Pos}(\mora)\mthen \monunit_{\Umon}^\posB\right](\posAel)=\upit \{\mora(\posAel)\}.
  \end{equation}
  By expanding the right-hand side, we get:
  \begin{equation}
    \begin{aligned}
      \monunit_{\Umon}^\posA\colon \Pos &\to \Pos\\
      \posAel&\mapsto \upit \{\posAel\}. \qquad \qquad (\cref{lem:unpack_part_2})
    \end{aligned}
  \end{equation}
  and
  \begin{equation}
    \begin{aligned}
      \Uendo(\mora)\colon \Pos &\to \Pos\\
      \posA'&\mapsto \upit \bigcup_{\posAel'\in \posA'}\{\mora(\posAel')\},
    \end{aligned}
  \end{equation}
  and hence
  \begin{equation}
    \begin{aligned}
      \left[\monunit_{\Umon}^\posA\mthen \Uendo(\mora)\right](\posAel)&=\upit \left( \bigcup_{\posAel'\in \upit \{\posAel\}} \{\mora(\posAel')\}\right)\\
      &=\upit \{\mora(\posAel)\}.
    \end{aligned}
  \end{equation}

  \emph{2)~$\moncomp_{\Umon}$ is a natural transformation}:
  We want to show that for every~$\mora\in \HomSet{\Pos}{\posA}{\posB}$, one has:
  \begin{equation}
    \Uendo(\Uendo(\mora))\mthen \moncomp_{\Umon}^{\posB}=\moncomp_{\Umon}^\posA \mthen \Uendo(\mora).
  \end{equation}
  Let's start with the left-hand side. We first look at~$\Uendo(\Uendo(\mora))$:
  \begin{equation}
    \begin{aligned}
      (\Uendo(\Uendo(\mora)))\colon \Pos&\to \Pos\\
      \posA''&\mapsto \upit \left( \bigcup_{\posA' \in \posA''}\left \{ \upit \left( \bigcup_{\posAel \in \posA'} \left\{ \mora(\posAel)\right\}\right)\right\}\right).
    \end{aligned}
  \end{equation}
  Furthermore, one has:
  \begin{equation}
  \begin{aligned}
    \Uendo(\Uendo(\mora))\mthen \moncomp_{\Umon}^{\posB}\colon \Pos&\to \Pos\\
    \posA''&\mapsto \bigcup_{\posB'\in \upit \left( \bigcup_{\posA'\in \posA''} \left\{ \upit \left( \bigcup_{\posAel \in \posA'} \{ \mora(\posAel)\}\right)\right\} \right)} \posB'\\
    &\mapsto \bigcup \upit \left( \bigcup_{\posA'\in \posA''} \left\{ \upit \left( \bigcup_{\posAel \in \posA'} \{ \mora(\posAel)\}\right)\right\} \right)\\
    &\mapsto \bigcup  \left( \bigcup_{\posA'\in \posA''} \left\{ \upit \left( \bigcup_{\posAel \in \posA'} \{ \mora(\posAel)\}\right)\right\} \right) \quad \quad (\text{\cref{lem:upperunionupper}})\\
    &\mapsto   \bigcup_{\posA'\in \posA''} \left\{ \upit \left( \bigcup_{\posAel \in \posA'} \{ \mora(\posAel)\}\right)\right\} \quad \quad (\text{\cref{lem:setunionset}})\\
    &\mapsto   \upit \left(\bigcup_{\posA'\in \posA''}  \bigcup_{\posAel \in \posA'} \{ \mora(\posAel)\}\right),
  \end{aligned}
  \end{equation}
  where we used the fact that the union of upper sets is the upper closure of the union of sets. By looking at the right-hand side, one has:
  \begin{equation}
  \begin{aligned}
    (\moncomp_{\Umon}^\posA \mthen \Uendo(\mora))\colon \Pos &\to \Pos\\
    \posA''&\mapsto \upit \left( \bigcup_{\posAel \in \bigcup_{\posA'\in \posA''}\posA'} \{ \mora(\posAel)\}\right)\\
    &\mapsto \upit \left( \bigcup_{\posA'\in \posA''}\bigcup_{\posAel \in \posA'}\{\mora(\posAel)\}\right),
  \end{aligned}
  \end{equation}
  proving that~$\moncomp_{\Umon}$ is a valid natural transformation.

  \emph{Left unitality holds}: We want to show that given a poset~$\posA\in \Ob_\Pos$, one has:
  \begin{equation}
    \monunit_{\Umon}^{\Uendo(\posA)}\mthen \moncomp_{\Umon}^{\posA}=\catid_{\Pos}^{\Uendo(\posA)}.
  \end{equation}
  One has:
  \begin{equation}
  \begin{aligned}
    \monunit_{\Umon}^{\Uendo(\posA)}\mthen \moncomp_{\Umon}^{\posA}\colon \Up \Uendo &\to \Up \Uendo\\
    \posA'&\mapsto \bigcup_{\posA''\in \upit \{ \posA'\}} \posA''\\
    &\mapsto \posA',
  \end{aligned}
  \end{equation}
  because the upper sets of an upper set~$\posA'$ are subsets of~$\posA'$ (and hence their union is~$\posA'$).

  \emph{Right unitality holds}: We want to show that given a poset~$\posA\in \Ob_\Pos$, one has:
  \begin{equation}
    \Uendo(\monunit_{\Umon}^{\Uendo})\mthen \moncomp_{\Umon}^{\posA}=\catid_{\UPos}^{\Uendo(\posA)}.
  \end{equation}
  One has:
  \begin{equation*}
    \begin{aligned}
    \Uendo(\monunit_{\Umon}^{\Uendo})\mthen \moncomp_{\Umon}^{\posA}\colon \Up \posA &\to \Up \posA\\
      \posA'&\mapsto \bigcup_{\posA''\in \upit \left( \bigcup_{\posAel \in \posA'}\{ \upit \{\posAel\}\}\right)}\posA''\\
      &\mapsto \bigcup \upit \left( \bigcup_{\posAel \in \posA'}\{\upit \{ \posAel\} \}\right)\\
      &\mapsto \upit \bigcup   \bigcup_{\posAel \in \posA'}\{\upit \{ \posAel\} \}\\
      &\mapsto \upit \bigcup_{\posAel \in \posA'}\upit \{ \posAel\}\\
      &\mapsto \upit \posA'\\
      &\mapsto \posA',
    \end{aligned}
  \end{equation*}
  where we use the facts that the union of upper sets is an upper set, the union of an upper sets is the upper closure of the union of sets, and that the upper closure of an upper set returns the upper set.

  \emph{Associativity holds}: We want to show that given a poset~$\posA\in \Ob_\Pos$, one has:
  \begin{equation}
    \Uendo(\moncomp_{\Umon}^{\posA})\mthen \moncomp_{\Umon}^{\posA}= \moncomp_{\Umon}^{\Uendo(\posA)}\mthen \moncomp_{\Umon}^{\posA}.
  \end{equation}
  We start by exploding the left-hand side of the equation. First, one has:
  \begin{equation*}
    \begin{aligned}
      \Uendo(\moncomp_{\Umon}^{\posA})(\posC)&=\upit \left( \bigcup_{\posB\in \posC}\bigcup_{\posA'\in \posB} \posA'\right).
    \end{aligned}
  \end{equation*}
  Therefore, we have:
  \begin{equation*}
    \begin{aligned}
      (\Uendo(\moncomp_{\Umon}^{\posA})\mthen \moncomp_{\Umon}^{\posA})(\posC)&=\bigcup_{\posD \in \upit \left( \bigcup_{\posB\in \posC}\bigcup_{\posA'\in \posB} \posA\right)} \posD\\
      &=\bigcup \upit \left( \bigcup_{\posB\in \posC}\bigcup_{\posA'\in \posB} \posA'\right)\\
      &=\upit \bigcup \bigcup_{\posB\in \posC}\bigcup_{\posA'\in \posB} \posA'\\
      &=\upit  \bigcup_{\posB\in \posC}\bigcup_{\posA'\in \posB} \posA'\\
      &= \bigcup_{\posB\in \posC}\bigcup_{\posA'\in \posB} \posA'
    \end{aligned}
  \end{equation*}
  Starting from right-hand side, instead, one has:
  \begin{equation*}
    \begin{aligned}
      (\moncomp_{\Umon}^{\Uendo(\posA)}\mthen \moncomp_{\Umon}^{\posA})(\posC)&=\bigcup_{\posA'\in \bigcup_{\posB\in \posC}\posB}\posA'\\
      &=\bigcup_{\posB\in \posC}\bigcup_{\posA'\in \posB} \posA'.
    \end{aligned}
  \end{equation*}
\end{proof}

\begin{lemma}
\label{lem:uposkleisli}
  $\UPos$ is the Kleisli category of~$\Umon$.
\end{lemma}