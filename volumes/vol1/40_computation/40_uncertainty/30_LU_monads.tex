% !TEX root = standalone.tex

%
%\section{Using monads to understand uncertainty}
%
%\AC{is this a thing?? }
%Take the $\mathsf{Unc}$ functor $\mathsf{Unc}\colon \DP\to \DP$ which
%\begin{compactenum}
%  \item Maps an object $P$ in \DP (poset) to its twisted arrow category $\twisted{P}$, representing a poset interval.
%  \item Maps a morphism in \DP $d\colon \funsp \profto \ressp$ to $\tup{\low d,\upp d}$, where
%  \begin{equation}
%    \begin{aligned}
%      \low d\colon \F{F_{\low}}&\profto \R{R_{\low}},\\
%      \upp d\colon \F{F_{\upp}}&\profto \R{R_{\upp}},
%    \end{aligned}
%  \end{equation}
%  and $\tup{\low d,\upp d}$ is a boolean profunctor of the form
%  \begin{equation}
%    \begin{aligned}
%      &\tup{\low d,\upp d}\colon \left(\F{F_{\low}}\times \F{F_{\upp}} \right)\op \times \left(\R{R_{\low}}\times \R{R_{\upp}} \right)\toinPos \Bool\\
%      &\tup{\tup{\F{f_{\low}},\F{f_{\upp}}}^*,\tup{\R{r_{\low}},\R{r_{\upp}}}}\mapsto \low d(\F{f_{\low}}^*,\R{r_{\low}})\wedge \upp d(\F{f_{\upp}}^*,\R{r_{\upp}})
%    \end{aligned}
%  \end{equation}
%\end{compactenum}
%
%\todo{write down better}
%Is this a functor?
%
%\begin{proof}
%  Consider two design problems $f\colon \F{A}\profto \R{B}$ and $g\colon \F{B}\profto \R{C}$.
%
%  We know that
%  \begin{equation}
%    \begin{aligned}
%      \unc(f)&\colon \left( \F{A_{\low}}\times \F{A_{\upp}}\right)\op \times \left( \R{B_{\low}}\times \R{B_{\upp}}\right)\toinPos \Bool\\
%      \unc(g)&\colon \left( \F{B_{\low}}\times \F{B_{\upp}}\right)\op \times \left( \R{C_{\low}}\times \R{C_{\upp}}\right)\toinPos \Bool.
%    \end{aligned}
%  \end{equation}
%
%  We have
%  \begin{equation}
%    \begin{aligned}
%      &\left(\unc(f)\then \unc(g)\right) (\tup{\ubar{a},\bar{a}}^*,\tup{\ubar{c},\bar{c}})\\
%      &=\bigvee_{\tup{\ubar{b},\bar{b}}\in B_{\low}\times B_{\upp}} \unc(f)(\tup{\ubar{a},\bar{a}}^*,\tup{\ubar{b},\bar{b}})\wedge \unc(g)(\tup{\ubar{b},\bar{b}}^*,\tup{\ubar{c},\bar{c}})\\
%      &= \bigvee_{\tup{\ubar{b},\bar{b}}\in B_{\low}\times B_{\upp}} \low f(\ubar{a}^*,\ubar{b})\wedge \upp f(\bar{a}^*,\bar{b})\wedge \low g(\ubar{b}^*,\ubar{c})\wedge \upp g(\bar{b}^*,\bar{c})\\
%      &=(\low f\then \low g)(\ubar{a}^*,\ubar{c})\wedge (\upp f\then \upp g)(\bar{a}^*,\bar{c})\\
%      &=\low f\then g (\ubar{a}^*,\ubar{c})\wedge \upp f\then g(\bar{a}^*,\bar{c})\\
%      &=\unc(f\then g)(\tup{\ubar{a},\bar{a}}^*,\tup{\ubar{c},\bar{c}}).
%    \end{aligned}
%  \end{equation}
%\end{proof}
%

\section{$\Lmon$ and $\Umon$ monads}
In this section we propose another example of monads related to posets and design problems. We start by defining the \emph{$\Uendo$ endofunctor}.
\begin{definition}[$\Uendo$ endofunctor]
  \label{def:Uendo}
  The \emph{$\Uendo$ endofunctor} has the form $\Uendo \colon \Pos \to \Pos$ and acts on objects and morphisms as follows:
  \begin{compactenum}
    \item \emph{On objects}: Given a poset $P\in \Ob_\Pos$, $\Uendo$ maps $P$ to its upper set\footnote{Recall that in \cref{lem:u_bounded_lat} we proved that the upper set is itself an object of \Pos.}.
    \item \emph{On morphisms}: Given posets $P,Q$, and a monotone map $f\colon P\to Q$, the $\Uendo$ endofunctor acts as:
    \begin{equation}
      \begin{aligned}
        \Uendo(f)\colon \Up P&\to \Up Q\\
        P'&\mapsto \upit \left( \bigcup_{p\in P'} \{f(p)\}\right).
      \end{aligned}
    \end{equation}
  \end{compactenum}
\end{definition}
We now want to prove that the $\Uendo$ endofunctor is an endofunctor, and the proof requires the following two facts.
\begin{lemma}
  \label{lem:unpack_u_functor}
  Given posets $P,Q$, a monotone map $f\colon P \to Q$, and a family of singleton sets $\{S_i\}_{i\in I}$, with $S_i=\{s_i\}$, $s_i\in P$, the following equality holds:
  \begin{equation}
    \label{eq:lemma_unpack}
    \upit\left( \bigcup_{p\in \upit \bigcup_{i\in I}S_i}\{f(p)\}\right)= \upit \left( \bigcup_{i\in I} \{f(s_i)\}\right).
  \end{equation}
\end{lemma}
\begin{proof}
  We first want to show that:
  \begin{equation}
    \label{eq:unpack_1}
    \underbrace{\upit \left(\bigcup_{p\in \upit \bigcup_{i\in I}S_i}\{f(p)\} \right)}_{\star}\subseteq \upit \underbrace{\left( \bigcup_{i\in I}\{f(s_i)\}\right)}_{\diamond}.
  \end{equation}
  Let's take a
  \begin{equation}
    q\in \upit\left( \bigcup_{p\in \upit \bigcup_{i\in I}S_i}\{f(p)\}\right).
  \end{equation}
  If we have such a $q$, it means that there exists a
  \begin{equation}
    q'\in \bigcup_{p\in \upit\bigcup_{i\in I}S_i}\{f(p)\}
  \end{equation}
  such that~$q'\ordleq_Q q$, and hence there is a~$p'\in \upit \bigcup_{i\in I} S_i$ such that $q'=f(p')$. Consequently, there must exist an $i'\in I$ such that $s_{i'}\ordleq_P p'$. The monotonicity of $f$ implies:
  \begin{equation}
    f(s_{i'})\ordleq_Q f(p')=q'\ordleq_Q q.
  \end{equation}
  We know that $s_{i'}\in \diamond$ and any $q^*\in Q$ satisfying $f(s_{i'})\ordleq_Q q^*$ belongs to $\upit \diamond$. Therefore,~$\star\subseteq \upit \diamond$, which proves the validity of \cref{eq:unpack_1}.

  We now want to show that:
  \begin{equation}
    \label{eq:unpack_2}
    \upit \left(\bigcup_{p\in \upit \bigcup_{i\in I}S_i}\{f(p)\} \right)\supseteq \upit \left( \bigcup_{i\in I}\{f(s_i)\}\right).
  \end{equation}
  By now taking a
  \begin{equation}
    q\in \upit \left( \bigcup_{i\in I}\{f(s_i)\}\right),
  \end{equation}
  we know that there is a $i'\in I$ such that $f(s_{i'})\ordleq_Q q$. Furthermore, we know that $f(s_{i'})\in \diamond$. Therefore, any $q^*\ordgeq_Q f(s_{i'})$ must be in $\upit \diamond$, meaning that $q\in \star$, and proving the validity of \cref{eq:unpack_2}.

  The validity of \cref{eq:unpack_1} and \cref{eq:unpack_2} implies \cref{eq:lemma_unpack}.
\end{proof}

\begin{lemma}
  \label{lem:unpack_part_2}
  Given posets $P,Q$ and a monotone map $f\colon P\to Q$, we have:
  \begin{equation}
    \upit \left( \bigcup_{p'\in \upit \{p\}} \{f(p')\}\right)=\upit \{f(p)\}.
  \end{equation}
\end{lemma}
\begin{proof}
  The proof follows from \cref{lem:unpack_u_functor}, by considering a family of singleton sets consisting solely of the set $\{p\}$.
\end{proof}
We can now show that the $\Uendo$ endofunctor is indeed a functor.
\begin{lemma}
  The $\Uendo$ endofunctor is indeed a functor.
\end{lemma}
\begin{proof}
  $\Uendo$ has a valid form and given a poset $P$, maps $\catid_P$ to $\catid_{\Up P}$. We now need to show that $\Uendo$ fulfills morphism composition. Consider maps $f\colon P \to Q$ and $g\colon Q \to R$. We have:
  \begin{equation}
    \label{eq:ufunctor_1}
    \begin{aligned}
      U(f\then g)\colon \Pos &\to \Pos\\
      P'&\mapsto \upit \left( \bigcup_{p\in P'} \{g(f(p))\}\right).
    \end{aligned}
  \end{equation}
  On the other hand, we have:
  \begin{equation}
    \begin{aligned}
      U(f)\colon \Pos &\to \Pos\\
      P'&\mapsto \upit \left( \bigcup_{p\in P'}\{f(p)\}\right),
    \end{aligned}
  \end{equation}
  and
  \begin{equation}
    \begin{aligned}
      U(g)\colon \Pos &\to \Pos\\
      Q'&\mapsto \upit \left( \bigcup_{q\in Q'}\{g(q)\}\right),
    \end{aligned}
  \end{equation}
  leading to
  \begin{equation}
    \label{eq:ufunctor_2}
    \begin{aligned}
      U(f)\then U(g)\colon \Pos &\to \Pos\\
      P'&\mapsto \upit\left( \bigcup_{q\in \upit \bigcup_{p\in P'}\{f(p)\}}\{g(q)\}\right)\\
      &\mapsto \upit \left(\bigcup_{p\in P'} \{ g(f(p))\}\right). \qquad \qquad (\cref{lem:unpack_u_functor})
    \end{aligned}
  \end{equation}
  Since \cref{eq:ufunctor_1} and \cref{eq:ufunctor_2} are equivalent, $\Uendo$ is a functor.
\end{proof}
Having proven that $\Uendo$ is a valid functor, we are now ready to define the $\Umon$ monad.
\begin{definition}[$\Umon$ monad]
  \label{def:Umon}
  The \emph{$\Umon$ monad} on \Pos consists of:
  \begin{compactenum}
    \item The $\Uendo$ endofunctor (\cref{def:ufunctor}).
    \item The unit natural transformation $\monunit_{\Umon} \colon \funid_{\Pos}\nto \Uendo$, which associates to every object $P\in \Ob_\Pos$ a morphisms in \Pos given by:
    \begin{equation}
      \begin{aligned}
        \monunit_{\Umon}^P\colon P &\to \Up P\\
        p&\mapsto \upit \{p\}.
      \end{aligned}
    \end{equation}
    \item The compositiona natural transformation $\moncomp_{\Umon}\colon \Uendo\then \Uendo\nto \Uendo$, which associates to every $P\in \Ob_\Pos$ the morphism in \Pos given by:
    \begin{equation}
      \begin{aligned}
        \moncomp_{\Umon}^P\colon \Up{(\Up{P})}&\to \Up P\\
        P''&\mapsto \bigcup_{P'\in P''}P'.
      \end{aligned}
    \end{equation}
  \end{compactenum}
\end{definition}


\begin{lemma}
  The $\Umon$ monad is indeed a monad.
\end{lemma}
\begin{proof}
  To show that $\Umon$ is indeed a monad, we need to show the following:
  \begin{compactenum}
    \item $\monunit_{\Umon}$ is a natural transformation;
    \item $\moncomp_{\Umon}$ is a natural transformation;
    \item left unitality holds;
    \item right unitality holds;
    \item associativity holds;
  \end{compactenum}
  We prove them in order.

  \emph{1)~$\monunit_{\Umon}$ is a natural transformation}: We need to show that for any~$\mora\in \HomSet{\Pos}{P}{Q}$, we have:
  \begin{equation}
    \funid_{\Pos}(\mora)\mthen \monunit_{\Umon}^Q=\monunit_{\Umon}^P\mthen \Uendo(\mora).
  \end{equation}
  By expanding the left-hand side, we obtain:
  \begin{equation}
    \left[\funid_{\Pos}(\mora)\mthen \monunit_{\Umon}^Q\right](p)=\upit \{\mora(p)\}.
  \end{equation}
  By expanding the right-hand side, we get:
  \begin{equation}
    \begin{aligned}
      \monunit_{\Umon}^P\colon \Pos &\to \Pos\\
      p&\mapsto \upit \{p\}. \qquad \qquad (\cref{lem:unpack_part_2})
    \end{aligned}
  \end{equation}
  and
  \begin{equation}
    \begin{aligned}
      \Uendo(\mora)\colon \Pos &\to \Pos\\
      P'&\mapsto \upit \bigcup_{p'\in P'}\{\mora(p')\},
    \end{aligned}
  \end{equation}
  and hence
  \begin{equation}
    \begin{aligned}
      \left[\monunit_{\Umon}^P\mthen \Uendo(\mora)\right](p)&=\upit \left( \bigcup_{p'\in \upit \{p\}} \{\mora(p')\}\right)\\
      &=\upit \{\mora(p)\}.
    \end{aligned}
  \end{equation}

  \emph{2)~$\moncomp_{\Umon}$ is a natural transformation}:
  \todo{finish/continue}
\end{proof}
