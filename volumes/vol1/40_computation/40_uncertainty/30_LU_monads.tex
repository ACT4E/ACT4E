% !TEX root = chapter-standalone.tex

%
%\section{Using monads to understand uncertainty}
%
%\AC{is this a thing?? }
%Take the $\mathsf{Unc}$ functor $\mathsf{Unc}\colon \DP\to \DP$ which
%\begin{enumerate}
%  \item Maps an object $P$ in \DP (poset) to its twisted arrow category $\twisted{P}$, representing a poset interval.
%  \item Maps a morphism in \DP $d\colon \funsp \profto \ressp$ to $\tup{\low d,\upp d}$, where
%  \begin{equation}
%    \begin{aligned}
%      \low d\colon \F{F_{\low}}&\profto \R{R_{\low}},\\
%      \upp d\colon \F{F_{\upp}}&\profto \R{R_{\upp}},
%    \end{aligned}
%  \end{equation}
%  and $\tup{\low d,\upp d}$ is a boolean profunctor of the form
%  \begin{equation}
%    \begin{aligned}
%      &\tup{\low d,\upp d}\colon \left(\F{F_{\low}}\times \F{F_{\upp}} \right)\op \times \left(\R{R_{\low}}\times \R{R_{\upp}} \right)\toinPos \Bool\\
%      &\tup{\tup{\F{f_{\low}},\F{f_{\upp}}}^*,\tup{\R{r_{\low}},\R{r_{\upp}}}}\mapsto \low d(\F{f_{\low}}^*,\R{r_{\low}})\wedge \upp d(\F{f_{\upp}}^*,\R{r_{\upp}})
%    \end{aligned}
%  \end{equation}
%\end{enumerate}
%
%Is this a functor?
%
%\begin{proof}
%  Consider two design problems $f\colon \F{A}\profto \R{B}$ and $g\colon \F{B}\profto \R{C}$.
%
%  We know that
%  \begin{equation}
%    \begin{aligned}
%      \unc(f)&\colon \left( \F{A_{\low}}\times \F{A_{\upp}}\right)\op \times \left( \R{B_{\low}}\times \R{B_{\upp}}\right)\toinPos \Bool\\
%      \unc(g)&\colon \left( \F{B_{\low}}\times \F{B_{\upp}}\right)\op \times \left( \R{C_{\low}}\times \R{C_{\upp}}\right)\toinPos \Bool.
%    \end{aligned}
%  \end{equation}
%
%  We have
%  \begin{equation}
%    \begin{aligned}
%      &\left(\unc(f)\then \unc(g)\right) (\tup{\ubar{a},\bar{a}}^*,\tup{\ubar{c},\bar{c}})\\
%      &=\bigvee_{\tup{\ubar{b},\bar{b}}\setin B_{\low}\times B_{\upp}} \unc(f)(\tup{\ubar{a},\bar{a}}^*,\tup{\ubar{b},\bar{b}})\wedge \unc(g)(\tup{\ubar{b},\bar{b}}^*,\tup{\ubar{c},\bar{c}})\\
%      &= \bigvee_{\tup{\ubar{b},\bar{b}}\setin B_{\low}\times B_{\upp}} \low f(\ubar{a}^*,\ubar{b})\wedge \upp f(\bar{a}^*,\bar{b})\wedge \low g(\ubar{b}^*,\ubar{c})\wedge \upp g(\bar{b}^*,\bar{c})\\
%      &=(\low f\then \low g)(\ubar{a}^*,\ubar{c})\wedge (\upp f\then \upp g)(\bar{a}^*,\bar{c})\\
%      &=\low f\then g (\ubar{a}^*,\ubar{c})\wedge \upp f\then g(\bar{a}^*,\bar{c})\\
%      &=\unc(f\then g)(\tup{\ubar{a},\bar{a}}^*,\tup{\ubar{c},\bar{c}}).
%    \end{aligned}
%  \end{equation}
%\end{proof}
%

\section{$\Lmon$ and~$\Umon$ monads}
In this section, we propose another example of monads related to posets and upper/lower sets.
We start by defining the \emph{$\Uendo$ endofunctor}.
\begin{definition}[$\Uendo$ endofunctor]
    \label{def:Uendo}
    The \emph{$\Uendo$ endofunctor} has the form~$\Uendo \colon \Pos \to \Pos$ and acts on objects and morphisms as follows:
    \begin{enumerate}
        \item \emph{On objects}: Given a poset~$\posA \setin \Ob_\Pos$,~$\Uendo$ maps~$\posA$ to its upper set\footnote{Recall that in \cref{lem:u_bounded_lat} we proved that the upper set is itself an object of \Pos.
              }.
        \item \emph{On morphisms}: Given posets~$\posA,\posB$, and a monotone map~$\mora \colon \posA\to \posB$, the~$\Uendo$ endofunctor acts as:
              \begin{equation}
                  \begin{aligned}
                      \Uendo(\mora)\colon \Up \posA & \to \Up \posB \\
                      \posA'                        & \mapsto \upit \left( \bigsetunion_{\posAel\setin \posA'} \makeset{\mapa(\posAel)}\right).
                  \end{aligned}
              \end{equation}
    \end{enumerate}
\end{definition}

We now want to prove that the~$\Uendo$ endofunctor is an endofunctor.

\begin{lemma}
    \label{lem:Uendo-is-functor}
    The~$\Uendo$ endofunctor is indeed a functor.
\end{lemma}

\begin{proof}
    $\Uendo$ has a valid form and given a poset~$\posA$, maps~$\catid_\posA$ to~$\catid_{\Up \posA}$.
    \todotextjira{244}{.
        I think the compatibility with identities is non-trivial and should be shown here.
        I think there is a mistake: the source and target should be UP and UR respectively, not ``Pos''.
        Not clear to me what ``valid form'' means}
    We now need to show that~$\Uendo$ is compatible with morphism composition.
    Consider maps~$\mora\colon \posA \to \posB$ and~$\morb\colon \posB \to \posC$.
    We have:
    \begin{equation}
        \label{eq:ufunctor_1}
        \begin{aligned}
            \Uendo(\mora\mthen \morb)\colon \Pos & \to \Pos \\
            \posA'                               & \mapsto \upit \left( \bigsetunion_{\posAel\setin \posA'} \makeset{\morb(\mora(\posAel))}\right).
        \end{aligned}
    \end{equation}
    %
    On the other hand, we have:
    %
    \begin{equation}
        \begin{aligned}
            \Uendo(\mora)\colon \Pos & \to \Pos \\
            \posA'                   & \mapsto \upit \left( \bigsetunion_{\posAel\setin \posA'}\makeset{\mora(\posAel)}\right),
        \end{aligned}
    \end{equation}
    %
    and
    %
    \begin{equation}
        \begin{aligned}
            \Uendo(\morb)\colon \Pos & \to \Pos \\
            \posB'                   & \mapsto \upit \left( \bigsetunion_{\posBel\setin \posB'}\makeset{\morb(\posBel)}\right),
        \end{aligned}
    \end{equation}
    leading to
    \begin{equation}
        \label{eq:ufunctor_2}
        \begin{aligned}
            \Uendo(\mora)\mthen \Uendo(\morb)\colon \Pos & \to \Pos \\
            \posB'                                       & \mapsto \upit\left( \bigsetunion_{\posBel\setin \upit \bigsetunion_{\posAel\setin \posB'}\makeset{\mora(\posAel)}}\makeset{\morb(\posBel)}\right) \\
                                                         & \mapsto \upit \left(\bigsetunion_{\posAel\setin \posA'} \makeset{ \morb(\mora(\posAel))}\right).
            \qquad \qquad (\cref{lem:unpack_u_functor})
        \end{aligned}
    \end{equation}
    Since \cref{eq:ufunctor_1} and \cref{eq:ufunctor_2} are equivalent, $\Uendo$ is a functor.
\end{proof}
Having proven that~$\Uendo$ is a valid functor, we are now ready to define the~$\Umon$ monad.
\begin{definition}[$\Umon$ monad]
    \label{def:Umon}
    The \emph{$\Umon$ monad} on \Pos consists of:
    \begin{enumerate}
        \item The~$\Uendo$ endofunctor (\cref{def:Uendo}).
        \item The unit natural transformation~$\monunit_{\Umon} \colon \funid_{\Pos}\nto \Uendo$, which associates to every object~$\posA \setin \Ob_\Pos$ a morphisms in \Pos given by:
              \begin{equation}
                  \begin{aligned}
                      \monunit_{\Umon}^\posA\colon \posA & \to \Up \posA \\
                      \posAel                            & \mapsto \upit \makeset{\posAel}.
                  \end{aligned}
              \end{equation}
        \item The compositional natural transformation~$\moncomp_{\Umon}\colon \Uendo\then \Uendo\nto \Uendo$, which associates to every~$\posA\setin \Ob_\Pos$ the morphism in \Pos given by:
              \begin{equation}
                  \begin{aligned}
                      \moncomp_{\Umon}^\posA\colon \Up{(\Up{\posA})} & \to \Up \posA \\
                      \posA''                                        & \mapsto \bigsetunion_{\posA'\setin \posA''}\posA'.
                  \end{aligned}
              \end{equation}
    \end{enumerate}
\end{definition}

Before showing that this indeed defines a monad, we list some results which will be instrumental later.
\begin{lemma}
    \label{lem:upperunionupper}
    Given a poset~$\posA$ and a family of upper sets of~$\posA$, denoted~$\makeset{\posA_i}_{i\setin \stylesets{I}}$ (with index set~$\stylesets{I}$), their union is also an upper set.
\end{lemma}

\begin{lemma}
    \label{lem:setunionset}
    Given a set~$\setA$ and a family of subsets of~$\setA$, denoted~$\makeset{\setA_i}_{i\setin \stylesets{I}}$ (with index set~$\stylesets{I}$), one has:
    \begin{equation}
        \bigsetunion \bigsetunion_{i\setin \stylesets{I}}\makeset{\setA_i} = \bigsetunion_{i\setin \stylesets{I}}\makeset{\setA_i}.
    \end{equation}
\end{lemma}

\begin{lemma}
    The~$\Umon$ monad is indeed a monad.
\end{lemma}
\begin{proof}
    To show that~$\Umon$ is indeed a monad, we need to show the following:
    \begin{enumerate}
        \item $\monunit_{\Umon}$ is a natural transformation;
        \item $\moncomp_{\Umon}$ is a natural transformation;
        \item left unitality holds;
        \item right unitality holds;
        \item associativity holds;
    \end{enumerate}
    We prove them in order.

    \emph{1)~$\monunit_{\Umon}$ is a natural transformation}:
    We need to show that for any morphism $\mora\setin \HomSet{\Pos}{\posA}{\posB}$, we have:
    %
    \begin{equation}
        \funid_{\Pos}(\mora)\mthen \monunit_{\Umon}^\posB=\monunit_{\Umon}^\posA\mthen \Uendo(\mora).
    \end{equation}
    %
    By expanding the left-hand side, we obtain:
    %
    \begin{equation}
        \left[\funid_{\Pos}(\mora)\mthen \monunit_{\Umon}^\posB\right](\posAel)=\upit \makeset{\mora(\posAel)}.
    \end{equation}
    %
    By expanding the right-hand side, we get:
    \todotext{@Gioele: dead reference \str{unpack_part_2}}
    %
    \begin{equation}
        \begin{aligned}
            \monunit_{\Umon}^\posA\colon \Pos & \to \Pos \\
            \posAel                           & \mapsto \upit \makeset{\posAel}.
            \qquad \qquad (\XXX)
        \end{aligned}
    \end{equation}
    %
    and
    %
    \begin{equation}
        \begin{aligned}
            \Uendo(\mora)\colon \Pos & \to \Pos \\
            \posA'                   & \mapsto \upit \bigsetunion_{\posAel'\setin \posA'}\makeset{\mora(\posAel')},
        \end{aligned}
    \end{equation}
    and hence
    \begin{equation}
        \begin{aligned}
            \left[\monunit_{\Umon}^\posA\mthen \Uendo(\mora)\right](\posAel) & =\upit \left( \bigsetunion_{\posAel'\setin \upit \makeset{\posAel}} \makeset{\mora(\posAel')}\right) \\
                                                                             & =\upit \makeset{\mora(\posAel)}.
        \end{aligned}
    \end{equation}
    %
    \emph{2)~$\moncomp_{\Umon}$ is a natural transformation}:
    We want to show that for every~$\mora\setin \HomSet{\Pos}{\posA}{\posB}$, one has:
    \begin{equation}
        \Uendo(\Uendo(\mora))\mthen \moncomp_{\Umon}^{\posB}=\moncomp_{\Umon}^\posA \mthen \Uendo(\mora).
    \end{equation}
    Let's start with the left-hand side.
    We first look at~$\Uendo(\Uendo(\mora))$:
    \begin{equation}
        \begin{aligned}
            (\Uendo(\Uendo(\mora)))
            \colon \Pos & \to \Pos \\
            \posA''     & \mapsto \upit \left( \bigsetunion_{\posA' \setin \posA''}
            \makesett{
                    \upit \left( \bigsetunion_{\posAel \setin \posA'} \makesett{ \mora(\posAel)}\right)}
            \right).
        \end{aligned}
    \end{equation}
    Furthermore, one has:
    \begin{equation}
        \begin{aligned}
            \Uendo(\Uendo(\mora))\mthen \moncomp_{\Umon}^{\posB}\colon \Pos & \to \Pos \\
            \posA''                                                         & \mapsto \bigsetunion_{\posB'\setin \upit \left( \bigsetunion_{\posA'\setin \posA''} \makesett{ \upit \left( \bigsetunion_{\posAel \setin \posA'} \makeset{ \mora(\posAel)}\right)} \right)} \posB' \\
                                                                            & \mapsto \bigsetunion \upit \left( \bigsetunion_{\posA'\setin \posA''} \makesett{ \upit \left( \bigsetunion_{\posAel \setin \posA'} \makeset{ \mora(\posAel)}\right)} \right) \\
                                                                            & \mapsto \bigsetunion  \left( \bigsetunion_{\posA'\setin \posA''} \makesett{ \upit \left( \bigsetunion_{\posAel \setin \posA'} \makeset{ \mora(\posAel)}\right)} \right) \quad \quad (\text{\cref{lem:upperunionupper}}) \\
                                                                            & \mapsto   \bigsetunion_{\posA'\setin \posA''} \makesett{ \upit \left( \bigsetunion_{\posAel \setin \posA'} \makeset{ \mora(\posAel)}\right)} \quad \quad (\text{\cref{lem:setunionset}}) \\
                                                                            & \mapsto   \upit \left(\bigsetunion_{\posA'\setin \posA''}  \bigsetunion_{\posAel \setin \posA'} \makeset{ \mora(\posAel)}\right),
        \end{aligned}
    \end{equation}
    where we used the fact that the union of upper sets is the upper closure of the union of sets.
    By looking at the right-hand side, one has:
    \begin{equation}
        \begin{aligned}
            (\moncomp_{\Umon}^\posA \mthen \Uendo(\mora))
            \colon \Pos & \to \Pos \\
            \posA''     & \mapsto \upit \left( \bigsetunion_{\posAel \setin \bigsetunion_{\posA'\setin \posA''}\posA'} \makeset{ \mora(\posAel)}\right) \\
                        & \mapsto \upit \left( \bigsetunion_{\posA'\setin \posA''}\bigsetunion_{\posAel \setin \posA'}\makeset{\mora(\posAel)}\right),
        \end{aligned}
    \end{equation}
    proving that~$\moncomp_{\Umon}$ is a valid natural transformation.

    \emph{Left unitality holds}: We want to show that given a poset~$\posA\setin \Ob_\Pos$, one has:
    \begin{equation}
        \monunit_{\Umon}^{\Uendo(\posA)}\mthen \moncomp_{\Umon}^{\posA}=\catid_{\Pos}^{\Uendo(\posA)}.
    \end{equation}
    One has:
    \begin{equation}
        \begin{aligned}
            \monunit_{\Umon}^{\Uendo(\posA)}\mthen \moncomp_{\Umon}^{\posA}\colon \Up \Uendo & \to \Up \Uendo \\
            \posA'                                                                           & \mapsto \bigsetunion_{\posA''\setin \upit \makeset{ \posA'}} \posA'' \\
                                                                                             & \mapsto \posA',
        \end{aligned}
    \end{equation}
    because the upper sets of an upper set~$\posA'$ are subsets of~$\posA'$ (and hence their union is~$\posA'$).

    \emph{Right unitality holds}: We want to show that given a poset~$\posA\setin \Ob_\Pos$, one has:
    \begin{equation}
        \Uendo(\monunit_{\Umon}^{\Uendo})\mthen \moncomp_{\Umon}^{\posA}=\catid_{\UPos}^{\Uendo(\posA)}.
    \end{equation}
    One has:
    \begin{equation*}
        \begin{aligned}
            \Uendo(\monunit_{\Umon}^{\Uendo})\mthen \moncomp_{\Umon}^{\posA}\colon \Up \posA & \to \Up \posA \\
            \posA'                                                                           & \mapsto \bigsetunion_{\posA''\setin \upit \left( \bigsetunion_{\posAel \setin \posA'}\makeset{ \upit \makeset{\posAel}}\right)}\posA'' \\
                                                                                             & \mapsto \bigsetunion \upit \left( \bigsetunion_{\posAel \setin \posA'}\makeset{\upit \makeset{ \posAel} }\right) \\
                                                                                             & \mapsto \upit \bigsetunion   \bigsetunion_{\posAel \setin \posA'}\makeset{\upit \makeset{ \posAel} } \\
                                                                                             & \mapsto \upit \bigsetunion_{\posAel \setin \posA'}\upit \makeset{ \posAel} \\
                                                                                             & \mapsto \upit \posA' \\
                                                                                             & \mapsto \posA',
        \end{aligned}
    \end{equation*}
    where we use the facts that the union of upper sets is an upper set, the union of an upper sets is the upper closure of the union of sets, and that the upper closure of an upper set returns the upper set.

    \emph{Associativity holds}: We want to show that given a poset~$\posA\setin \Ob_\Pos$, one has:
    \begin{equation}
        \Uendo(\moncomp_{\Umon}^{\posA})\mthen \moncomp_{\Umon}^{\posA}= \moncomp_{\Umon}^{\Uendo(\posA)}\mthen \moncomp_{\Umon}^{\posA}.
    \end{equation}
    We start by exploding the left-hand side of the equation.
    First, one has:
    \begin{equation*}
        \begin{aligned}
            \Uendo(\moncomp_{\Umon}^{\posA})(\posC) & =\upit \left( \bigsetunion_{\posB\setin \posC}\bigsetunion_{\posA'\setin \posB} \posA'\right).
        \end{aligned}
    \end{equation*}
    Therefore, we have:
    \begin{equation*}
        \begin{aligned}
            (\Uendo(\moncomp_{\Umon}^{\posA})\mthen \moncomp_{\Umon}^{\posA})(\posC)
             & =\bigsetunion_{\posD \setin \upit \left( \bigsetunion_{\posB\setin \posC}\bigsetunion_{\posA'\setin \posB} \posA\right)} \posD \\
             & =\bigsetunion \upit \left( \bigsetunion_{\posB\setin \posC}\bigsetunion_{\posA'\setin \posB} \posA'\right) \\
             & =\upit \bigsetunion \bigsetunion_{\posB\setin \posC}\bigsetunion_{\posA'\setin \posB} \posA' \\
             & =\upit  \bigsetunion_{\posB\setin \posC}\bigsetunion_{\posA'\setin \posB} \posA' \\
             & = \bigsetunion_{\posB\setin \posC}\bigsetunion_{\posA'\setin \posB} \posA'
        \end{aligned}
    \end{equation*}
    Starting from right-hand side, instead, one has:
    \begin{equation*}
        \begin{aligned}
            (\moncomp_{\Umon}^{\Uendo(\posA)}\mthen \moncomp_{\Umon}^{\posA})(\posC)
             & =\bigsetunion_{\posA'\setin \bigsetunion_{\posB\setin \posC}\posB}\posA' \\
             & =\bigsetunion_{\posB\setin \posC}\bigsetunion_{\posA'\setin \posB} \posA'.
        \end{aligned}
    \end{equation*}
\end{proof}

\begin{lemma}
    \label{lem:uposkleisli}
    $\UPos$ is the Kleisli category of~$\Umon$.
\end{lemma}
