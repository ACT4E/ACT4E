% !TEX root = chapter-standalone.tex


\section{Monotonicity and fixed points}
\label{sec:Monotonicity-and-fixed}

We will use Kleene's theorem, a celebrated result that is used in disparate fields.
It is used in computer science for defining denotational semantics~(see \eg~\cite{manes86}).
It is used in embedded systems for defining the semantics of models of computation~(see, \eg~\cite{lee10}).

\begin{definition}[Directed set]
    \label{def:directed-set}
    A set~$S\subseteq\posA$ is \emph{directed} if each pair of elements in~$S$ has an upper bound: for all~$a,b\in S$, there exists~$c\in S$ such that~$a\posleq c$ and~$b\posleq c$.
\end{definition}

\begin{definition}[Completeness]
    \label{def:cpo}A poset is a \emph{directed complete partial order} (\DCPO) if each of its directed subsets has a supremum (least of upper bounds).
    It is a \emph{complete partial order} (\CPO) if it also has a bottom.
\end{definition}

\begin{example}
    [Completion of $\nonNegReals$ to~$\nonNegRealsComp$]
    \label{exa:Rcomp}The set of real numbers~\reals is not a \CPO, because it lacks a bottom.
    The nonnegative reals~$\nonNegReals=\{x\in\reals \colon x\geq0\}$ have a bottom~$\posbot=0$, however, they are not a \DCPO because some of their directed subsets do not have an upper bound.
    For example, take~$\nonNegReals$, which is a subset of~$\nonNegReals$.
    Then~$\nonNegReals$ is directed, because for each~$a,b\in\nonNegReals$, there exists~$c=\max\{a,b\}\in\nonNegReals$ for which~$a\Rleq c$ and~$b\Rleq c$.
    One way to make~$\tup{ \nonNegReals,\Rleq} $ a \CPO is by adding an artificial top element~$\postop$, by defining~$\nonNegRealsComp\triangleq\nonNegReals\cup\{\postop\}$, and extending the partial order~$\Rleq$ so that~$a\Rleq\postop$ for all~$a\in\nonNegReals$.
\end{example}

A property of maps that will be important are monotonicity and the stronger property of \scottcontinuity.

\begin{definition}[\scottcontinuity]
    \label{def:scott}
    A map~$f:\posA\rightarrow\posB$ between DCPOs is \emph{\scottcontinuous{}} iff for each directed subset~$D\subseteq\posA$, the image~$f(D)$ is directed, and $f(\Sup D)=\Sup f(D).$
\end{definition}
\begin{remark}
    \scottcontinuity implies monotonicity.
\end{remark}

\begin{remark}
    \scottcontinuity does not imply topological continuity.
    A map from the CPO $\tup{\Rcomp,\Rleq}$ to itself is \scottcontinuous iff it is nondecreasing and left-continuous.
    For example, the ceiling function $x\mapsto\left\lceil x\right\rceil $~ is \scottcontinuous (\cref{fig:ceil}).
\end{remark}

\captionsideleft{\label{fig:ceil}}{\includegraphics[scale=0.33]{gmcdptro_ceil}}

A \emph{fixed point} of $f:\posA\to\posA$ is a point~$x$ such that $f(x)=x$.
\begin{definition}
    \label{def:least-fixed}
    A \emph{least fixed point} of~$f:\posA\to\posA$ is the minimum (if it exists) of the set of fixed points of~$f$:
    \begin{equation}
        \label{eq:lfp-one}
        \lfp(f)\,\,\definedas\,\,\min_{\posleq}\,\{x\in\posA\colon f(x)=x\}.
    \end{equation}
    \todotext{So if multiple minima exist, we say there is no least fixed point?
    If yes, perhaps say this explicitly, for clarity?}

    The equality in \cref{eq:lfp-one} can be relaxed to ``$\posleq$''.
\end{definition}

The least fixed point need not exist.
Monotonicity of the map~$f$ plus completeness is sufficient to ensure existence.

\begin{lemma}[{\cite[CPO Fixpoint Theorem II, 8.22]{davey02}}]
    \label{lem:CPO-fix-point-2}If~$\posA$ is a \CPO and~$f:\posA\rightarrow\posA$
    is monotone, then $\lfp(f)$ exists.
\end{lemma}


With the additional assumption of \scottcontinuity, Kleene's algorithm is a systematic procedure to find the least fixed point.

\begin{lemma}[{Kleene's fixed-point theorem~\cite[CPO fixpoint theorem I, 8.15]{davey02}}]
    \label{lem:kleene-1}Assume $\posA$ is a \CPO, and~$f:\posA\rightarrow\posA$ is \scottcontinuous.
    Then the least fixed point of~$f$ is the supremum of the Kleene ascent chain
    \begin{equation*}
        \posbot\posleq f(\posbot)\posleq f(f(\posbot))\posleq\cdots\posleq f^{(n)}(\posbot)\leq\cdots.
    \end{equation*}
\end{lemma}
