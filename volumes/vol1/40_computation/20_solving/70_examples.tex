% !TEX root = chapter-standalone.tex

\section{Extended Numerical Examples}
\label{sec:Numerical-examples}

This example considers the choice of different battery technologies for a robot.
The goals of this example are:
1) to show how \SY{design problems} can be composed;
2) to show how to define hard constraints and precedence between resources to be minimized;
3) to show how even relatively simple models can give very complex trade-offs surfaces;
and 4) to introduce MCDPL, a formal language for the description of co-design problems.

\begin{figure*}[p]
    \centering
    \subfloat[\label{fig:battery_nicad}
        MCDPL code equivalent to equations \crefrange{eq:mass}{eq:cost}.
    ]{
        \begin{centering}
            \includegraphics[scale=0.55]{batteries_battery_code}
        \end{centering}
    }
    \subfloat[\label{fig:Co-design-diagram}
        Co-design diagram generated by PyMCDP from code in panel~(b).
    ]{%
        \centering
        \hspace{2cm}\includegraphics[scale=0.35]{batteries_battery_parsed}\hspace{2cm}
    }\medskip{}

    \subfloat[\label{fig:Tree-representation-of}
        Tree representation using $\dppar$/$\dpseries$
        of diagram in panel (c).
    ]{
        \centering
        \includegraphics[width=10cm]{batteries_nicad_tree}
    }
    \par

    \caption{Panel (c) shows the co-design diagram generated from the code in~(b).
        Panel (d) shows a tree representation (series, parallel) for the diagram.
        The edges show the types of functionality and resources.
        The leaves are labeled with the Python class used internally by the interpreter PyMCDP.
    }
\end{figure*}

\subsection{Language and interpreter/solver}

MCDPL is a modeling language to describe CDPs and their compositions.
It is inspired by CVX and ``disciplined convex programming''~\cite{grant08graph}.
MCDPL is even more disciplined than CVX; for example, multiplying by a negative number is a \emph{syntax} error.
The figures are generated by PyMCDP, an interpreter and solver for CDPs, which implements the techniques described in these sections.
An in-depth description of MCDPL is available in the next volume of this series.

\subsection{Model of a battery}
\begin{marginfigure}
    \centering
    \includegraphics[scale=0.3]{batteries_battery_dp}
    \caption{Interface of battery design problem.}
    \label{fig:battery}
\end{marginfigure}

The choice of a battery can be modeled as a DPI (\cref{fig:battery}) with functionalities \F{capacity {[}J{]}} and \F{number of missions} and with resources \R{mass {[}kg{]}}, \R{cost {[}\${]}} and ``\R{maintenance}'', defined as the number of times that the battery needs to be repl\-aced over the lifetime of the robot.

Each battery technology is described by the three parameters specific energy, specific cost, and lifetime (number of cycles):
\begin{align}
    \rho   & \definedas\text{specific energy [Wh/kg]}, \\
    \alpha & \definedas\text{specific cost [Wh/\$]}, \\
    c      & \definedas\text{battery lifetime [\# of cycles]}.
\end{align}
The relation between functionality and resources is described by three nonlinear monotone constraints:
\begin{align}
    \R{\text{mass}}        & \geq\F{\text{capacity}}/\rho,\label{eq:mass} \\
    \R{\text{maintenance}} & \geq\left\lceil \F{\text{missions}}/c\right\rceil,\label{eq:maintenance} \\
    \R{\text{cost}}        & \geq\left\lceil \F{\text{missions}}/c\right\rceil (\F{\text{capacity}}/\alpha).
    \label{eq:cost}
\end{align}

\Cref{fig:battery_nicad} shows the MCDPL code that describes the model corresponding to \crefrange{eq:mass}{eq:cost}.
The diagram in \cref{fig:Co-design-diagram} is automatically generated from the code.
\cref{fig:Tree-representation-of}~shows a tree representation of the diagram using the $\dpseries$/$\dppar$ operators.

\subsection{Competing battery technologies}

The parameters for the battery technologies used in this example are shown in~\cref{tab:batteries}.

\begin{table}[h]
    \caption{Specifications of common batteries technologies}
    \centering{}\setlength\extrarowheight{0.5pt}{\footnotesize{}}
    \begin{tabular}{crrr}
        \multirow{2}{*}{{\footnotesize{}\tableColors}\emph{\footnotesize{}technology}} & \emph{\footnotesize{}energy density} & \emph{\footnotesize{}specific cost} & \emph{\footnotesize{}operating life}\tabularnewline
                                                                                       & {\footnotesize{}{[}
        Wh/kg{]}}                                                                      & {\footnotesize{}{[}Wh/\${]}}         & \# cycles\tabularnewline
        {\footnotesize{}NiMH}                                                          & {\footnotesize{}100}                 & {\footnotesize{}3.41}               & {\footnotesize{}500 }\tabularnewline
        {\footnotesize{}NiH2}                                                          & {\footnotesize{}45}                  & {\footnotesize{}10.50}              & {\footnotesize{}20000}\tabularnewline
        {\footnotesize{}LCO}                                                           & {\footnotesize{}195}                 & {\footnotesize{}2.84}               & {\footnotesize{}750}\tabularnewline
        {\footnotesize{}LMO}                                                           & {\footnotesize{}150}                 & {\footnotesize{}2.84}               & {\footnotesize{}500}\tabularnewline
        {\footnotesize{}NiCad}                                                         & {\footnotesize{}30}                  & {\footnotesize{}7.50}               & {\footnotesize{}500}\tabularnewline
        {\footnotesize{}SLA}                                                           & {\footnotesize{}30}                  & {\footnotesize{}7.00}               & {\footnotesize{}500}\tabularnewline
        {\footnotesize{}LiPo}                                                          & {\footnotesize{}250}                 & {\footnotesize{}2.50}               & {\footnotesize{}600}\tabularnewline
        {\footnotesize{}LFP}                                                           & {\footnotesize{}90}                  & {\footnotesize{}1.50}               & {\footnotesize{}1500}\tabularnewline
    \end{tabular}
    \label{tab:batteries}
\end{table}

Each row of the table is used to describe a model as in \cref{fig:battery_nicad} by plugging in the specific values in lines 12--14.

Given the different models, we can define their coproduct (\cref{fig:Co-product-of-battery}) using the MCDPL code in~\cref{fig:batteries_code}.

\begin{figure}[h]
    \subfloat[Co-product of battery technologies \label{fig:Co-product-of-battery}]{
        \includegraphics[scale=0.33]{batteries_batteries_dia}
    }
    \subfloat[\label{fig:batteries_code}
        Batteries.mcdp]{
        \centering
        \includegraphics[scale=0.66]{batteries_batteries_code}
    }

    \caption{
        The coproduct of \SY{design problems} describes the choices among different technologies.
        The MCDPL keyword for the coproduct is ``choose''.
    }
    \label{fig:batteriesbig}
\end{figure}

\subsection{Introducing other variations or objectives}

The \SY{design problem} for the battery has two functionalities (\F{capacity} and \F{number of missions}) and three resources (\R{cost}, \R{mass}, and \R{maintenance}).
Thus, it describes a family of multi-objective optimization problems, of the type ``Given \F{capacity} and \F{missions}, minimize $\tup{\R{\text{cost}},\R{\text{mass}},\R{\text{maintenance}}}$''.
We can further extend the class of optimization problems by introducing other hard constraints and by choosing which resource to prioritize.
This can be done by composition of \SY{design problems}; that is, by creating a larger DP that contains the original DP as a subproblem, and contains some additional degenerate DPs that realize the desired semantics.

For example, suppose that we would like to find the optimal solution(s) such that: 1) The mass does not exceed 3 kg; 2) The mass is minimized as a primary objective, while cost/maintenance are secondary objectives.

This semantics can be described by the co-design diagram in~\cref{fig:diagram}, which contains two new symbols.
The DP labeled ``3 kg'' implements the semantics of hard constraints.
It has one functionality ($\funsp=\F{\Rcomp^{\text{kg}}}$) and zero resources~($\ressp=\R{\One}$).
The poset~$\One=\makeset{\tup{} }$ has exactly two \SY{antichains}: $\Emptyset$ and $\makeset{\tup{} }$.
These represent ``infeasible'' and ``feasible'', respectively.
The DP is described by the map
\todotext{\alphubel: fix minipage}
\quad\quad
\begin{minipage}[c]{6.5cm}
    \begin{align}
        \ftor:\F{\Rcomp^{\text{kg}}} & \rightarrow\R{\antichains\One}, \\
        \fun                         & \mapsto\begin{cases}
                                                  \R{\makeset{\tup{} }}, & \text{if }\fun\leq\text{3 kg}, \\
                                                  \R{\Emptyset},         & \text{if }\fun>\text{3 kg}.
                                              \end{cases}
    \end{align}

\end{minipage}\quad\includegraphics[scale=0.45]{batteries_3kg}

\smallskip{}

\noindent The block labeled~``$\postop$'' is similarly defined and always returns ``feasible'', so it has the effect of ignoring \R{cost} and \R{maintenance} as objectives.
The only resource edge is the one for \R{mass}, which is then the only objective.

The MCDPL code is shown in~\cref{fig:diagram_code}.
Note the intuitive interface: the user can directly write ``mass required by battery $\leq$ 3 kg'' and ``ignore maintenance required by battery'', which is compiled to ``maintenance required by battery $\leq\postop$''.

\begin{figure}
    \begin{centering}
        \subfloat[
            \label{fig:diagram}
            Co-design diagram that expresses hard constraints for \R{mass}.
        ]{
            \centering
            \begin{minipage}[t]{8.6cm}
                \includegraphics[scale=0.4]{batteries_batteries_constraints}
            \end{minipage}
        }
    \end{centering}
    \begin{centering}
        \subfloat[\label{fig:diagram_code}
            MCDPL code equivalent to diagram in (a).
        ]{
            \begin{centering}
                \includegraphics[width=6cm]{batteries_mixed_code}
            \end{centering}
        }
    \end{centering}
    \smallskip{}

    \caption{
        Composition of DPs can express hard constraints and precedence of objectives.
        In this case, there is a hard constraint on the \R{mass}.
        Because there is only one outgoing edge for \R{mass}, and the \R{cost} and \R{maintenance} are terminated by a dummy constraint ($x\posleq\postop$), the semantics of the diagram is that the objective is to minimize the \R{mass} as primary objective.
    }
\end{figure}

This relatively simple model for energetics already shows the complexity of CDPs.
\Cref{fig:mainbattery} shows the optimal choice of the battery technology as a function of capacity and number of missions, for several slight variations of the problem that differ in constraints and objectives.
For each battery technology, the figures show whether at each operating point the technology is the optimal choice, and how many optimal choices there are.
Some results are intuitive.
For example, \cref{fig:min_mass} shows that if the only objective is minimizing \R{mass}, then the optimal choice is simply the technology with the largest specific energy (LiPo).
The decision boundaries become complex when considering nonlinear objectives.
For example, \cref{fig:min_cost} shows the case where the objective is to minimize the \R{cost}, which, defined by~\cref{eq:cost}, is nonlinearly related to both \F{capacity} and \F{number of missions}.
When considering multi-objective problems, such as minimizing jointly $\tup{\R{\text{mass}},\R{\text{cost}}} $~(\cref{fig:min_mass_cost}) or~$\tup{\R{\text{mass}},\R{\text{cost}},\R{\text{maintenance}}} $ (\cref{fig:min_mass_cost}), there are multiple non-dominated solutions.

\begin{figure*}
    \begin{centering}
        \subfloat[
            \label{fig:drone}\label{fig:drone_dia}
            Co-design diagram corresponding to~\crefrange{eq:drone_eq_first}{eq:drone_eq_last}.
        ]{\centering{}\includegraphics[scale=0.38]{batteries_overall}}

    \end{centering}
    \begin{centering}
        \subfloat[\label{fig:drone_code}
            MCDPL code for~\crefrange{eq:drone_eq_first}{eq:drone_eq_last}.
            The ``instance'' statements refer to previously defined models for batteries (\cref{fig:batteries_code}) and actuation (not shown).
        ]{
            \begin{centering}
                \begin{minipage}[t]{8.6cm}
                    \centering
                    \includegraphics[scale=0.55]{batteries_overall_code}
                \end{minipage}
            \end{centering}
        }
    \end{centering}
    \subfloat[\label{fig:drone_tree}
        Tree representation for the CDP.
        Yellow/green rounded ovals are $\dpseries$/$\dppar$ junctions.
        There is one coproduct junction, signifying the choice between different battery technologies, and one~$\dploop$ junction, at the root of the tree.
    ]{
        \begin{centering}
            \includegraphics[width=8.6cm]{batteries_drone_tree}
        \end{centering}
    }

    \subfloat[\label{fig:drone-endurance-missions}
        Relation between \F{endurance} and \F{number of missions} and \R{cost} and \R{mass}.
    ]{
        \begin{centering}
            \includegraphics[width=8.6cm]{batteries_drone_plane1}
        \end{centering}
    }

    \subfloat[\label{fig:drone-endurance-payload}
        Relation between \F{endurance}
        and \F{payload} and \R{cost} and \R{mass}.
    ]{
        \centering
        \includegraphics[width=8.6cm]{batteries_drone_plane2}
    }

    \caption{
        In panel (c), the \F{payload} is fixed to 100 g and \F{extra power} is set to 1~W.
        In panel (d), the \F{number of missions} is fixed to 400 and \F{extra power} is set to 1~W.
        The last two values, marked with ``$\times$'', are not feasible.
    }
    \label{fig:dronebigfig}
\end{figure*}

\subsection{From component to system co-design}

The rest of the section reuses the battery DP into a larger co-design problem that considers the co-design of actuation together with energetics for a drone~(\cref{fig:drone_dia}).
We will see that the decision boundaries change dramatically, which shows that the optimal choices for a component cannot be made in isolation from the system.

The functionality of the drone's subsystem considered~(\cref{fig:drone}) are parametrized by \F{endurance}, \F{number of missions}, \F{extra
    power} to be supplied, and \F{payload}.
We model ``actuation''
as a \SY{design problem} with functionality \F{lift {[}N{]}} and resources \R{cost}, \R{mass} and \R{power}, and we assume that power is a quadratic function of lift~(\cref{fig:actuation}).
Any other \SY{monotone map} could be used.

\captionsideleft{\label{fig:actuation}}{
    \includegraphics[scale=0.33]{batteries_actuation_dp}
}

The co-design constraints that combine energetics and actuation are
    {\small{}
        \begin{align}
            \text{battery }\F{\text{capacity}} & \geq\text{total power}\Ptimes \F{\text{endurance}},\label{eq:drone_eq_first} \\
            \text{total power}                 & =\text{actuation }\R{\text{power}}+\F{\text{extra power}},\nonumber \\
            \text{weight}                      & =\text{total mass}\cartprod\text{gravity},\nonumber \\
            \text{actuation}\,\F{\text{lift}}  & \geq\text{weight},\nonumber \\
            \text{labor cost}                  & =\text{cost per replacement}\cartprod\text{battery }\R{\text{maintenance}},\nonumber \\
            \R{\text{total cost}}              & =\text{battery}\,\R{\text{cost}}+\text{actuation}\,\R{\text{cost}}+\text{labor cost},\nonumber \\
            \R{\text{total mass}}              & =\text{battery}\,\R{\text{mass}}+\text{actuation}\,\R{\text{mass}}+\F{\text{payload}}.
            \label{eq:drone_eq_last}
        \end{align}
    }
The co-design graph contains recursive constraints: the power for
actuation depends on the total weight, which depends on the mass of
the battery, which depends on the capacity to be provided, which depends
on the power for actuation.
The MCDPL code for this model is shown
in \cref{fig:drone_code}; it refers to the previously defined models
for ``batteries'' and ``actuation''.

The co-design problem is now complex enough that we can appreciate
the compositional properties of CDPs to perform a qualitative analysis.
Looking at~\cref{fig:drone}, we know that there is a monotone relation
between any pair of functionality and resources, such as \F{payload}
and \R{cost}, or \F{endurance} and \R{mass}, even without knowing
exactly what are the models for battery and actuation.

When fully expanded, the co-design graph (too large to display) contains
110 nodes and 110 edges.
It is possible to remove all cycles by removing
only one edge (\eg, the $\R{\text{energy}}\leq\text{\F{\text{capacity}}}$
constraint), so the design complexity (\cref{def:design-complexity})
is equal to~$\posetwidth(\Rcomp)=1$.
The
tree representation is shown in~\cref{fig:drone_tree}.
Because the co-design diagram contains cycles, there is a~$\dploop$ operator at the root of the tree, which implies we need to solve a \SY{least fixed point} problem.
Because of the scale of the problem, it is not possible to show the map~\ftor explicitly, like we did in \cref{eq:expression} for the previous example.
The \SY{least fixed point} sequence converges to 64 bits machine precision in 50--100 iterations.

To visualize the multidimensional relation
\begin{equation}
    \ftor\colon\F{\Rcomp\Ptimes\Rcomp^{\text{s}}\Ptimes\Rcomp^{\text{W}}\Ptimes\Rcomp^{\text{g}}}\rightarrow\R{\antichains(\Rcomp^{\text{kg}}\Ptimes\Rcomp^{\USD})},
\end{equation}
we need to project across 2D slices.
\cref{fig:drone-endurance-missions}~shows
the relation when the functionality varies in a \SY{chain} in the space
\F{endurance}/\F{missions}, and~\cref{fig:drone-endurance-payload}
shows the results for a \SY{chain} in the space \F{endurance}/\F{payload}.

Finally, \cref{fig:drone_choice} shows the optimal choices of battery technologies in the \F{endurance}/\F{missions} space, when one wants to minimize \R{mass}, \R{cost}, or~$\tup{\R{\text{mass}},\R{\text{cost}}} $.
The decision boundaries are completely different from those in~\cref{fig:mainbattery}.
This shows that it is not possible to optimize a component separately from the rest of the system, if there are cycles in the co-design diagram.

\begin{figure*}
    \begin{centering}
        \hfill\includegraphics[width=0.8\textwidth]{batteries_legend}\rule{2cm}{0pt}

    \end{centering}
    \begin{centering}
        \subfloat[\label{fig:min_joint_dia}]{
            \begin{centering}
                \includegraphics[scale=0.33]{batteries_min_joint_dia}

            \end{centering}
            \vspace{5mm}

        }\subfloat[\label{fig:min_joint}]{
            \begin{centering}
                \includegraphics[scale=0.45]{batteries_min_joint}

            \end{centering}
        }

    \end{centering}
    \begin{centering}
        \subfloat[\label{fig:min_cost_dia}]{
            \begin{centering}
                \includegraphics[scale=0.33]{batteries_min_cost_dia}

            \end{centering}
        }\subfloat[\label{fig:min_cost}]{
            \begin{centering}
                \includegraphics[scale=0.45]{batteries_min_cost}

            \end{centering}
        }

    \end{centering}
    \begin{centering}
        \subfloat[]{
            \begin{centering}
                \includegraphics[scale=0.33]{batteries_min_mass_dia}

            \end{centering}
        }\subfloat[\label{fig:min_mass}]{
            \begin{centering}
                \includegraphics[scale=0.45]{batteries_min_mass}
            \end{centering}
        }
    \end{centering}
    \begin{centering}
        \subfloat[]{
            \begin{centering}
                \includegraphics[scale=0.33]{batteries_min_mass_cost_dia}
            \end{centering}
        }\subfloat[\label{fig:min_mass_cost}]{
            \begin{centering}
                \includegraphics[scale=0.45]{batteries_min_cost_mass}
            \end{centering}
        }

    \end{centering}
    \caption{This figure shows the optimal decision boundaries for the different battery technologies for the \SY{design problem} ``batteries'', defined as the coproduct of all battery technologies (\cref{fig:batteriesbig}).
        Each row shows a different variation of the problem.
        The first row (panels \emph{a}--\emph{b}) shows the case where the objective function is the product of~$\tup{\R{\text{mass}},\R{\text{cost}},\R{\text{maintenance}}} $.
        The shape of the symbols shows how many minimal solutions exists for a particular value of the functionality~$\tup{\F{\text{capacity}},\F{\text{missions}}} $.
        In this case, there are always three or more minimal solutions.
        The second row (panels \emph{c}--\emph{d}) shows the decision boundaries when minimizing only the scalar objective~$\R{\text{cost}}$, with a hard constraint on \R{mass}.
        The hard constraints make some combinations of the functionality infeasible.
        Note how the decision boundaries are nonconvex, and how the formalism allows defining slight variations of the problem.
    }
    \label{fig:mainbattery}
\end{figure*}
\begin{figure*}
    \begin{centering}
        \subfloat[]{
            \begin{centering}
                \includegraphics[scale=0.33]{batteries_drone_min_mass_cost_dia}
            \end{centering}
        }
        \subfloat[]{
            \begin{centering}
                \includegraphics[scale=0.45]{batteries_drone_min_mass_cost}
            \end{centering}
        }
    \end{centering}
    \begin{centering}
        \subfloat[]{
            \begin{centering}
                \includegraphics[scale=0.33]{batteries_drone_min_cost_dia}
            \end{centering}
        }
        \subfloat[]{
            \begin{centering}
                \includegraphics[scale=0.45]{batteries_drone_min_cost}
            \end{centering}
        }
    \end{centering}
    \begin{centering}
        \subfloat[]{
            \begin{centering}
                \includegraphics[scale=0.33]{batteries_drone_min_mass_dia}
            \end{centering}
        }
        \subfloat[]{
            \begin{centering}
                \includegraphics[scale=0.45]{batteries_drone_min_mass}
            \end{centering}
        }
    \end{centering}
    \caption{This figure shows the decision boundaries for the different values of battery technologies for the integrated actuation-energetics model described in \cref{fig:dronebigfig}.
        Please see the caption of~\cref{fig:mainbattery} for an explanation of the symbols.
        Notice how in most cases the decision boundaries are different from those in~\cref{fig:mainbattery}: this is an example in which one component cannot be optimized by itself without taking into account the rest of the system.
    }
    \label{fig:drone_choice}
\end{figure*}
