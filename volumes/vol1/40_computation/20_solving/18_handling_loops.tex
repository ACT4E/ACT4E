\section{Handling loops}

We are close to having a complete solution.
The only part that is missing is dealing with loops (trace).

\begin{marginfigure}
    \includegraphics[scale=0.33]{gmcdp_sloop2}
    \caption{}
    \label{fig:gmcdp_sloop2}
    \todographicsjira{711}{\alphubel: @Gioele: remake figure}
\end{marginfigure}

First, we will work with a particular form of loops, called ``the Conway form'',
shown in~\cref{fig:gmcdp_sloop2}.
This corresponds to working with \SY{design problems} of the type
\begin{equation}
    \adp \colon \funspa \Ptimes \funspb \profto \resspb,
\end{equation}
The new feedback operator has signature
\begin{equation}
    \dploop\colon (\funspa \Ptimes \funspb \profto \resspb) \to (\funspa \profto \resspb)
\end{equation}
%
We can do this without loss of generality, because we can re-write the trace
with this new operator.
We leave \cref{fig:can-rewrite} as a graphical proof that this is possible.
Hagesawa~\cite{hasegawa02uniformity} discusses the equivalence in detail.

\begin{figure*}[h]
    \hspace*{\fill}
    \subfloat[\label{fig:loop_generalb}]{
        \includegraphics[scale=0.33]{gmcdp_loop_general}
    }
    \hspace*{\fill}
    \subfloat[\label{fig:loop_general2b}]{
        \includegraphics[scale=0.33]{gmcdp_loop_general2}
    }
    \hspace*{\fill}
    \caption{We can rewrite the trace in Conway's form.}
    \label{fig:can-rewrite}
    \todographicsjira{711}{\alphubel: @Gioele: Re-do graphics with symbols. loopb = trace}
\end{figure*}

The following theorem establishes a closed form for~$\ftor_{\dploop(\adp)}$ as a \SY{least fixed point}.
Here on we consider~$\fAressp$ as a \SY{poset} with the order given by
\begin{equation}
    \prfdoubleperiod{
        \RR_\R{1} \ \posleqof{\fAressp} \ \RR_\R{2}
    }{
        \upit \RR_\R{1} \ \posleqof{\fuppersets \ressp}\ \upit \RR_\R{2}
    }
\end{equation}

\begin{theorem}
    \label{prop:loop-continuous}
    For any DP~$\adp$ of the right shape, we can compute~$\ftor_{\dploop(\adp)}$ as follows:
    \begin{equation}
        \label{eq:loop_fixpoint1}
        \ftor_{\dploop(\adp)}\colon\funan{1}\mapsto\upit\lfp(\Phi_{\funan{1}}),
    \end{equation}
    that is, as the \emph{\SY{least fixed point}} of a map~$\Phi_{\funan{1}}$ defined as
    \begin{eqnarray}
        \Phi_{\funan{1}}\colon\fAressp & \to     & \fAressp,\label{eq:bigphi1} \\
        \RR                            & \mapsto & \Min_{\resleq}\bigsetunion_{\res\setin\RR}\ftor_{\adp}(\funan{1}, \res)\ \setintersection\ \upit\res.
        \nonumber
    \end{eqnarray}
\end{theorem}
\begin{proof}
    The diagram in \cref{fig:gmcdp_sloop2} implies that the map~$\ftor_{\dploop(\adp)}$ can be described as:
    \begin{align}
        \ftor_{\dploop(\adp)}\colon\funspn{1} & \to\fAressp,\label{eq:loopproblem} \\
        \funan{1}                             & \mapsto\begin{cases}
                                                           \with          & \res,\funan{2}\setin\ressp,                  \\
                                                           \Min_{\resleq} & \res,                                        \\
                                                           \subto         & \res\setin\ftor_{\adp}(\funan{1},\funan{2}), \\
                                                                          & \res\resleq\funan{2}.
                                                       \end{cases}
    \end{align}
    Denote by~$\ftor_{\funan{1}}$ the map~$\ftor_{\adp}$ with the first element fixed:
    \begin{equation}
        \ftor_{\funan{1}}\colon\funan{2}\mapsto\ftor_{\adp}(\funan{1},\funan{2}).
    \end{equation}
    Rewrite~$\res\setin\ftor_{\adp}(\funan{1},\funan{2})$ in~\cref{eq:loopproblem} as
    \begin{equation}
        \res\setin\ftor_{\funan{1}}(\funan{2}).
        \label{eq:h2}
    \end{equation}
    Let~\res be a feasible solution, but not necessarily minimal.
    \Cref{lem:antichain-write} implies that the constraint~\cref{eq:h2} can be rewritten as
    \begin{equation}
        \makeset{\res}=\ftor_{\funan{1}}(\funan{2})\setintersection\upit\res.
        \label{eq:h3}
    \end{equation}
    Because~$\funan{2}\posgeq\res$, and $\ftor_{\funan{1}}$ is \SY{Scott continuous}, it follows that~$\ftor_{\funan{1}}(\funan{2})\posgeq_{\fAressp}\ftor_{\funan{1}}(\res)$.
    Therefore, by \cref{lem:antichain_inter}, we have
    \begin{equation}
        \makeset{\res}\posgeq_{\fAressp}\ftor_{\funan{1}}(\res)\setintersection\upit\res.
        \label{eq:fea}
    \end{equation}
    This is a recursive condition that all feasible~\res must satisfy.

    Let~$\RR\setin\fAressp$ be an \SY{antichain} of feasible resources, and let~\res be a generic element of~\ressp.
    Tautologically, rewrite~$\RR$ as the minimal elements of the union of the singletons containing its elements:
    \begin{equation}
        \RR=\Min_{\resleq}\bigsetunion_{\res\setin\RR}\ \makeset{\res}.
        \label{eq:condition3}
    \end{equation}
    Substituting~\cref{eq:fea} in~\cref{eq:condition3} we obtain (cf
    \cref{lem:antichain_union})
    \begin{equation}
        \RR\posgeq_{\fAressp}\Min_{\resleq}\bigsetunion_{\res\setin\RR}\ftor_{\funan{1}}(\res)\ \setintersection\ \upit\res.
        \label{eq:recursive}
    \end{equation}
    %
    Converse: It is also true that if an \SY{antichain}~$\RR$ satisfies~\cref{eq:recursive} then all~$\res\setin\RR$ are feasible.
    The constraint~\cref{eq:recursive} means that for any~$\resan{0}\setin\RR$ on the left side, we can find a~$\resan{1}$ on the right side so that~$\resan{0}\posgeq_{\ressp}\resan{1}$.
    The point~$\resan{1}$ needs to belong to one of the sets of which we take the union; say that it comes from~$\resan{2}\setin\RR$, so that~$\resan{1}\setin\ftor_{\funan{1}}(\resan{2})\ \setintersection\ \upit\resan{2}$.
    Summarizing:
    %
    \begin{equation}
        \forall\resan{0}\setin\RR\colon\ \exists\resan{1}\colon\ (\resan{0}\posgeq_{\ressp}\resan{1})\ \booland\ (\exists\resan{2}\setin\RR\colon\ \resan{1}\setin\ftor_{\funan{1}}(\resan{2})\ \setintersection\ \upit\resan{2}).
        \label{eq:conc}
    \end{equation}
    %
    Because~$\resan{1}\setin\ftor_{\funan{1}}(\resan{2})\,\setintersection\,\upit\resan{2}$, we can conclude that~$\resan{1}\setin\upit\resan{2}$, and therefore~$\resan{1}\posgeq_{\ressp}\resan{2}$, which together with~$\resan{0}\posgeq_{\ressp}\resan{1}$, implies~$\resan{0}\posgeq_{\ressp}\resan{2}$.
    We have concluded that there exist two points~$\resan{0},\resan{2}$ in the \SY{antichain}~$\RR$ such that~$\resan{0}\posgeq_{\ressp}\resan{2}$; therefore, they are the same point:~$\resan{0}=\resan{2}$.
    Because~$\resan{0}\posgeq_{\ressp}\resan{1}\posgeq_{\ressp}\resan{2}$, we also conclude that~$\resan{1}$ is the same point as well.
    We can rewrite~\cref{eq:conc} by using~$\resan{0}$ in place of~$\resan{1}$ and~$\resan{2}$ to obtain~$\forall\resan{0}\setin\RR\colon \resan{0}\setin\ftor_{\funan{1}}(\resan{0})$,
    which means that~$\resan{0}$ is a feasible resource.

    We have concluded that all \SY{antichains} of feasible resources~$\RR$ satisfy~\cref{eq:recursive}, and conversely, if an \SY{antichain}~$\RR$ satisfies~\cref{eq:recursive}, then it is an \SY{antichain} of feasible resources.

    \Cref{eq:recursive} is a recursive constraint for~$\RR$, of the kind
    \begin{equation}
        \Phi_{\funan{1}}(\RR)\posleqof{\fAressp}\RR,
    \end{equation}
    with the map~$\Phi_{\funan{1}}$ defined by
    \begin{eqnarray}
        \Phi_{\funan{1}}\colon \fAressp & \to     & \fAressp,\label{eq:bigphi} \\
        \RR                             & \mapsto & \Min_{\resleq}\bigsetunion_{\res\setin\RR}\ftor_{\funan{1}}(\res)\ \setintersection\ \upit\res.\nonumber
    \end{eqnarray}
    If we want the \emph{minimal} resources, we are looking for the \emph{least} \SY{antichain}:
    \begin{equation}
        \min_{\posleqof{\fAressp}}\makeset{\,\RR\setin\fAressp\colon\ \Phi_{\funan{1}}(\RR)\posleqof{\fAressp}\RR\,},
    \end{equation}
    which is equal to the \emph{least fixed point }of~$\Phi_{\funan{1}}$.
    Therefore, the map~$\ftor_{\dploop(\adp)}$ can be written as
    \begin{equation}
        \ftor_{\dploop(\adp)}\colon\funan{1}\mapsto\lfp(\Phi_{\funan{1}}).
        \label{eq:loop_fixpoint}
    \end{equation}
    \cref{lem:dagger} shows that~$\lfp(\Phi_{\funan{1}})$ is \SY{Scott continuous} in~$\funan{1}$.
\end{proof}
\begin{lemma}
    \label{lem:antichain-write}
    Let~$\subA$ be an \SY{antichain} in~\posA.
    Then
    \begin{equation}
        \prfdoubleperiod{
            \ela\setin \subA
        }{
            \makeset{\ela}=\subA\,\setintersection\upit \ela
        }
    \end{equation}
\end{lemma}

\begin{lemma}
    \label{lem:antichain_inter}
    For~$\subA,\subB\setin\fantichains\posA$, and~$\setA \setsubseteq \setB$,
    $\subA\posleqof{\fAressp}\subB$ implies~$\subA \setintersection \setA \posleqof{\fAressp}\subB\setintersection \setA$.
\end{lemma}

\begin{lemma}
    \label{lem:antichain_union}
    For~$\subA,\subB,\subC,\subD\setin\fantichains\posA$,~$\subA\posleqof{\fAressp}\subC$
    and~$\subB\posleqof{\fAressp}\subD$ implies~$\subA\setunion \subB\posleqof{\fAressp}\subC\setunion \subD.
    $
\end{lemma}

\begin{lemma}
    \label{lem:dagger}
    Let~$\mora\colon\posA\Ptimes\posB\toinPos\posB$ be \SY{Scott continuous}.
    For each~$\ela\setin\posAset$, define the map
    \begin{equation}
        \mora_{\ela}:\elb\mapsto \mora(\ela,\elb)
    \end{equation}
    Then the map
    \begin{equation}
        \mora^{\dagger}:\ela\mapsto\lfp(\mora_{\ela})
    \end{equation}
    is \SY{Scott continuous}.
\end{lemma}
\begin{proof}
    Davey and Priestly~\cite{davey02} leave this as Exercise~8.26.
    A proof is found in Gierz~\etal~\cite[Exercise II-2.29]{gierz03continuous}.
\end{proof}

