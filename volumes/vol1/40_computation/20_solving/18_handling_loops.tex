
\section{Handling loops}

We are close to having a complete solution.
The only part that is missing is dealing with loops (trace).

\begin{marginfigure}
    \includegraphics[scale=0.33]{gmcdp_sloop2}
    \caption{}
    \label{fig:gmcdp_sloop2}
    \todographicsjira{711}{@Gioele: remake figure}
\end{marginfigure}

First, we will work with a particular form of loops, called ``the Conway form'',
shown in~\cref{fig:gmcdp_sloop2}.
This corresponds to working with design problems of the type
\begin{equation}
    \adp \colon \funspa \times \funspb  \profto \resspb,
\end{equation}
The new feedback operator has signature
\begin{equation}
    \dploop: (\funspa \times \funspb \profto \resspb) \to (\funspa \profto \resspb)
\end{equation}

We can do this without loss of generality, because we can re-write the trace
with this new operator.
We leave \cref{fig:can-rewrite} as a graphical proof that this is possible.
Hagesawa~\cite{hasegawa02uniformity} discusses the equivalence in detail.

\begin{figure*}[h]
    \hspace*{\fill}
    \subfloat[\label{fig:loop_generalb}]{
        \includegraphics[scale=0.33]{gmcdp_loop_general}
    }
    \hspace*{\fill}
    \subfloat[\label{fig:loop_general2b}]{
        \includegraphics[scale=0.33]{gmcdp_loop_general2}
    }
    \hspace*{\fill}
    \caption{We can rewrite the trace in Conway's form.}
    \label{fig:can-rewrite}
    \todographicsjira{711}{Re-do graphics with symbols. loopb = trace}
\end{figure*}

The following theorem establishes a closed form for $\ftor_{\dploop(\adp)}$
as a least fixed point.
Here on we consider $\fAressp$ as a poset with the order given by
\begin{equation}
    \prfdoubleperiod{
        \RR_1 \  \posleq_{\fAressp} \ \RR_2
    }{
        \upit \RR_1 \ \posleq_{\fuppersets \ressp}\  \upit \RR_2
    }
\end{equation}

\begin{theorem}
    \label{prop:loop-continuous}
    For any DP $\adp$ of the right shape, we can compute $ \ftor_{\dploop(\adp)}$
    as follows:
    \begin{equation}\label{eq:loop_fixpoint1}
        \ftor_{\dploop(\adp)}\colon\fun_{1}\mapsto\upit\lfp(\Phi_{\fun_{1}}),
    \end{equation}
    that is, as the \emph{least fixed point} of a map $\Phi_{\fun_{1}}$ defined as
    \begin{eqnarray}
        \Phi_{\fun_{1}}:\fAressp & \to     & \fAressp,\label{eq:bigphi1} \\
        \RR                      & \mapsto & \Min_{\resleq}\bigsetunion_{\res\setin\RR}\ftor_{\adp}(\fun_{1}, \res)\ \setintersection\ \upit\res.
        \nonumber
    \end{eqnarray}
\end{theorem}
\begin{proof}
    The diagram in \cref{fig:gmcdp_sloop2} implies that the map~$\ftor_{\dploop(\adp)}$
    can be described as:
    \begin{align}
        \ftor_{\dploop(\adp)}\colon\funsp_{1} & \to\fAressp,\label{eq:loopproblem} \\
        \fun_{1}                              & \mapsto\begin{cases}
                                                           \with          & \res,\fun_{2}\setin\ressp,                 \\
                                                           \Min_{\resleq} & \res,                                      \\
                                                           \subto         & \res\setin\ftor_{\adp}(\fun_{1},\fun_{2}), \\
                                                                          & \res\resleq\fun_{2}.
                                                       \end{cases}
    \end{align}
    Denote by~$\ftor_{\fun_{1}}$ the map~$\ftor_{\adp}$ with the first element fixed:
    \begin{equation*}
        \ftor_{\fun_{1}}\colon\fun_{2}\mapsto\ftor_{\adp}(\fun_{1},\fun_{2}).
    \end{equation*}
    Rewrite $\res\setin\ftor_{\adp}(\fun_{1},\fun_{2})$ in~\cref{eq:loopproblem} as
    \begin{equation}
        \res\setin\ftor_{\fun_{1}}(\fun_{2}).
        \label{eq:h2}
    \end{equation}
    Let~\res be a feasible solution, but not necessarily minimal.
    Because of \cref{lem:antichain-write}, the constraint~\cref{eq:h2} can be rewritten as
    \begin{equation}
        \{\res\}=\ftor_{\fun_{1}}(\fun_{2})\setintersection\upit\res.
        \label{eq:h3}
    \end{equation}
    Because $\fun_{2}\posgeq\res$, and $\ftor_{\fun_{1}}$ is \scottcontinuous, it follows that~$\ftor_{\fun_{1}}(\fun_{2})\posgeq_{\fAressp}\ftor_{\fun_{1}}(\res)$.
    Therefore, by \cref{lem:antichain_inter}, we have
    \begin{equation}
        \{\res\}\posgeq_{\fAressp}\ftor_{\fun_{1}}(\res)\setintersection\upit\res.
        \label{eq:fea}
    \end{equation}
    This is a recursive condition that all feasible~\res must satisfy.

    Let $\RR\setin\fAressp$ be an antichain of feasible resources, and let~\res be a generic element of~\ressp.
    Tautologically, rewrite~$\RR$ as the minimal elements of the union of the singletons containing its elements:
    \begin{equation}
        \RR=\Min_{\resleq}\bigsetunion_{\res\setin\RR}\ \{\res\}.
        \label{eq:condition3}
    \end{equation}
    Substituting~\cref{eq:fea} in~\cref{eq:condition3} we obtain (cf
    \cref{lem:antichain_union})
    \begin{equation}
        \RR\posgeq_{\fAressp}\Min_{\resleq}\bigsetunion_{\res\setin\RR}\ftor_{\fun_{1}}(\res)\ \setintersection\ \upit\res.
        \label{eq:recursive}
    \end{equation}
    %
    Converse: It is also true that if an antichain~$\RR$ satisfies~\cref{eq:recursive} then all~$\res\setin\RR$ are feasible.
    The constraint~\cref{eq:recursive} means that for any~$\res_{0}\setin\RR$ on the left side, we can find a~$\res_{1}$ on the right side so that~$\res_{0}\posgeq_{\ressp}\res_{1}$.
    The point~$\res_{1}$ needs to belong to one of the sets of which we take the union; say that it comes from $\res_{2}\setin\RR$, so that $\res_{1}\setin\ftor_{\fun_{1}}(\res_{2})\ \setintersection\ \upit\res_{2}$.
    Summarizing:
    %
    \begin{equation}
        \forall\res_{0}\setin\RR:\ \exists\res_{1}\colon\ (\res_{0}\posgeq_{\ressp}\res_{1})\ \wedge\ (\exists\res_{2}\setin\RR\colon\ \res_{1}\setin\ftor_{\fun_{1}}(\res_{2})\ \setintersection\ \upit\res_{2}).
        \label{eq:conc}
    \end{equation}
    %
    Because~$\res_{1}\setin\ftor_{\fun_{1}}(\res_{2})\,\setintersection\,\upit\res_{2}$, we can conclude that~$\res_{1}\setin\upit\res_{2}$, and therefore~$\res_{1}\posgeq_{\ressp}\res_{2}$, which together with~$\res_{0}\posgeq_{\ressp}\res_{1}$, implies~$\res_{0}\posgeq_{\ressp}\res_{2}$.
    We have concluded that there exist two points~$\res_{0},\res_{2}$ in the antichain~$\RR$ such that~$\res_{0}\posgeq_{\ressp}\res_{2}$; therefore, they are the same point:~$\res_{0}=\res_{2}$.
    Because~$\res_{0}\posgeq_{\ressp}\res_{1}\posgeq_{\ressp}\res_{2}$, we also conclude that~$\res_{1}$ is the same point as well.
    We can rewrite~\cref{eq:conc} by using~$\res_{0}$ in place of~$\res_{1}$ and~$\res_{2}$ to obtain~$\forall\res_{0}\setin\RR:\res_{0}\setin\ftor_{\fun_{1}}(\res_{0})$,
    which means that~$\res_{0}$ is a feasible resource.

    We have concluded that all antichains of feasible resources~$\RR$ satisfy~\cref{eq:recursive}, and conversely, if an antichain~$\RR$ satisfies~\cref{eq:recursive}, then it is an antichain of feasible resources.

    \Cref{eq:recursive} is a recursive constraint for~$\RR$, of the kind
    \begin{equation*}
        \Phi_{\fun_{1}}(\RR)\posleqof{\fAressp}\RR,
    \end{equation*}
    with the map~$\Phi_{\fun_{1}}$ defined by
    \begin{eqnarray}
        \Phi_{\fun_{1}}:\fAressp & \to     & \fAressp,\label{eq:bigphi} \\
        \RR                      & \mapsto & \Min_{\resleq}\bigsetunion_{\res\setin\RR}\ftor_{\fun_{1}}(\res)\ \setintersection\ \upit\res.\nonumber
    \end{eqnarray}
    If we want the \emph{minimal} resources, we are looking for the \emph{least} antichain:
    \begin{equation*}
        \min_{\posleqof{\fAressp}}\{\,\RR\setin\fAressp\colon\ \Phi_{\fun_{1}}(\RR)\posleqof{\fAressp}\RR\,\},
    \end{equation*}
    which is equal to the \emph{least fixed point }of~$\Phi_{\fun_{1}}$.
    Therefore, the map $\ftor_{\dploop(\adp)}$ can be written as
    \begin{equation}
        \ftor_{\dploop(\adp)}\colon\fun_{1}\mapsto\lfp(\Phi_{\fun_{1}}).
        \label{eq:loop_fixpoint}
    \end{equation}
    \cref{lem:dagger} shows that $\lfp(\Phi_{\fun_{1}})$ is \scottcontinuous in~$\fun_{1}$.
\end{proof}
\begin{lemma}
    \label{lem:antichain-write}
    Let~$\subA$ be an antichain in~$\posA$.
    Then
    \begin{equation*}
        \prfdoubleperiod{
            \ela\setin \subA
        }{
            \makeset{\ela}=\subA\,\setintersection\upit \ela
        }
    \end{equation*}
\end{lemma}

\begin{lemma}
    \label{lem:antichain_inter}
    For~$A,B\setin\fantichains\posA$, and~$S\setsubseteq P$,
    $A\posleqof{\fAressp}B$ implies $A\setintersection S\posleqof{\fAressp}B\setintersection S$.
\end{lemma}

\begin{lemma}
    \label{lem:antichain_union}
    For~$A,B,C,D\setin\fantichains\posA$, $A\posleqof{\fAressp}C$
    and $B\posleqof{\fAressp}D$ implies $A\setunion B\posleqof{\fAressp}C\setunion D.
    $
\end{lemma}

\begin{lemma}
    \label{lem:dagger}
    Let~$\mora\colon\posA\Ctimes\posB\toinPos\posB$ be \scottcontinuous.
    For each~$\ela\setin\posAset$, define the map
    \begin{equation}
        \mora_{\ela}:\elb\mapsto \mora(\ela,\elb)
    \end{equation}
    Then the map
    \begin{equation}
        \mora^{\dagger}:\ela\mapsto\lfp(\mora_{\ela})
    \end{equation} is \scottcontinuous.
\end{lemma}
\begin{proof}
    Davey and Priestly~\cite{davey02} leave this as Exercise~8.26.
    A proof is found in Gierz~\etal~\cite[Exercise II-2.29]{gierz03continuous}.
\end{proof}

