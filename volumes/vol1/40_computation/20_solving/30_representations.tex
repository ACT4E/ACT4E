% !TEX root = chapter-standalone.tex


\section{Computational representation of DPIs}
\label{sec:solving-representation-of-dpis}

\linkvideo{spring2021-functorial-comp-a:solving-queries:multi-loop} % Canonical form


It is useful to also describe a design problem as a map from functionality to resources or viceversa that abstracts over implementations.

A useful analogy is the state space representation \emph{vs} the transfer function representation of a linear time-invariant system: the state space representation is richer, but we only need the transfer function to characterize the input-output response.

\todojira{218}{define the map from functionality to upper sets (somewhere already)}

\begin{definition}
  \label{def:ftor}
  Given a DPI $\tup{ \funsp,\ressp,\impsp,\prov,\req} $,
  define the map~$\ftor\colon \funsp\to\Aressp$ that associates
  to each functionality~\fun the objective function of~\cref{prob:problem1},
  which is the set of minimal resources necessary to realize~\fun:
  \begin{eqnarray*}
    \ftor\colon \funsp & \to & \Aressp,\\
    \fun& \mapsto & \resMin\{\req(\imp)\mid\left(\imp\in\impsp\right)\,\wedge\,\left(\fun\posleq\prov(\imp)\right)\}.
  \end{eqnarray*}
  If a certain functionality~\fun is infeasible, then~$\ftor(\fun)=\emptyset$.
\end{definition}
\captionsideleft{\label{fig:setup_h-1}}{\includegraphics[scale=0.33]{gmcdp_setup_h}}

\todojira{219}{add remark of when this might fail (antichains do not capture the uppersets)}


\begin{example}
  In the case of the motor design problem, the map~\ftor assigns
  to each pair of~$\tup{\F{\text{speed}},\F{\text{torque}}}$
  the achievable trade-off of \R{cost}, \R{mass}, and other resources~(\cref{fig:motor-trade-offs}).
  The antichains are depicted as continuous curves, but they could also
  be composed by a finite set of points.

\end{example}

\captionsideleft{\label{fig:motor-trade-offs}}{
  \includegraphics[scale=0.33]{gmcdp_motor_tradeoffs}
}

By construction, \ftor is monotone (\cref{def:monotone}), which means that
\begin{equation*}
  \fun_{1}\funleq\fun_{2}\quad\Imp\quad\ftor(\fun_{1})\posleq_{\Aressp}\ftor(\fun_{2}),
\end{equation*}
where~$\posleq_{\Aressp}$ is the order on antichains defined in \cref{lem:antichains-are-poset}.

Monotonicity of~\ftor means that if the functionality~\fun is increased the antichain of resources will go ``up'' in the poset of antichains~$\Aressp$,
and at some point it might reach the top of~$\Aressp$, which is the empty set, meaning that the problem is not feasible.


\todojira{221}{describe equally the map $\rtof$}

\todojira{222}{Style of whole section.}