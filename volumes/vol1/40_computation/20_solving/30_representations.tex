% !TEX root = chapter-standalone.tex


\section{Computational representation of DPIs}
\label{sec:solving-representation-of-dpis}

\linkvideo{spring2021-functorial-comp-b:solving-queries:multi-loop} % Canonical form


It is useful to also describe a design problem as a map from functionality to resources or viceversa that abstracts over implementations.

A useful analogy is the state space representation \emph{vs} the transfer function representation of a linear time-invariant system: the state space representation is richer, but we only need the transfer function to characterize the input-output response.

\begin{definition}
    \label{def:ftor}
    Given a DPI~$\tup{ \funsp,\ressp,\impsp,\prov,\req} $,
    we denote by~$\ftor\colon \funsp\toinPos\Aressp$ the monotone map that associates
    to each functionality~\fun  the set of minimal resources necessary to realize~\fun:
    \begin{eqnarray*}
        \ftor\colon \funsp & \toinPos & \Aressp,\\
        \fun& \mapsto & \resMin\{\req(\imp)\mid\left(\imp\in\impsp\right)\,\wedge\,\left(\fun\posleq\prov(\imp)\right)\}.
    \end{eqnarray*}
    If a certain functionality~\fun is infeasible, then~$\ftor(\fun)=\emptyset$.
\end{definition}

\begin{figure}
    \centering
    \includegraphics[scale=0.33]{gmcdp_setup_h}
    \caption{}
    \label{fig:setup_h-1}
\end{figure}

\todo{Say somewhere in the above how we are ordering antichains. Also, is "objective function" above a technical term? Is it clear what is meant?}

\begin{remark}
    The minimal elements of any upper set form an antichain, and one might be tempted to represent upper sets via antichains.
    However, one does not always have a one-to-one correspondence between antichains and upper sets.
    Particularly, the correspondence exists for partial orders satisfying the \emph{descending chain condition}, and does not hold in general.
    In short, the condition is satisfied if in the poset \todo{In the poset or resources or in the poset of upper subsets of the poset or resources? And/or this equivalent?} no \emph{infinite descending chain} exists.
    For instance, consider~$\tup{\mathbb{Z},\leq}$. The chain $-1, -2, -3, \ldots$ is an infinite descending chain.
\end{remark}


\begin{example}
    In the case of the motor design problem, the map~\ftor assigns
    to each pair of~$\tup{\F{\text{speed}},\F{\text{torque}}}$
    the achievable trade-off of \R{cost}, \R{mass}, and other resources~(\cref{fig:motor-trade-offs}).
    The antichains are depicted as continuous curves, but they could also
    be composed of a finite set of points.

\end{example}
\begin{figure}
    \includegraphics[scale=0.33]{gmcdp_motor_tradeoffs}
    \caption{}
    \label{fig:motor-trade-offs}
\end{figure}


By construction, \ftor is monotone (\cref{def:monotone}), which means that
\begin{equation*}
    \fun_{1}\funleq\fun_{2}\quad\Imp\quad\ftor(\fun_{1})\posleq_{\Aressp}\ftor(\fun_{2}),
\end{equation*}.

\todotextjira{299}{I commented out the below because the lemma is not included in public. }
%where~$\posleq_{\Aressp}$ is the order on antichains defined in \cref{lem:antichains-are-poset}.

Monotonicity of~\ftor means that if the functionality~\fun is increased the antichain of resources will go ``up'' in the poset of antichains~$\Aressp$,
and at some point it might reach the top of~$\Aressp$, which is the empty set, meaning that the problem is not feasible.


\begin{definition}
    \label{def:rtof}
    Given a DPI~$\tup{ \funsp,\ressp,\impsp,\prov,\req}$,
    define the map~$\rtof\colon \ressp\toinPos \Afunsp$ that associates
    to each resource~\res the set of maximal functionalities which can be realized with~$\res$:
    \begin{eqnarray*}
        \rtof\colon \ressp & \toinPos & \Afunsp,\\
        \res& \mapsto & \funMax\{\prov(\imp)\mid\left(\imp\in\impsp\right)\,\wedge\,\left(\res\posgeq\req(\imp)\right)\}.
    \end{eqnarray*}
    If a certain resource~\res only leads to infeasible functionalities, then~$\rtof(\res)=\emptyset$.
\end{definition}

\todo{J: Is $\varphi$ a good choice of notation here? It seems better to choose a notation that is more similar to $h$. The distance/clash between latin and greek seems too large to me.}

\todojira{254}{Do example as for ftor.}
