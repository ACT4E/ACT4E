% !TEX root = chapter-standalone.tex

\section{Monotonicity as compositional property}
%\label{sec:Monotone-Co-Design-Problems}

The first main computational result is an invariance result.

\todojira{256}{Restate the theorem with style and wording we use in the book.}
\begin{theorem}
    \label{thm:CDP-monotone}The class of MCDPs is closed with respect to interconnection.
\end{theorem}

\begin{proof}
    \cref{prop:reduction} has shown that any interconnection of design
    problems can be described using the three operators~$\dppar$, $\dpseries$,
    and~$\dploop$.
    Therefore, we just need to check that monotonicity
    in the sense of~\cref{def:DPI} is preserved by each operator
    separately.
    This is done below in~\cref{prop:dppar-monotone,prop:loop-continuous}.
\end{proof}

\todojira{257}{All of these are conseguence of functor and DP, UPos formalization.
    What should we do?}

\begin{proposition}
    \label{prop:dppar-monotone}If~$\adp_{1}$ and~$\adp_{2}$ are
    monotone (\cref{def:DPI}), then also the composition~$\dppar(\adp_{1},\adp_{2})$
    is monotone.
\end{proposition}
\begin{proof}
    We need to refer to the definition of $\dppar$ in \cref{def:parallel}
    and check the conditions in \cref{def:DPI}.
    If~$\funsp_{1},\funsp_{2},\ressp_{1},\ressp_{2}$
    are CPOs, then~$\funsp_{1}\cartprod\funsp_{2}$ and~$\ressp_{1}\cartprod\ressp_{2}$

    are CPOs as well.

    From \cref{def:ftor} and \cref{eq:dppar-exec} we know \ftor can
    be written as
    \begin{align*}
        \ftor\colon \funsp_{1}\cartprod\funsp_{2} & \to\antichains(\ressp_{1}\cartprod\ressp_{2})                     \\
        \tup{ \fun_{1},\fun_{2}}                  & \mapsto\resMin\{\tup{ \req_{1}(\imp_{1}),\req_{2}(\imp_{2})} \mid 
        \left(\tup{ \imp_{1},\imp_{2}} \in\impsp_{1}\cartprod\impsp_{2}\right)
        \wedge\,\left(\tup{ \fun_{1},\fun_{2}} \posleq\tup{ \prov_{1}(\imp_{1}),\prov_{2}(\imp_{2})} \right)\}. 
    \end{align*}
    All terms factorize in the two components, giving:{
    \begin{align*}
        \ftor\!:\!\tup{ \fun_{1},\fun_{2}} \mapsto & \!\!\!\quad\Min_{\ressp_{1}}\{\tup{ \req_{1}(\imp_{1})} \mid\left(\imp\in\impsp_{1}\right)\,\wedge\,\left(\fun_{1}\posleq\prov_{1}(\imp_{1})\right)\}      \\
                                                   & \!\!\!\cartprod\Min_{\ressp_{2}}\{\tup{ \req_{2}(\imp_{2})} \mid\left(\imp\in\impsp_{2}\right)\,\wedge\,\left(\fun_{2}\posleq\prov_{2}(\imp_{2})\right)\}, 
    \end{align*}
    }which reduces to
    \begin{eqnarray}
        \ftor\colon\tup{ \fun_{1},\fun_{2}} & \mapsto & \ftor_{1}(\fun_{1})\acprod\ftor_{2}(\fun_{2}).\label{eq:isproduct} 
    \end{eqnarray}
    The map \ftor is \scottcontinuous iff $\ftor_{1}$ and $\ftor_{2}$
    are~\cite[Lemma II.2.8]{gierz03continuous}.
\end{proof}

\begin{proposition}
    \label{prop:series-monotone}If $\adp_{1}$ and $\adp_{2}$ are
    monotone (\cref{def:DPI}), then also the composition $\dpseries(\adp_{1},\adp_{2})$
    is monotone.
\end{proposition}
\begin{proof}
    From the definition of $\dpseries$ (\cref{def:series-composition}),
    the semantics of the interconnection is captured by this problem:
    \begin{equation}
        \ftor:\fun_{1}\mapsto\begin{cases}
            \with                       & \res_{1},\fun_{2}\in\ressp_{1},\quad\res_{2}\in\ressp_{2}, \\
            \Min_{\posleq_{\ressp_{2}}} & \res_{2},                                                  \\
            \subto                      & \res_{1}\in\ftor_{1}(\fun_{1}),                            \\
                                        & \res_{1}\posleq_{\ressp_{1}}\fun_{2},                      \\
                                        & \res_{2}\in\ftor_{2}(\fun_{2}).                            
        \end{cases}\label{eq:dede}
    \end{equation}
    The situation is described by \cref{fig:series_mono1-2}.
    The point~$\fun_{1}$
    is fixed, and thus~$\ftor(\fun_{1})$ is a fixed antichain in~$\ressp_{1}$.
    For each point $\res_{1}\in\ftor(\fun_{1})$, we can choose a $\fun_{2}\posgeq\res_{1}$.
    For each~$\fun_{2}$, the antichain~$\ftor_{2}(\fun_{2})$ traces
    the solution in $\ressp_{2}$, from which we can choose~$\res_{2}$.

    \captionsideleft{\label{fig:series_mono1-2}}{\includegraphics[scale=0.45]{gmcdp_monotone_proof_mono1}}

    \noindent Because~$\ftor_{2}$ is monotone, $\ftor_{2}(\fun_{2})$
    is minimized when~$\fun_{2}$ is minimized, hence we know that the
    constraint~$\res_{1}\posleq\fun_{2}$ will be tight.
    We can then
    conclude that the objective does not change introducing the constraint~$\res_{1}=\fun_{2}$.
    The problem is reduced to:

    \begin{equation}
        \ftor:\fun_{1}\mapsto\begin{cases}
            \with                       & \fun_{2}\in\ressp_{1},\quad\res_{2}\in\ressp_{2}, \\
            \Min_{\posleq_{\ressp_{2}}} & \res_{2},                                         \\
            \subto                      & \fun_{2}\in\ftor_{1}(\fun_{1}),                   \\
                                        & \res_{2}\in\ftor_{2}(\fun_{2}).                   
        \end{cases}\label{eq:dede-2}
    \end{equation}
    Minimizing $\res_{2}$ with the only constraint being~$\res_{2}\in\ftor_{2}(\fun_{2})$,
    and with~$\ftor_{2}(\fun_{2})$ being an antichain, the solutions
    are all and only~$\ftor_{2}(\fun_{2})$.
    Hence the problem is reduced
    to
    \begin{equation}
        \ftor:\fun_{1}\mapsto\begin{cases}
            \with                       & \fun_{2}\in\ressp_{1},          \\
            \Min_{\posleq_{\ressp_{2}}} & \ftor_{2}(\fun_{2}),            \\
            \subto                      & \fun_{2}\in\ftor_{1}(\fun_{1}). 
        \end{cases}\label{eq:dede-2-1}
    \end{equation}
    The solution is simply
    \begin{equation}
        \ftor:\fun_{1}\mapsto\Min_{\posleq_{\ressp_{2}}}\bigcup_{\fun_{2}\in\ftor_{1}(\fun_{1})}\ftor_{2}(\fun_{2}).\label{eq:ora}
    \end{equation}
    This map is \scottcontinuous because it is the composition of \scottcontinuous
    maps.
\end{proof}

\begin{proposition}
    \label{prop:loop-continuous}If $\adp$ is monotone (\cref{def:DPI}),
    so is~$\dploop(\adp)$.
\end{proposition}
\begin{proof}
    The diagram in \cref{fig:sloop} implies that the map~$\ftor_{\dploop(\adp)}$
    can be described as:
    \begin{align}
        \ftor_{\dploop(\adp)}\colon\funsp_{1} & \to\Aressp,\label{eq:loopproblem}                \\
        \fun_{1}                              & \mapsto\begin{cases}
                                                           \with          & \res,\fun_{2}\in\ressp,                 \\
                                                           \Min_{\resleq} & \res,                                   \\
                                                           \subto         & \res\in\ftor_{\adp}(\fun_{1},\fun_{2}), \\
                                                                          & \res\resleq\fun_{2}.                    
                                                       \end{cases} 
    \end{align}
    Denote by~$\ftor_{\fun_{1}}$ the map~$\ftor_{\adp}$ with the
    first element fixed:
    \begin{equation*}
        \ftor_{\fun_{1}}\colon\fun_{2}\mapsto\ftor_{\adp}(\fun_{1},\fun_{2}).
    \end{equation*}
    Rewrite $\res\in\ftor_{\adp}(\fun_{1},\fun_{2})$ in~\cref{eq:loopproblem}
    as
    \begin{equation}
        \res\in\ftor_{\fun_{1}}(\fun_{2}).\label{eq:h2}
    \end{equation}
    Let~\res be a feasible solution, but not necessarily minimal.
    Because of \cref{lem:antichain-write}, the constraint~\cref{eq:h2} can
    be rewritten as
    \begin{equation}
        \{\res\}=\ftor_{\fun_{1}}(\fun_{2})\cap\upit\res.\label{eq:h3}
    \end{equation}
    Because $\fun_{2}\posgeq\res$, and $\ftor_{\fun_{1}}$ is \scottcontinuous,
    it follows that~$\ftor_{\fun_{1}}(\fun_{2})\posgeq_{\Aressp}\ftor_{\fun_{1}}(\res)$.
    Therefore, by \cref{lem:antichain_inter}, we have
    \begin{equation}
        \{\res\}\posgeq_{\Aressp}\ftor_{\fun_{1}}(\res)\cap\upit\res.\label{eq:fea}
    \end{equation}
    This is a recursive condition that all feasible~\res must satisfy.

    Let $\RR\in\Aressp$ be an antichain of feasible resources, and
    let~\res be a generic element of~\ressp.
    Tautologically, rewrite~$\RR$
    as the minimal elements of the union of the singletons containing
    its elements:
    \begin{equation}
        \RR=\Min_{\resleq}\bigcup_{\res\in\RR}\ \{\res\}.\label{eq:condition3}
    \end{equation}
    Substituting~\cref{eq:fea} in~\cref{eq:condition3} we obtain (cf
    \cref{lem:antichain_union})
    \begin{equation}
        \RR\posgeq_{\Aressp}\Min_{\resleq}\bigcup_{\res\in\RR}\ftor_{\fun_{1}}(\res)\ \cap\ \upit\res.\label{eq:recursive}
    \end{equation}
    %
    {[}Converse: It is also true that if an antichain~$\RR$ satisfies~\cref{eq:recursive}
    then all~$\res\in\RR$ are feasible.
    The constraint~\cref{eq:recursive}
    means that for any~$\res_{0}\in\RR$ on the left side, we can find
    a~$\res_{1}$ in the right side so that~$\res_{0}\posgeq_{\ressp}\res_{1}$.
    The point~$\res_{1}$ needs to belong to one of the sets of which
    we take the union; say that it comes from $\res_{2}\in\RR$, so
    that $\res_{1}\in\ftor_{\fun_{1}}(\res_{2})\ \cap\ \upit\res_{2}$.
    Summarizing:
    {
    \begin{equation}
        \forall\res_{0}\in\RR:\ \exists\res_{1}\colon\ (\res_{0}\posgeq_{\ressp}\res_{1})\ \wedge\ (\exists\res_{2}\in\RR\colon\ \res_{1}\in\ftor_{\fun_{1}}(\res_{2})\ \cap\ \upit\res_{2}).\label{eq:conc}
    \end{equation}
    }Because~$\res_{1}\in\ftor_{\fun_{1}}(\res_{2})\,\cap\,\upit\res_{2}$,
    we can conclude that~$\res_{1}\in\upit\res_{2}$, and therefore~$\res_{1}\posgeq_{\ressp}\res_{2}$,
    which together with~$\res_{0}\posgeq_{\ressp}\res_{1}$, implies~$\res_{0}\posgeq_{\ressp}\res_{2}$.
    We have concluded that there exist two points~$\res_{0},\res_{2}$
    in the antichain~$\RR$ such that~$\res_{0}\posgeq_{\ressp}\res_{2}$;
    therefore, they are the same point:~$\res_{0}=\res_{2}$.
    Because~$\res_{0}\posgeq_{\ressp}\res_{1}\posgeq_{\ressp}\res_{2}$,
    we also conclude that~$\res_{1}$ is the same point as well.
    We can
    rewrite~\cref{eq:conc} by using~$\res_{0}$ in place of~$\res_{1}$
    and~$\res_{2}$ to obtain~$\forall\res_{0}\in\RR:\res_{0}\in\ftor_{\fun_{1}}(\res_{0})$,
    which means that~$\res_{0}$ is a feasible resource.{]}

    We have concluded that all antichains of feasible resources~$\RR$
    satisfy~\cref{eq:recursive}, and conversely, if an antichain~$\RR$
    satisfies~\cref{eq:recursive}, then it is an antichain of feasible
    resources.

    Equation!\cref{eq:recursive} is a recursive constraint for~$\RR$,
    of the kind
    \begin{equation*}
        \Phi_{\fun_{1}}(\RR)\posleq_{\Aressp}\RR,
    \end{equation*}
    with the map~$\Phi_{\fun_{1}}$ defined by
    \begin{eqnarray}
        \Phi_{\fun_{1}}:\Aressp & \to     & \Aressp,\label{eq:bigphi}                                                           \\
        \RR                     & \mapsto & \Min_{\resleq}\bigcup_{\res\in\RR}\ftor_{\fun_{1}}(\res)\ \cap\ \upit\res.\nonumber 
    \end{eqnarray}
    If we want the \emph{minimal} resources, we are looking for the \emph{least}
    antichain:
    \begin{equation*}
        \min_{\posleq_{\Aressp}}\{\,\RR\in\Aressp\colon\ \Phi_{\fun_{1}}(\RR)\posleq_{\Aressp}\RR\,\},
    \end{equation*}
    which is equal to the \emph{least fixed point }of~$\Phi_{\fun_{1}}$.
    Therefore, the map $\ftor_{\dploop(\adp)}$ can be written as
    \begin{equation}
        \ftor_{\dploop(\adp)}\colon\fun_{1}\mapsto\lfp(\Phi_{\fun_{1}}).\label{eq:loop_fixpoint}
    \end{equation}
    \cref{lem:dagger} shows that $\lfp(\Phi_{\fun_{1}})$ is \scottcontinuous
    in~$\fun_{1}$.
\end{proof}

\begin{lemma}
    \label{lem:antichain-write}Let~$A$ be an antichain in~$\posA$.
    Then
    \begin{equation*}
        a\in A\qquad\equiv\qquad\{a\}=A\,\cap\upit a.
    \end{equation*}
\end{lemma}

\begin{lemma}
    \label{lem:antichain_inter}For~$A,B\in\antichains\posA$, and~$S\subseteq P$,
    $A\posleq_{\Aressp}B$ implies $A\cap S\posleq_{\Aressp}B\cap S$.
\end{lemma}

\begin{lemma}
    \label{lem:antichain_union}For~$A,B,C,D\in\antichains\posA$, $A\posleq_{\Aressp}C$
    and $B\posleq_{\Aressp}D$ implies $A\cup B\posleq_{\Aressp}C\cup D.$
\end{lemma}

\begin{lemma}
    \label{lem:dagger}Let~$f\colon\posA\cartprod\posB\to\posB$
    be \scottcontinuous.
    For each~$x\in\posA$, define $f_{x}:y\mapsto f(x,y).$

    Then $f^{\dagger}:x\mapsto\lfp(f_{x})$ is \scottcontinuous.
\end{lemma}
\begin{proof}
    Davey and Priestly~\cite{davey02} leave this as Exercise~8.26.
    A proof is found in Gierz~\etal~\cite[Exercise II-2.29]{gierz03continuous}.
\end{proof}

