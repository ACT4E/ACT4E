% !TEX root = chapter-standalone.tex


\section{Complexity of the solution}
\linkvideo{spring2021-functorial-comp-b:solving-queries:complexity} % Complexity

\subsection{Complexity of fixed point iteration}

Consider first the case of an MCDP that can be described as~$\dprob=\dploop(\dprob_{0})$,
where~$\dprob_{0}$ is an MCDP that is described only using the~$\dpseries$
and~$\dppar$ operators. Suppose that~$\dprob_{0}$ has resource
space~\ressp. Then evaluating~\ftor for~$\dprob$ is equivalent
to computing a least fixed point iteration on the space of antichains~$\Aressp$.
This allows to give worst-case bounds on the number of iterations.

\begin{proposition}
  \label{prop:complexity}
  Suppose that~$\dprob=\dploop(\dprob_{0})$ and~$\dprob_{0}$ has resource space~$\ressp_{0}$ and evaluating~$\ftor_{0}$ takes at most~$c$ computation. Then we can obtain the following bounds for the algorithm's resources usage:

  \smallskip{}
  \begin{tabular}{cc}
    memory & $O(\posetwidth(\ressp_{0}))$\tabularnewline
    number of steps & $O(\posetheight(\antichains\ressp_{0}))$\tabularnewline
    total computation & $O(\posetwidth(\ressp_{0})\times\posetheight(\antichains\ressp_{0})\times c)$\tabularnewline
  \end{tabular}

\end{proposition}
\begin{proof}
  The memory utilization is bounded by~$\posetwidth(\ressp_{0})$, because the state is an antichain, and~$\posetwidth(\ressp_{0})$ is the size of the largest antichain. The iteration happens in the space~$\antichains\ressp_{0}$, and we are constructing an ascending chain, so it can take at most~$\posetheight(\antichains\ressp_{0})$ steps to converge. Finally, in the worst case the map~$\ftor_{0}$ needs to be evaluated once for each element of the antichain for each step.
\end{proof}
These worst case bounds are strict.
\begin{example}
  Consider solving $\dprob=\dploop(\dprob_{0})$ with $\dprob_{0}$
  defined by $\ftor_{0}\colon\tup{ \tup{ } ,x} \mapsto x+1$
  with $x\in\overline{\natnumbers}$. Then the least fixed point equation
  is equivalent to solving~$\min\{x\colon\Psi(x)\leq x\}$ with~$\Psi:x\mapsto x+1$.
  The iteration~$R_{k+1}=\Psi(R_{k})$ converges to~$\postop$ in~$\posetheight(\overline{\natnumbers})=\aleph_{0}$
  steps.
\end{example}

\begin{remark}
  Making more precise claims requires additional more restrictive assumptions
  on the spaces involved. For example, without adding a metric on~\ressp,
  it is not possible to obtain properties such as linear or quadratic
  convergence.
\end{remark}

\begin{remark}[Invariance to re-parameterization]
  All the results given in this paper are invariant to any order-preserving
  re-parameterization of all the variables involved.
\end{remark}

\subsection{Relating complexity to the graph properties}

\cref{prop:complexity} above assumes that the MCDP is already in the form $\dprob=\dploop(\dprob_{0})$, and relates the complexity to
the poset~$\ressp_{0}$.
Here we relate the results to the graph structure of an MCDP.

Take an MCDP~$\dprob=\tup{ \funsp,\ressp,\tup{ \cdpiN,\mathcal{E}} } $.
To put~$\dprob$ in the form $\dprob=\dploop(\dprob_{0})$ according
to the procedure in~\cref{sec:Decomposition}, we need to find an arc
feedback set (AFS) of the graph~$\tup{ \cdpiN,\mathcal{E}} $.
Given a AFS~$F\subset\mathcal{E}$, then the resource space~\ressp for a~$\dprob_{0}$ such that~$\dprob=\dploop(\dprob_{0})$ is the product of the resources spaces along the edges: $\ressp_{0}=\prod_{e\in F}\ressp_{e}.$

Now that we have a relation between the AFS and the complexity of the iteration, it is natural to ask what is the optimal choice of AFS\textemdash which, so far, was left as an arbitrary choice.
The AFS should be chosen as to minimize one of the performance measures in~\cref{prop:complexity}.

Of the three performance measures in~\cref{prop:complexity}, the most fundamental appears to be~$\posetwidth(\ressp_{0})$, because that is also an upper bound on the number of distinct minimal solutions.
Hence we can call it ``design complexity'' of the MCDP.
\begin{definition}
  \label{def:design-complexity}Given a graph~$\tup{ \cdpiN,\mathcal{E}} $
  and a labeling of each edge~$e\in\mathcal{E}$ with a poset~$\ressp_{e}$,
  the \emph{design complexity~}$\text{DC}(\tup{ \cdpiN,\mathcal{E}} )$
  is defined as
  \begin{equation}
    \text{DC}(\tup{ \cdpiN,\mathcal{E}} )=\min_{F\text{ is an AFS}}\posetwidth(\prod_{e\in F}\ressp_{e}).\label{eq:complexity}
  \end{equation}
\end{definition}
In general, width and height of posets are not additive with respect
to products; therefore, this problem does not reduce to any of the
known variants of the minimum arc feedback set problem, in which
each edge has a weight and the goal is to minimize the sum of the
weights.

\subsection{Considering relations with infinite cardinality}

This analysis shows the limitations of the simple solution presented
so far: it is easy to produce examples for which $\posetwidth(\ressp_{0})$
is infinite, so that one needs to represent a continuum of solutions.

\begin{example}
  Suppose that the platform to be designed must travel \F{a distance~$d$
      {[}m{]}}, and we need to choose the \R{endurance~$T$~{[}s{]}}
  and \R{the velocity~$v$~{[}m/s{]}}. The relation among the quantities
  is~${\colF d}\leq{\colR T\,v}.$ This is a design problem described
  by the map
  \begin{eqnarray*}
    \ftor:{\colF\Rcomp} & \rightarrow & {\colR\antichains\Rcomp\times\Rcomp,}\\
    {\colF d} & \mapsto & \{\langle{\colR T},{\colR v}\rangle\in{\colR\Rcomp\times\Rcomp}:\,{\colF d}={\colR T}\,{\colR v}\}.
  \end{eqnarray*}
  For each value of ${\colF d}$, there is a continuum of solutions.
\end{example}
One approach to solving this problem would be to discretize the functionality~\funsp
and the resources~\ressp by sampling and/or coarsening. However,
sampling and coarsening makes it hard to maintain completeness and
consistency.

One effective approach, outside of the scope of this paper, that allows
to use finite computation is to \emph{approximate the design problem}
\emph{itself}, rather than the spaces~$\funsp,\ressp$, which are
left as possibly infinite. The basic idea is that an infinite antichain
can be bounded from above and above by two antichains that have a
finite number of points. This idea leads to an algorithm that, given
a prescribed computation budget, can compute an inner and outer approximation
to the solution antichain~\cite{mcdp_icra_uncertainty_arxiv}.

