% !TEX root = chapter-standalone.tex

\section{Solution of MCDPs}
\label{sec:Solution-of-Monotone}
\linkvideo{spring2021-functorial-comp-b:solving-queries:solving-series} % Series composition
\linkvideo{spring2021-functorial-comp-b:solving-queries:solving-parallel} % Parallel composition
\linkvideo{spring2021-functorial-comp-b:solving-queries:solving-loop} % Loop composition
\linkvideo{spring2021-functorial-comp-b:solving-queries:or-and-and} % Join and Meet in DP

The second main result is that the map \ftor of a MCDP has an explicit expression in terms of the maps~\ftor of the subproblems.

\begin{theorem}
    \label{thm:CDP-solvig}
    The map~\ftor for an MCDP has an explicit expression in terms of the maps \ftor of its subproblems, defined recursively using the rules in \cref{tab:correspondence}.
\end{theorem}

\begin{table}[h]
    \caption{Recursive expressions for \ftor.}
    \label{tab:correspondence}
    \centering{}\setlength\extrarowheight{5pt}\normalsize
    \begin{tabular}{ccc}
        series     & $\adp=\dpseries(\adp_{1},\adp_{2})$ & $\ftor=\ftor_{1}\opseries\ftor_{2}$\tabularnewline
        parallel   & $\adp=\dppar(\adp_{1},\adp_{2})$    & $\ftor=\ftor_{1}\oppar\ftor_{2}$\tabularnewline
        feedback   & $\adp=\dploop(\adp_{1})$            & $\ftor=\ftor_{1}^{\oploop}$\tabularnewline
        co-product & $\adp=\adp_{1}\sqcup\adp_{2}$       & $\ftor=\ftor_{1}\opcoprod\ftor_{2}$\tabularnewline
    \end{tabular}
\end{table}

\begin{proof}
    These expressions were derived in the proofs of~\cref{prop:dppar-monotone,prop:loop-continuous}.
    The operators $\opseries,\oppar,\oploop,\opcoprod$ are defined in \cref{def:opseries1}.
\end{proof}
\begin{definition}
    [Series operator~$\opseries$]
    \label{def:opseries1}
    For two maps~$\ftor_{1}\colon\funsp_{1}\rightarrow\Aressp_{1}$
    and~$\ftor_{2}\colon\funsp_{2}\rightarrow\Aressp_{2}$, if~$\ressp_{1}=\funsp_{2}$
    , define
    \begin{align*}
        {\displaystyle \ftor_{1}\opseries\ftor_{2}\colon\funsp_{1}}
                 & \rightarrow\Aressp_{2}, \\
        \fun_{1} & \mapsto\Min_{\posleqof{\ressp_{2}}}\bigsetunion_{s\setin\ftor_{1}(\fun)}\ftor_{2}(s).
    \end{align*}
\end{definition}

\begin{forslides}
    \begin{equation}
        \label{eq:h_example_berk_series}
        \begin{aligned}
            {\displaystyle \ftor_{\stylefunctors{\mathbf{f}}\mthen \stylefunctors{\mathbf{g}}}\colon\funspa}
                                          & \mto\antichains \resspc \\
            \ensuremath{{\colF a}}\xspace & \mapsto\Min_{\posleqof{\resspc}}\bigsetunion_{s\setin\ftor_{\stylefunctors{\mathbf{f}}}(\ensuremath{{\colF a}}\xspace)}\ftor_{\stylefunctors{\mathbf{g}}}(s).
        \end{aligned}
    \end{equation}

    \begin{equation}
        \label{eq:h_example_berk_series_u}
        \begin{aligned}
            {\displaystyle \ftor_{\stylefunctors{\mathbf{f}}\mthen \stylefunctors{\mathbf{g}}}\colon\funspa}
                                          & \mto\uppersets \resspc \\
            \ensuremath{{\colF a}}\xspace & \mapsto\bigsetunion_{s\setin\ftor_{\stylefunctors{\mathbf{f}}}(\ensuremath{{\colF a}}\xspace)}\ftor_{\stylefunctors{\mathbf{g}}}(s).
        \end{aligned}
    \end{equation}

    \begin{equation}
        \label{eq:h_example_berk_par}
        \begin{aligned}
            \ftor_{\stylefunctors{\mathbf{f}}}\mtimescat\ftor_{\stylefunctors{\mathbf{g}}}:(\funspa \cartprod\funspc) & \mto\antichains(\resspb \cartprod\resspd), \\
            \tup{\ensuremath{{\colF a}}\xspace,\ensuremath{{\colF c}}\xspace}                                         & \mapsto\ftor_{\stylefunctors{\mathbf{f}}}(\ensuremath{{\colF a}}\xspace)\acprod\ftor_{\stylefunctors{\mathbf{g}}}(\ensuremath{{\colF c}}\xspace),\nonumber
        \end{aligned}
    \end{equation}

    \begin{equation}
        \label{eq:h_example_berk_par_u}
        \begin{aligned}
            \ftor_{\stylefunctors{\mathbf{f}}}\mtimescat\ftor_{\stylefunctors{\mathbf{g}}}:(\funspa \cartprod\funspc) & \mto\uppersets(\resspb \cartprod\resspd), \\
            \tup{\ensuremath{{\colF a}}\xspace,\ensuremath{{\colF c}}\xspace}                                         & \mapsto\ftor_{\stylefunctors{\mathbf{f}}}(\ensuremath{{\colF a}}\xspace)\acprod\ftor_{\stylefunctors{\mathbf{g}}}(\ensuremath{{\colF c}}\xspace),\nonumber
        \end{aligned}
    \end{equation}

    \begin{equation}
        \label{eq:h_example_berk_cop}
        \begin{aligned}
            \ftor_{\stylefunctors{\mathbf{f}}}\vee\ftor_{\stylefunctors{\mathbf{g}}}:\funspa & \mto\antichains \resspb, \\
            \ensuremath{{\colF a}}\xspace                                                    & \mapsto\Min_{\posleqof{\resspb}}\left(\ftor_{\stylefunctors{\mathbf{f}}}(\ensuremath{{\colF a}}\xspace )\setunion\ftor_{\stylefunctors{\mathbf{g}}}(\ensuremath{{\colF a}}\xspace )\right).
        \end{aligned}
    \end{equation}

    \begin{equation}
        \label{eq:h_example_berk_cop_u}
        \begin{aligned}
            \ftor_{\stylefunctors{\mathbf{f}}}\vee\ftor_{\stylefunctors{\mathbf{g}}}:\funspa & \mto\uppersets \resspb, \\
            \ensuremath{{\colF a}}\xspace                                                    & \mapsto \ftor_{\stylefunctors{\mathbf{f}}}(\ensuremath{{\colF a}}\xspace )\setunion\ftor_{\stylefunctors{\mathbf{g}}}(\ensuremath{{\colF a}}\xspace ).
        \end{aligned}
    \end{equation}

    \begin{equation}
        \label{eq:h_example_berk_prod}
        \begin{aligned}
            \ftor_{\stylefunctors{\mathbf{f}}}\wedge\ftor_{\stylefunctors{\mathbf{g}}}:\funspa & \mto\antichains \resspb, \\
            \ensuremath{{\colF a}}\xspace                                                      & \mapsto\Min_{\posleqof{\resspb}}\left(\ftor_{\stylefunctors{\mathbf{f}}}(\ensuremath{{\colF a}}\xspace )\setintersection\ftor_{\stylefunctors{\mathbf{g}}}(\ensuremath{{\colF a}}\xspace )\right).
        \end{aligned}
    \end{equation}

    \begin{equation}
        \label{eq:h_example_berk_prod_u}
        \begin{aligned}
            \ftor_{\stylefunctors{\mathbf{f}}}\wedge\ftor_{\stylefunctors{\mathbf{g}}}:\funspa & \mto\uppersets \resspb, \\
            \ensuremath{{\colF a}}\xspace                                                      & \mapsto\ftor_{\stylefunctors{\mathbf{f}}}(\ensuremath{{\colF a}}\xspace )\setintersection\ftor_{\stylefunctors{\mathbf{g}}}(\ensuremath{{\colF a}}\xspace ).
        \end{aligned}
    \end{equation}

    \includesag{example_poset_fruits}

    \includesag{poset_reals}

    \includesag{poset_nats}

    \begin{equation}
        \label{eq:ressp1}
        \ressp_\R{1}
    \end{equation}

    \begin{equation}
        \label{eq:ressp2}
        \ressp_\R{2}
    \end{equation}

    \begin{equation}
        \label{eq:posleqDP}
        \posleq_\DP
    \end{equation}

    \begin{equation}
        \label{eq:ressp12}
        \ressp_\R{1}\times \ressp_\R{2}
    \end{equation}
\end{forslides}

\begin{definition}
    [Parallel operator $\oppar$] \label{def:oppar}
    For two maps $\ftor_{1}\colon\funsp_{1}\rightarrow\Aressp_{1}$ and $\ftor_{2}\colon\funsp_{2}\rightarrow\Aressp_{2}$, define
    \begin{align}
        \ftor_{1}\oppar\ftor_{2}:(\funsp_{1}\cartprod\funsp_{2}) & \rightarrow\antichains(\ressp_{1}\cartprod\ressp_{2}),\label{eq:oppar} \\
        \tup{ \fun_{1},\fun_{2}}                                 & \mapsto\ftor_{1}(\fun_{1})\acprod\ftor_{2}(\fun_{2}),\nonumber
    \end{align}
    where $\acprod$ is the product of two antichains.
\end{definition}

\begin{definition}
    [Feedback operator $\oploop$]
    \label{def:oploop1}
    For $\ftor:\funsp_{1}\cartprod\ressp\rightarrow\Aressp$,
    define
    \begin{align}
        \ftor^{\oploop}:\funsp_{1} & \rightarrow\Aressp,\nonumber \\
        \fun_{1}                   & \mapsto\lfp\left(\Psi_{\fun_{1}}^{\ftor}\right),
    \end{align}
    where~$\Psi_{\fun_{1}}^{\ftor}$ is defined as
    \begin{align}
        \Psi_{\fun_{1}}^{\ftor}:\Aressp & \rightarrow\Aressp,\nonumber \\
        {\colR R}                       & \mapsto\Min_{\posleqof{\ressp}}\bigsetunion_{\res\in{\colR R}}\ftor(\fun_{1},\res)\ \setintersection\upit\res.
        \label{eq:phi}
    \end{align}
\end{definition}

\begin{definition}
    [Coproduct operator $\opcoprod$]
    \label{def:opcoprod}
    For $\ftor_{1},\ftor_{2}:\funsp\rightarrow\Aressp$,
    define
    \begin{align*}
        \ftor_{1}\opcoprod\ftor_{2}:\funsp & \rightarrow\Aressp, \\
        \fun                               & \mapsto\Min_{\posleqof{\ressp}}\left(\ftor_{1}(\fun)\setunion\ftor_{2}(\fun)\right).
    \end{align*}
\end{definition}
