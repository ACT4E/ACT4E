
\subsection{Expressivity of MCDPs}
The results are significant because MCDPs induce a rich family of optimization problems.

We are not assuming, let alone strong properties like convexity, even weaker properties like differentiability or continuity of the constraints.
In fact, we are not even assuming that functionality and resources are continuous spaces; they could be arbitrary discrete posets.
Indeed, in that case, completeness and \scottcontinuity are trivially satisfied.

\todotext{Improve writing style and clarity of above paragraph.}

Moreover, even assuming topological continuity of all spaces and maps considered, MCDPs are strongly not convex.
What makes them nonconvex is the possibility of introducing feedback interconnections.
To show this, we will give an example of a 1-dimensional problem with a continuous~\ftor for which the feasible set is disconnected (and in particular non-convex).

\begin{marginfigure}
    \centering
    %\subfloat[\label{fig:Simple-DP}]{\centering{}\includegraphics[scale=0.43]{gmcdptro_nonconvex1b}}
    \subfloat[\label{fig:Simple-DP}]{\centering{}\includesag{simple-dp-nonconvex}} \\
    \subfloat[\label{fig:nonconvex3}]{\centering{}\includegraphics[scale=0.43]{gmcdptro_nonconvex3}}
    \caption{One feedback connection and a topologically continuous~\ftor are sufficient to induce a disconnected feasible set.}
    \label{fig:ceil-1}
\end{marginfigure}

\begin{example}[Non-convexity]
    \label{exa:one}
    Consider the CDPI in \cref{fig:Simple-DP}.
    The \uline{m}inimal resources~$M\setsubseteq\uppersets \ressp$ are the objectives of this optimization problem:
    \begin{equation*}
        M\definedas\begin{cases}
            \with          & \fun,\res\setin\funsp=\ressp, \\
            \Min_{\posleq} & \res,                         \\
                           & \res\setin\ftor(\fun),        \\
                           & \res\posleq\fun.
        \end{cases}
    \end{equation*}
    The \uline{fea}sible set~$\feasibleset{}\setsubseteq\funsp\cartprod\ressp$ is the set of functionality and resources that satisfy the constraints~$\res\setin\ftor(\fun)$ and~$\res\posleq\fun$:

    \begin{equation}
        \feasibleset{}=\left\{ \tup{\fun,\res}\setin\funsp\cartprod\ressp\colon (\res\setin\ftor(\fun))\wedge(\res\posleq\fun)\right\} .
        \label{eq:feasible}
    \end{equation}
    The \uline{p}rojection~$\feasiblesetprojfun$ of~$\feasibleset{}$ to the functionality space is:
    \begin{equation*}
        \feasiblesetprojfun=\left\{ \fun\mid\tup{\fun,\res} \setin\feasibleset{}\right\}.
    \end{equation*}
    In the scalar case ($\funsp=\ressp=\tupp{\nonNegRealsComp,\leq}$), the map~$\ftor\colon\funsp\toinPos\uppersets \ressp$ is simply a map~$\ftor\colon\F{\nonNegRealsComp}\to \uppersets\R{\nonNegRealsComp}$.
    The set~$\feasiblesetprojfun$ of feasible functionality is described by
    \begin{equation}
        \feasiblesetprojfun=\{\fun\setin\nonNegRealsComp\colon \ftor(\fun)\leq\fun\}.
        \label{eq:Pfeasible}
    \end{equation}
    \cref{fig:nonconvex3} shows an example of a continuous map~\ftor that gives a disconnected feasible set~$\feasiblesetprojfun$.
    Moreover,~$\feasiblesetprojfun$ is disconnected under any order-preserving nonlinear re-parametrization.

\end{example}
