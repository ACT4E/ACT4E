% !TEX root = chapter-standalone.tex

\section{Domain theory and fixed points}
\label{sec:Monotonicity-and-fixed}

In this section we recall some fundamentals of domain theory.
It is used in computer science for defining denotational semantics~(see \eg~\cite{manes86}).
It is used in embedded systems for defining the semantics of models of computation~(see, \eg~\cite{lee10}).
What we need from domain theory is the least necessary to define \emph{least fixed points} and to use Kleene's theorem.

Domain theory builds on order theory by defining ``directed'' and ``complete'' partial orders.
These attributes plays the same role as compactness in analysis: they will be used to make sure that certain sequences can converge to a fixed point.

\subsection{Directed and complete partial orders}

\begin{definition}[Directed set]
    \label{def:directed-set}
    In a poset $\posA = \tupp{\posAset, \posAleq}$, we say that a set~$\subA\subseteq\posAset$ is \emph{directed} if each pair of elements in~$\subA$ has an upper bound: for all~$\posela,\poselb \setin \subA$, there exists~$\poselc\setin \subA$ such that~$\posela\posleq \poselc$ and~$\poselb\posleq \poselc$.
\end{definition}

\todotext{@Andrea: verify condition: must }

\begin{definition}[Completeness]
    \label{def:cpo}
    A poset is a \emph{directed complete partial order} (\DCPO) if each of its directed subsets has a supremum (least of upper bounds).
    It is a \emph{complete partial order} (\CPO) if it also has a bottom.
\end{definition}

\begin{example}
    [Completion of $\nonNegReals$ to~$\nonNegRealsComp$]
    \label{exa:Rcomp}
    The poset $\tupp{\reals, \Rleq}$ is not a \CPO, because it lacks a bottom.

    The nonnegative reals~$\nonNegReals=\{x\setin\reals \colon x\geq0\}$ have a bottom~$\posbot=0$, however, they are not a \DCPO because some of their directed subsets do not have an upper bound.
    For example, take~$\nonNegReals$, which is a subset of~$\nonNegReals$.
    Then~$\nonNegReals$ is directed, because for each~$a,b\setin\nonNegReals$, there exists~$c=\max\{a,b\}\setin\nonNegReals$ for which~$a\Rleq c$ and~$b\Rleq c$.

    One way to make~$\tupp{ \nonNegReals,\Rleq} $ a \CPO is by adding an artificial top element~$\postop$ that we think as ``a point at infinitely''.
    We can define then the completion
    \begin{equation}
        \nonNegRealsComp\definedas\nonNegReals\setunion\{\postop\},
    \end{equation} and extending the partial order~$\Rleq$ so that~$a\Rleq\postop$ for all~$a\setin\nonNegReals$.
\end{example}

\subsection{\scottcontinuity}

\scottcontinuity is a property of maps on DCPOs that is slightly stronger than monotonicity.

\begin{definition}[\scottcontinuity]
    \label{def:scott}
    A map~$\mora\colon\posA\toinPos\posB$ between DCPOs is \emph{\scottcontinuous{}} iff for each directed subset~$D\subseteq\posA$, the image~$\mora(D)$ is directed, and
    \begin{equation}\label{eq:scott-continuity-cond}
        \mora(\Sup D)=\Sup \mora(D).
    \end{equation}
\end{definition}
\begin{lemma}
    \scottcontinuity implies monotonicity.
\end{lemma}
\begin{proof}
    Consider a map~$\mora\colon\posA\toinPos\posB$ that is \scottcontinuous.
    Take two elements $\posela, \poselb \setin \posA$ such that $\posela\posleq\poselb$.
    The set $ S =  \makeset{\posela, \poselb}$ is directed.
    It follows that $\mora(S) = \makeset{\mora(\posela), \mora(\poselb)}$ is directed, which means that at least one
    of $\mora(\posela) \posleq \mora(\poselb)$ and $\mora(\poselb) \posleq \mora(\posela)$ are true.
    From \cref{eq:scott-continuity-cond}, we know that
    \begin{equation}
        \mora(\Sup S) = \mora(\poselb) = \Sup \makeset{\mora(\posela), \mora(\poselb)},
    \end{equation}
    which implies that $\mora(\posela) \posleq \mora(\poselb)$.
    Therefore, $f$ is monotone.
\end{proof}

\begin{marginfigure}
    \includegraphics[scale=0.53]{gmcdptro_ceil}
    \caption{The ceiling function is \scottcontinuous.}
    \label{fig:gmcdp_ceil}
    \todographics{@Gioele: redo this diagram}
\end{marginfigure}

\begin{remark}
    \scottcontinuity does not imply ``topological'' continuity.
    A map from the CPO $\tupp{\Rcomp,\Rleq}$ to itself is \scottcontinuous iff it is nondecreasing and left-continuous.
    For example, the ceiling function $x\mapsto\left\lceil x\right\rceil $~ is \scottcontinuous (\cref{fig:gmcdp_ceil}).
\end{remark}

\subsection{Least fixed points}

\begin{definition}[Least fixed points]
    \label{def:least-fixed}
    A \emph{fixed point} of $\mora\colon\posA\to\posA$ is a point~$x$ such that $\mora(x)=x$.

    A \emph{least fixed point} of~$\mora\colon\posA\to\posA$ is the minimum (if it exists) of the set of fixed points of~$f$:
    \begin{equation}
        \label{eq:lfp-one}
        \lfp(f)\,\,\definedas\,\,\min_{\posleq}\,\{x\setin\posA\colon \mora(x)=x\}.
    \end{equation}
    \todotextjira{423}{@Andrea: So if multiple minima exist, we say there is no least fixed point?
        If yes, perhaps say this explicitly, for clarity?
    }

    The equality in \cref{eq:lfp-one} can be relaxed to ``$\posleq$''.
\end{definition}

The least fixed point need not exist.
Monotonicity of the map~$f$ plus completeness is sufficient to ensure existence.

\begin{lemma}[{\cite[CPO Fixpoint Theorem II, 8.22]{davey02}}]
    \label{lem:CPO-fix-point-2}
    If~$\posA$ is a \CPO and~$\mora\colon\posA\toinPos\posA$
    is monotone, then $\lfp(f)$ exists.
\end{lemma}

With the additional assumption of \scottcontinuity, Kleene's algorithm is a systematic procedure to find the least fixed point.

\begin{lemma}[{Kleene's fixed-point theorem~\cite[CPO fixpoint theorem I, 8.15]{davey02}}]
    \label{lem:kleene-1}
    Assume $\posA$ is a \CPO, and~$\mora\colon\posA\toinPos\posA$ is \scottcontinuous.
    Then the least fixed point of~$f$ is the supremum of the Kleene ascent chain
    \begin{equation*}
        \posbot\posleq \mora(\posbot)\posleq \mora(\mora(\posbot))\posleq\cdots\posleq f^{(n)}(\posbot)\leq\cdots.
    \end{equation*}
\end{lemma}
