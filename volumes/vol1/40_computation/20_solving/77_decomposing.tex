% !TEX root = chapter-standalone.tex

\section{Decomposition of CDPs}
\label{sec:Decomposition}

\linkvideo{spring2021-functorial-comp-b:solving-queries:multi-loop} % Canonical form

\label{sec:Decomposing2}
This section shows how to describe an arbitrary interconnection of \SY{design problems} using only three composition operators.
More precisely, for each CDPI with a set of atoms~$\cdpiN$, there is an equivalent one that is built from $\dpseries$/$\dppar$/$\dploop$ applied to the set of atoms~$\cdpiN$ plus some extra ``plumbing'' (identities, multiplexers).

% \subsection{Equivalence}

% The definition of equivalence below ensures that two equivalent DPIs have the same map from functionality to resources, while one of the DPIs may have a slightly larger implementation space.
% \begin{definition}
%     Two DPIs $\tup{\funsp,\ressp,\impsp_{1},\prov_{1},\req_{1}} $ and $\langle\funsp,\ressp,$ $\impsp_{2},\prov_{2},\req_{2}\rangle$ are \emph{equivalent} if there exists a map~$\varphi:\impsp_{2}\rightarrow\impsp_{1}$ such that~$\prov_{2}=\prov_{1}\circ\varphi$ and~$\req_{2}=$ $\req_{1}\circ\varphi$.
% \end{definition}

% \subsection{Plumbing}

% We also need to define ``trivial DPIs'', which serve as ``plumbing''.
% These can be built by taking a map $f:\funsp\rightarrow\ressp$ and lifting it to the definition of a DPI.
% The implementation space of a trivial DPI is a copy of the functionality space and there is a 1-to-1 correspondence between functionality and implementation.
% \begin{definition}[Trivial DPIs]
%     Given a map~$f:\funsp\rightarrow\ressp$, we can lift it to define a trivial DPI $\triv(f)=\tup{\funsp,\ressp,\funsp,\mapid_{\funsp},f} $, where~$\mapid_{\funsp}$ is the identity on~\funsp.
% \end{definition}
\begin{proposition}
    \label{prop:reduction}
    Given a CDPI $\tup{\funsp,\ressp, \tup{\cdpiN,\mathcal{E}} } $, we can find an equivalent CDPI obtained by applying the operators $\dppar/\dpseries/\dploop$ to a set of atoms~$\cdpiN'$ that contains~$\cdpiN$ plus a set of trivial DPIs.
    Furthermore, one instance of~$\dploop$ is sufficient.
\end{proposition}
\begin{proof}
    We show this constructively.
    We will temporarily remove all cycles from the graph, to be reattached later.
    To do this, find an \emph{arc feedback set} (AFS) $F\setsubseteq\mathcal{E}$.
    An AFS is a set of edges that, when removed, remove all cycles from the graph (see~\cite{golovach15incremental}).
    For example, the CDPI represented in~\cref{fig:cdpi_comp1} has a minimal AFS that contains the edge~$\text{c}\rightarrow\text{a}$~(\cref{fig:cdpi_comp2}).

    Remove the~AFS~$F$ from~$\mathcal{E}$ to obtain the reduced edge set~$\mathcal{E}'=\mathcal{E}\setcomplement F$.
    The resulting graph~$\tup{\cdpiN,\mathcal{E}'} $ does not have cycles, and can be written as a series-parallel graph, by applying the operators~$\dppar$ and~$\dpseries$ from a set of nodes~$\cdpiN'$.
    The nodes~$\cdpiN'$ will contain~$\cdpiN$, plus some extra ``connectors'' that are trivial DPIs.

    \begin{marginfigure}
        \includegraphics[scale=0.33]{gmcdptro_cdpi_comp3}
        \caption{}
        \label{fig:cdpi_comp3}
    \end{marginfigure}

    Find a weak topological ordering of~$\cdpiN$.
    Then the graph~$\tup{\cdpiN,\mathcal{E}'} $ can be written as the series of~$|\cdpiN|$ subgraphs, each containing one node of~$\cdpiN$.
    In the example, the weak topological ordering is~$\tup{\text{a},\text{b},\text{c}} $ and there are three subgraphs (\cref{fig:cdpi_comp3}).

    \begin{marginfigure}
        \includegraphics[scale=0.33]{gmcdptro_cdpi_comp4}
        \caption{}
        \label{fig:cdpi_comp4}
    \end{marginfigure}

    Each subgraph can be described as the parallel interconnection of a node~$\cdpin\setin\cdpiN$ and some extra connectors.
    For example, the second subgraph in the graph can be written as the parallel interconnection of node~$\text{b}$ and the identity $\triv(\mapid)$ (\cref{fig:cdpi_comp4}).

    After this is done, we just need to ``close the loop'' around the edges in the AFS~$F$ to obtain a CDPI that is equivalent to the original one.
    Suppose the AFS~$F$ contains only one edge.
    Then one instance of the~$\dploopb$ operator is sufficient~(\cref{fig:cdpi_comp5}).
    In this example, the tree representation (\cref{fig:cdpi_comp6}) is
    \begin{equation}
        \dploopb(\dpseries(\dpseries(\text{a},\dppar(\mapid,\text{b})),\text{c}).
    \end{equation}
    If the AFS contains multiple edges, then, instead of closing one loop at a time, we can always rewrite multiple nested loops as only one loop by taking the product of the edges.
    For example, a diagram like the one in~\cref{fig:nested1} can be rewritten as~\cref{fig:nested2}.
    This construction is analogous to the construction used for the analysis of process networks~\cite{lee10} (and any other construct involving a \SY{traced monoidal category}).
    Therefore, it is possible to describe an arbitrary graph of \SY{design problems} using only one instance of the ~$\dploop$ operator.
\end{proof}

\begin{figure}[tbh]
    \subfloat[\label{fig:cdpi_comp1}]{
        \centering{}\includegraphics[scale=0.33]{gmcdptro_cdpi_comp1}
    }
    \hfill{}
    \subfloat[\label{fig:cdpi_comp2}]{
        \centering{}\includegraphics[scale=0.33]{gmcdptro_cdpi_comp2}
    }
    \hfill{}
    \subfloat[\label{fig:cdpi_comp2b}]{
        \centering{}
        \includegraphics[scale=0.33]{gmcdptro_cdpi_comp2b}
    }

    \caption{An example co-design diagram with three nodes~$\cdpiN=\makeset{\text{a},\text{b},\text{c}}$,
        in which a minimal arc feedback set is $\makeset{\text{c}\rightarrow\text{a}}$.
    }
\end{figure}

\begin{figure}
    \hfill{}
    \subfloat[\label{fig:cdpi_comp5}]{
        \centering{}
        \includegraphics[scale=0.33]{gmcdptro_cdpi_comp5}}
    \hfill{}
    \subfloat[\label{fig:cdpi_comp6}]{
        \centering{}
        \includegraphics[scale=0.33]{gmcdptro_cdpi_comp6}}
    \hfill{}

    \caption{Tree representation for the co-design diagram in~\cref{fig:cdpi_comp1}.}
\end{figure}

\begin{figure}[h]
    \subfloat[\label{fig:nested1}]{
        \centering
        \includegraphics[scale=0.33]{gmcdptro_nested1}
    }\subfloat[\label{fig:nested2}]{
        \centering
        \includegraphics[scale=0.33]{gmcdptro_nested2}
    }
    \caption{If there are nested loops in a co-design diagram, they can be rewritten as one loop, by taking the product of the edges.}
    \label{fig:If-there-are}
    \todographics{@Gioele: remake figures}
\end{figure}
