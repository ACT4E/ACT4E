\section{Uncertain DPs}

We are going to build a theory of approximation for DPs.

\begin{marginfigure}
    \includegraphics[scale=0.33]{unc_ftorLU}
    \caption{}
    \label{fig:udp-bounds}
\end{marginfigure}

We will consider objects that we call Uncertain DPs (UDPs).
These are ordered pairs (intervals) of DPs.
Each pair can be interpreted as upper and lower bounds for resource consumption~(\cref{fig:udp-bounds}).

\todo{Add section references below}
In the previous chapters, we have already shown that:
\begin{itemize}
    \item For each pairs of functionality and resources $\funsp$, $\ressp$, the hom-set $\HomSet\DP\funsp\ressp$
          is a poset.
    \item Moreover, the poset is also a bounded lattice.
          The top and bottom are the DPs that are always true and false, respectively.
\end{itemize}

We can then apply the twisted construction of a poset (\cref{def:twisted-arrow}) to the hom-set, and obtain the poset
\begin{equation}
    \text{UDP}_{\funsp,\ressp} \definedas \posint \HomSet\DP\funsp\ressp
\end{equation}

We obtain:

\begin{definition}[Uncertain DPs]
    An Uncertain DP (UDP)~$\boldsymbol{u}$ is an interval of DPs~$\interv{\udpL\boldsymbol{u}}{\udpU\boldsymbol{u}}$
    such that~$\udpL\boldsymbol{u}\dpleq\udpU\boldsymbol{u}$.
\end{definition}
\begin{figure}[h!]
    \includegraphics[scale=0.43]{unc_udpdef}
    \caption{}
\end{figure}

\begin{definition}
    [Partial order $\udpleq$]
    A UDP~$\udpa$ precedes another UDP~$\udpb$ if the interval~$\interv{\udpL\udpa}{\udpU\udpa}]$
    is contained in the interval~$\interv{\udpL\udpa}{\udpU\udpa}]$ (\cref{fig:udpspace}):
    \begin{equation}
        \udpa\udpleq\udpb\quad\equiv\quad\udpL\udpb\dpleq\udpL\udpa\dpleq\udpU\udpa\dpleq\udpU\udpb.
    \end{equation}
\end{definition}

\begin{figure}[h!]
    \includegraphics[scale=0.43]{unc_udpab2}\includegraphics[scale=0.43]{unc_udpab}
    \caption{}
    \label{fig:udpspace}
\end{figure}

The partial order~$\udpleq$ has a top~$\postop_{\udpsp}=\tupp{\posbot_{\dpsp},\postop_{\dpsp}}.
$
This pair describes maximum uncertainty about the DP: we do not know
if the DP is feasible with 0 resources~($\posbot_{\dpsp}$), or if it
is completely infeasible~($\postop_{\dpsp}$).

A DP~$\ftor$ is equivalent to a degenerate UDP~$\tupp{\ftor,\ftor}$.

A UDP~$\boldsymbol{u}$ is a bound for a DP~$\ftor$ if~$\boldsymbol{u}\udpleq\tupp{\ftor,\ftor}$,
or, equivalently, if $\udpL\boldsymbol{u}\udpleq\ftor\udpleq\udpU\boldsymbol{u}$.

\begin{figure}[h!]
    \includegraphics[scale=0.43]{unc_dpcones2}\includegraphics[scale=0.43]{unc_dpcones}
    \caption{}
    \label{fig:pyr1}
\end{figure}

A pair $\tupp{\ftor,\ftor}$ is a minimal element of~$\udpsp$, because it cannot be dominated by any other.
Thus, we can imagine  the space $\udpsp$ as a pyramid~(\cref{fig:pyr1}), with the  space~$\dpsp$ forming the base.
The base represents non-uncertain  DPs.
The top of the pyramid is~$\postop_{\udpsp}$, which represents  maximum uncertainty.

%
%
% We will be able to propagate this interval uncertainty through an
% arbitrary interconnection of DPs.
% The result presented to the user
% will be a \emph{pair} of antichains \textemdash{} a lower and an upper
% bound for the resource consumption.

\begin{ctdefinition}[Category of uncertain DPs]
    \label{def:UDP}
    The category \iindex{\UDP} is defined by:
    \begin{enumerate}
        \item \emph{Objects}: posets.
        \item \emph{Morphisms}: given two posets~$\funsp$ and~$\ressp$, the hom-set~$\HomSet{\UDP}{\funsp}{\ressp}$ is the set of all intervals of DPs:
              \begin{equation}
                  \HomSet{\UDP}{\funsp}{\ressp}=\posint \HomSet\DP\funsp\ressp
              \end{equation}
        \item \emph{Composition operation}: composition is given by
              \begin{equation}
                  \mora \mthen \morb \definedas \interv{ \udpL\mora\mthen \udpL\morb}{\udpU\mora\mthen \udpU\morb}
              \end{equation}
        \item \emph{Identity morphism}: given a poset~$\posA$, the identity morphism is the degenerate interval of the DP identity
              \begin{equation}
                  \catid^{\UDP}_{\posA} \definedas \interv{\catid^{\DP}_\posA}{\catid^{\DP}_\posA}.
              \end{equation}
    \end{enumerate}
\end{ctdefinition}

\section{Approximation results}
\label{sec:Approximation-results}

The main result of this section is a relaxation result stated as \cref{thm:udpsem-monotone}
below.

Suppose that we have an MCDP composed of many DPs, and one of those is~$\ftor_{a}$~(\cref{fig:consider1}).

\begin{figure}[h!]
    \includegraphics[scale=0.33]{unc_f1}
    \caption{}
    \label{fig:consider1}
\end{figure}

Suppose that we can find two DPs $\boldsymbol{\mathsf{L}}$, $\boldsymbol{\mathsf{U}}$ that bound the DP~$\ftor_{a}$~(\cref{fig:consider2}).

\begin{figure}[h!]
    \includegraphics[scale=0.33]{unc_f2}
    \caption{}
    \label{fig:consider2}
\end{figure}

This can model either (a)~uncertainty in our knowledge
of~$\ftor_{a}$, or (b)~a relaxation that we willingly introduce.

Then we can consider the pair~$\boldsymbol{\mathsf{L}}$,
$\boldsymbol{\mathsf{U}}$ as a UDP~$\left\langle \boldsymbol{\mathsf{L}},\boldsymbol{\mathsf{U}}\right\rangle $
and we can plug it in the original MCDP in place of~$\ftor_{a}$~(\cref{fig:luinside}).

\begin{figure}[h!]
    \includegraphics[scale=0.33]{unc_f3}
    \caption{}
    \label{fig:luinside}
\end{figure}
Given the semantics of interconnections of UDPs (\cref{def:semantics-udp}),
this is equivalent to considering a pair of MCDPs, in which we choose
either the lower bound or the upper bound~(\cref{fig:pair}).

\begin{figure}[h!]
    \includegraphics[width=1\columnwidth]{unc_result}
    \caption{}
    \label{fig:pair}
\end{figure}

We can then show that the solution of the original MCDP
is bounded below and above by the solution of the new pair of MCDPs~(\cref{fig:domin}).

\begin{figure}[h!]
    \includegraphics[scale=0.33]{unc_f4}
    \caption{}
    \label{fig:domin}
\end{figure}
This result generalizes for any number of substitutions.

\subsubsection*{Formal statement}

First, we define a partial order on the valuations.
A valuation precedes
another if it gives more information on each DP.
\begin{definition}
    \label{def:For-two-valuations}
    [Partial order $\posleqof{V}$ on valuations]
    For two valuations~$\val_{1},\val_{2}:\atoms\rightarrow\udpsp$,
    say that~$\val_{1}\posleqof{V}\val_{2}$ if~$\val_{1}(a)\udpleq\val_{2}(a)$
    for all~$a\setin\atoms$.
\end{definition}
At this point, we have enough machinery in place that we can simply
state the result as ``the semantics is monotone in the valuation''.
\begin{theorem}[$\udpsem$ is monotone in the valuation]
    \label{thm:udpsem-monotone}
    If $\val_{1}\posleqof{V}\val_{2}$, then
    \begin{equation}
        \udpsem\llbracket\left\langle \atoms,\atree,\val_{1}\right\rangle \rrbracket\udpleq\udpsem\llbracket\left\langle \atoms,\atree,\val_{2}\right\rangle \rrbracket.
    \end{equation}
\end{theorem}
The proof is given in Appendix~\cref{subsec:proof-main-result}
in the supplementary materials.

This result says that we can swap any DP in a MCDP with a UDP relaxation, obtain a UMCDP, which we can solve to obtain inner and outer approximations  to the solution of the original MCDP.
This shows that considering  uncertainty in the MCDP framework is easy; as the problem reduces  to solving a pair of problems instead of one.

