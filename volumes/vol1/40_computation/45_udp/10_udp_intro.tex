\section{Uncertain DPs}

We are going to build a theory of approximation for DPs.

\begin{marginfigure}
    \includegraphics[scale=0.33]{unc_ftorLU}
    \caption{}
    \label{fig:udp-bounds}
    \todographicsjira{744}{\alphubel: @Gioele: redo picture}
\end{marginfigure}

We will consider objects that we call Uncertain DPs (UDPs).
These are ordered pairs (intervals) of DPs.
Each pair can be interpreted as upper and lower bounds for resource consumption~(\cref{fig:udp-bounds}).

\todo{@AC: Add section references below}

In the previous chapters, we have already shown that:
\begin{itemize}
    \item For each pairs of functionality and resources~$\funsp$, $\ressp$, the hom-set~$\HomSet\DP\funsp\ressp$
          is a poset.
    \item Moreover, the \SY{poset} is also a \SY{bounded lattice}.
          The top and bottom are the DPs that are always true and false, respectively.
\end{itemize}

We can then apply the twisted construction of a \SY{poset} (\cref{def:twisted-arrow}) to the hom-set, and obtain the poset
\begin{equation}
    \posint \HomSet\DP\funsp\ressp.
\end{equation}
%
The elements are intervals of DPs, which we call ``uncertain DPs''.

\begin{definition}[Uncertain DPs]\label{def:uncertain-dp}
    \SYNDEF{uncertain DP}
    An Uncertain DP (UDP)~$\boldsymbol{u}$ is an interval of DPs~$\interv{\udpL\boldsymbol{u}}{\udpU\boldsymbol{u}}$.
    % such that~$\udpL\boldsymbol{u}\dpleq\udpU\boldsymbol{u}$.
\end{definition}

\begin{figure}[h!]
    \includegraphics[scale=0.43]{unc_udpdef}
    \caption{}
    \todographicsjira{744}{\alphubel: @Gioele: redo picture, make DP using \DP}
\end{figure}

\begin{definition}
    [Partial order $\udpleq$]\label{def:udpleq}
    A UDP~$\udpa$ precedes another UDP~$\udpb$ if the interval~$\interv{\udpL\udpa}{\udpU\udpa}$
    is contained in the interval~$\interv{\udpL\udpa}{\udpU\udpa}$ (\cref{fig:udpspace}):
    \begin{equation}
        \udpa\udpleq\udpb\quad\equiv\quad\udpL\udpb\dpleq\udpL\udpa\dpleq\udpU\udpa\dpleq\udpU\udpb.
    \end{equation}
\end{definition}

\begin{figure}[h!]
    \includegraphics[scale=0.43]{unc_udpab2}\includegraphics[scale=0.43]{unc_udpab}
    \caption{}
    \label{fig:udpspace}
    \todographicsjira{744}{\alphubel: @Gioele: redo picture}
\end{figure}

The partial order~$\udpleq$ has a top~$\postop_{\UDP}=\interv {\posbot_{\dpsp}} {\postop_{\dpsp}}.
$
This pair describes maximum uncertainty about the DP: we do not know
if the DP is feasible with 0 resources~($\posbot_{\dpsp}$), or if it
is completely infeasible~($\postop_{\dpsp}$).

A DP~$\ftor$ is equivalent to a degenerate UDP~$\interv\ftor\ftor$.

A UDP~$\boldsymbol{u}$ is a bound for a DP~$\ftor$ if~$\boldsymbol{u}\udpleq\interv\ftor\ftor$,
or, equivalently, if $\udpL\boldsymbol{u}\udpleq\ftor\udpleq\udpU\boldsymbol{u}$.

\begin{figure}[h!]
    \includegraphics[scale=0.43]{unc_dpcones2}\includegraphics[scale=0.43]{unc_dpcones}
    \caption{}
    \label{fig:pyr1}
    \todographicsjira{744}{\alphubel: @Gioele: redo picture}
\end{figure}

A pair~$\interv\ftor\ftor$ is a minimal element of~$\UDP$, because it cannot be dominated by any other.
Thus, we can imagine the space $\UDP$ as a pyramid~(\cref{fig:pyr1}), with the space~$\dpsp$ forming the base.
The base represents non-uncertain DPs.
The top of the pyramid is~$\postop_{\UDP}$, which represents maximum uncertainty.

%
%
% We will be able to propagate this interval uncertainty through an
% arbitrary interconnection of DPs.
% The result presented to the user
% will be a \emph{pair} of antichains \textemdash{} a lower and an upper
% bound for the resource consumption.

\subsection{A category of uncertain DPs}

\begin{ctdefinition}[Category of uncertain DPs]
    \SYNDEF{category of uncertain DPs}
    \label{def:UDP}
    The category \UDP is defined by:
    \begin{enumerate}
        \item \emph{Objects}: \SY{posets}.
        \item \emph{Morphisms}: given two \SY{posets}~$\funsp$ and~$\ressp$, the hom-set~$\HomSet{\UDP}{\funsp}{\ressp}$ is the set of all intervals of DPs:
              \begin{equation}
                  \HomSet{\UDP}{\funsp}{\ressp}=\posint \HomSet\DP\funsp\ressp
              \end{equation}
        \item \emph{Composition operation}: composition is given by
              \begin{equation}
                  \morab \definedas \interv{ \udpL\mora\mthen \udpL\morb}{\udpU\mora\mthen \udpU\morb}
              \end{equation}
        \item \emph{Identity morphism}: given a poset~\posA, the \SY{identity morphism} is the degenerate interval of the DP identity
              \begin{equation}
                  \catid^{\UDP}_{\posA} \definedas \interv{\catid^{\DP}_\posA}{\catid^{\DP}_\posA}.
              \end{equation}
    \end{enumerate}
\end{ctdefinition}

We can prove that this category inherits the properties of \DP: it is traced monoidal and locally posetal.

\todotextjira{745}{\alphubel: @Gioele: write proof for properties of \UDP}

\section{Approximation strategy}
\label{sec:Approximation-results}

% The main result of this section is a relaxation result stated as \cref{thm:udpsem-monotone}
% below.

Suppose that we have a CDP composed of many DPs, among which one named~$\adp$ (\cref{fig:consider1}).

\begin{figure}[h!]
    \includegraphics[scale=0.33]{unc_f1}
    \caption{}
    \label{fig:consider1}
    \todographicsjira{744}{\alphubel: @Gioele: redo picture.
        Label $\adp$ rather than $\ftor_a$.
    }
\end{figure}

Suppose that we can find two DPs $\boldsymbol{\mathsf{L}}$, $\boldsymbol{\mathsf{U}}$ that bound the DP~$\adp$~(\cref{fig:consider2}).

\begin{figure}[h!]
    \includegraphics[scale=0.33]{unc_f2}
    \caption{}
    \label{fig:consider2}
    \todographicsjira{744}{\alphubel: @Gioele: redo picture}
\end{figure}

This can model either (a)~uncertainty in our knowledge of~$\adp$, or (b)~a relaxation that we willingly introduce.

Then we can consider the pair~$\boldsymbol{\mathsf{L}}$,
$\boldsymbol{\mathsf{U}}$ as a UDP~$\interv {\boldsymbol{\mathsf{L}}} {\boldsymbol{\mathsf{U}}}$,
and we can plug it in the original CDP in place of~$\adp$~(\cref{fig:luinside}).

\begin{figure}[h!]
    \includegraphics[scale=0.33]{unc_f3}
    \caption{}
    \label{fig:luinside}
    \todographicsjira{744}{\alphubel: @Gioele: redo picture}
\end{figure}

Given the semantics of interconnections of UDPs, this is equivalent to considering a pair of DPs, in which we choose either the lower bound or the upper bound~(\cref{fig:pair}).

\begin{figure}[h!]
    \includegraphics[width=1\columnwidth]{unc_result}
    \caption{}
    \label{fig:pair}
    \todographicsjira{744}{\alphubel: @Gioele: redo picture.
        Need to use $\interv\cdot\cdot$ rather than the brackets.
    }
\end{figure}

We can then show that the solution of the original DP is bounded below and above by the solution of the new pair of DPs~(\cref{fig:domin}).

\begin{figure}[h!]
    \includegraphics[scale=0.33]{unc_f4}
    \caption{}
    \label{fig:domin}
    \todographicsjira{744}{\alphubel: @Gioele: redo picture}
\end{figure}

\begin{proposition}\label{prop:udpsem-monotone}
    Consider any UDP diagram of signature~${\colF\posC }\profto {\colR\posD}$ with a hole
    of type~${\colF\posA} \profto {\colR\posB}$, as a function
    \begin{equation}
        \varphi\colon \HomSet{\UDP}{\colF\posA}{\colR\posB} \to \HomSet{\UDP}{\colF\posC}{\colR\posD},
    \end{equation}
    Then the function $\varphi$ is monotone.
    This result generalizes for any number of holes.
\end{proposition}

% \subsubsection*{Formal statement}

% First, we define a partial order on the valuations.
% A valuation precedes another if it gives more information on each DP.
% \begin{definition}
%     \label{def:For-two-valuations}
%     [Partial order $\posleqof{V}$ on valuations]
%     For two valuations~$\val_{1},\val_{2}:\atoms\rightarrow\udpsp$,
%     say that~$\val_{1}\posleqof{V}\val_{2}$ if~$\val_{1}(a)\udpleq\val_{2}(a)$
%     for all~$a\setin\atoms$.
% \end{definition}
% At this point, we have enough machinery in place that we can simply state the result as ``the semantics is monotone in the valuation''.
% \begin{theorem}[$\udpsem$ is monotone in the valuation]
%     \label{thm:udpsem-monotone}
%     If $\val_{1}\posleqof{V}\val_{2}$, then
%     \begin{equation}
%         \udpsem\llbracket\left\langle \atoms,\atree,\val_{1}\right\rangle \rrbracket\udpleq\udpsem\llbracket\left\langle \atoms,\atree,\val_{2}\right\rangle \rrbracket.
%     \end{equation}
% \end{theorem}
% The proof is given in Appendix~\cref{subsec:proof-main-result} in the supplementary materials.

This result says that we can swap any DP in a CDP with a UDP relaxation, obtain an uncertain CDP, which we can solve to obtain inner and outer approximations to the solution of the original CDP.
This shows that considering uncertainty in the CDP framework is easy; as the problem reduces to solving a pair of problems instead of one.

\section[Parametric uncertainty]{Dealing with parametric uncertainty}
\label{sec:Applications}

We proceed to show three example applications of the theory:
\begin{enumerate}
    \item The first example deals with \emph{parametric uncertainty}.
    \item The second example deals with the idea of ``relaxation of a scalar relation''.
          This is equivalent to accepting a tolerance for a given variable, in exchange for a reduced number of iterations.
    \item The third example deals with the relaxation of relations with infinite cardinality.
          In particular, it shows how we can obtain consistent estimates with a finite and prescribed amount of computation.
\end{enumerate}

We use the same model in all cases, show in \cref{fig:Example1}.
\begin{figure*}[b!]
    \centering
    \includegraphics[scale=0.43]{unc_bigproblem}
    \caption{
        % Monotone Co-Design Problems (MCDPs) can capture much of the complexity of the optimal robot design process.
        % The user defines a co-design diagram by hierarchical composition and arbitrary interconnection of primitive ``design problems'', modeled as monotone relations between \F{functionality} and \R{resources}.
        The semantics of the drone design exercises is the minimization of the \R{total mass} and \R{cost} of the platform, subject to functionality constraints (\F{distance}, \F{payload}, \F{number of missions}).
        We discuss how to introduce uncertainty in this framework, which allows, for example, to introduce parametric uncertainty in the definition of components properties (\eg, specific cost of batteries).
    }
    \label{fig:Example1}
\end{figure*}
\vfill\pagebreak

\subsection{Application: Dealing with Parametric Uncertainty\label{sec:Application-uncertainty}}

To instantiate the model in~\cref{fig:Example1}, we need to obtain numbers for energy density, specific cost, and operating life for all batteries technologies we want to examine.

By browsing Wikipedia, we can find the figures in~\cref{tab:unc_batteries}.

\begin{table}[h]
    \caption{Specifications of common batteries technologies}
    \label{tab:unc_batteries}
    \centering
    \begin{tabular}{crr@{\extracolsep{0pt}.}lr}
        \multirow{2}{*}{{\footnotesize{}\tableColors}\emph{\footnotesize{}technology}} & \emph{\footnotesize{}energy density}             & \multicolumn{2}{c}{\emph{\footnotesize{}specific cost}} & \emph{\footnotesize{}operating life}\tabularnewline
                                                                                       & {\footnotesize{}{[}
        Wh/kg{]}}                                                                      & \multicolumn{2}{c}{{\footnotesize{}{[}Wh/\${]}}} &
        {\footnotesize \# cycles}\tabularnewline
        {\footnotesize{}NiMH}                                                          & {\footnotesize{}100}                             & {\footnotesize{}3}                                      & {\footnotesize{}41 }                                & {\footnotesize{}500 }\tabularnewline
        {\footnotesize{}NiH2}                                                          & {\footnotesize{}45}                              & {\footnotesize{}10}                                     & {\footnotesize{}50 }                                & {\footnotesize{}20000}\tabularnewline
        {\footnotesize{}LCO}                                                           & {\footnotesize{}195}                             & {\footnotesize{}2}                                      & {\footnotesize{}84}                                 & {\footnotesize{}750}\tabularnewline
        {\footnotesize{}LMO}                                                           & {\footnotesize{}150}                             & {\footnotesize{}2}                                      & {\footnotesize{}84 }                                & {\footnotesize{}500}\tabularnewline
        {\footnotesize{}NiCad}                                                         & {\footnotesize{}30}                              & {\footnotesize{}7}                                      & {\footnotesize{}50 }                                & {\footnotesize{}500}\tabularnewline
        {\footnotesize{}SLA}                                                           & {\footnotesize{}30}                              & {\footnotesize{}7}                                      & {\footnotesize{}00}                                 & {\footnotesize{}500}\tabularnewline
        {\footnotesize{}LiPo}                                                          & {\footnotesize{}150}                             & {\footnotesize{}2}                                      & {\footnotesize{}50}                                 & {\footnotesize{}600}\tabularnewline
        {\footnotesize{}LFP}                                                           & {\footnotesize{}90}                              & {\footnotesize{}1}                                      & {\footnotesize{}50}                                 & {\footnotesize{}1500}\tabularnewline
    \end{tabular}
\end{table}

Should we trust those figures?
Fortunately, we can easily deal with possible mistrust by introducing uncertain DPs.

Formally, we replace the DPs for\emph{ energy density}, \emph{specific cost}, \emph{operating life} in~\cref{fig:Example1} with the corresponding Uncertain DPs with a configurable uncertainty.
We can then solve the UDPs to obtain a lower bound and an upper bound to the solutions that can be presented to the user.

\Cref{fig:unc_battery_uncertain} shows the relation between the provided \F{endurance} and the minimal \R{total mass} required, when using uncertainty of $5\%$, $10\%$, $25\%$ on the numbers above.
Each panel shows two curves: the lower bound (best case analysis) and the upper bound (worst case analysis).
In some cases, the lower bound is feasible, but the upper bound is not.
For example, in panel~\emph{b}, for 10\% uncertainty, we can conclude that, notwithstanding the uncertainty, there exists a solution for endurance~$\leq1.3\,\text{hours}$, while for higher endurance, because the upper bound is infeasible, we cannot conclude that there is a solution \textemdash{} though, because the lower bound is feasible, we cannot conclude that a solution does not exist~(\cref{fig:unc_battery_uncertain}c).
\vfill
\begin{figure*}[h]
    \centering
    \includegraphics[scale=0.53]{unc_battery_uncertain}
    \caption{
        Uncertain relation between \F{endurance} and the minimal \R{total mass} required, obtained by solving the example in \cref{fig:Example1} for different values of the uncertainty on the characteristics of the batteries.
        As the uncertainty increases, there are no solutions for the worst case.
    }
    \label{fig:unc_battery_uncertain}
\end{figure*}

\section{Introducing tolerances}
\label{sec:Application-tolerance}

Another application of the theory is the introduction of tolerances for any variable in the optimization problem.
For example, one might not care about the variations of the battery mass below, say,~$1\,\text{g}$.
One can then introduce a $\pm1\,\text{\ensuremath{\text{g}} }$ uncertainty in the definition of the problem by adding a UDP hereby called ``uncertain identity''.

\begin{marginfigure}
    \centering
    \subfloat{\includegraphics[scale=0.53,trim=0 22mm 0 0,clip=true]{unc_approx1a}}

    \vspace{10mm}
    \subfloat{\includegraphics[scale=0.53,trim=0 22mm 0 0,clip=true]{unc_approx1b}}

    \vspace{10mm}
    \subfloat{\includegraphics[scale=0.53,trim=0 22mm 0 0,clip=true]{unc_approx1c}}

    \caption{
        The identity and its two relaxations~$\ufloor_{\alpha}$ and $\uceil_{\alpha}$.
    }
    \label{fig:identity_approximation}
    \todographicsjira{744}{\alphubel: @Gioele: redo pictures.
        No need to include the definitions of the functions.
    }
\end{marginfigure}

\subsubsection{The uncertain identity}

Let~$\alpha>0$ be a step size.
Define~$\funfloor_{\alpha}$ and~$\funceil_{\alpha}$ to be the floor and ceiling functions with step size~$\alpha$~(\cref{fig:identity_approximation}):
\begin{equation}
    \defmapcomma{
        \funfloor_{\alpha}
    }{
        \nonNegReals
    }{
        \toinPos
    }{
        \nonNegReals
    }{
        x
    }{
        \alpha \cdot \funfloor(x/\alpha)
    }
\end{equation}
\begin{equation}
    \defmapperiod{
        \funceil_{\alpha}
    }{
        \nonNegReals
    }{
        \toinPos
    }{
        \nonNegReals
    }{
        x
    }{
        \alpha \cdot \funceil(x/\alpha)
    }
\end{equation}
It is easy to verify that for all~$\alpha$ and for all~$x$,
\begin{equation}
    \funfloor_{\alpha}(x) \leq x \leq \funceil_{\alpha}(x).
\end{equation}
Then, define $\ufloor_{\alpha}$ and $\uceil_\alpha$ as the \SY{companions} (see \cref{def:comp_conj}) of those two functions:
\begin{align}
    \ufloor_{\alpha} & = \comp{\funfloor_\alpha}, \\
    \uceil_{\alpha}  & = \comp{\funceil_\alpha} .
\end{align}
By construction,
\begin{equation}
    \ufloor_{\alpha}\dpleq\catid\dpleq\uceil_{\alpha}.
\end{equation}
Therefore, we can pair the two DPs to create a UDP that we call the ``uncertain identity'' as
\begin{equation}
    \UId_{\alpha}\definedas \interv {\ufloor_{\alpha}} {\uceil_{\alpha}}.
\end{equation}
For~$0<\alpha<\beta$, it holds that
%
\begin{equation}
    \UId\posltof{\UDP}\UId_{\alpha}\posltof{\UDP}\UId_{\beta}.
\end{equation}
%
Therefore, the sequence~$\UId_{\alpha}$ is a descending \SY{chain} that converges to~$\UId$ as~$\alpha\to0$~(\cref{fig:other}).

\begin{figure}[b]
    \hfill
    \includegraphics[scale=0.43]{unc_uid1}
    \hfill
    \includegraphics[scale=0.43]{unc_uid2}
    \hfill
    \caption{}
    \label{fig:other}
    \todographicsjira{744}{\alphubel: @Gioele: redo picture.}
\end{figure}

\vfill\pagebreak

\subsubsection{Approximations in CDPs}

We can take any edge, or multiple edges, in a CDP and apply this relaxation.

\begin{marginfigure}
    \includegraphics[scale=0.43]{unc_introduce}
    \caption{}
    \label{fig:introduce}
    \todographicsjira{744}{\alphubel: @Gioele: redo picture}
\end{marginfigure}

Formally, given an edge, we first introduce an identity~$\UId$, and then relax it using the uncertain identity~$\UId_{\alpha}$,
as shown in~\cref{fig:introduce}.

If the original DP is $\adp$, we have constructed a sequence $\adp_\alpha$ for which, by construction,
%
\begin{equation}
    \adp\udpleq\adp_{\alpha},
\end{equation}
%
which means that~$\adp_{\alpha}$ is a bound for $\adp$.
This means that we can solve~$\udpL\adp_{\alpha}$ and~$\udpU\adp_{\alpha}$ and obtain upper and lower bounds for~$\adp$.
Furthermore, by varying~$\alpha$, we can construct an approximating sequence of DPs whose solution will converge to the solution of the original CDP.

% Mathematically, given an MCDP~$\tupp{ \atoms,\atree,\val}$, we generate a UMCDP~$\tupp{\atoms,\atree,\val_{\alpha}}$, where the new valuation~$\val_{\alpha}$ agrees with~$\val$ except on a particular atom~$a\setin\atoms$, which is replaced by the series of the original~$\val(a)$ and the approximation~$\text{UId}_{\alpha}$:
% %
% \begin{equation}
%     \val_{\alpha}(a) \definedas \dpseries(\UId_{\alpha},\val(a))
% \end{equation}
% %
% Call the original and approximated DPs~$\adp$ and~$\adp_{\alpha}$:
% %
% \begin{equation}
%     \begin{array}{ccc}
%         \adp\definedas\udpsem\left\llbracket \left\langle \atoms,\atree,\val\right\rangle \right\rrbracket , &  & \adp{}_{\alpha}\definedas\udpsem\left\llbracket \left\langle \atoms,\atree,\val_{\alpha}\right\rangle \right\rrbracket .
%     \end{array}
% \end{equation}
% % Because $\val\posleqof{V}\val_{\alpha}$ (in the sense of~\cref{def:For-two-valuations,}), \cref{thm:udpsem-monotone} implies that
% From previous results, we know
% %
%

\subsubsection{Numerical results}

This procedure was applied to the example model in~\cref{fig:Example1} by introducing a tolerance to the ``power'' variable for the actuation.
The tolerance~$\alpha$ is chosen at logarithmic intervals between~$0.01\,\text{mW}$ and~$1\,\text{W}$.

\cref{fig:mass}~shows the solutions of the minimal mass required for~$\udpL\adp_{\alpha}$ and~$\udpU\adp_{\alpha}$, as a function of~$\alpha$. \cref{fig:mass} confirms the consistency results predicted by the theory.

First, if the solutions for both~$\udpL\adp_{\alpha}$ and~$\udpU\adp_{\alpha}$ exist, then they are ordered ($\udpL\adp_{\alpha}(\fun)\posleq\udpU\adp_{\alpha}(\fun)$).
Second, as~$\alpha$ decreases, the interval shrinks.
Third, the bounds are consistent (the solution for the original DP is always contained in the bound).

Next, it is interesting to consider the computational complexity.
\Cref{fig:num_iterations}~shows the number of iterations as a function of the resolution~$\alpha$, and the trade-off of the uncertainty of the solution and the computational resources spent.
This shows that this approximation scheme is an effective way to reduce the computation load while maintaining a consistent estimate.

\vfill
\begin{figure}[h]
    \subfloat[\label{fig:mass}]{
        \includegraphics[scale=0.5]{unc_approx2a}
    }%
    \subfloat[\label{fig:num_iterations}]{
        \includegraphics[scale=0.5]{unc_approx2c}
    }
    \caption{
        Results of model in \cref{fig:Example1} when tolerance is applied to the actuation \R{power} resource.
    }
    \todographicsjira{744}{\alphubel: @Gioele: do SURGERY on figure.
        Cover up the existing writing using
        $\udpL\adp_{\alpha}(\fun), \udpU\adp_{\alpha}(\fun)$.
    }
\end{figure}

\section[Uncomputable DPs]{Relaxation for uncomputable DPs}
\label{sec:Application-relax}

Another way in which uncertain DPs can be used is to construct approximations of DPs that are not computable.
For example, consider a relation like
%
\begin{equation}
    \label{eq:qun}
    {\colF\text{travel\_distance}}
    \leq{\colR\text{velocity}}\cdot{\colR\text{endurance}},
\end{equation}
%
which appears in the model in \cref{fig:Example1}.
If we take these three quantities in \cref{eq:qun} as belonging to~\reals, then, for each value of the \F{travel distance}, there are infinite pairs of~$\tup{{\colR\text{velocity}},{\colR\text{endurance}}} $ that are feasible.
This makes the problem not computable, as the solution cannot be represented as a finite \SY{antichain}.

Even if we are happy to first approximation \reals using the (finite) floating point numbers, the representation that \SY{antichain} would take billions of elements.

Using uncertainty, we can avoid considering all combinations by creating a sequence of uncertain DPs that use finite and prescribed computation.

\todotext{@AC: Mention the finite dp}

\subsubsection{Relaxations for addition}

\begin{marginfigure}
    \centering
    \includesag{dp_relax}
    \caption{}
    \label{fig:example-invplus}
\end{marginfigure}

Consider a monotone relation between some functionality~$\funan{1}\setin\nonNegReals$ and resources~$\resan{1},\resan{2}\setin\nonNegReals$ described by the constraint that~$\funan{1}\leq\resan{1}+\resan{2}$ (\cref{fig:example-invplus}).
For example, this could represent the case where there are two batteries providing the power~$\funan{1}$, and we need to decide how much to allocate to the first~($\resan{1}$) or the second~($\resan{2}$).

The formal definition of this constraint as an DP is
\begin{equation}
    \defmapperiod{\overline{+}}
    {\colF\nonNegReals}
    {\to}
    {\colR\antichains(\nonNegReals\Ctimes\nonNegReals)}
    {\funan{1}}
    {\makeset{\tupp{ \ela,\funan{1}-\ela} \mid \ela \setin \interv0{\funan1}}}
\end{equation}

Note that, for each value~$\funan{1}$,~$\overline{+}(\funan{1})$ is a set of infinite cardinality.

We will now define two sequences of relaxations for~$\overline{+}$ with a fixed number of solutions~$n\geq1$.

\subsubsection*{Using uniform sampling}

We will first define a sequence of UDPs~$S_{n}$ based on uniform sampling.
Let~$\udpU S_{n}$ consist of~$n$ points sampled on the segment with extrema~$\tup{0,\funan{1}}$ and~$\tup{\funan{1},0} $.
For~$\udpL S_{n}$, sample~$n+1$ points on the segment and take the \emph{meet} of successive points~(\cref{fig:make_lower}).

\begin{figure}[h]
    \includegraphics[scale=0.43]{unc_make_lower}
    \caption{}
    \label{fig:make_lower}
\end{figure}

The first elements of the sequences are shown in~\cref{fig:approx_invplus}.
One can easily prove that~$\udpL S_{n}\dpleq\overline{+}\dpleq\udpU S_{n}$, and thus~$S_{n}$ is a relaxation of~$\overline{+}$, in the sense that~$\overline{+}\udpleq S_{n}$.
Moreover,~$S_{n}$ converges to $\overline{+}$ as $n\rightarrow\infty$.

\begin{figure}[h]
    \centering
    \includegraphics[scale=0.43]{unc_sampling}
    \caption{Approximations to $\overline{+}$ using the uniform sampling sequence~$S_{n}$. }
    \label{fig:approx_invplus}
\end{figure}

However, note that the convergence is not monotonic:~$S_{n+1}{\not\posleq}_{\UDP}S_{n}$.
The situation can be represented graphically as in~\cref{fig:notchain}.
The sequence~$S_{n}$ eventually converges to~$\overline{+}$, but it is not a descending \SY{chain}.
This means that it is not true, in general, that the solution to the CDP obtained by plugging in~$S_{n+1}$ gives smaller bounds than~$S_{n}$.

\subsubsection*{Relaxation based on Van Der Corput sequence}
\label{sec:van-der-corput}
We can easily create an approximation sequence~$V:\natnumbers\to\UDP$ that converges monotonically using Var Der Corput (VDC) sampling~\cite[Section 5.2]{LaValle2006Planning}.
Let~$\vdc(n)$ be the VDC sequence of~$n$ elements in the interval~$[0,1]$.
The first elements of the VDC are
\begin{equation}
    0,0.5,0.25,0.75,0.125,\dots
\end{equation}
The sequence is guaranteed to satisfy~$\vdc(n)\setsubseteq\vdc(n+1)$ and to minimize the discrepancy.
The upper bound~$\udpU V_{n}$ is defined as sampling the segment with extrema~$\tupp{ 0,\funan{1}} $ and~$\tupp{\funan{1},0}$ using the VDC sequence:
\begin{equation}
    \udpU V_{n}\colon\funan{1}\mapsto\makeset{ \tupp{\funan{1}\ela,\funan{1}(1-\ela)} \mid \ela\setin\vdc(n)}.
\end{equation}
The lower bound~$\udpL V_{n}$ is defined by taking meets of successive points, according to the procedure in~\cref{fig:make_lower}.

\begin{figure}[h]
    \centering
    \includegraphics[scale=0.40]{unc_samplingb}
    \caption{
        Approximations to $\overline{+}$ using the Van Der Corput sequence~$V_{n}$.
    }
    \label{fig:Vn}
\end{figure}

For this sequence, we can prove that not only~$\overline{+}\udpleq V_{n}$, but also that the convergence is uniform, in the sense that~$\overline{+}\udpleq V_{n+1}\udpleq V_{n}$.
The situation is represented graphically in~\cref{fig:convergence_pyramid}: the sequence is a descending \SY{chain} that converges to~$\overline{+}$.

\subsubsection{Inverse of multiplication}

The case of multiplication can be treated analogously to the case of addition.
By taking the logarithm, the inequality~$\funan{1}\leq\resan{1}\resan{2}$ can be rewritten as~$\log(\funan{1})\leq\log(\resan{1})+\log(\resan{2})$.
So we can repeat the constructions done for addition.
The VDC sequence are shown in~\cref{fig:approx_invmult}.

\begin{figure}[h]
    \centering
    \includegraphics[scale=0.40]{unc_sampling2b}
    \caption{
        Van Der Corput relaxations for the relation $\fun_{1}\leq\res_{1}\res_{2}$.}
    \label{fig:approx_invmult}
\end{figure}

\subsection{Numerical example}

We have applied this relaxation to the relation ${\colF\text{travel distance}}\leq{\colR\text{velocity}}\times{\colR\text{endurance}}$ in the CDP in~\cref{fig:Example1}.
Thanks to this theory, we can obtain estimates of the solutions using bounded computation, even though that relation has infinite cardinality.

\Cref{fig:invplus1}~shows the result using uniform sampling, and~\cref{fig:invplus2} shows the result using VDC sampling.
As predicted by the theory, uniform sampling does not give monotone convergence, while VDC sampling does.
\vfill

\begin{figure}[h]
    \centering
    \subfloat[\label{fig:notchain}
        Qualitative behavior for $S_{n}$]{
        \centering
        \includegraphics[scale=0.43]{unc_convergence_pyramid}}
    \subfloat[\label{fig:convergence_pyramid}Qualitative behavior for $V_{n}$]{
        \centering
        \includegraphics[scale=0.43]{unc_convergence_pyramid2}
    }

    \centering
    \subfloat[\label{fig:invplus1}Numerical results for $S_{n}$]{
        \centering
        \adjustbox{max width=5.0cm}{%
            \includegraphics[scale=0.43]{unc_convergence}
        }
    }\subfloat[\label{fig:invplus2}Numerical results for $V_{n}$]{
        \centering
        \adjustbox{max width=5.0cm}{%
            \includegraphics[scale=0.43]{unc_convergence2}%
        }
    }

    \caption{
        Solutions to the example in~\cref{fig:Example1}, applying relaxations for the relation ${\colF\text{travel\_distance}}\leq{\colR\text{velocity}}\times{\colR\text{endurance}}$ using the uniform sampling sequence and the VDC sampling sequence.
        The uniform sampling sequence~$S_{n}$ does not converge monotonically (panel~\emph{a}); therefore the progress is not monotonic~(panel\emph{~c}).
        Conversely, the Van Der Corput sequence~$V_{n}$ is a descending \SY{chain} (panel~\emph{b}), which results in monotonic progress (panel~\emph{d}).
    }
\end{figure}

%\section{Conclusions and future work}
%
%Monotone Co-Design Problems (MCDPs) provide a compositional theory
%of ``co-design'' that describes co-design constraints among different
%subsystems in a complex system, such as a robotic system.
%
%This paper dealt with the introduction of uncertainty in the framework,
%specifically, interval uncertainty.
%
%Uncertainty can be used in two roles. First, it can be used to describe
%limited knowledge in the models. For example, in \cref{sec:Application-uncertainty},
%we have seen how this can be applied to model mistrust about numbers
%from Wikipedia. Second, uncertainty allows to generate relaxations
%of the problem. We have seen two applications: introducing an allowed
%tolerance in one particular variable (\cref{sec:Application-tolerance}),
%and dealing with relations with infinite cardinality using bounded
%computation resources (\cref{sec:Application-relax}).
%
%Future work includes strengthening these results. For example, we
%are not able to predict the resulting uncertainty in the solution
%before actually computing it; ideally, one would like to know how
%much computation is needed (measured by the number of points in the
%antichain approximation) for a given value of the uncertainty that
%the user can accept.

% \section{Proofs}

% \subsection{Proofs of well-formedness of \cref{def:semantics-udp}}

% As some preliminary business, we need to prove that \cref{def:semantics-udp}
% is well-formed, in the sense that the way the semantics function~$\udpsem$
% is defined, it returns a UDP for each argument.
% This is not obvious
% from~\cref{def:semantics-udp}.

% For example, for $\udpsem\llbracket\atoms,\dpseries(\atree_{1},\atree_{2}),\val\rrbracket$,
% the definition gives values for~$\udpL\udpsem\llbracket\atoms,\dpseries(\atree_{1},\atree_{2}),\val\rrbracket$
% and~$\udpU\udpsem\llbracket\atoms,\dpseries(\atree_{1},\atree_{2}),\val\rrbracket$
% separately, without checking that
% \begin{equation}
%     \udpL\udpsem\llbracket\atoms,\dpseries(\atree_{1},\atree_{2}),\val\rrbracket\dpleq\udpU\udpsem\llbracket\atoms,\dpseries(\atree_{1},\atree_{2}),\val\rrbracket.
% \end{equation}
% The following lemma provides the proof for that.
% \begin{lemma}
%     \label{lem:udpsem-well-formed}
%     \Cref{def:semantics-udp} is well-formed, in the sense that
%     \begin{equation}
%         \udpL\udpsem\llbracket\langle\atoms,\dpseries(\atree_{1},\atree_{2}),\val\rangle\rrbracket\dpleq\udpU\udpsem\llbracket\langle\atoms,\dpseries(\atree_{1},\atree_{2}),\val\rangle\rrbracket,\label{eq:wf1}
%     \end{equation}
%     \begin{equation}
%         \udpL\udpsem\llbracket\left\langle \atoms,\dppar(\atree_{1},\atree_{2}),\val\right\rangle \rrbracket\dpleq\udpU\udpsem\llbracket\left\langle \atoms,\dppar(\atree_{1},\atree_{2}),\val\right\rangle \rrbracket,\label{eq:wf2}
%     \end{equation}
%     \begin{equation}
%         \udpL\udpsem\llbracket\left\langle \atoms,\dploop(\atree),\val\right\rangle \rrbracket\dpleq\udpU\udpsem\llbracket\left\langle \atoms,\dploop(\atree),\val\right\rangle \rrbracket.\label{eq:wf3}
%     \end{equation}
% \end{lemma}
% \begin{proof}
%     Proving \cref{eq:wf1}\textemdash \cref{eq:wf3} can be
%     reduced to proving the following three results, for any $x,y\setin\udpsp$:
%     \begin{align*}
%         \left(\udpL x\opseries\udpL y\right) & \dpleq\left(\udpU x\opseries\udpU y\right), \\
%         \left(\udpL x\oppar\udpL y\right)    & \dpleq\left(\udpU x\oppar\udpU y\right), \\
%         \left(\udpL x\right)^{\oploop}       & \dpleq\left(\udpU x\right)^{\oploop}.
%     \end{align*}
%     These are given in \cref{lem:well-formed-series}, \cref{lem:well-formed-par},
%     \cref{lem:well-formed-loop}.
% \end{proof}
% \begin{lemma}
%     \label{lem:well-formed-series}$\left(\udpL x\opseries\udpL y\right)\dpleq\left(\udpU x\opseries\udpU y\right)$.
% \end{lemma}
% \begin{proof}
%     First prove that~$\opseries$ is monotone in each argument (proved
%     as~\cref{lem:series-monotone}).
%     Then note that
%     \begin{equation}
%         \left(\udpL x\opseries\udpL y\right)\dpleq\left(\udpL x\opseries\udpU y\right)\dpleq\left(\udpU x\opseries\udpU y\right).
%     \end{equation}
% \end{proof}
% \begin{lemma}
%     \label{lem:well-formed-par}$\left(\udpL x\oppar\udpL y\right)\dpleq\left(\udpU x\oppar\udpU y\right)$.
% \end{lemma}
% \begin{proof}
%     The proof is entirely equivalent to the proof of~\cref{lem:well-formed-series}.
%     First prove that~$\dppar$ is monotone in each argument (proved as~\cref{lem:par-monotone}).
%     Then note that
%     \begin{equation}
%         \left(\udpL x\oppar\udpL y\right)\dpleq\left(\udpL x\oppar\udpU y\right)\dpleq\left(\udpU x\oppar\udpU y\right).
%     \end{equation}
% \end{proof}

% \begin{lemma}
%     \label{lem:well-formed-loop}$\left(\udpL x\right)^{\oploop}\dpleq\left(\udpU x\right)^{\oploop}$.
% \end{lemma}
% \begin{proof}
%     This follows from the fact that~$\oploop$ is monotone (\cref{lem:loop-monotone}).
% \end{proof}

% \subsection{Monotonicity lemmas for DP}

% These lemmas are used in the proofs above.
% \begin{lemma}
%     \label{lem:series-monotone}$\opseries:\dpsp\times\dpsp\rightarrow\dpsp$
%     is monotone on~$\langle\dpsp,\dpleq\rangle$.
% \end{lemma}
% \begin{proof}
%     In~\cref{def:opseries}, $\opseries$ is defined as follows
%     for two maps~$\ftor_{1}\colon\funsp_{1}\toinPos\Aressp_{1}$ and~$\ftor_{2}\colon\funsp_{2}\toinPos\Aressp_{2}$:
%     \begin{equation}
%     {\displaystyle \ftor_{1}\opseries\ftor_{2}=\Min_{\posleqof{\ressp_{2}}}\uparrow\bigsetunion_{s\setin\ftor_{1}(\fun)}\ftor_{2}(s)}
%         .
%     \end{equation}
%     It is useful to decompose this expression as the composition of three
%     maps:
%     \begin{equation}
%         \ftor_{1}\opseries\ftor_{2}=m\circ g[\ftor_{2}]\circ\ftor_{1},
%     \end{equation}
%     where~``$\circ$'' is the usual map composition, and~$g$ and~$m$
%     are defined as follows:
%     \begin{align*}
%         g[\ftor_{2}]:\Aressp_{1} & \rightarrow\Uressp_{2}, \\
%         R                        & \mapsto\uparrow\bigsetunion_{s\setin R}\ftor_{2}(s),
%     \end{align*}
%     and
%     \begin{align*}
%         m:\Uressp_{2} & \to \Aressp_{2}, \\
%         R             & \mapsto\Min_{\posleqof{\ressp_{2}}}
%         R.
%     \end{align*}

%     From the following facts:
%     \begin{itemize}
%         \item $m$ is monotone.
%         \item $g[\ftor_{2}]$ is monotone in $\ftor_{2}$.
%         \item $f_{1}\circ f_{2}$ is monotone in each argument if the other argument
%         is monotone.
%     \end{itemize}
%     Then the thesis follows.
% \end{proof}

% \begin{lemma}
%     \label{lem:par-monotone}$\oppar:\dpsp\times\dpsp\rightarrow\dpsp$
%     is monotone on~$\langle\dpsp,\dpleq\rangle$.
% \end{lemma}
% \begin{proof}
%     The definition of $\oppar$ (\cref{def:opmaps}) is:
%     \begin{align*}
%         \ftor_{1}\oppar\ftor_{2}:(\funsp_{1}\times\funsp_{2}) & \rightarrow\antichains(\ressp_{1}\times\ressp_{2}), \\
%         \left\langle \fun_{1},\fun_{2}\right\rangle           & \mapsto\ftor_{1}(\fun_{1})\times\ftor_{2}(\fun_{2}).
%     \end{align*}
%     Because of symmetry, it suffices to prove that $\oppar$ is monotone
%     in the first argument, leaving the second fixed.

%     We need to prove that for any two DPs $\ftor_{a},\ftor_{b}$ such
%     that
%     \begin{equation}
%         \ftor_{a}\dpleq\ftor_{b},\label{eq:Ikno}
%     \end{equation}
%     and for any fixed $\overline{\ftor}$, then
%     \begin{equation}
%         \ftor_{a}\oppar\overline{\ftor}\dpleq\ftor_{b}\oppar\overline{\ftor}.
%     \end{equation}
%     Let $R=\overline{\ftor}(\fun_{2})$.
%     Then we have that
%     \begin{align*}
%     [\ftor_{a}\oppar\overline{\ftor}]
%     (\fun_{1},\fun_{2})
%         & =\ftor_{a}(\fun_{1})\acprod R,\\makeset{}
%         [\ftor_{b}\oppar\overline{\ftor}](\fun_{1},\fun_{2}) & =\ftor_{b}(\fun_{1})\acprod R.
%     \end{align*}
%     Because of \cref{eq:Ikno}, we know that
%     \begin{equation}
%         \ftor_{a}(\fun_{1})\posleqof{\Aressp_{1}}\ftor_{b}(\fun_{1}).
%     \end{equation}
%     So the thesis follows from proving that the product of antichains is monotone~(\cref{lem:product-monotone}).
% \end{proof}
% \begin{lemma}
%     \label{lem:product-monotone}
%     The product of antichains~$\acprod:\Aressp_{1}\times\Aressp_{2}\rightarrow\antichains(\ressp_{1}\times\ressp_{2})$
%     is monotone.
% \end{lemma}

% \begin{lemma}
%     \label{lem:loop-monotone}$\oploop:\dpsp\rightarrow\dpsp$ is monotone
%     on~$\langle\dpsp,\dpleq\rangle$.
% \end{lemma}
% \begin{proof}
%     Let $\ftor_{1}\dpleq\ftor_{2}$.
%     Then we can prove that $\ftor_{1}^{\oploop}\dpleq\ftor_{2}^{\oploop}$.
%     From the definition of~$\oploop$~(\cref{def:oploop}), we have that
%     \begin{align*}
%         \ftor_{1}^{\oploop}(\fun_{1}) & =\lfp(\Psi_{\fun}^{\ftor_{1}}), \\
%         \ftor_{2}^{\oploop}(\fun_{2}) & =\lfp(\Psi_{\fun}^{\ftor_{2}}),
%     \end{align*}
%     with~$\Psi_{\fun_{1}}^{\ftor}$ defined as
%     \begin{align*}
%         \Psi_{\fun_{1}}^{\ftor}:\Aressp & \rightarrow\Aressp, \\
%         {\colR R}                       & \mapsto\Min_{\posleqof{\ressp}}\bigsetunion_{\res\setin{\colR R}}\ftor(\fun_{1},\res)\ \setintersection\uparrow\res.
%     \end{align*}
%     The least fixed point operator $\lfp$ is monotone, so we are left
%     to check that the map
%     \begin{equation}
%         \ftor\mapsto\Psi_{\fun_{1}}^{\ftor}
%     \end{equation}
%     is monotone.
%     That is the case, because if $\ftor_{1}\dpleq\ftor_{2}$
%     then
%     \begin{equation}
%         \left[\bigsetunion_{\res\setin{\colR R}}\ftor_{1}(\fun_{1},\res)\ \setintersection\uparrow\res\right]\posleqof{\Aressp}\left[\bigsetunion_{\res\setin{\colR R}}\ftor_{2}(\fun_{1},\res)\ \setintersection\uparrow\res\right].
%     \end{equation}
% \end{proof}

% \subsection{Monotonicity of semantics $\dpsem$}

% \begin{lemma}[$\dpsem$ is monotone in the valuation]
%     \label{lem:dpsem-monotone}
%     Suppose that~$\val_{1},\val_{2}:\atoms\rightarrow\dpsp$
%     are two valuations for which it holds that~$\val_{1}(a)\dpleq\val_{2}(a)$.
%     Then~$\dpsem\llbracket\left\langle \atoms,\atree,\val_{1}\right\rangle \rrbracket\dpleq\dpsem\llbracket\left\langle \atoms,\atree,\val_{2}\right\rangle \rrbracket$.
% \end{lemma}
% \begin{proof}
%     Given the recursive definition of \cref{def:dpsem}, we need
%     to prove this just for the base case and for the recursive cases.

%     The base case, given in \cref{eq:base}, is
%     \begin{equation}
%         \dpsem\llbracket\left\langle \atoms,a,\val\right\rangle \rrbracket\definedas\val(a),\qquad\text{for all}\ a\setin\atoms.
%     \end{equation}
%     We have
%     \begin{align*}
%         \dpsem\llbracket\left\langle \atoms,\atree,\val_{1}\right\rangle \rrbracket & =\val_{1}(a) \\
%         \dpsem\llbracket\left\langle \atoms,\atree,\val_{2}\right\rangle \rrbracket & =\val_{2}(a)
%     \end{align*}
%     and $\val_{1}(a)\dpleq\val_{2}(a)$ by assumption.

%     For the recursive cases (\crefrange{eq:series}{eq:loop}),
%     the thesis follows from the monotonicity of $\opseries$, $\oppar$,
%     $\oploop$, proved in \cref{lem:par-monotone}, \cref{lem:series-monotone},
%     \cref{lem:loop-monotone}.
% \end{proof}

% \subsection{Proof of the main result, \cref{thm:udpsem-monotone}}

% \label{subsec:proof-main-result}

% We restate the theorem.

% \textbf{Theorem~\ref{thm:udpsem-monotone}}. \emph{If
%     \begin{equation}
%         \val_{1}\posleqof{V}\val_{2}
%     \end{equation}
%     then
%     \begin{equation}
%         \udpsem\llbracket\left\langle \atoms,\atree,\val_{1}\right\rangle \rrbracket\udpleq\udpsem\llbracket\left\langle \atoms,\atree,\val_{2}\right\rangle \rrbracket.
%     \end{equation}
% }
% \begin{proof}
%     From the definition of $\udpsem$ and $\dpsem$, we can derive that
%     \begin{align}
%         \udpL\udpsem\llbracket\langle\atoms,\atree,\val\rangle\rrbracket & =\dpsem\llbracket\langle\atoms,\atree,\udpL\circ\val\rangle\rrbracket.
%         \label{eq:equiv1}
%     \end{align}
%     In particular, for $\val=\val_{1}$,
%     \begin{equation}
%         \udpL\udpsem\llbracket\langle\atoms,\atree,\val_{1}\rangle\rrbracket=\dpsem\llbracket\langle\atoms,\atree,\udpL\circ\val_{1}\rangle\rrbracket.
%         \label{eq:w1}
%     \end{equation}
%     Because $\val_{1}(a)\udpleq\val_{2}(a),$ from \cref{lem:dpsem-monotone},
%     \begin{equation}
%         \dpsem\llbracket\langle\atoms,\atree,\udpL\circ\val_{1}\rangle\rrbracket\dpleq\dpsem\llbracket\langle\atoms,\atree,\udpL\circ\val_{2}\rangle\rrbracket.
%         \label{eq:w2}
%     \end{equation}
%     From~\cref{eq:equiv1} again,
%     \begin{equation}
%         \dpsem\llbracket\langle\atoms,\atree,\udpL\circ\val_{2}\rangle\rrbracket=\udpL\udpsem\llbracket\langle\atoms,\atree,\val_{2}\rangle\rrbracket.
%         \label{eq:w3}
%     \end{equation}
%     From \cref{eq:w1},~\cref{eq:w2}, \cref{eq:w3} together,
%     \begin{equation}
%         \udpL\udpsem\llbracket\langle\atoms,\atree,\val_{1}\rangle\rrbracket\dpleq\udpL\udpsem\llbracket\langle\atoms,\atree,\val_{2}\rangle\rrbracket.
%     \end{equation}
%     Repeating the same reasoning for~$\udpU$, we have
%     \begin{equation}
%         \udpU\udpsem\llbracket\langle\atoms,\atree,\val_{2}\rangle\rrbracket\dpleq\udpU\udpsem\llbracket\langle\atoms,\atree,\val_{1}\rangle\rrbracket
%     \end{equation}
%     ,
%     \begin{equation}
%         \udpsem\llbracket\langle\atoms,\atree,\val_{1}\rangle\rrbracket\udpleq\udpsem\llbracket\langle\atoms,\atree,\val_{2}\rangle\rrbracket.
%     \end{equation}
% \end{proof}

% \vfill\pagebreak

% % \devel{
% \section{Software}

% \section{Source code}

% The implementation is available at the repository \url{http://github.com/AndreaCensi/mcdp/},
% in the branch ``uncertainty\_sep16''.

% \section{Virtual machine }

% A VMWare virtual machine is available to reproduce the experiments
% at the URL \url{https://www.dropbox.com/sh/nfpnfgjh9hpcgvh/AACVZfdVXxMoVqTYiHWaOwHAa?
%     dl=0}.

% To reproduce the figures, log in with user password ``mcdp''/``mcdp''.
% Then execute the following commands:

% \includepdf[pages={-}]{mcdp_icra_uncertainty_models.pdf}
% }
