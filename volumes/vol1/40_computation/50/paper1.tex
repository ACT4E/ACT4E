\paragraph{Abstract}
One of the challenges of modern engineering, and robotics in particular, is designing complex systems, composed of many subsystems, rigorously and with optimality guarantees.
This paper introduces a theory of co-design that describes ``design problems'', defined as tuples of ``functionality space'', ``implementation space'', and ``resources space'', together with a feasibility relation that relates the three spaces.
Design problems can be interconnected together to create ``co-design problems'', which describe possibly recursive co-design constraints among subsystems.
A co-design problem induces a family of optimization problems of the type ``find the minimal resources needed to implement a given functionality''; the solution is an antichain (Pareto front) of resources.
A special class of co-design problems are Monotone Co-Design Problems (MCDPs), for which functionality and resources are complete partial orders and the feasibility relation is monotone and Scott continuous.
The induced optimization problems are multi-objective, nonconvex, nondifferentiable, noncontinuous, and not even defined on continuous spaces; yet, there exists a complete solution.
The antichain of minimal resources can be characterized as a least fixed point, and it can be computed using Kleene's algorithm.
The computation needed to solve a co-design problem can be bounded by a function of a graph property that quantifies the interdependence of the subproblems.
These results make us much more optimistic about the problem of designing complex systems in a rigorous way.

\section{Introduction}
One of the great engineering challenge of this century is dealing with the design of ``complex'' systems.
A complex system is complex because its components cannot be decoupled; otherwise, it would be just a (simple) product of simple systems.
The \emph{design}
of a complex system is complicated because of the ``co-design constraints'', which are the constraints that one subsystem induces on another.
This paper is an attempt towards formalizing and systematically solving the problem of ``co-design'' of complex systems with recursive design constraints.

\subsection{Robotic systems as the prototype of complex systems}

Robotics is the prototypical example of a field that includes heterogeneous multi-domain co-design constraints.
The design of a robotic system involves the choice of physical components, such as the actuators, the sensors, the power supply, the computing units, the network links, \etc.
Not less important is the choice of the software components, including perception, planning, and control modules.
All these components induce co-design constraints on each other.
Each physical component has SWAP characteristics such as its shape (which must contained somewhere), weight (which adds to the payload), power (which needs to be provided by something else), excess heat (which must be dissipated somehow), \etc.
Analogously, the software components have similar co-design constraints.
For example, a planner needs a state estimate.
An estimator provides a state estimate, and requires the data from a sensor, which requires the presence of a sensor, which requires power.
Everything costs money to buy or develop or license.

What makes system design problems non trivial is that the constraints might be recursive.
This is a form of \emph{feedback} in the problem of design~(\cref{fig:intro}).
For example, a battery provides power, which is used by actuators to carry the payload.
A larger battery provides more power, but it also increases the payload, so more power is needed.
Extremely interesting trade-offs arise when considering constraints between the mechanical system and the embodied intelligence.
For control, typically a better state estimate saves energy in the execution, but requires better sensors (which increase the cost and the payload) or better computation (which increases the power consumption).

\begin{figure}[t]
    \begin{centering}
        \includegraphics[width=8.6cm]{gmcdptro_intro}
        \par
    \end{centering}
    \caption{A \emph{design problem }is a relation that relates the implementations available to the \F{functionality provided}
        and the \R{resources required}, both represented as partially ordered sets.
        A\emph{ co-design problem }is the interconnection of two or more design problems.
        An edge in a co-design diagram like in the figure represent a \emph{co-design constraint}: the resources required by the first design problem are a lower bound for the functionality to be provided by the second.
        The optimization problem to be solved is: find the solutions that are minimal in resources usage, given a lower bound on the functionality to be provided. }
    \label{fig:intro}
\end{figure}

\subsection{Contribution: A Principled Theory of Co-Design}

This paper describes a theory to deal with arbitrarily complex co-design problems in a principled way.
A~\emph{design problem} is defined as a tuple of functionality space, implementation space, and a resources space, plus the two maps that relate an implementation to functionality provided and resources required.
A design problem defines a family of optimization problems of the type ``find the minimal resources needed to implement a given functionality''.
A \emph{co-design problem} is an interconnection of design problems according to an arbitrary graph structure, including feedback connections.
Monotone Co-Design Problems (MCDPs) are the composition of design problems for which both functionality and resources are complete partial orders, and the relation between functionality implemented and resources needed is monotone (order-preserving) and \scottcontinuous.
The first main result in this paper (\cref{thm:CDP-monotone}) is that the class of MCDPs is closed with respect to interconnection.
The second main result (\cref{thm:CDP-solvig}) is that there exists a systematic procedure to solve an MCDP, assuming there is a procedure to solve the primitive design problems.
The solution of an MCDP\textemdash a Pareto front, or ``antichain'' of minimal resources\textemdash can be found by solving a least fixed point iteration in the space of antichains.
The complexity of this iteration depends on the structure of the co-design diagram.

This paper is a generalization of previous work~\cite{censi15monotone}, where the interconnection was limited to one cycle.
A conference version of this work appeared as~\cite{censi15same}.

\subsection{Outline}
\devel{
    \dots recalls necessary background about partial orders.
    \Cref{sec:Design-Problems} defines co-design problems. \cref{sec:Optimization}
    contains a brief statement of results. \cref{sec:threeoperators} describes composition operators for design problems. \cref{sec:Decomposition}
    shows how any interconnection of design problems can be described using three composition operators (series, parallel, feedback).
    \XXX
    describes the invariance of a monotonicity property that is preserved by the composition operators. \cref{sec:Solution-of-Monotone} describes solution algorithms for MCDPs. \cref{sec:Numerical-examples} shows numerical examples.
}

%\cref{sec:Discussion-of-related} discusses related work.

Uppercase ``$\Min$'' will denote the \emph{minimal} elements of a set.
The minimal elements are the elements that are not dominated by any other in the set.
Lowercase ``$\min$'' denotes\emph{ the least} element, an element that dominates all others, if it exists.
(If~$\min S$ exists, then~$\Min S=\{\min S\}$.)

The set of minimal elements of a set are an antichain, so~$\Min$
is a map from the power set $\powerset(\posA)$ to the antichains~$\antichains\posA$:
\begin{align*}
    \Min\colon\powerset(\posA) & \rightarrow\antichains\posA,                               \\
    S                          & \mapsto\{x\in S:\ (y\in S)\wedge(y\posleq x)\Imp(x=y)\ \}.
\end{align*}

$\Max$ and $\max$ are similarly defined.

\subsection{Order on antichains}

The upper closure operator ``$\upit$'' maps a subset of a poset to an upper set.
\begin{definition}[Upper closure]
    The operator~$\upit$ maps a subset to the smallest upper set that includes it:
    \begin{eqnarray*}
        \upit\colon\powerset(\posA) & \rightarrow & \uppersets\posA,\\
        S & \mapsto & \{y\in\posA:\exists\,x\in S:x\posleq y\}.
    \end{eqnarray*}
\end{definition}

\begin{figure}[h]
    \centering
    \includegraphics[scale=0.4]{gmcdp_antichains_upsets}
    \caption{}
    \label{fig:antichains_upsets}
\end{figure}
%\captionsideleft{\label{fig:antichains_upsets}}{\includegraphics[scale=0.4]{gmcdp_antichains_upsets}}

By using the upper closure operator, we can define an order on antichains using the order on the upper sets~(\cref{fig:antichains_upsets}).
\begin{lemma}
    \label{lem:antichains-are-poset}$\antichains\posA$ is a poset with the relation~$\posleq_{\antichains\posA}$ defined by
    \[
        A\posleq_{\antichains\posA}B\qquad\equiv\qquad\upit A\supseteq\upit B.
    \]
\end{lemma}
In the poset $\tup{\antichains\posA,\posleq_{\antichains\posA}}$, the top is the empty set:$\top_{\antichains\posA}=\emptyset.$ If a bottom for $\posA$ exists, then the bottom for~$\antichains\posA$
is the singleton containing only the bottom for~$\posA$: $\bot_{\antichains\posA}=\{\bot_{\posA}\}.$

Call ``Monotone Co-Design Problems'' (MCDPs) the set of CDPIs for which all subproblems respect the conditions in~\cref{def:DPI-monotone}.
I will show two main results:

1) A \textbf{modeling result} (\cref{thm:CDP-monotone}) says that the class of MCDPs is closed with respect to arbitrary interconnections.
Therefore, given a co-design diagram, such as the one in~\cref{fig:shipping}, if we know that each design problem is an MCDP, we can conclude that the diagram represents an MCDP as well.

2) An \textbf{algorithmic result }(\cref{thm:CDP-solvig}) says that the functionality-resources map~$\ftor$ for the entire MCDP has an explicit expression in terms of the maps~$\{\ftor_{\cdpin},\,\cdpin\in\cdpiN\}$ for the subproblems.
If there are cycles in the co-design diagram, the map~$\ftor$ involves the solution of a\emph{ least fixed point} equation in the space of antichains.
This equation can be solved using Kleene's algorithm to find the antichain containing all minimal solutions at the same time.
%
%
%\section{Conclusions}
%
%This paper described a mathematical theory of co-design, in which
%the primitive objects are design problems, defined axiomatically as
%relations between functionality, resources, and implementation. Monotone
%Co-Design Problems (MCDPs) are the interconnection of design problems
%whose functionality and resources are complete partial orders and
%the relation is \scottcontinuous. These were shown to be non-convex,
%non-differentiable, and not even defined on continuous spaces. Yet,
%MCDPs have a systematic solution procedure in the form of a least
%fixed point iteration, whose complexity depends on a measure of interdependence
%between design problems\textemdash it is easier to design a system
%composed of subsystems that are only loosely coupled. Based on this
%theory, it is possible to create modeling languages and optimization
%tools that allow the user to quickly define and solve multi-objective
%design problems in heterogeneous domains.

