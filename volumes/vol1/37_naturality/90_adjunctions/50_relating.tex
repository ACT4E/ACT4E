% !TEX root = chapter-standalone.tex

\section{Relating the two definitions}
\label{relate-adj-defs}

We start first with the ``hom-set definition'' of \SY{adjunction}, and show how to obtain the ``(co)unit definition''.
Given an \SY{adjunction}~$\funa \dashv \funb$ from a category \CatC to a category \CatD, we have, by \cref{def:adj-iso} a \SY{natural isomorphism}~$\adjtau$ with components
\begin{equation}
    \adjtau_{\Obja,\Objb} \colon \HomSet{\CatD}{\funa(\Obja)}{\Objb} \mto \HomSet{\CatC}{\Obja}{\funb(\Objb)}.
\end{equation}
From this data we can construct the unit and counit of the \SY{adjunction} as follows.

Given an object~$\Obja$ of~\CatC, we define
%
\begin{equation}
    \nunit_\Objc \colon \Obja \mto \funb(\funa(\Obja))
\end{equation}
%
to be the image under~$\adjtau_{\Obja, \funa(\Obja)}$ of~$\catidat{\funa(\Obja)} \setin \HomSet{\CatD}{\funa (\Obja)}{\funa(\Obja)}$.

Given an object~$\Objb$ of~\CatD, we define
\begin{equation}
    \ncounit_\Objb \colon \funa(\funb(\Objb)) \mto \Objb
\end{equation}
to the image under~$\adjtau_{\funb(\Objb), \Objb}^{-1}$ of~$\catidat{\funb(\Objb)} \setin \HomSet\CatD{\funb(\Objb)}{\funb(\Objb)}$.

\begin{exercise}
    \label{ex:eta-epsilon}

    Show that if we define~$\nunit$ and~$\ncounit$ in terms of their components as above, then they do indeed define natural transformations
    \begin{equation}
        \nunit\colon \funidat{\CatC} \nto \funb\fthen \funa
    \end{equation}
    and
    \begin{equation}
        \ncounit\colon \funb \fthen \funa \nto \funidat{\CatD}
    \end{equation}
    respectively.
    In other words, check the naturality conditions for~$\nunit$ and~$\ncounit$.
\end{exercise}
\todotextjira{145}{\alphubel: @JL: Solution of \cref{ex:eta-epsilon}.}

\begin{solution}
    \missingsolution
\end{solution}

\begin{exercise}
    \label{ex:eta-epsilon-triangle}
    Show that~$\nunit$ and~$\ncounit$, as defined above, satisfy the triangle identities stated in \cref{def:adj-counit}.
\end{exercise}
\todotextjira{146}{\alphubel: @JL: Solution of \cref{ex:eta-epsilon-triangle}.}

\begin{solution}
    \missingsolution
\end{solution}

Now let's start with the ``(co)unit definition'' of \SY{adjunction} and see how to obtain the ``hom-set definition''.

Given the unit~$\nunit$ and counit~$\ncounit$, we can construct the components~$\adjtau_{\Obja,\Objb}$ of the \SY{natural transformation}~$\adjtau$ as follows.
Given~$\mora \setin \Hom_{\CatD}(\funa(\Obja),\Objb)$, we define
\begin{equation}
    \adjtau_{\Obja,\Objb}(\mora) = \nunit_\Obja \mthen \funb(\mora).
\end{equation}
Similarly, given~$\morb \setin \Hom_{\CatC}(\Obja,\funb(\Objb))$, the inverse component is given by
\begin{equation}
    \adjtau_{\Obja,\Objb}^{-1}(\morb) = \funa(\morb) \mthen \ncounit_\Objb.
\end{equation}

\begin{exercise}
    \label{ex:tau}
    Show that~$\adjtau_{\Obja,\Objb}$ and~$\adjtau_{\Obja,\Objb}^{-1}$ are indeed functions which are inverses of each other.
\end{exercise}
\todotextjira{147}{\alphubel: @JL: Solution of \cref{ex:tau}.}

\begin{solution}
    \missingsolution
\end{solution}

\begin{exercise}
    \label{ex:tau2}
    Show that the functions~$\adjtau_{\Obja,\Objb}$ do assemble to a natural transformation
    \begin{equation}
        \adjtau  \colon \Hom_{\CatD}(\funa(-),- ) \nto \Hom_{\CatC}{-}{\funb(-)}
    \end{equation}
    between \SY{functors}~$\CatCop \Ctimes \CatD \fto \Set $.
\end{exercise}
\todotextjira{148}{\alphubel: @JL: Solution of \cref{ex:tau2}.}

\begin{solution}
    \missingsolution
\end{solution}
