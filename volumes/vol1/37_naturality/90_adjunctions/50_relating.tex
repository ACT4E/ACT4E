% !TEX root = chapter-standalone.tex

\section{Relating the two definitions}
\label{relate-adj-defs}

Let's start first with the ``hom-set definition'' of adjunction, and show how to obtain the ``(co)unit definition''.
Given an adjunction~$\funa \dashv \funb$ from a category \CatC to a category \CatD, we have, by \cref{def:adj-iso} a natural isomorphism~$\tau$ with components
\begin{equation*}
    \tau_{X,Y} \colon \HomSet{\CatD}{\funa(\Obja)}{\Objb} \mto \HomSet{\CatC}{\Obja}{\funb(\Objb)}.
\end{equation*}
From this data we can construct the unit and counit of the adjunction as follows.

Given an object~$\Obja$ of~\CatC, we define
\begin{equation*}
    \eta_\Objc \colon \Obja \to \funb(\funa(\Obja))
\end{equation*}
to be the image under~$\tau_{\Obja, \funa(\Obja)}$ of~$\catid_{\funa(\Obja)} \in \HomSet{\CatD}{\funa (\Obja)}{\funa(\Obja)}$.
\todotextjira{515}{Need to use macros rather than $\eta$, $\epsilon$.}
And given an object~$\Objb$ of~\CatD, we define
\begin{equation*}
    \epsilon_\Objb \colon \funa(\funb(\Objb)) \to \Objb
\end{equation*}
to the the image under~$\tau_{\funb(\Objb), \Objb}^{-1}$ of~$\catid_{\funb(\Objb)} \in \HomSet\CatD{\funb(\Objb)}{\funb(\Objb)}$.

\begin{exercise}
    \label{ex:eta-epsilon}

    Show that if we define~$\eta$ and~$\epsilon$ in terms of their components as above, then they do indeed define natural transformations
    \begin{equation*}
        \eta\colon \catid_{\CatC} \nto \funb\fthen \funa
    \end{equation*}
    and
    \begin{equation*}
        \epsilon\colon \funb \fthen \funa \nto \catid_{\CatD}
    \end{equation*}
    respectively.
    In other words, check the naturality conditions for~$\eta$ and~$\epsilon$.
\end{exercise}
\begin{solution}
    \missingsolution
    \todotextjira{145}{@J: Solution of \cref{ex:eta-epsilon}.}
\end{solution}

\begin{exercise}
    \label{ex:eta-epsilon-triangle}
    Show that~$\eta$ and~$\epsilon$, as defined above, satisfy the triangle identites stated in \cref{def:adj-counit}.
\end{exercise}
\begin{solution}
    \missingsolution
    \todotextjira{146}{@J: Solution of \cref{ex:eta-epsilon-triangle}.}
\end{solution}

Now let's start with the ``(co)unit definition'' of adjunction and see how to obtain the ``hom-set definition''.

Given the unit~$\eta$ and counit~$\epsilon$, we can construct the components~$\tau_{\Obja,\Objb}$ of the natural transformation~$\tau$ as follows.
Given~$\mora \in \Hom_{\CatD}(\funa(\Obja),\Objb)$, we define
\begin{equation*}
    \tau_{\Obja,\Objb}(\mora) = \eta_\Obja \mthen \funb(\mora).
\end{equation*}
Similarly, given~$\morb \in \Hom_{\CatC}(\Obja,\funb(\Objb))$, the inverse component is given by
\begin{equation*}
    \tau_{\Obja,\Objb}^{-1}(\morb) = \funa(\morb) \mthen \epsilon_\Objb.
\end{equation*}

\begin{exercise}
    \label{ex:tau}
    Show that~$\tau_{\Obja,\Objb}$ and~$\tau_{\Obja,\Objb}^{-1}$ are indeed functions which are inverses of each other.
\end{exercise}
\begin{solution}
    \missingsolution
    \todotextjira{147}{@J: Solution of \cref{ex:tau}.}
\end{solution}

\begin{exercise}
    \label{ex:tau2}
    Show that the functions~$\tau_{\Obja,\Objb}$ do assemble to a natural transformation
    \begin{equation*}
        \tau  \colon \Hom_{\CatD}(\funa(-) ,- ) \nto \Hom_{\CatC}{-}{\funb(-)}
    \end{equation*}
    between functors~$\CatC\op \Ctimes \CatD \fto \Set $.
\end{exercise}
\begin{solution}
    \missingsolution
    \todotextjira{148}{@J: Solution of \cref{ex:tau2}.}
\end{solution}
