% !TEX root = chapter-standalone.tex

% \section{The duality of design}

    \section{Formal concept analysis}
    \todojira{143}{\bernina:
        An example of a Galois connection between functionalities and resources could be treated here to motivate the \SY{adjunction} discussion}

Formal concept analysis (FCA) is a mathematical theory which formalizes the relationships, and in particular hierarchies, that appear when we consider a set of \emph{things} -- these are called \emph{objects} in FCA -- together with a set of \emph{attributes} that these things may or may not have. (We will use the word ``things'' instead of ``objects'', because we are already using the word ``objects'' in the category-theory sense.) 

The most basic set-up for formal concept analysis is to start with a triple $\tup{\things, \attributes, \incidence}$, where $\things$ is a set of things (``G'' stands for the german word ``Gegenst\"ande''), $\attributes$ is a set of attributes (``M'' stands for the german word ``Merkmale''), and $\incidence \subseteq \things \cartprod \attributes$ is a relation that encodes which objects are associated with which attributes (``I'' stands for the german word ``Inzidenz''). The triple $\tup{\things, \attributes, \incidence}$ is called a \emph{formal context}.

Here is a (very simplified) example in the context of ``private means of transportation''. We consider the set of things to be the following means of transportation
\begin{equation}\label{eq:fca-example-transport-things}
\things = \makeset{ \text{classic car}, \text{hybrid car}, \text{electric car}, \\
\text{classic bike}, \text{hybrid bike}, \text{electric scooter}, \text{skateboard}}
\end{equation}
and we consider the set of attributes 
\begin{equation}\label{eq:fca-example-transport-attributes}
\attributes = \makeset{\text{fast}, \text{electric}, \text{gas}, \text{muscle}, \text{cheap}},
\end{equation}
which describe aspects such how the means of transportation are powered, their relative cost, or if they can go fast enough to move on a highway, for example. 
%To make notation more economical, we'll abbreviate the elements of these sets as
%\begin{equation}
%\things = \makeset{ \text{CC}, \text{HC}, \text{EC}, \\
%\text{CB}, \text{HB}, \text{ES}, \text{SB}}
%\end{equation}
%and 
%\begin{equation}
%\attributes = \makeset{\text{F}, \text{E}, \text{G}, \text{H}, \text{C}}.
%\end{equation}
We define the relation $\incidence \subseteq \things \cartprod \attributes$ via the following table
\begin{center}
\begin{tabular}{c|c|c|c|c|c}\label{fca-table-private-transportation}
 & fast & electric & gas & muscle & cheap \\
\hline 
classic car & $\times$ & & $\times$ & & \\
\hline
hybrid car & $\times$ &$\times$ &$\times$ &  & \\
 \hline
electric car & $\times$ &$\times$ & & &  \\
 \hline
classic bike & & & &$\times$ & $\times$ \\
 \hline
hybrid bike & &$\times$ & &$\times$ & \\
 \hline
electric scooter & &$\times$ & & &$\times$ \\
 \hline
skateboard & & & &$\times$ &$\times$ \\
\hline
\end{tabular} 
\end{center}
where a cross ``$\times$'' indicates when a thing and an attribute are associated with each other.  


\subsection{Induced monotone maps}

For each element $\ela$ of $\things$, we can consider the set $\incidence_\sharp(\makeset{\ela})$ of attributes that are associated with $\ela$. This corresponds to reading off in the the above table where there are crosses ``$\times$'' in the row labeled by $\ela$. For example
\begin{equation}\label{eq:fca-example-pushforward-1}
\incidence_\sharp(\makeset{\text{hybrid car}}) = \makeset{\text{fast}, \text{electric}, \text{gas}}. 
\end{equation}


More generally, given a subset $\setA \subseteq \things$, can can consider the largest set $\incidence_\sharp(\setA)$ of attributes which all elements of $\setA$ have in common. For example 
\begin{equation}\label{eq:fca-example-pushforward-2}
\incidence_\sharp(\makeset{\text{hybrid car}, \text{electric car}}) = \makeset{\text{fast}, \text{electric}}. 
\end{equation}
Here, ``$\text{gas}$'' is not an element of $\incidence_\sharp(\makeset{\text{hybrid car}, \text{electric car}})$ because ``$\text{gas}$'' is associated with $\text{hybrid car}$ but not with $\text{electric car}$.  

Similarly, 
\begin{equation}\label{eq:fca-example-pushforward-3}
\incidence_\sharp(\makeset{\text{classic car}, \text{hybrid car}, \text{electric car}}) = \makeset{\text{fast}}. 
\end{equation}
Also we have, for example, 
\begin{equation}\label{eq:fca-example-pushforward-4}
\incidence_\sharp(\makeset{\text{classic car}, \text{classic bike}}) = \emptyset
\end{equation}
because ``\text{classic car}'' and ``\text{classic bike}'' have \emph{no} attributes from the set $\attributes$ in common. In general we have 
\begin{equation}\label{eq:fca-example-pushforward-5}
\incidence_\sharp(\setA) = \bigcap_{\ela \setin \setA} \incidence_\sharp(\makeset{\ela}).
\end{equation}
and the operation ``$\incidence_\sharp$'' defines a function 
\begin{equation}\label{eq:fca-example-pushforward-6}
\incidence_\sharp \colon \powerset \things \mto \powerset \attributes.
\end{equation}

Observe that the larger $\setA$ is, the smaller $\incidence_\sharp(\setA)$ will be. Formulated more mathematically, we have
\begin{equation}\label{eq:fca-example-pushforward-7}
\setA \subseteq \setA' \quad \implies \quad \incidence_\sharp(\setA) \supseteq \incidence_\sharp(\setA').
\end{equation}
Another way of saying this is to say that $\incidence_\sharp$ is a monotone map of posets 
\begin{equation}\label{eq:fca-example-pushforward-8}
\incidence_\sharp \colon \tup{\powerset \things, \subseteq} \mto \tup{\powerset \attributes, \supseteq}. 
\end{equation}
Or, equivalently, we can say that $\incidence_\sharp$ is a monotone map
\begin{equation}\label{eq:fca-example-pushforward-9}
\incidence_\sharp \colon \tup{\powerset \things, \subseteq} \mto \tup{\powerset \attributes, \subseteq}\op. 
\end{equation}

Note that we can also define a similar map in the other direction: there is a function
\begin{equation}\label{eq:fca-example-pullback-1}
\incidence_\flat \colon \powerset \attributes \mto \powerset \things
\end{equation}
defined such that for any subset $\setB \subseteq \attributes$, the set $\incidence_\flat(\setB)$ is the largest set of elements of $\things$ such that the attributes in $\setB$ apply to all of the elements of $\incidence_\flat(\setB)$. 
For example, 
\begin{equation}
\incidence_\flat(\makeset{\text{muscle}, \text{cheap}}) = \makeset{\text{classic bike}, \text{skateboard}}.
\end{equation}
The map $\incidence_\flat$  is also order-reversing with respect to inclusion of sets: if we start with a larger set of attributes, then set of things that these all apply to will be smaller. Thus we have a monotone map 
\begin{equation}\label{eq:fca-example-pullback-2}
\incidence_\flat \colon \tup{\powerset \attributes, \subseteq}\op \mto \tup{\powerset \things, \subseteq}.
\end{equation}
We will also want to use its opposite, the monotone map
\begin{equation}\label{eq:fca-example-pullback-3}
\incidence_\flat\op \colon \tup{\powerset \attributes, \subseteq} \mto \tup{\powerset \things, \subseteq}\op.
\end{equation}

In the following, we will try to keep track of when there is an superscript ``$( - )\op $''; however sometimes it will be convenient to use the notations $\incidence_\sharp$ and $\incidence_\flat$ both for these maps \emph{and} their opposites (in particular, on the level of objects they are the same function). 

A key observation is that $\incidence_\sharp$ and $\incidence_\flat$ are ``complementary'' in the following sense. For any $\setA \subseteq \things$ and any $\setB \subseteq \attributes$ we have 
\begin{equation}\label{eq:concept-adjunction}
\incidence_\sharp(\setA) \supseteq \setB \quad \Leftrightarrow \quad \setA \subseteq \incidence_\flat(\setB).
\end{equation}
This equivalence formalizes the (nearly tautological-seeming) statement that a set $\setB$ is contained in the largest set of attributes which apply to all members of $\setA$ (meaning: the attributes $\setB$ apply to all elements of $\setA$) if, and only if, $\setA$ is contained in the largest set of things to which all attributes $\setB$ apply (again meaning: the attributes $\setB$ apply to all elements of $\setA$). 

Despite seeming tautological, we can use the equivalence \cref{eq:concept-adjunction} to make non-trivial observations. One consequence of \cref{eq:concept-adjunction} is that the monotone maps 
\begin{equation}\label{eq:fca-closure-operator-things}
\incidence_\sharp \mthen \incidence_\flat \colon \tup{\powerset \things, \subseteq} \mto \tup{\powerset \things, \subseteq}
\end{equation}
and 
\begin{equation}\label{eq:fca-closure-operator-attributes}
\incidence_\flat \mthen \incidence_\sharp \colon \tup{\powerset \attributes, \subseteq}\op \mto \tup{\powerset \attributes, \subseteq}\op.
\end{equation}
are examples of what are called a \emph{closure operator} and \emph{interior operator}, respectively.


\subsection{Closure and interior operators}

\begin{definition}\label{def:closure-operator}
Let $\posA = \tup{\posAset, \posleq}$ be poset. A \emph{closure operator} on $\posA$ is

\constit
\begin{enumerate}
\item a monotone map $\mora \colon \posA \mto \posA$; 
\end{enumerate}

\condit
\begin{enumerate}
\item \text{Extensivitity:} \  $\ela \posleq \mora(\ela) \quad \forall \ela \setin \posAset$;
\item \text{Idempotence:} \ $\mora (\mora(\ela)) =  \mora(\ela) \quad \forall \ela \setin \posAset$. 
\end{enumerate}
\end{definition}

\begin{definition}\label{def:interior-operator}
Let $\posA = \tup{\posAset, \posleq}$ be poset. An \emph{interior operator} on $\posA$ is

\constit
\begin{enumerate}
\item a monotone map $\mora \colon \posA \mto \posA$; 
\end{enumerate}

\condit
\begin{enumerate}
\item \text{Intensivitity:} \  $\mora(\ela) \posleq \ela \quad \forall \ela \setin \posAset$;
\item \text{Idempotence:} \ $\mora (\mora(\ela)) =  \mora(\ela) \quad \forall \ela \setin \posAset$. 
\end{enumerate}
\end{definition}

The notions of closure and interior operator are dual in the following sense. 

\begin{lemma}
If $\mora \colon \posA \mto \posA$ is a closure (interior) operator, then $\mora\op \colon \posA\op \mto \posA\op$ is an interior (closure) operator. 
\end{lemma}

In this section, for simplicity, we will work mainly in terms of closure operators. 

\begin{lemma}\label{lem:cfa-closure-operators}
The monotone maps 
\begin{equation}
\incidence_\sharp \mthen \incidence_\flat \colon \tup{\powerset \things, \subseteq} \mto \tup{\powerset \things, \subseteq}
\end{equation}
and 
\begin{equation}
\incidence_\flat\op \mthen \incidence_\sharp\op \colon \tup{\powerset \attributes, \subseteq} \mto \tup{\powerset \attributes, \subseteq}.
\end{equation}
are closure operators. 
\end{lemma}

\begin{proof}
Let's check that $\incidence_\sharp \mthen \incidence_\flat \colon \tup{\powerset \things, \subseteq} \mto \tup{\powerset \things, \subseteq}$ is a closure operator, using \cref{eq:concept-adjunction}. We omit the proof for $\incidence_\flat\op \mthen \incidence_\sharp\op$, which may be done analogously. 


To show the first condition in the definition of closure operator, fix a set of things $\setA \subseteq \things$. In the situation of \cref{eq:concept-adjunction}, choose $\setB = \incidence_\sharp(\setA)$. Since $\incidence_\sharp(\setA) \supseteq \incidence_\sharp(\setA)$ is true, \cref{eq:concept-adjunction} implies that $\setA \subseteq \incidence_\flat(\incidence_\sharp(\setA))  = (\incidence_\sharp \mthen \incidence_\flat)(\setA)$. 

Now let's consider the second condition. Applying the monotone map $\incidence_\sharp \mthen \incidence_\flat$ to the relation $\setA \subseteq (\incidence_\sharp \mthen \incidence_\flat)(\setA)$, we have
\begin{equation}
(\incidence_\sharp \mthen \incidence_\flat)(\setA)  \subseteq  (\incidence_\sharp \mthen \incidence_\flat \mthen \incidence_\sharp \mthen \incidence_\flat)(\setA).
\end{equation}
Thus we are finished when we show the inclusion in the other direction. By the first condition, we know that 
\begin{equation}
\incidence_\sharp(\setA) \subseteq \incidence_\sharp( \incidence_\flat (\incidence_\sharp(\setA))) = (\incidence_\sharp \mthen \incidence_\flat \mthen \incidence_\sharp)(\setA).
\end{equation}
Applying the order-reversing map $\incidence_\flat$ to both sides of this inclusion then gives 
\begin{equation}
(\incidence_\sharp \mthen \incidence_\flat)(\setA) \supseteq (\incidence_\sharp \mthen \incidence_\flat \mthen \incidence_\sharp \mthen \incidence_\flat)(\setA)
\end{equation}
as desired. 
\end{proof}

Closure and interior operators arise in various contexts in mathematics. Often we are interested in the elements which are in the images of these operators. These are called \emph{closed elements} and \emph{open elements}, respectively. We will use the term \emph{fixed-points} to refer to both of these cases without needing to specify whether we are working with a closure or an interior operator.  

\begin{definition}\label{def:closure-closed-elements}
Let $\posA = \tup{\posAset, \posleq}$ be a poset, $\mora \colon \posA \mto \posA$ a monotone map, and $\ela \setin \posAset$ an arbitrary element of $\posA$.

If $\mora$ is a closure (interior) operator, then $\mora(\ela) \setin \posAset$ is called the \emph{closure} (\emph{interior}) of $\ela$, and $\ela$ is called \emph{closed} (\emph{open}) if $\mora(\ela) = \ela$. In both cases, when $\mora(\ela) = \ela$, we say that $\ela$ is a fixed-point of $\mora$. 

The set of fixed-points of $\mora$ will be denoted $\posAset_\mora$, or by $\posAset_{\text{fix}}$ when the operator $\mora$ in question is clear. 
\end{definition}

\begin{remark}\label{rem:closed-elements-are-image}
Note that if $\mora \colon \posA \mto \posA$ is a closure or interior operator, then the set of fixed points $\posAset_{\text{fix}}$ coincides with the image of $\mora$.  

On the one hand, any element $\elb$ of the form $\elb = \mora(\ela)$ is a fixed-point, because
\begin{equation}
\mora(\mora(\ela)) = \mora(\ela)
\end{equation}
by the idempotence property. 

On the other hand, if $\elb \setin \posAset$ is a fixed point, then by definition $\elb = \mora(\elb)$ is in the image of $\mora$.
\end{remark}

\begin{remark}\label{rem:closed-elements-are-image}
If $\ela$ is a fixed-point of a closure/interior operator $\mora \colon \posA \mto \posA$, then $\ela$ is also a fixed-point of $\mora\op \colon \posA\op \mto \posA\op$. 
\end{remark}


Returning now to formal concept analysis, let's look at closures and closed elements for the closure operators 
$$\incidence_\sharp \mthen \incidence_\flat \colon \tup{\powerset \things, \subseteq} \mto \tup{\powerset \things, \subseteq}$$  
and 
$$\incidence_\flat\op \mthen \incidence_\sharp\op \colon \tup{\powerset \attributes, \subseteq} \mto \tup{\powerset \attributes, \subseteq} $$ 
in terms of our simple example. 

For example, let $\setA = \makeset{\text{classic car}, \text{electric car}}$. Then 
\begin{equation}
\incidence_\sharp(\setA) = \makeset{\text{fast}}
\end{equation}
and 
\begin{equation}
\incidence_\flat(\incidence_\sharp(\setA)) = \incidence_\flat(\makeset{\text{fast}}) = \makeset{\text{classic car}, \text{hybrid car}, \text{electric car}}.
\end{equation}
So $\setA = \makeset{\text{classic car}, \text{electric car}}$ is not a closed element of $\tup{\powerset \things, \subseteq}$. Its closure contains the element ``$\text{hybrid car}$'' which is not in $\setA$.

Or consider $\setB = \makeset{\text{electric}, \text{muscle}}$. Then
\begin{equation}
\incidence_\flat(\setB) = \makeset{\text{hybrid bike}}
\end{equation}
and 
\begin{equation}
\incidence_\sharp(\incidence_\flat(\setB)) = \incidence_\sharp(\makeset{\text{hybrid bike}}) = \makeset{\text{electric}, \text{muscle}} = \setB.
\end{equation}
We find here that $\makeset{\text{hybrid bike}}$ \emph{is} a closed element of $\tup{\powerset \attributes, \subseteq}$. 

In general, given a set $\setA$ of things, its closure $(\incidence_\sharp \mthen \incidence_\flat)(\setA)$ is the largest set of things that share the attributes in $\incidence_\sharp(\setA)$. And $\incidence_\sharp(\setA)$ is the largest set of attributes shared by $\setA$. Thus we may say:

 ``\emph{$(\incidence_\sharp \mthen \incidence_\flat)(\setA)$ is the maximal set of things that share the same attributes as are shared by $\setA$}.'' 
 
Or, put another way, taking the closure of $\setA$ is a way of enlarging $\setA$ without decreasing the set of associated shared attributes. Closing $\setA$ is adding those things to $\setA$ that come ``for free'' in the sense that, by adding them, we are not losing shared attributes. 

A similar point of view of course also applies to closing sets of attributes with respect to the closure operator $\incidence_\flat \mthen \incidence_\sharp$.

\subsection{Concepts}

\begin{definition}\label{def:cfa-concept}
Let $\tup{\things, \attributes, \incidence}$ be a formal context in the sense of formal concept analysis. A \emph{concept} is a pair $\tup{\setA, \setB} \setin \powerset\things \cartprod \powerset\attributes$ such that 
\begin{equation}\label{eq:def-concept-equations}
\incidence_\sharp(\setA) = \setB \quad \text{and} \quad \incidence_\flat(\setB) = \setA.
\end{equation}
For a concept $\tup{\setA, \setB}$, the set $\setA$ of things is called the \emph{extent} of the concept, and the set $\setB$ of attributes is called the \emph{intent} of the concept. 

We denote the set of all concepts for the context $\tup{\things, \attributes, \incidence}$ by $\mathcal{B}\tup{\things, \attributes, \incidence}$. (Here ``B'' comes from the German term ``Begriffe''.) 
\end{definition}

The set $\mathcal{B}\tup{\things, \attributes, \incidence}$ of concepts for a formal context has a natural partial order structure. We set
\begin{equation}\label{eq:partial-order-concepts}
\tup{\setA_1, \setB_1} \posleq \tup{\setA_2, \setB_2}
\end{equation}
if $\setA_1 \subseteq \setA_2$ and $\setB_1 \supseteq \setB_2$. (In fact, by the definition of a concept, if one of the latter inclusions holds, then so must the other, so we only need to require one of them.) When \cref{eq:partial-order-concepts} holds, we say that $\tup{\setA_1, \setB_1}$ is a \emph{subconcept} of $\tup{\setA_2, \setB_2}$


\begin{lemma}\label{lem:extents-and-intents-are-closed}
If $\tup{\setA, \setB}$ is a concept, then $\setA$ and $\setB$ are closed elements of $\tup{\powerset \things, \subseteq}$ and $\tup{\powerset \attributes, \subseteq}$, respectively. 
\end{lemma}

\begin{proof}
For $\setA$ we have 
\begin{equation}
\incidence_\flat(\incidence_\sharp(\setA)) = \incidence_\flat(\setB) = \setA
\end{equation}
using both the equations \cref{eq:def-concept-equations}.
The case for $\setB$ is analogous. 
\end{proof}

\begin{lemma}\label{lem:closed-elements-define-concepts}
If $\setA \in \tup{\powerset \things, \subseteq}$ is closed, then $\incidence_\sharp(\setA)$ is closed and $\tup{\setA, \incidence_\sharp(\setA)}$ is a concept. 

Similarly, if $\setB \in \tup{\powerset \attributes, \subseteq}$ is closed, then $\incidence_\flat(\setB)$ is closed and $\tup{\incidence_\flat(\setB), \setB)}$ is a concept.
\end{lemma}

\begin{proof}
We show only the first statement. We have 
\begin{equation}
(\incidence_\flat \mthen \incidence_\sharp) (\incidence_\sharp(\setA)) = \incidence_\sharp( \incidence_\flat (\incidence_\sharp(\setA))) = \incidence_\sharp( (\incidence_\sharp \mthen \incidence_\sharp)(\setA)) = \incidence_\sharp(\setA), 
\end{equation}
so $\incidence_\sharp(\setA)$ is closed. That $\tup{\setA, \incidence_\flat(\setA)}$ is a concept is clear, since $\incidence_\flat(\incidence_\sharp(\setA)) = \setA$. 
\end{proof}




\begin{lemma}
The posets of fixed points $\tup{\powerset \things_{\text{fix}}, \subseteq}$ and $\tup{\powerset \attributes_{\text{fix}}, \subseteq}\op$ are isomorphic via the restrictions of $\incidence_\sharp$ and $\incidence_\flat$, and each is isomorphic to the poset $\tup{\mathcal{B}\tup{\things, \attributes, \incidence}, \posleq}$ via its projections onto its first and second factors, respectively.
\end{lemma}

\begin{proof}
This follows from \cref{lem:extents-and-intents-are-closed}, \cref{lem:closed-elements-define-concepts}, and the definition of the ordering on $\tup{\mathcal{B}\tup{\things, \attributes, \incidence}, \posleq}$. 
\end{proof}

\section{Galois connections}\label{sec:galois-connections}

\todotextjira{405}{\bernina: @Gioele: GZ: Adjust given new flow.
    We don't know what this is yet.}

\begin{ctdefinition}[Monotone Galois Connection]\label{def:monotone-galois-connection}
    A \maindef{Galois connection} between \SY{posets}~\posA and~\posB is a pair of \SY{monotone maps}
    $\mapa\colon \posA\to \posB$ and~$\mapb\colon \posB\to \posA$ such that for all~$\posAel\setin \posA$,~$\posBel\setin \posB$:
    \begin{equation}
        \prfdoubleperiod{\mapa(\posAel) \posBleq \posBel}{\posAel \posAleq \mapb(\posBel)}
        %\mapa(\posAel) \posBleq \posBel \quad \Leftrightarrow \quad \posAel \posAleq \mapb(\posBel).
    \end{equation}
  In this case $\mora$ is called the \emph{left adjoint} and $\morb$ is called the \emph{right adjoint}. We use the short-hand notation $\mora \dashv \morb$ to say that $\mora$ and $\morb$ form a Galois connection, or we draw a globular diagram like so:  
    \begin{equation}
    \middlesag{globular_galois_connection}
\end{equation}
\end{ctdefinition}

\begin{lemma}\label{lem:alternative-def-Galois-connection}
Monotone maps $\mapa\colon \posA\to \posB$ and~$\mapb\colon \posB\to \posA$ form a Galois connection if and only if the following hold: 
\begin{enumerate}
\item $\posAel\posAleq \mapb(\mapa(\posAel)) \quad \forall \posAel\setin \posA$;
\item $\mapa(\mapb(\posBel)) \posBleq \posBel \quad \forall \posBel\setin \posB$. 
\end{enumerate}
\end{lemma}

  This is equivalent to ask, for all~$\posAel\setin \posA$,~$\posBel\setin \posB$, that:
    \begin{equation}
        (\posAel\posAleq \mapb(\mapa(\posAel)))
        \booland (\posBel\posBleq \mapa(\mapb(\posBel))).
    \end{equation}

\begin{ctdefinition}[Antitone Galois Connection]\label{def:antitone-galois-connection}
    An \maindef{antitone Galois connection} between~\posA and~\posB is a pair of \SY{antitone maps} $\mapa\colon \posA\to \posB$ and $\mapb \colon \posB\to \posA$ such that for all~$\posAel\setin \posA$, $\posBel\setin \posB$:
    \begin{equation}
        \prfdoubleperiod{\posBel \posBleq \mapa(\posAel)}{\posAel \posAleq \mapb(\posBel)}
        %\posBel \posBleq \mapa(\posAel) \quad \Leftrightarrow \quad \posAel \posAleq \mapb(\posBel).
    \end{equation}
\end{ctdefinition}

\begin{remark}\label{rem:not-focusing-on-antitone-Galois-connections}
The underlying function of an antitone map $\mapa\colon \posA\to \posB$ defines a monoton map $\mapa\colon \posA\to \posB\op$ (or a monotone map $\mapa \colon \posA \op \to \posB$). Every antitone Galois connection $\mapa \colon \posA \to \posB$ and~$\mapb \colon \posB \to \posA$ defines a Galois connection $\mapa \colon \posA \to \posB\op$ and~$\mapb \colon \posB\op \to \posA$. 
\end{remark}

Because of the above remark, and because we prefer to work with monotone maps (since they are morphisms of posets), we will not focus on antitone Galois connections. However, it is useful to be aware of this definition, since it is sometimes used in the literature and sometimes more natural in the context of certain examples. We do include the follow alternative characterization of antitone Galois connections, analogous to the one above for Galois connections. 

\begin{lemma}\label{lem:alternative-def-antitone-Galois-connection}
Antitone maps $\mapa\colon \posA\to \posB$ and~$\mapb\colon \posB\to \posA$ form an antitone Galois connection if and only if the following hold: 
\begin{enumerate}
\item $\posAel\posAleq \mapb(\mapa(\posAel)) \quad \forall \posAel\setin \posA$;
\item $\posBel\posBleq \mapa(\mapb(\posBel)) \quad \forall \posBel\setin \posB$. 
\end{enumerate}
\end{lemma}



    
\begin{lemma}\label{lem:closure-interior-operators-from-Galois-connection}
If $\mapa \colon \posA \to \posB$ and~$\mapb \colon \posB \to \posA$ form a Galois connection, then 
\begin{equation}\label{eq:closure-from-Galois}
\mapa \mthen \mapb \colon \posA \to \posA
\end{equation}
is a closure operator and
\begin{equation}\label{eq:interior-from-Galois}
\mapb \mthen \mapa \colon \posB \to \posB
\end{equation}
is an interior operator. 
\end{lemma}    
    
    
   





\devel{
\section{Galois connections and co-design}\label{sec:galois-connections-codesign}


Consider a design problem~$d\colon \posAop \Ptimes \posB \toinPos \Bool$.
We can define the maps that work on single functionality and resources:
%
\begin{equation}
    \begin{aligned}
        \theta\colon \posA & \to \posUB \\
        \posAel            & \mapsto \makeset{\posBel\setin \posB \colon d(\posAopel, \posBel) },
    \end{aligned}
\end{equation}
%
\begin{equation}
    \begin{aligned}
        \psi\colon \posB & \to \posLA \\
        \posBel          & \mapsto \makeset{\posAel\setin \posA \colon d(\posAopel, \posBel) }.
    \end{aligned}
\end{equation}
We can define the maps that work on multiple functionality and resources:
\begin{equation}
    \label{eq:galoisalfa}
    \begin{aligned}
        \alpha\colon \posLA & \to \posUB \\
        S                   & \mapsto \makeset{\posBel\setin \posB \colon \exists \posAel\setin S\colon \ d(\posAel,\posBel)},
    \end{aligned}
\end{equation}
Alternatively, we can write
\begin{equation}
    \begin{aligned}
        \alpha \colon \posLA & \to \posUB \\
        S                    & \mapsto \bigsetunion_{\posAel\setin S} \theta(\posAel).
    \end{aligned}
\end{equation}
%
\begin{equation}
    \label{eq:galoisbeta}
    \begin{aligned}
        \beta\colon \posUB & \to \posLA \\
        T                  & \mapsto \makeset{\posAel\setin \posA  \colon \exists \posBel\setin T\colon d(\posAel,\posBel)},
    \end{aligned}
\end{equation}
%
\noindent Alternatively, we can write
\begin{equation}
    \begin{aligned}
        \beta \colon \posUB & \to \posLA \\
        T                   & \mapsto \bigsetunion_{\posBel\setin T} \psi(\posBel).
    \end{aligned}
\end{equation}
%
\begin{equation}
    \label{eq:galoisdelta}
    \begin{aligned}
        \delta \colon \posLA & \to \posUB \\
        S                    & \mapsto \makeset{\posBel\setin \posB \colon \forall {\posAel\setin S}\colon d(\posAel,\posBel)},
    \end{aligned}
\end{equation}
Alternatively, we can write
\begin{equation}
    \begin{aligned}
        \delta\colon \posLA & \to \posUB \\
        S                   & \mapsto \bigsetintersection_{\posAel\setin S}\theta(\posAel).
    \end{aligned}
\end{equation}
%
\begin{equation}
    \label{eq:galoisgamma}
    \begin{aligned}
        \gamma \colon \posUB & \to \posLA \\
        T                    & \mapsto \makeset{\posAel\setin \posA \colon \forall {\posBel\setin T}\colon d(\posAel,\posBel)},
    \end{aligned}
\end{equation}
Alternatively, we can write
\begin{equation}
    \begin{aligned}
        \delta\colon \posUB & \to \posLA \\
        T                   & \mapsto \bigsetintersection_{\posBel\setin T}\psi(\posBel).
    \end{aligned}
\end{equation}
%
Properties of these maps are reported in \cref{tab:galoisproperties}.

\begin{table*}[h!]
    \centering
    \begin{tabular}{c|l|l|c|c|c|c|c}
        $\star$  & X                                   & Y                    & $\star(\bot)$                             & $\star(\postop)$                           & $A\posleqof{X}
        B$       & $\star(A\vee_X B)$                  & $\star(A\wedge_X B)$ \\
        \hline
        $\alpha$ & $\posLA$                            & $\posUB$             & $\alpha(\Emptyset)=\Emptyset$             & $\alpha(\posA)\geq_{\posUB} \alpha(\cdot)$
                 & $\alpha(A)\geq_{\posUB} \alpha(B)$
                 & $\alpha(A)\vee_{\posLA}\alpha(B)$
                 & $\alpha(A)\wedge_{\posLA}\alpha(B)$ \\
        \hline
        $\beta$  & $\posUB$                            & $\posLA$             & $\beta(\posB)\geq_{\posLA}\beta(\cdot)$   & $\beta(\Emptyset)=\Emptyset$
                 & $\beta(A)\geq_{\posLA} \beta(B)$
                 & $\beta(A)\vee_{\posLA}\beta(B)$
                 & $\beta(A)\wedge_{\posLA}\beta(B)$ \\
        \hline
        $\delta$ & $\posLA$                            & $\posUB$             & $\delta(\Emptyset)=\posB$                 & $\delta(\posA)\geq_{\posUB}\delta(\cdot)$  & $\delta(A)\leq_{\posUB} \delta(B)$
                 & $\delta(A)\wedge_{\posUB}\delta(B)$
                 & $\delta(A)\vee_{\posUB}\delta(B)$ \\
        \hline
        $\gamma$ & $\posUB$                            & $\posLA$             & $\gamma(\posB)\leq_{\posLA}\gamma(\cdot)$ & $\gamma(\Emptyset)=\posA$                  & $\gamma(A) \leq_{\posLA} \gamma(B)$
                 & $\gamma(A)\wedge_{\posLA}\gamma(B)$
                 & $\gamma(A)\vee_{\posLA}\gamma(B)$
    \end{tabular}
    \caption{Properties of $\alpha,\beta,\delta,\gamma$}
    \label{tab:galoisproperties}
\end{table*}

\begin{lemma}
    \label{lem:deltagammamonotone}
    $\delta$ and $\gamma$ are \SY{monotone maps}.
\end{lemma}
\begin{proof}
    We first prove that~$\delta$ is a \SY{monotone map}.
    Given~$A,B\setin \posLA$ with~$A\setsubseteq B$, we have
    \begin{equation}
        \begin{aligned}
            \delta(A) & =\makeset{\posBel\setin \posB\colon \forall \posAel \setin A\colon d(\posAel,\posBel)} \\
                      & \setsupseteq \makeset{\posBel\setin \posB\colon \forall \posAel\setin B \colon d(\posAel,\posBel)} \\
                      & =\delta(B),
        \end{aligned}
    \end{equation}
    meaning that~$A\posleqof{\posLA} B \Imp \delta(A)\posleqof{\posUB} \delta(B)$.
    We now prove that~$\gamma$ is a \SY{monotone map}.
    Given~$C,D\setin \posUB$, with~$C\setsupseteq D$, we have
    \begin{equation}
        \begin{aligned}
            \gamma(C) & =\makeset{\posAel\setin \posA\colon \forall \posBel \setin C\colon d(\posAel,\posBel)} \\
                      & \setsubseteq \makeset{\posAel\setin \posA\colon \forall \posBel \setin D \colon d(\posAel,\posBel)} \\
                      & =\gamma(D),
        \end{aligned}
    \end{equation}
    meaning that~$C\posleqof{\posUB} D \Imp \gamma(C)\posleqof{\posLA} \gamma(D)$.
\end{proof}

\begin{lemma}
    \label{lem:alfabetaantitone}
    $\alpha$ and $\beta$ are \SY{antitone maps}.
\end{lemma}
\begin{proof}
    We first prove that~$\alpha$ is an \SY{antitone map}.
    Given~$A,B\setin \posLA$, with~$A\setsubseteq B$, we have
    \begin{equation}
        \begin{aligned}
            \alpha(A) & =\makeset{ \posBel\setin \posB \colon \exists \posAel\setin A\colon d(\posAel,\posBel)} \\
                      & \setsubseteq \makeset{\posBel\setin \posB \colon \exists \posAel\setin B\colon d(\posAel,\posBel)} \\
                      & =\alpha(B),
        \end{aligned}
    \end{equation}
    meaning that~$A\posleqof{\posLA} B\Imp \alpha(A) \posgeq_{\posUB} \alpha(B)$.
    We now prove that~$\beta$ is an \SY{antitone map}.
    Given~$C,D\setin \posUB$, with~$C\setsupseteq D$, we have
    \begin{equation}
        \begin{aligned}
            \beta(C) & =\makeset{\posAel\setin \posA\colon \exists \posBel\setin C\colon d(\posAel,\posBel)} \\
                     & \setsupseteq \makeset{\posAel\setin \posA\colon \exists \posBel\setin D \colon d(\posAel,\posBel)} \\
                     & =\beta(D),
        \end{aligned}
    \end{equation}
    meaning that~$C\posleqof{\posUB} D \Imp \beta(C)\posgeq_{\posUB} \beta(D)$.
\end{proof}

\begin{lemma}\label{lem:deltagamma-monotone-galois}
    $(\delta, \gamma)$~forms a \SYN{monotone Galois connection}{\textbf{monotone}
        Galois connection} between~$\posLA$ and~$\posUB$.
\end{lemma}
\begin{proof}
    In~\cref{lem:deltagammamonotone} we proved that~$\delta$ and~$\gamma$ are \SY{monotone maps}.
    We now need to show that for any \SY{lower set}~$L\setsubseteq \posA$ of functionalities and \SY{upper set} $U\setsubseteq \posB$ of resources, we have
    \begin{equation}
        L\setsubseteq\gamma(U) \iff \delta(L)\setsupseteq U
    \end{equation}
    The left-hand side says that if~$\posAel\setin L$, then for all~$\posBel \setin U$ we have $d(\posAel,\posBel)=\true$.
    The right-hand side says that if~$\posBel\setin U$ then for all~$\posAel \setin L$,~$d(\posAel,\posBel)=\true$.
    Both are equivalent to~$\forall \posAel\setin L,\posBel\setin U$: $d(\posAel,\posBel)=\true$, and hence to each other.
    In formulas:
    \begin{equation}
        \begin{aligned}
            L \setsubseteq \gamma(U) & \equiv L\setsubseteq \makeset{\posAel \setin \posA\colon \forall \posBel\setin U\colon d(\posAel,\posBel)} \\
                                     & \equiv \forall \posAel\setin L, \posBel\setin U \colon d(\posAel,\posBel)=\true \\
                                     & \equiv \forall \posBel\setin U, \posAel\setin L \colon d(\posAel,\posBel)=\true \\
                                     & \equiv U\setsubseteq \makeset{\posBel\setin \posB\colon \forall \posAel\setin L\colon d(\posAel,\posBel)=\true } \\
                                     & \equiv U\setsubseteq \delta(L).
        \end{aligned}
    \end{equation}
\end{proof}
% We need to prove that, for $a\setin F$, $b\setin R$:
% \begin{equation}
% \label{eq:gammadeltafirst}
%     a\leq_{LF} \gamma(\delta(a)),
% \end{equation}
% and
% \begin{equation}
% \label{eq:gammadeltasec}
% b\geq_{UR}\delta(\gamma(b))
% \end{equation}
% \begin{itemize}
%     \item Let's start from \cref{eq:gammadeltafirst}. We know that $a\leq_{LF} \gamma(\delta(a))$ means $a\setsubseteq \gamma(\delta(a))$. Assume this is not true, \ie  $\exists x\setin a \colon (x\setin a)\wedge (x \not\setin \gamma(\delta(a)))$. Following \cref{eq:galoisdelta}, we know that if $y\setin \delta(a)$, $d(x',y)=\true \ \forall x'\setin a$. Following \cref{eq:galoisgamma}, we know that if $w\setin \gamma(\delta(a))$, $d(w,y)=\true \ \forall y\setin \delta(a)$. But from before, we know that for each $y\setin \delta(a)$, we have $d(x',y)=\true$, for all $x'\setin a$, meaning that $\gamma(\delta(a))$ must include $x'$, $\forall x' \setin a$. This contradicts the initial assumption.
%     \item Let's continue with \cref{eq:gammadeltasec}. We know that $b\geq_{UR} \delta(\gamma(b))$ means $b \setsubseteq \delta(\gamma(b))$. Assume this is not true, \ie  $\exists x\setin b\colon (x\setin b)\wedge (x \not\setin \delta(\gamma(b)))$. From \cref{eq:galoisgamma}, we know that if $y \setin \gamma(b)$, $d(y,x')=\true$, for all $x'\setin b$. Following \cref{eq:galoisdelta}, we know that if $w\setin \delta(\gamma(b))$, $d(y,w)=\true$, for all $y\setin \gamma(b)$. But from before, we know that for each $y\setin \gamma(b)$, we have $d(y,x')=\true$, for all $x'\setin b$, meaning that $\delta(\gamma(b))$ must include $x'$, $\forall x' \setin b$. This contradicts the initial assumption.
% \end{itemize}

\begin{lemma}\label{lem:alpha-beta-not-antitone}
    $(\alpha, \beta)$ does not form an \SYN{antitone Galois connection}{\textbf{antitone}
        Galois connection} between~$\posLA$ and~$\posUB$.
\end{lemma}
\begin{proof}
    In~\cref{lem:alfabetaantitone} we have proved that$\alpha$ and~$\beta$ are \SY{antitone maps}.
    For~$L\setin \posA$,~$U\setin \posB$, we want to show that the following does not hold:
    \begin{equation}
        \label{eq:alfabetafirst}
        L\setsubseteq \beta(\alpha(L))
    \end{equation}
    and
    \begin{equation}
        \label{eq:alfabetasec}
        U\setsupseteq \alpha(\beta(U)).
    \end{equation}
    %

    \paragraph{Example}
    Consider~$d$ as the \SY{design problem} which is always not feasible (the empty \SY{profunctor}), which means~$d(\posAel,\posBel)=\false$,~$\forall \posAel\setin \posA,\posBel\setin \posB$.
    Take any~$L\setin \posA$.
    We know that~$\alpha(L)=\Emptyset$, and~$\beta(\alpha(L))=\beta(\Emptyset)=\Emptyset$.
    But~$L\setsubseteq \Emptyset$ is not true.
\end{proof}

}