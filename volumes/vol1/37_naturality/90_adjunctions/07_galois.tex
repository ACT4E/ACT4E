% !TEX root = chapter-standalone.tex

% \section{The duality of design}

    \section{Formal concept analysis}
    \todojira{143}{\bernina:
        An example of a Galois connection between functionalities and resources could be treated here to motivate the \SY{adjunction} discussion}

Formal concept analysis (FCA) is a mathematical theory which formalizes the relationships, and in particular hierarchies, that appear when we consider a set of \emph{things} -- these are called \emph{objects} in FCA -- together with a set of \emph{attributes} that these things may or may not have. (We will use the word ``things'' instead of ``objects'', because we are already using the word ``objects'' in the category-theory sense.) 

The most basic set-up for formal concept analysis is to start with a triple $\tup{\things, \attributes, \incidence}$, where $\things$ is a set of things (``G'' stands for the german word ``Gegenst\"ande''), $\attributes$ is a set of attributes (``M'' stands for the german word ``Merkmale''), and $\incidence \subseteq \things \cartprod \attributes$ is a relation that encodes which objects are associated with which attributes (``I'' stands for the german word ``Inzidenz''). The triple $\tup{\things, \attributes, \incidence}$ is called a \emph{formal context}.

Here is a (very simplified) example in the context of ``private means of transportation''. We consider the set of things to be the following means of transportation
\begin{equation}\label{eq:fca-example-transport-things}
\things = \makeset{ \text{classic car}, \text{hybrid car}, \text{electric car}, \\
\text{classic bike}, \text{hybrid bike}, \text{electric scooter}, \text{skateboard}}
\end{equation}
and we consider the set of attributes 
\begin{equation}\label{eq:fca-example-transport-attributes}
\attributes = \makeset{\text{fast}, \text{electric}, \text{gas}, \text{muscle}, \text{cheap}},
\end{equation}
which describe aspects such how the means of transportation are powered, their relative cost, or if they can go fast enough to move on a highway, for example. 
%To make notation more economical, we'll abbreviate the elements of these sets as
%\begin{equation}
%\things = \makeset{ \text{CC}, \text{HC}, \text{EC}, \\
%\text{CB}, \text{HB}, \text{ES}, \text{SB}}
%\end{equation}
%and 
%\begin{equation}
%\attributes = \makeset{\text{F}, \text{E}, \text{G}, \text{H}, \text{C}}.
%\end{equation}
We define the relation $\incidence \subseteq \things \cartprod \attributes$ via the following table
\begin{center}
\begin{tabular}{c|c|c|c|c|c}\label{fca-table-private-transportation}
 & fast & electric & gas & muscle & cheap \\
\hline 
classic car & $\times$ & & $\times$ & & \\
\hline
hybrid car & $\times$ &$\times$ &$\times$ &  & \\
 \hline
electric car & $\times$ &$\times$ & & &  \\
 \hline
classic bike & & & &$\times$ & $\times$ \\
 \hline
hybrid bike & &$\times$ & &$\times$ & \\
 \hline
electric scooter & &$\times$ & & &$\times$ \\
 \hline
skateboard & & & &$\times$ &$\times$ \\
\hline
\end{tabular} 
\end{center}
where a cross ``$\times$'' indicates when a thing and an attribute are associated with each other.  


\subsection{Induced monotone maps}

For each element $\ela$ of $\things$, we can consider the set $\incidence_*(\makeset{\ela})$ of attributes that are associated with $\ela$. This corresponds to reading off in the the above table where there are crosses ``$\times$'' in the row labeled by $\ela$. For example
\begin{equation}\label{eq:fca-example-pushforward-1}
\incidence_*(\makeset{\text{hybrid car}}) = \makeset{\text{fast}, \text{electric}, \text{gas}}. 
\end{equation}


More generally, given a subset $\setA \subseteq \things$, can can consider the largest set $\incidence_*(\setA)$ of attributes which all elements of $\setA$ have in common. For example 
\begin{equation}\label{eq:fca-example-pushforward-2}
\incidence_*(\makeset{\text{hybrid car}, \text{electric car}}) = \makeset{\text{fast}, \text{electric}}. 
\end{equation}
Here, ``$\text{gas}$'' is not an element of $\incidence_*(\makeset{\text{hybrid car}, \text{electric car}})$ because ``$\text{gas}$'' is associated with $\text{hybrid car}$ but not with $\text{electric car}$.  

Similarly, 
\begin{equation}\label{eq:fca-example-pushforward-3}
\incidence_*(\makeset{\text{classic car}, \text{hybrid car}, \text{electric car}}) = \makeset{\text{fast}}. 
\end{equation}
Also we have, for example, 
\begin{equation}\label{eq:fca-example-pushforward-4}
\incidence_*(\makeset{\text{classic car}, \text{classic bike}}) = \emptyset
\end{equation}
because ``\text{classic car}'' and ``\text{classic bike}'' have \emph{no} attributes from the set $\attributes$ in common. In general we have 
\begin{equation}\label{eq:fca-example-pushforward-5}
\incidence_*(\setA) = \bigcap_{\ela \setin \setA} \incidence_*(\makeset{\ela}).
\end{equation}
and the operation ``$\incidence_*$'' defines a function 
\begin{equation}\label{eq:fca-example-pushforward-6}
\incidence_* \colon \powerset \things \mto \powerset \attributes.
\end{equation}

Observe that the larger $\setA$ is, the smaller $\incidence_*(\setA)$ will be. Formulated more mathematically, we have
\begin{equation}\label{eq:fca-example-pushforward-7}
\setA \subseteq \setA' \quad \implies \quad \incidence_*(\setA) \supseteq \incidence_*(\setA').
\end{equation}
Another way of saying this is to say that $\incidence_*$ is a monotone map of posets 
\begin{equation}\label{eq:fca-example-pushforward-8}
\incidence_* \colon \tup{\powerset \things, \subseteq} \mto \tup{\powerset \attributes, \supseteq}. 
\end{equation}
Or, equivalently, we can say that $\incidence_*$ is a monotone map
\begin{equation}\label{eq:fca-example-pushforward-9}
\incidence_* \colon \tup{\powerset \things, \subseteq} \mto \tup{\powerset \attributes, \subseteq}\op. 
\end{equation}

Note that we can also define a similar map in the other direction: there is a function
\begin{equation}\label{eq:fca-example-pullback-1}
\incidence^* \colon \powerset \attributes \mto \powerset \things
\end{equation}
defined such that for any subset $\setB \subseteq \attributes$, the set $\incidence^*(\setB)$ is the largest set of elements of $\things$ such that the attributes in $\setB$ apply to all of the elements of $\incidence^*(\setB)$. 
For example, 
\begin{equation}
\incidence^*(\makeset{\text{muscle}, \text{cheap}}) = \makeset{\text{classic bike}, \text{skateboard}}.
\end{equation}
The map $\incidence^*$  is also order-reversing with respect to inclusion of sets: if we start with a larger set of attributes, then set of things that these all apply to will be smaller. Thus we have a monotone map 
\begin{equation}\label{eq:fca-example-pullback-2}
\incidence^* \colon \tup{\powerset \attributes, \subseteq} \mto \tup{\powerset \things, \subseteq}\op
\end{equation}
or, equivalently, a monotone map
\begin{equation}\label{eq:fca-example-pullback-3}
\incidence^* \colon \tup{\powerset \attributes, \subseteq}\op \mto \tup{\powerset \things, \subseteq}.
\end{equation}

In the following, we will find it convenient to use the notations $\incidence_*$ and $\incidence^*$, irrespective of whether we are thinking of there being an ``$( - )\op $'' on the target or source poset (it will be convenient to switch seamlessly between these different points of view). 

A key observation is that $\incidence_*$ and $\incidence^*$ are ``complementary'' in the following sense. For any $\setA \subseteq \things$ and any $\setB \subseteq \attributes$ we have 
\begin{equation}\label{eq:concept-adjunction}
\incidence_*(\setA) \supseteq \setB \quad \Leftrightarrow \quad \setA \subseteq \incidence^*(\setB).
\end{equation}
This equivalence formalizes the (nearly tautological-seeming) statement that a set $\setB$ is contained in the largest set of attributes which apply to all members of $\setA$ (meaning: the attributes $\setB$ apply to all elements of $\setA$) if, and only if, $\setA$ is contained in the largest set of things to which all attributes $\setB$ apply (again meaning: the attributes $\setB$ apply to all elements of $\setA$). 

Despite seeming tautological, we can use the equivalence \cref{eq:concept-adjunction} to make non-trivial observations. One consequence of \cref{eq:concept-adjunction} is that the monotone maps 
\begin{equation}\label{eq:fca-closure-operator-things}
\incidence_* \mthen \incidence^* \colon \tup{\powerset \things, \subseteq} \mto \tup{\powerset \things, \subseteq}
\end{equation}
and 
\begin{equation}\label{eq:fca-closure-operator-attributes}
\incidence^* \mthen \incidence_* \colon \tup{\powerset \attributes, \supseteq} \mto \tup{\powerset \attributes, \supseteq}.
\end{equation}
are examples of what are called \emph{closure operators}.


\subsection{Closure operators}

\begin{definition}\label{def:closure-operator}
Let $\posA = \tup{\posAset, \posleq}$ be poset. A \emph{closure operator} on $\posA$ is

\constit
\begin{enumerate}
\item a monotone map $\mora \colon \posA \mto \posA$; 
\end{enumerate}

\condit
\begin{enumerate}
\item \text{Extensivitity:} \  $\ela \posleq \mora(\ela) \quad \forall \ela \setin \posAset$;
\item \text{Idempotence:} \ $\mora (\mora(\ela)) =  \mora(\ela) \quad \forall \ela \setin \posAset$. 
\end{enumerate}



\end{definition}

\begin{lemma}\label{lem:cfa-closure-operators}
The monotone maps 
\begin{equation}
\incidence_* \mthen \incidence^* \colon \tup{\powerset \things, \subseteq} \mto \tup{\powerset \things, \subseteq}
\end{equation}
and 
\begin{equation}
\incidence^* \mthen \incidence_* \colon \tup{\powerset \attributes, \supseteq} \mto \tup{\powerset \attributes, \supseteq}.
\end{equation}
are closure operators. 
\end{lemma}

\begin{proof}
Let's check that $\incidence_* \mthen \incidence^* \colon \tup{\powerset \things, \subseteq} \mto \tup{\powerset \things, \subseteq}$ is a closure operator, using \cref{eq:concept-adjunction}. We omit the proof for $\incidence^* \mthen \incidence_*$, which may be done analogously. 


To show the first condition in the definition of closure operator, fix a set of things $\setA \subseteq \things$. In the situation of \cref{eq:concept-adjunction}, choose $\setB = \incidence_*(\setA)$. Since $\incidence_*(\setA) \supseteq \incidence_*(\setA)$ is true, \cref{eq:concept-adjunction} implies that $\setA \subseteq \incidence^*(\incidence_*(\setA))  = (\incidence_* \mthen \incidence^*)(\setA)$. 

Now let's consider the second condition. Applying the monotone map $\incidence_* \mthen \incidence^*$ to the relation $\setA \subseteq (\incidence_* \mthen \incidence^*)(\setA)$, we have
\begin{equation}
(\incidence_* \mthen \incidence^*)(\setA)  \subseteq  (\incidence_* \mthen \incidence^* \mthen \incidence_* \mthen \incidence^*)(\setA).
\end{equation}
Thus we are finished when we show the inclusion in the other direction. By the first condition, we know that 
\begin{equation}
\incidence_*(\setA) \subseteq \incidence_*( \incidence^* (\incidence_*(\setA))) = (\incidence_* \mthen \incidence^* \mthen \incidence_*)(\setA).
\end{equation}
Applying the order-reversing map $\incidence^*$ to both sides of this inclusion then gives 
\begin{equation}
(\incidence_* \mthen \incidence^*)(\setA) \supseteq (\incidence_* \mthen \incidence^* \mthen \incidence_* \mthen \incidence^*)(\setA)
\end{equation}
as desired. 
\end{proof}

Closure operators arise in various contexts in mathematics. Often we are interested in the elements which are in the image of a closure operator. These are called \emph{closed elements}. 

\begin{definition}\label{def:closure-closed-elements}
Let $\posA = \tup{\posAset, \posleq}$ be poset, $\mora \colon \posA \mto \posA$ a closure operator, and $\ela \setin \posAset$ an arbitrary element of $\posA$. The element $\mora(\ela) \setin \posAset$ is called the \emph{closure} of $\ela$, and $\ela$ is called \emph{closed} if $\mora(\ela) = \ela$. 

We denote by $\posA_\mora$ the poset of elements of $\posA$ that are closed with respect to $\mora$, or we use $\posA_{cl}$ when the closure operator in question is clear. 
\end{definition}

\begin{remark}\label{rem:closed-elements-are-image}
Note that if $\mora \colon \posA \mto \posA$ is a closure operator, then for any $\ela \setin \posAset$, the element $\mora(\ela)$ is closed, because
\begin{equation}
\mora(\mora(\ela)) = \mora(\ela)
\end{equation}
by the idempotence property of closure operators. 
\end{remark}


Returning now to formal concept analysis, let's look at closures and closed elements for the closure operators 
$$\incidence_* \mthen \incidence^* \colon \tup{\powerset \things, \subseteq} \mto \tup{\powerset \things, \subseteq}$$  
and 
$$\incidence^* \mthen \incidence_* \colon \tup{\powerset \attributes, \supseteq} \mto \tup{\powerset \attributes, \supseteq} $$ 
in terms of our simple example. 

For example, let $\setA = \makeset{\text{classic car}, \text{electric car}}$. Then 
\begin{equation}
\incidence_*(\setA) = \makeset{\text{fast}}
\end{equation}
and 
\begin{equation}
\incidence^*(\incidence_*(\setA)) = \incidence^*(\makeset{\text{fast}}) = \makeset{\text{classic car}, \text{hybrid car}, \text{electric car}}.
\end{equation}
So $\setA = \makeset{\text{classic car}, \text{electric car}}$ is not a closed element of $\tup{\powerset \things, \subseteq}$. Its closure contains the element ``$\text{hybrid car}$'' which is not in $\setA$.

Or consider $\setB = \makeset{\text{electric}, \text{muscle}}$. Then
\begin{equation}
\incidence^*(\setB) = \makeset{\text{hybrid bike}}
\end{equation}
and 
\begin{equation}
\incidence_*(\incidence^*(\setB)) = \incidence_*(\makeset{\text{hybrid bike}}) = \makeset{\text{electric}, \text{muscle}} = \setB.
\end{equation}
We find here that $\makeset{\text{hybrid bike}}$ \emph{is} a closed element of $\tup{\powerset \attributes, \subseteq}$. 

In general, given a set $\setA$ of things, its closure $(\incidence_* \mthen \incidence^*)(\setA)$ is the largest set of things that share the attributes in $\incidence_*(\setA)$. And $\incidence_*(\setA)$ is the largest set of attributes shared by $\setA$. Thus we may say:

 ``\emph{$(\incidence_* \mthen \incidence^*)(\setA)$ is the maximal set of things that share the same attributes as are shared by $\setA$}.'' 
 
Or, put another way, taking the closure of $\setA$ is a way of enlarging $\setA$ without decreasing the set of associated shared attributes. Closing $\setA$ is adding those things to $\setA$ that come ``for free'' in the sense that, by adding them, we are not losing shared attributes. 

A similar point of view of course also applies to closing sets of attributes with respect to the closure operator $\incidence^* \mthen \incidence_*$.

\subsection{Concepts}

\begin{definition}\label{def:cfa-concept}
Let $\tup{\things, \attributes, \incidence}$ be a formal context in the sense of formal concept analysis. A \emph{concept} is a pair $\tup{\setA, \setB} \setin \powerset\things \cartprod \powerset\attributes$ such that 
\begin{equation}\label{eq:def-concept-equations}
\incidence_*(\setA) = \setB \quad \text{and} \quad \incidence^*(\setB) = \setA.
\end{equation}
For a concept $\tup{\setA, \setB}$, the set $\setA$ of things is called the \emph{extent} of the concept, and the set $\setB$ of attributes is called the \emph{intent} of the concept. 

We denote the set of all concepts for the context $\tup{\things, \attributes, \incidence}$ by $\mathcal{B}\tup{\things, \attributes, \incidence}$. (Here ``B'' comes from the German term ``Begriffe''.) 
\end{definition}

The set $\mathcal{B}\tup{\things, \attributes, \incidence}$ of concepts for a formal context has a natural partial order structure. We set
\begin{equation}\label{eq:partial-order-concepts}
\tup{\setA_1, \setB_1} \posleq \tup{\setA_2, \setB_2}
\end{equation}
if $\setA_1 \subseteq \setA_2$ and $\setB_1 \supseteq \setB_2$. (In fact, by the definition of a concept, if one of the latter inclusions holds, then so must the other, so we only need to require one of them.) When \cref{eq:partial-order-concepts} holds, we say that $\tup{\setA_1, \setB_1}$ is a \emph{subconcept} of $\tup{\setA_2, \setB_2}$


\begin{lemma}\label{lem:extents-and-intents-are-closed}
If $\tup{\setA, \setB}$ is a concept, then $\setA$ and $\setB$ are closed elements of $\tup{\powerset \things, \subseteq}$ and $\tup{\powerset \attributes, \supseteq}$, respectively. 
\end{lemma}

\begin{proof}
For $\setA$ we have 
\begin{equation}
\incidence^*(\incidence_*(\setA)) = \incidence^*(\setB) = \setA
\end{equation}
using both the equations \cref{eq:def-concept-equations}.
The case for $\setB$ is analogous. 
\end{proof}

\begin{lemma}\label{lem:closed-elements-define-concepts}
If $\setA \in \tup{\powerset \things, \subseteq}$ is closed, then $\incidence_*(\setA)$ is closed and $\tup{\setA, \incidence_*(\setA)}$ is a concept. 

Similarly, if $\setB \in \tup{\powerset \attributes, \supseteq}$ is closed, then $\incidence^*(\setB)$ is closed and $\tup{\incidence^*(\setB), \setB)}$ is a concept.
\end{lemma}

\begin{proof}
We show only the first statement. We have 
\begin{equation}
(\incidence^* \mthen \incidence_*) (\incidence_*(\setA)) = \incidence_*( \incidence^* (\incidence_*(\setA))) = \incidence_*( (\incidence_* \mthen \incidence_*)(\setA)) = \incidence_*(\setA), 
\end{equation}
so $\incidence_*(\setA)$ is closed. That $\tup{\setA, \incidence^*(\setA)}$ is a concept is clear, since $\incidence^*(\incidence_*(\setA)) = \setA$. 
\end{proof}




\begin{lemma}
The posets of closed elements $\tup{\powerset \things, \subseteq}_{cl}$ and $\tup{\powerset \attributes, \supseteq}_{cl}$ are isomorphic via the restrictions of $\incidence_*$ and $\incidence^*$, and each is isomorphic to the poset $\tup{\mathcal{B}\tup{\things, \attributes, \incidence}, \posleq}$ via its projections onto its first and second factors, respectively.
\end{lemma}

\begin{proof}
This follows straighforwardly from \cref{lem:extents-and-intents-are-closed}, \cref{lem:closed-elements-define-concepts}, and the definition of the ordering on $\tup{\mathcal{B}\tup{\things, \attributes, \incidence}, \posleq}$. 
\end{proof}

\section{Galois connections}\label{sec:galois-connections}

\begin{ctdefinition}[Monotone Galois Connection]\label{def:monotone-galois-connection}
    A \maindef{monotone Galois connection} between \SY{posets}~\posA and~\posB is a pair of \SY{monotone maps}
    $\mapa\colon \posA\to \posB$ and~$\mapb\colon \posB\to \posA$ such that for all~$\posAel\setin \posA$,~$\posBel\setin \posB$:
    \begin{equation}
        \prfdoubleperiod{\mapa(\posAel) \posBleq \posBel}{\posAel \posAleq \mapb(\posBel)}
        %\mapa(\posAel) \posBleq \posBel \quad \Leftrightarrow \quad \posAel \posAleq \mapb(\posBel).
    \end{equation}
    This is equivalent to ask, for all~$\posAel\setin \posA$,~$\posBel\setin \posB$, that:
    \begin{equation}
        (\posAel\posAleq \mapb(\mapa(\posAel)))
        \booland (\posBel\posBleq \mapa(\mapb(\posBel))).
    \end{equation}
\end{ctdefinition}

\begin{ctdefinition}[Antitone Galois Connection]\label{def:antitone-galois-connection}
    An \maindef{antitone Galois connection} between~\posA and~\posB is a pair of \SY{antitone maps} $\mapa\colon \posA\to \posB$ and $\mapb \colon \posB\to \posA$ such that for all~$\posAel\setin \posA$, $\posBel\setin \posB$:
    \begin{equation}
        \prfdoubleperiod{\posBel \posBleq \mapa(\posAel)}{\posAel \posAleq \mapb(\posBel)}
        %\posBel \posBleq \mapa(\posAel) \quad \Leftrightarrow \quad \posAel \posAleq \mapb(\posBel).
    \end{equation}
    This is equivalent to ask for all~$\posAel\setin \posA$,~$\posBel\setin \posB$:
    \begin{equation}
        (\posAel \posAleq \mapb(\mapa(\posAel)))
        \booland (\posBel \posBleq \mapa(\mapb(\posBel))).
    \end{equation}
\end{ctdefinition}

\todotextjira{405}{\bernina: @Gioele: GZ: Adjust given new flow.
    We don't know what this is yet.
}
\todotext{\alphubel: we are not using the right symbols $\adp$}

\devel{
\section{Galois connections and co-design}\label{sec:galois-connections-codesign}


Consider a design problem~$d\colon \posAop \Ptimes \posB \toinPos \Bool$.
We can define the maps that work on single functionality and resources:
%
\begin{equation}
    \begin{aligned}
        \theta\colon \posA & \to \posUB \\
        \posAel            & \mapsto \makeset{\posBel\setin \posB \colon d(\posAopel, \posBel) },
    \end{aligned}
\end{equation}
%
\begin{equation}
    \begin{aligned}
        \psi\colon \posB & \to \posLA \\
        \posBel          & \mapsto \makeset{\posAel\setin \posA \colon d(\posAopel, \posBel) }.
    \end{aligned}
\end{equation}
We can define the maps that work on multiple functionality and resources:
\begin{equation}
    \label{eq:galoisalfa}
    \begin{aligned}
        \alpha\colon \posLA & \to \posUB \\
        S                   & \mapsto \makeset{\posBel\setin \posB \colon \exists \posAel\setin S\colon \ d(\posAel,\posBel)},
    \end{aligned}
\end{equation}
Alternatively, we can write
\begin{equation}
    \begin{aligned}
        \alpha \colon \posLA & \to \posUB \\
        S                    & \mapsto \bigsetunion_{\posAel\setin S} \theta(\posAel).
    \end{aligned}
\end{equation}
%
\begin{equation}
    \label{eq:galoisbeta}
    \begin{aligned}
        \beta\colon \posUB & \to \posLA \\
        T                  & \mapsto \makeset{\posAel\setin \posA  \colon \exists \posBel\setin T\colon d(\posAel,\posBel)},
    \end{aligned}
\end{equation}
%
\noindent Alternatively, we can write
\begin{equation}
    \begin{aligned}
        \beta \colon \posUB & \to \posLA \\
        T                   & \mapsto \bigsetunion_{\posBel\setin T} \psi(\posBel).
    \end{aligned}
\end{equation}
%
\begin{equation}
    \label{eq:galoisdelta}
    \begin{aligned}
        \delta \colon \posLA & \to \posUB \\
        S                    & \mapsto \makeset{\posBel\setin \posB \colon \forall {\posAel\setin S}\colon d(\posAel,\posBel)},
    \end{aligned}
\end{equation}
Alternatively, we can write
\begin{equation}
    \begin{aligned}
        \delta\colon \posLA & \to \posUB \\
        S                   & \mapsto \bigsetintersection_{\posAel\setin S}\theta(\posAel).
    \end{aligned}
\end{equation}
%
\begin{equation}
    \label{eq:galoisgamma}
    \begin{aligned}
        \gamma \colon \posUB & \to \posLA \\
        T                    & \mapsto \makeset{\posAel\setin \posA \colon \forall {\posBel\setin T}\colon d(\posAel,\posBel)},
    \end{aligned}
\end{equation}
Alternatively, we can write
\begin{equation}
    \begin{aligned}
        \delta\colon \posUB & \to \posLA \\
        T                   & \mapsto \bigsetintersection_{\posBel\setin T}\psi(\posBel).
    \end{aligned}
\end{equation}
%
Properties of these maps are reported in \cref{tab:galoisproperties}.

\begin{table*}[h!]
    \centering
    \begin{tabular}{c|l|l|c|c|c|c|c}
        $\star$  & X                                   & Y                    & $\star(\bot)$                             & $\star(\postop)$                           & $A\posleqof{X}
        B$       & $\star(A\vee_X B)$                  & $\star(A\wedge_X B)$ \\
        \hline
        $\alpha$ & $\posLA$                            & $\posUB$             & $\alpha(\Emptyset)=\Emptyset$             & $\alpha(\posA)\geq_{\posUB} \alpha(\cdot)$
                 & $\alpha(A)\geq_{\posUB} \alpha(B)$
                 & $\alpha(A)\vee_{\posLA}\alpha(B)$
                 & $\alpha(A)\wedge_{\posLA}\alpha(B)$ \\
        \hline
        $\beta$  & $\posUB$                            & $\posLA$             & $\beta(\posB)\geq_{\posLA}\beta(\cdot)$   & $\beta(\Emptyset)=\Emptyset$
                 & $\beta(A)\geq_{\posLA} \beta(B)$
                 & $\beta(A)\vee_{\posLA}\beta(B)$
                 & $\beta(A)\wedge_{\posLA}\beta(B)$ \\
        \hline
        $\delta$ & $\posLA$                            & $\posUB$             & $\delta(\Emptyset)=\posB$                 & $\delta(\posA)\geq_{\posUB}\delta(\cdot)$  & $\delta(A)\leq_{\posUB} \delta(B)$
                 & $\delta(A)\wedge_{\posUB}\delta(B)$
                 & $\delta(A)\vee_{\posUB}\delta(B)$ \\
        \hline
        $\gamma$ & $\posUB$                            & $\posLA$             & $\gamma(\posB)\leq_{\posLA}\gamma(\cdot)$ & $\gamma(\Emptyset)=\posA$                  & $\gamma(A) \leq_{\posLA} \gamma(B)$
                 & $\gamma(A)\wedge_{\posLA}\gamma(B)$
                 & $\gamma(A)\vee_{\posLA}\gamma(B)$
    \end{tabular}
    \caption{Properties of $\alpha,\beta,\delta,\gamma$}
    \label{tab:galoisproperties}
\end{table*}

\begin{lemma}
    \label{lem:deltagammamonotone}
    $\delta$ and $\gamma$ are \SY{monotone maps}.
\end{lemma}
\begin{proof}
    We first prove that~$\delta$ is a \SY{monotone map}.
    Given~$A,B\setin \posLA$ with~$A\setsubseteq B$, we have
    \begin{equation}
        \begin{aligned}
            \delta(A) & =\makeset{\posBel\setin \posB\colon \forall \posAel \setin A\colon d(\posAel,\posBel)} \\
                      & \setsupseteq \makeset{\posBel\setin \posB\colon \forall \posAel\setin B \colon d(\posAel,\posBel)} \\
                      & =\delta(B),
        \end{aligned}
    \end{equation}
    meaning that~$A\posleqof{\posLA} B \Imp \delta(A)\posleqof{\posUB} \delta(B)$.
    We now prove that~$\gamma$ is a \SY{monotone map}.
    Given~$C,D\setin \posUB$, with~$C\setsupseteq D$, we have
    \begin{equation}
        \begin{aligned}
            \gamma(C) & =\makeset{\posAel\setin \posA\colon \forall \posBel \setin C\colon d(\posAel,\posBel)} \\
                      & \setsubseteq \makeset{\posAel\setin \posA\colon \forall \posBel \setin D \colon d(\posAel,\posBel)} \\
                      & =\gamma(D),
        \end{aligned}
    \end{equation}
    meaning that~$C\posleqof{\posUB} D \Imp \gamma(C)\posleqof{\posLA} \gamma(D)$.
\end{proof}

\begin{lemma}
    \label{lem:alfabetaantitone}
    $\alpha$ and $\beta$ are \SY{antitone maps}.
\end{lemma}
\begin{proof}
    We first prove that~$\alpha$ is an \SY{antitone map}.
    Given~$A,B\setin \posLA$, with~$A\setsubseteq B$, we have
    \begin{equation}
        \begin{aligned}
            \alpha(A) & =\makeset{ \posBel\setin \posB \colon \exists \posAel\setin A\colon d(\posAel,\posBel)} \\
                      & \setsubseteq \makeset{\posBel\setin \posB \colon \exists \posAel\setin B\colon d(\posAel,\posBel)} \\
                      & =\alpha(B),
        \end{aligned}
    \end{equation}
    meaning that~$A\posleqof{\posLA} B\Imp \alpha(A) \posgeq_{\posUB} \alpha(B)$.
    We now prove that~$\beta$ is an \SY{antitone map}.
    Given~$C,D\setin \posUB$, with~$C\setsupseteq D$, we have
    \begin{equation}
        \begin{aligned}
            \beta(C) & =\makeset{\posAel\setin \posA\colon \exists \posBel\setin C\colon d(\posAel,\posBel)} \\
                     & \setsupseteq \makeset{\posAel\setin \posA\colon \exists \posBel\setin D \colon d(\posAel,\posBel)} \\
                     & =\beta(D),
        \end{aligned}
    \end{equation}
    meaning that~$C\posleqof{\posUB} D \Imp \beta(C)\posgeq_{\posUB} \beta(D)$.
\end{proof}

\begin{lemma}\label{lem:deltagamma-monotone-galois}
    $(\delta, \gamma)$~forms a \SYN{monotone Galois connection}{\textbf{monotone}
        Galois connection} between~$\posLA$ and~$\posUB$.
\end{lemma}
\begin{proof}
    In~\cref{lem:deltagammamonotone} we proved that~$\delta$ and~$\gamma$ are \SY{monotone maps}.
    We now need to show that for any \SY{lower set}~$L\setsubseteq \posA$ of functionalities and \SY{upper set} $U\setsubseteq \posB$ of resources, we have
    \begin{equation}
        L\setsubseteq\gamma(U) \iff \delta(L)\setsupseteq U
    \end{equation}
    The left-hand side says that if~$\posAel\setin L$, then for all~$\posBel \setin U$ we have $d(\posAel,\posBel)=\true$.
    The right-hand side says that if~$\posBel\setin U$ then for all~$\posAel \setin L$,~$d(\posAel,\posBel)=\true$.
    Both are equivalent to~$\forall \posAel\setin L,\posBel\setin U$: $d(\posAel,\posBel)=\true$, and hence to each other.
    In formulas:
    \begin{equation}
        \begin{aligned}
            L \setsubseteq \gamma(U) & \equiv L\setsubseteq \makeset{\posAel \setin \posA\colon \forall \posBel\setin U\colon d(\posAel,\posBel)} \\
                                     & \equiv \forall \posAel\setin L, \posBel\setin U \colon d(\posAel,\posBel)=\true \\
                                     & \equiv \forall \posBel\setin U, \posAel\setin L \colon d(\posAel,\posBel)=\true \\
                                     & \equiv U\setsubseteq \makeset{\posBel\setin \posB\colon \forall \posAel\setin L\colon d(\posAel,\posBel)=\true } \\
                                     & \equiv U\setsubseteq \delta(L).
        \end{aligned}
    \end{equation}
\end{proof}
% We need to prove that, for $a\setin F$, $b\setin R$:
% \begin{equation}
% \label{eq:gammadeltafirst}
%     a\leq_{LF} \gamma(\delta(a)),
% \end{equation}
% and
% \begin{equation}
% \label{eq:gammadeltasec}
% b\geq_{UR}\delta(\gamma(b))
% \end{equation}
% \begin{itemize}
%     \item Let's start from \cref{eq:gammadeltafirst}. We know that $a\leq_{LF} \gamma(\delta(a))$ means $a\setsubseteq \gamma(\delta(a))$. Assume this is not true, \ie  $\exists x\setin a \colon (x\setin a)\wedge (x \not\setin \gamma(\delta(a)))$. Following \cref{eq:galoisdelta}, we know that if $y\setin \delta(a)$, $d(x',y)=\true \ \forall x'\setin a$. Following \cref{eq:galoisgamma}, we know that if $w\setin \gamma(\delta(a))$, $d(w,y)=\true \ \forall y\setin \delta(a)$. But from before, we know that for each $y\setin \delta(a)$, we have $d(x',y)=\true$, for all $x'\setin a$, meaning that $\gamma(\delta(a))$ must include $x'$, $\forall x' \setin a$. This contradicts the initial assumption.
%     \item Let's continue with \cref{eq:gammadeltasec}. We know that $b\geq_{UR} \delta(\gamma(b))$ means $b \setsubseteq \delta(\gamma(b))$. Assume this is not true, \ie  $\exists x\setin b\colon (x\setin b)\wedge (x \not\setin \delta(\gamma(b)))$. From \cref{eq:galoisgamma}, we know that if $y \setin \gamma(b)$, $d(y,x')=\true$, for all $x'\setin b$. Following \cref{eq:galoisdelta}, we know that if $w\setin \delta(\gamma(b))$, $d(y,w)=\true$, for all $y\setin \gamma(b)$. But from before, we know that for each $y\setin \gamma(b)$, we have $d(y,x')=\true$, for all $x'\setin b$, meaning that $\delta(\gamma(b))$ must include $x'$, $\forall x' \setin b$. This contradicts the initial assumption.
% \end{itemize}

\begin{lemma}\label{lem:alpha-beta-not-antitone}
    $(\alpha, \beta)$ does not form an \SYN{antitone Galois connection}{\textbf{antitone}
        Galois connection} between~$\posLA$ and~$\posUB$.
\end{lemma}
\begin{proof}
    In~\cref{lem:alfabetaantitone} we have proved that$\alpha$ and~$\beta$ are \SY{antitone maps}.
    For~$L\setin \posA$,~$U\setin \posB$, we want to show that the following does not hold:
    \begin{equation}
        \label{eq:alfabetafirst}
        L\setsubseteq \beta(\alpha(L))
    \end{equation}
    and
    \begin{equation}
        \label{eq:alfabetasec}
        U\setsupseteq \alpha(\beta(U)).
    \end{equation}
    %

    \paragraph{Example}
    Consider~$d$ as the \SY{design problem} which is always not feasible (the empty \SY{profunctor}), which means~$d(\posAel,\posBel)=\false$,~$\forall \posAel\setin \posA,\posBel\setin \posB$.
    Take any~$L\setin \posA$.
    We know that~$\alpha(L)=\Emptyset$, and~$\beta(\alpha(L))=\beta(\Emptyset)=\Emptyset$.
    But~$L\setsubseteq \Emptyset$ is not true.
\end{proof}

}