% !TEX root = chapter-standalone.tex



\section{Adjunctions:  hom-set definition}
\label{sec:adjunctions-hom-set-definition}
In this section we give a definition of \SY{adjunction} which can be viewed as an analogy with the following situation in linear algebra.
Suppose~$V$ and~$W$ are finite-dimensional real \SY{vector spaces}, equipped with inner products~$(-\com -)_V$ and~$(- \com -)_W$, respectively.
The adjoint of a linear map~$F\colon V \to W$ is a linear map $F^*\colon W \sto V$ such that
\begin{equation}
    (Fv, w)
    _W = (v, F^*w)_V, \quad \forall v \setin V, w \setin W.
\end{equation}

\begin{ctdefinition}[Adjunction, Version 1]
    \label{def:adj-iso}
    \label{def:cat-adjunction-v1}
    Let \CatC and \CatD be categories.
    An \maindef{adjunction} from \CatC to \CatD is given by the following data:
    \begin{enumerate}
        \item A \SY{functor}~$\funl\colon \CatC \fto \CatD$, called the \emph{left adjoint};
        \item A \SY{functor}~$\funr\colon \CatD \fto \CatC$, called the \emph{right adjoint};
        \item A \SY{natural isomorphism}~$\adjtau: \HomSet\CatD{\funl -}{-} \nto \HomSet\CatC{-}{\funr}$ between functors $\CatC\op \Ctimes \CatD \fto \Set$. 
    \end{enumerate}
    We use the notation~$\funl \adjunction \funr$ to indicate that~$\funl$ and~$\funr$ form an \SY{adjunction}, with $\funl$ the \SY{left adjoint} and $\funr$ the \SY{right adjoint}.
\end{ctdefinition}


\todotext{J: @J: include this example somewhere: (copypaste from seven sketches) ``considering a poset as a category is right adjoint to turning a category into a poset by poset reflection."}
