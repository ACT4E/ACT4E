% !TEX root = chapter-standalone.tex

\section[Product-Hom adjunction]{Example of a ``Product-Hom'' adjunction}
We will consider an \SY{adjunction} between the category \Set and itself which is a basic representative of a certain ``type'' of \SY{adjunction} that appears all over mathematics.
This type of \SY{adjunction} might be called a ``Product-Hom'' \SY{adjunction}.

Fix a set~\setB and consider the \SY{functors}~$\funa$ and~$\funb$ which act as follows.
Given a set~\setA,
\begin{equation}\label{eq:product-with-a-set-functor-objects}
    \funa(\setA) = \setB \cartprod \setA
\end{equation}
and
\begin{equation}\label{eq:set-expontiation-functor-objects}
    \funb(\setA) = \HomSet{\Set}{\setB}{\setA} \eqqcolon \setA^\setB.
\end{equation}
\todotextjira{474}{\alphubel: @JL: too synthentic, not readable}
Given a morphism~$\mora\colon \setA \mto \setAprime$,
\begin{equation}\label{eq:product-with-a-set-functor-morphisms}
    \funa(\mora) = \mora \funcprod \catidat\setB
\end{equation}
and
\begin{equation}\label{eq:set-exponentiation-functor-morphisms}
    \begin{aligned}
        \funb(\mora)\colon \setA^\setB & \sto \setAprime^\setB \\
        \morb                          & \mapsto \morb \mthen \mora.
    \end{aligned}
\end{equation}
%
These \SY{functors} are part of an adjunction
%
\begin{equation}\label{eq:prod-hom-adjunction-diagram}
    \middlesag{091_adjunctions_prod_hom}
\end{equation}
In terms of \cref{def:adj-iso}, there is a natural isomorphism
\begin{equation}\label{eq:prod-hom-adjunction-transpose-nat-iso}
    \stylenat{\tau}\colon \HomSet{\Set}{\funa(- )}{-} \ntolong \HomSet{\Set}{-}{\funb( - )}
\end{equation}
whose component at~$\tup{\setA,\setC}$ is the isomorphism
\begin{equation}\label{eq:prod-hom-adjunction-transpose-components}
    \stylenat{\tau}_{\setA,\setC} \colon \HomSet\Set{\setB\cartprod \setA}{\setC} \mto \HomSet{\Set}{\setA}{\setC^\setB}
\end{equation}
given by ``partial evaluation''.
Namely, given~$\mora \colon \setB \cartprod \setA \mto \setC$, this is mapped by~$\stylenat{\tau}_{\setA,\setC}$ to the function~$\stylenat{\tau}\mora \colon \setA \sto \setC^\setB$,~$\setAel \mapsto f( -, \setAel)$.

In terms of \cref{def:adj-counit}, the component at~\setA of the unit and co-unit, respectively, are
\begin{equation}\label{eq:prod-hom-adjunction-unit}
    \begin{aligned}
        \equivunit_\setA \colon \setA & \to  (\setB \cartprod \setA)^\setB \\
        \setAel                       & \mapsto (\setBel \mapsto \tup{\setAel,\setBel})
    \end{aligned}
\end{equation}
and
\begin{equation}\label{eq:prod-hom-adjunction-co-unit}
    \begin{aligned}
        \equivcounit_\setA \colon \setB \cartprod (\setA^\setB) & \sto \setA \\
        \tup{\setBel,\mora}                                     & \mapsto \mora(\setBel)
    \end{aligned}
\end{equation}
