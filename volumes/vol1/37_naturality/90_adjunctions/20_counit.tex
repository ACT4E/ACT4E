% !TEX root = chapter-standalone.tex

\section{Adjunctions: (co)unit definition}

Recall from \cref{def:isomorphism}: in a category \CatC, a morphism~$\mora \colon \Obja \mto \Objb$ is an isomorphism if there exists a morphism~$\morb \colon \Obja \mto \Objb$ such that
\begin{equation}
    \mora \mthen \morb = \catid_\Obja \qqand \morb \mthen \mora = \catid_\Objb.
\end{equation}

Now let's think about this definition in the case where \CatC is the category \Category of categories.
We will consider weakenings of the notion of isomorphism in this setting, and this will lead to a second (but equivalent) definition of adjunction.
The precisely relationship between the two definitions will be spelled out \cref{relate-adj-defs}.

The idea of ``weakening'' the notion of isomorphism of categories is as follows.
Given functors
\begin{equation}
    \middlesag{091_functors_adjunctions}
\end{equation}
%
instead of requiring the equations
%
\begin{equation}
    \funl \fthen \funr = \funid_\CatC  \qqand  \funr \fthen \funl = \funid_\CatD,
\end{equation}
%
we can replace the equality symbols with 2-morphisms!
We'll do it like this:
%
\begin{equation}
    \funl \fthen \funr \ \overset{\equivunit}{\ntolong}\  \funid_\CatC \qqand \funr \fthen \funl \ \overset{\equivcounit}{\nfromlong}\  \funid_\CatD.
\end{equation}
The last two relationships can also be depicted in the following more geometric manner:
\begin{equation}
    \middlesag{091_2_mor_a},
    \hspace{1cm}
    \middlesag{091_2_mor_b}.
\end{equation}

\begin{ctdefinition}[Equivalence of categories]
    \label{def:cat-equivalence}
    Let \CatC and \CatD be categories.
    An \emph{equivalence} between \CatC and \CatD is the following data:
    \begin{enumerate}
        \item A functor~$\funl \colon \CatC \fto \CatD$;
        \item A functor~$\funr \colon \CatD \fto \CatC$;
        \item Natural isomorphisms~$\equivunit \colon \funid_\CatC \nto \funl \fthen \funr$ and~$\equivcounit \colon  \funr \fthen \funl \nto \funid_\CatD$.
    \end{enumerate}
\end{ctdefinition}

\begin{ctdefinition}[Adjunction, Version 2]
    \label{def:adj-counit}
    \label{def:cat-adjunction-v2}
    Let \CatC and \CatD be categories.
    An \emph{\iindex{adjunction}} from \CatC to \CatD is given by the following data, satisfying the following conditions.

    \underline{Data:}
    \begin{enumerate}
        \item A functor~$\funl \colon \CatC \fto \CatD$ (the \emph{left adjoint});
        \item A functor~$\funr \colon \CatD \fto \CatC$ (the \emph{right adjoint});
        \item Natural transformations~$\equivunit \colon \funid_\CatC \nto \funl  \fthen \funr $ and~$\equivcounit\colon \funr \fthen \funl \nto \funid_\CatD$
    \end{enumerate}

    \underline{Conditions:}
    \begin{enumerate}
        \item For all objects~$\Obja$ of \CatC, it holds that
              \begin{equation}
                  \funl \equivunit_\Obja \mthen \equivcounit_{\funl\Obja} = \catid_{\funl\Obja} \qqand
                  \equivunit_{\funr\Objb} \mthen \funr\equivcounit_\Objb = \catid_{\funr\Objb}
              \end{equation}
              \text{\ie } that the following diagrams commute:
              \begin{equation}
                  \middlesag{091_adjunction_def}.
              \end{equation}
    \end{enumerate}

    The 2-morphisms~$\equivunit$ and~$\equivcounit$ are called the \emph{unit} and \emph{counit} of the adjunction.
    An adjunction is called an \emph{adjoint equivalence} if the unit and counit are natural isomorphisms.
\end{ctdefinition}

\begin{remark}
    The conditions (triangle identities) from \cref{def:adj-counit} are ``hidden'' in \cref{def:adj-iso} in the condition that $\tau$ be an natural isomorphism.
    In \cref{relate-adj-defs} we spell out how the two definitions are related.
\end{remark}
