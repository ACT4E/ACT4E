% !TEX root = chapter-standalone.tex

\section{Monoidal categories}
\label{sec:parallelism-mon-cat}

\todotext{\alphubel: Need to redo introduction here}
% We now generalize from \SY{monoidal posets} to \SY{monoidal categories}.
%
% So far, we have described a single way to compose morphisms of a category: the~$\mthen$ operation.
% However, category theory allows defining other ways of composing morphisms, adding structure to the basic category defined in~\cref{def:categorymain}.

\linkvideo{spring2021-par-feedback:mon-cat:mon-cat-def} % Definition of monoidal categories
\begin{ctdefinition}[Monoidal category]
    \label{def:monoidal-cat}
    \SYNDEF{monoidal category}
    A \emph{monoidal structure} on a category~\CatC is specified by:

    \constit
    \begin{enumerate}
        \item A \SY{functor}~$\mtimescat \colon \CatC \Ctimes \CatC\fto \CatC$, called the \emph{monoidal product}.
        \item An object~$\idmoncat \setin \ObC$, called the \emph{monoidal unit}.
        \item A \SY{natural isomorphism}, called the \emph{associator}, whose components are of the type
              \begin{equation}
                  \associator_{\Obja,\Objb,\Objc}\colon (\Obja \mtimescatob \Objb )\mtimescatob \Objc \mtoiso \Obja\mtimescatob (\Objb \mtimescatob \Objc) \quad \quad \Obja,\Objb,\Objc\setin \ObC.
              \end{equation}
        \item A \SY{natural isomorphism}, called the \emph{left unitor}, whose components are of the type
              \begin{equation}
                  \leftunitor_\Obja \colon \idmoncat\mtimescatob \Obja \mtoiso \Obja \quad \quad \Obja\setin \ObC.
              \end{equation}
        \item A \SY{natural isomorphism}, called the \emph{right unitor}, whose components are of the type
              \begin{equation}
                  \rightunitor_\Obja \colon \Obja \mtimescatob \idmoncat \mtoiso \Obja \quad \quad \Obja\setin \ObC.
              \end{equation}
    \end{enumerate}

    \condit\\
    For all~$\Obja,\Objb,\Objc,\Objd\setin \ObC$, the following diagrams must commute:
    \begin{enumerate}
        \item Triangle identities.
              \equationsag{30_triangle_identity}{eq:30_triangle_identity}
        \item Pentagon identities.
    \end{enumerate}
    %%
    \equationsag{30_pentagon_identity}{eq:30_pentagon_identity}
    A category equipped with a monoidal structure is called a \emph{monoidal category}.
    If the components of the associator, left unitor, and right unitor are all equalities, one calls the category \emph{strict} monoidal.
\end{ctdefinition}

\begin{remark}
    Note that in the constituents listed in \cref{def:monoidal-cat} we specified \SY{natural isomorphisms} $\associator$, $\leftunitor$, and $\rightunitor$ simply in terms of their components.
    You may be wondering: which \SY{functors} are the respective source and target of these natural transformations?
    Since it is a mouthful to write, this information is often left to be inferred from the components given.
    Let us quickly illustrate how to see, from the components, which \SY{functors} are involved.
    Take, for example, the left unitor.
    Its components are
    %
    \begin{equation}
        \leftunitor_\Obja \colon \idmoncat\mtimescatob \Obja \mtoiso \Obja \quad \quad \Obja\setin \ObC,
    \end{equation}
    %
    so, if $\funa$ and $\funb$ denote the \SY{functors} which are the source and target of $\leftunitor$, the \SY{functor} $\funa$ must act on objects by $\funa(\Obja) = \idmoncat \mtimescatob \Obja$ and $\funb$ must act by $\funb(\Obja) = \funa(\Obja)$.
    The ``obvious'' or ``canonical'' choice then (given that we are considering \emph{any} \SY{monoidal category}) is that $\funb$ is the \SY{identity functor} and that $\funa$ is the \SY{functor} which acts on morphisms by mapping $\mora \colon \Obja \mto \Objb$ to
    %
    \begin{equation}
        \catid_\idmoncat \mtimescatmor \mora \colon   \idmoncat\mtimescatob \Obja \mto  \idmoncat\mtimescatob \Objb.
    \end{equation}
    %
    Note that the components of the left unitor $\leftunitor$ are indexed by one variable $\Obja \setin \ObC$, while the associator $\associator$ is indexed by \emph{three} variables!
    The associator is therefore a \SY{natural transformation} between two \SY{functors} of the type
    %
    \begin{equation}
        \CatC \Ctimes \CatC \Ctimes \CatC \fto \CatC.
    \end{equation}
    %
    Can you guess which \SY{functors} of this type are meant in \cref{def:monoidal-cat} to be the source and target of $\associator$?
\end{remark}

\showslides{
    \begin{forslides}
        \begin{equation}
            \label{eq:set_mon_prod_0}
            \tup{\Set,\cartprod,\singleton}
        \end{equation}
        \begin{equation}
            \label{eq:set_mon_prod_1}
            \mapa\colon \setA\mto \stylesets{A'}, \quad \mapb\colon \setB\mto \stylesets{B'}
        \end{equation}
    \end{forslides}
}

\begin{example}
    We digest the definition of \SY{monoidal category} with an explanatory example.
    We consider the structure~$\tup{\Set,\cartprod,\singleton}$ and show that it indeed forms a \SY{monoidal category}.
    First, we specify how the monoidal product (\SY{cartesian product} here) acts on objects and morphisms in \Set (it is a \SY{functor}).
    Given~$\setA,\setB\setin \Obof{\Set}$,~$\setA\cartprod \setB$ is the \SY{cartesian product} of sets, and given~$\mapa\colon \setA\mto \stylesets{A'}$,~$\mapb\colon \setB\mto \stylesets{B'}$, we have:
    \begin{equation}
        \label{eq:set_mon_prod_2}
        \begin{aligned}
            (\mora \funcprod \morb)
            \colon \setA\cartprod \setB & \mtoiso \stylesets{A'}\cartprod \stylesets{B'} \\
            \tup{\setAel,\setBel}       & \mapsto \tup{\mapa(\setAel),\mapb(\setBel)}.
        \end{aligned}
    \end{equation}
    Furthermore, given any~$\setA,\setB,\setC\setin \Obof{\Set}$, we specify the associator~$\associator_{\setA,\setB,\setC}$:
    \begin{equation}
        \label{eq:set_mon_prod_3}
        \begin{aligned}
            \associator_{\setA,\setB,\setC}\colon (\setA\cartprod \setB)\cartprod \setC & \mto \setA\cartprod (\setB\cartprod \setC) \\
            \tup{\tup{\setAel,\setBel},\setCel}                                         & \mapsto \tup{\setAel, \tup{\setBel,\setCel}}
        \end{aligned}
    \end{equation}
    This defines an isomorphism (I can go ``back and forth'', by switching the tuple separation).
    We now need to check whether~$\associator$ is natural.
    We check this graphically, using the commutative diagram in \cref{fig:monoidal_set_ass_nat}.

    \begin{figure}[h!]
        \centering
        \includesag{115_set_mon_ass}
        \caption{}
        \label{fig:monoidal_set_ass_nat}
    \end{figure}

    Given an object~$\setA\setin \Obof\Set$, the unitor~$\leftunitor_\setA$ is given by:
    \begin{equation}
        \label{eq:set_mon_prod_4}
        \begin{aligned}
            \leftunitor_\setA\colon \singleton \cartprod \setA & \mtoiso \setA \\
            \tup{\singletonel,\setAel}                         & \mapsto \setAel.
        \end{aligned}
    \end{equation}

    Again, this defines an isomorphism.
    Consider a morphism~$\mapa\colon \setA\mto \stylesets{A'}$.
    We now prove naturality graphically (\cref{fig:monoidal_set_unit_nat}).

    \begin{figure}[h!]
        \centering
        \includesag{115_set_mon_unit_nat}
        \caption{}
        \label{fig:monoidal_set_unit_nat}
    \end{figure}

    Analogously, given an object~$\setA\setin \Obof{\Set}$, the unitor isomorphism~$\rightunitor_\setA$ is given by:
    \begin{equation}
        \label{eq:set_mon_prod_5}
        \begin{aligned}
            \rightunitor_\setA\colon \setA\cartprod \singleton & \mtoiso \setA \\
            \tup{\setAel,\singletonel}                         & \mapsto \setAel.
        \end{aligned}
    \end{equation}
    The proof for naturality is analogous to the one of~$\leftunitor_\setA$.
    We now need to check whether the triangle and pentagon identities are satisfied.
    We start by the triangle.
    Given~$\setA,\setB\setin \Obof{\Set}$, the proof is displayed in \cref{fig:set_mon_triangle}.

    \begin{figure}[h]
        \centering
        \includesag{115_set_mon_tri}
        \caption{}
        \label{fig:set_mon_triangle}
    \end{figure}

    We now prove the pentagon identity.
    Given~$\setA,\setB,\setC,\setD\setin \Obof\Set$, the proof is reported in \cref{fig:set_mon_pent}.

    \begin{figure*}[h]
        \centering
        \includesag{115_set_mon_pentagon}
        \caption{}
        \label{fig:set_mon_pent}
    \end{figure*}
\end{example}

\begin{remark}[We cannot define monoidal \emph{semi}-categories]
    Note that in \cref{def:monoidal-cat} we used the category's identities in the diagrams.
    It is not possible to ``patch'' the definition to work with \SY{semicategories}.
\end{remark}

\section{Examples of monoidal categories}

\todotextjira{358}{\alphubel: @JL: Introduce string diagrams}

\todotextjira{270}{\alphubel: @Gioele: Add example of \SY{monoidal categories} connected to dynamical systems.}

\begin{example}
    \label{ex:Vect-symmetric-monoidal}
    The category $\VectR$ is can be equipped with a monoidal structure where the monoidal product is the tensor product of real \SY{vector spaces}.
    It can also be equipped with a different monoidal structure where the monoidal product is the direct sum of real \SY{vector spaces}.
\end{example}

\todojira{188}{\bernina: @Gioele: Review/revisit example \cref{ex:robot}.}
\begin{example}[Robot configurations]
    \label{ex:robot}
    Consider~$\reals^2$, discretized as a two-dim\-ensional grid, representing locations (cells) which a robot can reach.
    The configuration space of the robot is $\reals^2\cartprod \Theta$, where $\Theta=[0,2\pi)$.
    A specific configuration $\tup{x,y,\theta}$ is characterized at each time by the position of the robot $x,y\setin \reals$ and its orientation $\theta \setin \Theta$.
    The action space of the robot is $\mathcal{A}=\makeset{\mathsf{stay},\leftarrow, \rightarrow, \uparrow, \downarrow}$.
    This is a category, where each configuration of the robot is an object, and morphisms are robot actions which change configurations.
    Each configuration has the \SY{identity morphism} which does not change it ($\mathsf{stay}$).
    Composition of morphisms is given by concatenation of actions (\cref{fig:robotcategory}).
    Assuming the existence of multiple robots $r_i=\tup{x_i,y_i,\theta_i}$, it is possible to define a product $r_i\mtimescat r_j$, which is to be intended as ``we have a robot at configuration $r_i$ and another one at configuration $r_j$''.
    However, this cannot be a proper monoidal product, because two robots cannot have the same configuration (physically, they cannot lie on each other), and hence $r_i\mtimescat r_i$ does not exist.
    By assuming that two robots could share the same configuration, this would be a valid monoidal product.
    \begin{figure}[tbh]
        \centering
        \includesag{120_robotcategory}
        \caption{Example of the robot category. }
        \label{fig:robotcategory}
    \end{figure}
\end{example}

\begin{publictodo}
    We will provide more examples of monoidal structures on the previous categories that we have seen.
\end{publictodo}

\begin{gradedexercise}[\exname{VectTensorMonStructure}]
    \label{ex:VectTensorMonStructure}
    What are straightforward choices of monoidal unit, associator, and left/right unitors which, together with the tensor product as monoidal product, equip $\VectR$ with a monoidal structure?

    In this exercise, simply write down how you think each of these pieces of data would be defined -- it is not asked that you prove that they do indeed form a monoidal structure (that would be much more involved).
\end{gradedexercise}

\todotextjira{187}{\bernina: @Gioele: Style of solutions is wrong, adjust}

\solutionof{VectTensorMonStructure}

\subsubsection*{\DP is a symmetric monoidal category}
\begin{lemma}
    \label{lem:symmetricmonoidaldp}
    For any~$\posA,\posB \setin\Obof\DP$, the \SY{design problem}~$\braiding_{\posA,\posB}\colon \F{\posgenA} \Ptimes \F{\posgenB} \profto \R{\posgenB} \Ctimes \R{\posgenA}$ given by
    \begin{equation}
        \braiding_{\posA,\posB}(\tup{\F{\posgenAel_1},\F{\posgenBel_1}}\Fop, \tup{\R{\posgenBel_2},\R{\posgenAel_2}})\definedas \pars{\F{\posgenAel_1}\posleqof\posA \R{\posgenAel_2}}\booland \pars{\F{\posgenBel_1}\posleqof\posB \R{\posgenBel_2}}
    \end{equation}
    constitutes the braiding operation for a symmetric monoidal structure on~$\tupp{\DP, \mtimescat,\singleton}$.
    In other words,~$\tup{\DP, \mtimescat, \singleton, \braiding}$ is a \SY{symmetric monoidal category}.
\end{lemma}

\begin{proof}
    In this proof, given two elements~$\posAel_1,\posBel_2$ of a poset~\posA, we denote for brevity~$\posAel_\styleelements{1} \posleqof\posA \posAel_\styleelements{2}$ by~$\posAel_\styleelements{1} \posleq \posAel_\styleelements{2}$.
    To prove that~$\braiding_{\posA,\posB}$ is an isomorphism, we use \cref{def:monoidal-cat} and show~$\braiding_{\posA,\posB}\mthen \braiding_{\posB,\posA}=\catidat{\posA\Ptimes \posB}$.
    We have
    \begin{equation}
        \begin{aligned}
             & \pars{ \braiding_{\posA,\posB}\fthen \braiding_{\posB,\posA}} \pars{ \tup{\F{\posgenAel_1},\F{\posgenBel_1}}\Fop, \tup{\R{\posgenAel_2},\R{\posgenBel_2}}} \\
             & =\bigvee_{\tup{\posBel,\posAel}\setin \posB\Ptimes \posA}\braiding_{\posA,\posB}(\tup{\F{\posgenAel_1},\F{\posgenBel_1}}\Fop, \tup{\RposgenBel,\R{\posgenAel}})\booland \braiding_{\posB,\posA}(\tup{\FposgenBel,\FposgenAel}\Fop, \tup{\R{\posgenAel_2},\R{\posgenBel_2}}) \\
             & =\pars{ (\F{\posgenAel_1}\posleq \R{\posgenAel}) \booland (\F{\posgenBel_1}\posleq \RposgenBel)}\booland \pars{(\FposgenAel\posleq \R{\posgenAel_2}) \booland (\FposgenBel\posleq \R{\posgenBel_2})} \\
             & =(\F{\posgenAel_1}\posleq \R{\posgenAel_2})\booland (\F{\posgenBel_1}\posleq \R{\posgenBel_2}) \\
             & =\catidat{\posA\Ptimes \posB}(\tup{\F{\posgenAel_1},\F{\posgenBel_1}}\Fop, \tup{\R{\posgenAel_2},\R{\posgenBel_2}}).
        \end{aligned}
    \end{equation}
    This also shows the second triangle identity:~$\braiding_{\posA,\posB}$ is its own identity.
    For naturality, consider two morphisms (design problems)~$\adpa\colon \funposA_\F{1}\profto \resposB_\R{1}$ and $\adpb\colon \funposA_\F{2}\profto \resposB_\R{2}$.
    For brevity, denote~$\braiding_{\posB_\stylepos{1}\cartprod \posB_\stylepos{2},\posB_\stylepos{2}\cartprod \posB_\stylepos{1}}$ by~$\braiding_\posB$ and~$\braiding_{\posA_\stylepos{1}\cartprod \posA_\stylepos{2},\posA_\stylepos{2}\cartprod \posA_\stylepos{1}}$ by~$\braiding_\posA$.
    We have
    \begin{equation}
        \begin{aligned}
             & \pars{(\adpa\mtimescat \adpb)\fthen \braiding_\posgenB }\pars{ \tup{\F{\posgenAel_1},\F{\posgenAel_2}}\Fop, \tup{\R{\posgenBel_2},\R{\posgenBel_1}}} \\
             & =\bigvee_{\tup{\posBel,\posBel\elprime}\setin \posB_\stylepos{1}\cartprod \posB_\stylepos{2}} \pars{\adpa\mtimescat \adpb} \pars{ \tup{\F{\posgenAel_1},\F{\posgenAel_2}}\Fop, \tup{\RposgenBel,\R{\posgenBel'}}}\booland \braiding_\posB\pars{\tup{\FposgenBel,\F{\posgenBel'}}\Fop, \tup{\R{\posgenBel_2},\R{\posgenBel_1}} } \\
             & =\bigvee_{\tup{\posBel,\posBel\elprime}\setin \posB_\stylepos{1}\cartprod \posB_\stylepos{2}}(\adpa(\F{\posgenAel_1^*},\RposgenBel)\booland \adpb(\F{\posgenAel_2^*},\R{\posgenBel'}))\booland (\pars{\FposgenBel\posleq \R{\posgenBel_1}} \booland \pars{\F{\posgenBel'}\posleq \R{\posgenBel_2}}) \\
             & = \adpa(\F{\posgenAel_1^*},\R{\posgenBel_1}) \booland \adpb(\F{\posgenAel_2^*},\R{\posgenBel_2}),
        \end{aligned}
    \end{equation}
    where the last step comes from the monotonicity of~$\adpa$ and~$\adpb$.
    Similarly,
    \begin{equation}
        \begin{aligned}
             & \pars{ \braiding_\posA \fthen (\adpb\mtimescat \adpa)}\pars{ \tup{\F{\posgenAel_1},\F{\posgenAel_2}}\Fop, \tup{\R{\posgenBel_2},\R{\posgenBel_1}}} \\
             & =\bigvee_{\tup{\posAel,\posAel\elprime}\setin \posA_\stylepos{2}\cartprod \posA_\stylepos{1}}\braiding_\posgenA\pars{\tup{\F{\posgenAel_1},\F{\posgenAel_2}}\Fop, \tup{\R{\posgenAel},\R{\posgenAel}'} }\booland \pars{\adpb\mtimescat \adpa} \pars{ \tup{\FposgenAel,\F{\posgenAel'}}\Fop, \tup{\R{\posgenBel_2},\R{\posgenBel_1}}} \\
             & =\bigvee_{\tup{\posAel,\posAel\elprime}\setin \posA_\stylepos{2}\cartprod \posA_\stylepos{1}}(\pars{\F{\posgenAel_1}\posleq \R{\posgenAel'}}\booland \pars{\F{\posgenAel_2}\posleq \R{\posgenAel}}) \booland (\adpb(\FposgenAelop,\R{\posgenBel_2})\booland \adpa(\F{\posgenAel'^*},\R{\posgenBel_1})) \\
             & = \adpb(\F{\posgenAel_1^*},\R{\posgenBel_1}) \booland \adpa(\R{\posgenAel_2^*},\R{\posgenBel_2}).
        \end{aligned}
    \end{equation}
    To show the first triangle identity, we write
    \begin{equation}
        \begin{aligned}
             & \pars{\braiding_{\singleton \cartprod \posA}\fthen \rightunitor_\posA}\pars{ \tup{\F{\singletonel},\F{\posgenAel_1}}\Fop,\R{\posgenAel_2}} \\
             & =\bigvee_{\tup{\F{\posgenAel'},\R{\singletonel}}\setin \posA\Ptimes \singleton}\braiding_{\singleton \cartprod \posA}\pars{ \tup{\F{\singletonel},\F{\posgenAel_1}}\Fop, \tup{\R{\posgenAel'},\R{\singletonel}}}\booland \rightunitor_\posA\pars{ \tup{\F{\posgenAel'},\F{\singletonel}}\Fop,\R{\posgenAel_2}} \\
             & =\bigvee_{\tup{\posAel\elprime,\singletonel}\setin \posA\Ptimes \singleton} \pars{\F{\singletonel}\posleq \R{\singletonel}} \booland \pars{\F{\posgenAel_1}\posleq \R{\posgenAel'}}\booland \pars{\F{\posgenAel'}\posleq \R{\posgenAel_2}} \\
             & =\F{\posgenAel_1}\posleq \R{\posgenAel_2} \\
             & =\leftunitor_\posA\pars{ \tup{\F{\singletonel},\F{\posgenAel_1}}\Fop,\R{\posgenAel_2}}.
        \end{aligned}
    \end{equation}
    The hexagon identities are more verbose.
    Consider~$\posA,\posB,\posC\setin \Obof\DP$.
    For brevity, we denote~$\associator_{\posA,\posB,\posC}$ by~$\associator$,~$\braiding_{\posA,\posB}\mtimescatmor \catid_\posC$ by~$\braiding'$,~$\catid_\posB \mtimescatmor \braiding_{\posA,\posC}$ by~$\braiding''$,~$(\posB\Ptimes \posA)\Ptimes \posC$ as~$\Diamond$, and~$\posB\Ctimes (\posA\Ptimes \posC)$ as~$\Delta$.

    Recall that
    \begin{equation}
        \begin{aligned}
            \braiding' \pars{\tup{\tup{\F{\posgenAel_1},\F{\posgenBel_1}},\F{\posgenCel_1}}\Fop, \tup{\R{\posgenBel_2}, \tup{\R{\posgenAel_2},\R{\posgenCel_2}}} } & =
            \pars{ (\F{\posgenAel_1}\posleq \R{\posgenAel_2})  \booland (\F{\posgenBel_1}\posleq \R{\posgenBel_2})}\booland (\F{\posgenCel_1}\posleq \R{\posgenCel_2}) \\
                                                                                                                                                                   & = (\F{\posgenAel_1}\posleq \R{\posgenAel_2})  \booland (\F{\posgenBel_1}\posleq \R{\posgenBel_2}) \booland (\F{\posgenCel_1}\posleq \R{\posgenCel_2}).
        \end{aligned}
    \end{equation}
    We have
    \begin{equation}
        \begin{aligned}
             & \pars{\braiding' \fthen \associator } \pars{\tup{\tup{\F{\posgenAel_1},\F{\posgenBel_1}},\F{\posgenCel_1}}\Fop, \tup{\R{\posgenBel_2}, \tup{\R{\posgenAel_2},\R{\posgenCel_2}}} } \\
             & =\bigvee_{\tup{\tup{\posBel,\posAel},\posCel}\setin \Diamond}\braiding' \pars{ \tup{\tup{\F{\posgenAel_1},\F{\posgenBel_1}},\F{\posgenCel_1}}\Fop, \tup{\tup{\RposgenBel,\R{\posgenAel}},\R{\posgenCel}}}\booland \associator \pars{ \tup{\tup{\FposgenBel,\FposgenAel},\F{\posgenCel}}\Fop, \tup{\R{\posgenBel_2}, \tup{\R{\posgenAel_2},\R{\posgenCel_2}}}} \\
             & =\bigvee_{\tup{\tup{\posBel,\posAel},\posCel}\setin \Diamond} \pars{\pars{\F{\posgenAel_1}\posleq \R{\posgenAel} }\booland \pars{ \F{\posgenBel_1}\posleq \RposgenBel}\booland \pars{\F{\posgenCel_1}\posleq \R{\posgenCel}}}\booland \\
             & \pars{\pars{\FposgenBel\posleq \R{\posgenBel_2}}\booland \pars{ \FposgenAel\posleq \R{\posgenAel_2}}\booland \pars{\F{\posgenCel}\posleq \R{\posgenCel_2}}} \\
             & =\pars{\F{\posgenBel_1}\posleq \R{\posgenBel_2} }\booland \pars{\F{\posgenAel_1}\posleq \R{\posgenAel_2} }\booland \pars{ \F{\posgenCel_1}\posleq \R{\posgenCel_2}} \\
             & =\underbrace{\associator\pars{\tup{\tup{\F{\posgenBel_1},\F{\posgenAel_1}},\F{\posgenCel_1}}^*, \tup{\R{\posgenBel_2}, \tup{\R{\posgenAel_2},\R{\posgenCel_2}}}}}_{\star}.
        \end{aligned}
    \end{equation}
    Furthermore, consider
    \begin{equation}
        \begin{aligned}
             & \pars{ \star \fthen \braiding''}\pars{ \tup{\tup{\F{\posgenAel_1},\F{\posgenBel_1}},\F{\posgenCel_1}}\Fop, \tup{\R{\posgenBel_3}, \tup{\R{\posgenCel_3},\R{\posgenAel_3}}}} \\
             & =\bigvee_{\tup{\posBel_\styleelements{2}, \tup{\posAel_\styleelements{2},\posCel_\styleelements{2}}}\setin \Delta} \star \pars{\tup{\tup{\F{\posgenAel_1},\F{\posgenBel_1}},\F{\posgenCel_1}}\Fop, \tup{\R{\posgenBel_2}, \tup{\R{\posgenAel_2},\R{\posgenCel_2}}} }\booland \\
             & \braiding'' \pars{\tup{\F{\posgenBel_2}, \tup{\F{\posgenAel_2},\F{\posgenCel_2}}}^*, \tup{\R{\posgenBel_3}, \tup{\R{\posgenCel_3},\R{\posgenAel_3}}} } \\
             & =\bigvee_{\tup{\posgenBel_\styleelements{2}, \tup{\posgenAel_\styleelements{2},\posgenCel_\styleelements{2}}}\setin \Delta}(\pars{\F{\posgenBel_1}\posleq \R{\posgenBel_2} }\booland \pars{\F{\posgenAel_1}\posleq \R{\posgenAel_2} }\booland \pars{ \F{\posgenCel_1}\posleq \R{\posgenCel_2}}) \booland \\
             & \pars{\pars{\F{\posgenBel_2}\posleq \R{\posgenBel_3}}\booland \pars{\F{\posgenAel_2}\posleq \R{\posgenAel_3}} \booland \pars{\F{\posgenCel_2}\posleq \R{\posgenCel_3}}} \\
             & =(\F{\posgenAel_1}\posleq \R{\posgenAel_3}) \booland (\F{\posgenBel_1}\posleq \R{\posgenBel_3}) \booland (\F{\posgenCel_1}\posleq \R{\posgenCel_3}).
        \end{aligned}
    \end{equation}
    It is easy to see that the other direction in the hexagon commutative diagram commutes as well.
    With this we have proved that~$\braiding$ is a valid braiding operation and hence that~$\tup{\DP, \mtimescat, \singleton, \braiding}$ is a \SY{symmetric monoidal category}.
\end{proof}
