% !TEX root = chapter-standalone.tex

\section{Monoidal categories}
\label{sec:parallelism-mon-cat}

\todotext{\alphubel: Need to redo introduction here}
% We now generalize from \SY{monoidal posets} to \SY{monoidal categories}.
%
% So far, we have described a single way to compose morphisms of a category: the~$\mthen$ operation.
% However, category theory allows defining other ways of composing morphisms, adding structure to the basic category defined in~\cref{def:categorymain}.

\linkvideo{spring2021-par-feedback:mon-cat:mon-cat-def} % Definition of monoidal categories
\begin{ctdefinition}[Monoidal category]
    \label{def:monoidal-cat}
    \SYNDEF{monoidal category}
    A \emph{monoidal structure} on a category~\CatC is specified by:

    \constit
    \begin{enumerate}
        \item A \SY{functor}~$\mtimescat \colon \CatC \Ctimes \CatC\fto \CatC$, called the \emph{monoidal product}.
        \item An object~$\idmoncat \setin \ObC$, called the \emph{monoidal unit}.
        \item A \SY{natural isomorphism}, called the \emph{associator}, whose components are of the type
              \begin{equation}
                  \associator_{\Obja,\Objb,\Objc}\colon (\Obja \mtimescatob \Objb )\mtimescatob \Objc \mtoiso \Obja\mtimescatob (\Objb \mtimescatob \Objc) \quad \quad \Obja,\Objb,\Objc\setin \ObC.
              \end{equation}
        \item A \SY{natural isomorphism}, called the \emph{left unitor}, whose components are of the type
              \begin{equation}
                  \leftunitor_\Obja \colon \idmoncat\mtimescatob \Obja \mtoiso \Obja \quad \quad \Obja\setin \ObC.
              \end{equation}
        \item A \SY{natural isomorphism}, called the \emph{right unitor}, whose components are of the type
              \begin{equation}
                  \rightunitor_\Obja \colon \Obja \mtimescatob \idmoncat \mtoiso \Obja \quad \quad \Obja\setin \ObC.
              \end{equation}
    \end{enumerate}

    \condit\\
    For all~$\Obja,\Objb,\Objc,\Objd\setin \ObC$, the following diagrams must commute:
    \begin{enumerate}
        \item Triangle identities.
              \equationsag{30_triangle_identity}{eq:30_triangle_identity}
        \item Pentagon identities.
    \end{enumerate}
    %%
    \equationsag{30_pentagon_identity}{eq:30_pentagon_identity}
    A category equipped with a monoidal structure is called a \emph{monoidal category}.
    If the components of the associator, left unitor, and right unitor are all equalities, one calls the category \emph{strict} monoidal.
\end{ctdefinition}

\begin{remark}
    Note that in the constituents listed in \cref{def:monoidal-cat} we specified \SY{natural isomorphisms} $\associator$, $\leftunitor$, and $\rightunitor$ simply in terms of their components.
    You may be wondering: which \SY{functors} are the respective source and target of these natural transformations?
    Since it is a mouthful to write, this information is often left to be inferred from the components given.
    Let us quickly illustrate how to see, from the components, which \SY{functors} are involved.
    Take, for example, the left unitor.
    Its components are
    %
    \begin{equation}
        \leftunitor_\Obja \colon \idmoncat\mtimescatob \Obja \mtoiso \Obja \quad \quad \Obja\setin \ObC,
    \end{equation}
    %
    so, if $\funa$ and $\funb$ denote the \SY{functors} which are the source and target of $\leftunitor$, the \SY{functor} $\funa$ must act on objects by $\funa(\Obja) = \idmoncat \mtimescatob \Obja$ and $\funb$ must act by $\funb(\Obja) = \Obja$.
    The ``obvious'' or ``canonical'' choice then (given that we are considering \emph{any} \SY{monoidal category}) is that $\funb$ is the \SY{identity functor} and that $\funa$ is the \SY{functor} which acts on morphisms by mapping $\mora \colon \Obja \mto \Objb$ to
    %
    \begin{equation}
        \catid_\idmoncat \mtimescatmor \mora \colon \idmoncat\mtimescatob \Obja \mto \idmoncat\mtimescatob \Objb.
    \end{equation}
    %
    Note that the components of the left unitor $\leftunitor$ are indexed by one variable $\Obja \setin \ObC$, while the associator $\associator$ is indexed by \emph{three} variables!
    The associator is therefore a \SY{natural transformation} between two \SY{functors} of the type
    %
    \begin{equation}
        \CatC \Ctimes \CatC \Ctimes \CatC \fto \CatC.
    \end{equation}
    %
    Can you guess which \SY{functors} of this type are meant in \cref{def:monoidal-cat} to be the source and target of $\associator$?
\end{remark}

\showslides{
    \begin{forslides}
        \begin{equation}
            \label{eq:set_mon_prod_0}
            \tup{\Set,\cartprod,\singleton}
        \end{equation}
        \begin{equation}
            \label{eq:set_mon_prod_1}
            \mapa\colon \setA\mto \stylesets{A'}, \quad \mapb\colon \setB\mto \stylesets{B'}
        \end{equation}
    \end{forslides}
}

\begin{example}
    We digest the definition of \SY{monoidal category} with an explanatory example.
    We consider the structure~$\tup{\Set,\cartprod,\singleton}$ and show that it indeed forms a \SY{monoidal category}.
    First, we specify how the monoidal product (\SY{cartesian product} here) acts on objects and morphisms in \Set (it is a \SY{functor}).
    Given~$\setA,\setB\setin \Obof{\Set}$,~$\setA\cartprod \setB$ is the \SY{cartesian product} of sets, and given~$\mapa\colon \setA\mto \stylesets{A'}$,~$\mapb\colon \setB\mto \stylesets{B'}$, we have:
    \begin{equation}
        \label{eq:set_mon_prod_2}
        \begin{aligned}
            (\mora \funcprod \morb)
            \colon \setA\cartprod \setB & \mtoiso \stylesets{A'}\cartprod \stylesets{B'} \\
            \tup{\setAel,\setBel}       & \mapsto \tup{\mapa(\setAel),\mapb(\setBel)}.
        \end{aligned}
    \end{equation}
    Furthermore, given any~$\setA,\setB,\setC\setin \Obof{\Set}$, we specify the associator~$\associator_{\setA,\setB,\setC}$:
    \begin{equation}
        \label{eq:set_mon_prod_3}
        \begin{aligned}
            \associator_{\setA,\setB,\setC}\colon (\setA\cartprod \setB)\cartprod \setC & \mto \setA\cartprod (\setB\cartprod \setC) \\
            \tup{\tup{\setAel,\setBel},\setCel}                                         & \mapsto \tup{\setAel, \tup{\setBel,\setCel}}
        \end{aligned}
    \end{equation}
    This defines an isomorphism (I can go ``back and forth'', by switching the tuple separation).
    We now need to check whether~$\associator$ is natural.
    We check this graphically, using the commutative diagram in \cref{fig:monoidal_set_ass_nat}.

    \begin{figure}[h!]
        \centering
        \includesag{115_set_mon_ass}
        \caption{}
        \label{fig:monoidal_set_ass_nat}
    \end{figure}

    Given an object~$\setA\setin \Obof\Set$, the unitor~$\leftunitor_\setA$ is given by:
    \begin{equation}
        \label{eq:set_mon_prod_4}
        \begin{aligned}
            \leftunitor_\setA\colon \singleton \cartprod \setA & \mtoiso \setA \\
            \tup{\singletonel,\setAel}                         & \mapsto \setAel.
        \end{aligned}
    \end{equation}

    Again, this defines an isomorphism.
    Consider a morphism~$\mapa\colon \setA\mto \stylesets{A'}$.
    We now prove naturality graphically (\cref{fig:monoidal_set_unit_nat}).

    \begin{figure}[h!]
        \centering
        \includesag{115_set_mon_unit_nat}
        \caption{}
        \label{fig:monoidal_set_unit_nat}
    \end{figure}

    Analogously, given an object~$\setA\setin \Obof{\Set}$, the unitor isomorphism~$\rightunitor_\setA$ is given by:
    \begin{equation}
        \label{eq:set_mon_prod_5}
        \begin{aligned}
            \rightunitor_\setA\colon \setA\cartprod \singleton & \mtoiso \setA \\
            \tup{\setAel,\singletonel}                         & \mapsto \setAel.
        \end{aligned}
    \end{equation}
    The proof for naturality is analogous to the one of~$\leftunitor_\setA$.
    We now need to check whether the triangle and pentagon identities are satisfied.
    We start by the triangle.
    Given~$\setA,\setB\setin \Obof{\Set}$, the proof is displayed in \cref{fig:set_mon_triangle}.

    \begin{figure}[h]
        \centering
        \includesag{115_set_mon_tri}
        \caption{}
        \label{fig:set_mon_triangle}
    \end{figure}

    We now prove the pentagon identity.
    Given~$\setA,\setB,\setC,\setD\setin \Obof\Set$, the proof is reported in \cref{fig:set_mon_pent}.

    \begin{figure*}[h]
        \centering
        \includesag{115_set_mon_pentagon}
        \caption{}
        \label{fig:set_mon_pent}
    \end{figure*}
\end{example}

\begin{remark}[We cannot define monoidal \emph{semi}-categories]
    Note that in \cref{def:monoidal-cat} we used the category's identities in the diagrams.
    It is not possible to ``patch'' the definition to work with \SY{semicategories}.
\end{remark}

\section{Examples of monoidal categories}

\todotextjira{358}{\alphubel: @JL: Introduce string diagrams}

\todotextjira{270}{\alphubel: @Gioele: Add example of \SY{monoidal categories} connected to dynamical systems.}

\begin{example}
    \label{ex:Vect-symmetric-monoidal}
    The category $\VectR$ is can be equipped with a monoidal structure where the monoidal product is the tensor product of real \SY{vector spaces}.
    It can also be equipped with a different monoidal structure where the monoidal product is the direct sum of real \SY{vector spaces}.
\end{example}

\todotext{@J: write up the definition of the tensor product for vector spaces in detail}

\todojira{188}{\bernina: @Gioele: Review/revisit example \cref{ex:robot}.}
\begin{example}[Robot configurations]
    \label{ex:robot}
    Consider~$\reals^2$, discretized as a two-dim\-ensional grid, representing locations (cells) which a robot can reach.
    The configuration space of the robot is $\reals^2\cartprod \Theta$, where $\Theta=[0,2\pi)$.
    A specific configuration $\tup{x,y,\theta}$ is characterized at each time by the position of the robot $x,y\setin \reals$ and its orientation $\theta \setin \Theta$.
    The action space of the robot is $\mathcal{A}=\makeset{\mathsf{stay},\leftarrow, \rightarrow, \uparrow, \downarrow}$.
    This is a category, where each configuration of the robot is an object, and morphisms are robot actions which change configurations.
    Each configuration has the \SY{identity morphism} which does not change it ($\mathsf{stay}$).
    Composition of morphisms is given by concatenation of actions (\cref{fig:robotcategory}).
    Assuming the existence of multiple robots $r_i=\tup{x_i,y_i,\theta_i}$, it is possible to define a product $r_i\mtimescat r_j$, which is to be intended as ``we have a robot at configuration $r_i$ and another one at configuration $r_j$''.
    However, this cannot be a proper monoidal product, because two robots cannot have the same configuration (physically, they cannot lie on each other), and hence $r_i\mtimescat r_i$ does not exist.
    By assuming that two robots could share the same configuration, this would be a valid monoidal product.
    \begin{figure}[tbh]
        \centering
        \includesag{120_robotcategory}
        \caption{Example of the robot category. }
        \label{fig:robotcategory}
    \end{figure}
\end{example}

\begin{publictodo}
    We will provide more examples of monoidal structures on the previous categories that we have seen.
\end{publictodo}

\begin{gradedexercise}[\exname{VectTensorMonStructure}]
    \label{ex:VectTensorMonStructure}
    What are straightforward choices of monoidal unit, associator, and left/right unitors which, together with the tensor product as monoidal product, equip $\VectR$ with a monoidal structure?

    In this exercise, simply write down how you think each of these pieces of data would be defined -- it is not asked that you prove that they do indeed form a monoidal structure (that would be much more involved).
\end{gradedexercise}

\todotextjira{187}{\bernina: @Gioele: Style of solutions is wrong, adjust}

\solutionof{VectTensorMonStructure}

