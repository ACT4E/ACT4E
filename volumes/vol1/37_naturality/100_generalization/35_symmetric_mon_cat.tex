% !TEX root = chapter-standalone.tex

\linkvideo{spring2021-par-feedback:braided-moncat} % Braided monoidal categories

\begin{ctdefinition}[\iindex{Braided monoidal category}]
    \label{def:braided_moncat}
    A \emph{braided monoidal category} is a monoidal category~$\tupp{\CatC,\mtimescat, \idmoncat, \associator, \leftunitor, \rightunitor}$ equipped with a \emph{braiding}, which is specified by

    \constit
    \begin{enumerate}
        \item A natural isomorphism~$\braiding$, called the braiding, whose components are of the type
              \begin{equation}
                  \label{eq:braiding_symmetry}
                  \braiding_{\Obja,\Objb}\colon (\Obja\mtimescat \Objb)\mtoiso (\Objb\mtimescat \Obja), \quad \Obja,\Objb \setin \ObC.
              \end{equation}
              Explicitly, this means that for any~$\mora_1\colon \Obja_1\mto \Objb_1$ and~$\mora_2\colon \Obja_2\mto \Objb_2$, the following diagram
              \begin{center}
                  \includesag{50_sym_1}
              \end{center}
    \end{enumerate}

    \condit
    \begin{enumerate}
        \item \emph{Hexagon identities:} Given any objects~$\Obja,\Objb,\Objc\setin \ObC$, the following diagrams must commute.
    \end{enumerate}
    \begin{center}
        \includesag{50_hex_1}
    \end{center}
    \begin{center}
        \includesag{50_hex_2}
    \end{center}
\end{ctdefinition}

\begin{remark}
    If $\tup{\CatC,\mtimescat, \idmoncat, \associator, \leftunitor, \rightunitor, \braiding}$ is a braided monoidal category, one can show that the following diagram commutes for all~$\Obja \setin \ObC$.
    \equationsag{50_sym_2}{eq:50_sym_2}
\end{remark}

\section{Symmetric monoidal categories}

\linkvideo{spring2021-par-feedback:symmetric-moncat} % Symmetric monoidal categories
\begin{ctdefinition}[\iindex{Symmetric monoidal category}]
    \label{def:sym-mon-cat}
    A \emph{symmetric monoidal category} is a braided monoidal category~$\tupp{\CatC,\mtimescat, \idmoncat, \associator, \leftunitor, \rightunitor, \braiding}$ for which the braiding satisfies the symmetry condition
    \begin{equation}
        \label{eq:braiding-symmetry}
        \braiding_{\Obja,\Objb} \mthen \braiding_{\Objb,\Obja} = \catid_{\Obja \mtimescat \Objb}
    \end{equation}
    for all~$\Obja, \Objb \setin \ObC$.
\end{ctdefinition}

\begin{remark}
    If~$\braiding$ is a natural isomorphism such that it is a candidate to be a braiding on a given monoidal category, and if, additionally, it satisfies \cref{eq:braiding_symmetry}, then the two hexagon identities are equivalent, and so only one of them needs to be checked.
\end{remark}
