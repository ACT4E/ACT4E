% !TEX root = chapter-standalone.tex

\section{Symmetric monoidal categories}

\linkvideo{spring2021-par-feedback:braided-moncat} % Braided monoidal categories

\begin{ctdefinition}[Braided monoidal category]
    \label{def:braided-monoidal-category}
    A \maindef{braided monoidal category} is a \SY{monoidal category}~$\tupp{\CatC, \mtimescat, \idmoncat, \associator, \leftunitor, \rightunitor}$ eq\-uip\-ped with a \emph{braiding}, which is specified by

    \constit
    \begin{enumerate}
        \item A \SY{natural isomorphism}~$\braiding$, called the \maindef{braiding}, whose components are of the type \label{def:braiding}
              \begin{equation}
                  \label{eq:braiding_symmetry}
                  \braiding_{\Obja,\Objb}\colon (\Obja\mtimescatob \Objb)\mtoiso (\Objb\mtimescatob \Obja), \quad \Obja,\Objb \setin \ObC.
              \end{equation}
              Explicitly, this means that for any~$\mora_1\colon \Obja_1\mto \Objb_1$ and~$\mora_2\colon \Obja_2\mto \Objb_2$, the following diagram commutes:
              \equationsag{50_sym_1}{eq:50_sym_1}
    \end{enumerate}

    \condit
    \begin{enumerate}
        \item \emph{Hexagon identities:}
              Given any objects~$\Obja,\Objb,\Objc\setin \ObC$, the following diagrams must commute.
    \end{enumerate}
    \equationsag{50_hex_1}{eq:50_hex_1}
    \equationsag{50_hex_2}{eq:50_hex_2}
\end{ctdefinition}

\begin{remark}
    If $\tup{\CatC,\mtimescat, \idmoncat, \associator, \leftunitor, \rightunitor, \braiding}$ is a \SY{braided monoidal category}, we can show that the following diagram commutes for all~$\Obja \setin \ObC$.
    \equationsag{50_sym_2}{eq:50_sym_2}
\end{remark}

\linkvideo{spring2021-par-feedback:symmetric-moncat} % Symmetric monoidal categories
\begin{ctdefinition}[Symmetric monoidal category]
    \label{def:sym-mon-cat}
    A \maindef{symmetric monoidal category} is a \SY{braided monoidal category} $\tupp{\CatC, \mtimescat, \idmoncat, \associator, \leftunitor, \rightunitor, \braiding}$ for which the braiding satisfies the symmetry condition
    \begin{equation}
        \label{eq:braiding-symmetry}
        \braiding_{\Obja,\Objb} \mthen \braiding_{\Objb,\Obja} = \catidat{\Obja \mtimescatob \Objb}
    \end{equation}
    for all~$\Obja, \Objb \setin \ObC$.
\end{ctdefinition}

\begin{remark}
    If~$\braiding$ is a \SY{natural isomorphism} such that it is a candidate to be a braiding on a given \SY{monoidal category}, and if, additionally, it satisfies \cref{eq:braiding-symmetry}, then the two hexagon identities are equivalent, and so only one of them needs to be checked.
\end{remark}

\subsubsection*{\DP is a symmetric monoidal category}

We define a monoidal product $\mtimescat$ for $\DP$ on objects by
\begin{equation}\label{eq:DP-mon-prod-objects}
    \posA \mtimescat \posB = \posA \Ptimes \posB
\end{equation}
and on morphisms by
\begin{equation}
    \label{eq:dp-mon-prod-morphisms}
    \prfperiod{
        \adpa\colon \posAop\Ptimes \posC \toinPos \Bool
    }{\qquad}{
        \adpb\colon \posB\posop\Ptimes \posD \toinPos \Bool
    }{
        \defmapcomma{
            \adpa \mtimescat \adpb
        }{
            (\posA \Ptimes \posB)\op
            \Ptimes
            \pars{\posC \Ptimes \posD}
        }{
            \toinPos
        }{
            \Bool
        }{
            \tup{\tup{a, c}^*, \tup{b, d}}
        }{
            \adpa(a^*, b) \booland \adpb(c^*, d)
        }
    }
\end{equation}

\begin{lemma}
    \label{lem:symmetricmonoidaldp}
    There is a \SY{symmetric monoidal category} $\tup{\DP, \mtimescat, \singleton, \associator, \leftunitor, \rightunitor, \braiding}$ where the braiding is given by the \SY{design problem}~$\braiding_{\posA,\posB}\colon \F{\posgenA} \Ptimes \F{\posgenB} \profto \R{\posgenB} \Ctimes \R{\posgenA}$ with
    \begin{equation}
        \braiding_{\posA,\posB}(\tup{\F{\posgenAel_1},\F{\posgenBel_1}}\Fop, \tup{\R{\posgenBel_2},\R{\posgenAel_2}})\definedas \pars{\F{\posgenAel_1}\posleqof\posA \R{\posgenAel_2}}\booland \pars{\F{\posgenBel_1}\posleqof\posB \R{\posgenBel_2}}
    \end{equation}
    for any~$\posA,\posB \setin\Obof\DP$.
\end{lemma}

\devel{

    \todotext{J: various things with the below proof are strange / fishy!}
    \begin{proof}
        In this proof, given two elements~$\posAel_1,\posBel_2$ of a poset~\posA, we denote for brevity~$\posAel_\styleelements{1} \posleqof\posA \posAel_\styleelements{2}$ by~$\posAel_\styleelements{1} \posleq \posAel_\styleelements{2}$.
        To prove that~$\braiding_{\posA,\posB}$ is an isomorphism, we use \cref{def:monoidal-cat} and show~$\braiding_{\posA,\posB}\mthen \braiding_{\posB,\posA}=\catidat{\posA\Ptimes \posB}$.
        We have
        \begin{equation}
            \begin{aligned}
                 & \pars{ \braiding_{\posA,\posB}\fthen \braiding_{\posB,\posA}} \pars{ \tup{\F{\posgenAel_1},\F{\posgenBel_1}}\Fop, \tup{\R{\posgenAel_2},\R{\posgenBel_2}}} \\
                 & =\bigvee_{\tup{\posBel,\posAel}\setin \posB\Ptimes \posA}\braiding_{\posA,\posB}(\tup{\F{\posgenAel_1},\F{\posgenBel_1}}\Fop, \tup{\RposgenBel,\R{\posgenAel}})\booland \braiding_{\posB,\posA}(\tup{\FposgenBel,\FposgenAel}\Fop, \tup{\R{\posgenAel_2},\R{\posgenBel_2}}) \\
                 & =\pars{ (\F{\posgenAel_1}\posleq \R{\posgenAel}) \booland (\F{\posgenBel_1}\posleq \RposgenBel)}\booland \pars{(\FposgenAel\posleq \R{\posgenAel_2}) \booland (\FposgenBel\posleq \R{\posgenBel_2})} \\
                 & =(\F{\posgenAel_1}\posleq \R{\posgenAel_2})\booland (\F{\posgenBel_1}\posleq \R{\posgenBel_2}) \\
                 & =\catidat{\posA\Ptimes \posB}(\tup{\F{\posgenAel_1},\F{\posgenBel_1}}\Fop, \tup{\R{\posgenAel_2},\R{\posgenBel_2}}).
            \end{aligned}
        \end{equation}
        This also shows the second triangle identity:~$\braiding_{\posA,\posB}$ is its own identity.
        For naturality, consider two morphisms (design problems)~$\adpa\colon \funposA_\F{1}\profto \resposB_\R{1}$ and $\adpb\colon \funposA_\F{2}\profto \resposB_\R{2}$.
        For brevity, denote~$\braiding_{\posB_\stylepos{1}\cartprod \posB_\stylepos{2},\posB_\stylepos{2}\cartprod \posB_\stylepos{1}}$ by~$\braiding_\posB$ and~$\braiding_{\posA_\stylepos{1}\cartprod \posA_\stylepos{2},\posA_\stylepos{2}\cartprod \posA_\stylepos{1}}$ by~$\braiding_\posA$.
        We have
        \begin{equation}
            \begin{aligned}
                 & \pars{(\adpa\mtimescat \adpb)\fthen \braiding_\posgenB }\pars{ \tup{\F{\posgenAel_1},\F{\posgenAel_2}}\Fop, \tup{\R{\posgenBel_2},\R{\posgenBel_1}}} \\
                 & =\bigvee_{\tup{\posBel,\posBel\elprime}\setin \posB_\stylepos{1}\cartprod \posB_\stylepos{2}} \pars{\adpa\mtimescat \adpb} \pars{ \tup{\F{\posgenAel_1},\F{\posgenAel_2}}\Fop, \tup{\RposgenBel,\R{\posgenBel'}}}\booland \braiding_\posB\pars{\tup{\FposgenBel,\F{\posgenBel'}}\Fop, \tup{\R{\posgenBel_2},\R{\posgenBel_1}} } \\
                 & =\bigvee_{\tup{\posBel,\posBel\elprime}\setin \posB_\stylepos{1}\cartprod \posB_\stylepos{2}}(\adpa(\F{\posgenAel_1^*},\RposgenBel)\booland \adpb(\F{\posgenAel_2^*},\R{\posgenBel'}))\booland (\pars{\FposgenBel\posleq \R{\posgenBel_1}} \booland \pars{\F{\posgenBel'}\posleq \R{\posgenBel_2}}) \\
                 & = \adpa(\F{\posgenAel_1^*},\R{\posgenBel_1}) \booland \adpb(\F{\posgenAel_2^*},\R{\posgenBel_2}),
            \end{aligned}
        \end{equation}
        where the last step comes from the monotonicity of~$\adpa$ and~$\adpb$.
        Similarly,
        \begin{equation}
            \begin{aligned}
                 & \pars{ \braiding_\posA \fthen (\adpb\mtimescat \adpa)}\pars{ \tup{\F{\posgenAel_1},\F{\posgenAel_2}}\Fop, \tup{\R{\posgenBel_2},\R{\posgenBel_1}}} \\
                 & =\bigvee_{\tup{\posAel,\posAel\elprime}\setin \posA_\stylepos{2}\cartprod \posA_\stylepos{1}}\braiding_\posgenA\pars{\tup{\F{\posgenAel_1},\F{\posgenAel_2}}\Fop, \tup{\R{\posgenAel},\R{\posgenAel}'} }\booland \pars{\adpb\mtimescat \adpa} \pars{ \tup{\FposgenAel,\F{\posgenAel'}}\Fop, \tup{\R{\posgenBel_2},\R{\posgenBel_1}}} \\
                 & =\bigvee_{\tup{\posAel,\posAel\elprime}\setin \posA_\stylepos{2}\cartprod \posA_\stylepos{1}}(\pars{\F{\posgenAel_1}\posleq \R{\posgenAel'}}\booland \pars{\F{\posgenAel_2}\posleq \R{\posgenAel}}) \booland (\adpb(\FposgenAelop,\R{\posgenBel_2})\booland \adpa(\F{\posgenAel'^*},\R{\posgenBel_1})) \\
                 & = \adpb(\F{\posgenAel_1^*},\R{\posgenBel_1}) \booland \adpa(\R{\posgenAel_2^*},\R{\posgenBel_2}).
            \end{aligned}
        \end{equation}
        To show the first triangle identity, we write
        \begin{equation}
            \begin{aligned}
                 & \pars{\braiding_{\singleton \cartprod \posA}\fthen \rightunitor_\posA}\pars{ \tup{\F{\singletonel},\F{\posgenAel_1}}\Fop,\R{\posgenAel_2}} \\
                 & =\bigvee_{\tup{\F{\posgenAel'},\R{\singletonel}}\setin \posA\Ptimes \singleton}\braiding_{\singleton \cartprod \posA}\pars{ \tup{\F{\singletonel},\F{\posgenAel_1}}\Fop, \tup{\R{\posgenAel'},\R{\singletonel}}}\booland \rightunitor_\posA\pars{ \tup{\F{\posgenAel'},\F{\singletonel}}\Fop,\R{\posgenAel_2}} \\
                 & =\bigvee_{\tup{\posAel\elprime,\singletonel}\setin \posA\Ptimes \singleton} \pars{\F{\singletonel}\posleq \R{\singletonel}} \booland \pars{\F{\posgenAel_1}\posleq \R{\posgenAel'}}\booland \pars{\F{\posgenAel'}\posleq \R{\posgenAel_2}} \\
                 & =\F{\posgenAel_1}\posleq \R{\posgenAel_2} \\
                 & =\leftunitor_\posA\pars{ \tup{\F{\singletonel},\F{\posgenAel_1}}\Fop,\R{\posgenAel_2}}.
            \end{aligned}
        \end{equation}
        The hexagon identities are more verbose.
        Consider~$\posA,\posB,\posC\setin \Obof\DP$.
        For brevity, we denote~$\associator_{\posA,\posB,\posC}$ by~$\associator$,~$\braiding_{\posA,\posB}\mtimescatmor \catid_\posC$ by~$\braiding'$,~$\catid_\posB \mtimescatmor \braiding_{\posA,\posC}$ by~$\braiding''$,~$(\posB\Ptimes \posA)\Ptimes \posC$ as~$\Diamond$, and~$\posB\Ctimes (\posA\Ptimes \posC)$ as~$\Delta$.

        Recall that
        \begin{equation}
            \begin{aligned}
                \braiding' \pars{\tup{\tup{\F{\posgenAel_1},\F{\posgenBel_1}},\F{\posgenCel_1}}\Fop, \tup{\R{\posgenBel_2}, \tup{\R{\posgenAel_2},\R{\posgenCel_2}}} } & =
                \pars{ (\F{\posgenAel_1}\posleq \R{\posgenAel_2}) \booland (\F{\posgenBel_1}\posleq \R{\posgenBel_2})}\booland (\F{\posgenCel_1}\posleq \R{\posgenCel_2}) \\
                                                                                                                                                                       & = (\F{\posgenAel_1}\posleq \R{\posgenAel_2}) \booland (\F{\posgenBel_1}\posleq \R{\posgenBel_2}) \booland (\F{\posgenCel_1}\posleq \R{\posgenCel_2}).
            \end{aligned}
        \end{equation}
        We have
        \begin{equation}
            \begin{aligned}
                 & \pars{\braiding' \fthen \associator } \pars{\tup{\tup{\F{\posgenAel_1},\F{\posgenBel_1}},\F{\posgenCel_1}}\Fop, \tup{\R{\posgenBel_2}, \tup{\R{\posgenAel_2},\R{\posgenCel_2}}} } \\
                 & =\bigvee_{\tup{\tup{\posBel,\posAel},\posCel}\setin \Diamond}\braiding' \pars{ \tup{\tup{\F{\posgenAel_1},\F{\posgenBel_1}},\F{\posgenCel_1}}\Fop, \tup{\tup{\RposgenBel,\R{\posgenAel}},\R{\posgenCel}}}\booland \associator \pars{ \tup{\tup{\FposgenBel,\FposgenAel},\F{\posgenCel}}\Fop, \tup{\R{\posgenBel_2}, \tup{\R{\posgenAel_2},\R{\posgenCel_2}}}} \\
                 & =\bigvee_{\tup{\tup{\posBel,\posAel},\posCel}\setin \Diamond} \pars{\pars{\F{\posgenAel_1}\posleq \R{\posgenAel} }\booland \pars{ \F{\posgenBel_1}\posleq \RposgenBel}\booland \pars{\F{\posgenCel_1}\posleq \R{\posgenCel}}}\booland \\
                 & \pars{\pars{\FposgenBel\posleq \R{\posgenBel_2}}\booland \pars{ \FposgenAel\posleq \R{\posgenAel_2}}\booland \pars{\F{\posgenCel}\posleq \R{\posgenCel_2}}} \\
                 & =\pars{\F{\posgenBel_1}\posleq \R{\posgenBel_2} }\booland \pars{\F{\posgenAel_1}\posleq \R{\posgenAel_2} }\booland \pars{ \F{\posgenCel_1}\posleq \R{\posgenCel_2}} \\
                 & =\underbrace{\associator\pars{\tup{\tup{\F{\posgenBel_1},\F{\posgenAel_1}},\F{\posgenCel_1}}^*, \tup{\R{\posgenBel_2}, \tup{\R{\posgenAel_2},\R{\posgenCel_2}}}}}_{\star}.
            \end{aligned}
        \end{equation}
        Furthermore, consider
        \begin{equation}
            \begin{aligned}
                 & \pars{ \star \fthen \braiding''}\pars{ \tup{\tup{\F{\posgenAel_1},\F{\posgenBel_1}},\F{\posgenCel_1}}\Fop, \tup{\R{\posgenBel_3}, \tup{\R{\posgenCel_3},\R{\posgenAel_3}}}} \\
                 & =\bigvee_{\tup{\posBel_\styleelements{2}, \tup{\posAel_\styleelements{2},\posCel_\styleelements{2}}}\setin \Delta} \star \pars{\tup{\tup{\F{\posgenAel_1},\F{\posgenBel_1}},\F{\posgenCel_1}}\Fop, \tup{\R{\posgenBel_2}, \tup{\R{\posgenAel_2},\R{\posgenCel_2}}} }\booland \\
                 & \braiding'' \pars{\tup{\F{\posgenBel_2}, \tup{\F{\posgenAel_2},\F{\posgenCel_2}}}^*, \tup{\R{\posgenBel_3}, \tup{\R{\posgenCel_3},\R{\posgenAel_3}}} } \\
                 & =\bigvee_{\tup{\posgenBel_\styleelements{2}, \tup{\posgenAel_\styleelements{2},\posgenCel_\styleelements{2}}}\setin \Delta}(\pars{\F{\posgenBel_1}\posleq \R{\posgenBel_2} }\booland \pars{\F{\posgenAel_1}\posleq \R{\posgenAel_2} }\booland \pars{ \F{\posgenCel_1}\posleq \R{\posgenCel_2}}) \booland \\
                 & \pars{\pars{\F{\posgenBel_2}\posleq \R{\posgenBel_3}}\booland \pars{\F{\posgenAel_2}\posleq \R{\posgenAel_3}} \booland \pars{\F{\posgenCel_2}\posleq \R{\posgenCel_3}}} \\
                 & =(\F{\posgenAel_1}\posleq \R{\posgenAel_3}) \booland (\F{\posgenBel_1}\posleq \R{\posgenBel_3}) \booland (\F{\posgenCel_1}\posleq \R{\posgenCel_3}).
            \end{aligned}
        \end{equation}
        It is easy to see that the other direction in the hexagon commutative diagram commutes as well.
        With this we have proved that~$\braiding$ is a valid braiding operation and hence that~$\tup{\DP, \mtimescat, \singleton, \braiding}$ is a \SY{symmetric monoidal category}.
    \end{proof}
}

