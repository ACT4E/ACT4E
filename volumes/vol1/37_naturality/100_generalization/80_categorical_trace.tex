% !TEX root = chapter-standalone.tex

\section{Trace for monoidal categories}
\linkvideo{spring2021-par-feedback:feedback} % Feedback
\todotextjira{199}{\alphubel: @JL: Write something about how we already saw string diagrams, and that (co)evaluation maps allow for bending the wires around to change the their direction, and that we can encode dual objects by adding a ``flow'' direction to the wires.
    Then motivate the notion of trace in a \SY{monoidal category} by asking what we get when we bend a wire around and let the outout of a morphism flow back and be it's input -- ``closing the loop''}

%\linkvideo{spring2021-par-feedback:mon-cat:string-diag} % String diagrams

\begin{ctdefinition}[Trace of an endomorphism]
    \label{def:trace_endo}
    Let~$\tup{\CatC, \mtimescat, \idmoncat, \associator, \leftunitor, \rightunitor, \braiding}$ be a \SY{symmetric monoidal category}.
    Let~$\Obja \setin \ObC$ be dualizable and let
    \begin{equation}
        \mora \colon \Obja \mto \Obja.
    \end{equation}
    The \emph{trace} of~$\mora$ is the morphism
    \begin{equation}
        \Tr(\mora) \colon \idmoncat \mto \idmoncat
    \end{equation}
    defined by
    \equationsag{endo_trace}{eq:endo_trace}
    %\begin{equation}
    %\label{eq:endomorphism-gen-trace}
    %\idmoncat \overset{\coev_\Obja}{\mto} \Obja \mtimescatat \Obja^\vee  \overset{\mora \mtimescatmor \catid_{\Obja^\vee}}{\mto} \Obja \mtimescatob \Obja^\vee  \overset{\braiding}{\mto}  \Obja^\vee \mtimescatob \Obja \overset{\ev_\Obja}{\mto} \idmoncat.
    %\end{equation}
\end{ctdefinition}

\begin{gradedexercise}[\exname{LinearAlgebraTrace}]
    \label{ex:LinearAlgebraTrace}
    Let~\CatC be the category of finite-dimensional real \SY{vector spaces} and~\reals-linear maps.
    \todotextjira{516}{\alphubel: @JL: Didn't we call this already something different?}
    We have seen that this category is symmetric monoidal when equipped with the usual tensor product as monoidal product.
    Furthermore, in \cref{sec:dual-objects} we saw that every object in this category is dualizable.

    Fix a finite-dimensional real \SY{vector space}~$\styleobj{V}$, and let~$\makeset{e_1,\ldots,e_n }$ be a basis of it.
    We saw that a choice of dual object for~$\styleobj{V}$ is given by~$\styleobj{V}^* = \Hom(\styleobj{V}, \reals)$, together with the evaluation map
    \begin{equation}
        \defmapperiod{
            \ev_\styleobj{V}
        }{
            \styleobj{V}^* \otimes \styleobj{V}
        }{
            \mto
        }{
            \reals
        }{
            \tup{l, v}
        }{
            l(v)
        }
    \end{equation}
    and the co-evaluation map
    \begin{equation}
        \defmapperiod{
            \coev_\styleobj{V}
        }{
            \reals
        }{
            \mto
        }{
            \styleobj{V} \otimes \styleobj{V}^*
        }{
            \lambda
        }{
            \lambda \sum_{i=1}^n e_i \otimes e_i^*
        }
    \end{equation}
    where $\makeset{e_1^*,\ldots,e_n^* }$ is the basis dual to the one we chose for~$\styleobj{V}$.

    Let~$\mora\colon \styleobj{V} \mto \styleobj{V}$ be a linear \SY{endomorphism} -- that is,~$\mora \setin \Hom_{\CatC}(\styleobj{V}, \styleobj{V})$.
    Compute the trace~$\Tr(\mora) \setin \Hom_{\CatC}(\reals, \reals)$ of~$\mora$ according to \cref{def:trace_endo}, and explain why it is the linear map ``multiplication by the trace of~$\mora$'', where ``trace'' in this latter phrase is the usual notion that we know from linear algebra.
\end{gradedexercise}
\solutionof{LinearAlgebraTrace}

\begin{gradedexercise}[\exname{DPSnakeTrace}]
    \label{ex:DPSnakeTrace}

    In this exercise we work with the category $\DP$ of posets and design problems, equipped with the symmetric monoidal structure where the monoidal product is the cartesian product of posets.
    The braiding
    $$\braiding_{\posgenA,\posgenB}\colon \F{\posgenA} \Ptimes \F{\posgenB} \profto \R{\posgenB} \Ctimes \R{\posgenA}$$
    is defined by
    \begin{equation}
        \braiding_{\posgenA,\posgenB}(\tup{\F{\posgenAel_1},\F{\posgenBel_1}}\Fop, \tup{\R{\posgenBel_2},\R{\posgenAel_2}})\definedas \pars{\F{\posgenAel_1}\posleqof\posA \R{\posgenAel_2}}\booland \pars{\F{\posgenBel_1}\posleqof\posB \R{\posgenBel_2}}.
    \end{equation}
    In the following you are free to use the identification
    \begin{equation}
        (\posgenA \Ptimes \posgenB)\op = \posgenA\op \Ptimes \posgenB\op
    \end{equation}
    for any posets $\posgenA$, $\posgenB$.
    Also, recall that $(\posgenA\op)\op = \posgenA$.

    Let us define the following duality data:
    \begin{itemize}
        \item $\posgenA\dual \definedas \posgenA\op$
        \item $\ev_\posgenA \colon \F{(\posgenA\op \Ptimes \posgenA)}\op \Ptimes \R{\{ \bullet \}} \mto \Bool, \quad \tup{\tup{\ela^*, \elb}^*, \bullet} \mapsto \elb \posleqof\posgenA \ela$
        \item $\coev_\posgenA \colon \F{\{ \bullet \}}\op \Ptimes \R{(\posgenA \Ptimes \posgenA\op)} \mto \Bool, \quad \tup{\bullet, \tup{\ela, \elb^*}} \mapsto \elb \posleqof\posgenA \ela$
    \end{itemize}

    Your tasks:

    \begin{enumerate}
        \item
              Guess the definitions of the associator $\associator$ and the unitors $\leftunitor$, $\rightunitor$ for the monoidal category $\DP$, check that each has the correct type, and justify why each of them does indeed define a morphism in the category $\DP$.
        \item
              Guess the definitions of $\leftunitor^{-1}$ and $\rightunitor^{-1}$ and check for one of them that it does indeed define the inverse morphism.
        \item
              Check that $\ev_\posgenA$ and $\coev_\posgenA$, as defined above, are morphisms in $\DP$.
        \item
              For an arbitrary poset $\posgenA$ and the duality data given above, prove that this snake equation
              \begin{equation}
                  \leftunitor_{\posgenA}^{-1} \mthen (\coev_\posgenA \mtimescatmor \catidat \posgenA) \mthen \associator_{\posgenA, \posgenA\op, \posgenA} \mthen(\catidat\posgenA \mtimescatmor \ev_\posgenA) \mthen \rightunitor_{\posgenA} = \catidat\posgenA
              \end{equation}
              holds.
    \end{enumerate}
\end{gradedexercise}
\solutionof{DPSnakeTrace}

\begin{widepar}
    \begin{ctdefinition}[Trace of a generalized endomorphism]
        \label{def:trace_gen_endo}
        Let~$\tup{\CatC, \mtimescat, \idmoncat, \associator, \leftunitor, \rightunitor, \braiding}$ be a \SY{symmetric monoidal category}.
        Let~$\Obja \setin \ObC$ be dualizable and let
        \begin{equation}
            \mora \colon (\Objb \mtimescatob \Obja) \mto( \Objc \mtimescatob \Obja).
        \end{equation}
        The \emph{trace over}~$\Obja$ of~$\mora$ is the morphism
        \begin{equation}
            \Tr_{\Objb, \Objc}^\Obja (\mora) \colon \Objb \mto \Objc
        \end{equation}
        defined by
        \equationsag{endo_trace_gen}{eq:endo_trace_gen}
    \end{ctdefinition}
\end{widepar}

\begin{ctdefinition}[Traced monoidal category]
    \label{def:traced-monoidal-cat}
    \label{def:traced-monoidal-category}
    We say that a \SY{symmetric monoidal category}~$\tup{\CatC, \mtimescat, \idmoncat, \associator, \leftunitor, \rightunitor, \braiding}$ is \emph{traced} if it is equipped with a family of functions
    \begin{equation}
        \Tr_{\Obja,\Objb}^\Objc\colon \HomSet{\CatC}{\Obja \mtimescatob \Objc}{\Objb\mtimescatob \Objc}\to \HomSet{\CatC}{\Obja}{\Objb},
    \end{equation}
    satisfying the following axioms:
    \begin{enumerate}
        \item \emph{Naturality in $\Obja$:} For any morphisms $\mora\colon \Obja\mtimescatob \Objc \mto \Objb\mtimescatob \Objc$ and $\morb \colon \Obja' \mto \Obja$,
              \begin{equation}\label{eq:trace-naturality-1}
                  \Tr_{\Obja', \Objb}^\Objc ( (\morb \mtimescatmor \catidat\Objc) \mthen \mora) = \morb \mthen \Tr_{\Obja, \Objb}^\Objc (\mora)
              \end{equation}
        \item \emph{Naturality in $\Objb$:} For any morphisms $\mora\colon \Obja\mtimescatob \Objc \mto \Objb\mtimescatob \Objc$ and $\morb \colon \Objb \mto \Objb'$,
              \begin{equation}\label{eq:trace-naturality-2}
                  \Tr_{\Obja, \Objb'}^\Objc ( \mora \mthen (\morb \mtimescatmor \catidat\Objc) ) = \Tr_{\Obja, \Objb}^\Objc (\mora) \mthen \morb
              \end{equation}
        \item \emph{Dinaturality in $\Objc$:}
              For any morphisms $\mora \colon \Obja \mtimescatob \Objc \mto \Objb \mtimescatob \Objc'$ and $\morb \colon \Objc' \mto \Objc$,
              \begin{equation}\label{eq:trace-dinaturality}
                  \Tr_{\Obja, \Objb}^{\Objc} (\mora \mthen (\catidat\Objb \mtimescatmor \morb)) = \Tr_{\Obja, \Objb}^{\Objc'}((\catidat\Obja \mtimescatmor \morb) \mthen \mora).
              \end{equation}

        \item \emph{Vanishing I:}
              For any morphisms~$\mora \colon \Obja\mto \Objb$ in \CatC,
              \begin{equation}
                  \label{eq:vanishing_1}
                  \Tr_{\Obja,\Objb}^\idmoncat (\mora)= \rightunitor^{-1}_{\Obja} \mthen \mora \mthen \rightunitor_\Objb.
              \end{equation}
        \item \emph{Vanishing II:}
              For any morphism~$\mora \colon (\Obja\mtimescatob \Objc) \mtimescatob \Objd \mto (\Objb\mtimescatob \Objc) \mtimescatob \Objd$ in \CatC,
              \begin{equation}
                  \label{eq:vanishing_2}
                  \Tr_{\Obja,\Objb}^\Objc\pars{
                      \Tr_{\Obja \mtimescatob \Objc, \Objb \mtimescatob \Objc}^\Objd(\mora)} = \Tr_{\Obja,\Objb}^{\Objc\mtimescatob \Objd}(\associator_{\Obja, \Objc, \Objd} \mthen \mora \mthen \associator^{-1}_{\Objb, \Objc, \Objd}).
              \end{equation}
        \item \emph{Superposing:}
              For any morphism~$\mora\colon \Obja\mtimescatob \Objc \mto \Objb\mtimescatob \Objc$ in \CatC,
              \begin{equation}
                  \label{eq:superposing}
                  \Tr_{\Obje\mtimescatob \Obja,\Obje\mtimescatob \Objb}^{\Objc}(\associator_{\Obje, \Obja, \Objc} \mthen \catidat\Obje\mtimescatmor \mora \mthen \associator^{-1}_{\Obje, \Objb, \Objc} )=\catidat\Obje\mtimescatmor \Tr_{\Obja,\Objb}^\Objc(\mora).
              \end{equation}
        \item \emph{Yanking:}
              \begin{equation}
                  \label{eq:yanking}
                  \Tr_{\Objc,\Objc}^\Objc\pars{\braiding_{\Objc,\Objc}}=\catidat\Objc.
              \end{equation}
    \end{enumerate}
\end{ctdefinition}

\begin{remark}
    Other variants of the definition of a \SY{traced monoidal category} can be found in the literature.
    For instance, some include a more general version of the superposing law, see \cref{lem:general-superposing-law} below.
\end{remark}

\begin{lemma}
    \label{lem:general-superposing-law}
    Let~$\tup{\CatC, \mtimescat,\idmoncat_{\CatC}, \braiding, \Tr}$ be a \SY{traced monoidal category}.
    Then a more general version of the superposing law holds: for any morphisms~$\mora\colon \Obja\mtimescatob \Objc \mto \Objb \mtimescatob \Objc$ and~$\morb \colon \Objd \mto \Obje$,
    \begin{equation}
        \label{eq:superposing_2}
        \Tr_{\Objd\mtimescatob \Obja,\Obje\mtimescatob \Objb}^{\Objc}(\morb \mtimescatmor \mora)=\morb \mtimescatmor \Tr_{\Obja,\Objb}^\Objc(\mora).
    \end{equation}
\end{lemma}
\missingproof

\todotextjira{165}{\alphubel: @JL: Do the proof.
    Also, this could be made into a graded exercise.
}

\begin{gradedexercise}[\exname{TracingRelations}]
    \label{eq:TracingRelations}

    Given the associative stacking category $\tup{\RelL, \mtimescatob, \mtimescatmor}$, as defined in \cref{def:RelL}, consider the trace operation
    $$\Tr_{\Obja,\Objb}^\Objc \colon \HomSet{\RelL}{\Obja \cprod \Objc}{\Objb \cprod \Objc}\to \HomSet{\RelL}{\Obja}{\Objb}$$
    which is defined, for a morphism $\relA \setin\HomSet{\CatC}{\Obja \cprod \Objc}{\Objb\cprod \Objc}$, by
    \begin{equation}
        \Tr_{\Obja,\Objb}^\Objc(\relA) = \{ \tup{\ela, \elb} \setin\Obja \cartprod \Objb \mid \exists \elc \setin\Objc \colon \tup{\ela \tupconcat \elc, \elb \tupconcat \elc} \setin\relA \}.
    \end{equation}

    Your task is to check that this definition satisfies the following two trace axioms:

    \begin{enumerate}

        \item \emph{Vanishing II}:

              For any relation~$\relA \colon \Obja \cprod \Objc \cprod \Objd \mto \Objb\cprod \Objc \cprod \Objd$ in $\RelL$,
              \begin{equation}
                  \Tr_{\Obja,\Objb}^{\Objc\cprod \Objd}(\relA) = \Tr_{\Obja,\Objb}^\Objc\pars{
                      \Tr_{\Obja \cprod \Objc, \Objb \cprod \Objc}^\Objd(\relA)}.
              \end{equation}

        \item \emph{Superposing}:

              For any relations~$\relA \colon \Obja\cprod \Objc \mto \Objb\cprod \Objc$ and $\relB \colon \Obje \mto \Objf$ in $\RelL$,
              \begin{equation}
                  \Tr_{\Obje\cprod \Obja,\Objf\cprod \Objb}^{\Objc}(\relB \mtimescatmor \relA) = \relB \mtimescatmor \Tr_{\Obja,\Objb}^\Objc(\relA).
              \end{equation}
    \end{enumerate}
\end{gradedexercise}

\solutionof{TracingRelations}

\begin{gradedexercise}[\exname{RelDualsTrace}]
    \label{ex:RelDualsTrace}

    In this exercise we work with the category $\Rel$ of sets and relations, equipped with the symmetric monoidal structure where the monoidal product is the cartesian product of sets.
    The braiding is
    \begin{equation}
        \braiding_{\setA, \setB} \colon \makeset{ \tup{\tup{\ela, \elb},\tup{\elb', \ela'}} \setin (\setA \cartprod \setB) \cartprod (\setB \cartprod \setA) \mid \ela = \ela', \elb = \elb' }.
    \end{equation}

    This symmetric monoidal category is compact closed when we let the dual $\setA \dual$ of any set $\setA$ be the set itself, $\setA \dual = \setA$, and we define evaluation and co-evaluation by
    \begin{equation}
        \ev_\setA \colon \setA \cartprod \setA \mto \singleton, \quad \ev_\setA = \makeset{\tup{\tup{\ela, \elb}, \bullet} \setin(\setA \cartprod \setA) \cartprod \singleton \mid \ela = \elb}
    \end{equation}
    and
    \begin{equation}
        \coev_\setA \colon \singleton \mto \setA \cartprod \setA, \quad \coev_\setA = \makeset{\tup{\bullet, \tup{\ela, \elb}} \setin\singleton \cartprod (\setA \cartprod \setA) \mid \ela = \elb}
    \end{equation}
    respectively.

    Your tasks:
    \begin{enumerate}
        \item
              Let $\relB \colon \setA \mto \setB$ be a (generic) morphism in $\Rel$.
              Compute the dual morphism $\relB\dual \colon \setB\dual \mto \setA\dual$.

        \item
              Let $\relA \colon \setA \cartprod \setC \mto \setB \cartprod \setC$ be a morphism in $\Rel$.
              Show that the trace of $\relA$, given by the composition
              \begin{equation}
                  \rightunitor^{-1}_{\setA} \mthen (\catid_\setA \mtimescatmor \coev_\setC) \mthen \associator^{-1}_{\setA, \setC, \setC} \mthen (\relA \mtimescatmor \catid_\setC) \mthen \associator_{\setB, \setC, \setC} \mthen (\catid_\setB \mtimescatmor \braiding_{\setA, \setA}) \mthen (\catid_\setB \mtimescatmor \ev_\setC) \mthen \rightunitor_\setB
              \end{equation}
              is equal to the relation
              \begin{equation}
                  \makeset{\tup{\ela, \elb} \setin\setA \cartprod \setB \mid \exists \elc \setin \setC \colon \tup{\tup{\ela, \elc},\tup{\elb,\elc}} \setin \relA}.
              \end{equation}
    \end{enumerate}
\end{gradedexercise}
\solutionof{RelDualsTrace}

\begin{gradedexercise}[\exname{DPSnakeTracePart2}]
    \label{ex:DPSnakeTracePart2}

    In this exercise we work again with the category $\DP$ of posets and design problems, equipped with the symmetric monoidal structure where the monoidal product is the cartesian product of posets.
    In the following we make the identification
    \begin{equation}
        (\posgenA \Ptimes \posgenB)\op = \posgenA\op \Ptimes \posgenB\op
    \end{equation}
    for any posets $\posgenA$, $\posgenB$.
    Also, recall that $(\posgenA\op)\op = \posgenA$.

    In components, the associator for $\DP$ is
    \begin{equation}
        \associator_{\posgenA, \posgenB, \posgenC} \colon ((\posgenA \Ptimes \posgenB) \Ptimes \posgenC)\op \Ptimes (\posgenA \Ptimes (\posgenB \Ptimes \posgenC)) \mto \Bool
    \end{equation}
    with
    \begin{equation}
        \associator_{\posgenA, \posgenB, \posgenC}(\tup{\tup{\F{\posgenAel_1\Fop},\F{\posgenBel_1\Fop}}, \F{\posgenCel_1\Fop}},\tup{\R{\posgenAel_2}, \tup{\R{\posgenBel_2}, \R{\posgenCel_2}}}) = \F{\posgenAel_1} \posleq \R{\posgenAel_2} \booland \F{\posgenBel_1} \posleq \R{\posgenBel_2} \booland \F{\posgenCel_1} \posleq \R{\posgenCel_2},
    \end{equation}
    and the left unitor is
    \begin{equation}
        \leftunitor_{\posgenA} \colon (\stylepos{\singleton} \Ptimes \posgenA)\op \Ptimes \posgenA \mto \Bool
    \end{equation}
    with
    \begin{equation}
        \leftunitor_{\posgenA} (\tup{\tup{\F{\bullet\Fop}, \F{\posgenAel_1\Fop}}, \R{\posgenAel_2}}) = \F{\posgenAel_1} \posleq \R{\posgenAel_2}.
    \end{equation}
    The right unitor is analogous.

    The braiding
    $$\braiding_{\posgenA,\posgenB}\colon \F{\posgenA} \Ptimes \F{\posgenB} \profto \R{\posgenB} \Ctimes \R{\posgenA}$$
    is
    \begin{equation}
        \braiding_{\posgenA,\posgenB}(\tup{\F{\posgenAel_1\Fop},\F{\posgenBel_1\Fop}}, \tup{\R{\posgenBel_2},\R{\posgenAel_2}})\definedas \pars{\F{\posgenAel_1}\posleqof\posA \R{\posgenAel_2}}\booland \pars{\F{\posgenBel_1}\posleqof\posB \R{\posgenBel_2}}.
    \end{equation}

    We define the following duality data, with respect to which $\DP$ is compact closed:
    \begin{itemize}
        \item $\posgenA\dual \definedas \posgenA\op$
        \item $\ev_\posgenA \colon \F{(\posgenA\op \Ptimes \posgenA)}\op \Ptimes \R{\{ \bullet \}} \mto \Bool, \quad \tup{\tup{\ela^*, \elb}^*, \bullet} \mapsto \elb \posleqof\posgenA \ela$
        \item $\coev_\posgenA \colon \F{\{ \bullet \}}\op \Ptimes \R{(\posgenA \Ptimes \posgenA\op)} \mto \Bool, \quad \tup{\bullet, \tup{\ela, \elb^*}} \mapsto \elb \posleqof\posgenA \ela$
    \end{itemize}

    Your task: given a morphism $\adp \colon \F{\posgenA} \Ptimes \F{\posgenC} \profto \R{\posgenB} \Ctimes \R{\posgenC}$ in $\DP$, show that the design problem $\F{\posgenA} \profto \R{\posgenB}$ given by the following composition
    \begin{equation}
        \rightunitor^{-1}_{\posgenA} \mthen (\catid_\posgenA \mtimescatmor \coev_\posgenC) \mthen \associator^{-1}_{\posgenA, \posgenC, \posgenC\op} \mthen (\adp \mtimescatmor \catid_{\posgenC\op}) \mthen \associator_{\posgenB, \posgenC, \posgenC\op} \mthen (\catid_\posgenB \mtimescatmor \braiding_{\posgenC, \posgenC\op}) \mthen (\catid_\posgenB \mtimescatmor \ev_\posgenC) \mthen \rightunitor_\posgenB
    \end{equation}
    is equal to the design problem $\F{\posgenA} \profto \R{\posgenB}$ given by
    \begin{equation}
        \begin{aligned}
            \funposA\op \Ptimes \resposB    & \toinPos \Bool, \\
            \tup{\funposAopel, \RposgenBel} & \mapsto \bigvee_{\posgenCel \setin \posgenC}
            \adpa(\tup{\FposgenAel, \F{\posgenCel}}\Fop,
            \tup{\RposgenBel, \R{\posgenCel}}).
        \end{aligned}
    \end{equation}
\end{gradedexercise}
\solutionof{DPSnakeTracePart2}