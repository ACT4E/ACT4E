% !TEX root = chapter-standalone.tex

\section{Feedback in category theory}
\linkvideo{spring2021-par-feedback:feedback} % Feedback
\todotextjira{199}{\alphubel: @JL: Write something about how we already saw string diagrams, and that (co)evaluation maps allow for bending the wires around to change the their direction, and that we can encode dual objects by adding a ``flow'' direction to the wires.
    Then motivate the notion of trace in a \SY{monoidal category} by asking what we get when we bend a wire around and let the outout of a morphism flow back and be it's input -- ``closing the loop''}

%\linkvideo{spring2021-par-feedback:mon-cat:string-diag} % String diagrams

\begin{ctdefinition}[Trace of an endomorphism]
    \label{def:trace_endo}
    Let~$\tup{\CatC, \mtimescat, \idmoncat, \associator, \leftunitor, \rightunitor, \braiding}$ be a \SY{symmetric monoidal category}.
    Let~$\Obja \setin \ObC$ be dualizable and let
    \begin{equation}
        \mora \colon \Obja \mto \Obja.
    \end{equation}
    The \emph{trace} of~$\mora$ is the morphism
    \begin{equation}
        \Tr(\mora) \colon \idmoncat \mto \idmoncat
    \end{equation}
    defined by
    \equationsag{endo_trace}{eq:endo_trace}
    %\begin{equation}
    %\label{eq:endomorphism-gen-trace}
    %\idmoncat \overset{\coev_\Obja}{\mto} \Obja \mtimescatat \Obja^\vee  \overset{\mora \mtimescatmor \catid_{\Obja^\vee}}{\mto} \Obja \mtimescatob \Obja^\vee  \overset{\braiding}{\mto}  \Obja^\vee \mtimescatob \Obja \overset{\ev_\Obja}{\mto} \idmoncat.
    %\end{equation}
\end{ctdefinition}

\begin{gradedexercise}[\exname{LinearAlgebraTrace}]
    \label{ex:LinearAlgebraTrace}
    Let~\CatC be the category of finite-dimensional real \SY{vector spaces} and~\reals-linear maps.
    \todotextjira{516}{\alphubel: @JL: Didn't we call this already something different?}
    We have seen that this category is symmetric monoidal when equipped with the usual tensor product as monoidal product.
    Furthermore, in \cref{sec:dual-objects} we saw that every object in this category is dualizable.

    Fix a finite-dimensional real \SY{vector space}~$\styleobj{V}$, and let~$\makeset{e_1,\ldots,e_n }$ be a basis of it.
    We saw that a choice of dual object for~$\styleobj{V}$ is given by~$\styleobj{V}^* = \Hom(\styleobj{V}, \reals)$, together with the evaluation map
    \begin{equation}
        \defmapperiod{
            \ev_\styleobj{V} 
        }{
            \styleobj{V}^* \otimes \styleobj{V}
        }{
            \mto
        }{
            \reals 
        }{
            \tup{l, v}
        }{
            l(v)
        }
    \end{equation}
    and the co-evaluation map
    \begin{equation}
        \defmapperiod{
            \coev_\styleobj{V} 
        }{
            \reals
        }{
            \mto
        }{
            \styleobj{V}  \otimes \styleobj{V}^*
        }{
            \lambda 
        }{
            \lambda \sum_{i=1}^n e_i \otimes e_i^*
        }
    \end{equation}
    where $\makeset{e_1^*,\ldots,e_n^* }$ is the basis dual to the one we chose for~$\styleobj{V}$.

    Let~$\mora\colon \styleobj{V} \mto \styleobj{V}$ be a linear \SY{endomorphism} -- that is,~$\mora \setin \Hom_\CatC(\styleobj{V}, \styleobj{V})$.
    Compute the trace~$\Tr(\mora) \setin \Hom_\CatC(\reals, \reals)$ of~$\mora$ according to \cref{def:trace_endo}, and explain why it is the linear map ``multiplication by the trace of~$\mora$'', where ``trace'' in this latter phrase is the usual notion that we know from linear algebra.
\end{gradedexercise}
\solutionof{LinearAlgebraTrace}

\begin{widepar}
    \begin{ctdefinition}[Trace of a generalized endomorphism]
        \label{def:trace_gen_endo}
        Let~$\tup{\CatC, \mtimescat, \idmoncat, \associator, \leftunitor, \rightunitor, \braiding}$ be a \SY{symmetric monoidal category}.
        Let~$\Obja \setin \ObC$ be dualizable and let
        \begin{equation}
            \mora \colon (\Objb \mtimescatob \Obja) \mto( \Objc \mtimescatob \Obja).
        \end{equation}
        The \emph{trace over}~$\Obja$ of~$\mora$ is the morphism
        \begin{equation}
            \Tr_{\Objb, \Objc}^\Obja (\mora) \colon \Objb \mto \Objc
        \end{equation}
        defined by
        \equationsag{endo_trace_gen}{eq:endo_trace_gen}
    \end{ctdefinition}
\end{widepar}

\todojira{361}{\alphubel: @JL: The above needs to be fixed to account for coherence morphisms in the non-strict case, and also for the definition of trace below}

%\devel{
%\begin{forslides}
%
%\begin{equation}\label{eq:slides-trace-1}
%\idmoncat \overset{\coev_\Obja}{\mto} \Obja \mtimescatob \Obja^\vee  \overset{\mora \mtimescatmor \catid_{\Obja^\vee}}{\mto} \Obja \mtimescatob \Obja^\vee  \overset{\braiding}{\mto}  \Obja^\vee \mtimescatob \Obja \overset{\ev_\Obja}{\mto} \idmoncat.
%\end{equation}
%
%\begin{equation}\label{eq:slides-trace-2}
%\Objb \overset{\catid_\Objb \mtimescat \coev_\Obja}{\mto} \Objb \mtimescatob \Obja \mtimescatob \Obja^\vee   \overset{\mora \mtimescatmor \catid_{\Obja^\vee}}{\mto} \Objc \mtimescatob \Obja \mtimescatob \Obja^\vee  \overset{\catid_\Objc \mtimescatmor \braiding}{\mto}  \Objc \mtimescatob \Obja^\vee \mtimescatob \Obja \overset{\catid_\Objc \mtimescat \ev_\Obja}{\mto} \Objc.
%\end{equation}
%
%\begin{equation}\label{eq:slides-trace-3}
%\Tr(\mora) \setin \Hom(\idmoncat, \idmoncat)
%\end{equation}
%
%\begin{equation}\label{eq:slides-trace-4}
%\Tr_{\Objb, \Objc}^\Obja (\mora) \setin \Hom(\Objb, \Objc)
%\end{equation}
%
%\begin{equation}\label{eq:slides-trace-5}
%\tup{\CatC,\mtimesC,\idmoncat_{\CatC}, \braiding}
%\end{equation}
%
%\end{forslides}
%}

\begin{ctdefinition}[Traced monoidal category]
    \label{def:traced-monoidal-cat}
    \label{def:traced-monoidal-category}
    We say that a \SY{symmetric monoidal category}~$\tup{\CatC, \mtimescat, \idmoncat, \associator, \leftunitor, \rightunitor, \braiding}$ is \emph{traced} if it is equipped with a family of functions
    \begin{equation}
        \Tr_{\Obja,\Objb}^\Objc\colon \HomSet{\CatC}{\Obja \mtimescatob \Objc}{\Objb\mtimescatob \Objc}\to \HomSet{\CatC}{\Obja}{\Objb},
    \end{equation}
    satisfying the following axioms:
    \begin{enumerate}
        \item \emph{Naturality in $\Obja$:} For any morphisms $\mora\colon \Obja\mtimescatob \Objc \mto \Objb\mtimescatob \Objc$ and $\morb \colon \Obja' \mto \Obja$,
              \begin{equation}
                  \Tr_{\Obja', \Objb}^\Objc ( (\morb \mtimescatmor \catidat\Objc) \mthen \mora) = \morb \mthen \Tr_{\Obja, \Objb}^\Objc (\mora)
              \end{equation}
        \item \emph{Naturality in $\Objb$:} For any morphisms $\mora\colon \Obja\mtimescatob \Objc \mto \Objb\mtimescatob \Objc$ and $\morb \colon \Objb \mto \Objb'$,
              \begin{equation}
                  \Tr_{\Obja, \Objb'}^\Objc ( \mora \mthen (\morb \mtimescatmor \catidat\Objc) ) =  \Tr_{\Obja, \Objb}^\Objc (\mora) \mthen \morb
              \end{equation}
        \item \emph{Dinaturality in $\Objc$:} For any morphisms $\mora \colon \Obja \mtimescatob \Objc \mto \Objb \mtimescatob \Objc'$ and $\morb \colon \Objc' \mto \Objc$,
              \begin{equation}
                  \Tr_{\Obja, \Objb}^{\Objc} (\mora \mthen (\catidat\Objb \mtimescatmor \morb)) = \Tr_{\Obja, \Objb}^{\Objc'}((\catidat\Obja \mtimescatmor \morb) \mthen \mora).
              \end{equation}

        \item \emph{Vanishing I:}
              For any morphisms~$\mora \colon \Obja\mto \Objb$ in \CatC,
              \begin{equation}
                  \label{eq:vanishing_1}
                  \Tr_{\Obja,\Objb}^\idmoncat (\mora)= \leftunitor^{-1}_{\Obja} \mthen \mora \mthen \leftunitor_\Objb.
              \end{equation}
        \item \emph{Vanishing II:}
              For any morphism~$\mora \colon \Obja\mtimescatob \Objc \mtimescatob \Objd \mto \Objb\mtimescatob \Objc \mtimescatob \Objd$ in \CatC,
              \begin{equation}
                  \label{eq:vanishing_2}
                  \Tr_{\Obja,\Objb}^{\Objc\mtimescatob \Objd}(\mora)=\Tr_{\Obja,\Objb}^\Objc\pars{
                      \Tr_{\Obja \mtimescatob \Objc, \Objb \mtimescatob \Objc}^\Objd(\mora)}.
              \end{equation}
        \item \emph{Superposing:}
              For any morphism~$\mora\colon \Obja\mtimescatob \Objc \mto \Objb\mtimescatob \Objc$ in \CatC,
              \begin{equation}
                  \label{eq:superposing}
                  \Tr_{\Obje\mtimescatob \Obja,\Obje\mtimescatob \Objb}^{\Objc}(\catidat\Obje\mtimescatmor \mora)=\catidat\Obje\mtimescatmor \Tr_{\Obja,\Objb}^\Objc(\mora).
              \end{equation}
        \item \emph{Yanking:}
              \begin{equation}
                  \label{eq:yanking}
                  \Tr_{\Objc,\Objc}^\Objc\pars{\braiding_{\Objc,\Objc}}=\catidat\Objc.
              \end{equation}
    \end{enumerate}
\end{ctdefinition}

\begin{remark}
    Other variants of the definition of a \SY{traced monoidal category} can be found in the literature.
    For instance, some include a more general version of the superposing law, see \cref{lem:general-superposing-law} below.
\end{remark}

\begin{lemma}
    \label{lem:general-superposing-law}
    Let~$\tup{\CatC, \mtimescat,\idmoncat_{\CatC}, \braiding, \Tr}$ be a \SY{traced monoidal category}.
    Then a more general version of the superposing law holds: for any morphisms~$\mora\colon \Obja\mtimescatob \Objc \mto \Objb \mtimescatob \Objc$ and~$\morb \colon \Objd \mto \Obje$,
    \begin{equation}
        \label{eq:superposing_2}
        \Tr_{\Objd\mtimescatob \Obja,\Obje\mtimescatob \Objb}^{\Objc}(\morb \mtimescatmor \mora)=\morb \mtimescatmor \Tr_{\Obja,\Objb}^\Objc(\mora).
    \end{equation}
\end{lemma}
\missingproof

\todotextjira{165}{\alphubel: @JL: Do the proof.
    Also, this could be made into a graded exercise.
}

