% !TEX root = chapter-standalone.tex

\section{Dualizable objects}
\label{sec:dual-objects}
\todotextjira{420}{\bernina: @JL: Seems like that here string diagrams would help.}
There is a concept of ``duality'' for objects in \SY{monoidal category} which we will introduce with an illustrative example.

\linkvideo{spring2021-par-feedback:dualizability} % Dualizability
We have seen in \cref{ex:Vect-symmetric-monoidal} that the category $\CatC = \VectR$ of real \SY{vector spaces} is symmetric monoidal, with tensor product as the monoidal product.
Given a \SY{vector space} $\styleobj{V} $, its \emph{linear dual} is the real vector space
\begin{equation}
    \label{eq:lin-dual-vecsp}
    \styleobj{V} ^{*} := \makeset{ \text{linear maps } \styleobj{V} \mto \reals } = \HomSet{\CatC}{\styleobj{V}}{ \reals}.
\end{equation}
%
Recall from linear algebra the following fact about any \SY{vector space} $\styleobj{V} $:
%
\begin{equation}
    \label{eq:double-dual-fin-dim}
    \styleobj{V}  \simeq (\styleobj{V} ^{*})^* \text{ if and only if } \dim \styleobj{V}  < \infty.
\end{equation}
%
One might say that the finite-dimensional real \SY{vector spaces} are characterizable based on their behavior in this way with respect to the operation of taking the linear dual.

We will develop an alternative formulation of this fact, based on the notion of a \emph{dualizable object}.
This notion will make sense in the setting of any (symmetric) \SY{monoidal category}, and we will see then, that \cref{eq:double-dual-fin-dim} translates to the statement
\begin{equation}
    \label{eq:dualizable-fin-dim}
    \styleobj{V}  \setin \Obof{\VectR} \text{ is dualizable if and only if } \dim \styleobj{V}  < \infty.
\end{equation}
%
Key protagonists in this reformulation are \emph{evaluation} and \emph{coevaluation} maps.

In the following,  $\styleobj{V} $ denotes a \emph{finite-dimensional} real \SY{vector space}.
The evaluation map $\ev_\styleobj{V} $ associated to $\styleobj{V} $ is
\begin{equation}
    \label{eq:evaluation-vecsp}
    \defmapperiod{
        \ev_\styleobj{V}
    }{
        \styleobj{V}^* \otimes \styleobj{V}
    }{
        \mto
    }{
        \reals
    }{
        \tup{l, v}
    }{
        l(v)
    }
\end{equation}
In other words, given $\tup{l, v}$, the map $\ev_\styleobj{V} $ evaluates $l$ at $v$.

The coevaluation map $\coev_\styleobj{V} $ associated to $\styleobj{V} $ is slightly trickier to describe.
Let $\makeset{e_1, \dots, e_n }$ be a basis of $\styleobj{V}$, and let $\makeset{e_1^*, \dots, e_n^* }$ be the corresponding dual basis of~$\styleobj{V}^*$.
Then
\begin{equation}
    \label{eq:coevaluation-vecsp}
    \defmapperiod{
        \coev_\styleobj{V}
    }{
        \reals
    }{
        \mto
    }{
        \styleobj{V}  \otimes \styleobj{V}^*
    }{
        \lambda
    }{
        \lambda \sum_{i=1}^n e_i \otimes e_i^*
    }
\end{equation}
It turns out that this map is independent of the choice of basis.
One way to think of this coevaluation map is to recall that $\styleobj{V}  \otimes \styleobj{V}^* \simeq \Hom(\styleobj{V}, \styleobj{V})$.
Under this identification, $\coev_\styleobj{V}$ maps the scalar $\lambda$ to the linear \SY{endomorphism} of $\styleobj{V}$ which is ``multiplication by $\lambda$''.
(In terms of matrices, this is a diagonal matrix, with $\lambda$ at every entry of the diagonal.)

\todotext{Don't use \str{styleobj}, use macros.
    Otherwise it's very difficult to edit -- ``write only'' phenomenon}

Recall that as part of the monoidal structure on $\VectR$ we have the left and right unitors
\begin{equation}
    \leftunitor_\styleobj{V} \colon \idmoncat\mtimescatob \styleobj{V} \mtoiso \styleobj{V} \quad \quad \styleobj{V} \setin \Obof{\VectR}
\end{equation}
\begin{equation}
    \rightunitor_\styleobj{V} \colon \styleobj{V} \mtimescatob \idmoncat \mtoiso \styleobj{V} \quad \quad \styleobj{V} \setin \Obof{\VectR}.
\end{equation}
The evaluation and coevaluation maps defined above satisfy the following equations:
\begin{equation}
    \label{eq:dualizability-cond-1}
    \leftunitor_{\styleobj{V}}^{-1} \mthen (\coev_\styleobj{V} \mtimescatmor \catid_\styleobj{V}) \mthen \associator_{\styleobj{V}, \styleobj{V}^*, \styleobj{V}} \mthen (\catid_\styleobj{V} \mtimescatmor \ev_\styleobj{V}) \mthen \rightunitor_{\styleobj{V}} = \catidat{\styleobj{V}}
\end{equation}
and
\begin{equation}
    \label{eq:dualizability-cond-2}
    \rightunitor_{\styleobj{V}^*}^{-1} \mthen (\catidat{\styleobj{V}^*} \mtimescatmor \coev_\styleobj{V})  \mthen \associator_{\styleobj{V}^*, \styleobj{V}, \styleobj{V}^*}^{-1} \mthen (\ev_\styleobj{V} \mtimescatmor \catidat{\styleobj{V}^*}) \mthen \leftunitor_{\styleobj{V}^*} = \catidat{\styleobj{V}^*}.
\end{equation}
%

\begin{gradedexercise}[\exname{VectSnakeEquations}]
    \label{ex:VectSnakeEquations}
    Check \cref{eq:dualizability-cond-1,eq:dualizability-cond-2} by direct calculation, assuming that $\styleobj{V}$ is a finite-dimensional real \SY{vector space}.
\end{gradedexercise}

\solutionof{VectSnakeEquations}

The equations \cref{eq:dualizability-cond-1} and \cref{eq:dualizability-cond-2} form the basis for the general notion of dualizability in a \SY{monoidal category}.
\begin{ctdefinition}[Dualizable object]
    \label{def:dualizable-object}
    Let~$\tup{\CatC,\mtimesC,\idmoncat_{\CatC}}$ be a \SY{monoidal category}, and let~$\Obja \setin \ObC$.
    A \maindef{right dual object} of~$\Obja$ is specified by:

    \constit
    \begin{enumerate}
        \item an object $\Objadual \setin \ObC$;
        \item an evaluation map $\ev_\Obja\colon \Objadual \mtimescatob \Obja \mto \idmoncat$; \label{def:ev}
        \item a coevaluation map $\coev_\Obja\colon \idmoncat \mto \Obja \mtimescatob \Objadual$;\label{def:coev}
    \end{enumerate}

    \condit
    \begin{enumerate}
        \item $\leftunitor_{\Obja}^{-1} \mthen (\coev_\Obja \mtimescatmor \catidat\Obja) \mthen  \associator_{\Obja, \Objadual, \Obja} \mthen(\catidat\Obja \mtimescatmor \ev_\Obja) \mthen \rightunitor_{\Obja} = \catidat\Obja$;
        \item $\rightunitor_{\Objadual}^{-1} \mthen (\catidat{\Objadual} \mtimescatmor \coev_\Obja)  \mthen \associator_{\Objadual, \Obja, \Objadual}^{-1} \mthen (\ev_\Obja \mtimescatmor \catidat{\Objadual}) \mthen \leftunitor_{\Objadual}  = \catidat{\Objadual}.
              $
    \end{enumerate}
\end{ctdefinition}

\begin{definition}\label{def:right-dualizable}
    An object $\Obja$ in a \SY{monoidal category} is called \maindef{right dualizable} if there exists, in the category, a right dual object of $\Obja$.
\end{definition}

\begin{remark}\label{rem:left-dualizability}
    There is an analogous definition of \maindef{left dual object} and \maindef{left dualizability}.
    One can show that when the \SY{monoidal category} in question is symmetric, then left dual objects can be seen as right dual objects, and vice versa.
    In this case, we then speak simply of \maindef{dualizability}.
\end{remark}

%\devel{
%\begin{forslides}
%
%\begin{equation}\label{eq:slides-parallelism-1}
%\tup{\reals, \leq, {{+}},  0}
%\end{equation}
%
%\begin{equation}\label{eq:slides-parallelism-2}
%\tup{\reals, \leq,, {{\cdot}},  1}
%\end{equation}
%
%\begin{equation}\label{eq:slides-parallelism-3}
%\tup{\reals_{>0}, \leq, {{\cdot}},  1}
%\end{equation}
%
%\begin{equation}\label{eq:slides-parallelism-4}
%\tup{\Bool, \wedge, \true}
%\end{equation}
%
%\begin{equation}\label{eq:slides-parallelism-5}
%\styleobj{U}
%\end{equation}
%
%\begin{equation}\label{eq:slides-parallelism-6}
%\styleobj{V}
%\end{equation}
%
%\begin{equation}\label{eq:slides-parallelism-7}
%\styleobj{W}
%\end{equation}
%
%\begin{equation}\label{eq:slides-parallelism-8}
%\styleobj{X}
%\end{equation}
%
%\begin{equation}\label{eq:slides-parallelism-9}
%\styleobj{Y}
%\end{equation}
%
%\begin{equation}\label{eq:slides-parallelism-10}
%\styleobj{Z}
%\end{equation}
%
%\begin{equation}\label{eq:slides-parallelism-11}
%\styleobj{U^*}
%\end{equation}
%
%\begin{equation}\label{eq:slides-parallelism-12}
%\styleobj{V^*}
%\end{equation}
%
%\begin{equation}\label{eq:slides-parallelism-13}
%\styleobj{W^*}
%\end{equation}
%
%\begin{equation}\label{eq:slides-parallelism-14}
%\styleobj{X^\vee}
%\end{equation}
%
%\begin{equation}\label{eq:slides-parallelism-15}
%\styleobj{Y^\vee}
%\end{equation}
%
%\begin{equation}\label{eq:slides-parallelism-16}
%\styleobj{Z^\vee}
%\end{equation}
%
%\begin{equation}\label{eq:slides-parallelism-17}
%\ev_\Obja: \Obja^\vee \mtimescatob \Obja \mto \idmoncat
%\end{equation}
%
%\begin{equation}\label{eq:slides-parallelism-18}
%\coev_\Obja: \idmoncat \mto \Obja \mtimescatob \Obja^\vee
%\end{equation}
%
%\begin{equation}\label{eq:slides-parallelism-19}
%(\coev_\Obja \mtimescatmor \catid_\Obja) \then (\catid_\Obja \mtimescat \ev_\Obja) = \catid_\Obja
%\end{equation}
%
%\begin{equation}\label{eq:slides-parallelism-20}
%(\catid_{\Obja^*} \mtimescat \coev_\Obja)  \then (\ev_\Obja \mtimescatmor \catid_{\Obja^*})  = \catid_{\Obja^*}\end{equation}
%
%\begin{equation}\label{eq:slides-parallelism-21}
%\tup{\CatC,\mtimesC,\idmoncat_{\CatC}}
%\end{equation}
%
%\begin{equation}\label{eq:slides-parallelism-22}
%\CatC = \VectR
%\end{equation}
%
%\begin{equation}\label{eq:slides-parallelism-23}
%\VectR
%\end{equation}
%
%\end{forslides}
%}
