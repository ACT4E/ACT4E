
\section{Other examples of subcategories in engineering}

In engineering it is very common to look at specific types of functions; in many cases, the properties of a certain type of function are preserved by function composition, and so they form a category.

\subsubsection{\Injset forms a subcategory of \Set}
\begin{definition}[\iindex{Injective function}]
    \label{def:injective-function}
    Let~$\mora \colon \Obja\to \Objb$ be a function. The function~$\mora$ is \emph{injective} if, for all~$\obja,\obja'\in \Obja$ holds:~$\mora(\obja)=\mora(\obja')\implies \obja=\obja'$.
\end{definition}


\begin{example}
    We can define a category \iindex{\Injset} which has the same objects as \Set but restricts the morphisms to be \emph{injective functions}.
    We want to show that \Injset is a subcategory of \Set. Composition and identity morphisms are defined as in \Set.

    Since~$\Ob_{\Injset}=\Ob_{\Set}$, the first condition of \cref{def:subcategory} is satisfied. Injective functions are a particular type of functions: this satisfies the second condition. Given~$\Obja\in \Ob_{\Injset}$, the identity morphism~$\id_\Obja\in \Hom_{\Set}(\Obja,\Obja)$ corresponds to the identity morphism in~$\Hom_{\Injset}(\Obja,\Obja)$, i.e., the identity function is injective. This proves the third condition. To check the fourth condition, consider two morphisms~$\mora \in \Hom_\Set(\Obja,\Objb)$,~$\morb \in \Hom_\Set(\Objb,\Objc)$ such that~$\mora \in \Hom_\Injset(\Obja,\Objb)$ and~$\morb\in \Hom_\Injset(\Objb,\Objc)$. From the injectivity of~$\mora,\morb$, we know that given~$\obja,\obja'\in \Obja$, $\mora(\obja)=\mora(\obja') \Leftrightarrow \obja=\obja'$ and~$\objb,\objb'\in \Obja$,~$\morb(\objb)=\morb(\objb') \Leftrightarrow \objb=\objb'$. Furthermore, we have:
    \begin{equation*}
        \begin{aligned}
        (\mora\then \morb)(\obja)
            =(\mora \then \morb)(\obja')&\implies \mora(\obja)=\mora(\obja')\\
            &\implies \obja=\obja',
        \end{aligned}
    \end{equation*}
    which proves the fourth condition of \cref{def:subcategory}, i.e. that the composition of injective functions is injective.
\end{example}



\book{
    \begin{definition}[Continuous functions]
        Let~$f\colon \reals\to \reals$ be a function. We call~$f$ \emph{continuous} at~$c\in \reals$ if~$\lim_{x\to c}f(x)=f(c)$;~$f$ is continuous over~$\reals$ if the condition is satisfied for all~$c\in \reals$.

        \begin{example}
            We can define a category~$\Cat{Cont}$ which~$\Ob_\Cat{Cont}=\reals$ and in which the morphisms are given by continuous functions. Composition and identity are as in~\Set. We want to show that $\Cat{Cont}$ is a subcategory of~\Set.
            \todo{write down formally and use that composition of continuous is continuous}
        \end{example}
    \end{definition}
    \todo{Continuous functions (topologies)}
    \todo{Differentiable functions: Set to Manifolds}

    \begin{definition}[Differentiable functions]
        A function $f\colon U\subset \reals\to \reals$, defined on an open set $U$, is \emph{differentiable} at $a\in U$ if the derivative
        \begin{equation}
            f'(a)=\lim_{h\to 0} \frac{f(a+h)-f(a)}{h}
        \end{equation}
        exists; $f$ is differentiable on $U$ if it is differentiable at every point of $U$.
    \end{definition}

    \begin{example}
        the composition of differentiable functions is differentiable
    \end{example}


    \todo{Lipschitz bounded}

    \begin{definition}[Lipschitz continuous function]
        A real valued function $f\colon \reals\to \reals$ is called \emph{Lipschitz} continuous if there exists a positive real constant $\kappa$ such that, for all $x_1,x_2\in \reals$:
        \begin{equation}
            \vert f(x_1)-f(x_2)\vert \leq \kappa \vert x_1-x_2\vert.
        \end{equation}
    \end{definition}

    \begin{example}
        the composition of differentiable functions is differentiable
    \end{example}

    \todo{smooth}
    \todo{cont diff, composition is compdiff}
%\begin{exercise}
%Check that \Set, as specified above, does in fact define a category.
%\end{exercise}

    \subsubsection{Generalization outside of R}
    Generalization to more general spaces.
    We used the fact thath R is:
    \begin{itemize}
        \item For defining continuous functions, we used the fact that R is topological space (minimum needed for defining a continuous function.). In fact, the real definitoin of continuous function is:

        \todo{ADD: continuous function}

        \item To define Lipschitz we needed the fact that R is a metric space
        \item For differentiable, smooth, you define this on Manifolds. Exists tangent space.
    \end{itemize}

    Hence in general, the objects of these are different, so it's not really a relation of subcategory, that requires
    the objects to be the same. However we will see later that you can generalize this notion using functors.

    \todo{functor F:C→D that is both injective on objects and a faithful functor.}
}

