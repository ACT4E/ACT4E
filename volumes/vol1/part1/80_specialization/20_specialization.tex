% !TEX root = standalone.tex
\label{sec:specialization}



\section{Notion of subcategory}
\index{subcategory}
\begin{ctdefinition}[Subcategory]
    \label{def:subcategory}
    A \emph{\iindex{subcategory}}~\CatD of a category~\CatC is a category for which:
    \begin{compactenum}
        \item All the objects in~$\ObD$ are in~$\ObC$;
        \item For any objects~$\Obja,\Objb\in \ObD$,~$\Hom_\CatD(\Obja,\Objb)\subseteq \Hom_\CatC(\Obja,\Objb)$;
        \item If~$\Obja\in \ObD$, then~$\id_\Obja \in \Hom_\CatC(\Obja,\Obja)$ is in~$\Hom_\CatD(\Obja,\Obja)$ and acts as its identity morphism;
        \item If~$\mora \colon \Obja \to \Objb$ and~$\morb\colon \Objb\to \Objc$ in \CatD, then the composite~$\mora \then \morb$ in \CatC is in~\CatD and represents the composite in~\CatD.
    \end{compactenum}
\end{ctdefinition}

Two important examples of subcategory are the following.

\begin{example}[Finite Sets]
    \iindex{\FinSet} is the category of finite sets and all functions between them. It is a subcategory of the category \Set of sets and functions. While an object~$\Obja \in \Ob_\Set$ is a set with arbitrary cardinality,~$\Ob_{\FinSet}$ only includes sets which have finitely many elements. Objects of \FinSet are in \Set, but the converse is not true. Furthermore, given~$\Obja,\Objb\in \Ob_\FinSet$, we take~$\Hom_{\FinSet}(\Obja,\Objb)=\Hom_{\Set}(\Obja,\Objb)$.
\end{example}


\begin{example}[\Set and \Rel]
    The category \Set is a subcategory of \Rel. To show this, we need to prove the conditions presented in \cref{def:subcategory}.
    \begin{enumerate}
        \item In both \Rel and \Set, the collection of objects is all sets.
        \item Given~$\Obja,\Objb\in \Ob_{\Set}$, we know that~$\Hom_{\Set}(\Obja,\Objb)\subseteq \Hom_{\Rel}(\Obja,\Objb)$, i.e., that all functions between sets~$\Obja,\Objb$ are a particular subset of all relations between~$\Obja,\Objb$.
        \item For each~$\Obja \in \Ob_{\Set}$, the identity relation~$\id_\Obja=\{\tup{\obja,\obja'}\in \Obja\times \Obja \mid \obja=\obja'\}$ corresponds to the identity function~$\id_\Obja \colon \Obja \to \Obja$ in \Set.
        \item Let~$R\subseteq \Obja\times \Objb$ and $S\subseteq \Objb \times \Objc$ be relations which are functions. We need to show that their composition in \Rel, expressed as~$R\then S\subseteq \Obja\times \Objc$, is again a function. This was proven in \cref{lem:comprelfun}.
    \end{enumerate}

\end{example}

