% !TEX root = standalone.tex


\section{Relations}\label{sec:connection-relations}

A basic mathematical notion which underlies the above discussion is that of a \textbf{binary relation}.


\begin{definition}[Binary relation]
  \label{def:binary-relation}
  A \emph{\iindex{binary relation}} from a set~$X$ to a set~$Y$ is a subset of the Cartesian product $X\times Y$.
\end{definition}

\begin{remark}
  We will often drop the word ``binary'' and simply use the name ``relation''.
\end{remark}

%\begin{remark}
%There is also the notion of an $n$-ary relation, where $n$ can be any natural number. An $n$-nary relations is a subset of a cartesian product of the form $X_1 \times \dots \times X_n$.
%\end{remark}

If~$X$ and~$Y$ are finite sets, we can depict a relation~$R \subseteq X \times Y$ graphically as in \cref{fig:example_rel}. For each element~$\tup{x,y} \in X \times Y$, we draw an arrow from~$x$ to~$y$ if and only if~$\tup{x,y} \in R \subseteq X \times Y$.

\begin{figure}[h!]
  \centering
  \includesag{30_rel_1}
%\includegraphics[width=0.5\linewidth]{pics/dist_net_7.png}
  \caption{\label{fig:example_rel}}
\end{figure}

We can also depict this relation graphically as a subset of~$X \times Y$ in a ``coordinate system way'', as in \cref{fig:example_rel_coord}. The shaded grey area is the subset~$R$ defining the relation.
%
%\begin{comment}
%    \begin{figure}[h!]
%        \centering
%        \includegraphics[width=0.5\linewidth]{dist_net_8}
%        \caption{}
%    \end{figure}
%\end{comment}

\begin{figure}[h!]
  \begin{center}
    \includesag{30_rel_graph}
  \end{center}
  \caption{Relations visualized in ``coordinate systems''.}
  \label{fig:example_rel_coord}
\end{figure}

\begin{exercise}
  Let~$X = Y = \{1, 2, 3, 4 \}$ and consider the relation~$R \subseteq X \times Y$ defined by
  \begin{equation}
    R = \{ \tup{x,y} \in X \times Y \mid x \leq y \}.
  \end{equation}

  Visualize the relation~$R$ via the method in \cref{fig:example_rel} and \cref{fig:example_rel_coord} each.
\end{exercise}

The visualization in \cref{fig:example_rel} hints at the fact that we can think of a relation~$R \subseteq X \times Y$ as a \emph{morphism} from~$X$ to~$Y$.

\begin{ctdefinition}[Category \Rel]
  \label{def:Rel}
  The category \iindex{\Rel} of relations \Rel is given by:
  \begin{compactenum}
    \item \emph{Objects}: The objects of this category are all sets.
    \item \emph{Morphisms}: Given sets~$\Obja, \Objb$, the homset~$\Hom_{\Rel}(\Obja,\Objb)$ consists of all
    relations~$R\subseteq \Obja\times \Objb$.
    \item \emph{Identity morphisms}: Given a set~$\Obja$, its identity morphism is
    \begin{equation}
      1_\Obja \coloneqq \{ \tup{\obja,\objb} \mid  \obja = \objb \}.
    \end{equation}
    \item \emph{Composition}: Given relations~$R \colon \Obja\to \Objb$,~$S\colon \Objb\to \Objc$, their composition is given by
    \begin{equation}
      \label{RelCompRule}
      R \then S \coloneqq \{\tup{\obja,\objc} \mid  \exists \objb \in \Objb \colon \ \left(\tup{\obja,\objb} \in R\right) \wedge \left(\tup{\objb,\objc} \in S\right)\}.
    \end{equation}
  \end{compactenum}
\end{ctdefinition}

To illustrate the composition rule in \cref{RelCompRule} for relations, let's consider a simple example, involving sets~$\Obja$,~$\Objb$, and~$\Objc$, and relations~$R \colon \Obja \to \Objb$ and $S \colon \Objb \to \Objc$, as depicted graphically below in \cref{fig:example_rel_composable}.
\begin{figure}[h!]
  \centering
  \includesag{30_rel_2}
  \caption{Relations compatible for composition.}
  \label{fig:example_rel_composable}
\end{figure}
Now, according to the rule in \cref{RelCompRule}, the composition~$R \then S \subseteq \Obja \times \Objc$ will be such that~$\tup{\obja,\objc} \in R \then S$ if and only if there exists some~$\objb \in \Objb$ such that~$\tup{\obja,\objb} \in R$ and~$\tup{\objb,\objc} \in S$, which, graphically, means that for~$\tup{\obja,\objc}$ to be an element of the relation~$R \then S$,~$\obja$ and~$\objb$ need to be connected by at least one sequence of two arrows such that the target of the first arrow is the source of the second. For example, in \cref{fig:example_rel_composable}, there is an arrow from~$\obja_2$ to~$\objb_3$, and from there on to~$\objc_1$, and therefore, in the composition~$R \then S$ depicted in \cref{fig:example_rel_composed}, there is an arrow from~$\obja_2$ to~$\objc_1$.
\begin{figure}[h!]
  \centering
  \includesag{30_rel_3}
%\includegraphics[width=0.5\linewidth]{pics/dist_net_10.png}
  \caption{Composition of relations.}
  \label{fig:example_rel_composed}
\end{figure}

\begin{remark}
  Relations with the same source and target can be \emph{compared} via inclusion. Given~$R\subseteq X\times Y$ and~$R'\subseteq X\times Y$, we can ask whether~$R\subseteq R'$ or~$R'\subseteq R$.
\end{remark}
