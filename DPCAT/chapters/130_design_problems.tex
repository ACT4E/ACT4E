% !TEX root = ../CategoricalCoDesign.tex
\section{Defining design problems}
\subsection{Defining design problems from catalogues}
\begin{shaded}
\begin{definition}[Span]
Given a category $\Cat{C}$, a \emph{span} from an object $x$ to an object $y$ is a diagram of the form
\begin{equation}
\includesag{51_span}
\end{equation}
where $z$ is some other object of $\Cat{C}$. 
\end{definition}
\end{shaded}
\begin{example}
Consider the category $\mathbf{trek}$, introduced in \cref{sec:trekking}. An example of span in this category is reported in~\cref{fig:exmountains}.
\begin{figure}[h!]
\begin{center}
\includesag{130_mountains}
\end{center}
\caption{Example of span in $\mathbf{trek}$. \label{fig:exmountains}}
\end{figure}
Recall that $\mathsf{Matterhorn}$, $\mathsf{Jungfrau}$, and $\mathsf{Pilatus}$ are objects of $\mathbf{trek}$, and the arrows are morphisms in $\mathbf{trek}$.
\end{example}

\begin{definition}[Catalogue] \label{def:catalogue}
A \emph{catalogue} is a span in $Pos$.
It thus consists of 3 posets $I$,$\F{F}$, $\R{R}$.
We call them implementation space, functionality space, requirements spaces. We need to define two maps $\text{provides} \colon I \to \F{F}$
and $\text{requires} \colon I \to \R{R}$ (mnemonics for \emph{evaluation}).
\begin{equation}
\includesag{130_catalogue}
\end{equation}
\end{definition}
\todo{Let's use "provides" and "requires" rather than exec/eval}

\begin{definition}[Design problem induced by a catalogue]
Every \emph{catalogue} $\tup{I,\exec,\eval}$ induces a design problem of the form $d\colon \exec(I)\tickar \eval(I)$.
\end{definition}
\todo{F to R}
\begin{example}
\todo{Add example once we decide the leading one}
\end{example}

