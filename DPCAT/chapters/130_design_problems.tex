% !TEX root = ../CategoricalCoDesign.tex
\section{Creating design problems}
\subsection{Creating design problems from catalogues}
\begin{shaded}
\begin{definition}[Span]
Given a category~$\Cat{C}$, a \emph{span} from an object~$x$ to an object~$y$ is a diagram of the form
\begin{equation}
\includesag{51_span}
\end{equation}
where~$z$ is some other object of~$\Cat{C}$. 
\end{definition}
\end{shaded}
\begin{example}
Consider the category~$\mathbf{trek}$, introduced in \cref{sec:trekking}. An example of span in this category is reported in~\cref{fig:exmountains}.
\begin{figure}[h!]
\begin{center}
\includesag{130_mountains}
\end{center}
\caption{Swiss peaks can be thought of as a span in~$\mathbf{trek}$. \label{fig:exmountains}}
\end{figure}
Recall that~$\mathsf{Matterhorn \ Peak}$,~$\mathsf{Jungfrau \ Peak}$, and~$\mathsf{Pilatus \ Peak}$ are objects of $\mathbf{trek}$, and the arrows are morphisms in~$\mathbf{trek}$ (paths from one location to the other).
\end{example}

\begin{definition}[Catalogue] \label{def:catalogue}
A \emph{catalogue} is a span in~$\Pos$.
It thus consists of 3 posets~$I$,~$\F{F}$,~$\R{R}$.
We call them implementation space, functionality space, and requirements space, respectively. We need to define two map~$\prov \colon I \to \F{F}$ (an implementation \textbf{prov}ides a functionality) and~$\req\colon I \to \R{R}$ (an implementation \textbf{req}uires resources). 
\begin{equation}
\includesag{130_catalogue}
\end{equation}
\end{definition}

\begin{definition}[Design problem induced by a catalogue]
Every catalogue~$\tup{I,\prov,\req}$ \emph{induces} a design problem of the form~$d\colon \F{F}\tickar \R{R}$, with
\begin{equation}
    \begin{aligned}
    d\colon \F{F}\op \times \R{R}&\to \Bool\\
    \tup{\F{f}^*,\R{r}}&\mapsto \bigvee_{i\in I}\left(\prov(i)\ordleq_{\F{F}}\F{f} \right)\wedge \left( \req(i)\ordgeq_{\R{R}}\R{r}\right)
    \end{aligned}
\end{equation}
\end{definition}

\begin{example}
\GZ{here take again the example of the motors from previous sections}
The choice of a battery can be modeled as a design problem with the provided capacity~$\F{C_\mathrm{prov}}$ (in \unit[]{Joules}) and the number of missions~$\F{M_\mathrm{b}}$ (unitless) as functionalities, and its cost~$\R{c_\mathrm{b}}$ (in \unit[]{USD}), its mass~$\R{m_\mathrm{b}}$ (in \unit[]{kg}), and the number of its replacements during the lifetime of a robot~$\R{R_\mathrm{b}}$ (unitless) as resources. The design problem is depicted in~\cref{fig:battery}. 

\begin{figure}[tbh]
\begin{center}
    \begin{tikzpicture}[DP]
    \node[dp={2}{3}] (bat) {Battery};
    \draw[runconn, runame={$c_\mathrm{b}$}] (bat_res1){};
    \draw[runconn, runame={$m_\mathrm{b}$}] (bat_res2){};
    \draw[runconn, runame={$R_\mathrm{b}$}] (bat_res3){};
    \draw[funconn, funame={$C_\mathsf{prov}$}] (bat_fun1){};
    \draw[funconn, funame={$M_\mathsf{b}$}] (bat_fun2){};
\end{tikzpicture}
\end{center}
\caption{The battery design problem.\label{fig:battery}}
\end{figure}

The relation between functionalities and resources is described as follows. First, mass and provided capacity are related as
\begin{equation}
    \R{m_\mathrm{b}}\succeq \frac{\F{C_\mathrm{prov}}}{\rho_\mathrm{b}},
\end{equation}
where $\rho_\mathrm{b}$ represents the specific energy of the battery, expressed in $\unitfrac[]{J}{kg}$. Second, battery replacements and missions are related as
\begin{equation}
    \R{R_\mathrm{b}}\succeq \ceil{\frac{\F{M_\mathrm{b}}}{\ell}},
\end{equation}
where $\ell$ represents the battery lifetime, expressed in number of cycles. Finally, cost, missions, and provided capacity are related as
\begin{equation}
    \R{c_\mathrm{b}}\succeq  \ceil{\frac{\F{M_\mathrm{b}}}{\ell}} \frac{\F{C_\mathrm{prov}}}{\alpha},
\end{equation}
where $\alpha$ represents the battery's specific cost, expressed in \unitfrac[]{J}{\$}. An example of the specifications of battery technologies is given in~\cite{censi2015}, and is shown in~\cref{tab:battery}.

\begin{table}[tbh]
\begin{center}
\begin{tabular}{cccc}
Technology&Specific energy [\unitfrac[]{J}{kg}]&Specific cost [\unitfrac[]{J}{\$}]&Life [\# cycles]\\
\hline
$\mathsf{NiMH}$&100.0&3.41&500\\
$\mathsf{NiH2}$&45.0&10.5&20,000\\
$\mathsf{LCO}$&195.0&2.84&750\\
$\mathsf{LMO}$&150.0&2.84&500\\
$\mathsf{NiCad}$&30.0&7.50&500\\
$\mathsf{SLA}$&30.0&7.00&500\\
$\mathsf{LiPo}$&250.0&2.50&600\\
$\mathsf{LFP}$&90.0&1.50&1,500
\end{tabular}
\end{center}
\caption{Specifications of common battery technologies~\cite{censi2015}. \label{tab:battery}}
\end{table}
\todo{finish with design problem description}
\end{example}

