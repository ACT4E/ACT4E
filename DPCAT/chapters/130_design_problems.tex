% !TEX root = ../CategoricalCoDesign.tex
\section{Defining design problems}
\subsection{Defining design problems from catalogues}
\todo{span theory here}
\begin{shaded}
\begin{definition}[Span]
Given a category $\Cat{C}$, a \emph{span} from an object $x$ to an object $y$ is a diagram of the form
\begin{center}
\includesag{51_span}
\end{center}
where $z$ is some other object of $\Cat{C}$. 
\end{definition}
\end{shaded}
\subsection{Companion and Conjoint}
We round out our discussion of $\DP$ by introducing two formulae for transforming monotone maps in $\Poset$ into design problems in $\DP$. Each monotone map $f$ can be transformed into two design problems, called its \emph{companion} $\comp{f}$ and \emph{conjoint} $\conj{f}$. Many of the design problems that we have introduced can be realized as companions and conjoints of appropriate monotone maps.

\begin{definition}[Companion and conjoint]
\label{def:comp_conj}
Let $\cP $ and $\cQ $ be posets, and suppose that $f\colon\cP \toinPos \cQ $ is a monotone map. We define its \emph{companion} in $\DP$, denoted $\comp{f}\colon \F{\cP} \tickar \R{\cQ}$,
and its \emph{conjoint}, denoted $\conj{f}\colon \F{\cQ} \tickar \R{\cP}$ as
\begin{equation}
\comp{f}(\F{p}^*,\R{q})\coloneqq f(\F{p}) \leq_\cQ \R{q}
\qquad\text{and}\qquad
\conj{f}(\F{q}^*,\R{p})\coloneqq \F{q} \leq_\cQ f(\R{p}).
\end{equation}
\end{definition}

\GZ{
\begin{example}The identity design problem $\id_A\colon \F{A} \tickar \R{A}$ is the companion (and the conjoint) of the identity map $\id_A'\colon A \toinPos A$. This is easy to check as
\begin{equation}
    \begin{aligned}
    \comp{\id}_A'(\F{a_1}^*,\R{a_2})&=\id_A'(\F{a_1})\leq_A \R{a_2}\\
    &=\F{a_1}\leq \R{a_2}\\
    &=\id_A(\F{a_1}^*,\R{a_2}).
    \end{aligned}
\end{equation}
\end{example}}

\begin{example}The coproduct injections $\iota_A, \iota_B$ for design problems are the companions of the coproduct injections for the disjoint union.
\end{example}

\begin{example}The product projections $\pi_A, \pi_B$ for design problems are the conjoint of the coproduct injections for the disjoint union.\end{example}

We can also re-define the sum $\vee$ and intersection $\wedge$ using companions and conjoints, which allows us to introduce some useful constructions.

\begin{definition}[Diagonal function]
Define the \emph{diagonal function} $\Delta_P\colon P \to P \times P$:
\begin{equation}
\begin{aligned}
    \Delta_P \colon P & \to P \times P, \\
             p & \mapsto \tup{p, p}.
\end{aligned}
\end{equation}
\end{definition}

\begin{definition}[Codiagonal function]
Define the \emph{codiagonal function} $\Diamond_P\colon P+P \to P $:
\begin{equation}
\begin{aligned}
    \Diamond_P \colon P + P & \to P,  \\
            \tup{1,p} & \mapsto p, \\
            \tup{2,p} & \mapsto p.
\end{aligned}
\end{equation}
\end{definition}

Using the diagonal function, \cref{lemma:intersection} can be rewritten as the following Lemma.

\begin{lemma}
    Given $f, g\colon \F{A} \tickar \R{B}$, we have:
    \begin{equation}
        f \vee g =  \conj{\Diamond}_A \then (f + g)\then \comp{\Diamond}_B.
    \end{equation}
\end{lemma}

Similarly, using the codiagonal function, one can prove the following Lemma.
\begin{lemma}
    Given $f, g\colon \F{A} \tickar \R{B}$, we have:
    \begin{equation}
        f \wedge g = \comp{\Delta}_A ;\then(f + g) \then \conj{\Delta}_B.
    \end{equation}
\end{lemma}
\todo{write proofs here}
\begin{proof}Just write out the composition.\end{proof}
Unlike $\conj{\Diamond} = \text{split}$ and $\comp{\Diamond} = \text{fuse}$, $\comp{\Delta}$ and $\conj{\Delta}$ do not have an intuitive diagrammatic representation.