% !TEX root = ../CategoricalCoDesign.tex
\section{The Traced Monoidal Category of Design Problems}
\todo{Add exhaustive motivation for this in engineering, touching all the properties we are going to define later and their utility in real engineering life (parallel, loop, series, choice, why orders why posets}
\todo{Important, quantitative does not mean numerical, but comparable here}
\todo{Maybe through real world example, like robot?}

Category theory is a lightweight, economical language for talking about \emph{structure}: It allows us to cleanly abstract from the set-theoretic data of upper sets and antichains, as found in~\cite{censi16}, in favor of more intrinsic definitions of ``the things that engineers do''.

For example, an engineer will often plug the output of one design component as the input to another design component, in a series. She will consider combinations of components in parallel. And she will sometimes use components (or combinations of components) that consume their own output, e.g. a car motor that also has to carry its own weight.

\begin{equation*}
\includesag{50_washer}
\tag{series}
\end{equation*}

\AP{hopefully dirty and clean should be switched}

\begin{equation*}
\includesag{50_parallel}
\tag{parallel}
\end{equation*}

\todo{Change example in something more real}
\begin{equation*}
\includesag{50_loop}
\tag{loop}
\end{equation*}

In this section, we will show that the operations of forming series, parallels, and loops of design problems correspond to the composition, monoidal product, and trace in the category of design problems, $\DP$. These three properties then give a convenient semantics---that of traced symmetric monoidal categories---for the diagrams we use throughout this paper.




\subsection{to put back up}


Characterizing the class of all design problems as a category allows us to define several additional operations on design problems. From now on, we will also use the category-theoretic names, for example `composition' instead of `series'. Following \cite{censi16}, we begin with the monoidal product (a.k.a. `parallel'), the biproduct, and the trace (a.k.a. `loop'). % In the following, we will study different ways to interconnect and compose design problems, in addition to the series composition represented by ``;''.

\begin{figure}[h!]
\centering
\begin{subfigure}{0.2\textwidth}
\centering
\includesag{50_sum_series}
\caption{Series: $(f \then g)$.}
\end{subfigure}
\hspace{10mm} % add space between figures
\begin{subfigure}{0.2\textwidth}
\centering
\includesag{50_sum_parallel}
\caption{Parallel: $f \otimes g$.}
\end{subfigure}
\hspace{10mm} % add space between figures
\begin{subfigure}{0.2\textwidth}
\centering
\includesag{50_sum_biproduct}
\caption{Biproduct: $f + g$.}
\end{subfigure}
\hspace{10mm} % add space between figures
\begin{subfigure}{0.2\textwidth}
\centering
\includesag{50_sum_loop}
\caption{Loop: $\Tr f$.}
\end{subfigure}
\label{fig:diagrams}
\end{figure}

\begin{table}[t!]
    \centering
\begin{tabular}{c|c|c|crl}
    series &
    $f:A\tickar B$&
    $g:B\tickar C$&
    $f\then g:$&$A$&$\tickar C$ \\
    %
    sum &
    $f:A\tickar B$ &
    $g:A\tickar B$ &
    $f\vee g:$&$A$&$\tickar B$ \\
    %
    intersection &
    $f:A\tickar B$ &
    $g:A\tickar B$ &
    $f\wedge g:$&$A$&$\tickar B$ \\
    %
    monoidal product &
    $f:A\tickar C$&
    $g:B\tickar D$ &
    $f\otimes g:$&$A\times B$&$\tickar C \times D$ \\
    %
    product &
    $f:A\tickar C$&
    $g:A\tickar D$ &
    $f\times g:$&$A $&$\tickar C + D$ \\
    %
    coproduct &
    $f:A\tickar C$&
    $g:B\tickar C$ &
    $f\sqcup g:$&$A + B $&$\tickar C$ \\
    %
    biproduct &
    $f:A\tickar B$ &
    $g:A\tickar B$ &
    $f+ g:$&$A + A$&$\tickar B + B$ \\
    %
    trace &
    $f: C \times A \tickar C \times B$ &
    -&
    $\Tr_{A,B}^C(f) :$&$A$&$\tickar B$
\end{tabular}
    \caption{Various composition operations on design problems (i.e. morphisms) in $\DP$.}
\end{table}