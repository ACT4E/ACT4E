% !TEX root = ../CategoricalCoDesign.tex
\section{Thinking about resource composition}
\label{sec:productset}
\subsection{Products}
Is ``$X \times Y$'' the same as ``$Y \times X$''?
It depends on the context. Intuitively, for the categories of resources treated in this work, we would not make a distinction between ``having~$X$ and~$Y$'' and ``having~$Y$ and~$X$''.
However, there are contexts in which this is not valid. For example, if we are
using~$X \times Y$ to mean that we will have the resource~$X$ today, and the
resource~$Y$ tomorrow, then~$Y \times X$ would not be the same as$X \times Y$. 

In the contexts in which this symmetry holds, we call the category ``symmetric monoidal'': We will talk about this more in detail in \cref{sec:feedbackindesign}.

The word ``symmetric'' is well-suited, because to say that the two objects are equivalent, or ``isomorphic'', we can postulate that we always have a way to get
``$X \times Y$'' starting from ``$Y \times X$'' and viceversa. Diagrammatically, this is depicted by the two arrows that connect them~(\cref{fig:e17}).

\begin{figure}[h!]
    \centering
    \includesag{30_dpcatfig_e17}
    \caption{Objects equivalence in the symmetric case. \label{fig:e17}}
\end{figure}

Consider the following example: To generate~$\mathsf{motion}$, one needs \emph{both} an~$\mathsf{engine}$ \emph{and} some~$\mathsf{fuel}$.

We can consider~$\mathsf{motion}$,~$\mathsf{engine}$, and~$\mathsf{fuel}$ as resources and draw them as points (\cref{fig:e11}).

\begin{figure}[h!]
    \centering
    \includesag{30_dpcatfig_e11}
    \caption{Motion generation as resources. \label{fig:e11}}
\end{figure}

However, it is not clear how to represent the arrow corresponding to~$\mathsf{fuel} + \mathsf{engine} = \mathsf{motion}$, because arrows only have one head and one tail.
\JL{Perhaps this particular example would be confusing to the reader, since earlier ``engine'' was used as a morphism, while here it is being used as an object (??)}
One clean way to formalize this is to expand the objects in the category, by postulating that, if two resources~$X$ and~$Y$ exist, then there exists also the resource ``$X$ \emph{and}~$Y$''.

In our example, we can postulate the existence of an object $\mathsf{engine} \emph{ and }\mathsf{fuel}$ and then draw the arrow from it to motion (\cref{fig:e13}).

\begin{figure}[h!]
    \centering
    \includesag{30_dpcatfig_e13}
    \caption{Engine and fuel can generate motion. \label{fig:e13}}
\end{figure}

We call this construction ``product'', because it is equivalent to
the Cartesian product of sets.

\begin{definition}[Cartesian product of sets]
\label{def:cartesian-product}
   Given two sets~$A,B$, their \emph{cartesian product} is denoted~$A\times  B$
   and defined as 
   \begin{equation}
       A\times  B=\{ \tup{a,b}\mid a\in A\text{ and } b\in B\}.
   \end{equation}
\end{definition}

\begin{example}
Consider the sets~$\{\diamond,\star\}$ and $\{\dagger, \ast\}$. Their product can be represented as in \cref{fig:cartesian-product}.
\begin{figure}[h!]
    \centering
    \includesag{50_cartesian_product}
    \caption{Example of cartesian product of two sets.\label{fig:cartesian-product}}
\end{figure}
\end{example}

Given the Cartesian product of two sets, we can define two \emph{projection maps}~$\pi_1$ and~$\pi_2$, which, given an element of the product, will return the first or the second element, respectively:
\begin{equation}
\begin{aligned}
    \pi_1(\tup{a,b}) &\coloneqq a,\\
    \pi_2(\tup{a,b}) &\coloneqq b.
\end{aligned}
\end{equation}

In our case, we would be able to say that, given both~$\mathsf{resource}_1$ and~$\mathsf{resource}_2$ together, we can recover~$\mathsf{resource}_1$ and~$\mathsf{resource}_2$ separately~(\cref{fig:resource-product}).

\begin{figure}[h!]
    \centering
    \includesag{30_recover}
    \caption{Two projection maps. \label{fig:resource-product}}
\end{figure}


Later, we will see other examples of products, such as products of partially ordered sets.
All these products are an instance of a more general concept of product that exists in category
theory.
 
\begin{shaded}
\begin{definition}[Categorical Product]
Let~$\CatC$ be a category and let~$A,B\in \CatC$ be objects. The \emph{product} of~$A$ and~$B$ is an object~$A \times B \in \CatC$ together with \emph{projection morphisms}~$\pi_A \colon A \times B \to A$ and~$\pi_B \colon A \times B \to B$, such that, given any~$X \in \CatC$ and morphisms~$f \colon X \to A, g \colon X \to B$, there exists a \emph{unique} morphism~$(\prodMap{f}{g}) \colon X \to A \times B$ such that~$f = (\prodMap{f}{g})\then \pi_A$ and~$g=(\prodMap{f}{g})\then \pi_B$:
\begin{equation}
\includesag{50_defproduct}
\end{equation}
\end{definition}
\end{shaded}

The formulation, slightly technical, hints at uniqueness properties or ``universality''. The definition implies that if a product exists, then it is unique, in the sense that all the admissible products can be derived from each other. Therefore, when presented with a new category, one can guess what construction might be the product, and check the definition to make sure that it is indeed \emph{the} product.

\GZ{JL: need to reformulate this last part}
