% !TEX root = ../CategoricalCoDesign.tex
\section{Thinking about alternatives}
\subsubsection{Coproduct}

There exists a dual notion to ``product'' that is called ``coproduct''. Here is the motivation in the context of engineering design.
Suppose that we are considering a hybrid car that contains two engines: An
electric engine and an internal combustion engine. Both can produce motion, from
different sources of energy. The electric engine uses electric energy; the
internal combustion engine uses gasoline. The situation is as in \cref{fig:e16a}.

\begin{figure}[h!]
    \centering
    \includesag{30_dpcatfig_e14}
    %\includegraphics[scale=0.33]{dpcatfig_e14}
    \caption{\label{fig:e16a}}
\end{figure}

From this we would like to conclude that we can obtain motion
from \emph{either} gas \emph{or}
electric energy~(\cref{fig:e16b}).

\begin{figure}[h!]
    \centering
    \includesag{30_dpcatfig_e15}
    %\includegraphics[scale=0.33]{dpcatfig_e15}
    \caption{\label{fig:e16b}}
\end{figure}

To define the idea of ``\emph{either gas \emph{or} electric energy}'' we can
refer to the idea of disjoint union of sets (\cref{def:disjoint-union}). The
disjoint union of two sets is a set that contains two complete copies of the two
sets. If an element is contained in both sets, there will be two distinct copies
in the disjoint union.

\begin{definition}[Disjoint union]
\label{def:disjoint-union}
    The \emph{disjoint union} of two sets $A$ and $B$ is denoted $A + B$
    and it is defined as
    \begin{equation}
        A + B=\{\tup{1,a}\mid a\in A\}\cup\{\tup{2,b}\mid b\in B\}.
    \end{equation}
\end{definition}


\begin{figure}[h!]
    \begin{center}
    \includesag{30_disjoint_union}
    \end{center}
    \caption{Example of disjoint union.}
\end{figure}


We can define the disjoint union of a set with itself; this is equivalent
to having two distinct copies of the set~(\cref{fig:disjointself}).

\begin{figure}[h!]
\begin{center}
\includesag{30_disjoint_union_self}
    \caption{Disjoint union of a set with itself \label{fig:disjointself}.}
\end{center}
\end{figure}

The dsjoint union is a particular instance of the notion of ``coproduct''.
The following definition is the generic definition of coproducts for an arbitrary category.

\begin{shaded}
\begin{definition}[Coproduct]
Let $\CatC$ be a category and let $A, B \in \CatC$ be objects. The \emph{coproduct} of $A$ and $B$ is an object $A \sqcup B \in \CatC$ together with two \emph{inclusion maps} $\iota_A \colon A \to A \sqcup B $ and $\iota_B \colon B \to A  \sqcup B $, such that, given any $X \in \CatC$ and morphisms $f\colon  A \to X, g \colon B \to X$, there exists a \emph{unique} morphism $(\coprodMap{f}{g}) \colon A \sqcup B \to X$ such that $f = \iota_A\then (\coprodMap{f}{g})$ and $g = \iota_B \then (\coprodMap{f}{g})$.
\end{definition}
\end{shaded}

\begin{example}

The two inclusion maps for the disjoint union are $\iota_1\colon A \to A + B$ and $\iota_2\colon B \to A + B$, defined as:
\begin{equation}
\begin{aligned}
    \iota_1(a) &= \tup{1, a},\\
    \iota_2(b) &= \tup{2, b}.
\end{aligned}
\end{equation}

The map $\coprodMap{f}{g}$ for the disjoint union is
\begin{equation}
\begin{aligned}
    \coprodMap{f}{g} \colon  A + B &\to X \\
    y &   \mapsto
    \begin{cases}
        f(y), & \text{if } y \in A, \\
        g(y), & \text{if } y \in B.
    \end{cases}
\end{aligned}
\end{equation}
\end{example}

\todo{figure that shows the injections: from ``gas'' I can get ``gas or electricity''.}


\todo{collocate it well}
$X \sqcup Y$ is different from $Y \sqcup X$, but the two are isomorphic~(\cref{fig:e16}).

\begin{figure}[h!]
    \centering
    \includesag{30_dpcatfig_e16}
    \caption{\label{fig:e16}}
\end{figure}
