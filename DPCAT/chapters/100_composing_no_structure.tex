% !TEX root = ../CategoricalCoDesign.tex
\section{Composing without structure}
\AC{Here we need a whole discussion.}
\begin{definition}[Category~$\Rel$]
    The category $\Rel$ is defined by:
    \begin{compactenum}
    \item \emph{Objects}: The objects of this category are all sets.
    \item \emph{Morphisms}: The morphisms between any pair of sets~$X, Y$
    are relations from~$X$ to~$Y$, $R\subseteq X\times Y$.
    \item \emph{Identity morphism}: The identity morphism for the set~$X$
    is the diagonal morphism $\delta_X \colon X\to X\times X$.
    \item \emph{Composition operation}: The composition operation of two relations $R \colon X\to Y$, $S\colon Y\to Z$ is given by
    \begin{align}
    (x,z) \in (R \then S) \qquad \equiv \qquad  \exists y \in Y, \ (x,y) \in R \wedge (y,z) \in S.  	
    \end{align}
    \end{compactenum}
\end{definition}

\AC{We are using rangle/langle for tuples, right?}
\begin{lemma}
    The category~$\Cat{Set}$ is a subcategory of~$\Cat{Rel}$.
\end{lemma}
\begin{proof}
	We need to prove the conditions presented in \cref{def:subcategory}.
	\begin{enumerate}[(i)]
	\item If a morphism of $\Rel$ $f \colon X\to Y$ is in $\Set$, then so are the objects $X$ and $Y$. Both $\Rel$ and $\Set$ have sets as objects. If a morphism of $\Rel$ $f\colon X\to Y$ is in $\Set$, then a relation $F\subseteq X\times Y$ between set $X$ and set $Y$ exists. This relation can be expressed in $\Set$ as $f\colon X\to Y$ and hence the objects $X$ and $Y$ exist.
	\item Two morphisms in $\Rel$ $f\colon X\to Y$ and $g\colon Y\to Z$ are relations $F\subseteq X\times Y$, $G\subseteq Y\times Z$. If they are in $\Set$, they can be written as $f\colon X\to Y$ and $g\colon Y\to Z$ and their composition is in $\Set$ as well. 
	\item If an object of $\Rel$ $X$ is in $\Set$, then so is the identity morphism $\text{Id}_X$. This is true: for every object $X$ of $\Set$ there exists the identity morphism $\text{Id}_X:X\to X$. The diagonal morphism can be expressed as $\text{Id}_X:X\to X$.
	\end{enumerate}
\end{proof}
