% !TEX root = ../CategoricalCoDesign.tex

\todo{Add whole discussion}

\begin{remark}[Opposite relations]
\label{remark:oppositerel}
Any relation~$R\colon A\to B$ induces an opposite relation (or transpose relation, reverse relation) \begin{equation}
    R^\intercal =\{\tup{b,a}\in B\times A\colon \tup{a,b}\in R\} \subseteq B\times A.
\end{equation}
Note tha~ $\left( R^\intercal\right)^\intercal = R$
\end{remark}

\begin{definition}[Category~$\Rel$]
    The category~$\Rel$ is defined by:
    \begin{enumerate}
    \item \emph{Objects}: The objects of this category are all sets.
    \item \emph{Morphisms}: The morphisms between any pair of sets~$X, Y$
    are relations~$R\subseteq X\times Y$.
    \item \emph{Identity morphism}: The identity morphism for the set~$X$
    is the diagonal morphism~$\delta_X \colon X\to X\times X$.
    \item \emph{Composition of morphisms}: The composition of two relations (morphisms)~$R \colon X\to Y$,~$S\colon Y\to Z$ is given by
    \begin{equation}
    R \then S \coloneqq \{\tup{x,z} \equiv  \exists y \in Y, \ \left(\tup{x,y} \in R\right) \wedge \left(\tup{y,z} \in S\right)\}.
    \end{equation}
    \end{enumerate}
\end{definition}
\begin{remark}
The opposite category (\cref{def:oppositecat}) of~$\Rel$ has the same objects of~$\Rel$ and as morphisms its opposite relations (\cref{remark:oppositerel}).
\end{remark}

\begin{lemma}
    The category~$\Cat{Set}$ is a subcategory (\cref{def:subcategory}) of~$\Cat{Rel}$.
\end{lemma}
\begin{proof}
	We need to prove the conditions presented in \cref{def:subcategory}.
	\begin{enumerate}[(i)]
	\item If a morphism of~$\Rel$ $f \colon X\to Y$ is in~$\Set$, then so are the objects~$X$ and~$Y$. Both~$\Rel$ and~$\Set$ have sets as objects. If a morphism~$f\colon X\to Y$ of~$\Rel$ is in~$\Set$, then a relation~$F\subseteq X\times Y$ between set~$X$ and set~$Y$ exists. This relation can be expressed in~$\Set$ as~$f\colon X\to Y$ and hence the objects~$X$ and~$Y$ exist.
	\item Two morphisms~$f\colon X\to Y$ and~$g\colon Y\to Z$ in~$\Rel$ are relations~$F\subseteq X\times Y$,~$G\subseteq Y\times Z$. If they are in~$\Set$, they can be written as functions~$f\colon X\to Y$ and~$g\colon Y\to Z$ and their composition is in~$\Set$ as well. 
	\item If an object of~$\Rel$ $X$ is in~$\Set$, then so is the identity morphism~$\text{Id}_X$. This is true: for every object~$X$ of~$\Set$ there exists the identity morphism~$\text{Id}_X\colon X\to X$. The diagonal morphism can be expressed as~$\text{Id}_X\colon X\to X$.
	\end{enumerate}
\end{proof}
