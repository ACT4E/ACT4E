% !TEX root = ../CategoricalCoDesign.tex

\section{Thinking about how things connect to each other}
Currency categories illustrated how one can use category theory to think about things transforming into each other. In this section, we want to think about how things connect to each other.
\subsection{Trekking}
\label{sec:trekking}
Consider a geographical region whose locations are expressed through coordinates $(x,y)\in \mathbb{R}^2$, \text{e.g.} as given by a map of that region. Furthermore, consider a function $\mathsf{alt}: \mathbb{R}^2 \longrightarrow \mathbb{R}_{\geq 0}$ which, for a known location, returns its altitude (for simplicity, we assume that alt is a smooth function). 
%\begin{equation}
% \mathsf{alt}: \left \{ 
%    \begin{tabular}{ccc}
%    $\mathbb{R}^2$ & $\longrightarrow$ & $\mathbb{R}_{\geq 0}$ \\
%    $(x,y)$ & $\mapsto$ &  $a.$ \\
%    \end{tabular}
%\right .
%\end{equation}

We can think about this situation using a category, call it $\mathbf{trek}$, where objects are geographical locations $\tup{x,y}\in \mathbb{R}^2$ and morphisms are continuous paths between them. The identity morphism for each location consists of the trivial path (i.e. not moving), and composition is given by concatenation of paths. 
\JL{We need to be more precise about what ``continuous path'' means here! The typical mathematical definition of paths from topology is as a function of a (``time'') parameter, and leads to a well-known situation where concatenation is not an associative operation on the nose... and/or there is also issue that there are crazy kinds of continuous paths, such as space-filling curves... perhaps this example can be modified a bit to capture the basic idea, but avoid the math issues...}

Suppose that a human can only traverse trails which have a maximum inclination of $\alpha>0$ when going uphill and $\beta>0$ when going downhill. 
We can now think of the aforementioned human, wanting to go from a location $\tup{x,y}$ to a location $\tup{v,w}$. Finding a path consists of finding at least a morphism in $\Hom_\mathbf{trek}(\tup{x,y},\tup{v,w})$ satisfying the condition on the maximum inclinations $\alpha$ and $\beta$.

\todo{Eventually, make an example by taking altimetry of a famous zone, and compute reachability paths from certain points.}


Using the terminology from Section [INSERT REF], we can see that $\mathbf{trek}$ is the free category on a graph with vertices given by geographical locations $\tup{x,y}\in \mathbb{R}^2$ and arrows given by paths between them. In particular, a valid path $p\colon \tup{x,y}\mapsto \tup{v,w}$ for the human to be able to reach a destination, has not to exceed the maximum inclination $\alpha$ when going up and the maximum inclination $\beta$ when going down.


\subsection{Mobility}

For a specific mode of transportation, say a car, we can define a graph~$G_\mathrm{c}=\tup{V_\mathrm{c},A_\mathrm{c},s_\mathrm{c},t_\mathrm{c}}$, where~$V_\mathrm{c}$ represents geographical locations which the car can reach and~$A_\mathrm{c}$ represents the paths it can take (e.g. roads). Similarly, we consider a graph~$G_\mathrm{s}=(V_\mathrm{s},A_\mathrm{s},s_\mathrm{s},t_\mathrm{s})$, representing the subway system of a city, with stations~$V_\mathrm{s}$ and subway lines going through paths~$A_\mathrm{s}$, and a graph $G_\mathrm{b}=\tup{V_\mathrm{b},A_\mathrm{b},s_\mathrm{b},t_\mathrm{b}}$, representing onboarding and offboarding at airports. In the following, we want to express intermodality: the phenomenon that someone might travel to a certain intermediate location in a car and then take the subway to reach their final destination. 

By considering the graph~$G=(V,A,s,t)$, with~$V=V_\mathrm{c}\cup V_\mathrm{s}$ and~$A=A_\mathrm{c}\cup A_\mathrm{s}$, we obtain the desired intermodality graph. Graph~$G$ can be seen as a new category, with objects~$V$ and morphisms~$A$.
\begin{example}
Consider the car category, describing your road trip in California, with
\begin{equation*}
    V_\mathrm{c}=\{\mathsf{SFO}_\mathrm{c},\mathsf{S.}\ \mathsf{Mateo},\mathsf{Half} \ \mathsf{Moon}\ \mathsf{Bay},\mathsf{SBP}_\mathrm{c},\mathsf{Lake}\ \mathsf{Balboa},\mathsf{LAX}_\mathrm{c}\},
\end{equation*}
and arrows as in~\cref{fig:carcat}. The nodes represent typical touristic road trip checkpoints in California and the arrows famous highways connecting them.

\begin{figure}[h!]
\begin{center}
\includesag{30_carcategory}
\end{center}
\caption{The car category. \label{fig:carcat}}
\end{figure}

Furthermore, consider the flight category with $V_\mathrm{f}=\{\mathsf{SFO}_\mathrm{f},\mathsf{SJC},\mathsf{SBP}_\mathrm{f},\mathsf{LAX}_\mathrm{f}\}$ and arrows as in~\cref{fig:flight}. The nodes represent airports in california and the arrows represent connections, offerend by specific flight companies.

\begin{figure}[h!]
\begin{center}
\includesag{30_flight}
\end{center}
\caption{The flight category. \label{fig:flight}}
\end{figure}

We then consider the boarding category, with nodes
\begin{equation*}
    V_\mathrm{b}=\{\mathsf{SFO}_\mathrm{f},\mathsf{SFO}_\mathrm{c},\mathsf{SBP}_\mathrm{f},\mathsf{SBP}_\mathrm{c},\mathsf{LAX}_\mathrm{f},\mathsf{LAX}_\mathrm{c}\}
\end{equation*}
and arrows as in~\cref{fig:boarding}. Nodes represent airports and airport parkings, and arrows represent the onboarding and offboarding paths one has to walk to get from the parkings to the airport and vice-versa.

\begin{figure}[h!]
\begin{center}
\includesag{30_boarding}
\end{center}
\caption{The boarding category. \label{fig:boarding}}
\end{figure}

The combination of the three, which we call the \emph{intermodal graph}, can be represented as a graph, with \textcolor{red}{red} arrows for the car network, \textcolor{blue}{blue} arrows for the flight network, \textcolor{green}{green} arrows for the boarding network, and black dashed arrows for intermodal morphisms, arising from composition of morphisms involving multiple modes (\cref{fig:intermodal}). Imagine that you are in the parking of $\mathsf{LAX}$ airport and you want to reach $\mathsf{S. Mateo}$. From there, you will e.g. onboard to a $\mathsf{United}$ flight to $\mathsf{SFO}_\mathrm{f}$, will then offboard reaching the parking lot $\mathsf{SFO}_\mathrm{c}$, and drive on highway $\mathsf{US-101}$ reaching $\mathsf{S. Mateo}$. This is intermodality.

\begin{figure}[h!]
\begin{center}
\includesag{30_intermodal}
\end{center}
\caption{Intermodal graph. The dashed arrows represent intermodal morphisms, and we depict just one of them for simplicity. \label{fig:intermodal}}
\end{figure}
\end{example}

The intermodal network category is the free category on the graph illustrated in \cref{fig:intermodal}.

\GZ{Maybe not needed? If so, rewrite}
\JL{I think this is nice to include! But then maybe for clarity we would want to actually remove the dashed arrow in Figure 18, since that is not part of the graph we are thinking to take the free category of. Technically we could also leave it in, but that seems to confuse the situation, no?}
\begin{comment}
\begin{shaded}
\begin{definition}[Union of categories]
Given two categories $\CatC,\CatD$, one can create the \emph{union} $\Cat{E}$ of the two, which is composed of:
\begin{compactenum}
\item \emph{Objects:} $\Ob_\Cat{E}=\Ob_\CatC \cup \Ob_\CatD$.
\item \emph{Morphisms:} A morphism $f$ is given by considering the followng. If $f\in \Hom_\CatC(X,Y)$, then $f\in \Hom_\Cat{E}(X,Y)$. If $f\in \Hom_\CatD(X,Y)$, then $f\in \Hom_\Cat{E}(X,Y)$.
\item \emph{Identity morphism:} For any morphism in $\Cat{E}$, the identity morphism remais the same as in the original category.
\item \emph{Composition operation}: The composition of morphisms remains the same.
\end{compactenum}
\end{definition}
\end{shaded}
\end{comment}

%\subsection{Generating categories from graphs}
%What we sketched is the previous sections has deeper roots. In the following, we will introduce the concept of \emph{free categories on graphs}.
%
%\begin{shaded}
%\begin{definition}[Free category on a graph]
%Consider any graph $G=(V,A,s,t)$. We can define a category $\mathbf{Free}(G)$, called the \emph{free category on $G$}. Its objects are the vertices $V$, and given vertices $a\in V$ and $b\in V$, the morphisms $\mathbf{Free}(G)(a,b)$ are the paths from $a$ to $b$. 
%%A path is a sequence of ``consecutive'' edges, \text{i.e.} the source of a subsequent edge is equal to the target of its predecessor. We also formally allow for ``empty paths'', \text{i.e.} a sequence of "zero"-many edges which starts and ends at the same vertex. 
%The identity morphism $id_a$ is defined to be the empty path starting and ending at a vertex $a \in V$, and composition of morphisms is given by concatenation of paths.
%\end{definition}
%\end{shaded}
%
%
%With these two new definitions, we can see that $\mathbf{trek}$ is the free category on a graph with vertices given by geographical locations $\tup{x,y}\in \mathbb{R}^2$ and arrows given by paths between them. In particular, a valid path $p\colon \tup{x,y}\mapsto \tup{v,w}$ for the human to be able to reach a destination, has not to exceed the maximum inclination $\alpha$ when going up and the maximum inclination $\beta$ when going down.

