% !TEX root = ../CategoricalCoDesign.tex
\section{Thinking about how things connect to each other}
\subsection{Trekking}


\subsection{Generating categories from graphs}

\begin{definition}[Graph]
A \emph{graph} $G=\tup{V,A,s,t}$ consists of a set of vertices $V$, a set of arrows $A$, and two functions $s,t\colon A\to V$, called the \emph{source} and \emph{target} functions, respectively. Given $a\in A$ with $s(a)=v$ and $t(a)=w$, one says that $a$ is an arrow from $v$ to $w$. A path in $G$ is a sequence of arrows such that the target of one arrow is the source of the next. Note that a path of length 1 (one single arrow) from a vertex to itself always exists. We call it \emph{trivial}.
\end{definition}

\begin{shaded}
\begin{definition}[Free category on a graph]
Consider any graph $G=(V,A,s,t)$. One can define the category $\mathbf{Free}(G)$, called the \emph{free category on $G$}. Its objects are the vertices $V$, and its morphisms from $c\in V$ to $d\in V$ are the paths from $c$ to $d$. The identity morphism of an object $c\in V$ is the trivial path at $c$. Composition of morphisms is given by paths' concatenation.
\end{definition}
\end{shaded}


\subsection{Mobility}

\AC{Ok now you have Free category of a graph; inevitable to sketch
the application of mobility graph.}

\AC{(What operation on categories is the adjoining of two different mobility graphs?)}

\subsection{The Set category}


\begin{shaded}
\begin{definition}[Category $\Set$]
    The category $\Set$ is defined by:
    \begin{compactenum}
    \item \emph{Objects}: The objects of this category are all sets.
    \item \emph{Morphisms}: The morphisms between any pair of sets $X, Y$
    are maps from $X$ to $Y$.
    \item \emph{Identity morphism}: The identity morphism for the set $X$
    is the identity function $\text{Id}_X$.
    \item \emph{Composition operation}: The composition operation is function
    composition.
    \end{compactenum}
\end{definition}
\end{shaded}