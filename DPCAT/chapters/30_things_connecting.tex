% !TEX root = ../CategoricalCoDesign.tex
\section{Thinking about how things connect to each other}
The $\mathbf{Curr}$ category was a good example of how one can use category theory to think about things transforming into each other. In this section, we want to think about how things connect to each other.
\subsection{Trekking}
\label{sec:trekking}
Consider geographical locations expressed through coordinates $\tup{x,y}\in \mathbb{R}^2$. Furthermore, consider a map $\mathsf{alt}$ which, for a known location, returns its altitude (assuming smooth terrain):
\begin{equation}
    \begin{aligned}
    \mathsf{alt}\colon \mathbb{R}^2&\to \mathbb{R}_{\geq 0}\\
    \tup{x,y}&\to a.
    \end{aligned}
\end{equation}
Consider a human which can traverse trails with a maximum inclination of $\alpha>0$ when going up and $\beta>0$ when going down a trail. We can think about this as a category (call it $\mathbf{trek}$), where objects are geographical locations $\tup{x,y}\in \mathbb{R}^2$ and morphisms are continuous paths between them. The identity morphism for each location consists of the trivial path (i.e. not moving), and composition is given by concatenation of paths. We can now think of the aforementioned human, willing to go from a location $\tup{x,y}$ to a location $\tup{v,w}$. Finding a path consists of finding at least a morphism in $\Hom_\mathbf{trek}(\tup{x,y},\tup{v,w})$ satisfying the condition on the maximum inclination $\alpha$.


\todo{Eventually, make an example by taking altimetry of a famous zone, and compute reachability paths from certain points.}


\subsection{Mobility}
Consider now the example of mobility. For a specific mode of transportation, say a car, we can define a graph $G_\mathrm{c}=(V_\mathrm{c},A_\mathrm{c},s_\mathrm{c},t_\mathrm{c})$, where $V$ represents geographical locations that the car can reach in a city and $A_\mathrm{c}$ represents the paths it can take (e.g. roads). Similarly, consider a graph $G_\mathrm{s}=(V_\mathrm{s},A_\mathrm{s},s_\mathrm{s},t_\mathrm{s})$, representing the subway system of a city, with stations $V_\mathrm{s}$ and lines going through paths $A_\mathrm{s}$. In the following, we want to express intermodality, e.g. someone reaching a certain intermediate location on a car and then deciding to take the subway to reach the final destination. By considering the graph $G=(V,A,s,t)$, with $V=V_\mathrm{c}\cup V_\mathrm{s}$ and $A=A_\mathrm{c}\cup A_\mathrm{s}$, one obtains the desired intermodality. The arising graph can be seen as a new category, with objects $V$ and morphisms $A$.
\begin{example}
Consider the car category, describing your road trip in California, with $V_\mathrm{c}=\{\mathsf{SFO},\mathsf{S.}\ \mathsf{Mateo},\mathsf{Half-Moon}\ \mathsf{Bay},\mathsf{SBP},\mathsf{Lake}\ \mathsf{Balboa},\mathsf{LAX}\}$ and arrows as in~\cref{fig:carcat}.

\begin{figure}[h!]
\begin{center}
\includesag{30_carcategory}
\end{center}
\caption{The car category. \label{fig:carcat}}
\end{figure}

Furthermore, consider the flight category with $V_\mathrm{f}=\{\mathsf{SFO},\mathsf{SJC},\mathsf{SBP},\mathsf{LAX}\}$ and arrows as in~\cref{fig:subcat}.


\begin{figure}[h!]
\begin{center}
\includesag{30_flight}
\end{center}
\caption{Flight category. \label{fig:subcat}}
\end{figure}
The union of the two, which we call the \emph{intermodal graph}, can be represented as a graph, with \textcolor{red}{red} arrows for the car network, \textcolor{blue}{blue} arrows for the flight network (\cref{fig:intermodal}), and black dashed arrows for intermodal morphisms (arising from composition of morphisms, involving more modes).

\begin{figure}[h!]
\begin{center}
\includesag{30_intermodal}
\end{center}
\caption{Intermodal graph. The dashed arrows represent intermodal morphisms, and we depict just one of them for simplicity. \label{fig:intermodal}}
\end{figure}
\end{example}


\noindent This concept is formalized in category theory with the union of categories
\begin{shaded}
\begin{definition}[Union of categories]
Given two categories $\CatC,\CatD$, one can create the \emph{union} $\Cat{E}$ of the two, which is composed of:
\begin{compactenum}
\item \emph{Objects:} $\Ob_\Cat{E}=\Ob_\CatC \cup \Ob_\CatD$.
\item \emph{Morphisms:} A morphism $f$ is given by considering the followng. If $f\in \Hom_\CatC(X,Y)$, then $f\in \Hom_\Cat{E}(X,Y)$. If $f\in \Hom_\CatD(X,Y)$, then $f\in \Hom_\Cat{E}(X,Y)$.
\item \emph{Identity morphism:} For any morphism in $\Cat{E}$, the identity morphism remais the same as in the original category.
\item \emph{Composition operation}: The composition of morphisms remains the same.
\end{compactenum}
\end{definition}
\end{shaded}

\subsection{Generating categories from graphs}
What we sketched is the previous sections has deeper roots. In the following, we will introduce the concept of \emph{free categories on graphs}.
\begin{definition}[Graph]
A \emph{graph} $G=\tup{V,A,s,t}$ consists of a set of vertices $V$, a set of arrows $A$, and two functions $s,t\colon A\to V$, called the \emph{source} and \emph{target} functions, respectively. Given $a\in A$ with $s(a)=v$ and $t(a)=w$, one says that $a$ is an arrow from $v$ to $w$. A path in $G$ is a sequence of arrows such that the target of one arrow is the source of the next. Note that a path of length 1 (one single arrow) from a vertex to itself always exists. We call it \emph{trivial}.
\end{definition}

\begin{shaded}
\begin{definition}[Free category on a graph]
Consider any graph $G=(V,A,s,t)$. One can define the category $\mathbf{Free}(G)$, called the \emph{free category on $G$}. Its objects are the vertices $V$, and its morphisms from $c\in V$ to $d\in V$ are the paths from $c$ to $d$. The identity morphism of an object $c\in V$ is the trivial path at $c$. Composition of morphisms is given by paths' concatenation.
\end{definition}
\end{shaded}


With these two new definitions, we can see that $\mathbf{trek}$ is the free category on a graph with vertices given by geographical locations $\tup{x,y}\in \mathbb{R}^2$ and arrows given by paths between them. Particularly, a valid path $p\colon \tup{x,y}\mapsto \tup{v,w}$ for the human to be able to reach a destination, has not to exceed the maximum inclination $\alpha$ when going up and the maximum inclination $\beta$ when going down.

\subsection{The Set category}


\begin{shaded}
\begin{definition}[Category $\Set$]
    The category $\Set$ is defined by:
    \begin{compactenum}
    \item \emph{Objects}: The objects of this category are all sets.
    \item \emph{Morphisms}: The morphisms between any pair of sets $X, Y$
    are maps from $X$ to $Y$.
    \item \emph{Identity morphism}: The identity morphism for the set $X$
    is the identity function $\text{Id}_X$.
    \item \emph{Composition operation}: The composition operation is function
    composition.
    \end{compactenum}
\end{definition}
\end{shaded}

Note that $\Set$ is precisely defined by the four points above. However, there are categories where objects are sets which differ from $\Set$ in the morphisms, as outlined in the following example.
\begin{example}
\label{ex:hasseinclusion}
Given a set $X=\{a,b,c\}$, consider its power set $\powerset(X)$. Define sets as the objects of this new category and define the morphisms to be inclusions (\cref{fig:powersetcat}).
\begin{figure}[h!]
\begin{center}
\includesag{40_dpcatfig_power}
\end{center}
\caption{Power set as a category. \label{fig:powersetcat}}
\end{figure}
The identity morphism of each set is the inclusion with itself (every set is a subset of itself). Composition is given by composition of inclusions, i.e. if $X\subseteq Y \subseteq Z$, then $X\subseteq Z$. This is a category which has sets as objects, but which is different from $\Set$.
\end{example}