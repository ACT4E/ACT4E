% !TEX root = ../CategoricalCoDesign.tex
\subsection{Thinking about specifity of constraints}

\JL{Maybe we could make this its own chapter about ``attributes and constraints'' and cook up some set-based examples, and then use those as motivation to talk about the category of sets and the notion of subcategory... perhaps some examples related to David Spivak's database framework could be useful here.... below is a first attempt/draft of this. }


Suppose we want to buy an electric motor for a robot that we are building, and for this we consult a catalogue for electric motors. The catalogue might be organized as a large table, where on the left-hand side there is a column listing all available motors (identified with a model name), and the remaining columns correspond to different attributes that each of the models of motor might have, such as the name of the manufacturer, the size dimensions, the weight, the maximum power, the price, etc.

As a simplified illustration:
\begin{table}[h]
    \centering
    \begin{tabular}{c|c|c|c|c|c}
         Motor ID & Manufacturer& Size & Weight & Max Power & Price \\
         \hline
         $\mathsf{Model1}$&Company $\mathsf{B}$ & 2 x 3 x 4& 10 & &\unit[259]{\$}\\
         $\mathsf{Model2}$&Company $\mathsf{A}$ &2 x 3 x 4& 20 & &\unit[109]{\$}\\
         $\mathsf{Model3}$&Company $\mathsf{B}$ &2 x 3 x 4& 5 & &\unit[124]{\$}\\
         $\mathsf{Model4}$&Company $\mathsf{C}$ &2 x 3 x 4& 30 & &\unit[399]{\$}\\
         $\mathsf{Model5}$&Company $\mathsf{A}$ &2 x 3 x 4& 45 & &\unit[245]{\$}  \\
        $\mathsf{Model6}$&Company $\mathsf{D}$ & 2 x 3 x 4& 20 & &\unit[89]{\$}\\
        $\mathsf{Model7}$&Company $\mathsf{B}$ & 2 x 3 x 4& 15 &&\unit[130]{\$}
    \end{tabular}
    \caption{A simplified catalogue of motors.
    }
    \label{tab:currencycompanies}
\end{table}

One useful way to think of tables of data is in terms of sets and functions. In this example, we can consider the set
$$
M := \{ \mathsf{Model1}, \mathsf{Model2}, \mathsf{Model3}, \mathsf{Model4}, \mathsf{Model5}, \mathsf{Model6} \}
$$ 
of models of motors, as well as the set $C := \{ \A, \B, \C, \D \}$ of manufacturing companies, the set $S$ of possible motor sizes, the set $W$ of possible weights, the set $J$ of possible maximal powers, and the set $P$ of possible prices. Each attribute of a motor may be thought of as a function from the set $M$ to set of possible values for the given attribute. For example, there is a function from $\mathsf{Manufacturer}: M \longrightarrow C$ which maps each model to the corresponding company that manufactures it. So, according to Table \ref{tab:currencycompanies}, we have \text{e.g.} $\mathsf{Manufacturer}(\mathsf{Model1}) = \B$, and $\mathsf{Manufacturer}(\mathsf{Model2}) = \A$, etc.

Now consider that the catalogue of motors might not have seven entries, as in our simple table above, but has in fact thousands of entries, and is implemented digitally as a database. 

[TO BE CONTINUED]

\

\

\
\begin{shaded}
\begin{definition}[Subcategory]
\label{def:subcategory}
	Given a category $\Cat{A}$, a \emph{subcategory} $\Cat{B}$ consists of a subcollection of the collection of objects and morphisms of $\Cat{A}$ such that:
	\begin{enumerate}[(i)]
	\item If a morphism $f \colon x\to y$ is in $\Cat{B}$, then so are the objects $x$ and $y$.
	\item If the morphisms $f\colon x\to y$ and $g\colon y\to z$ are in $\Cat{B}$, then so is their composite $f\then g\colon x\to z$.
	\item If $x$ is in $\Cat{B}$, then so is the identity morphism $\text{Id}_x$.
	\end{enumerate}
\end{definition}
\end{shaded}

\JL{I would say the lemma below is obsolete (because of the changes in the currency example), and can be deleted}
\begin{lemma}
    $\mathbf{Curr}$ is a subcategory of $\Set$.
\end{lemma}

\begin{proof}
    We need to check the conditions presented in \cref{def:subcategory}.
    \begin{enumerate}[(i)]
        \item If a morphism of $\Set$ $f\colon X\to Y$ is in $\mathbf{Curr}$, it means automatically that both $X$ and $Y$ are singletons, containing currencies, and that $f$ can be represented as a currency exchanger.
        \item If morphisms of $\Set$ $f\colon X\to Y$ and $g\colon Y \to Z$ are in $\mathbf{Curr}$, then from (i) we know that $X,Y,Z$ are as well, that $f,g$ are currency exchangers, and that we can write their composition.
        \item If $X$ is in $\mathbf{Curr}$, then the identity morphism $\text{Id}_X$ in $\Set$ can be written as an identity currency exchanger.
    \end{enumerate}
\end{proof}