% !TEX root = ../CategoricalCoDesign.tex
\subsection{Thinking about specifity of constraints}
\begin{shaded}
\begin{definition}[Subcategory]
\label{def:subcategory}
	Given a category $\Cat{A}$, a \emph{subcategory} $\Cat{B}$ consists of a subcollection of the collection of objects and morphisms of $\Cat{A}$ such that:
	\begin{enumerate}[(i)]
	\item If a morphism $f \colon x\to y$ is in $\Cat{B}$, then so are the objects $x$ and $y$.
	\item If the morphisms $f\colon x\to y$ and $g\colon y\to z$ are in $\Cat{B}$, then so is their composite $f\then g\colon x\to z$.
	\item If $x$ is in $\Cat{B}$, then so is the identity morphism $\text{Id}_x$.
	\end{enumerate}
\end{definition}
\end{shaded}

\begin{lemma}
    $\mathbf{Curr}$ is a subcategory of $\Set$.
\end{lemma}

\begin{proof}
    We need to check the conditions presented in \cref{def:subcategory}.
    \begin{enumerate}[(i)]
        \item If a morphism of $\Set$ $f\colon X\to Y$ is in $\mathbf{Curr}$, it means automatically that both $X$ and $Y$ are singletons, containing currencies, and that $f$ can be represented as a currency exchanger.
        \item If morphisms of $\Set$ $f\colon X\to Y$ and $g\colon Y \to Z$ are in $\mathbf{Curr}$, then from (i) we know that $X,Y,Z$ are as well, that $f,g$ are currency exchangers, and that we can write their composition.
        \item If $X$ is in $\mathbf{Curr}$, then the identity morphism $\text{Id}_X$ in $\Set$ can be written as an identity currency exchanger.
    \end{enumerate}
\end{proof}