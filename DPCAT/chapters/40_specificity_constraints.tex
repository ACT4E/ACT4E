% !TEX root = ../CategoricalCoDesign.tex
\section{Thinking about attributes, sameness, and constraints}

\JL{Maybe we could make this its own chapter about ``attributes and constraints'' and cook up some set-based examples, and then use those as motivation to talk about the category of sets and the notion of subcategory... perhaps some examples related to David Spivak's database framework could be useful here.... below is a first attempt/draft of this. }

\subsection{Sets, functions, databases}

Suppose we want to buy an electric motor for a robot that we are building, and for this we consult a catalogue of electric motors. The catalogue might be organized as a large table, where on the left-hand side there is a column listing all available motors (identified with a model name), and the remaining columns correspond to different attributes that each of the models of motor might have, such as the name of the company that manufactures the motor, the size dimensions, the weight, the maximum power, the price, etc.

As a simplified illustration:
\begin{table}[h]
    \centering
    \begin{tabular}{c|c|c|c|c|c}
         Motor ID & Company& Size & Weight & Max Power & Price \\
         \hline
         $\mathsf{Model1}$&Company $\mathsf{B}$ & 2 x 3 x 4& 10 & &\unit[259]{\$}\\
         $\mathsf{Model2}$&Company $\mathsf{A}$ &2 x 3 x 4& 20 & &\unit[109]{\$}\\
         $\mathsf{Model3}$&Company $\mathsf{B}$ &2 x 3 x 4& 5 & &\unit[124]{\$}\\
         $\mathsf{Model4}$&Company $\mathsf{C}$ &2 x 3 x 4& 30 & &\unit[399]{\$}\\
         $\mathsf{Model5}$&Company $\mathsf{A}$ &2 x 3 x 4& 45 & &\unit[245]{\$}  \\
        $\mathsf{Model6}$&Company $\mathsf{D}$ & 2 x 3 x 4& 20 & &\unit[89]{\$}\\
        $\mathsf{Model7}$&Company $\mathsf{B}$ & 2 x 3 x 4& 15 &&\unit[130]{\$}
    \end{tabular}
    \caption{A simplified catalogue of motors.
    }
    \label{tab:currencycompanies}
\end{table}

One useful way to think of tables of data is in terms of sets and functions. In this example, we can consider the set
$$
M := \{ \mathsf{Model1}, \mathsf{Model2}, \mathsf{Model3}, \mathsf{Model4}, \mathsf{Model5}, \mathsf{Model6} \}
$$
of models of motors, as well as t he set $C := \{ \A, \B, \C, \D \}$ of manufacturing companies, the set $S$ of possible motor sizes, the set $W$ of possible weights, the set $J$ of possible maximal powers, and the set $P$ of possible prices. Each attribute of a motor may be thought of as a function from the set $M$ to set of possible values for the given attribute. For example, there is a function $\mathsf{Company}: M \longrightarrow C$ which maps each model to the corresponding company that manufactures it. So, according to Table \ref{tab:currencycompanies}, we have \text{e.g.} $\mathsf{Company}(\mathsf{Model1}) = \B$, and $\mathsf{Company}(\mathsf{Model2}) = \A$, etc.

Note that in ``real life'', the catalogue of motors might not have seven entries, as in our simple table above, but has in fact hundreds of entries, and is implemented digitally as a database, \text{i.e.} a collection of interrelated tables. In this case, we will want to be able to search and filter the data based on various criteria. Many natural operations on tables and databases may be described simply in terms of operations with functions. We will use this setting as a way to introduce compositional aspects of working with sets and functions, and a preview of how this might be useful for thinking, in particular, about databases.

Sticking with our simple example table above, suppose, for instance, that we want to consider only motors from Company $\mathsf{B}$. In terms of the function
\begin{equation*}
\mathsf{Company}\colon M \longrightarrow C
\end{equation*}
this corresponds to considering the preimage $\mathsf{Company}^{-1}(\{ B \}) = \{ \mathsf{Model1}, \mathsf{Model3}, \mathsf{Model7} \}$, which is a subset of the set $M$. Or, we may want to consider only motors which cost between 100 and 200 dollars. In terms of the obvious function
\begin{equation*}
\mathsf{Price}\colon M \longrightarrow P,
\end{equation*}
this means we wish to restrict ourselves to the preimage
\begin{equation*}
\mathsf{Price}^{-1}(\{ 109, 124, 130\}) = \{ \mathsf{Model2}, \mathsf{Model3}, \mathsf{Model7} \} \subseteq M.
\end{equation*}

Now suppose we wish to add a column to our table for ``volume'', because we may want to only consider motors that have, at most, a certain volume. For this we define a set $V$ of possible volumes (let's take $V = \mathbb{R}_{\geq 0}$, the non-negative real numbers), and define a function
\begin{equation*}
\mathsf{Multiply}\colon S \longrightarrow V, \ \tup{l, w, h} \longmapsto l \cdot w \cdot h,
\end{equation*}
which maps any size of motor to its corresponding volume by multiplying together the given numbers for length, width, and heigth.  Now we can compose this function with the function
\begin{equation*}
\mathsf{Size}\colon M \longrightarrow S
\end{equation*}
to obtain a function
\begin{equation*}
\mathsf{Volume}\colon M \longrightarrow V,
\end{equation*}
which defines a new column in our table. The composition of functions is usually written as $\mathsf{Volume} = \mathsf{Multiply} \circ \mathsf{Size}$, however we stick to our convention of writing $\mathsf{Volume} = \mathsf{Size} \then \mathsf{Multiply}$. Schematically, we can represent what we did as a diagram, as in Figure \cref{fig:diagram_functions}.

%We call such a diagram \textbf{commutative}, because the composition $\mathsf{Size} \then \mathsf{Multiply}$ is equal to the function $\mathsf{Volume}$ (in fact, we defined $\textsf{Volume}$ this way).


\begin{figure}[h!]
\begin{center}
\includesag{40_dpcatfig_data_comm_diag}
\end{center}
\caption{A diagram of functions \label{fig:diagram_functions}}
\end{figure}

We can interpret the arrows in this diagrams as being part of a category, one where $M$, $S$, and $V$ are among the objects, and where the functions $\mathsf{Size}$, $\mathsf{Multiply}$ and $\mathsf{Volume}$ are morphisms. But we probably want to consider the other sets associated with our database as also part of this category, and the other functions which we defined so far, too. One idea might be to just include all the sets and functions that we've defined so far, as well as all possible compositions of those functions, and obtain a category (maybe call it $\mathsf{Database}$?) in a way that is similar to how one can build a category from a graph (Section [REF]). This would be an option. However, we may want soon to add new sets and functions to our database framework, or think about new kinds of functions between them that we hadn't considered before. And we might not want to re-think each time precisely which category we are working with.

A helpful concept here is to think of our specific sets and functions as living in a very (very) large category which contains all possible sets as its objects and all possible functions as its morphisms. This category is know as the category of sets, and it is an important protagonist in category theory. We'll denote it by $\mathsf{Set}$. It is a short exercise to check that the following does indeed define a category. 

\begin{shaded}
\begin{definition}[Category of sets]
    The category $\Set$ is defined by:
    \begin{compactenum}
    \item \emph{Objects}: all sets.
    \item \emph{Morphisms}: given sets $X$ and  $Y$, the homset $\Set(X,Y)$ is the set of all functions from $X$ to $Y$.
    \item \emph{Identity morphism}: given a set $X$, its identity morphism $\text{1}_X$ is
    is the identity function $X \rightarrow X, \ \text{1}_X(x) = x$.
    \item \emph{Composition operation}: the composition operation is the usual composition of functions.
    \end{compactenum}
\end{definition}
\end{shaded}

%\begin{exercise}
%Check that $\Set$, as specified above, does in fact define a category.
%\end{exercise}

\subsection{Sameness in category theory}

One nice thing about the category of sets is that we are all used to working with sets and functions. And many concepts that are familiar in the setting of sets and functions can actually be reformulated in a way which makes sense for lots of other categories, if not for all categories. It can be fun, and insightful, to see known definitions transformed into ``category theory language''. For example: the notion of a bijective function is a familiar concept. There are least two ways of saying what it means for a function $f : X \rightarrow Y$ of sets to be bijective:
\begin{itemize}
\item[Definition 1:] ``$f:X \rightarrow Y$ is bijective if it is both injective and surjective''\footnote{Recall: a function $f:X \rightarrow Y$ is injective if... and surjective if... }
\item[Definition 2:] ``$f: X \rightarrow Y$ is bijective if there exists a function $g: Y \rightarrow X$ such that $f \then g = id_X$ and $g \then f = id_Y$''. 
\end{itemize}

It is a short proof to show that the above two definitions are equivalent. It turns out that both versions have useful generalizations in category theory, and it also turns out that these generalizations do not always imply each other! So, two equivalent ways of saying that a function is bijective give rise to non-equivalent concepts in category theory. Definition 2 above is the more fundamental variant for category theory; it corresponds to the basic notion of an ``isomorphism''. 

\begin{shaded}
\begin{definition}[Isomorphism]
Let $\CatC$ be a category, let $x \in \CatC$ and $y \in \CatC$ be objects, and let $f: x \rightarrow y$ be a morphism. We say that $f$ is an \textbf{isomorphism} if there exists a morphism $g: y \rightarrow x$ such that $f \then g = id_x$ and $g \then f = id_y$. 
\end{definition}
\end{shaded}

Note that this definition is basically identical with Definition 2 above for bijective functions. This generalization to any category $\CatC$ is completely straightforward, because all the ingredients for the formulation of Definition 2 make clear sense in any category; namely, all we use are morphisms, their composition, the notion of identity morphisms, and the notion of equality of morphisms (for equations such as ``$f \then g = id_x$'').
 
\


\subsection{Constraints I: subobjects}

\


\subsection{Constraints II: subcategories}




\
\begin{shaded}
\begin{definition}[Subcategory]
\label{def:subcategory}
	Given a category $\Cat{A}$, a \emph{subcategory} $\Cat{B}$ consists of a subcollection of the collection of objects and morphisms of $\Cat{A}$ such that:
	\begin{enumerate}[(i)]
	\item If a morphism $f \colon x\to y$ is in $\Cat{B}$, then so are the objects $x$ and $y$.
	\item If the morphisms $f\colon x\to y$ and $g\colon y\to z$ are in $\Cat{B}$, then so is their composite $f\then g\colon x\to z$.
	\item If $x$ is in $\Cat{B}$, then so is the identity morphism $\text{Id}_x$.
	\end{enumerate}
\end{definition}
\end{shaded}
