\section{Lifting SOMETHING}
\todo{add general formulas}
We round out our discussion of $\DP$ by introducing two formulae for transforming monotone maps in $\Poset$ into design problems in $\DP$. Each monotone map $f$ can be transformed into two design problems, called its \emph{companion} $\comp{f}$ and \emph{conjoint} $\conj{f}$. Many of the design problems that we have introduced can be realized as companions and conjoints of appropriate monotone maps.

\begin{definition}[Companion and conjoint]
\label{def:comp_conj}
Let $\cP $ and $\cQ $ be posets, and suppose that $f\colon\cP \toinPos \cQ $ is a monotone map. We define its \emph{companion} in $\DP$, denoted $\comp{f}\colon \F{\cP} \tickar \R{\cQ}$,
and its \emph{conjoint}, denoted $\conj{f}\colon \F{\cQ} \tickar \R{\cP}$ as
\begin{equation}
\comp{f}(\F{p}^*,\R{q})\coloneqq f(\F{p}) \leq_\cQ \R{q}
\qquad\text{and}\qquad
\conj{f}(\F{q}^*,\R{p})\coloneqq \F{q} \leq_\cQ f(\R{p}).
\end{equation}
\end{definition}

\begin{proposition}\label{prop:comp_conj}
Both the companion and conjoint constructions from \cref{def:comp_conj} are functorial: They preserve identities and composition.
\end{proposition}
\begin{proof}
We will show that the companion and conjoint are functors of the following forms:
\begin{equation}
\comp{(\cdot)}\colon\Poset\to\DP
\qquad\text{and}\qquad
\conj{(\cdot)}\colon\Poset\to\DP\op
\end{equation}
First, we see that they send the identity monotone map $f(p)=p$ to the unit $\Unit{\cP }$ for any poset $\cP $, because 
\begin{equation}
    \begin{aligned}
        \comp{\id}(\F{p_1}^*,\R{p_2})&= (\F{p_1} \leq_{\cP} \R{p_2})\\
        &=\conj{\id}(\F{p_1}^*,\R{p_2})
    \end{aligned}
\end{equation}

Now suppose that $f\colon  \cP \toinPos \cQ $ and $g\colon \cQ \toinPos \cR$ are given. We first show that $\conj{g}\then\conj{f}=\conj{f\then g}$.
For any $p\in P$ and $r\in R$, one has
\begin{equation}
\begin{aligned}
	\left(\conj{g}\then \conj{f}\right)(\F{r}^*,\R{p})
	&=\bigvee_{q\in Q} \conj{g}(\F{r}^*,\R{q})\wedge\conj{f}(\F{q}^*,\R{p})\\
	&=\bigvee_{q\in Q} (\F{r}\leq_R g(\R{q})) \wedge (\F{q}\leq_Q f(\R{p})) \\
	&= \F{r}\leq_R g(f(\R{p}))\\
    &=\left(\conj{f\then g}\right)(\F{r}^*,\R{p}).
\end{aligned}
\end{equation}
Similarly, we can prove that $\comp{f}\then \comp{g}=\comp{f\then g}$:
\begin{equation}
    \begin{aligned}
    \left(\comp{f}\then \comp{g}\right)(\F{p}^*,\R{r})&=\bigvee_{q\in Q} \comp{f}(\F{p}^*,\R{q})\wedge\comp{g}(\F{q}^*,\R{r})\\
    &=\bigvee_{q\in Q} (f(\F{p})\leq_Q \R{q})\wedge (g(\F{q})\leq_R \R{r})\\
    &=g(f(\F{p}))\leq_R \R{r}\\
    &=\left(\comp{f\then g}\right)(\F{p}^*,\R{r}).
    \end{aligned}
\end{equation}
\end{proof}


\begin{example}The identity design problem $\id_A\colon \F{A} \tickar \R{A}$ is the companion (and the conjoint) of the identity map $\id_A'\colon A \toinPos A$. This is easy to check as
\begin{equation}
    \begin{aligned}
    \comp{\id}_A'(\F{a_1}^*,\R{a_2})&=\id_A'(\F{a_1})\leq \R{a_2}\\
    &=\F{a_1}\leq \R{a_2}\\
    &=\id_A(\F{a_1}^*,\R{a_2}).
    \end{aligned}
\end{equation}
\end{example}

\begin{example}The coproduct injections $\iota_A, \iota_B$ for design problems are the companions of the coproduct injections for the disjoint union.
\end{example}

\begin{example}The product projections $\pi_A, \pi_B$ for design problems are the conjoint of the coproduct injections for the disjoint union.
\end{example}

\todo{Prima def generale e poi definizione per DP}
\todo{esempi vari}

\subsection{Interesting implications}
Consider a poset $A$, which can be thought of a map $f\colon 1\to A$. By taking the companion of $f$ one gets
\begin{equation}
\begin{aligned}
    \comp{f}\colon \F{1}&\tickar \R{A}\\
    \tup{\F{1},\R{a}}&\mapsto f(1)\leq \R{a}.
\end{aligned}
\end{equation}
By taking the conjoint, one gets
\begin{equation}
\begin{aligned}
    \conj{f}\colon \F{A}&\tickar \R{1}\\
    \tup{\F{a}^*,\R{1}}&\mapsto \F{a}\leq f(\R{1}).
\end{aligned}
\end{equation}
These two cases represent design problems with constant resources and functionalities.

