\section{Relationship between products}
\subsection{Biproduct, Product, and Coproduct of Design Problems}
\begin{example}
Just as Beau is about to connect the engine diagram to the larger design problem for the X103, his supervisor, Elly May, comes up behind him and catches a glance at his diagram. ``Beau, that's almost right: Jeb-XX~$\wedge$ Bob-Roc does indeed work as an approximation of a generic ``engine'' design problem. The problem is that the choice of engine is so important and expensive that it isn't up to the engineering team---it's up to the politicians! So take~$f \vee g$ out and stick the one reported in~\cref{fig:exbiproduct} instead.
\begin{figure}[h!]
\begin{center}
\includesag{50_rival_pol}
\end{center}
\caption{The choice of engine is up to the politicians: Biproduct. \label{fig:exbiproduct}}
\end{figure}

We'll come back and adjust the parameter to either the Jeb XX or the Bob-Roc after the politicians make their choice.''
\end{example}
\begin{definition}[Biproduct of design problems]
Given design problems~$f\colon \F{A} \tickar \R{B}$ and~$g\colon \F{C} \tickar \R{D}$, their \emph{biproduct}~$(f + g)\colon \F{A} + \F{C} \tickar \R{B} + \R{D}$ is defined by
\begin{equation}
\begin{aligned}
    (f + g)\colon (\F{A} + \F{C})\op  \times (\R{B} + \R{D}) & \toinPos \Bool,  \\
            \tup{ \tup{1,\F{a}}^*, \tup{1,\R{b}}} & \mapsto f(\F{a}^*, \R{b}), \\
            \tup{ \tup{2,\F{c}}^*, \tup{1,\R{b}}} & \mapsto \false, \\
            \tup{ \tup{1,\F{a}}^*, \tup{2,\R{d}}} & \mapsto \false, \\
            \tup{ \tup{2,\F{c}}^*, \tup{2,\R{d}}} & \mapsto g(\F{c}^*, \R{d}),
\end{aligned}
\end{equation}
and represented as in~\cref{fig:biproductdp}.

\begin{figure}[h!]
\begin{center}
    \includesag{52_biproduct}
\end{center}
\caption{Diagrammatic representation of the biproduct of design problems. \label{fig:biproductdp}}
\end{figure}
In particular, when~$f, g\colon \F{A} \tickar \R{B}$,~$(f + g)\colon \F{A} + \F{A} \tickar \R{B} + \R{B}$ is defined by
\begin{equation}
\begin{aligned}
    (f + g) \colon (\F{A} + \F{A})\op  \times (\R{B} + \R{B}) & \toinPos \Bool,  \\
            \tup{ \tup{1, \F{a}}^*, \tup{1, \R{b}}} & \mapsto f(\F{a}^*, \R{b}), \\
            \tup{ \tup{2, \F{a}}^*, \tup{1, \R{b}}} & \mapsto \false, \\
            \tup{ \tup{1, \F{a}}^*, \tup{2, \R{b}}} & \mapsto \false, \\
            \tup{ \tup{2, \F{a}}^*, \tup{2, \R{b}}} & \mapsto g(\F{a}^*, \R{b}).
\end{aligned}
\end{equation}
\end{definition}


Assume$f,g \colon \F{A} \tickar \R{B}$. Intuitively,~$f+g$ can be thought of as: pick either~$f$ or~$g$, then throw away the other one, whereas on any~$\tup{\F{a}^*,\R{b}}$,~$f \vee g$ always picks the better (more feasible) of either~$f(\F{a}^*,\R{b})$ or~$g(\F{a}^*,\R{b})$. Note that~$f+g$ introduces an extra parameter, since the choice of~$f$ or~$g$ has to be hard-coded into the larger design problem.


In general, the biproduct is defined first on objects of the category, so what we are calling a biproduct of design
problems is actually the unique map derived from the biproduct~$A + B$ of posets in~$\DP$, namely the disjoint union. In the more general setting~$f\colon \F{A} \tickar \R{B}$ and~$g \colon \F{C} \tickar \R{D}$, the biproduct on objects~$\F{A} + \F{C}$ models~$\F{A}$ and $\F{C}$ as \emph{interchangeable entities}:~$f+g$ may output one functionality from either~$\F{A}$ or $\F{C}$, and similarly it requires only one resource from either~$\R{B}$ or~$\R{D}$. We can emphasize one or the other condition by letting~$\F{A} = \F{C}$ or~$\R{B} =\R{D}$ above, thus deriving the product of design problems and coproduct of design problems, respectively.


\begin{definition}[Coproduct of design problems]
\label{define:coproduct}
Given two design problems~$f\colon \F{A} \tickar \R{C}$ and~$g \colon \F{B} \tickar \R{C}$, their \emph{coproduct}~$(f \sqcup g)\colon \F{A} + \F{B} \tickar \R{C}$ is defined by
\begin{equation}
\begin{aligned}
    (f \sqcup g) \colon (\F{A} + \F{B})\op \times \R{C} & \toinPos \Bool,  \\
            \tup{ \F{a}^*, \R{c}} & \mapsto f(\F{a}^*, \R{c}), \\
            \tup{ \F{b}^*, \R{c}} & \mapsto g(\F{b}^*, \R{c}),
\end{aligned}
\end{equation}
and represented as in~\cref{fig:coproductdp}.

\begin{figure}[h!]
\begin{center}
    \includesag{52_coproduct}
\end{center}
\caption{Diagrammatic representation of the coproduct of design problems. \label{fig:coproductdp}}
\end{figure}
\end{definition}

\begin{definition}[Product of design problems]
\label{define:product}
Given design problems~$f\colon \F{C} \tickar \R{A}$,~$g \colon \F{C} \tickar \R{B}$, their \emph{product}~$(f \times g)\colon \F{C} \tickar \R{A} + \R{B}$ is defined by
\begin{equation}
\begin{aligned}
    (f \times g) \colon \F{C}\op  \times (\R{A} + \R{B}) & \toinPos \Bool,  \\
            \tup{\F{c}^*, \R{a}} & \mapsto f(\F{c}^*, \R{a}), \\
            \tup{\F{c}^*, \R{b}} & \mapsto g(\F{c}^*, \R{b}),
\end{aligned}
\end{equation}
and represented as in~\cref{fig:productdp}.

\begin{figure}[h!]
\begin{center}
    \includesag{52_product}
\end{center}
\caption{Diagrammatic representation of the product of design problems. \label{fig:productdp}}
\end{figure}

\end{definition}


To show that~$A + B$ is in fact a biproduct in the categorical sense, we need to show that it satisfies certain properties. These properties guarantee that the biproduct is the most ``efficient'' way of combining two objects in~$~\DP$ in a certain sense, and that the resulting combination is unique when it exists. Specifically, we need to show that~$A + B$ is both a product~$(+, \pi_A, \pi_B)$ and a coproduct~$(+, \iota_A, \iota_B)$ in~$\DP$, and satisfies an extra coherence condition:~$\pi_B \circ \iota_B = \id_B$.

\begin{shaded}
\begin{definition}[Initial and terminal object]
Let~$\CatC$ be a category and let~$A \in \CatC$ be an object. We say that~$A$ is an \emph{initial object} if, for all~$B \in\CatC$, the hom-set~$\Hom_\CatC(A,B)$ has exactly one element. We say that~$A$ is a \emph{terminal object} if, for all~$D\in\CatC$, the hom-set~$\Hom_\CatC(D,A)$ has exactly one element.
\end{definition}

\begin{definition}[Finite coproducts and products]
We say that~$\CatC$ \emph{has finite coproducts} if it has an initial object and every pair of objects in~$\CatC$ has a coproduct.
We say that~$\CatC$ \emph{has finite products} if it has a terminal object and every pair of objects in~$\CatC$ has a product.
\end{definition}
\end{shaded}

\begin{example}
    The category~$\Poset$ has finite coproducts and finite products.
    The coproduct~$\sqcup_\Poset$ is the disjoint union~$+$.
    The product~$\times_\Poset$ is the Cartesian product~$\times$.
\end{example}

\begin{lemma}
    The category~$\DP$ has finite coproducts and finite products.
    The coproduct~$\sqcup_\DP$ is the disjoint union~$+$ (\cref{def:disjoint-union}).
    The product~$\times_\DP$ is also the disjoint union~$+$.
\end{lemma}

\begin{proof}
Let~$\iota_A$ be the injection of posets~$A \to A+B$. We define the design problems
\begin{equation}
    \begin{aligned}
    \comp{\iota_A} \colon \F{A} &\tickar \R{A} + \R{B}\\
    \tup{\F{a}^*,\R{c}} &\mapsto \iota_A(\F{a})\ordleq_{A+B}\R{c},
    \end{aligned}
\end{equation}
and 
\begin{equation}
    \begin{aligned}
    \comp{\iota_B} \colon \F{B} &\tickar \R{A} + \R{B}\\
    \tup{\F{b}^*,\R{c}} &\mapsto \iota_B(\F{b})\ordleq_{A+B}\R{c}.
    \end{aligned}
\end{equation}

Recall the coproduct of design problems in Definition~\ref{define:coproduct}. To show that~$A+B$ is a coproduct (of posets) in~$\DP$, we need to show that the coproduct~$f \sqcup g$ of design problems is unique and that it satisfies~$f = \comp{\iota_A} \then (f \sqcup g)$ and~$g = \comp{\iota_B} \then (f \sqcup g)$. This can be verified by simply writing out the composition. In the following we denote~$\ordleq_{A+B}$ by $\ordleq$. One has
\begin{equation}
    \begin{aligned}
    \left(\comp{\iota_A}\then (f\sqcup g)\right)(\F{a}^*,\R{c})&=\bigvee_{x' \in A+B}\comp{\iota_A}(\F{a}^*,\R{x'})\wedge (f\sqcup g)(\F{x'}^*,\R{c})\\
    &=\bigvee_{\tup{1,x} \in A+B}
    \left(\iota_A(\F{a})\ordleq\tup{1,\R{x}}\right)\wedge  f(\F{x}^*,\R{c})  \vee \bigvee_{\tup{2,x} \in A+B}
    \left(\iota_A(\F{a})\ordleq \tup{2,\R{x}}\right)\wedge  g(\F{x}^*,\R{c})\\
    &=\bigvee_{\tup{1,x} \in A+B}
    \left( \left(\F{a}\ordleq \R{x}\right) \wedge  f(\F{x}^*,\R{c})\right) \vee \bigvee_{\tup{2,x} \in A+B}
    \left( \false \wedge  g(\F{x}^*,\R{c}) \right)\\
    &=\bigvee_{\tup{1,x} \in A+B} \left(\F{a}\ordleq \R{x}\right) \wedge  f(\F{x}^*,\R{c})\\
    &=\bigvee_{x\in A}\left( \F{a}\ordleq \R{x}\right) \wedge f(\F{x}^*,\R{c})\\
    &=\left( \id_A \then f\right)\tup{\F{a}^*,\R{c}}\\
    &=f(\F{a}^*,\R{c}).
    \end{aligned}
\end{equation}
Similarly:
\begin{equation}
    \begin{aligned}
    \left(\comp{\iota_B}\then (f\sqcup g)\right)(\F{b}^*,\R{c})&=\bigvee_{x' \in A+B}\comp{\iota_B}(\F{b}^*,\R{x})\wedge (f\sqcup g)(\F{x}^*,\R{c})\\
    &=\bigvee_{\tup{1,x} \in A+B}
    \left(\iota_B(\F{b})\ordleq\tup{1,\R{x}}\right)\wedge  f(\F{x}^*,\R{c})  \vee \bigvee_{\tup{2,x} \in A+B}
    \left(\iota_B(\F{b})\ordleq \tup{2,\R{x}}\right)\wedge  g(\F{x}^*,\R{c})\\
    &=\bigvee_{\tup{1,x} \in A+B}
    \left(\false \wedge  f(\F{x}^*,\R{c})\right) \vee \bigvee_{\tup{2,x} \in A+B}
    \left(\left(\F{b}\ordleq \R{x}\right) \wedge  g(\F{x}^*,\R{c}) \right)\\
    &=\bigvee_{\tup{2,x} \in A+B} \left(\F{b}\ordleq \R{x}\right) \wedge  g(\F{x}^*,\R{c})\\
    &=g(\F{b}^*,\R{c}).
    \end{aligned}
\end{equation}

The proof that the disjoint union~$+$ is also a categorical product in~$\DP$ is analogous, now with~$f \colon \F{C} \tickar \R{A}$,~$g \colon \F{C} \tickar \R{B}$, and replacing the injections~$\iota_A$ with projections:
\begin{equation}
    \begin{aligned}
    \comp{\pi_A} \colon \F{A}+\F{B} &\tickar \R{A}\\
    \tup{\F{c}^*,\R{a}}&\mapsto \pi_A(\F{c})\ordleq_A \R{a},
    \end{aligned}
\end{equation}
and
\begin{equation}
    \begin{aligned}
    \comp{\pi_B} \colon \F{A}+\F{B} &\tickar \R{B}\\
    \tup{\F{c}^*,\R{b}}&\mapsto \pi_B(\F{c})\ordleq_B \R{b},
    \end{aligned}
\end{equation}
and the coproduct~$f \sqcup g$ with the product (on morphisms)~$f \times g \colon \F{C} \tickar \R{A} + \R{B}$ from Definition~\ref{define:product}. We have
\begin{equation}
    \begin{aligned}
    \left((f\times g)\then \comp{\pi_A}\right)(\F{c}^*,\R{a})&=\bigvee_{x'\in A+B} (f\times g)(\F{c}^*,\R{x'}) \wedge \comp{\pi_A}(\F{x'}^*,\R{a})\\
    &=\bigvee_{\tup{1,x}\in A+B}\left( f(\F{c}^*,\R{x}) \wedge \comp{\pi}_A(\tup{1,\F{x}}^*,\R{a})\right) \vee 
    \bigvee_{\tup{2,x}\in A+B}\left( g(\F{c}^*,\R{x}) \wedge \comp{\pi}_A(\tup{1,\F{x}}^*,\R{a})\right)\\
    &=\bigvee_{\tup{1,x}\in A+B}\left( f(\F{c}^*,\R{x}) \wedge (\F{x}\ordleq \R{a})\right) \vee 
    \bigvee_{\tup{2,x}\in A+B}\left( g(\F{c}^*,\R{x}) \wedge \false \right)\\
    &=\bigvee_{\tup{1,x}\in A+B} f(\F{c}^*,\R{x}) \wedge (\F{x}\ordleq \R{a})\\
    &=f(\F{c}^*,\R{a}).
    \end{aligned}
\end{equation}
Similarly:
\begin{equation}
    \begin{aligned}
    \left((f\times g)\then \comp{\pi_B}\right)(\F{c}^*,\R{b})&=\bigvee_{x'\in A+B} (f\times g)(\F{c}^*,\R{x'}) \wedge \comp{\pi_B}(\F{x'}^*,\R{b})\\
    &=\bigvee_{\tup{1,x}\in A+B}\left( f(\F{c}^*,\R{x}) \wedge \comp{\pi}_B(\tup{1,\F{x}}^*,\R{b})\right) \vee 
    \bigvee_{\tup{2,x}\in A+B}\left( g(\F{c}^*,\R{x}) \wedge \comp{\pi}_B(\tup{2,\F{x}}^*,\R{b})\right)\\
    &=\bigvee_{\tup{1,x}\in A+B}\left( f(\F{c}^*,\R{x}) \wedge \false \right) \vee 
    \bigvee_{\tup{2,x}\in A+B}\left( g(\F{c}^*,\R{x}) \wedge (\F{x}\ordleq \R{b}) \right)\\
    &=\bigvee_{\tup{2,x}\in A+B} g(\F{c}^*,\R{x}) \wedge (\F{x}\ordleq \R{b}) \\
    &=g(\F{c}^*,\R{b}).
    \end{aligned}
\end{equation}
\end{proof}

\begin{remark}Where it is clear from context, we will not distinguish between the injection of posets~$\iota_A$ and its corresponding design problem~$\comp{\iota_A}$, and refer to both by~$\iota_A$. The same holds for the projections~$\pi_A$ and~$\comp{\pi_A}$.
\end{remark}

\begin{table}[b]
\begin{small}
\begin{center}
\begin{tabular}{llll}
    Category&$\Set$&$\Pos$&$\DP$\\
    \hline
    Objects&sets&posets&posets\\
    Morphisms &functions&monotone functions&design problems\\
    Product (objects) &Cartesian product& Cartesian product& disjoint union\\
    Product (morphisms) &(not used)&(not used)&$\times_\DP$\\
    Coproduct (objects) &disjoint union&disjoint union&disjoint union\\
    Coproduct (morphisms) &(not used)&(not used)&$\sqcup_\DP$\\
    Biproduct (morphisms)& none&none& disjoint union\\
    Tensor Product &$\times$ or~$\sqcup$&$\times$&$\times$\\
    Initial object &$\emptyset$&$\emptyset$&$\emptyset$\\
    Terminal object &$\singleton$&$\singleton$&$\singleton$
\end{tabular}
\end{center}
\end{small}
\caption{A comparison of $\Pos, \Set$, and $\DP$.}
\end{table}

\begin{comment}
\begin{table}[b]
\resizebox{\textwidth}{!}{
\begin{tabular}{ccccccccccc}
    category &
    objects & morphisms &
    product (ob) & product (morph) & coproduct (ob) & coproduct (morph) & biproduct (morph) & tensor product &
    initial  & terminal\\
    \hline

    $\Cat{Set}$ &
    sets & functions &
    Cartesian product & (not used) & disjoint union & (not used) &
    none & $\times$ \emph{or} $\sqcup$ &
    $\emptyset$ & $\{\ast\}$ \\

    $\Cat{Pos}$ &
    posets & monotone maps &
    Cartesian product & (not used) & disjoint union & (not used) &
    none & $\times$ &
    $\emptyset$  & $\{\ast\}$\\

    $\DP$ &
    posets & design problems &
    disjoint union & $\times_\DP$ & disjoint union & $\sqcup_\DP$  &
    disjoint union & $\times$ &
    $\emptyset$ & $\{\ast\}$\\

    % category &
    % objects & morphisms &
    % product & coproduct & biproduct & tensor product
    % initial  & terminal\\
\end{tabular}}
\caption{A comparison of $\Pos, \Set$, and $\DP$.}
\end{table}
\end{comment}

\begin{shaded}
\begin{definition}[Zero object and morphism]
We call an object that is both initial and terminal the \emph{zero object} and we indicate it as~$0$. For any other pairs of objects~$A, B\in\CatC$, there is a unique morphism of the form~$A \to 0\to B$; we call it the \emph{zero morphism} and denote it~$0_{A,B}\colon A \to B$.
\end{definition}
\end{shaded}

\begin{example}
$\Pos$ has an initial object~$\emptyset$ and a terminal object~$\singleton$, but no zero object.
\end{example}

\begin{lemma}
In the category~$\DP$, the empty poset~$\emptyset$ is a zero object. The zero morphism~$0_{\F{\cP},\R{\cQ}}\colon \F{\cP} \tickar \R{\cQ}$ is the design problem that is always infeasible.
\end{lemma}
\begin{proof}
    First of all,~$\emptyset$ is an initial object in~$\$DP$. Indeed, for any poset~$\cP $, there is a unique design problem~$\F{\emptyset} \tickar \R{\cP}$ given by the unique monotone map~$\emptyset=\F{\emptyset}\op\times \R{\cP} \to\Bool$. Similarly, there is a unique design problem~$\F{\cP} \tickar \R{\emptyset}$, so~$\emptyset$ is also terminal. For any posets~$\cP ,\cQ$, the zero map~$0 \colon \F{\cP} \tickar \R{\cQ}$ is the monotone map~$\F{\cP} \op\times \R{\cQ} \toinPos \Bool$ sending everything to~$\false$.
\end{proof}

\begin{shaded}
\begin{definition}[Biproduct]
Let~$\CatC$ be a category with a zero-object~$0$, and let~$C_1,C_2\in\CatC$ be objects. Suppose that~$\tup{D,\pi_1,\pi_2,\iota_1,\iota_2}$ is a tuple such that~$\tup{D,\pi_1,\pi_2}$ is a product of~$C_1$ and~$C_2$ and~$\tup{D,\iota_1,\iota_2}$ is a coproduct of~$C_1$ and~$C_2$. We say that it is a \emph{biproduct} if, for each~$1\ordleq i,j\ordleq 2$, we have
\begin{equation}
\iota_i\then \pi_j=
\begin{cases}
	\id_{C_i}, &\text{ if }i=j,\\
	0, &\text{ if }i\neq j,
\end{cases} \label{eq:biproduct-condition}
\end{equation}
as maps~$C_i\to C_j$ in~$\CatC$.
\end{definition}
\end{shaded}

\begin{lemma}
The disjoint union is a biproduct for~$\DP$.
\end{lemma}
\begin{proof}
    We have already shown that the disjoint union is both
    a product and a coproduct. We just need to verify that~\cref{eq:biproduct-condition} holds
    for the design problems~$\pi_A, \pi_B, \iota_A, \iota_B$.
    This amounts to checking the four conditions:
    \begin{equation}
    \begin{aligned}
        (\iota_A\then \pi_A) &= \id_{A}, \\
        (\iota_A\then \pi_B) &= 0_{A,B}, \\
        (\iota_B\then \pi_A) &= 0_{B,A},\\
        (\iota_B\then \pi_B) &= \id_{B}.
    \end{aligned}
    \end{equation}
    We check only the first two, as the other two are similar.
    To check the second condition, we compute an explicit expression for~$(\iota_A\then \pi_B)\colon \F{A} \tickar \R{B}$, using the definition
    of design problem series composition:
    \begin{equation}
    \begin{aligned}
        (\iota_A\then \pi_B) \colon  \F{A}\op\times \R{B} & \toinPos \Bool \\
        \tup{\F{x}^*, \R{z}} &\mapsto
        \bigvee_{y \in A + B} \iota_A(\F{x}^*,\R{y}) \wedge \pi_B(\F{y}^*,\R{z}).
    \end{aligned}
    \end{equation}
    We can do a case analysis for~$y\in A+B$. Suppose~$y\in A$.
    Then~$\pi_B(\F{y}^*,\R{z}) = \false$. If~$y \in B$, then~$\iota_A(\F{x}^*,\R{y}) = \false$.
    Therefore, the sum is always false, and hence~$(\iota_A\then \pi_B) = 0_{A,B}$.
    For the first condition, we compute  an explicit expression for~$(\iota_A\then \pi_A) \colon \F{A} \tickar \R{A}$:
    \begin{equation}
    \begin{aligned}
        (\iota_A\then \pi_A) \colon  \F{A}\op\times \R{A} & \toinPos \Bool \\
        \tup{\F{x}^*, \R{z}} &\mapsto
        \bigvee_{y \in A + B} \iota_A(\F{x}^*,\R{y}) \wedge \pi_A(\F{y}^*,\R{z}).
    \end{aligned}
    \end{equation}
    %Note that we are summing over $y \in A + B$. 
    Let us divide the ``sum'' in two parts:
    \begin{equation}
    \bigvee_{y \in A} \iota_A(\F{x}^*,\R{y}) \wedge \pi_A(\F{y}^*,\R{z}) \quad\vee\quad
    \bigvee_{y \in B} \iota_A(\F{x}^*,\R{y}) \wedge \pi_A(\F{y}^*,\R{z}).
    \end{equation}
    For~$y \in B$,~$\iota_A(\F{x}^*,\R{y})$ and~$\pi_A(\F{y}^*,\R{z})$ are both false, and we can therefore ignore the second sum.
    From the definition of~$\iota_A$ we have that for~$y\in A$,~$ \iota_A(\F{x}^*,\R{y})=\F{x} \ordleq_A \R{y}$, and from the definition of~$\pi_A$ we have that for~$y\in A$, $\pi_A(\F{y}^*,\R{z})=\F{y} \ordleq_A \R{z}$. Thus we compute:
\begin{equation}
    \begin{aligned}
    \left(\iota_A\then \pi_A\right)(\F{x}^*, \R{z}) &= \bigvee_{y \in A} \iota_A(\F{x}^*,\R{y}) \wedge \pi_A(\F{y}^*,\R{z})  \\
     &= \bigvee_{y \in A} (\F{x} \ordleq_A \R{y}) \wedge  (\F{y} \ordleq_A \R{z}) \\
     &= \F{x} \ordleq_A \R{z}\\
     &= \id_A(\F{x}^*, \R{z})
\end{aligned}
\end{equation}
\end{proof}

\subsection{Relationship between intersection and monoidal prodcut}
To define the precise relationship between the monoidal product~$f \otimes g$ (\cref{def:monoidalproduct}) and the intersection~$f \wedge g$, we first define two operations,~$\mathsf{split} \colon \F{A} \tickar \R{A} \times \R{A}$ and~$\mathsf{fuse} \colon \F{A} \times \F{A} \tickar \R{A}$, which correspond to splitting and fusing wires in a diagram:
\begin{equation}
\begin{aligned}
    \mathsf{split} \colon \F{A}\op \times (\R{A} \times \R{A}) &\to_\Pos \Bool \\
    \tup{\F{x}^*, \tup{\R{y}, \R{z}}} &\mapsto (\F{x} \ordleq_A \R{y}) \wedge (\F{x} \ordleq_A \R{z})
\end{aligned}
\end{equation}
~
\begin{equation}
\begin{aligned}
    \mathsf{fuse} \colon (\F{A} \times \F{A})\op \times \R{A} &\to_\Pos \Bool \\
    \tup{ \tup{\F{x}^*, \F{y}^*}, \R{z}} &\mapsto (\F{x} \ordleq_A \R{z}) \wedge (\F{y} \ordleq_A \R{z}).
\end{aligned}
\end{equation}

\begin{lemma}
\label{lemma:intersection}
For~$f,g \colon \F{A} \tickar \R{B}$,
\begin{equation}
f \wedge g = \mathsf{split} \then (f \otimes g) \then \mathsf{fuse},
\end{equation}
as shown in~\cref{fig:lemmasplitfuse}.
\begin{figure}[h!]
\begin{center}
\includesag{50_split_1_2}
\end{center}
\caption{$f \wedge g = \mathsf{split} \then (f \otimes g) \then \mathsf{fuse}$. \label{fig:lemmasplitfuse}}
\end{figure}
\end{lemma}


\begin{proof}
Recall that 
    \begin{equation}
        (f\otimes g)(\tup{\F{a_1},\F{a_2}}^*,\tup{\R{b_1},\R{b_2}})=f(\F{a_1}^*,\R{b_1})\wedge g(\F{a_2}^*,\R{b_2}).
    \end{equation}
We have
    \begin{equation}
        \begin{aligned}
            \left(\left(f\otimes g\right) \then \mathsf{fuse}\right)(\tup{\F{a_1},\F{a_2}}^*,\R{b})&=\bigvee_{\tup{b',b''}\in B\times B}(f(\F{a_1}^*,\R{b'})\wedge g(\F{a_2}^*,\R{b''}))\wedge ((\F{b'}\ordleq \R{b}) \wedge (\F{b''}\ordleq \R{b}))\\
            &=f(\F{a_1}^*,\R{b})\wedge g(\F{a_2}^*,\R{b}).
        \end{aligned}
    \end{equation}
Thus, we have
    \begin{equation}
        \begin{aligned}
        (\mathsf{split}\then f\otimes g\then \mathsf{fuse})(\F{a}^*,\R{b})&=\bigvee_{\tup{a',a''}\in A\times A} \mathsf{split}(\F{a}^*,\tup{\R{a'},\R{a''}})\wedge (f(\F{a'}^*,\R{b})\wedge g(\F{a''}^*,\R{b}))\\
        &=\bigvee_{\tup{a',a''}\in A\times A}(\F{a}^*\ordleq \R{a'})\wedge (\F{a}^*\ordleq \R{a''})\wedge f(\F{a'}^*,\R{b})\wedge g(\F{a''}^*,\R{b})\\
        &=f(\F{a}^*,\R{b})\wedge g(\F{a}^*,\R{b})\\
        &=(f\wedge g)(\F{a}^*,\R{b}).
        \end{aligned}
    \end{equation}
\end{proof}

\subsection{FIND TITLE}

We can also re-define the sum~$\vee$ and intersection~$\wedge$ using companions and conjoints, which allows us to introduce some useful constructions.

\begin{definition}[Diagonal function]
Define the \emph{diagonal function}~$\Delta_P\colon P \to P \times P$:
\begin{equation}
\begin{aligned}
    \Delta_P \colon P & \to P \times P, \\
             p & \mapsto \tup{p, p}.
\end{aligned}
\end{equation}
\end{definition}

\begin{definition}[Codiagonal function]
Define the \emph{codiagonal function}~$\Diamond_P\colon P+P \to P $:
\begin{equation}
\begin{aligned}
    \Diamond_P \colon P + P & \to P,  \\
            \tup{1,p} & \mapsto p, \\
            \tup{2,p} & \mapsto p.
\end{aligned}
\end{equation}
\end{definition}

\noindent Using the diagonal function, \cref{lemma:intersection} can be rewritten as the following lemma.

\begin{lemma}
    Given~$f, g\colon \F{A} \tickar \R{B}$, we have:
    \begin{equation}
        f \vee g =  \conj{\Diamond}_A \then (f + g)\then \comp{\Diamond}_B.
    \end{equation}
\end{lemma}

\begin{proof}
First of all, note that 
\begin{equation}
    \begin{aligned}
    \conj{\Diamond}_A\colon \F{A}&\tickar \R{A}+\R{A}\\
    \tup{\F{a_1}^*,\tup{1,\R{a_2}}}&\mapsto \F{a_1}\ordleq \R{a_2}\\
    \tup{\F{a_1}^*,\tup{1,\R{a_3}}}&\mapsto \F{a_1}\ordleq \R{a_3}
    \end{aligned}
\end{equation}
and
\begin{equation}
    \begin{aligned}
    \comp{\Diamond}_B\colon \F{B}+\F{B}&\tickar \R{B}\\
    \tup{\tup{1,\F{b_1}}^*,\R{b_3}}&\mapsto \F{b_1}\ordleq \R{b_3}\\
    \tup{\tup{2,\F{b_2}}^*,\R{b_3}}&\mapsto \F{b_2}\ordleq \R{b_3}
    \end{aligned}
\end{equation}
We start by looking at~$\underbrace{\conj{\Diamond}_A\then (f+g)}_{\star}\colon \F{A} \tickar \R{B}+\R{B}$. 
\begin{equation}
    \begin{aligned}
    \star (\tup{\F{a}^*,\R{b}})&=\bigvee_{a'\in A+A} \conj{\Diamond}_A(\tup{\F{a}^*,\R{a'}})\wedge (f+g)(\tup{\F{a'}^*,\R{b}})\\
    &=\left( \bigvee_{\tup{1,a'}\in A+A} (\F{a}\ordleq \R{a'})\wedge f(\F{a'}^*,\R{b}) \right)\vee \left( \bigvee_{\tup{2,a'}\in A+A} (\F{a}\ordleq \R{a'})\wedge g(\F{a'}^*,\R{b}) \right)\\
    &=f(\F{a}^*,\R{b}) \vee g(\F{a}^*,\R{b}).
    \end{aligned}
\end{equation}

Let's now look at~$\star \then \comp{\Diamond}_B\colon \F{A} \tickar \R{B}$:
\begin{equation}
    \begin{aligned}
    (\star \then \comp{\Diamond}_B)(\F{a}^*,\R{b'})&=\bigvee_{b\in B+B} \star(\F{a}^*,\R{b})\wedge \comp{\Diamond}_B(\F{b}^*,\R{b'}) \\
    &=\left(\bigvee_{\tup{1,b}\in B+B} f(\F{a}^*,\R{b}) \wedge (\F{b}\ordleq \R{b'})\right) \vee 
    \left(\bigvee_{\tup{2,b}\in B+B} g(\F{a}^*,\R{b}) \wedge (\F{b}\ordleq \R{b'})\right)\\
    &=f(\F{a}^*,\R{b'})\vee g(\F{a}^*,\R{b'}).
    \end{aligned}
\end{equation}
\end{proof}

Similarly, using the codiagonal function, one can prove the following.
\begin{lemma}
    Given~$f, g\colon \F{A} \tickar \R{B}$, we have:
    \begin{equation}
        f \wedge g = \comp{\Delta}_A \then(f + g) \then \conj{\Delta}_B.
    \end{equation}
\end{lemma}
\begin{proof}
First, note that
\begin{equation}
    \begin{aligned}
    \comp{\Delta}_A \colon \F{A}&\tickar \R{A}\times \R{A}\\
    \tup{\F{a_1}^*,\tup{\R{a_2},\R{a_3}}}&\mapsto \Delta_A(\F{a_1})\leq \tup{\R{a_2},\R{a_3}}\\
    &= \tup{\F{a_1},\F{a_1}}\leq \tup{\R{a_2},\R{a_3}}\\
    &= (\F{a_1}\leq \R{a_2}) \wedge (\F{a_1}\leq \R{a_3}).
    \end{aligned}
\end{equation}
and 
\begin{equation}
    \begin{aligned}
    \conj{\Delta}_B \colon \F{B}\times \F{B}&\tickar \R{B}\\
    \tup{\tup{\F{b_1},\F{b_2}}^*,\R{b_3}}&\mapsto \tup{\F{b_1},\F{b_2}}\leq \Delta_B(\R{b_3})\\
    &= (\F{b_1}\leq \R{b_3}) \wedge (\F{b_2}\leq \R{b_3}).
    \end{aligned}
\end{equation}
We start by looking at $\comp{\Delta}_A \then (f+g) \colon \F{A}\tickar \R{B}+\R{B}$:
\begin{equation}
    \begin{aligned}
    \left(\comp{\Delta}_A\then (f+g)\right)\left(\tup{\F{a}^*,\R{b}}\right)&=\bigvee_{}
    \end{aligned}
\end{equation}
\todo{Adjust signatures, have to find a good way to write it down}
\end{proof}
Unlike $\conj{\Diamond} = \mathsf{split}$ and $\comp{\Diamond} = \mathsf{fuse}$, $\comp{\Delta}$ and $\conj{\Delta}$ do not have an intuitive diagrammatic representation.