% !TEX root = ../CategoricalCoDesign.tex
\section{Ordering design problems}

We claimed that category theory is an efficient language for talking about \emph{structure}, and showed how the category $\DP$ could accommodate all the basic operations required by a theory of formal engineering design. Here, we illustrate some of the applications and advantages of $\DP$ in reasoning about and solving design problems, starting with the fact that that $\DP$ is compact closed, which allows us to compose and reason about ``design problems of design problems''.

\begin{definition}[Order on $\DP$]
\label{def:DP_loc_pos}

Suppose that $A$ and $B$ are two posets, and that $f,g \colon \F{A} \tickar \R{B}$ are design problems. We say that $f$ \emph{implies} $g$, denoted $f \geq_{\DP} g$, if $f(\F{a}^*,\R{b}) \leq g(\F{a}^*,\R{b} )$ in $\Bool$, for all~$\F{a} \in \F{A}$
and~$\R{b} \in \R{B}$. In other words, if $f$ feasible implies that $g$ is feasible. We often denote the relation $f \geq_{\DP} g$
by the diagram reported in~\cref{fig:dpimplies}.

\begin{figure}[h!]
\begin{center}
\includesag{60_relation}
\end{center}
\caption{The design problem $f$ implies the design problem $g$. \label{fig:dpimplies}}
\end{figure}
\end{definition}


\begin{definition}[Lattice]
\label{def:lattice}
A lattice is a poset $\tup{P, \leq}$ with some additional properties:
\begin{enumerate}
    \item Given two points $p, q \in P$, it is always possible to define their least upper bound, called \emph{join}, and indicated as $p \vee q$.
    \item Given two points $p, q \in P$, it is always possible to define their greatest lower bound, called \emph{meet}, and indicated as $p \wedge q$.
\end{enumerate}
\end{definition}

If there is a least upper bound for the entire lattice, it is called
the \emph{top}. If the greatest lower bound exists it is called the \emph{bottom}. If both a top and a bottom exist, we call the lattice \emph{bounded}.

\begin{example}
    In \cref{ex:hasseinclusion} we presented the poset arising from the power set of a set $A$ and ordered via subset inclusion. This is a lattice, bounded by $A$ and by the empty set $\emptyset$. 
\end{example}

\begin{example}
Consider the set $\{1,2,3,6\}$ ordered by divisibility is a lattice. However, the set $\{1,2,3\}$ ordered by divisibility is not, since 2 and 3 lack a meet (\cref{fig:exlattice}).
\begin{figure}[h!]
\begin{center}
\includesag{40_dpcatfig_exlattice}
\end{center}
\caption{Example of lattice. \label{fig:exlattice}}
\end{figure}
\end{example}

\begin{remark}
We denote by $1_{\F{A},\R{B}}$ the design problem which is always feasible, for any functionality-resource pair $\F{A},\R{B}$. We denote by $0_{\F{A},\R{B}}$ the design problem which is never feasible, for any functionality-resource pair $\F{A},\R{B}$.
\end{remark}
\begin{lemma}
\label{lemma:dpboundedlattice}
$\Hom_\DP(\F{A},\R{B})$ is a bounded lattice with $\vee$ as meet, $\wedge$ as join, least upper bound  $1_{\F{A},\R{B}}$ and greatest lower bound $0_{\F{A},\R{B}}$.
\end{lemma}

\begin{proof}
First of all, we need to prove that $\Hom_\DP(\F{A},\R{B})$ is a poset. To prove this, we check the following:

\begin{itemize}
    \item \emph{Reflexivity}: Given $f\in \Hom_\DP(\F{A},\R{B})$, $f\geq_\DP f$ is always true.
    \item \emph{Antisymmetry}: Given $f,g\in \Hom_\DP(\F{A},\R{B})$, if $f\geq_\DP g$ and $g\geq_\DP f$, then $f=g$.
    \item \emph{Transitivity}: Given $f,g,h\in \Hom_\DP(\F{A},\R{B})$, $f\geq_\DP g$, and $g\geq_\DP h$, then $f\geq_\DP h$.
\end{itemize}
Therefore, $\Hom_\DP$ is a poset. Furthermore, consider two design problems $f,g\in \Hom_\DP(\F{A},\R{B})$. Their least upper bound (join) is $f\wedge g$, since it is the least design problem implying both $f$ and $g$. Their greatest lower bound (meet), instead, is $f\vee g$, since it is the greatest design problem implied by both $f$ and $g$. This proves that $\Hom_\DP$ is a lattice. To prove that it is bounded, we identify the top element as $1_{\F{A},\R{B}}$ (it implies all other design problems) and the bottom element as $0_{\F{A},\R{B}}$ (it is implied by all the other design problems).
\end{proof}

\todo{This part below should be moved to the 
second-order part}
In particular, the fact that $\Hom_\DP$ is a poset, means that design problems in $\Hom_\DP(A,B)$ can be counted as functionalities and/or resources in a design problem.

\begin{example}\label{ex:r&d}
Jeb's Spaceship Parts lost their last contract to Starshow Bob after submitting a completely inferior engine in every respect ($\text{Jeb-XX} \Imp \text{Bob-Roc}$). In response, they've decided to invest heavily in R\&D, which can be thought of as a design problem of its own: Given time and money as resources, what kind of engine technology can they produce as a `functionality'?

\begin{figure}[h!]
\begin{center}
\includesag{60_engine}
\end{center}
\caption{Example of order in $\DP$. \label{fig:orderdp}}
\end{figure}
\end{example}