% !TEX root = ../CategoricalCoDesign.tex
\section{Composing recognizing structure}
\subsection{An axiomatization of design problems}
We can now give our first definition of design problems.

\begin{definition}[Design Problems as Monotone Functions] \label{def:design-problem}
Suppose that $\F{F}, \R{R}$ are posets. A \emph{design problem} is a monotone function of the form
\begin{equation}
	\adp \colon  \F{F}\op \times \R{R} \to \Bool.
\end{equation}
We represent it by an arrow $\adp \colon \F{F} \tickar \R{R}$.
\end{definition}

\paragraph{Intended semantics}

When we consider a design problem $\adp\colon \F{F} \tickar \R{R}$, we imagine the poset~$\F{F}$ to represent the \textbf{f}unctionality to be provided, while the poset~$R$ represents the \textbf{r}esources required. The object $\adp \colon \F{F} \tickar \R{R}$ is a relation that describes which combinations of functionality and resources are feasible: for each $\F{f}^* \in \F{F}\op$ and $\R{r} \in \R{R}$, $\adp(\F{f}^*,\R{r})$ is a truth value, $\true$ or $\false$, which we call the \emph{feasibility of $\F{f}$ given $\R{r}$}. The value~$\adp(\F{f}^*,\R{r})$ is the answer to the question ``Is the functionality $\F{f}$ feasible with resources~$\R{r}$?''.

This is the basic fact of life in engineering: There is a price to pay for everything, and there are trade-offs to be made.

% At this point, while it is clear what these objects have to do with engineering design, it is not clear why these should be called design \emph{problems}. The answer will come later in the paper; for now, we anticipate that these objects will be very useful to represent and solve answer certain optimization problems typical of engineering design that can be stated in terms of maximizing functionality and minimizing resources.

The monotonicity of $\adp$ represents the two assumptions:

\begin{compactenum}
\item If $\F{f}$ is feasible with $\R{r}$, then any $\F{f'} \leq_\F{F} \F{f}$ is feasible with $\R{r}$.
\item If $\F{f}$ is feasible with $\R{r}$, then $\F{f}$ is feasible with any $\R{r'} \geq_\R{R} \R{r}$.
\end{compactenum}

\begin{example}
Imagine a truck to be driving at constant speed on a straight street. If it can cover 100 miles with 5 gallons of gasoline, it can also cover 80 miles with it. Furthermore, it will be able to cover the 100 miles also with 10 gallons of gasoline.
\end{example}

A design problem $\adp$ will satisfy these conditions if and only if it is represented by a monotone function.  % Crucially, we took $F\op$ instead of $F$ in the first argument of $\adp$.

We define the \emph{feasible set} $K_\adp$ as the subset of $\F{F} \op\times \R{R}$ for which~$\adp$ is the indicator function:
\begin{equation}
    K_\adp = \{ \tup{\F{f^*},\R{r}} \in \F{F} \op\times \R{R}  \mid
            \adp(\F{f}^*, \R{r}) = \true
                \}.
\end{equation}

The set $K_\adp$ is always an upper set. In fact, another way to define a design problem is to declare it as ``an upper
set in $\F{F}\op \times \R{R}$''. This is simpler than declaring it as ``a monotone map to $\Bool$''. However, the
definition as a monotone map will lend very easily to further generalization. The Boolean-valued design problems we are considering here do not distinguish between particular implementations; they only tell us if \emph{any} implementation or solution exists for given functionality and resources. We will define $\Set$-enriched design problems in \cref{sec:enriched}, which directly generalize Boolean-valued design problems and do distinguish between particular implementations.

\paragraph{Diagrammatic notation}

We represent design problems using a diagrammatic notation. One design problem $\adp \colon \F{F} \tickar \R{R}$ is represented as a box with functionality $\F{F}$ on the \emph{left} and resources $\R{R}$ on the \emph{right}:
\begin{equation*}
\includesag{50_diagrammatic}
\end{equation*}
It is inevitable that we will connect these diagrams.
\begin{example}An aerospace company, Jeb's Spaceship Parts, is designing a new rocket engine, the Bucket of Boom X100. The engine requires fuel and provides thrust, and so it can be modeled as a design problem where~$\text{Thrust}$ and~$\text{Fuel}$ are two totally-ordered sets representing their respective resources.

The corresponding diagram is
\begin{equation*}
	\includesag{50_engine}
\end{equation*}

Concretely, ``engine'' is represented as a monotone map
\begin{equation}
    \text{engine} \colon \F{\text{Thrust}}\op \times \R{\text{Fuel}} \toinPos \Bool.
\end{equation}

Assuming that the posets \R{Fuel}, \F{Thrust}$\op$ are finite, we can think of the ``engine'' design problem as a matrix, where each $(i,j)$-th entry is the answer to the question, ``is the amount of thrust $f_i$ feasible with the amount of fuel $r_j$?'':

\begin{equation}
\begin{blockarray}{ccccccc}
&&&& \text{Fuel} \\
 && r_1 = 0  & r_2 & r_3 & \hdots & r_m \\
\begin{block}{cc(ccccc)}
  & f_n^* = 0 & 0 & 0 & 0 & & 0 \\
  & f_{n-1}^* & 0 & 0 & 0 & & 1 \\
  \text{Thrust}\op & f_{n-2}^* & 0 & 1 & 1 & & 1 \\
  & \vdots &  &  &  & \ddots & \\
  & f_1^* & 1 & 1 & 1 & & 1 \\
\end{block}
\end{blockarray}
\end{equation}

Suppose we have tested or are given the performance data of a few different engines, i.e., possible solutions to the ``engine'' design problem, each with a fixed optimal fuel-thrust value. To illustrate the monotonicity assumption, we can render the data of ``engine'' as a graph:
%
\begin{equation*}
\centering
\includesag{50_engine_graph}
\end{equation*}
%
where the shaded regions cover the feasible solution set. This feasible solution set is always an \emph{upper set} (\cref{def:upperset}) in Thrust$\op \times$ Fuel, which is another way of characterizing the monotonicity of the design problem. The optimal solutions, indicated by dots, form an \emph{antichain} of solutions. We will come back to antichains in \cref{sec:computation}, when discussing how to compute optimal solutions of design problems.
\end{example}



%\JT{Do we need 2 large examples, or can we move the battery example to the notes?}
%
%\newcommand{\cCapacity}{\text{Capacity}}
%\newcommand{\cMass}{\text{Mass}}
%
%\begin{example}[Battery]
%   A battery is a store of elecrical energy.
%   We are interested in the relation between the capacity of the battery, measured
%   in joules, and the mass of the battery, measured in grams.
%   We will model the battery as a design problem
%   \[
%       \text{battery} : \cCapacity \tickar \cMass,
%   \]
%   with the two posets $\cCapacity$ and $\cMass$ defined as
%   $\cCapacity \doteq \langle\mathbb{R}_+^{\text{J}}, \leq\rangle$
%   and $\cMass \doteq \langle\mathbb{R}_+^{\text{g}}, \leq\rangle$, where ``$\leq$''' is the usual
%   order on $\mathbb{R}_+$.
%
%This is the corresponding diagram:
%   \[
%   \centering
%   \begin{tikzpicture}[oriented WD, bb min width =1.5cm, bby=2ex, bbx=.7cm,bb port length=3pt]
%       \node[bb port sep=0.8, bb={1}{1}, bb name={battery}] (dp) {};
%       \node [black, left = 0.2 of dp] {$\cCapacity$};
%       \node [black, right = 0.2 of dp] {$\cMass$};
%   \end{tikzpicture}
%   \]
%   The concrete representation of the design problem is
%   \[
%       \text{battery} \colon\cCapacity\op\times \cMass \toinPos \Bool,
%   \]
%%
%   In this case, the feasibility question is
%   \[
%           \text{battery}(c^*, m) = \true \quad \Leftrightarrow  \quad
%            \text{a battery of mass $m$ is sufficient to provide the capacity $c^*$.}
%   \]
%   It is easy to convince oneself that this ``$\text{battery}$'' is monotone from basic physics consideration.
%   Monotonicity is equivalent to the following two assertions:
%%
%   \newcommand{\qsmall}[1]{{\color{blue}#1}}
%   \newcommand{\qlarge}[1]{{\color{red}#1}}
%%
%\begin{enumerate}
%\item We can provide a smaller capacity with the same mass:
%   \begin{eqnarray}
%       \text{For all}\  \qlarge{c_2}  \geq_{\cCapacity} \qsmall{c_1},\quad
%       \text{a battery of mass $m$ is sufficient to provide the capacity $\qlarge{c_2}$} \\
%       \Rightarrow
%       \text{a battery of mass $m$ is sufficient to provide the capacity $\qsmall{c_1}$.}
%   \end{eqnarray}
%%
%\item A battery of larger mass can provide the same capacity:
%   \begin{eqnarray}
%       \text{For all}\ \qlarge{m_2} \geq_{\cMass} \qsmall{m_1},\quad
%       \text{a battery of mass $\qsmall{m_1}$ is sufficient to provide the capacity $c^*$} \\
%       \Rightarrow
%       \text{a battery of mass $\qlarge{m_2}$ is sufficient to provide the capacity $c^*$.}
%   \end{eqnarray}
%\end{enumerate}
%
%   Assume that there is a linear relation between mass and capacity,
%   and such relation is described by the energy density~$\rho$ [Wh/kg].
%   Then the minimum mass to provide~$c^*$ is~$m_\text{min} = m_0 + c^* / \rho.$
%   So we have
%   \[
%       \text{battery}(c^*, m) = \true \quad \Leftrightarrow \quad  m \geq m_0 + c^* / \rho.
%   \]
%   A visualization of the feasible set $\feasibleset{\text{battery}}$ is in \cref{fig:battery-1}.
%   Monotonicity means that if we fix one point $\tup{c^*,m}$,
%   if we increase $c^*$, we can only see at most one transition, from feasible to unfeasible.
%   Similarly, if we increase $m$, there is at most one transition from unfeasible to feasible.
%
%\end{example}
%
%\begin{figure}[h!]
%    \todo{figure battery-1}
%    \caption{\label{fig:battery-1}}
%\end{figure}


\subsection{Series composition of design problems}

We will define several ways to connect and compose design problems. The first and most basic way is the series composition, or just `composition'.

\begin{definition}[Series composition]
\label{def:dp-series}
Given two design problems $f \colon  \F{A} \tickar \R{B}$ and $g \colon \F{B} \tickar \R{C}$, we define their \emph{series composition}
$(f\then g)\ \colon  \F{A} \tickar \R{C}$
as
\begin{equation}
\label{eq:composition2}
\begin{aligned}
   (f \then g)  \colon \ \F{A}\op \times \R{C} & \toinPos  \Bool, \\
  \tup{\F{a}^*, \R{c}} &\mapsto \bigvee_{b\in B} f(\F{a}^*,b) \wedge g(b^*,\R{c}).
\end{aligned}
\end{equation}
or
\begin{equation}
\begin{aligned}
\label{eq:composition}
   (f \then g)\  \colon \ \F{A}\op \times \R{C} & \toinPos  \Bool,  \\
  \tup{\F{a}^*, \R{c}} &\mapsto \bigvee_{\R{b_1}\leq\F{b_2}, \R{b_1},\F{b_2}\in B} f(\F{a}^*,\R{b_1}) \wedge g(\F{b_2}^*,\R{c}).
\end{aligned}
\end{equation}
\end{definition}

\begin{lemma}
\label{lemma:composition_equivalency}
\cref{eq:composition2} and \cref{eq:composition} are equivalent.
\end{lemma}

\begin{proof}
First consider the direction \cref{eq:composition} $\implies$ \cref{eq:composition2}. In order for $\bigvee_{\R{b_1}\leq\F{b_2}, \R{b_1},\F{b_2}\in B} f(\F{a}^*,\R{b_1}) \wedge g(\F{b_2}^*,\R{c})$ to be true, there should exist some $\R{b_1}\leq \F{b_2}$ for which $f(\F{a}^*,\R{b_1}) \wedge g(\F{b_2}^*,\R{c})$ is true. However, due to the monotonicity of $f$, $f(\F{a}^*,\R{b_2}) \wedge g(\F{b_2}^*,\R{c})$ and \cref{eq:composition2} must be true as well. On the other hand, if $\bigvee_{\R{b_1}\leq\F{b_2}, \R{b_1},\F{b_2}\in B} f(\F{a}^*,\R{b_1}) \wedge g(\F{b_2}^*,\R{c})$ is false, then due to the equality, it is false for any $\R{b_1}=\F{b_2}$, and therefore all inner terms of \cref{eq:composition2} must be false as well.

The other direction, \cref{eq:composition2} $\implies$ \cref{eq:composition}, can be shown in a similar way. If \cref{eq:composition2} is true, then there must exist a $b'$ such that $f(\F{a}^*, \R{b'})=\true$ and $g(\F{b'}^*, \R{c})=\true$. Then, the inner term in \cref{eq:composition} is true for $\R{b_1}=\F{b_2}=b'$. If \cref{eq:composition2} is false, then there is no such $b'$ for which both $f(\F{a}^*, \R{b'})$ and $g(\F{b'}^*, \R{c})$ are true, but then due to the monotonicity of $f$ and $g$ they also cannot be true for any $\R{b_1} \leq \F{b_2} = b'$ or $b'=\R{b_1} \leq \F{b_2}$. Hence, \cref{eq:composition} must also be false.
\end{proof}
\begin{remark}
At first sight, \cref{eq:composition} might seem like a more verbose version of \cref{eq:composition2}. However, assume that we have the means to obtain the minimal antichain of the feasible set of resources that provide $\F{a}$ for the first term:
\begin{equation}
    \R{B_f} = \Min \{b\in B \mid f(\F{a}^*,\R{b})=\true\} \in \antichains B.
\end{equation}
This represents the minimal resources with which $f$ can provide $\F{a}$.

Assume further that we similarly have the means to obtain the maximal antichain of the feasible set of functionalities that $\R{c}$ provides for the second term:
\begin{equation}
    \F{B_g} = \Max \{b\in B \mid g(\F{b_2}^*,\R{c})=\true\} \in \antichains B,
\end{equation}
which represents the maximal functionality that $g$ can provide given $\R{c}$.

Then, \cref{eq:composition} implies that it suffices to only evaluate
\begin{equation}
    \bigvee_{\substack{\R{b_1}\leq\F{b_2} \\ \R{b_1}\in \R{B_f},~ \F{b_2}\in\F{B_g} }} f(\F{a}^*,\R{b_1}) \wedge g(\F{b_2}^*,\R{c}),
\end{equation}
which can be much more efficient than iterating over all $b\in B$.
\end{remark}


\paragraph{Intended semantics}
The series interconnection $f\then g$ judges a pair $\tup{\F{a}^*,\R{c}}$ as feasible if and only if there exists a $b \in B$ such that $f(\F{a}^*,\R{b})$ and $g(\F{b}^*,\R{c})$ are feasible.\footnote{In \eqref{eq:composition2} we could have written ``$\exists_{b\in B}$''
instead of ``$\bigvee_{b\in B}$''; the latter form highlights the connection
with operations on matrices. Given a set $I$ and a map $s\colon I\to\Bool$, we can define the boolean $\bigvee_{i\in I}s(i)$ by
\begin{equation*}
 \bigvee_{i\in I}s(i)\coloneqq
 \begin{cases}
 	\true&\text{ if there exists }i\in I\text{ for which }s(i)=\true,\\
 	\false&\text{ if there exists \emph{no} }i\in I\text{ for which }s(i)=\true.
 \end{cases}
 \end{equation*}
 For any $I$, if we have $i_0\in I$ then $s(i)\leq\bigvee_{i\in I}s(i)$. One can also check that for any $b\in\Bool$ or even set of booleans $t\colon J\to\Bool$, we have
 \begin{equation*}
     \bigvee_{i\in I}(b\wedge s(i))=b\wedge\left(\bigvee_{i\in I}s(i)\right)
 \quad\text{and}\quad
 \bigvee_{(i,j)\in I\times  J}\big(s(i)\wedge t(j)\big)=\left(\bigvee_{i\in I}s(i)\right)\wedge\left(\bigvee_{j\in J} t(j)\right)
 \end{equation*}
}

\begin{example}
After the Bucket of Boom X100 blew up on re-entry, Jeb's Spaceship Parts is building the X101. This time, they are making sure to take into account other aspects of the rocket design, such as the choice of propellant and nozzle.
\[
\includesag{50_X101}
\]
\end{example}

\begin{remark}
When viewing compositions (and larger diagrams) formed from these boxes, it is tempting to interpret the boxes as input-output processes. However, that would be misleading. The arrows do not represent information flow, materials flow, or energy flow. Design problems do not represent input-output processes but rather a static calculus of requirements---a requirements flow.
\end{remark}

Let us check that given design problems $f$ and $g$, $f\then g$ is in fact a design problem.
\begin{lemma}
    Composition as in~\cref{eq:composition2} is monotone in $a$ and $c$.
\end{lemma}
\begin{proof}
 We need to show that $[f\then g](a^*,c)$ is monotone in $a^*$ and $c$. This is true
 because it is a composition of monotone maps. Because~$f$ represents a design problem, $f(a^*,b)$ is monotone in~$a^*$,
 and similarly $g(b^*,c)$  is monotone in $c$. The conjunction ``$\wedge$'' is monotone in both variables, and likewise the $\vee$ operation.
 \end{proof}

We represent series in the diagrammatic notation.

\begin{equation}
\includesag{50_series_diag}
\end{equation}

We can show two important properties for the ``$\then$'' operation: Associativity and unitality.

\begin{lemma}
    The composition operation is associative, i.e. $(f\then g)\then h = f\then (g\then h)$.
\end{lemma}

\begin{proof}
To show that the operation is associative, we can use distributivity and commutativity in $\Bool$:
%
\begin{equation}
\begin{aligned}
\left[(f\then g) \then h\right] (\F{a}^*,\R{d})
&= \bigvee_{c \in C} \left (\ \bigvee_{b\in B } f(\F{a}^*,\R{b}) \wedge g(\F{b}^*,\R{c}) \right )  \wedge  h (\F{c}^*, \R{d}) \\
&= \bigvee_{c \in C} \left (\ \bigvee_{b\in B } f(\F{a}^*,\R{b})
\wedge g(\F{b}^*,\R{c}) \wedge h (\F{c}^*, \R{d})
    \right ) \\
&= \bigvee_{b \in B } f(\F{a}^*,\R{b}) \wedge \left ( \bigvee_{c\in C} g(\F{b}^*,\R{c}) \wedge h (\F{c}^*, \F{d}) \right ) = [f \then (g \then h )] (\F{a}^*, \R{d}).
\end{aligned}
\end{equation}
%
\end{proof}

Because of associativity, we can write~$f\then g\then h$ without ambiguity.
Associativity of composition is represented with the following diagram equation:


\begin{equation}
    \includesag{50_assoc_1_2_3}
\end{equation}

There exists an identity for the ``$\then$'' operation.
We define the identity $\id_A\colon A \tickar A$ as follows.

\begin{definition}[Identity]
\label{def:dp-identity}
For any poset $A$, the \emph{identity} $\id_A\colon \F{A} \tickar \R{A}$ is a map
\begin{equation}
\begin{aligned}
  \id_A \colon \F{A}\op \times \R{A} & \toinPos   \Bool, \label{eq:identity}\\
  \tup{\F{a_1}^*, \R{a_2}} & \mapsto \F{a_1} \leq_{A} \R{a_2}.
\end{aligned}
\end{equation}
\end{definition}

In the diagrammatic notation, we represent $\id_A$ by
\todo{discuss with AC}
    \begin{center}
    \begin{tikzpicture}[DP]
    \node[dp={1}{1}] (id) {$\id_A$};
    \draw[runconn, runame={}] (id_res1){};
    \draw[funconn, funame={}] (id_fun1){};
    \draw (id_fun1) to[in=180,out=0] (id_res1){};
\end{tikzpicture}
    \end{center}
since a design problem composed with the identity is equivalent to the original design problem.

\begin{lemma}
    The composition operation satisfies the left and right unit laws, i.e.
\begin{equation}
    \includesag{50_composition_unitality}
\end{equation}
\end{lemma}
\begin{proof}
Given $f\colon A\tickar B$, we need to show:
\begin{equation}
    \id_A \then f = f = f\then \id_B.
\end{equation}
In the following, we prove $\id_A \then f = f$. Proving $f\then \id_B=f$ is similar.
Consider the poset $\Bool$. Since for $x,y\in \Bool$, $x\cong y \Rightarrow x=y$ (also known as skeletality), we just need to show that $f\leq \id_A\then f$ and $\id_A\then f\leq f$.
We have
\begin{equation}
\begin{aligned}
        f(a,b)&=\true \wedge f(a,b)\\
        &\leq \id_A(a,a) \wedge f(a,b)\\
        &\leq \bigvee_{a'\in A}\id_A(a,a')\wedge f(a',b)\\
        &=(\id_A\then f)(a,b).
\end{aligned}
\end{equation}
For the other direction, we need to show that $\id_A\then f\leq f$, i.e.
\begin{equation}
    \bigvee_{a'\in A}\id_A(a,a')\wedge f(a',b) \leq f(a,b).
\end{equation}
This holds iff $\id_A(a,a')\wedge f(a',b) \leq f(a,b)$ for some $a'\in A$. If there is no such $a'$, then the inequality holds ($\false \leq \false$ and $\false \leq \true$). If there is such an $a'$, it means than $\id_A(a,a')=\true$ and $f(a',b)=\true$. We know that $\id_A(a,a')=\true \Leftrightarrow a\leq a'$, and hence $f(a,b)=\true$.
\end{proof}

\subsection{The category of design problems}

We claim and prove that the class of all design problems forms a category, called $\DP$.

\begin{definition}[Category $\DP$]
\label{define:DP}
The \emph{category of design problems, $\DP$}, consists of the following data:
%
\begin{compactenum}
\item \emph{Objects}: The objects of $\DP$ are posets.
\item \emph{Morphisms}: The morphisms of $\DP$ are design problems (\cref{def:design-problem}).
\item \emph{Identity morphism}: The identity morphism $\id_A \colon \F{A} \tickar \R{A}$ is given by \cref{def:dp-identity}.
\item \emph{Composition operation}: Given two morphisms $f \colon  \F{A} \tickar \R{B}$ and $g \colon \F{B} \tickar \R{C}$, their
composition $(f\then g)\colon  \F{A} \tickar \R{C}$ is
given by \cref{def:dp-series}.
\end{compactenum}
\end{definition}

\begin{lemma}
    $\DP$ is a category.
\end{lemma}
\begin{proof}
We have already shown that the composition operator ``$\then$'' is associative and unital, and that the composition of two design problems is a design problem (closure). Therefore, $\DP$ is a valid category.
\end{proof}