% !TEX root = ../CategoricalCoDesign.tex
\section{Enrichments}
\label{sec:enriched}
A design problem in ``enriched in $\Bool$'' answers the question, ``is it feasible to provide a given functionality $f$ with resources $r$?''.  A design problem ``enriched in $\Set$'' answers the question, ``which implementations provide $f$ with $r$?''. A design problem ``enriched in $\DP$'' answers the question, ``which design problems provide $f$ with $r$?'' If compact closure allows us to zoom out by studying design problems of design problems, then enrichment allows us to zoom in by studying the `subatomic' composition of design problems.

\begin{example}\label{ex:dpi_example}
Beau, the NASA engineer, stares hard at his screen. His boss, Elly May, only wants to know whether the rocket will fly, but he's the one that has to deal with all potential suppliers and technologies. Since NASA started letting any podunk junkyard company bid on rocket contracts, the number of potential parts had sky\emph{rocketed}. Intuitively, he wants to keep track of this extra parts information, so he defines, for every design problem $\adp : F \tickar R$, an extra set of implementations $I_\text{adp}$ and two functions prov $: I_\text{adp} \to F$ and req $: I_\text{adp} \to R$; we say that $i$ provides at most $f \in F$ and $i$ requires at least $r \in R$. Now the familiar $\text{engine} : \text{Thrust} \tickar \text{Fuel}$ design problem looks like this:
\begin{align*}
\text{engine} : \textF{Thrust}\op \times \textR{Fuel} &\toinPos \powerset(I_\text{engine}) \\
\tup{\F{t}^*,\R{f}} &\mapsto \{ i \in I_\text{engine} : \text{exec}(i) \leq_\text{Thrust} t, \text{eval}(i) \geq_\text{Fuel} f \}
\end{align*}
Beau is feeling pretty proud of himself. Now any time NASA wants a design to be built, he can just press a button and the computer will spit out all the parts needed!
\end{example}

In order to sharpen the intuition of the above example (implementation spaces of the sort above were actually implemented in \cite{censi}), we will need to learn more about the theory of enriched categories.

Some preliminaries.

Given that $\DP$ is compact closed and the external hom is a poset, it is a short, visual step to seeing that $\DP$ is also what we call a \emph{2-category}: a category where the (external) hom-object between any two objects is itself a category (remember that a poset is a kind of category, from \cref{???}). In other words, a 2-category is a category endowed with the extra structure of ``morphisms between morphisms''.

Another way of defining a 2-category is to say that it is ``enriched in $\Cat{Cat}$'', the category of categories, whose objects are (small) categories and morphisms are functors. More generally, to say that a category $\Cat{C}$ is \emph{enriched in $\Cat{D}$}, where $\Cat{D}$ is a monoidal category, is to say that $\Hom_\Cat{C}(A,B) \in \Cat{D}$ for all $A,B \in \Cat{C}$, and that composition of morphisms in $\Hom_\Cat{C}(A,B)$ respects the rules of composition in $\Cat{D}$.

\subsection{Enriched categories}

\begin{shaded}
\begin{definition}[Enriched category]
We say that a category \CatC is enriched in \CatD if:
\begin{compactenum}
    \item For all objects $x, y$ of \CatC, the set $\HomC(x,y)$ can be considered an object of \CatD;
    \item \CatD is a monoidal category (\cref{def:monoidal}),
    with monoidal product~$\otimesD$;
    \item For all objects $x, y, z$ of \CatC, there exists
    a certain morphism $m_{x,y,z}$ in \CatD,
    which goes from the object $\HomC(x, y) \otimesD \HomC(y, z)$ to the object $\HomC(x, z)$:
    \begin{equation} \label{eq:enriched-condition}
        m_{x,y,z} :
            \HomC(x, y)
            \otimesD
            \HomC(y, z)
            \to
            \HomC(x, z) .
    \end{equation}
\end{compactenum}
\end{definition}
\end{shaded}

% A locally-posetal 2-category is also known as a category enriched over posets.
%\begin{definition}\label{def:loc_pos_cat}
%A \emph{locally-posetal 2-category} is a category $\cat{C}$, with an additional pre-order structure $\leq$ on the hom-set $\cat{C}(p,q)$, for every two objects $p,q\in\cat{C}$, which additionally satisfies the condition that if $f\leq g$ in $\Hom(p,q)$ and $j\leq k$ in $\Hom(q,r)$, then $f;j\leq g;k$ in $\Hom(p,r)$.
%\end{definition}
%
%% The following is obvious, but we record it anyway.
%\begin{lemma}\label{lemma:loc_pos_op}
%If $\cat{C}$ is locally posetal, then so is $\cat{C}\op$.
%\end{lemma}

\begin{example}\label{ex:enrichbool}Every poset (as a category) is enriched in $\Bool$, since between any two elements $a,b$ of the poset, either the morphism $a \leq b$ exists ($\Hom_A(a,b) = \true$) or it does not ($\Hom_A(a,b) = \false$).\end{example}

\begin{example}The poset $\Bool$ is enriched in $\Bool$.\end{example}

\begin{example}The category $\Pos$ of posets is enriched in $\Pos$, where the partial order on monotone maps is given by $f \Imp g$ (i.e. for $f,g : A \to B$, $f(a) \leq g(a) \forall a \in A$).\end{example}

%\begin{proposition}\label{prop:Pos_loc_pos}
% Suppose that $f,g\colon\cP \to\cQ $ and $j,k\colon\cQ \to\cR $ are monotone maps with $f\Imp g$ and $j\Imp k$. Then $f;j\Imp g;k$ in $\Poset$.
% \[
% \begin{tikzcd}
% 	\cP \ar[r, bend left=35pt, "f"]\ar[r, bend right=35pt, "g"']\ar[r,phantom, "\Downarrow"]
% 	&\cQ \ar[r, bend left=35pt, "j"]\ar[r, bend right=35pt, "k"']\ar[r,phantom, "\Downarrow"]
% 	&\cR
% \end{tikzcd}
% \]
% \end{proposition}
% \begin{proof}
% Assume $f\Imp g$ and $j\Imp k$, and choose $p\in\cP $; we want to show that $j(f(p))\leq_R k(g(p))$. We have $f(p)\leqQ  g(p)$ by assumption and since $j$ is monotone, we have $j(f(p))\leq_R j(g(p))$. But since $j\Imp k$, we also have $j(g(p))\leq_R k(g(p))$, and the result follows by transitivity.
% \end{proof}
%
%With the above, we have shown that the category $\Poset$ actually has the structure of \emph{locally-posetal 2-category}.

\begin{example}$\Cat{Cat}$ is enriched in $\Cat{Cat}$.\end{example}

\begin{example}Every category is enriched in $\Set$.\end{example}



\subsubsection{Examples of enriched categories}
\begin{example}
    A poset~$\tup{P, \leq}$ can be consider a category enriched in the
    category~$\Bool$.

    First, recall the construction that makes each poset into a
    category~(\cref{sec:poset-as-category}).

    The poset~$P$ as a category is a category with the objects being the
    elements of~$P$, and with a morphism~$f: x \to y$ existing if and only if~$x
    \leq y$.

    $\Bool$ as a category contains two elements,~$\true$ and~$\false$, with
    the three morphism~$\false \to \true$, $\true \to \true$, and~$\false \to \false$. This is equivalent to say that there is a morphism between~$a, b \in \Bool$ if and only if~$a \Rightarrow b$. So we can set~$\Rightarrow\ \equiv\ \to_{\Bool}$.

    $\Bool$ can be also considered a monoidal category, by letting~$\otimes$ be
    the \emph{and} operation, so that~$a \otimes b\ =\ a \wedge b$.

    Looking at~$P$ again, we can show how it can be considered a category
    enriched in~$\Bool$. For any two points~$a, b$ of~$P$, either~$a \leq b$, or
    not; there are two choices. The hom-set~$\Hom(a, b)$ is either non-empty
    (if~$a \leq b$) or empty (if~$a \not\leq b$). We can make the correspondence
    that an empty hom-set corresponds to~$\false$ and a non-empty hom-set
    corresponds to~$\true$.

    Now we can verify that the condition~\eqref{eq:enriched-condition} holds. We
    know that, in a poset,~$x \leq y$ and~$y \leq z$ implies~$x \leq z$.
%
    Rewritten in the language of categories, this is:
    \begin{equation}
        (\Hom(x,y)\ \text{non-empty})
        \wedge
        (\Hom(y, z)\ \text{non-empty})
        \quad
        \Rightarrow
        \quad
        \Hom(x, z)\ \text{non-empty}.
    \end{equation}
    By making the identification~$\Rightarrow\ \equiv\ \to_{\Bool}$ and~$\wedge\ \equiv\ \otimes_{\Bool}$, we can rewrite the above as
    \begin{equation}
        (\Hom(x,y)\ \text{non-empty})
        \otimes_{\Bool}
        (\Hom(y, z)\ \text{non-empty})
        \quad
        \to_{\Bool}
        \quad
        \Hom(x, z)\ \text{non-empty},
    \end{equation}
    This is the specialization of~\eqref{eq:enriched-condition}
    for~\CatC a poset and~\CatD equal to~\Bool.
\end{example}

\todo{describe somewhere that \Set is a monoidal category}

\begin{example}
A category ``enriched in \Set'' is just a regular category, as we have defined it.

Recall that~\Set is a monoidal category with $\otimes_{\Set}$ equal to the Cartesian product $\times$.

Take an arbitrary category~\CatC. For all~$x, y$ in~\CatC, we know by definition
that~$\Hom(x,y)$ is a set.

Consider three objects~$x,y,z$ in~\CatC. We know from the definition of a
category that if there exists a morphism~$f: x \to y$ and another~$g: y \to z$,
then there also exists~$(f\then g)\colon x \to z$.

Consider now the set of all morphisms between $x, y$, given by~$\Hom(x,y)$, and
all morphisms between~$y$ and~$z$, given by~$\Hom(y, z)$. What is the relation
between those hom-sets and~$\Hom(x, z)$?

It is not quite true that~$\Hom(x, z)$ is the Cartesian product~$\Hom(x, y)
\times \Hom(y, z)$. If there are~$m$ morphisms between~$x$ and~$y$, and~$n$
morphisms between~$y$ and~$z$, there are not necessarily~$m \cdot n$ morphisms
from~$x$ to~$z$, because we are not guaranteed that all the compositions~$(f\then g)$
will be distinct morphisms.

What we are guaranteed is that all of the compositions will be mapped to something in~$\Hom(x, z)$; or, in other words, we are guaranteed that there
is a map $\phi$ of the type
\begin{equation}
    \phi \colon \Hom(x, y) \times \Hom(y, z) \rightarrow \Hom(x, z).
\end{equation}
This $\phi$ is a morphism in \Set, and it is the witness required by~\eqref{eq:enriched-condition}.

\end{example}



\subsection{Set-enriched DPs (DPI)}

\begin{proposition}\label{prop:DP_loc_pos}
$\DP$ is enriched in $\Pos$. Equivalently, it is a locally-posetal 2-category: a 2-category where the hom-categories are also posets.
\end{proposition}
\begin{proof}
Suppose that $f,g\colon\cP \tickar\cQ $ and $j,k\colon\cQ \tickar\cR $ are design problems and that $f\Imp g$ and $j\Imp k$. We need to show that $(f\then j)\Imp(g\then k)$ in $\DP$.
\begin{equation}
\includesag{77_dploc}
\end{equation}
Assume $f\Imp g$ and $j\Imp k$, and choose $p\in\cP $ and $r\in\cR $ such that $(f\then j)(p^*,r)=\true$; we must show that $(g\then k)(p^*,r)=\true$. By assumption, there is some $q\in Q$ such that $f(p^*,q)\wedge j(q^*,r)=\true$. But then also $g(p^*,q)\wedge k(q^*,r)=\true$, and this implies the result.
\end{proof}

Since posets are also categories, monotone maps between posets can be defined as functors between posets (qua categories). Further, posets are enriched in $\Bool$ (\cref{ex:enrichbool}), i.e. $\Hom_A(a,b) \in \Bool$ for all $A \in \Pos$. So when we say that a design problem $f$ is ``enriched in $\Bool$'', we mean that is not only a functor $f : A \tickar B$ between posets $A\op \times B$ and $\Bool$, but that its action on hom-sets is always a morphism in $\Bool$. Explicitly, $f : \Hom_{A\op \times B}(\tup{a_1^*,b_1}, \tup{a_2^*,b_2}) \to \Hom_\Bool(f(\tup{a_1^*,b_1}), f(\tup{a_2^*,b_2}))$ is a morphism in $\Bool$---in other words, one among the set $1 \leq_\Bool 1, 0 \leq_\Bool 0$, and $0 \leq_\Bool 1$ familiar from \cref{ex:bool}.

\begin{shaded}
\begin{definition}[Enriched functor]
\label{defn:enrichedfunctor}
Given two categories \CatC and \CatD enriched in the same monoidal category \CatV, an enriched functor $F: \CatC \to \CatD$ consists of:
\begin{compactenum}
\item A map $F\colon \ObC \to \ObD$ that maps objects of~\CatC to objects of~\CatD.
\item For each $x, y$ in $\ObC$, there exists a morphism in \CatV given by
\begin{equation}
    F_{x,y} \colon \HomC(x,y) \to \HomD(F(x), F(y)),
\end{equation}
such that composing maps ``across $F$'' respects the composition in \CatC and the unit in \CatV in the obvious ways:
    \begin{equation}
        \includesag{77_enriched_functor}
    \end{equation}
and
    \begin{equation}
        \includesag{77_enriched_functor_2}
    \end{equation}
where $\otimes$ and $\One$ are the monoidal product and monoidal unit in \CatV.
\end{compactenum}
\end{definition}
\end{shaded}

\begin{proposition}
Design problems in $\DP$ are Boolean-enriched profunctors.
\end{proposition}
\begin{proof}
In fact, we will show that any monotone map $f : A \toinPos B$ between posets is a Boolean-enriched functor. Then a design problem is simply a Boolean-enriched functor that happens to be a profunctor.

Monotonicity of $f$ signifies that $a_1 \leq_A a_2$ implies $f(a_1) \leq_B F(a_2)$. Rewriting this condition considering $A, B$ as categories enriched in $\Bool$, we have that
\begin{equation}
\Hom_A(a_1, a_2) \quad \Rightarrow \quad \Hom_B(f(a_1), f(a_2)).
\end{equation}
which can be re-stated as proclaiming the existence of a morphism $f_{a_1, a_2}$ in \Bool:
\begin{equation}
f_{a_1, a_2} \colon \Hom_A(a_1, a_2) \to_{\Bool} \Hom_B(f(a_1), f(a_2).
\end{equation}
It remains to check the diagrams in \cref{defn:enrichedfunctor}. The first is trivial since $a \leq_A a$ is true by definition in all posets:
\begin{equation}
\id_{a} \then f_{a,a} (\ast) = \id_{f(a) = \true}.
\end{equation}
The second is also relatively trivial: the monoidal product is just the conjunction $\wedge$ in $\Bool$, so commuting with the composition in the poset can be thought of as a verification of the monotonicity of $f$ via the natural $\Imp$ inside $\Bool$:
\begin{equation}
\begin{aligned}
(a \leq b) \wedge (b \leq c) & \quad \Rightarrow & (a \leq c) \\
\Downarrow & & \Downarrow \\
(f(a) \leq f(b)) \wedge (f(b) \leq f(c)) & \quad \Rightarrow & (f(a) \leq f(c))
\end{aligned}
\end{equation}
\end{proof}

Having defined morphisms in $\DP$ as Boolean-enriched design problems, we can directly generalize these Boolean-enriched design problems to $\Set$-enriched design problems in a new category, $\DPI$.

\begin{definition}[Category of $\Set$-enriched design problems]
The category of $\Set$-enriched design problems, $\DPI$, consists of the following data:
\begin{enumerate}
\item The objects of $\DPI$ are posets.
\item A morphism of $\DPI$ $\adp : F \tickar R$ can be represented as a tuple of the form $(I_\adp, \provides, \requires)$, where $I_\adp$ is a set, $\provides$ and $\requires$ are functions from $I_\adp$ to $F$ and $R$ respectively:
\begin{equation}
    F \xleftarrow{\provides} I \xrightarrow{\requires} R.
\end{equation}
from which we define $\adp : F \tickar R$ as a $\Set$-enriched profunctor as follows:
\begin{equation}
\begin{aligned}
\adp : F\op \times R &\toinPos \powerset(I_\adp) \\
\tup{f^*, r} &\mapsto \{ i \in I_\adp : (\provides(i) \leq_F f) \wedge (\requires(i) \geq_R r) \}.
\end{aligned}
\end{equation}
where the partial order on $\powerset(I)$ is given by subset inclusion.
 The interpretation is that the result of $\adp(f^*, r)$ is the set of ways to provide $f$ using $r$.
 We call $I$ the set of implementations.
 %Note that when $I_\adp = \{\ast\}$, $\DPI = \DP$.
%Where it is clear, we will also refer to these morphisms as design problems.% The monotonocity of a design problem $\adp$ in $\DPI$ describes a similar feasibility condition as in $\DP$: ...

\item The identity morphism $\id_A : A \tickar A$ is given by implementation set $I_{\id_A} = A$ and $\provides=\requires$ being
the identity on $A$. The profunctor is defined as
\begin{equation}
\begin{aligned}
\id_A : A\op \times A &\toinPos \powerset(A) \\
\tup{a^*,a'} &\mapsto (\uparrow a) \cap (\downarrow a')
\end{aligned}
\end{equation}

\item Given two morphisms $f\colon A \tickarimp{I_1, \provides_1, \requires_1} B$
and $g\colon B \tickarimp{I_2, \provides_2, \requires_2} C$, we can define their
series interconnection $(f\then g)\colon A \tickarimp{I, \provides, \requires} C$ as follows.
With reference to this diagram:
\begin{equation}
    A \xleftarrow{\provides_1} I_1 \xrightarrow{\requires_1} B
      \xleftarrow{\provides_2} I_2 \xrightarrow{\requires_2} C
\end{equation}
we let the implementation space be the \emph{pullback}
\begin{equation}
\begin{aligned}
I &= I_1 \times_B I_2 \doteq \{
        \tup{i_1,i_2} \in I_1 \times I_2 :
        \requires_1(i_1) \leq_{B} \provides_2(i_2)
    \},\\
\end{aligned}
\end{equation}
and the two maps $\provides$, $\requires$ defined as:
\begin{equation}
\begin{aligned}
        \requires  &:  \tup{i_1,i_2} \mapsto \requires_2(i_2) \\
        \provides  &:  \tup{i_1,i_2} \mapsto \provides_1(i_1).
\end{aligned}
\end{equation}
In terms of the profunctors, given two morphisms $f \colon A \tickar B$ and $g \colon B \tickar C$, their composition $(f\then g) \colon A \tickar C$ is given by:
\begin{equation}
\begin{aligned}
(f\then g) \colon A\op \times C &\toinPos \powerset(I_1 \times_B I_2) \\
\tup{a^*,c} &\mapsto \bigcup_{\substack{(b,b') \in B\op \times B\op \\ b \leq_B b'}} \Bigg[ f(a^*,b) \times g(b'^*,c) \Bigg].
\end{aligned}
\end{equation}
\end{enumerate}
\end{definition}

We check the basic properties of associativity and unitality.

\begin{lemma}[associativity]
Given three design problems with implementation
    $f\colon A \tickarimp{I_f, \provides_f, \requires_f} B$
and $g\colon B \tickarimp{I_g, \provides_g, \requires_g} C$,
and $h\colon C \tickarimp{I_h, \provides_h, \requires_h} C$,
it holds that $(f\then g)\then h \simeq f\then (g\then h).$
\end{lemma}

\begin{proof}
Here's the situation:
\begin{equation}
    A \xleftarrow{\provides_f} I_f \xrightarrow{\requires_f} B
      \xleftarrow{\provides_g} I_g \xrightarrow{\requires_g} C
      \xleftarrow{\provides_h} I_h \xrightarrow{\requires_h} D.
\end{equation}

For $(f\then g)$ we compute
\begin{equation}
\begin{aligned}
I_{f\then g} &= \{
                \tup{i_f,i_g} \in I_f \times I_g \colon
                \requires_f(i_f) \leq_{B} \provides_g(i_g)
            \}\\
        \requires_{f\then g}  &\colon  \tup{i_f,i_g} \mapsto \requires_g(i_g) \\
        \provides_{f\then g}  &\colon  \tup{i_f,i_g} \mapsto \provides_f(i_f).
\end{aligned}
\end{equation}
For $(f\then g)\then h$ we compute
\begin{equation}
\begin{aligned}
        I_{(f\then g)\then h} &= \{
                \tup{i_{f\then g},i_h} \in I_{f\then g} \times I_{h}\colon
                \requires_{f\then g}(i_{f\then g}) \leq_{C} \provides_h(i_h)
            \}\\
        \requires_{(f\then g)\then h}  &\colon  \tup{i_{f\then g},i_h} \mapsto \requires_h(i_h) \\
        \provides_{(f\then g)\then h}  &\colon  \tup{i_{f\then g},i_h} \mapsto \provides_{f\then g}(i_{f\then g}).
\end{aligned}
\end{equation}
By substituting from the previous equations, we have:
\begin{equation}
\begin{aligned}
        I_{(f\then g)\then h} &= \{
                \tup{\tup{i_f, i_g},i_h} \in (I_f \times I_g) \times I_{h}\colon
                (\requires_f(i_f) \leq_{B} \provides_g(i_g))
                \wedge
                (\requires_g(i_g) \leq_{C} \provides_h(i_h))
            \}\\
        \requires_{(f\then g)\then h}  &\colon  \tup{\tup{i_f, i_g},i_h} \mapsto \requires_h(i_h) \\
        \provides_{(f\then g)\then h}  &\colon  \tup{\tup{i_f, i_g},i_h} \mapsto \provides_f(i_f).
\end{aligned}
\end{equation}
Up to isomorphism
\begin{equation}
    (I_f \times I_g) \times I_{h} \ \simeq\  I_f \times (I_g \times I_{h}),
\end{equation}
the above is exactly what we would obtain for $f\then (g\then h)$, so we can say
that  $f\then(g\then h)\ \simeq\ (f\then g)\then h$.
\end{proof}


\begin{lemma}[unitality]
\AC{todo, non trivial}
\end{lemma}


Like $\DP$, $\DPI$ is also a traced symmetric monoidal category with monoidal product $\times$ and biproduct given by $+$; we will skip the proofs for $\DPI$ since most are directly analogous to those for $\DP$. We already saw an example of $\DPI$ in \cref{ex:dpi_example} above. It remains to verify that morphisms in $\DPI$ are indeed enriched in $\Set$:

\begin{proposition}
Design problems in $\DPI$ are $\Set$-enriched profunctors.
\end{proposition}
\begin{proof}
Fix a design problem $f \colon A\op \times B \toinPos \powerset(I)$. \AC{to finish}

\end{proof}

We introduce $\DPI$ mainly as a point of comparison; enriching in $\Set$ is the most obvious, if not the most elegant, way of representing and reasoning about implementations of design problems. Indeed, the way we defined morphisms is rather clunky (note how $I_\adp$ is a set rather than a poset, and the extra provisions for prov and req in the identity morphism), and it also essentially restricts implementations $I_\adp$ to being subsets of $F\op \times R$; i.e. an implementation rigidly provides a fixed functionality $f$ with a fixed $r$.

The far more natural option, from the perspective of enriched category theory, is to enrich design problems in $\DP$ itself.

\begin{example}
\begin{equation}
\begin{aligned}
\text{engine} \colon \text{Thrust}\op \times \text{Fuel} &\toinPos \powerset(\text{Engines}) \\
\tup{t^*,f} &\mapsto \{ e \in \text{Engines} \colon e(t^*, f) = \true \}
\end{aligned}
\end{equation}
where $\powerset$ indicates the powerset operation and Engines is the hom-poset $\Hom_\DP(\text{Thrust}, \text{Fuel})$ from \cref{ex:r&dproblem}.
\end{example}


\subsubsection{Operations on DPIs}


% \begin{table}[t!]
%     \centering
% \begin{tabular}{c|c|c|crl}
%     series &
%     $f:A\tickar B$&
%     $g:B\tickar C$&
%     $f;g:$&$A$&$\tickar C$ \\
%     %
%     sum &
%     $f:A\tickar B$ &
%     $g:A\tickar B$ &
%     $f\vee g:$&$A$&$\tickar B$ \\
%     %
%     intersection &
%     $f:A\tickar B$ &
%     $g:A\tickar B$ &
%     $f\wedge g:$&$A$&$\tickar B$ \\
%     %
%     monoidal product &
%     $f:A\tickar C$&
%     $g:B\tickar D$ &
%     $f\otimes g:$&$A\times B$&$\tickar C \times D$ \\
%     %
%     product &
%     $f:A\tickar C$&
%     $g:A\tickar D$ &
%     $f\times g:$&$A $&$\tickar C + D$ \\
%     %
%     coproduct &
%     $f:A\tickar C$&
%     $g:B\tickar C$ &
%     $f\sqcup g:$&$A + B $&$\tickar C$ \\
%     %
%     biproduct &
%     $f:A\tickar B$ &
%     $g:A\tickar B$ &
%     $f+ g:$&$A + A$&$\tickar B + B$ \\
%     %
%     trace &
%     $f: C \times A \tickar C \times B$ &
%     -&
%     $\Tr_{A,B}^C(f) :$&$A$&$\tickar B$
% \end{tabular}
%     \caption{Various composition operations on design problems (i.e. morphisms) in $\DP$.}
% \end{table}


We can now extend the ways we can compose to DPs to the case of DPIs.

%\begin{definition}[Category enriched in $\tup{\Bool,\true,\wedge}$]\label{def:cat_enriched_bool}
%A \emph{category enriched in $\tup{\Bool,\true,\wedge}$}, denoted $\cP =\tup{P,\leqP }$, consists of the following:
%\begin{description}
%\item[\quad 1) Reflexivity:] for any object $p\in P$, it is true that $p \leqP p$.
%\item[\quad 2) Transitivity:] for any three objects $p_1,p_2,p_3\in P$, if $p_1\leqP  p_2$ and $p_2\leqP p_3$, then $p_1 \leqP p_3$.
%\end{description}
%\end{definition}





% \begin{example}The only morphisms in $\Bool$ are $\false \leq \false$, $\true \leq \true$, and $\false \leq \true$. For posets and monotone maps, this makes sense, since there is at most one element in each hom-set $\hom_P(p_1,p_2)$ of any poset $P$, so the mapping $F : \hom_P(p_1,p_2) \to \hom_Q(q_1,q_2)$ is either $\false \to \false$ (i.e. $p_1 \not \leq p_2$ so $q_1 \not \leq q_2$), $\true \leq \true$ (i.e. $p_1 \leq p_2$ implies $q_1 \leq q_2$, which is just the naturality condition of being a functor), or $\false \leq \true$ (i.e. $p_1 \not \leq p_2$, and $q_1 \leq q_2$). Importantly, what is not allowed is $\true \to \false$, which contradicts the naturality condition of a functor between posets (i.e. that it is monotone).\end{example}

%\begin{definition}
%Given posets $\cP = \tup{P,\leqP }$ and $\cQ = \tup{Q,\leqQ }$, a \emph{boolean-enriched functor} $F\colon P\to Q$ consists of the following:
%\begin{description}
%	\item[\quad Objects:] a map $F\colon P\to Q$, from objects in $P$ to objects in $Q$.
%	\item[\quad Order:] for every $p_1,p_2\in P$ an assurance that $\cP(p_1, p_2)\leq_\Bool \cQ(F(p_1),F(p_2))$. In other words, if $p_1\leqP p_2$, then $F(p_1)\leqQ F(p_2)$.
%\end{description}
%We often denote $\Ob_F$ simply by $F$, e.g.\ writing $F(c)$ rather than $\Ob_F(c)$; it overloads the notation $F$, but is often easier to read.
%\end{definition}

%\begin{definition}
%Given two posets $\cP =\tup{P,\leqP }$ and $\cQ =\tup{Q,\leqQ }$, a \emph{boolean-enriched profunctor} \[f\colon \cP \tickar \cQ \] is a monotone map
%\[
%    f\colon\cP \op\times \cQ \toinPos \Bool.
%\]
%\end{definition}

% Recall that a $\Set$-enriched functor is just a functor, while a $\Set$-enriched category is just a category.
% Then:

%\begin{definition}
%Given two categories $\cat{C}=\tup{\Ob_\cat{C},\to_{\cat{C}}}$ and $\cat{D}=\tup{\Ob_\cat{D},\to_{\cat{D}}}$, a \emph{set-enriched functor} $F\colon\cat{C}\to\cat{D}$ consists of the following:
%\begin{description}
%	\item[\quad Objects:] a map $\Ob_F\colon \Ob_\cat{C}\to \Ob_\cat{D}$, from objects in \CatC to objects in \CatD.
%	\item[\quad Morphisms:] for every $c_1,c_2\in\Ob_\cat{C}$ a map $F\colon\cat{C}(c_1, c_2)\to_\Set\cat{D}(\Ob_F(c_1),\Ob_F(c_2))$. In other words, for every $f\colon c_1\to_\cat{C} c_2$ in \CatC, we associate a $F(f)\colon \Ob_F(c_1)\to_\cat{D} \Ob_F(c_2)$ in \CatD.
%	\item[\quad Identities:] for every $c\in\cat{C}$, we have $F(\id_c)=\id_{F(c)}$.
%	\item[\quad Composition:] for every $f\colon c_1\to_\cat{C} c_2$ and $g\colon c_2\to_\cat{C} c_3$ in \CatC, we have $F(f;g)=F(f);F(g)$ in \CatD.
%\end{description}
%We often denote $\Ob_F$ simply by $F$, e.g.\ writing $F(c)$ rather than $\Ob_F(c)$; it overloads the notation $F$, but is often easier to read.
%\end{definition}