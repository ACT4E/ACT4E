% !TEX root = ../CategoricalCoDesign.tex
\section{The duality of design}
\subsection{Galois connections}

\AC{ Define Galois connection. Introduce by saying that we want to represent the pair of maps (from and to) between
functionality and resources.}

\AC{Define the category with object Posets and morphisms Galois connections.}

\AC{Explain what this category  can and cannot represent. (It is a subcategory of DP. (preliminary exercise).) 

In ref to next section, reorganize the flow like this:

1) What can Pos represent and what not.

2) What can Galois represent.

3) What's missing and justifies DP.
}

\todo{This definition is now superseded, we give two more specific definitions (antitone and monotone connections) later.}
\begin{shaded}
\begin{definition}[Galois connection]
\label{def:galois}
Given two preorders $P$ and $Q$, a \emph{Galois connection} between them is a pair of monotone maps $f\colon P\to Q$, $g\colon Q\to P$ such that
\begin{equation}
    f(p)\leq q \Leftrightarrow p\leq g(q).
\end{equation}
One says that $f$ is the \emph{left adjoint} and $g$ is the \emph{right adjoint} of the Galois connection.
\end{definition}
\end{shaded}

\begin{definition}[Category $\mathbf{Gal}$]
The category $\mathbf{Gal}$ is specified by
\begin{compactenum}
    \item \emph{Objects:} Objects are posets.
    \item \emph{Morphisms:} Morphisms are Galois connections (\cref{def:galois}).
    \item \emph{Identity morphism:} For each object $C\in \Ob_\mathbf{Gal}$, we define the identity morphism a pair of identity functions, such that $\text{Id}(C)=C\leq C$.
\end{compactenum}
\end{definition}

\newpage
\subsection{Notebook AC/GZ}

\begin{lemma}
$\LF$ is a bounded lattice with 
\begin{equation}
    \{\LF,\leq_{\LF},\bot_{\LF},\top_{\LF},\vee_{\LF},\wedge_{\LF}\}=\{\LF,\subseteq,\emptyset,F,\cup,\cap\}.
\end{equation}
\end{lemma}
\begin{proof}
Consider the poset $\tup{\LF,\subseteq}$ and $p,q\in \LF$. The least upper bound of $p,q$ is $p\cup u$, and their greatest lower bound is $p\cap u$. This makes $\LF$ a lattice. Furthermore, $\LF$ is bounded because it has a bottom $\emptyset$ (included by all other elements of $\LF$) and a top $F$ (including all other elements of $\LF$).
\todo{finish}
\end{proof}
\begin{lemma}
$\UR$ is a bounded lattice with
\begin{equation}
    \{\UR,\leq_{\UR},\bot_{\UR},\top_{\UR},\vee_{\UR},\wedge_{\UR}\}=\{\UR,\supseteq,R,\emptyset,\cap,\cup\}.
\end{equation}
\begin{proof}
Consider the poset $\tup{\UR,\supseteq}$ and $p,q\in \UR$. The least upper bound of $p,q$ is $p\cap u$, and their greatest lower bound is $p\cup u$. This makes $\UR$ a lattice. Furthermore, $\UR$ is bounded because it has a bottom $R$ (including all other elements of $\UR$) and a top $\emptyset$ (included in all other elements of $\UR$).
\todo{finish}
\end{proof}
\end{lemma}
Note that the order on $\LF$ is $\subseteq$, and the order on $\UR$ is $\supseteq$.

Consider the profunctor $d\colon F\tickar R$. We can define the maps that work on single functionality and resources:

\begin{equation}
    \begin{aligned}
    \theta\colon F&\to \UR\\
    f&\mapsto \{r\in R \colon d(f, r) \},
    \end{aligned}
\end{equation}

\begin{equation}
    \begin{aligned}
    \psi\colon R&\to \LF\\
    r&\mapsto \{f\in F \colon d(f, r) \}.
    \end{aligned}
\end{equation}
We can define the maps that work on multiple functionality 
and resources:
\begin{equation}
\label{eq:galoisalfa}
    \begin{aligned}
    \alpha\colon \LF&\to \UR\\
    S&\mapsto \{r\in R \colon \exists f\in S\colon \ d(f,r)\},
    \end{aligned}
\end{equation}
Alternatively, we can write 
\begin{equation}
\begin{aligned}
\alpha \colon \LF &\to \UR \\
S&\mapsto \bigcup_{f\in S} \theta(f).
\end{aligned}
\end{equation}

\begin{equation}
\label{eq:galoisbeta}
    \begin{aligned}
    \beta\colon \UR&\to \LF\\
    T&\mapsto \{f\in F  \colon \exists r\in T\colon d(f,r)\},
    \end{aligned}
\end{equation}

\noindent Alternatively, we can write
\begin{equation}
\begin{aligned}
\beta \colon \UR &\to \LF \\
T&\mapsto \bigcup_{r\in T} \psi(r).
\end{aligned}
\end{equation}

\begin{equation}
\label{eq:galoisdelta}
    \begin{aligned}
    \delta \colon \LF&\to \UR\\
    S&\mapsto \{r\in R \colon \forall {f\in S}\colon d(f,r)\},
    \end{aligned}
\end{equation}
Alternatively, we can write
\begin{equation}
    \begin{aligned}
    \delta\colon \LF &\to \UR\\
    S&\mapsto \bigcap_{f\in S}\theta(f).
    \end{aligned}
\end{equation}

\begin{equation}
\label{eq:galoisgamma}
    \begin{aligned}
    \gamma \colon \UR&\to \LF\\
    T&\mapsto \{f\in F \colon \forall {r\in T}\colon d(f,r)\},
    \end{aligned}
\end{equation}
Alternatively, we can write
\begin{equation}
    \begin{aligned}
    \delta\colon \UR &\to \LF\\
    T&\mapsto \bigcap_{r\in T}\psi(r).
    \end{aligned}
\end{equation}

Properties of these maps are reported in \cref{tab:galoisproperties}.


\begin{table}[h!]
    \begin{center}
    \begin{tabular}{c|l|l|c|c|c|c|c}
         $\star$&X&Y&$\star(\bot)$&$\star(\top)$&$A\leq_X B$ &$\star(A\vee_X B)$&$\star(A\wedge_X B)$\\
         \hline
         $\alpha$&$\LF$&$\UR$&$\alpha(\emptyset)=\emptyset$&$\alpha(F)\geq_{\UR} \alpha(\cdot)$&$\alpha(A)\geq_{\UR} \alpha(B)$& $\alpha(A)\vee_{\LF}\alpha(B)$&$\alpha(A)\wedge_{\LF}\alpha(B)$\\
         \hline 
         $\beta$&$\UR$&$\LF$&$\beta(R)\geq_{\LF}\beta(\cdot)$&$\beta(\emptyset)=\emptyset$&$\beta(A)\geq_{\LF} \beta(B)$&$\beta(A)\vee_{\LF}\beta(B)$&$\beta(A)\wedge_{\LF}\beta(B)$\\
         \hline
         $\delta$&$\LF$&$\UR$&$\delta(\emptyset)=R$&$\delta(F)\geq_{\UR}\delta(\cdot)$&$\delta(A)\leq_{\UR} \delta(B)$&$\delta(A)\wedge_{\UR}\delta(B)$&$\delta(A)\vee_{\UR}\delta(B)$\\
         \hline
         $\gamma$&$\UR$&$\LF$&$\gamma(R)\leq_{\LF}\gamma(\cdot)$&$\gamma(\emptyset)=F$&$\gamma(A) \leq_{\LF} \gamma(B)$&$\gamma(A)\wedge_{\LF}\gamma(B)$&$\gamma(A)\vee_{\LF}\gamma(B)$
    \end{tabular}
    \caption{Properties of $\alpha,\beta,\delta,\gamma$}
    \label{tab:galoisproperties}
    \end{center}
\end{table}


\begin{comment}
\begin{align}
    \alpha(\emptyset_{F}) &= \emptyset_{R} \\
    \alpha(\bot_{LF}) &= \top_{UR} \\
    \alpha(F) &\subseteq \text{ any other }\alpha(\cdot) 
    \\
    A \subseteq B \ &\Rightarrow\ \alpha(A) \subseteq \alpha(B)
    \\
    A \leq_{LF} B \ &\Rightarrow\ \alpha(A) \geq_{UR} \alpha(B) \label{eq:1}\\
    \alpha(A \cup B) &= \alpha(A) ? \alpha(B) \\
    \alpha(A \cap B) &= \alpha(A) ? \alpha(B) 
\end{align}

\begin{align}
    \beta(\emptyset_{R}) &= \emptyset_{F} \\
    \beta(\top_{UR}) &= \bot_{LF} \\
    \beta(\bot_{UR}) &= ... \\
    A \subseteq B \ &\Rightarrow\ \beta(A) \subseteq \beta(B)     \\
    A \geq_{UR} B \ &\Rightarrow\ \beta(A) \leq_{LF} \beta(B) \label{eq:1} \\
    \beta(A \cup B) &= \beta(A) ? \beta(B) \\
    \beta(A \cap B) &= \beta(A) ? \beta(B)
\end{align}
$\delta, \gamma$:
\begin{align}
\delta(\emptyset_F) &= R  \\
    \delta(\bot_{LF}) &= \bot_{UR}  \\
    \delta(\top_{LF}) &= ... \\
    A \subseteq B \ &\Rightarrow\ \delta(A) \supseteq \delta(B)\\
    A \leq_{LF} B \ &\Rightarrow\ \delta(A) \leq_{UR} \delta(B)
    \\
    \delta(A \cup B) &= \delta(A) ? \delta(B) \\
    \delta(A \cap B) &= \delta(A) ? \delta(B)
\end{align}
\begin{align}
   \gamma(\emptyset_R) &= F \\
    \gamma(\top_{UR}) &= \top_{LF} \\
    \gamma(\bot_{UR}) &= ... \\
    A \subseteq B \ & \Rightarrow\ \gamma(A) \supseteq \gamma(B) \\
    A \leq_{UR} B \ & \Rightarrow\ \gamma(A) \leq_{LF} \gamma(B) \\
    \gamma(A \cup B) &= \gamma(A) ? \gamma(B) \\
    \gamma(A \cap B) &= \gamma(A) ? \gamma(B) \\
\end{align}
\end{comment}


\todo{should we stick to one of monotone/order preserving as terms?}
\begin{definition}[Monotone Galois Connection]
A monotone Galois connection between $A$ and $B$ is a pair of \textbf{order-preserving} functions $f:A\to B$ and $g:B\to A$ such that for all $a\in A$, $b\in B$:
\begin{equation}
    f(a) \leq_B b \quad \Leftrightarrow \quad a \leq_A g(b)
\end{equation}
This is equivalent to ask, for all $a\in A$, $b\in B$:
\begin{equation}
    a\leq_A g(f(a)) \qquad \text{and} \qquad b\geq_{B}f(g(b))
\end{equation}
\end{definition}



\begin{definition}[Antitone Galois Connection]
An antitone Galois connection between $A$ and $B$ is a pair of \textbf{order-reversing} functions $f\colon A\to B$ and $g \colon B\to A$ such that for all $a\in A$, $b\in B$:
\begin{equation}
    b \leq_B f(a) \quad \Leftrightarrow \quad a \leq_A g(b) 
\end{equation}
This is equivalent to ask for all $a\in A$, $b\in B$:
\begin{equation}
a \leq_A g(f(a))   \qquad \text{and} \qquad  b \leq_B f(g(b))
\end{equation}
\end{definition}


\begin{lemma} $(\delta, \gamma)$~forms a \textbf{monotone} Galois connection between $\LF$ and $\UR$.
\end{lemma}
\begin{proof}

First of all, we need to prove that $\delta$ and $\gamma$ are order-\textbf{preserving}, i,e. that if $L\subseteq L'$ then $\delta(L)\supseteq\delta(L')$ and if $U\supseteq U'$ then $\gamma(U)\subseteq\gamma(U')$. 
\todo{write proof}

\noindent We need to show that for any lower set $L\subseteq F$ of functionalities and upper set $U\subseteq R$ of resources, we have
\begin{equation}
L\subseteq\gamma(U) \iff \delta(L)\supseteq U
\end{equation}
The left-hand side says that if $f\in L$, then $\forall r \in U$ we have $d(f,r)=\true$. The right-hand side says that if $r\in U$ then $\forall f \in L$, $d(f,r)=\true$. Both are equivalent to $\forall f\in L,r\in U$, $d(f,r)=\true$, and hence to each other. In formulas:
\begin{equation}
    \begin{aligned}
    L \subseteq \gamma(U)&\equiv L\subseteq \{f \in F\colon \forall r\in U\colon d(f,r)\} \\
    &\equiv \forall f\in L, r\in U \colon d(f,r)=\true \\
    &\equiv \forall r\in U, f\in L \colon d(f,r)=\true\\
    &\equiv U\subseteq \{r\in R\colon \forall f\in L\colon d(f,r)=\true \}\\
    &\equiv U\subseteq \delta(L).
    \end{aligned}
\end{equation}
\end{proof}
% We need to prove that, for $a\in F$, $b\in R$:
% \begin{equation}
% \label{eq:gammadeltafirst}
%     a\leq_{LF} \gamma(\delta(a)),
% \end{equation}
% and
% \begin{equation}
% \label{eq:gammadeltasec}
% b\geq_{UR}\delta(\gamma(b))
% \end{equation}
% \begin{itemize}
%     \item Let's start from \cref{eq:gammadeltafirst}. We know that $a\leq_{LF} \gamma(\delta(a))$ means $a\subseteq \gamma(\delta(a))$. Assume this is not true, i.e. $\exists x\in a \colon (x\in a)\wedge (x \not\in \gamma(\delta(a)))$. Following \cref{eq:galoisdelta}, we know that if $y\in \delta(a)$, $d(x',y)=\true \ \forall x'\in a$. Following \cref{eq:galoisgamma}, we know that if $w\in \gamma(\delta(a))$, $d(w,y)=\true \ \forall y\in \delta(a)$. But from before, we know that for each $y\in \delta(a)$, we have $d(x',y)=\true$, for all $x'\in a$, meaning that $\gamma(\delta(a))$ must include $x'$, $\forall x' \in a$. This contradicts the initial assumption. 
%     \item Let's continue with \cref{eq:gammadeltasec}. We know that $b\geq_{UR} \delta(\gamma(b))$ means $b \subseteq \delta(\gamma(b))$. Assume this is not true, i.e. $\exists x\in b\colon (x\in b)\wedge (x \not\in \delta(\gamma(b)))$. From \cref{eq:galoisgamma}, we know that if $y \in \gamma(b)$, $d(y,x')=\true$, for all $x'\in b$. Following \cref{eq:galoisdelta}, we know that if $w\in \delta(\gamma(b))$, $d(y,w)=\true$, for all $y\in \gamma(b)$. But from before, we know that for each $y\in \gamma(b)$, we have $d(y,x')=\true$, for all $x'\in b$, meaning that $\delta(\gamma(b))$ must include $x'$, $\forall x' \in b$. This contradicts the initial assumption.
% \end{itemize}


\begin{lemma} $(\alpha, \beta)$ forms an  \textbf{antitone} Galois connection between $LF$ and $UR$.
\end{lemma}
\begin{proof}
Preliminarly note that $\alpha$ and $\beta$ are order \textbf{reversing}. For $a\in F$, $b\in R$, we want to show:
\begin{equation}
\label{eq:alfabetafirst}
    L\subseteq \beta(\alpha(L))
\end{equation}
and
\begin{equation}
\label{eq:alfabetasec}
     U\supseteq \alpha(\beta(U)).
\end{equation}
\paragraph{Counterexample} Consider $d$ as the design problem which is always not feasible (the empty profunctor), i.e. $d(f,r)=\false$, $\forall f\in F,r\in R$. Take any $L\in F$. We know that $\alpha(L)=\emptyset$, and $\beta(\alpha(L))=\beta(\emptyset)=\emptyset$. But $L\subseteq \emptyset$ is not true.

\begin{comment}
Or ...
\begin{itemize}
    \item \cref{eq:alfabetafirst} means that $x\in L \Rightarrow x \in \beta(\alpha(L))$. Let's check this. Given $x\in L$, we know that $y\in \alpha(x)$ if $\exists f\in x$ such that $d(f,y)=\true$. Furthermore, we know that $z\in \beta(\alpha(x))$ if $\exists r\in \alpha(x)$ such that $d(z,r)=\true$. We know that for all $r\in \alpha(x)$, there exists a $f\in x$ such that $d(f,r)=\true$, so at least $f\in \beta(\alpha(x))$. 
    \todo{how to generalize?}
    \item \cref{eq:alfabetasec} means that $x\in \alpha(\beta(U))\Imp x\in U$. Let's check this. Given $x\in U$, we know that $y\in \beta(x)$ if $\exists r\in x$ such that $d(y,r)=\true$. Furthermore, we know that $z\in \alpha(\beta(x))$ if $\exists f\in \beta(x)$ such that $d(f,z)=\true$. We know that for all $f\in \beta(x)$, there exists a $r\in x$ such that $d(f,r)=\true$, so at least $r\in \beta(\alpha(x))$.
    \todo{how to generalize}
\end{itemize}
\end{comment}

\end{proof}






