% !TEX root = ../CategoricalCoDesign.tex
\section{The duality of design}
\subsection{Galois connections}

\AC{Define the category with object Posets and morphisms Galois connections.}

\AC{Explain what this category  can and cannot represent. (It is a subcategory of DP. (preliminary exercise).) 

In ref to next section, reorganize the flow like this:

1) What can Pos represent and what not.

2) What can Galois represent.

3) What's missing and justifies DP.
}

\begin{definition}[Monotone Galois Connection]
A \emph{monotone Galois connection} between $A$ and $B$ is a pair of monotone maps $f\colon A\to B$ and $g\colon B\to A$ such that for all $a\in A$, $b\in B$:
\begin{equation}
    f(a) \leq_B b \quad \Leftrightarrow \quad a \leq_A g(b).
\end{equation}
This is equivalent to ask, for all $a\in A$, $b\in B$:
\begin{equation}
    a\leq_A g(f(a)) \qquad \text{and} \qquad b\geq_{B}f(g(b)).
\end{equation}
\end{definition}



\begin{definition}[Antitone Galois Connection]
An \emph{antitone Galois connection} between $A$ and $B$ is a pair of antitone maps $f\colon A\to B$ and $g \colon B\to A$ such that for all $a\in A$, $b\in B$:
\begin{equation}
    b \leq_B f(a) \quad \Leftrightarrow \quad a \leq_A g(b).
\end{equation}
This is equivalent to ask for all $a\in A$, $b\in B$:
\begin{equation}
a \leq_A g(f(a))   \qquad \text{and} \qquad  b \leq_B f(g(b)).
\end{equation}
\end{definition}

Consider the profunctor $d\colon F\tickar R$. We can define the maps that work on single functionality and resources:

\begin{equation}
    \begin{aligned}
    \theta\colon F&\to \UR\\
    f&\mapsto \{r\in R \colon d(f, r) \},
    \end{aligned}
\end{equation}

\begin{equation}
    \begin{aligned}
    \psi\colon R&\to \LF\\
    r&\mapsto \{f\in F \colon d(f, r) \}.
    \end{aligned}
\end{equation}
We can define the maps that work on multiple functionality 
and resources:
\begin{equation}
\label{eq:galoisalfa}
    \begin{aligned}
    \alpha\colon \LF&\to \UR\\
    S&\mapsto \{r\in R \colon \exists f\in S\colon \ d(f,r)\},
    \end{aligned}
\end{equation}
Alternatively, we can write 
\begin{equation}
\begin{aligned}
\alpha \colon \LF &\to \UR \\
S&\mapsto \bigcup_{f\in S} \theta(f).
\end{aligned}
\end{equation}

\begin{equation}
\label{eq:galoisbeta}
    \begin{aligned}
    \beta\colon \UR&\to \LF\\
    T&\mapsto \{f\in F  \colon \exists r\in T\colon d(f,r)\},
    \end{aligned}
\end{equation}

\noindent Alternatively, we can write
\begin{equation}
\begin{aligned}
\beta \colon \UR &\to \LF \\
T&\mapsto \bigcup_{r\in T} \psi(r).
\end{aligned}
\end{equation}

\begin{equation}
\label{eq:galoisdelta}
    \begin{aligned}
    \delta \colon \LF&\to \UR\\
    S&\mapsto \{r\in R \colon \forall {f\in S}\colon d(f,r)\},
    \end{aligned}
\end{equation}
Alternatively, we can write
\begin{equation}
    \begin{aligned}
    \delta\colon \LF &\to \UR\\
    S&\mapsto \bigcap_{f\in S}\theta(f).
    \end{aligned}
\end{equation}

\begin{equation}
\label{eq:galoisgamma}
    \begin{aligned}
    \gamma \colon \UR&\to \LF\\
    T&\mapsto \{f\in F \colon \forall {r\in T}\colon d(f,r)\},
    \end{aligned}
\end{equation}
Alternatively, we can write
\begin{equation}
    \begin{aligned}
    \delta\colon \UR &\to \LF\\
    T&\mapsto \bigcap_{r\in T}\psi(r).
    \end{aligned}
\end{equation}

\noindent Properties of these maps are reported in \cref{tab:galoisproperties}.


\begin{table}[h!]
    \begin{center}
    \begin{tabular}{c|l|l|c|c|c|c|c}
         $\star$&X&Y&$\star(\bot)$&$\star(\top)$&$A\leq_X B$ &$\star(A\vee_X B)$&$\star(A\wedge_X B)$\\
         \hline
         $\alpha$&$\LF$&$\UR$&$\alpha(\emptyset)=\emptyset$&$\alpha(F)\geq_{\UR} \alpha(\cdot)$&$\alpha(A)\geq_{\UR} \alpha(B)$& $\alpha(A)\vee_{\LF}\alpha(B)$&$\alpha(A)\wedge_{\LF}\alpha(B)$\\
         \hline 
         $\beta$&$\UR$&$\LF$&$\beta(R)\geq_{\LF}\beta(\cdot)$&$\beta(\emptyset)=\emptyset$&$\beta(A)\geq_{\LF} \beta(B)$&$\beta(A)\vee_{\LF}\beta(B)$&$\beta(A)\wedge_{\LF}\beta(B)$\\
         \hline
         $\delta$&$\LF$&$\UR$&$\delta(\emptyset)=R$&$\delta(F)\geq_{\UR}\delta(\cdot)$&$\delta(A)\leq_{\UR} \delta(B)$&$\delta(A)\wedge_{\UR}\delta(B)$&$\delta(A)\vee_{\UR}\delta(B)$\\
         \hline
         $\gamma$&$\UR$&$\LF$&$\gamma(R)\leq_{\LF}\gamma(\cdot)$&$\gamma(\emptyset)=F$&$\gamma(A) \leq_{\LF} \gamma(B)$&$\gamma(A)\wedge_{\LF}\gamma(B)$&$\gamma(A)\vee_{\LF}\gamma(B)$
    \end{tabular}
    \caption{Properties of $\alpha,\beta,\delta,\gamma$}
    \label{tab:galoisproperties}
    \end{center}
\end{table}

\begin{lemma}
\label{lemma:deltagammamonotone}
$\delta$ and $\gamma$ are monotone maps. 
\end{lemma}
\begin{proof}
We first prove that $\delta$ is a monotone map. Given $A,B\in \LF$ with $A\subseteq B$, one has 
\begin{equation}
    \begin{aligned}
    \delta(A)&=\{r\in R\colon \forall f \in A\colon d(f,r)\}\\
    &\supseteq \{r\in R\colon \forall f\in B \colon d(f,r)\}\\
    &=\delta(B),
    \end{aligned}
\end{equation}
meaning that $A\leq_{\LF} B \Imp \delta(A)\leq_{\UR} \delta(B)$.
We now prove that $\gamma$ is a monotone map. Given $C,D\in \UR$, with $C\supseteq D$, one has
\begin{equation}
    \begin{aligned}
    \gamma(C)&=\{f\in F\colon \forall r \in C\colon d(f,r)\}\\
    &\subseteq \{f\in F\colon \forall r \in D \colon d(f,r)\}\\
    &=\gamma(D),
    \end{aligned}
\end{equation}
meaning that $C\leq_{\UR} D \Imp \gamma(C)\leq_{\LF} \gamma(D)$.
\end{proof}

\begin{lemma}
\label{lemma:alfabetaantitone}
$\alpha$ and $\beta$ are antitone maps.
\end{lemma}
\begin{proof}
We first prove that $\alpha$ is an antitone map. Given $A,B\in \LF$, with $A\subseteq B$, one has
\begin{equation}
    \begin{aligned}
    \alpha(A)&=\{ r\in R\colon \exists f\in A\colon d(f,r)\}\\
    &\subseteq \{r\in R\colon \exists f\in B\colon d(f,r)\}\\
    &=\alpha(B),
    \end{aligned}
\end{equation}
meaning that $A\leq_{\LF} B\Imp \alpha(A) \geq_{\UR} \alpha(B)$.
We now prove that $\beta$ is an antitone map. Given $C,D\in \UR$, with $C\supseteq D$, one has
\begin{equation}
    \begin{aligned}
    \beta(C)&=\{f\in F\colon \exists r\in C\colon d(f,r)\}\\
    &\supseteq \{f\in F\colon \exists r\in D \colon d(f,r)\}\\
    &=\beta(D),
    \end{aligned}
\end{equation}
meaning that $C\leq_{\UR} D \Imp \beta(C)\geq_{\UR} \beta(D)$.
\end{proof}


\begin{lemma} $(\delta, \gamma)$~forms a \textbf{monotone} Galois connection between $\LF$ and $\UR$.
\end{lemma}
\begin{proof}
In~\cref{lemma:deltagammamonotone} we proved that $\delta$ and $\gamma$ are monotone maps. We now need to show that for any lower set $L\subseteq F$ of functionalities and upper set $U\subseteq R$ of resources, we have
\begin{equation}
L\subseteq\gamma(U) \iff \delta(L)\supseteq U
\end{equation}
The left-hand side says that if $f\in L$, then $\forall r \in U$ we have $d(f,r)=\true$. The right-hand side says that if $r\in U$ then $\forall f \in L$, $d(f,r)=\true$. Both are equivalent to $\forall f\in L,r\in U$, $d(f,r)=\true$, and hence to each other. In formulas:
\begin{equation}
    \begin{aligned}
    L \subseteq \gamma(U)&\equiv L\subseteq \{f \in F\colon \forall r\in U\colon d(f,r)\} \\
    &\equiv \forall f\in L, r\in U \colon d(f,r)=\true \\
    &\equiv \forall r\in U, f\in L \colon d(f,r)=\true\\
    &\equiv U\subseteq \{r\in R\colon \forall f\in L\colon d(f,r)=\true \}\\
    &\equiv U\subseteq \delta(L).
    \end{aligned}
\end{equation}
\end{proof}
% We need to prove that, for $a\in F$, $b\in R$:
% \begin{equation}
% \label{eq:gammadeltafirst}
%     a\leq_{LF} \gamma(\delta(a)),
% \end{equation}
% and
% \begin{equation}
% \label{eq:gammadeltasec}
% b\geq_{UR}\delta(\gamma(b))
% \end{equation}
% \begin{itemize}
%     \item Let's start from \cref{eq:gammadeltafirst}. We know that $a\leq_{LF} \gamma(\delta(a))$ means $a\subseteq \gamma(\delta(a))$. Assume this is not true, i.e. $\exists x\in a \colon (x\in a)\wedge (x \not\in \gamma(\delta(a)))$. Following \cref{eq:galoisdelta}, we know that if $y\in \delta(a)$, $d(x',y)=\true \ \forall x'\in a$. Following \cref{eq:galoisgamma}, we know that if $w\in \gamma(\delta(a))$, $d(w,y)=\true \ \forall y\in \delta(a)$. But from before, we know that for each $y\in \delta(a)$, we have $d(x',y)=\true$, for all $x'\in a$, meaning that $\gamma(\delta(a))$ must include $x'$, $\forall x' \in a$. This contradicts the initial assumption. 
%     \item Let's continue with \cref{eq:gammadeltasec}. We know that $b\geq_{UR} \delta(\gamma(b))$ means $b \subseteq \delta(\gamma(b))$. Assume this is not true, i.e. $\exists x\in b\colon (x\in b)\wedge (x \not\in \delta(\gamma(b)))$. From \cref{eq:galoisgamma}, we know that if $y \in \gamma(b)$, $d(y,x')=\true$, for all $x'\in b$. Following \cref{eq:galoisdelta}, we know that if $w\in \delta(\gamma(b))$, $d(y,w)=\true$, for all $y\in \gamma(b)$. But from before, we know that for each $y\in \gamma(b)$, we have $d(y,x')=\true$, for all $x'\in b$, meaning that $\delta(\gamma(b))$ must include $x'$, $\forall x' \in b$. This contradicts the initial assumption.
% \end{itemize}


\begin{lemma} $(\alpha, \beta)$ does not form an \textbf{antitone} Galois connection between $LF$ and $UR$.
\end{lemma}
\begin{proof}
In~\cref{lemma:alfabetaantitone} we have proved that $\alpha$ and $\beta$ are antitone maps. For $L\in F$, $U\in R$, we want to show that the following does not hold:
\begin{equation}
\label{eq:alfabetafirst}
    L\subseteq \beta(\alpha(L))
\end{equation}
and
\begin{equation}
\label{eq:alfabetasec}
     U\supseteq \alpha(\beta(U)).
\end{equation}
\paragraph{Example} Consider $d$ as the design problem which is always not feasible (the empty profunctor), i.e. $d(f,r)=\false$, $\forall f\in F,r\in R$. Take any $L\in F$. We know that $\alpha(L)=\emptyset$, and $\beta(\alpha(L))=\beta(\emptyset)=\emptyset$. But $L\subseteq \emptyset$ is not true.

\end{proof}






