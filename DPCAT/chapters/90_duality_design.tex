% !TEX root = ../CategoricalCoDesign.tex
\section{The duality of design}
\subsection{Galois connections}
\AC{ Define Galois connection. Introduce by saying that we want to represent the pair of maps (from and to) between
functionality and resources.}

\AC{Define the category with object Posets and morphisms Galois connections.}

\AC{Explain what this category  can and cannot represent. (It is a subcategory of DP. (preliminary exercise).) 

In ref to next section, reorganize the flow like this:

1) What can Pos represent and what not.

2) What can Galois represent.

3) What's missing and justifies DP.
}

\begin{shaded}
\begin{definition}[Galois connection]
\label{def:galois}
Given two preorders $P$ and $Q$, a \emph{Galois connection} between them is a pair of monotone maps $f\colon P\to Q$, $g\colon Q\to P$ such that
\begin{equation}
    f(p)\leq q \Leftrightarrow p\leq g(q).
\end{equation}
One says that $f$ is the \emph{left adjoint} and $g$ is the \emph{right adjoint} of the Galois connection.
\end{definition}
\end{shaded}

\begin{definition}[Category $\mathbf{Gal}$]
The category $\mathbf{Gal}$ is specified by
\begin{compactenum}
    \item \emph{Objects:} Objects are posets.
    \item \emph{Morphisms:} Morphisms are Galois connections (\cref{def:galois}).
    \item \emph{Identity morphism:} For each object $C\in \Ob_\mathbf{Gal}$, we define the identity morphism a pair of identity functions, such that $\text{Id}(C)=C\leq C$.
\end{compactenum}
\end{definition}