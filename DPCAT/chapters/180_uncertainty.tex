% !TEX root = ../CategoricalCoDesign.tex
\section{Thinking about uncertainty}
\todo{Add introduction on uncertainty}
\subsection{Monads}
\subsection{Uncertainty (to put back up)}

\begin{definition}[Bounded interval]
Given a poset $ \tup{P, \leq}$ and elements $x,y$ of $P$, we define the bounded interval $[x,y]_P$, $x$ to be the subset
\begin{equation}
    [x,y]_P\coloneqq \{z\colon P \mid x\leq z \leq y\}.
\end{equation}
Thinking of $P$ as a category, $[x,y]_P$ represents the bislice (under-over) category of $P$.
\end{definition}

\begin{definition}[Interval category]
The \emph{interval category} (also denoted \Cat{2}) is the category with two objects and one nontrivial morphism connecting them.
\end{definition}

\begin{shaded}
\begin{definition}[Natural transformation]
Let $\Cat{C}$ and $\Cat{D}$ be categories, and let $F,G\colon \Cat{C}\to \Cat{D}$ be functors. To specify a \emph{natural transformation} $\alpha\colon F\to G$
\begin{equation}
\includesag{55_natural_1}
\end{equation}
one specifies for each obect $c\in \CatC$ a morphism $\alpha_c\colon F(c)\to G(c)$ in $\Cat{D}$, called the $c$\emph{-component} of $\alpha$. For every morphism $f\colon c\to d$ in $\Cat{C}$, these components must satisfy the \emph{naturality condition}:
\begin{equation}
    F(f)\then \alpha_d = \alpha_c\then G(f),
\end{equation}
i.e. the following diagram must commute:
\begin{equation}
\includesag{55_natural_2}
\end{equation}
\end{definition}

\begin{remark}[Natural isomorphism]
A natural transformation $\alpha\colon F\to G$ is called a \emph{natural isomorphism} if each component $\alpha_c$ is an isomorphism in $\CatD$.
\end{remark}

\begin{definition}[Monad]
Let $\CatC$ be a category. A \emph{monad} on $\CatC$ consists of:
\begin{compactenum}
    \item A functor $T \colon \CatC \to \CatC$.
    \item A natural transformation $\eta \colon \id_\CatC \Rightarrow T$ called \emph{unit}.
    \item A natural transformation $\mu\colon TT\Imp T$ called \emph{composition} or \emph{multiplication}.
\end{compactenum}
The constituents must satisfy \emph{left and right unitality}
\begin{equation}
\includesag{55_monad_1}
\end{equation}
and \emph{associativity}
\begin{equation}
\includesag{55_monad_2}
\end{equation}
\end{definition}
\end{shaded}


\subsection{The Uncertainty Monad}