% !TEX root = ../CategoricalCoDesign.tex
\section{Posets with more structure}
In the following, we add some structure to the definition of a poset, by introducing \emph{monoidal posets} and \emph{lattices}.
\subsection{Monoidal posets}
\begin{definition}[Monoidal poset]
\label{def:monoidal_poset}
A \emph{monoidal structure} on a poset $\tup{P,\leq}$ consists of:
\begin{compactenum}
    \item An element $I\in P$, called \emph{monoidal unit}, and
    \item a function $\otimes\colon P\times P\to P$, called the \emph{monoidal product}. Note that we write $\otimes(p_1,p_2)=p_1\otimes p_2$, $p_1,p_2\in P$.
\end{compactenum}
The constituents must satisfy the following properties:
\begin{compactenum}[(a)]
    \item \emph{Monotonicity}: For all $p_1,p_2,q_1,q_2\in P$, if $p_1\leq q_1$ and $p_2\leq q_2$, then $p_1\otimes p_2\leq q_1\otimes q_2$.
    \item \emph{Unitality}: For all $p\in P$, $I\otimes p=p$ and $p\otimes I=p$.
    \item \emph{Associativity}: For all $p,q,r\in P$, $(p\otimes q)\otimes r=p\otimes (q\otimes r)$.
\end{compactenum}
A poset equipped with a monoidal structure $\tup{P,\leq,I,\otimes}$ is called a \emph{monoidal poset}.
\end{definition}

\begin{example}
Consider the real numbers $\mathbb{R}$ with the poset structure given by $\leq$. Consider 0 as monoidal unit and the operation $+\colon \mathbb{R}\times \mathbb{R}\to \mathbb{R}$ as mononidal product. It is easy to see that the conditions of~\cref{def:monoidal_poset} are satisfyied:
\begin{compactenum}[(a)]
    \item If $p_1\leq p_2$ and $q_1\leq q_2$, it is true that $p_1+p_2\leq q_1+q_2$, $p_1,p_2,q_1,q_2\in \mathbb{R}$.
    \item $0+p=p+0=0$, $p\in \mathbb{R}$.
    \item $(p+q)+r=p+(q+r)$, $p,q,r\in \mathbb{R}$.
\end{compactenum}
\end{example}
\subsection{Lattices}
\begin{definition}[Lattice]
\label{def:lattice}
A \emph{lattice} is a poset $\tup{P, \leq}$ with some additional properties:
\begin{compactenum}
    \item Given two points $p, q \in P$, it is always possible to define their least upper bound, called \emph{join}, and indicated as $p \vee q$.
    \item Given two points $p, q \in P$, it is always possible to define their greatest lower bound, called \emph{meet}, and indicated as $p \wedge q$.
\end{compactenum}
\end{definition}

\begin{remark}[Bounded lattices]
If there is a least upper bound for the entire lattice $A$, it is called
the \emph{top} ($\top$). If the greatest lower bound exists it is called the \emph{bottom} ($\bot$). If both a top and a bottom exist, we call the lattice \emph{bounded}, and denote it by $\tup{A,\leq,\vee,\wedge,\bot,\top}$.
\end{remark}

\begin{example}
    In \cref{ex:hasseinclusion} we presented the poset arising from the power set of a set $A$ and ordered via subset inclusion. This is a lattice, bounded by $A$ and by the empty set $\emptyset$. Note that this lattice possesses two (dual) monoidal structures $\tup{\powerset(A),\subseteq,\emptyset,\cup}$ and $\tup{\powerset(A),\subseteq,A,\cap}$.
\end{example}

\begin{example}
Consider the set $\{1,2,3,6\}$ ordered by divisibility. This is a lattice. However, the set $\{1,2,3\}$ ordered by divisibility is not, since 2 and 3 lack a meet (\cref{fig:exlattice}).
\begin{figure}[h!]
\begin{center}
\includesag{40_dpcatfig_exlattice}
\end{center}
\caption{Examples of lattice and non-lattice. \label{fig:exlattice}}
\end{figure}
\end{example}