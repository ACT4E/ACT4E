% !TEX root = ../CategoricalCoDesign.tex
\section{Parallel composition}
\begin{shaded}
\begin{definition}[Monoidal category]\label{def:monoidal_cat}
Given a category $\CatC$, a \emph{monoidal structure} on $\CatC$ consists of:
\begin{compactenum}
    \item An object $I\in \Ob(\CatC)$ called the \emph{monoidal unit}.
    \item A functor $\otimes \colon \CatC \times \CatC\to \CatC$, called the \emph{monoidal product}.
\end{compactenum}
The two constituents are subject to the natural isomorphisms:
\begin{compactenum}
    \item[a)] $\lambda_c \colon I\otimes c \cong c$ for every $c\in \Ob(\CatC)$,
    \item[b)] $\rho_c \colon c\otimes I \cong c$ for every $c\in \Ob(\CatC)$,
    \item[c)] $\alpha_{c,d,e}\colon (c\otimes d)\otimes e \cong c\otimes (d\otimes e)$ for every for every $c,d,e\in \Ob(\CatC)$.
\end{compactenum}
These isomorphisms are themselves required to satisfy the triangle identity
\begin{equation}
\includesag{30_triangle_identity}
\end{equation}
and the pentagon identity
\begin{equation}
\includesag{30_pentagon_identity}
\end{equation}
for $a,b,c,d\in \Ob(\CatC)$.
\noindent A category equipped with a monoidal structure is called a \emph{monoidal category}.
\end{definition}
\end{shaded}

In $\DP$, putting two design problems in parallel corresponds to forming their \emph{monoidal product}. 

\begin{example}
After the X101 spontaneously combusted in low Earth orbit, the astronauts at Jeb's Spaceship Parts go on strike. They demand that the engineers take into account safety and living conditions on the future X102. As long as the propulsion and life support systems of the X102 do not interact, we can simply tensor the two design problems representing these systems into one, big co-design problem.
\[
\includesag{50_engine_tensor_1_2}
\]
\end{example}

\begin{definition}[Monoidal product $\otimes$]\label{def:monoidalproduct}
Given two design problems $f \colon \F{A} \tickar \R{B}$ and $g \colon \F{C} \tickar \R{D}$,
the \emph{monoidal product} $f\otimes g \colon \F{A} \times \F{C} \tickar \R{B} \times \R{D}$ is their conjunction:
\begin{equation}
\begin{aligned}
f \otimes g \colon (\F{A} \times \F{C})\op \times (\R{B} \times \R{D}) & \toinPos \Bool, \\
\tup{\tup{\F{a}^*,\F{c}^*},\tup{\R{b},\R{d}}} &\mapsto f(\F{a}^*,\R{b}) \wedge g(\F{c}^*,\R{d}).
\end{aligned}
\end{equation}
\end{definition}

For $f \colon \F{A} \tickar \R{B}$ and $g \colon \F{C} \tickar \R{D}$, the product $[f \otimes g](\langle \F{a}^*, \F{c}^*\rangle, \langle \F{b},\F{d} \rangle)$ is true if \emph{both} $f(\F{a}^*,\R{b})$ and $g(\F{c}^*,\R{d})$ are true, and false otherwise.

\begin{equation}
\includesag{50_monoidal}
\end{equation}

\GZ{
\begin{lemma}
\label{lemma:monoidal_functorial}
The monoidal product $\otimes$ is functorial in $\DP$.
\end{lemma}
\begin{proof}
First, consider two posets $A,B\in \Ob_\DP$. We need to show that $\id_A\otimes \id_A = \id_{A\times B}$. It holds
\begin{equation}
\begin{aligned}
    \left( \id_{A}\otimes \id_{B}\right)
    \left( \tup{\F{a_1},\F{b_1}}^*,\tup{\R{a_2},\R{b_2}}\right)&=
    \id_A(\F{a_1}^*,\R{a_2})\wedge \id_B(\F{b_1}^*,\R{b_2})\\
    &=\left( \F{a_1}\leq_A \R{a_2}\right)\wedge \left( \F{b_1}\leq_B \R{b_2}\right)\\
    &=\tup{\F{a_1},\F{b_1}}\leq_{A\times B}\tup{\R{a_2},\R{b_2}}\\
    &=\id_{A\times B}\left( \tup{\F{a_1},\F{b_1}}^*,\tup{\R{a_2},\R{b_2}}\right).
\end{aligned}
\end{equation}
Furthermore, consider the design problems $f_1\colon \F{A_1} \tickar \R{B_1}$, $f_2\colon \F{A_2}\tickar \R{B_2}$, $g_1\colon \F{B_1}\tickar \R{C_1}$, $g_2\colon \F{B_2}\tickar \R{C_2}$. We need to show that $\left( (f_1\then g_1) \otimes (f_2\then g_2)\right)=\left( (f_1\otimes f_2)\then (g_1\otimes g_2)\right)$. It holds
\begin{equation}
\begin{aligned}
    \left( (f_1\then g_1) \otimes (f_2\then g_2)\right)\left( \tup{\F{a_1},\F{a_2}}^*,\tup{\R{c_1},\R{c_2}}\right)&=
    (f_1\then g_1)(\F{a_1}^*,\R{c_1})\wedge (f_2\then g_2)(\F{a_2}^*,\R{c_2})\\
    &=\left(\bigvee_{b_1\in B_1}\left( f_1(\F{a_1}^*,\R{b_1})\wedge g_1(\F{b_1}^*,\R{c_1})\right)\right) \wedge\left(\bigvee_{b_2\in B_2}\left( f_2(\F{a_2}^*,\R{b_2})\wedge g_2(\F{b_2}^*,\R{c_2})\right)\right)\\
    &=\bigvee_{b_1\in B_1}\bigvee_{b_2\in B_2} \left(f_1(\F{a_1}^*,\R{b_1})\wedge g_1(\F{b_1}^*,\R{c_1})\wedge f_1(\F{a_2}^*,\R{b_2})\wedge f_2(b_2^*,c_2) \right)\\
    &=\bigvee_{\tup{b_1,b_2}\in B_1\times B_2} \left(f_1(\F{a_1}^*,\R{b_1})\wedge f_2(\F{a_2}^*,\F{b_2})\wedge g_1(\F{b_1}^*,\R{c_1})\wedge g_2(\F{b_2}^*,\R{c_2}) \right)\\
    &=\left( (f_1\otimes f_2)\then (g_1\otimes g_2)\right)\left(\tup{\F{a_1},\F{a_2}}^*,\tup{\R{c_1},\R{c_2}} \right).
\end{aligned}
\end{equation}
\end{proof}}


\begin{lemma}
$\DP$ is a monoidal category with monoidal structure $\left(\otimes, \{1\}\right)$.
\end{lemma}
\begin{proof}
To show that $\DP$ is a monoidal category, we have to first identify the constituents presented in \cref{def:monoidal_cat}. First, recall $\{1\}$ to be the singleton: This is the monoidal unit. In \cref{lemma:monoidal_functorial} we have shown that $\otimes$ is a functor. Furthermore, we identify
\begin{itemize}
    \item $\lambda_A \colon \singleton \times A \tickar A$, for all $A\in \Ob_\DP$, is the left unitor. This is given by
    \begin{equation}
        \lambda_A\left( \tup{1,a_1}^*,a_2\right)\coloneqq a_1\leq_A a_2.
    \end{equation}
    To prove that this is an isomorphism, we define its inverse $\lambda_A^{-1}\colon A\tickar \singleton \times A$ and show that $\lambda_A\then \lambda_A^{-1}=\id_{\{1\}\times A}$ and $\lambda_A^{-1}\then \lambda_A=\id_{A}$. One has
    \begin{equation}
        \begin{aligned}
           \left( \lambda_A^{-1}\then \lambda_A\right)(\tup{a_1^*,a_2})&= \bigvee_{\tup{1,a}\in  \{1\}\times A} \lambda_A^{-1}(a_1^*,\tup{1,a})\wedge \lambda_A(\tup{1,a}^*,a_2)\\
           &= \bigvee_{\tup{1,a}\in  \{1\}\times A}(a_1\leq a) \wedge a\leq a_2\\
           &=a_1\leq a_2\\
           &=\id_A(\tup{a_1^*,a_2}).
        \end{aligned}
    \end{equation}
    Similarly, one can show that $\lambda_A\then \lambda_A^{-1}=\id_{\singleton \times A}$.
    \item $\rho_A\colon A \times \singleton \tickar A$, for all $A\in \Ob_\DP$, is the right unitor. This is given by 
    \begin{equation}
        \rho\left( \tup{a_1,1}^*,a_2\right)\coloneqq a_1\leq_A a_2.
    \end{equation}
    The proof that $\rho_A$ is an isomorphism is analogous to the one for $\lambda_A$.
    \item $\alpha_{A,B,C}\colon (A\times B)\times C \tickar A \times (B\times C)$ for all $A,B,C \in \Ob_\DP$, is the associator. It is given by
    \begin{equation}
        \alpha_{A,B,C}(\tup{\tup{a_1,b_1},c_1}^*,\tup{a_2,\tup{b_2,c_2}})\coloneqq a_1\leq_Aa_2 \wedge b_1\leq_B b_2\wedge c_1\leq_C c_2.
    \end{equation}
    To prove that $\alpha_{A,B,C}$ is an isomorphism, we define its inverse $\alpha_{A,B,C}^{-1}\colon A\times (B\times C) \tickar (A\times B)\times C$ and show $\alpha_{A,B,C}^{-1}\then \alpha_{A,B,C}=\alpha_{A,B,C}\then \alpha_{A,B,C}^{-1}= \id_{A\times B\times C}$. One has
    \begin{equation}
        \begin{aligned}
           \left( \alpha_{A,B,C}^{-1}\then \alpha_{A,B,C} \right)&(\tup{a_1,\tup{b_1,c_1}}^*,\tup{a_2,\tup{b_2,c_2}})\\
           &=\bigvee_{\tup{\tup{a,b},c}\in (A\times B)\times C}
           \alpha_{A,B,C}^{-1}(\tup{a_1,\tup{b_1,c_1}}^*,\tup{\tup{a,b},c})\wedge \alpha_{A,B,C}(\tup{\tup{a,b},c},\tup{a_2,\tup{b_2,c_2}})\\
           &=\bigvee_{\tup{\tup{a,b},c}\in (A\times B)\times C}\left( a_1\leq_A a \wedge b_1\leq_B b \wedge c_1\leq_C c\right)\wedge \left(a\leq_A a_2\wedge b\leq_B b_2 \wedge c\leq_C c_2\right)\\
           &=a_1\leq_Aa_2\wedge b_1\leq_B b_2 I \wedge c_1\leq_C c_2\\
           &=\id_{A\times B\times C}.
        \end{aligned}
    \end{equation}
    The proof for $\alpha_{A,B,C}\then \alpha_{A,B,C}^{-1}$ is analogous.
\end{itemize}
Therefore, $\DP$ is a monoidal category with monoidal structure $\tup{\otimes, \singleton}$.
\end{proof}