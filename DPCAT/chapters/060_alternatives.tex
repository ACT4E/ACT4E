% !TEX root = ../CategoricalCoDesign.tex
% \section{Thinking about alternatives}
\label{sec:coproductset}
\subsection{Coproduct}

There exists a dual notion to ``product'' that is called ``coproduct''. Suppose that we are considering a hybrid car that contains two engines: an
electric engine and an internal combustion engine. Both can produce $\mathsf{motion}$, but each from a different source of energy. The electric engine uses $\mathsf{electric}\ \mathsf{energy}$; the internal combustion engine uses $\mathsf{gasoline}$. The situation is as in \cref{fig:e16a}.

\begin{figure}[h!]
    \centering
    \includesag{30_dpcatfig_e14}
    \caption{Alternative ways to generate $\mathsf{motion}$. \label{fig:e16a}}
\end{figure}

From this we would like to conclude that we can obtain $\mathsf{motion}$ from \emph{either}~$\mathsf{gasoline}$ \emph{or}~$\mathsf{electric}\ \mathsf{energy}$ (\cref{fig:e16b}).

\begin{figure}[h!]
    \centering
    \includesag{30_dpcatfig_e15}
    \caption{We can generate $\mathsf{motion}$ from either $\mathsf{gasoline}$ or $\mathsf{electric} \ \mathsf{energy}$. \label{fig:e16b}}
\end{figure}

To define the idea of ``\emph{either $\mathsf{gasoline}\ \emph{or} \ \mathsf{electric}\ \mathsf{energy}$}'' we can refer to the idea of disjoint union of sets (\cref{def:disjoint-union}). The disjoint union of two sets is a set that contains two distinct copies of each of the two sets. If an element is contained in both sets, there will be two distinct copies of it in the disjoint union.

\begin{definition}[Disjoint union of sets]
\label{def:disjoint-union}
    The \emph{disjoint union} (or sum) of two sets~$A$ and~$B$ is denoted~$A + B$
    and it is defined as
    \begin{equation}
        A + B=\{\tup{1,a}\mid a\in A\}\cup\{\tup{2,b}\mid b\in B\}.
    \end{equation}
\end{definition}

\begin{example}
Consider the sets~$\{\star, \diamond\}$ and $\{\ast, \dagger\}$. Their disjoint union can be represented as in \cref{fig:disjoint}.
\begin{figure}[h!]
    \begin{center}
    \includesag{30_disjoint_union}
    \end{center}
    \caption{Example of a disjoint union of sets. \label{fig:disjoint}}
\end{figure}
\end{example}

We can define the disjoint union of a set with itself; this corresponds to having two distinct copies of the set~(\cref{fig:disjointself}).

\begin{figure}[h!]
\begin{center}
\includesag{30_disjoint_union_self}
    \caption{Disjoint union of a set with itself \label{fig:disjointself}.}
\end{center}
\end{figure}

The disjoint union is a particular instance of the notion of ``coproduct''.
The following definition is the general definition of coproducts for an arbitrary category.

\begin{shaded}
\begin{definition}[Coproduct]
\label{def:catcoproduct}
Let~$\CatC$ be a category and let~$A, B \in \CatC$ be objects. The \emph{coproduct} of~$A$ and~$B$ is an object~$A \sqcup B \in \CatC$ together with two \emph{inclusion morphisms}~$\iota_A \colon A \to A \sqcup B $ and~$\iota_B \colon B \to A  \sqcup B $, such that, given any~$X \in \CatC$ and morphisms $f\colon  A \to X, g \colon B \to X$, there exists a \emph{unique} morphism~$(\coprodMap{f}{g}) \colon A \sqcup B \to X$ such that~$f = \iota_A\then (\coprodMap{f}{g})$ and~$g = \iota_B \then (\coprodMap{f}{g})$. Diagrammatically:
\begin{equation}
\includesag{60_defcoproduct}
\end{equation}
\end{definition}
\end{shaded}

\begin{example}
The two inclusion maps for the disjoint union of sets are~$\iota_1\colon A \to A + B$ and~$\iota_2\colon B \to A + B$, defined as:
\begin{equation}
\begin{aligned}
    \iota_1(a) &= \tup{1, a},\\
    \iota_2(b) &= \tup{2, b}.
\end{aligned}
\end{equation}
Given maps~$f\colon A\to X$ and~$g\colon B\to X$ as in \cref{def:catcoproduct}, then the map~$\coprodMap{f}{g}$ for the disjoint union of sets is
\begin{equation}
\begin{aligned}
    \coprodMap{f}{g} \colon  A + B &\to X \\
    y &   \mapsto
    \begin{cases}
        f(y), & \text{if } y \in A, \\
        g(y), & \text{if } y \in B.
    \end{cases}
\end{aligned}
\end{equation}
\end{example}
Note that~$X \sqcup Y$ is different from~$Y \sqcup X$, but the two are isomorphic~(\cref{fig:e16}).

\begin{figure}[h!]
    \centering
    \includesag{30_dpcatfig_e16}
    \caption{$X \sqcup Y$ and~$Y \sqcup X$ are isomorphic. \label{fig:e16}}
\end{figure}
For the case of $\mathsf{motion}$ generation, the inclusion maps are as in \cref{fig:inclusiongas}.

\begin{figure}[h!]
    \centering
    \includesag{30_dpcatfig_inclusiongas}
    \caption{Inclusion maps for the motion generation problem. \label{fig:inclusiongas}}
\end{figure}
