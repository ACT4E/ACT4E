% !TEX root = ../CategoricalCoDesign.tex
\section{Thinking about tradeoffs}
So far, the discussion has been purely qualitative: While we discussed how
categories can describe the way that one resource can be turned into another,
this kind of modelling did not allow for quantitative statements. For example, it
is good to know that we can obtain motion from electric power, but, how fast can
we go with a certain amount of power?

To achieve a quantitative theory, we need to introduce two aspects.
First, we need to specify various degrees of resources and functionality.
For this, we will use the idea of partial orders.
Second, we need to describe how these various degrees map to each other.
For this, it is important to describe monotone maps.

\AC{ Why do we need the posets? make a concrete example. E.g. We can
have two resources, "time" and "money" that we care about when we prepare the cakes.

Introduce pareto fronts (antichain, upper sets, lower sets) with relation to
a specific example. }
\subsection{Partially ordered sets}

We will assume that functionality and resources
are \emph{partially-ordered sets} (or \emph{posets}).%


\gray{
Such ordering arise naturally in engineering as criteria for judging whether one design is better or worse than another.\AC{expand}
}

\begin{definition}[Partially ordered set]
    A \emph{partially-ordered set} is a tuple $ \tup{P, \leq}$,
where $P$ is a set (also called the \emph{carrier set}), together with a
relation $\leq$ on $P$ that is reflexive, transitive, and antisymmetric.%
\end{definition}

\todo{Recall definitions of the properties above.}



\begin{example}[Singleton poset]\label{ex:singleton}
If a set has only one element, say $\singleton$, then there is a unique order relation on it (\cref{fig:singleton}). We denote the resulting poset again by $\singleton$.
\end{example}
\begin{figure}[h!]
   \centering
   \includesag{40_dpcatfig_singleton}
   \caption{\label{fig:singleton}}
\end{figure}


\begin{example}[Booleans]\label{ex:bool}
The booleans $\Bool$ is a poset with carrier set $\{\true,\false\}$ and the order relation given by $b_1 \leq_\Bool b_2$ iff $b_1 \Imp b_2$, that is, $\false \leq_\Bool \true$. This relation should be familiar from the following table:
\[
\begin{array}{cc|ccc}
a & b & a \leq_\Bool b & a \wedge b & a \vee b \\ \hline
\true&\true&\true&\true&\true\\
\true&\false&\false&\false&\true\\
\false&\true&\true&\false&\true\\
\false&\false&\true&\false&\false
\end{array}
\]
In addition to the operation $\Imp\colon\Bool\times \Bool\to\Bool$, called \emph{implies}, there are also the familiar \emph{and} ($\wedge$) and \emph{or} ($\vee$) operations. Note that $\wedge$ and $\vee$ are commutative ($b\wedge c = c\wedge b$), whereas $\Imp$ is not.
\end{example}

\GZ{
\begin{figure}[h!]
   \centering
   %\includegraphics[scale=0.33]{dpcatfig_boolean}
   \includesag{40_dpcatfig_boolean}
   \caption{\label{fig:boolean}}
\end{figure}}
\AC{The problem with the above figure is that it is not clear what direction is $\leq$. 
Also what direction is $\Rightarrow$.}


\begin{example}[Reals]
The real numbers $\mathbb{R}$ is a poset with carrier $a\in \mathbb{R}$ and order relation given by the usual ordering $r_1 \leq r_2$.
\end{example}


A \emph{Hasse diagram} is a way to visualize a poset by being economical about the arrows that we use. In a Hasse diagram elements are points, and if $a \leq b$ then $a$ is drawn lower than $b$ and with an edge connected to it~(\cref{fig:dpcatfig_hasse}).

\GZ{
Category theory is all about organizing and layering structures. Given systems $A$ and $B$, we say that $A\leq B$ if, whenever $x$ is connected to $y$ in $A$, then $x$ is connected to $y$ in $B$.}


\begin{example}
\label{ex:hasseinclusion}

\AC{I would move this example after Set and do Set + inclusion to give an example
of two categories with same objects but different morphisms}
\GZ{but we don't have a notion of order there, right?}

\GZ{Given a set $X$ consider its power set $\powerset{X}$. We can give it an order by inclusion of subsets. Consider $X=\{a,b,c\}$: its powerset $\powerset{X}$ can be depicted as:}
\begin{center}
\includesag{40_dpcatfig_power}
\end{center}
\end{example}

\begin{figure}[h!]
   \centering
   \includesag{40_dpcatfig_hasse}
   \caption{\label{fig:hasse}}
\end{figure}

\begin{example}[Discrete partially ordered sets]
\label{ex:discreteposet}
{Every set $X$ can be considered as a \emph{discrete poset} $(X,=)$. Discrete posets are represented as collection of points:}
\begin{center}
\includesag{40_discrete}
\end{center}
\end{example}

\AC{Move later}
\begin{definition}[Opposite of a poset]

    The \emph{opposite} of a  poset $\langle A, \leq \rangle $ is the poset $\langle A\op, \leq\op\rangle $ that has the same elements as~$A$ and the reverse ordering.
    For a given~$x \in A$, we use~$x^*$ to represent its corresponding copy in~$A\op$;
    note that~$x$ and~$x^*$ are distinct.
    Reversing the order means that, for all $x,y\in A$,
    \begin{equation}
        x \leq y \quad \Leftrightarrow \quad y^* \leq\op x^*.
    \end{equation}

\end{definition}

\begin{figure}[h!]
   \centering
   \includesag{40_dpcatfig_opposite}
   \caption{\label{fig:opposite}}
\end{figure}


\begin{example}[Credit and debt]
   Let us define the set $\text{USD}=\{\$0.00,\$0.01,\$0.02,\dots\}$
   of all US dollars monetary quantities approximated to the cent.
   From this set we can define two posets:
       $\text{USD}^{+} = \tup{\text{USD}, \leq}$
       and $\text{USD}^{-} = \tup{\text{USD}, \geq}$
       that are opposite of each other.
   If the context is that, given two quantities~$\$1$ and~$\$2$,
   we prefer~$\$1$ to~$\$2$ (for example because it is a cost to pay to acquire a component), then we are working in~$\text{USD}^{+}$,
   otherwise we are working in~$\text{USD}^{-}$ (for example
   because it represents the price at which we are selling our product).

   Traditionally, in double-entry ledger systems, the numbers were not
   written with negative signs, but rather in color: red and black.
   From this convention we get the idioms ``being in the black''
   and ``being in the red''.

\end{example}


\subsection{Staring at Pareto fronts}
\AC{Design = compromise = pareto front.
Draw some pareto front (for continuous, discrete poset) and 
use those diagrams to explain the notions.
All these notions are very graphical.}

\subsubsection{Chains and Antichains} \label{sec:chains-antichains}


\begin{definition}[Chain in a poset]
\label{def:chain}
Given a poset $S$, a \emph{chain} is a sequence of points ${s_i}$ in~$S$ where two successives points are comparable:
\begin{equation}
    i \leq j \Rightarrow s_i \leq s_j.
\end{equation}
\end{definition}

 
\begin{definition}[Antichain in a poset]
\label{def:antichain}
An \emph{antichain} is a subset $S$ of a poset where no elements are comparable. If $a,b \in S$, then $a \leq b$ implies $a=b$.
\end{definition}

We denote the set of antichains of a poset $P$ by $\antichains P$.

\GZ{\begin{example}
Let's consider the poset $\tup{P,\leq}$ where $a\leq b$ if $a$ is a divisor of $b$ and $P=\{1,5,10,11,13,15\}$. A chain of $P$ is $\{1,5,10,15\}$. An antichain of $P$ is $\{10,11,13\}$.
\end{example}

\begin{example}
Consider \cref{ex:hasseinclusion}. Examples of chains are 
\begin{equation}
    \{\varnothing,\{a\},\{a,b\},\{a,b,c\}\}, \quad  \{\varnothing,\{b\},\{b,c\},\{a,b,c\}\}.
\end{equation}

Examples of antichains are
\begin{equation}
    \{\{a\},\{b\},\{c\}\}, \quad \{ \{a,b\},\{a,c\}, \{b,c\}\}.
\end{equation}
\end{example}}

\todo{add here comments, examples and figures}

\subsubsection{Upper and lower sets}

% also called cone
\begin{definition}[Upper set]
\label{def:upperset}An upper set is a subset $U$ of a poset $X$ such
that, if a point is inside, all points above it are inside as well.
In formulas:
\begin{equation}
\text{$U$ is an upperset} \equiv \forall x\in U, \forall y\in X\colon x\leq y \Imp y\in U.
% \begin{prooftree}
%      \AxiomC{x\in U}
%      \AxiomC{x \leq y}
%      \BinaryInfC{y \in U}
%  \end{prooftree}
\end{equation}
\end{definition}

\begin{definition}[Lower set]
\label{def:lowerset}
\todo[inline]{as above}
\end{definition}

\todo[inline]{Examples}

\subsubsection{From antichains to uppersets and viceversa}


\begin{definition}[Upper closure]
\label{def:upperclosure}
Given a point $p$ in a poset $P$, its upper closure $\uparrow p$ is the subset of $P$ that is ``above'' $p$:
    \begin{equation}
        \uparrow p = \{ q\in P \colon p \leq q\}.
    \end{equation}
    The upper closure of a set is the union of the upper closures of its points:
    \begin{equation}
        \uparrow S = \bigcup_{p\in S}\uparrow p.
    \end{equation}
\end{definition}

\AC{And lower closure}

\AC{Equivalently: upper set is closed to upper closure}

\AC{Give the functions that convert one to another and explain the subtletlies.
Example: take ${x\in R : x>0}$ as a poset that is not downward closed.}

\begin{definition}
\label{def:Min}
$\Min \colon \powerset(P) \to \antichains P$ is the monotone map that sends a subset $S$ of a poset to the minimal elements of that subset, i.e., those elements $a \in S$ such that $a \leq b$ for all $b \in S$.
\end{definition}
\AC{this can be empty}
\begin{definition}
\label{def:Max}
$\Max \colon \powerset(P) \to \antichains P$ is the monotone map that sends a subset $S$ of a poset to the maximal elements of that subset, i.e., those elements $a \in S$ such that $a \geq b$ for all $b \in S$.
\end{definition}
\AC{this can be empty}


\todo{Before that, define monotone map, power set notation, power poset relation, antichains relations}


\begin{definition}[Downward closed set]
\label{def:downward-closed-upperset}
An upper set $S$ is downward-closed in a poset $P$ if
\begin{equation}
    S =\, \uparrow \Min S
\end{equation}
The set of downward-closed upper sets of $P$ is denoted $\dcuppersets P$.
\end{definition}

\GZ{
\begin{example}[Upper and lower sets in $\Bool$]
The booleans $\{\false, \true \}$ form a preorder with $\false \leq \true:(\Bool,\leq)$ . The subset $\{\false\} \subseteq \Bool$ is not an upper set, since $\false \leq \true$ and $\true \notin \{\false \}$.	
\end{example}}

\todo{add here comments, examples and figures}