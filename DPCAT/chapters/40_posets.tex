% !TEX root = ../CategoricalCoDesign.tex
\section{Towards a quantitative theory: Posets and related notions}








\subsection{Why the category $\Cat{Pos}$ is not sufficient for design theory}


The category $\Pos$ of posets and monotone maps that we have described
can model many facts that are useful for design theory; however, there are also
limitations which motivate us to describe a more general category.
This section describes usefulness and limitations of $\Pos$.

\begin{example}[Battery]
    Consider the model of a battery where the capacity is the functionality
    and the mass of the battery is the resource. There is certainly
    a monotone map from capacity to mass; this map answers the question: ``Given a value of the capacity, what is the minimum mass needed?''. Conversely,
    in the other direction, the map that answers the question: ``Given a certain mass, what is the maximum capacity that can be provided?'' is also
    a monotone map.
\end{example}

\begin{figure}[h!]
    \centering
    % \includegraphics[scale=0.33]{dpcatfig_e2}
    \todo{Battery example}
    \caption{\label{fig:battery-example}}
\end{figure}

Therefore, at first sight it might seem that posets and monotone maps
would be sufficient to describe a quantitative theory of design.
However, there are more general relations to be modeled. It is easy enough
to describe examples in which having a simple map from functionality
to resources is not sufficient.

\begin{example}
Consider the design of a delivery drone, in which the functional
requirement is that the drone should be able to make a delivery
at a distance~$d$ and we need to reason about how powerful to make
it; in particular, we need to choose at what (average) velocity $v$ the drone  should travel and what is the optimal \emph{endurance} (mission duration). The relation between distance $d$, velocity $v$ and endurance~$T$ is $d=vT$. We can choose
to have either a fast drone and short missions, or a slow drone
and long missions. This is an interesting trade-off. Flying fast takes more energy, both for propulsion as well as computation (more things to observe). Flying too slow will also consume excessive energy because of the long mission duration.

If we consider $v$ and $T$ as given, then the map $v,T \mapsto vT$ is clearly a monotone function that gives the distance which the drone can cover. However, in the other direction, we do not have a simple map, but rather a 1-to-many relation. For each fixed value of the distance, there is an entire continuum of values of $v$ and $T$ which we can choose~(\cref{fig:drone-example-antichain}).

\begin{figure}[h!]
    \centering
    % \includegraphics[scale=0.33]{dpcatfig_e2}
    \todo{Drone example and antichain}
    \caption{\label{fig:drone-example}
    \label{fig:drone-example-antichain}}
\end{figure}

\end{example}

In other words, using $\Cat{Pos}$ it is not possible to make a theory of \emph{trade-offs}. The
next section introduces a more general category, called  the category $\Cat{DP}$ of
\emph{design problems}, which will allow to describe such a theory.
