% !TEX root = ../CategoricalCoDesign.tex
\section{Monotonicity is a fact of life}
\subsection{Monotone maps}
\AC{Add some motivation from design. E.g. generalization of "non-decreasing" and "non-increasing" from real numbers to partial orders}
\todo{I would skip the "isomorphism" definition because we can't use it yet. Move to the end}
\begin{definition}[Monotone map]
A monotone map between two posets
$\tup{A, \leq_A}$ and $\tup{B, \leq_B}$ is a map that preserves the ordering, in the sense that 
\begin{equation}
 a \leq_A b \quad \Imp \quad f(a) \leq_B f(b).
\end{equation}

\noindent A monotone map is an \emph{isomorphism} if the other direction
of the implication holds as well:
\begin{equation}
 a \leq_A b \quad \Leftrightarrow \quad f(a) \leq_B f(b).
\end{equation}
\end{definition}
\todo{the below uses "preorder" but we never defined it.}
\begin{remark}
Given a preorder $A$, $\id_A$ is monotone, since for $a_1,a_2\in A$, one has:
\begin{equation}
a_1\leq_A a_2 \Imp a_1=a_1\then \id_A \leq_A a_2\then \id_A = a_2.
\end{equation}
\end{remark}



\begin{example}
In \cref{ex:hasseinclusion} we presented the poset arising from the power set of a set $A=\{a,b,c\}$ and ordered via subset inclusion. The map $\vert \cdot \vert \colon \powerset(A)\to \mathbb{N}$ (cardinality), is a monotone map (\cref{fig:cardinality}).
\begin{figure}[h!]
\begin{center}
\includesag{40_dpcatfig_exmonotone}
\end{center}
\caption{The map from the power set of a set to cardinality is a monotone map. \label{fig:cardinality}}
\end{figure}
\end{example}

\noindent Note that monotonicity is a compositional property.

\begin{lemma}Given three posets $A, B, C$ and two monotone maps $f\colon A
\to B$ and  $g\colon B \to C$, the composite map $f\then g\colon  A \to C$ is
monotone as well.
\end{lemma}
\begin{proof}
Consider $a_1,a_2 \in A$, $b_1,b_2\in B$. We have 
\begin{equation}
\begin{aligned}
        a_1\leq_A a_2 &\Imp f(a_1)\leq_B f(a_2)\\ 
        b_1\leq_B b_2 &\Imp g(b_1)\leq_C g(b_2).
\end{aligned}
\end{equation}
By substituting the above, one has
\begin{equation}
    a_1\leq_A a_2 \Imp (f\then g)(a_1) \leq_C (f\then g)(a_2),
\end{equation}
which is the monotonicity condition for $(f\then g)$.
\end{proof}

\begin{lemma}
Consider a discrete poset $A$ and a poset $B$. Any map $f\colon A\to B$ is monotone.
\end{lemma}
\begin{proof}
Since $A$ is a discrete poset, one has
\begin{equation}
    a_1\leq_A a_2 \iff a_1=a_2.
\end{equation}
Therefore, one has
\begin{equation}
\begin{aligned}
    a_1\leq_A a_2 &\Imp a_1=a_2\\
    &\Imp f(a_1)=f(a_2)\\
    &\Imp f(a_1)\leq_B f(a_2).
\end{aligned}
\end{equation}
\end{proof}
Unless indicated otherwise, in this paper all maps between two posets
are assumed to be monotone or will turn out to be monotone.

\subsection{The category $\Pos$ of posets and monotone maps}
In the previous section, we described a poset as a category, where the objects are elements of the poset and the morphisms are given by the poset order. In this section, we want to lift this concept and describe a category in which the objects are posets themselves, and the morphisms are monotone functions between them. This category is called $\Pos$.

\begin{definition}[Category $\Pos$]
    The category $\Pos$ is defined by:
    \begin{compactenum}
    \item \emph{Objects}: The objects of this category are all posets.
    \item \emph{Morphisms}: The morphisms between any pair of posets $X, Y$
    are the monotone maps from $X$ to $Y$.
    \item \emph{Identity morphism}:  The identity morphism for the poset $X$
    is the identity function $\text{Id}_X$.
    \item \emph{Composition operation}: The composition operation is function
    composition.
    \end{compactenum}
\end{definition}

Occasionally we will write $f \colon A \to_{\Pos} B$ to emphasize that a monotone map between posets is a morphism in the category of posets, called~$\Pos$. Note that morphisms in $\Pos$ are morphisms between posets, which we previously defined as categories. The notion of ``morphisms between categories'' is formally described through functors.


\subsection{Why the category $\Cat{Pos}$ is not sufficient for design theory}


The category $\Pos$ of posets and monotone maps that we have described
can model many facts that are useful for design theory; however, there are also
limitations which motivate us to describe a more general category.
This section describes usefulness and limitations of $\Pos$.

\begin{example}[Battery]
    Consider the model of a battery where the capacity is the functionality
    and the mass of the battery is the resource (\cref{fig:battery-example}). There is certainly
    a monotone map from capacity to mass; this map answers the question: ``Given a value of the capacity, what is the minimum mass needed?''. Conversely,
    in the other direction, the map that answers the question: ``Given a certain mass, what is the maximum capacity that can be provided?'' is also
    a monotone map.
\end{example}

\begin{figure}[h!]
    \centering
    \begin{tikzcd}
    \bullet &\arrow[l] \bullet\\[-15pt]
    \text{mass} & \text{capacity}
    \end{tikzcd}
    \caption{Example of the design of a battery. \label{fig:battery-example}}
\end{figure}

Therefore, at first sight it might seem that posets and monotone maps
would be sufficient to describe a quantitative theory of design.
However, there are more general relations to be modeled. It is easy enough
to describe examples in which having a simple map from functionality
to resources is not sufficient.

\begin{example}
Consider the design of a delivery drone, in which the functional
requirement is that the drone should be able to make a delivery
at a distance~$d$ and we need to reason about how powerful to make
it. In particular, we need to choose at what (average) velocity $v$ the drone  should travel and what is the optimal \emph{endurance} (mission duration). The relation between distance $d$, velocity $v$ and endurance~$T$ is $d=vT$. We can choose
to have either a fast drone and short missions, or a slow drone
and long missions. This is an interesting trade-off. Flying fast takes more energy, both for propulsion as well as computation (more things to observe). Flying too slow will also consume excessive energy because of the long mission duration.

If we consider $v$ and $T$ as given, then the map $v,T \mapsto vT$ is clearly a monotone function that gives the distance which the drone can cover. However, in the other direction, we do not have a simple map, but rather a 1-to-many relation. For each fixed value of the distance, there is an entire continuum of values of $v$ and $T$ which we can choose~(\cref{fig:drone-example-antichain}).

\begin{figure}[h!]
    \centering
    % \includegraphics[scale=0.33]{dpcatfig_e2}
    \todo{Drone example and antichain}
    \caption{\label{fig:drone-example}
    \label{fig:drone-example-antichain}}
\end{figure}

\end{example}

In other words, using $\Cat{Pos}$ it is not possible to make a theory of \emph{trade-offs}. The
next section introduces a more general category, called  the category $\Cat{DP}$ of
\emph{design problems}, which will allow to describe such a theory.
\todo{We need a way to go back}
    
