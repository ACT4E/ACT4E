% !TEX root = ../CategoricalCoDesign.tex
\section{Thinking about restrictions and alternatives}
\subsection{Lattice}
\begin{definition}[Lattice]
\label{def:lattice}
A lattice is a poset $\tup{P, \leq}$ with some additional properties:
\begin{enumerate}
    \item Given two points $p, q \in P$, it is always possible to define their least upper bound, called \emph{join}, and indicated as $p \vee q$.
    \item Given two points $p, q \in P$, it is always possible to define their greatest lower bound, called \emph{meet}, and indicated as $p \wedge q$.
\end{enumerate}
\end{definition}

If there is a least upper bound for the entire lattice, it is called
the \emph{top}. If the greatest lower bound exists it is called the \emph{bottom}.

\begin{example}
    In \cref{ex:hasseinclusion} we presented the poset arising from the power set of a set $A$ and ordered via subset inclusion. This is a lattice, bounded by $A$ and by the empty set $\emptyset$. 
\end{example}

\begin{example}
Consider the set $\{1,2,3,6\}$ ordered by divisibility is a lattice. However, the set $\{1,2,3\}$ ordered by divisibility is not, since 2 and 3 lack a meet.
\begin{center}
\includesag{40_dpcatfig_exlattice}
\end{center}
\end{example}
\todo{add here comments, examples and figures}
\subsection{Union of Design Problems}

We define the \emph{union} $f \vee g$ to be the design problem which is feasible whenever \emph{either} $f$ or $g$ is feasible. This models $f$ and $g$ as interchangeable technologies; either one can replace the other.

\begin{definition}[Union of design problems]
Given two design problems $f \colon \F{A} \tickar \R{B}$ and $g \colon \F{A} \tickar \R{B}$, their \emph{union} is denoted $f \vee g \colon \F{A} \tickar \R{B}$ and defined by
\begin{equation}
\begin{aligned}
(f \vee g)\colon \F{A}\op \times \R{B} & \toinPos \Bool \\
\tup{\F{a}^*, \R{b}} & \mapsto f(\F{a}^*, \R{b}) \vee g(\F{a}^*, \R{b}).
\end{aligned}
\end{equation}
\begin{equation}
    \includesag{52_union}
\end{equation}
\end{definition}

\begin{example}
Jeb's Spaceship Parts is locked in a deadly rivalry with Starshow Bob to supply engines for the new X103 space orbiter. Neither knows the exact operational scenario that the X103 will encounter, but have provided a range of performance benchmarks for their engines,
\begin{comment}
\[
    \centering
    \includesag{50_rival_jeb}
    \quad
        \includesag{50_rival_bob}
\]
\end{comment}
\[
\includesag{50_rival_jeb_bob}
\]
Back at NASA headquarters, Beau has uploaded Jeb and Bob's data in order to construct the design problem
\[
      \includesag{50_rival_beau}
\]
\end{example}

%The union $f \vee g$ can be decomposed in terms of the \emph{biproduct} $f + g$ in $\DP$, which can be thought of as the \emph{maximally} disjunctive way of combining $f$ and $g$.

\subsection{Intersection of Design Problems}
Given two design problems $f, g \colon \F{A} \tickar \R{B}$, we can define a design problem $f \wedge g$ that is feasible if only if $f$ and $g$ are both feasible. We call $f \wedge g$ the \emph{intersection} of $f$ and $g$. One interpretation of $f \wedge g$ is that $f$ and $g$ are two slightly different models of the same process, and we want to make sure that the design is conservatively feasible for both models.

\begin{definition}[Intersection of design problems]
Given two design problems $f\colon \F{A} \tickar \R{B}$ and $g\colon \F{A} \tickar \R{B}$,
their \emph{intersection} is denoted $(f \wedge g)\colon \F{A} \tickar \R{B}$ and defined by
\begin{equation}
	\begin{aligned}
		(f \wedge g)\colon \F{A}\op \times \R{B} & \toinPos \Bool \\
		\tup{\F{a}^*, \R{b}} & \mapsto f(\F{a}^*, \R{b}) \wedge  g(\F{a}^*, \R{b}).
	\end{aligned}
\end{equation}

\begin{equation}
    \includesag{52_intersection}
\end{equation}
\end{definition}

We can directly generalize the intersection $f \wedge g$ by allowing $f$ and $g$ to have different domain and codomains, $f \colon \F{A} \tickar \R{B}$ and $g \colon \F{C} \tickar \R{D}$. We call this putting two design problems ``in parallel''.
\subsubsection{Relationship between intersection and monoidal prodcut}
To define the precise relationship between the monoidal product $f \otimes g$ (\cref{def:monoidalproduct}) and the intersection $f \wedge g$, we first define two operations, $\text{split} \colon \F{A} \tickar \R{A} \times \R{A}$ and $\text{fuse} \colon \F{A} \times \F{A} \tickar \R{A}$, which correspond to splitting and fusing wires in a diagram:
\begin{equation}
\begin{aligned}
    \text{split} \colon \F{A}\op \times (\R{A} \times \R{A}) &\to_\Pos \Bool \\
    \tup{\F{x}^*, \tup{\R{y}, \R{z}}} &\mapsto (\F{x} \leq_A \R{y}) \wedge (\F{x} \leq_A \R{z})
\end{aligned}
\end{equation}
~
\begin{equation}
\begin{aligned}
    \text{fuse} \colon (\F{A} \times \F{A})\op \times \R{A} &\to_\Pos \Bool \\
    \tup{ \tup{\F{x}^*, \F{y}^*}, \R{z}} &\mapsto (\F{x} \leq_A \R{z}) \wedge (\F{y} \leq_A \R{z}).
\end{aligned}
\end{equation}

\begin{lemma}
\label{lemma:intersection}
For $f,g \colon \F{A} \tickar \R{B}$,
\begin{equation}
f \wedge g = \text{split} \then f \otimes g \then \text{fuse},
\end{equation}
\end{lemma}
or, in diagrams:
\begin{equation}
\includesag{50_split_1_2}
\end{equation}
\GZ{
\begin{proof}
    Recall that $(f\otimes g)(\tup{a_1,a_2}^*,\tup{b_1,b_2})=f(a_1^*,b_1)\wedge g(a_2^*,b_2)$. Furthermore, one has
    \begin{equation*}
        \begin{aligned}
            (f\otimes g \then \text{fuse})(\tup{a_1,a_2}^*,b)&=\bigvee_{\tup{b',b''}\in B\times B}(f(a_1^*,b')\wedge g(a_2^*,b''))\wedge ((b'\leq b) \wedge (b''\leq b))\\
            &=\underbrace{f(a_1^*,b)\wedge g(a_2^*,b)}_{\star}.
        \end{aligned}
    \end{equation*}
    Furthermore, one has
    \begin{equation*}
        \begin{aligned}
        (\text{split}\then \star)(a^*,b)&=\bigvee_{\tup{a',a''}\in A\times A} \text{split}(a^*,\tup{a',a''})\wedge (f(a',b)\wedge g(a'',b))\\
        &=\bigvee_{\tup{a',a''}\in A\times A}(a^*\leq a')\wedge (a^*\leq a'')\wedge f(a',b)\wedge g(a'',b)\\
        &=f(a^*,b)\wedge g(a^*,b)\\
        &=(f\wedge g)(a^*,b).
        \end{aligned}
    \end{equation*}
\end{proof}}

\subsection{Biproduct, Product, and Coproduct of Design Problems}
\begin{example}
\todo{maybe last diagram leave out, adding decision variable to resources}
Just as Beau is about to connect it to the larger design problem for the X103, his supervisor, Elly May, comes up behind him and catches a glance at his diagram. ``Beau, that's almost right: Jeb-XX $\wedge$ Bob-Roc does indeed work as an approximation of a generic ``engine'' design problem. The problem is that the choice of engine is so important and expensive that it isn't up to the engineering team---it's up to the politicians! So take $f \vee g$ out and stick this one in there instead: 
\[
\includesag{50_rival_pol}
\]
We'll come back and adjust the parameter to either the Jeb XX or the Bob-Roc after the politicians make their choice.''
% even we're not crazy enough to swap out an engine in the middle of a launch! We need to decide beforehand whether we're going with Jeb or Bob. Chuck that thing out and put this one in there instead:

\end{example}
\begin{definition}[Biproduct of design problems]
Given two design problems $f\colon \F{A} \tickar \R{B}$ and $g\colon \F{C} \tickar \R{D}$, their \emph{biproduct} $(f + g)\colon \F{A} + \F{C} \tickar \R{B} + \R{D}$ is defined by
\begin{equation}
\begin{aligned}
    (f + g)\colon (\F{A} + \F{C})\op  \times (\R{B} + \R{D}) & \toinPos \Bool,  \\
            \tup{ \F{a}^*, \R{b}} & \mapsto f(\F{a}^*, \R{b}), \\
            \tup{ \F{c}^*, \R{b}} & \mapsto \false, \\
            \tup{ \F{a}^*, \R{d}} & \mapsto \false, \\
            \tup{ \F{c}^*, \R{d}} & \mapsto g(\F{c}^*, \R{d}).
\end{aligned}
\end{equation}
\begin{equation}
    \includesag{52_biproduct}
\end{equation}
In particular, when $f, g\colon \F{A} \tickar \R{B}$, $(f + g)\colon \F{A} + \F{A} \tickar \R{B} + \R{B}$ is defined by
\begin{equation}
\begin{aligned}
    (f + g) \colon (\F{A} + \F{A})\op  \times (\R{B} + \R{B}) & \toinPos \Bool,  \\
            \tup{ \tup{\F{1}, \F{a}}^*, \tup{\R{1}, \R{b}}} & \mapsto f(\F{a}^*, \R{b}), \\
            \tup{ \tup{\F{2}, \F{a}}^*, \tup{\R{1}, \R{b}}} & \mapsto \false, \\
            \tup{ \tup{\F{1}, \F{a}}^*, \tup{\R{2}, \R{b}}} & \mapsto \false, \\
            \tup{ \tup{\F{2}, \F{a}}^*, \tup{\R{2}, \R{b}}} & \mapsto g(\F{a}^*, \R{b}).
\end{aligned}
\end{equation}
\end{definition}


Assume $f,g \colon \F{A} \tickar \R{B}$. Intuitively, $f+g$ can be thought of as: Pick either $f$ or $g$, then throw away the other one, whereas on any $\tup{\F{a}^*,\R{b}}$, $f \vee g$ always picks the better (more feasible) of either $f(\F{a}^*,\R{b})$ or $g(\F{a}^*,\R{b})$. Note that $f+g$ introduces an extra parameter, since the choice of $f$ or $g$ has to be hard-coded into the larger design problem.


In general, the biproduct is defined first on objects of the category, so what we are calling a biproduct of design
problems is actually the unique map derived from the biproduct $A + B$ of posets in $\DP$, namely the disjoint union. In the more general setting $f\colon \F{A} \tickar \R{B}$ and $g \colon \F{C} \tickar \R{D}$, the biproduct on objects $\F{A} + \F{C}$ models $\F{A}$ and $\F{C}$ as \emph{interchangeable entities}: $f+g$ may output one functionality from either $\F{A}$ or $\F{C}$, and similarly it requires only one resource from either $\R{B}$ or $\R{D}$. We can emphasize one or the other condition by letting $\F{A} = \F{C}$ or $\R{B} =\R{D}$ above, thus deriving the product of design problems and coproduct of design problems, respectively.


\begin{definition}[Coproduct of design problems]
\label{define:coproduct}
Given two design problems $f\colon \F{A} \tickar \R{C}$ and $g \colon \F{B} \tickar \R{C}$, their \emph{coproduct} $(f \sqcup g)\colon \F{A} + \F{B} \tickar \R{C}$ is defined by
\begin{equation}
\begin{aligned}
    (f \sqcup g) \colon (\F{A} + \F{B})\op \times \R{C} & \toinPos \Bool,  \\
            \tup{ \F{a}^*, \R{c}} & \mapsto f(\F{a}^*, \R{c}), \\
            \tup{ \F{b}^*, \R{c}} & \mapsto g(\F{b}^*, \R{c}).
\end{aligned}
\end{equation}
\begin{equation}
    \includesag{52_coproduct}
\end{equation}
\end{definition}

\begin{definition}[Product of design problems]
\label{define:product}
Given design problems $f\colon \F{C} \tickar \R{A}$ and $g \colon \F{C} \tickar \R{B}$, their \emph{product} $(f \times g)\colon \F{C} \tickar \R{A} + \R{B}$ is defined by
\begin{equation}
\begin{aligned}
    (f \times g) \colon \F{C}\op  \times (\R{A} + \R{B}) & \toinPos \Bool,  \\
            \tup{\F{c}^*, \R{a}} & \mapsto f(\F{c}^*, \R{a}), \\
            \tup{\F{c}^*, \R{b}} & \mapsto g(\F{c}^*, \R{b}).
\end{aligned}
\end{equation}
\begin{equation}
    \includesag{52_product}
\end{equation}
\end{definition}


To show that $A + B$ is in fact a biproduct in the categorical sense, we need to show that it satisfies certain universal properties; these properties guarantee that the biproduct is the most ``efficient'' way of combining two objects in $\DP$ in a certain sense, and that the resulting combination is unique when it exists. Specifically, we need to show that $A + B$ is both a product $(+, \pi_A, \pi_B)$ and a coproduct $(+, \iota_A, \iota_B)$ in $\DP$, satisfying an extra coherence condition: $\pi_B \circ \iota_B = \id_B$.

We recall the definitions of product and coproduct from Section~\ref{sec:additionalconstructions}.

\todo{Ask AC: I would put here these definitions, we don't want to have a block in section 2}

\begin{shaded}
\begin{definition}[Initial object]
Let $\CatC$ be a category and let $A \in \CatC$ be an object. We say that $A$ is an \emph{initial object} if, for all $B \in\CatC$, the hom-set $\Hom_\CatC(A,B)$ has exactly one element. We say that $A$ is a \emph{terminal object} if, for all $D\in\CatC$, the hom-set $\Hom_\CatC(D,A)$ has exactly one element.
\end{definition}

\begin{definition}[Finite coproducts and products]
We say that $\CatC$ \emph{has finite coproducts} if it has an initial object and every pair of objects in $\CatC$ has a coproduct.
We say that $\CatC$ \emph{has finite products} if it has a terminal object and every pair of objects in $\CatC$ has a product.
\end{definition}
\end{shaded}

\begin{example}
    The category $\Poset$ has finite coproducts and finite products.
    The coproduct $\sqcup_\Poset$ is the disjoint union $+$.
    The product $\times_\Poset$ is the Cartesian product $\times$.
\end{example}

\begin{lemma}
    The category $\DP$ has finite coproducts and finite products.
    The coproduct $\sqcup_\DP$ is the disjoint union $+$ (\cref{def:disjoint-union}).
    The product $\times_\DP$ is also the disjoint union $+$ (\cref{def:cartesian-product}).
\end{lemma}

\begin{proof}
Let $\iota_A$ be the injection of posets $A \hookrightarrow A+B$. We define the design problem $\comp{\iota_A} \colon \F{A} \tickar \R{A} + \R{B}$ by returning true on $(\F{a}^*,\R{c})$ iff $\iota_A(\F{a}) \leq_{A+B} \R{c}$ for $\R{c} \in A + B$ and false otherwise, and similarly for $\comp{\iota_B}$.

Recall the coproduct of design problems in Definition~\ref{define:coproduct}. To show that $A+B$ is a coproduct (of posets) in $\DP$, we need to show that the coproduct $f \sqcup g$ of design problems is unique and that it satisfies $f = \comp{\iota_A} \then f \sqcup g$ and $g = \comp{\iota_B} \then f \sqcup g$. This can be verified by simply writing out the composition.


The proof that the disjoint union $+$ is also a categorical product in $\DP$ is analogous, now with $f \colon \F{X} \tickar \R{A}$, $g \colon \F{X} \tickar \R{B}$, and replacing the injections $\iota_A$ with projections $\comp{\pi_A} \colon \F{A}+\F{B} \tickar \R{A}$ and the coproduct $f \sqcup g$ with the product (on morphisms) $f \times g \colon \F{X} \tickar \R{A} + \R{B}$ from Definition~\ref{define:product}.
\end{proof}

\begin{remark}Where it is clear from context, we will not distinguish between the injection of posets $\iota_A$ and its corresponding design problem $\comp{\iota_A}$, and refer to both by $\iota_A$. Similarly with the projection $\pi_A$ and $\comp{\pi_A}$.
\end{remark}

\begin{table}[b]
\begin{tabular}{ccccccccccc}
    category &
    objects & morphisms &
    product (ob) & product (morph) & coproduct (ob) & coproduct (morph) & biproduct (morph) & tensor product &
    initial  & terminal\\
    \hline

    $\Cat{Set}$ &
    sets & functions &
    Cartesian product & (not used) & disjoint union & (not used) &
    none & $\times$ \emph{or} $\sqcup$ &
    $\emptyset$ & $\{\ast\}$ \\

    $\Cat{Pos}$ &
    posets & monotone maps &
    Cartesian product & (not used) & disjoint union & (not used) &
    none & $\times$ &
    $\emptyset$  & $\{\ast\}$\\

    $\DP$ &
    posets & design problems &
    disjoint union & $\times_\DP$ & disjoint union & $\sqcup_\DP$  &
    disjoint union & $\times$ &
    $\emptyset$ & $\{\ast\}$\\

    % category &
    % objects & morphisms &
    % product & coproduct & biproduct & tensor product
    % initial  & terminal\\
\end{tabular}
\caption{A comparison of $\Pos, \Set$, and $\DP$.}
\end{table}

\begin{shaded}
\begin{definition}[Zero object and morphism]
We call an object that is both initial and terminal the \emph{zero object} and we indicate it as $0$. For any other pairs of objects  $A, B\in\CatC$, there is a unique morphism of the form $A \to 0\to B$; we call it the \emph{zero morphism} and denote it $0_{A,B}\colon A \to B$.
\end{definition}
\end{shaded}

\begin{example}The category $\Poset$ has an initial object $\emptyset$ and a terminal object $\singleton$, but no zero object.\end{example}

\begin{lemma}
In the category $\DP$, the zero object is the empty poset $\emptyset$. The zero morphism $0_{\F{\cP},\R{\cQ}}\colon \F{\cP} \tickar \R{\cQ}$ is the design problem that is always infeasible (see \cref{def:zero}).
\end{lemma}
\begin{proof}
    The zero object is the empty poset $\emptyset$. Indeed, for any poset $\cP $, there is a unique design problem $\F{\emptyset} \tickar \R{\cP} $ given by the unique monotone map $\emptyset=\emptyset\op\times \cP \to\Bool$. Similarly, there is a unique design problem $\F{\cP} \tickar \R{\emptyset}$, so $\emptyset$ is both initial and terminal. For any posets $\cP ,\cQ $, the zero map $0 \colon \F{\cP} \tickar \R{\cQ}$ is the monotone map $\cP \op\times \cQ \toinPos \Bool$ sending everything to $\false$.
\end{proof}

\begin{shaded}
\begin{definition}[Biproduct]
Let $\CatC$ be a category with a zero-object $0$, and let $C_1,C_2\in\CatC$ be objects. Suppose that $\tup{D,\pi_1,\pi_2,\iota_1,\iota_2}$ is a tuple such that $\tup{D,\pi_1,\pi_2}$ is a product of $C_1$ and $C_2$ and $\tup{D,\iota_1,\iota_2}$ is a coproduct of $C_1$ and $C_2$. We say that it is a \emph{biproduct} if, for each $1\leq i,j\leq 2$, we have
\begin{equation}
(\iota_i\then \pi_j)=
\begin{cases}
	\id_{C_i}, &\text{ if }i=j,\\
	0, &\text{ if }i\neq j,
\end{cases} \label{eq:biproduct-condition}
\end{equation}
as maps $C_i\to C_j$ in $\CatC$.
\end{definition}
\end{shaded}

\begin{lemma}
The disjoint union is a biproduct for $\DP$.
\end{lemma}
\begin{proof}
    We have already shown that the disjoint union is both
    a product and a coproduct. We just need to verify that~\eqref{eq:biproduct-condition} holds
    for the design problems $\pi_A, \pi_B, \iota_A, \iota_B$.
    This accounts to checking the four conditions:
    \begin{equation}
    \begin{aligned}
        (\iota_A\then \pi_A) &= \id_{A}, \\
        (\iota_A\then \pi_B) &= 0_{A,B}, \\
        (\iota_B\then \pi_A) &= 0_{B,A},\\
        (\iota_B\then \pi_B) &= \id_{B}.
    \end{aligned}
    \end{equation}
    We check only the first two, as the other two are similar.

    To check the second condition, we compute an explicit expression for $(\iota_A\then \pi_B)\colon \F{A} \tickar \R{B}$, using the definition
    of design problem interconnection:
    \begin{equation}
    \begin{aligned}
        (\iota_A\then \pi_B) \colon  \F{A}\op\times \R{B} & \toinPos \Bool \\
                            \tup{\F{x}^*, \R{z}} &\mapsto
                            \bigvee_{y \in A + B} \iota_A(\F{x}^*,\R{y}) \wedge \pi_B(\F{y}^*,\R{z}).
    \end{aligned}
    \end{equation}
    We can do a case analysis for $y\in A+B$. Suppose $y\in A$.
    Then $\pi_B(\F{y}^*,\R{z}) = \false$. If $y \in B$, then $\iota_A(\F{x}^*,\R{y}) = \false$.
    Therefore, the sum is always false, and hence $(\iota_A\then \pi_B) = 0_{A,B}$.

    For the first condition, we compute  an explicit expression for $(\iota_A\then \pi_A) \colon \F{A} \tickar \R{A}$:
    \begin{equation}
    \begin{aligned}
        (\iota_A\then \pi_A) \colon  \F{A}\op\times \R{A} & \toinPos \Bool \\
                            \tup{\F{x}^*, \R{z}} &\mapsto
                            \bigvee_{y \in A + B} \iota_A(\F{x}^*,\R{y}) \wedge \pi_A(\F{y}^*,\R{z}).
    \end{aligned}
    \end{equation}
    Note that we are summing over $y \in A + B$. Let us divide the sum in two parts:
    \begin{equation}
    \bigvee_{y \in A} \iota_A(\F{x}^*,\R{y}) \wedge \pi_A(\F{y}^*,\R{z}) \quad\vee\quad
    \bigvee_{y \in B} \iota_A(\F{x}^*,\R{y}) \wedge \pi_A(\F{y}^*,\R{z}).
    \end{equation}
    For $y \in B$, $\iota_A(\F{x}^*,\R{y})$ and $\pi_A(\F{y}^*,\R{z})$ are both false, and we can therefore ignore the second sum.
    From the definition of $\iota_A$ we have that for $y\in A$, $ \iota_A(\F{x}^*,\R{y})=\F{x} \leq_A \R{y}$, and from the definition of $\pi_A$ we have that for $y\in A$, $\pi_A(\F{y}^*,\R{z})=\F{y} \leq_A \R{z}$. Thus we compute:
\begin{equation}
    \begin{aligned}
    [\iota_A\then \pi_A](\F{x}^*, \R{z}) &= \bigvee_{y \in A} \iota_A(\F{x}^*,\R{y}) \wedge \pi_A(\F{y}^*,\R{z})  \\
     &= \bigvee_{y \in A} (\F{x} \leq_A \R{y}) \wedge  (\F{y} \leq_A \R{z}) \\
     &= \F{x} \leq_A \R{z}\\
     &= \id_A(\F{x}^*, \R{z}).
\end{aligned}
\end{equation}
\end{proof}
