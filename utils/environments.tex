% !TEX root = ../../ACT4E-full.tex

\surroundwithmdframed[
  backgroundcolor=none,
  topline=false,
  rightline=false,
  bottomline=false,
  leftmargin=2pt,
  skipabove=0pt,
  skipbelow=0pt
]{proof}
%\surroundwithmdframed[
%  backgroundcolor=none,
%  topline=false,
%  rightline=false,
%  bottomline=false,
%  leftmargin=2pt,
%  skipabove=0pt,
%  skipbelow=0pt
%]{remark}
%\newenvironment{remark}{\textbf{Remark}}{\medskip}

\theoremstyle{definition}
\newtheorem{theorem}{Theorem}[chapter]
\newtheorem*{theorem*}{Theorem}

\newtheorem{proposition}[theorem]{Proposition}
\newtheorem*{proposition*}{Proposition}
\newtheorem{corollary}[theorem]{Corollary}
\newtheorem{lemma}[theorem]{Lemma}

\newtheorem*{lemma*}{Lemma}
\newtheorem*{problem}{Problem}
\newtheorem{remark}[theorem]{Remark}

\theoremstyle{definition}

\newtheorem{protodefinition}[theorem]{Definition}

% \newenvironment{definition}
%    {\colorlet{shadecolor}{blue!25}\begin{shaded}\begin{protodefinition}} % light blue for lemmas only
%    {\end{protodefinition}\end{shaded}}
\newenvironment{ctdefinition}
{\colorlet{shadecolor}{LightBlue!50}\begin{shaded*}
                                      \begin{protodefinition}} % light blue for lemmas only
                                      {\end{protodefinition}
\end{shaded*}}
\newtheorem{definition}[theorem]{Definition}
\newtheorem*{definition*}{Definition}
% \newtheorem{ctdefinition}[theorem]{Definition}


\newtheorem{construction}[theorem]{Construction}
\newtheorem{example}[theorem]{Example}

%\theoremstyle{remark}
\newtheorem*{warning}{Warning}
%\newtheorem{remark}[theorem]{Remark}
%\newtheorem{warning}[theorem]{Warning}


\newcounter{exercise}[part]
\newtheorem{protoexcercise}[exercise]{Exercise}
\newcounter{gradedexercise}[part]
\newtheorem{protogradedexcercise}[gradedexercise]{Graded exercise}

\newenvironment{gradedexercise}
{\colorlet{shadecolor}{LightGreen!50}\begin{shaded*}
                                      \begin{protogradedexcercise}} %
                                      {\end{protogradedexcercise}
\end{shaded*}}


\newenvironment{body}{\par}{}

\newcounter{codeexercise}[part]
\newtheorem{protocodeexercise}[codeexercise]{Code exercise}

\newenvironment{codeexercise}
{\colorlet{shadecolor}{LightPink!50}\begin{shaded*}
                                      \begin{protocodeexercise}} %
                                      {\end{protocodeexercise}
\end{shaded*}}

\newenvironment{gradedsolution}
{\colorlet{shadecolor}{LightPink!20}\begin{shaded*}}{\end{shaded*}}






% We cannot do this if we put figures inside of examples.
% \surroundwithmdframed[
%   topline=false,
%   rightline=false,
%   bottomline=false,
%   %leftmargin=\parindent,
%   %skipabove=\medskipamount,
%   %skipbelow=\medskipamount
% ]{example}

\newcommand{\classlisting}[1]{%
\begin{longcode}
\centering%
\caption{The \classname{#1} interface.}%
\vspace{-0.5cm}
\label{lst:#1}%
\classsource{#1}{}%
\end{longcode}%
}

\newcommand{\solutionof}[1]{%
\instructors{%
    \IfFileExists{\rootdir/ACT4E-solutions/CatProductPowerSet.tex}{
        \IfFileExists{\rootdir/ACT4E-solutions/#1.tex}{
            \input{\rootdir/ACT4E-solutions/#1.tex}
        }{
            \textbf{You need to create file ACT4E-solutions/#1.tex}
        }
    }{
        \textbf{You need to checkout the repo ACT4E-solutions}
    }
}%
}


\newcommand{\linkvideo}[1]{%
    
\begin{minipage}{10cm}
    \href{https://ACT4E.github.io/ACT4E/videos/spring2021-functors:semi-and-fun:cat-of-cat.html}{Watch online video (6 minutes).}
        
    \href{https://ACT4E.github.io/ACT4E/videos/spring2021-functors:semi-and-fun:cat-of-cat.html}{\includegraphics[height=3.5cm]{spring2021-functors:semi-and-fun:cat-of-cat/thumbnails.jpg}}
    \href{https://ACT4E.github.io/ACT4E/videos/spring2021-functors:semi-and-fun:cat-of-cat.html}{\includegraphics[height=2.5cm]{spring2021-functors:semi-and-fun:cat-of-cat/qrcode.png}}
\end{minipage}
%
}


%
\newcommand{\watch}[5]{%
\marginnote{
    \href{#1}{Watch: \emph{#2} (#3).}\\
    \href{#1}{\fbox{\includegraphics[height=2.5cm]{#4}}}\\%
    \href{#1}{\includegraphics[height=2cm]{#5}}%
}%
}

%
%\newcommand{\watch}[5]{%
%\begin{marginfigure}
%    \href{#1}{Watch \emph{#2}.\\#3}\\
%    \href{#1}{\fbox{\includegraphics[height=2cm]{#4}}}\\%
%    \href{#1}{\includegraphics[height=2cm]{#5}}%
%\end{marginfigure}
%}
%
%\newcommand{\watch}[5]{%
%    \href{#1}{Watch \emph{#2}.\\#3}\\
%    \href{#1}{\fbox{\includegraphics[height=2cm]{#4}}}\\%
%    \href{#1}{\includegraphics[height=2cm]{#5}}%
%}

% https://ethidsc.atlassian.net/browse/ACT4EBOOK-264
\newenvironment{publictodo}
{\colorlet{shadecolor}{LightPink!50}\begin{shaded*}
                                      } %
                                      {
\end{shaded*}}

%\newcommand\sectionbreak{\clearpage}

%\sectionfont{\clearpage}
%\let\oldsectionformat\sectionformat
%\renewcommand*{\sectionformat}{HELLO\clearpage H\oldsectionformat}
%\usepackage{titlesec}
%\titleformat{\section}[hang]{
%    \usefont{T1}{qhv}{b}{n}\selectfont} % "qhv" - TeX Gyre Heros, "b" - bold
%    {}
%    {0em}
%   {\hspace{-0.4pt}\Large \thesection\hspace{0.6em}}
\let\oldsection\section
\renewcommand\section{\FloatBarrier\clearpage\oldsection}
