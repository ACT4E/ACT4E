% !TEX root = ../../ACT4E-full.tex

% operators

%:section:dpcat/Operators: operators

%\DeclareMathOperator{\id}{id}
\newcommand{\id}{\operatorname{id}}


% \DeclareMathOperator{\Lax}{Lax}
\newcommand{\Lax}{\operatorname{Lax}}

% \DeclareMathOperator*{\Min}{Min}
\newcommand{\Min}{\operatorname*{Min}}
% \DeclareMathOperator*{\Inf}{Inf}
\newcommand{\Inf}{\operatorname*{Inf}}
% \DeclareMathOperator*{\Sup}{Sup}
\newcommand{\Sup}{\operatorname*{Sup}}
% \DeclareMathOperator*{\Max}{Max}
\newcommand{\Max}{\operatorname*{Max}}

\newcommand{\Ob}{\operatorname{Ob}}
% \DeclareMathOperator{\Ob}{\mathsf{Ob}}
\DeclareMathOperator{\MOb}{\lvert\mspace{2mu}\cdot\mspace{2mu}\rvert}
% \DeclareMathOperator{\Tr}{Tr}
\newcommand{\Tr}{\operatorname{Tr}}

%:section:dpcat/colors: Color
\definecolor{functionality}{rgb}{0.094869,0.500000,0.000000}
\definecolor{requirements}{rgb}{0.555789,0.000000,0.000000}
\definecolor{implementations}{RGB}{214,120,5}
\definecolor{functionalitylight}{rgb}{0.094869,0.500000,0.000000}
\definecolor{requirementslight}{rgb}{0.555789,0.000000,0.000000}
\definecolor{implementationslight}{RGB}{214,120,5}

\definecolor{setcolor}{rgb}{0.61, 0.87, 1.0}
\definecolor{darkgreen}{rgb}{0.0, 0.5, 0.0}
\definecolor{darkred}{rgb}{0.5, 0.0, 0.0}
\definecolor{brightpink}{rgb}{1.0, 0.65, 0.79}
\definecolor{applegreen}{rgb}{0.55, 0.71, 0.0}
\definecolor{custompurple}{rgb}{0.18,0.46,0.74}
\definecolor{custompink}{rgb}{0.73,0.83,0.91}

\def\dpgreen{darkgreen}
\def\dpred{darkred}

\definecolor{darkblue}{rgb}{0.0, 0.0, 0.55}
\definecolor{shadecolor}{named}{LightBlue}
\definecolor{transmuter}{RGB}{46,117,189}
\definecolor{transmuted}{RGB}{213,87,96}

\definecolor{upcolor}{named}{Purple}
\definecolor{downcolor}{named}{Orange}
\definecolor{antichaincolor}{named}{Brown}
% color commands
\newcommand{\colR}{\color{requirements}}
%:example: $\colR X$
\newcommand{\colF}{\color{functionality}}
%:example: $\colF X$
\newcommand{\colI}{\color{implementations}}
%:example: $\colI X$
\newcommand{\colH}{\color[rgb]{0.000000,0.400000,1.000000}}
%:example: $\colH X$
\newcommand{\colU}{\color{purple}}
%:example: $\colU X$
\newcommand{\colL}{\color{orange}}
%:example: $\colL X$
\newcommand{\colUp}{\color{upcolor}}
%:example: $\colUp X$
\newcommand{\colAnti}{\color{antichaincolor}}
%:example: $\colAnti X$
\newcommand{\colDown}{\color{downcolor}}
%:example: $\colDown X$
\newcommand{\colTransmuter}{\color{transmuter}}
%:example: $\colTransmuter X$
\newcommand{\colTransmuted}{\color{transmuted}}
%:example: $\colTransmuted X$
\newcommand{\blue}[1]{\textcolor{blue}{#1}}
\newcommand{\F}[1]{{\color{\dpgreen}#1}}
\newcommand{\Rdia}[1]{\color{\dpred}#1}
\newcommand{\R}[1]{{\color{\dpred}#1}}
\newcommand{\Fdia}[1]{\color{\dpgreen}#1}

\newcommand{\Rcol}[1]{{\colR #1}}
%:example: \Rcol{X}
\newcommand{\Fcol}[1]{{\colF #1}}
%:example: \Fcol{X}
\newcommand{\Icol}[1]{{\colI #1}}
%:example: \Icol{X}
\newcommand{\gray}[1]{{\color{gray}#1}}
%:example: \gray{X}
\newcommand{\bchanges}{\color[rgb]{0,0.3,0}}
%:example: \bchanges{X}
\newcommand{\changes}[1]{{\color[rgb]{0,0.3,0}#1}}
%:example: \changes{X}
\newcommand{\echanges}{\color[rgb]{0,0,0}}
%:example: \echanges{X}

%:section:dpcat/comments: Personal comments
\newcommand{\AC}[1]{{\color{blue}AC: #1}}
\newcommand{\GZ}[1]{{\color{green}GZ: #1}}
\newcommand{\DS}[1]{{\color{blue!50!red}DS says: #1}}
\newcommand{\JT}[1]{{\color{blue!30!green!30!black}JT: #1}}
%:example: \JT{blah}
\newcommand{\JL}[1]{{\color{magenta}JL: #1}}
%:example: \JL{blah}
%:section:dpcat/markers: stuff missing
\newcommand{\XXX}{{\color{red}XXX}}
\newcommand{\citeXXX}{{\color{red}[cite]}}
\newcommand{\todographics}[1]{\todo[color=red!70]{#1}}
%:example: \todographics{blah}
\newcommand{\todotext}[1]{\todo[color=red!50]{#1}}
%:example: \todotext{blah}
\newcommand{\todostructure}[1]{\todo[color=red!10]{#1}}
%:example: \todostructure{blah}

%:section:dpcat/status: Status markers
\newcommand{\readytoreview}[1]{\texorpdfstring{\color{darkgreen}{#1}}{#1}}
%:example: \readytoreview{blah}
\newcommand{\statusdraft}[1]{\texorpdfstring{\color{orange}{#1} [draft]}{#1}}
%:example: \statusdraft{blah}
\newcommand{\statusmissing}[1]{\color{darkred}{#1} [missing]}
%:example: \statusdraft{blah}

%\renewcommand{\readytoreview}[1]{#1}
%\renewcommand{\statusdraft}[1]{#1}
%
% Not sure why it does not work
%\newcommand{\readytoreview}[1]{%
%  \ifthenelse{\boolean{statuscolors}}{%
%      {\color{darkgreen}#1}%
%  }{%
%    #1%
%  }%
%}
%\newcommand{\statusdraft}[1]{%
%  \ifthenelse{\boolean{statuscolors}}{%
%      {\color{orange} #1  [draft] }%
%  }{%
%    #1%
%  }%
%}

% \DeclareFontFamily{U}{mathx}{\hyphenchar\font45}
% \DeclareFontShape{U}{mathx}{m}{n}{
%       <5> <6> <7> <8> <9> <10>
%       <10.95> <12> <14.4> <17.28> <20.74> <24.88>
%       mathx10
%       }{}
% \DeclareSymbolFont{mathx}{U}{mathx}{m}{n}
% \DeclareFontSubstitution{U}{mathx}{m}{n}
% \DeclareMathAccent{\widecheck}{0}{mathx}{"71}



%:section:dpcat/cat-commands: Category commands

\newcommand{\cat}[1]{\mathsf{#1}}
\newcommand{\Cat}[1]{\ensuremath{\mathbf{#1}}\xspace}

\newcommand{\Hom}{\operatorname{Hom}}
\newcommand{\HomSet}[3]{\Hom_{#1}\left({#2};{#3}\right)}


%:section:dpcat/categories-generic: Generic categories
\newcommand{\CatA}{\Cat{A}}
\newcommand{\CatB}{\Cat{B}}

\newcommand{\CatC}{\Cat{C}}
\newcommand{\CatD}{\Cat{D}}
\newcommand{\CatV}{\Cat{V}}
\newcommand{\HomC}{\Hom_{\CatC}}
\newcommand{\HomD}{\Hom_{\CatD}}

%:section:dpcat/named-categories: Named categories

\newcommand{\DP}{\Cat{DP}}
\newcommand{\UDP}{\Cat{DP}}
\newcommand{\DPI}{\Cat{DPI}}
\newcommand{\Draw}{\Cat{Draw}}

\newcommand{\Bool}{\Cat{Bool}}

\newcommand{\Berg}{\Cat{Berg}}
\newcommand{\Bergama}{\Cat{BergAma}}
\newcommand{\Berglazy}{\Cat{BergLazy}}

\newcommand{\Curr}{\Cat{Curr}}
\newcommand{\Category}{\Cat{Cat}}
\newcommand{\Vect}{\Cat{Vect}}
\newcommand{\FinVect}{\Cat{FinVect}}

\newcommand{\Rel}{\Cat{Rel}}
\newcommand{\FinSet}{\Cat{FinSet}}
\newcommand{\Set}{\Cat{Set}}
\newcommand{\Prof}{\mathbb{P}\Cat{rof}}
\newcommand{\Pos}{\Cat{Pos}}

\newcommand{\Injset}{\Cat{InjSet}}

\newcommand{\Graph}{\Cat{Grph}}

\newcommand{\Plans}{\Cat{Plans}}

\newcommand{\LTI}{\Cat{LTI}}
\newcommand{\Free}{\Cat{Free}}
\newcommand{\feas}{\Cat{Feas}}
\newcommand{\Trek}{\Cat{Trek}}
\newcommand{\Car}{\Cat{Car}}
\newcommand{\Flight}{\Cat{Flight}}



%:section:dpcat/transmutation: Transmutation

\newcommand{\transmuter}[1]{\textbf{\colTransmuter #1}\xspace}
\newcommand{\transmuted}[1]{\textbf{\colTransmuted #1}\xspace}
\newcommand{\technology}[1]{\mathsf{#1}}
\newcommand{\motor}{\transmuter{motor}}
\newcommand{\move}{\transmuter{move}}
\newcommand{\dynamo}{\transmuter{dynamo}}
\newcommand{\wheels}{\transmuter{wheels}}
\newcommand{\electricpower}{\transmuted{electricity}}
\newcommand{\rotationalmotion}{\transmuted{rotation}}
\newcommand{\translationalmotion}{\transmuted{translation}}

%:section:dpcat/sets: Well-known sets.
\newcommand{\reals}{\mathbb{R}}
\newcommand{\natnumbers}{\mathbb{N}}
\newcommand{\ratnumbers}{\mathbb{Q}}
\newcommand{\nonNegReals}{\reals_{\geq0}}
\newcommand{\nonNegRealsComp}{\overline{\reals}_{\geq0}}

\newcommand{\Rcomp}{\nonNegRealsComp}


\newcommand{\Rleq}{\mathrel{\MS{\leq}}}
\newcommand{\Nleq}{\mathrel{\MS{\leq}}}


%:section:dpcat/lower-upper-anti: Lower/upper sets, antichains
\newcommand{\antichains}{\mathcal{\colAnti A}}
%:def:def:antichain
\newcommand{\lowersets}{\mathscr{\colDown L}}
%:def:def:lowerset
\newcommand{\uppersets}{\mathscr{\colUp U}}
%:def:def:upperset
\newcommand{\powerset}{\mathscr{P}}

\newcommand{\dcuppersets}{\underline{\uppersets}} % Downward-closed upper sets


\newcommand{\upit}{{\colUp \pmb{\uparrow}\,}}
\newcommand{\downit}{{\colDown \pmb{\downarrow}\,}}
\newcommand{\Up}[1]{\uppersets#1}


%:section:dpcat/codesign-spaces: Co-design spaces
\newcommand{\LF}{\lowersets\funsp}
\newcommand{\UR}{\Up\ressp}

\newcommand{\Aressp}{{\antichains\ressp}}
\newcommand{\Uressp}{\UR}






%% B %%

\newcommand{\book}[1]{\ifthenelse{\boolean{bookversion}}{#1}{}}


%:section:dpcat/well-known-functions: Well-known functions


\newcommand{\ceil}[1]{\left \lceil #1 \right \rceil}
\newcommand{\funceil}{\textsf{ceil}}


%:section:dpcat/dom-codomain: domain/codomain/limits
\DeclareMathOperator{\cod}{cod}
\DeclareMathOperator*{\colim}{colim\;}
\DeclareMathOperator{\Coll}{Col}
\DeclareMathOperator{\dom}{dom}

%:section:dpcat/misc: Misc

\newcommand{\col}[1]{\mathrm{col(#1)}}
\newcommand{\companion}[1]{\hat{#1}}
\newcommand{\comp}[1]{\widehat{#1}}
\newcommand{\conjoint}[1]{\check{#1}}
\newcommand{\conj}[1]{\widecheck{#1}}
\newcommand{\Conw}{\aword{Conw}}
\newcommand{\coprodMap}[2]{{#1}\mathbf{+}{#2}}
\newcommand{\cP}{P}
\newcommand{\cQ}{Q}
\newcommand{\cR}{R}
%\newcommand{\Ctimes}{ \tikz[baseline=-.55ex] \node [inner sep=0pt,cross out,draw,line width=1pt,minimum size=1ex] (a) {};}
\newcommand{\Ctimes}{\mathbf{\times}}
\newcommand{\Cplus}{\mathbf{+}}



%% D %%



\DeclareMathOperator{\dvert}{Vert}

\newcommand{\dpiconwidth}{1.5cm}


%% E %%



%% F %%

%\newcommand{\feasibleset}[1]{F_{#1}}
%\newcommand{\fix}{\text{fix}}

%\newcommand{\fupd}{f^\mathrm{upd}}
%\newcommand{\frdt}{f^\mathrm{rdt}}


%:section:dpcat/functors: default functors

\newcommand{\funa}{F}
\newcommand{\funb}{G}
\newcommand{\func}{H}
\newcommand{\fund}{I}
\newcommand{\funfloor}{\textsf{floor}}


%:section:dpcat/morphisms: morphisms
\newcommand{\mora}{f}
\newcommand{\morb}{g}
\newcommand{\morc}{h}
\newcommand{\mord}{i}


%% N %%

%:section:dpcat/natural: natural transformations
\newcommand{\ntrafoa}{\alpha}
\newcommand{\ntrafob}{\beta}
\newcommand{\ntrafoc}{\gamma}
\newcommand{\ntrafod}{\delta}
\newcommand{\naturals}{\mathbb{N}}

%:section:dpcat/objects: objects
\newcommand{\ObC}{\Ob_{\CatC}}
\newcommand{\ObD}{\Ob_{\CatD}}
\newcommand{\One}{\mathbf{1}}
\newcommand{\Obja}{X}
\newcommand{\Objb}{Y}
\newcommand{\Objc}{Z}
\newcommand{\Objd}{W}
\newcommand{\obja}{x}
\newcommand{\objb}{y}
\newcommand{\objc}{z}
\newcommand{\objd}{w}


\newcommand{\op}{^{\mathrm{op}}}
\newcommand{\otimesD}{\otimes_{\CatD}}

%% P %%

\newcommand{\para}{\text{par}}


\newcommand{\prodMap}[2]{{#1}\times{#2}}


\newcommand{\qqand}{\qquad\text{and}\qquad}
 
\newcommand{\rtntte}{\textsf{rtntte}}
 

\newcommand{\singleton}{\{1\}}
\newcommand{\snack}[1]{\mathsf{#1}}

%% T %%





\newcommand{\textF}[1]{\text{\F{#1}}}
\newcommand{\textR}[1]{\text{\R{#1}}}
% \DeclareMathOperator{\then}{\fatsemi}
% \DeclareMathOperator{\after}{\circ}

\newcommand{\then}{\mathbin{\fatsemi}}
\newcommand{\after}{\mathbin{\circ}}
\newcommand{\thing}[1]{\text{#1}}



\newcommand{\twisted}[1]{\mathrm{Tw}\left(#1\right)}


%% U %%
\newcommand{\Unit}[1]{\id_{#1}}

\newcommand{\upp}{\mathsf{Upp}}
\newcommand{\ubar}[1]{\underaccent{\bar}{#1}}


\newcommand{\unc}{\mathsf{Unc}}

%:section:dpcat/tuples: supper for tuples

% support for tuples
\newcommand{\pair}[2]{\langle#1,#2\rangle}
%:example: $\pair{a}{b}$
\newcommand{\tup}[1]{\left\langle#1\right\rangle}
%:example: $\tup{\posA, \posAleq}$
\newcommand{\tupp}[1]{\langle#1\rangle}
%:example: $\tupp{\posA, \posAleq}$

% The one below supposedly would allow splitting tuples over newlines,
% but I could not make it work - AC
%\makeatletter
%\newcommand\tup[1]{%
%  \@tempcnta=0
%  \left\langle
%  \@for\@ii:=#1\do{%
%    \@insertbreakingcomma
%    \@ii
%  }%
%  \right\rangle
%}
%\def\@insertbreakingcomma{%
%  \ifnum \@tempcnta = 0 \else\,,\ \linebreak[1] \fi
%  \advance\@tempcnta\@ne
%}
%\makeatother

%:section:dpcat/booleans: Booleans

\newcommand{\true}{\top}
\newcommand{\false}{\bot}

%:section:dpcat/arrows: Arrows


\newcommand{\slashedrightarrow}{\relbar\joinrel\relbar\joinrel\mapstochar\joinrel\rightarrow}
\newcommand{\profto}{\mathrel{\slashedrightarrow}}

\newcommand{\toinPos}{\to_{\Pos}}
\newcommand{\toiso}{\overset{\sim}{\to}}
\newcommand\too{\longrightarrow}
\newcommand{\To}[1]{\xrightarrow{#1}}
%:example: $a \To f b$

\newcommand{\Imp}{\Rightarrow}
\newcommand{\embedsin}{\hookrightarrow}


%:section:dpcat/dp: DP
\newcommand{\fun}{\ensuremath{{\colF f}}\xspace}
\newcommand{\res}{\ensuremath{{\colR r}}\xspace}
\newcommand{\imp}{\ensuremath{{\colI i}}\xspace}
\newcommand{\funsp}{\ensuremath{{\colF F}}\xspace}
\newcommand{\ressp}{\ensuremath{{\colR R}}\xspace}
\newcommand{\impsp}{\ensuremath{{\colI I}}\xspace}
\newcommand{\prov}{{\colF\aword{prov}}} % provides
\newcommand{\req}{{\colR\aword{req}}} % requires
%:section:dpcat/dp/top-bottom: Top and bottom
\newcommand{\restop}{\top_{\ressp}}
\newcommand{\resbot}{\bot_{\ressp}}
\newcommand{\funtop}{\top_{\funsp}}
\newcommand{\funbot}{\bot_{\funsp}}
 
%:section:dpcat/orders: Symbols for orders

\newcommand{\poslt}{\prec}
\newcommand{\posgeq}{\succeq}
\newcommand{\ordleq}{\preceq}
\newcommand{\posleq}{\preceq}
\newcommand{\ordgeq}{\succeq}
\newcommand{\leqP}{\posleq_{\cP}}
\newcommand{\leqQ}{\posleq_{\cQ}}

\newcommand{\funleq}{\posleq_{\funsp}}
\newcommand{\resleq}{\posleq_{\ressp}}
\newcommand{\fungeq}{\posgeq_{\funsp}}
\newcommand{\resgeq}{\posgeq_{\ressp}}



\newcommand{\adp}{d}



% antichains


\newcommand{\resMin}{{\Min_{\resleq}}}


%:section:dpcat/abbrevs: Abbreviations 
\newcommand{\etal}{{et\,al.}\xspace} % avoid!
\newcommand{\eg}{{e.g.},\xspace}% avoid!
\newcommand{\etc}{{etc.}\xspace}% avoid!
\newcommand{\ie}{{i.e.},\xspace}% avoid!


\newcommand{\subto}{\text{s.t.}} % used in optimization problem
\newcommand{\with}{\text{using}} % used in optimization problem





\newcommand{\cdpiN}{\mathcal{V}}
\newcommand{\cdpin}{v}
\newcommand{\cdpinA}{v_1}
\newcommand{\cdpinB}{v_2}
\newcommand{\cdpiresind}{i}
\newcommand{\cdpifunind}{j}
\newcommand{\cdpiresindA}{i_1}
\newcommand{\cdpifunindB}{j_2}
\newcommand{\dpinumf}{\textrm{n}_f}
\newcommand{\dpinumr}{\textrm{n}_r}
\newcommand{\cdpinnumf}{{\dpinumf}_{\cdpin}}
\newcommand{\cdpinnumr}{{\dpinumr}_{\cdpin}}
\newcommand{\cdpiE}{\mathcal{E}}




\newcommand{\unconnectedfun}{\mathsf{UF}} % XXX: not a good choice
\newcommand{\unconnectedres}{\mathsf{UR}}

\newcommand{\rtof}{\ensuremath{{\colH \varphi}}\xspace}
\newcommand{\ftor}{\ensuremath{{\colH h}}\xspace}
\newcommand{\ftoR}{\ensuremath{{\colH H}}\xspace}
\newcommand{\ftorL}{\ensuremath{{\colL h_L}}\xspace}
\newcommand{\ftorU}{\ensuremath{{\colU h_U}}\xspace}

% R with units
\newcommand{\Runit}[1]{\reals^{\!\!\textrm{[#1]}}}

%\renewcommand{\Aressp}{{\colR\antichains\ressp}}


\newcommand{\scottcontinuous}{Scott continuous\xspace}
\newcommand{\scottcontinuity}{Scott continuity\xspace}

\newcommand{\posA}{\Psi}

\newcommand{\posAleq}{\mathrel{{\posleq_\posA}}}

\newcommand{\posAMin}{\mathop{{\Min_{\posAleq}}}}

\newcommand{\posAmin}{\mathop{{\min_{\posAleq}}}}
\newcommand{\posAmax}{\mathop{{\max_{\posAleq}}}}


% Typesetting


\newcommand*{\vcenteredhbox}[1]{\begingroup
\setbox0=\hbox{#1}\parbox{\wd0}{\box0}\endgroup}

\newcommand{\captionsideleft}[2]{
    \medskip
    \begin{minipage}{1.8cm}{
        \hfill
        \protect\captionof{figure}{#1}}\end{minipage}
    \begin{minipage}{6.6cm}
        \vcenteredhbox{{#2}}
        \hfill
    \end{minipage}
    \medskip
}

\newcommand{\triv}{\mathsf{Triv}}

%:section:dpcat/uncertainty: Uncertainty paper

\newcommand{\ufloor}{{\colL\aword{floor}}}
\newcommand{\uceil}{{\colU\aword{ceil}}}

%\captionsetup[subtable]{font={rm,md,sc,footnotesize},position=top}
%\setlength{\abovecaptionskip}{0pt}
%\setlength{\belowcaptionskip}{0pt}
%
%\setlength{\columnsep}{10pt}
%\setlength{\intextsep}{2pt}
%

% udps

\newcommand{\udpa}{\boldsymbol{u}_a}
\newcommand{\udpb}{\boldsymbol{u}_b}
\newcommand{\udpL}{{\colL \boldsymbol{\mathsf{L}}}}
\newcommand{\udpU}{{\colU \boldsymbol{\mathsf{U}}}}
\newcommand{\udpsp}{\UDP}
\newcommand{\udpleq}{\posleq_{\udpsp}}

\newcommand{\dpsp}{\DP}
\newcommand{\dpleq}{\posleq_\dpsp}


%:section:dpcat/units: Units
\newcommand{\currency}[1]{\textbf{\colTransmuted #1}\xspace}
\newcommand{\USD}{\currency{USD}}
\newcommand{\SGD}{\currency{SGD}}
\newcommand{\CHF}{\currency{CHF}}
\newcommand{\EUR}{\currency{EUR}}

%\newcommand{\stdcurr}{\currency{BTC}} % not transmuted
\def\bitcoinA{%
  \leavevmode
  \vtop{\offinterlineskip %\bfseries
    \setbox0=\hbox{B}%
    \setbox2=\hbox to\wd0{\hfil\hskip-.03em
    \vrule height .3ex width .15ex\hskip .08em
    \vrule height .3ex width .15ex\hfil}
    \vbox{\copy2\box0}\box2}}
\newcommand{\stdcurr}{\bitcoinA{}} %




\newcommand{\chapterfirst}[3]{
\vfill%
\begin{minipage}{\textwidth}%
#3%
\end{minipage}%
\vfill%
\par%
\minitoc%
\par%
\vfill%
\includegraphics[width=\textwidth]{#1}
\par%
\smallskip%
\begin{minipage}{\textwidth}%
\small #2%
\end{minipage}%
\vfill
\clearpage%
}


%:section:dpcat/dp-symbols: DP

\newcommand{\dprob}{\aword{dp}}
\newcommand{\dpseries}{\aword{series}}
\newcommand{\dppar}{\aword{par}}
\newcommand{\dploop}{\aword{loop}}

\newcommand{\dploopb}{\aword{loopb}}
\newcommand{\terms}{\aword{Terms}}
%
\newcommand{\udpsem}{\Phi}
%
\newcommand{\dpsem}{\varphi}
%
\newcommand{\atoms}{\mathcal{A}}
\newcommand{\atree}{\boldsymbol{\aword{T}}}
\newcommand{\val}{\boldsymbol{v}}
%
\newcommand{\ops}{\aword{ops}}
%
%
\newcommand{\acprod}{\mathbin{\boldsymbol{\times}}}
%
\newcommand{\oploop}{\dagger}
\newcommand{\opseries}{\mathbin{\varocircle}}
\newcommand{\oppar}{\mathbin{\varotimes}}
\newcommand{\opcoprod}{\mathbin{\varovee}}
%
\newcommand{\UId}{\aword{UId}}
\newcommand{\vdc}{\aword{vdc}}


\newcommand{\idFunc}{\aword{Id}}
\newcommand{\tableColors}{\rowcolors{2}{green!4}{blue!4}}
\newcommand{\tableColorsTwo}{\rowcolors{3}{green!4}{blue!4}}


%:section:dpcat/domain: Domain theory

\newcommand{\lfp}{\aword{lfp}}
\newcommand{\CPOs}{\textsc{CPO}s\xspace}
\newcommand{\CPO}{\textsc{CPO}\xspace}
\newcommand{\DCPOs}{\textsc{DCPO}s\xspace}
\newcommand{\DCPO}{\textsc{DCPO}\xspace}
\newcommand{\posB}{\mathcal{Q}}
%\newcommand{\posC}{\mathcal{R}}


\newcommand{\posetwidth}{\aword{width}}
\newcommand{\posetheight}{\aword{height}}




% need these to be here to compile the tikz separately
%:section:dpcat/db-example: Used in set example

\newcommand{\Company}{\transmuter{Company}}
\newcommand{\SanyoDenki}{\transmuter{Sanyo Denki}}
\newcommand{\Price}{\transmuter{Price}}
\newcommand{\Volume}{\transmuter{Volume}}
\newcommand{\Size}{\transmuter{Size}}
\newcommand{\Multiply}{\transmuter{Multiply}}

%:section:dpcat/dp-problems: Used in set example

\newcommand{\Feasibility}{\aword{Feasibility}}
\newcommand{\FeasibleImp}{\aword{FeasibleImp}}
\newcommand{\FixFunMinReq}{\aword{FixFunMinReq}}
\newcommand{\FixResMinFun}{\aword{FixResMinFun}}


%:section:dpcat/miscellanea: To categorize

\newcommand{\MS}[1]{{\color{blue}#1}}


\newcommand{\RR}{{\colR \alpha}} % Antichain



\newcommand{\iindex}[1]{#1\index{#1}}
\newcommand{\aword}[1]{\ensuremath{\mathsf{#1}}\xspace}
\newcommand{\idmon}{1}


\newcommand{\low}{\aword{Low}}

\newcommand{\Lo}[1]{\mathsf{L}#1}


\newcommand{\myvee}{\textstyle{\bigvee}}
\newcommand{\mycup}{\textstyle{\bigcup}}
\newcommand{\M}{\mathsf{M}}
\newcommand{\T}{\mathsf{T}}