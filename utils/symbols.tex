% !TEX root = ../../ACT4E-full.tex

%%%%%%%%%%%%%%%%%%%%%%%%% Sets
%:section:sets: Sets

\newcommand{\stylesets}[1]{{\color{setcolor}\mathbf{#1}}}
\newcommand{\styleelements}[1]{{\color{setcolor}{#1}}}
%:section:sets/generic: Generic sets and elements
\newcommand{\setA}{\stylesets{A}}
%:nomenc:\setA,\setB,\setC:Generic names for sets.
\newcommand{\setB}{\stylesets{B}}
%:nomenc-exclude:
\newcommand{\setC}{\stylesets{C}}
%:nomenc-exclude:
\newcommand{\subA}{\stylesets{S}}
%:nomenc:\subA,\subB:Generic names for subsets.
\newcommand{\subB}{\stylesets{T}}
%:nomenc-exclude:


\newcommand{\ela}{\styleelements{x}}
%:nomenc:\ela,\elb,\elc:Generic names for elements of sets.
\newcommand{\elb}{\styleelements{y}}
%:nomenc-exclude:
\newcommand{\elc}{\styleelements{z}}
%:nomenc-exclude:
\newcommand{\stylemaps}[1]{{\color{setcolor}#1}}
\newcommand{\mapa}{\stylemaps{f}}
%:nomenc:\mapa,\mapb,\mapc:Generic names for maps between sets.
\newcommand{\mapb}{\stylemaps{g}}
%:nomenc-exclude:
\newcommand{\mapc}{\stylemaps{h}}
%:nomenc-exclude:


\newcommand{\cod}{\operatorname{cod}}
\newcommand{\dom}{\operatorname{dom}}

%:section:sets/known: Well-known sets.
\newcommand{\reals}{\ensuremath{\mathbb{R}}} % Real numbers
\newcommand{\natnumbers}{\ensuremath{\mathbb{N}}} % Natural numbers: $0, 1, 2, \dots$
\newcommand{\wnumbers}{\ensuremath{\mathbb{Z}}} % Integers: $0, 1, -1, 2, -2, \dots$
\newcommand{\ratnumbers}{\ensuremath{\mathbb{Q}}} % Rational numbers
\newcommand{\posReals}{\reals_{>0}} % Positive real numbers
\newcommand{\nonNegReals}{\reals_{\geq0}} % Non-negative real numbers
\newcommand{\nonNegRealsComp}{\overline{\reals}_{\geq0}} % Completion of non-negative real numbers.

\newcommand{\Rcomp}{\nonNegRealsComp} % Not sure - need to check
%:nomenc-exclude:
\newcommand{\singletonel}{\bullet} % element of the singleton
\newcommand{\singleton}{\{\singletonel\}} % singleton set

% R with units
\newcommand{\Runit}[1]{\reals^{\!\!\textrm{[#1]}}}

%:section:sets/constructors: Constructors
\newcommand{\powerset}{\mathscr{P}}
%:nomenc:\powerset \setA: Power set of $\setA$.
%:def:ex:hasseinclusion

%:section:sets/operations: Operations
\newcommand{\cartprod}{\times} % set product
%:nomenc:\setA\cartprod\setB:Cartesian product of two sets.
%:def:def:cartesian-product
\newcommand{\setdisunion}{+} % Disjoint union
%:nomenc:\setA\setdisunion\setB:Disjoint union of two sets.
%:def:def:disjoint-union

\newcommand{\disunionA}[1]{\tup{1, #1}}
%:nomenc:\disunionA{a}, \disunionA{b}:Decorated elements of disjoint union
%:def:def:disjoint-union
\newcommand{\disunionB}[1]{\tup{2, #1}}
\newcommand{\injA}{\stylemorph{\iota}_1}
%:nomenc:\injA,\injB:Injections into $\setA \setdisunion \setB$.
%:def:def:disjoint-union
\newcommand{\injB}{\stylemorph{\iota}_2}
%:nomenc-exclude:

%%%%%%%%%%%%%%%%%%%%%%%%% Relations
%:section:relations:Relations
\newcommand{\relstyle}[1]{\mathbf{\color{blue} #1}} % style for relation (temporary)

\newcommand{\relA}{\relstyle{R}}
%:nomenc:\relA, \relB:Generic relation names.
\newcommand{\relB}{\relstyle{S}}
%:nomenc-exclude:

\newcommand{\reltransp}{^{\intercal}}
%:nomenc:\relA\reltransp:Transpose of a relation $\relA$.

%%%%%%%%%%%%%%%%%%%%%%%%% Posets
%:section:posets: Posets


%:section:posets/generic:Generic poset names
\newcommand{\posA}{\stylesets{P}}
%:nomenc:\posA, \posB, \posC: Generic posets
%:def:def:poset
\newcommand{\posB}{\stylesets{Q}}
%:nomenc-exclude:
\newcommand{\posC}{\Omega}
%:nomenc-exclude:
%:nomenc:\posAel, \posBel, \posCel: Generic elements of posets
\newcommand{\posAel}{\styleelements{\psi}}
%:nomenc-exclude:
\newcommand{\posBel}{\styleelements{\phi}}
%:nomenc-exclude:
\newcommand{\posCel}{\omega}
%:nomenc-exclude:

\newcommand{\posAleq}{\mathrel{{\posleq_\posA}}}
\newcommand{\posDPleq}{\mathrel{{\posleq_\DP}}}
%:nomenc-exclude:
\newcommand{\posAgeq}{\mathrel{{\posgeq_\posA}}}
%:nomenc-exclude:
\newcommand{\posBleq}{\mathrel{{\posleq_\posB}}}
%:nomenc-exclude:
\newcommand{\posCleq}{\mathrel{{\posleq_\posC}}}
%:nomenc-exclude:
\newcommand{\posAMin}{\mathop{{\Min_{\posAleq}}}}
%:nomenc-exclude:
\newcommand{\posAmin}{\mathop{{\min_{\posAleq}}}}
%:nomenc-exclude:
\newcommand{\posAmax}{\mathop{{\max_{\posAleq}}}}
%:nomenc-exclude:
\newcommand{\posAA}{\antichains\posA}
%:nomenc-exclude:
\newcommand{\posPA}{\powerset\posA}
%:nomenc-exclude:
\newcommand{\posUA}{\uppersets\posA}
%:nomenc-exclude:
\newcommand{\posLA}{\lowersets\posA}
%:nomenc-exclude:
\newcommand{\posAAleq}{\posleq_{\posAA}}
%:nomenc-exclude:


%:section:posets/operations-sets:Operations on sets
\newcommand{\Min}{\operatorname*{Min}}
%:nomenc:\Min_{\posAleq} \subA : Minimal elements of the subset $\subA$.
%:def:def:Min
\newcommand{\Max}{\operatorname*{Max}}
%:nomenc:\Max_{\posAleq} \subA : Maximal elements of the subset $\subA$.
%:def:def:Max


\newcommand{\Inf}{\operatorname*{Inf}}
\newcommand{\Sup}{\operatorname*{Sup}}


\newcommand{\upit}{{\colUp \pmb{\uparrow}\,}} % up closure
%:nomenc:\upit \subA: Upper closure of $\subA$.
%:def:def:upperclosure

\newcommand{\downit}{{\colDown \pmb{\downarrow}\,}} % down closure
%:nomenc:\downit \subA: Lower closure of $\subA$.
%:def:def:lowerclosure

%:section:posets/operations-elements:Operations on elements

\newcommand{\join}{\pmb{\vee}} % Join
%:nomenc:\ela \join \elb: Join of two elements $\ela$, $\elb$
%:def:def:lattice

\newcommand{\meet}{\pmb{\wedge}} % Meet
%:nomenc:\ela \meet \elb: Meet of two elements $\ela$, $\elb$
%:def:def:lattice

%:section:posets/constructors: Constructors
\newcommand{\antichains}{\mathcal{\colAnti A}} % antichain symbols
%:nomenc:\antichains \posA: Antichains of $\posA$.
%:def:def:antichain
\newcommand{\lowersets}{\mathscr{\colDown L}}
%:nomenc:\lowersets \posA: Lower sets of $\posA$.
%:def:def:lowerset
\newcommand{\uppersets}{\mathscr{\colUp U}}
%:nomenc:\uppersets \posA: Upper sets of $\posA$.
%:def:def:upperset

\newcommand{\dcuppersets}{\underline{\uppersets}} % Downward-closed upper sets
%:nomenc:\dcuppersets \posA: Downward-closed upper sets of $\posA$.
%:def:def:downward-closed-upperset


\newcommand{\Rleq}{\mathrel{\MS{\leq}}} % $\leq$ for $\reals$ (making sure to use right one)
%:nomenc-exclude:
\newcommand{\Nleq}{\mathrel{\MS{\leq}}} % $\leq$ for $\natnumbers$ (making sure to use right one)
%:nomenc-exclude:
\newcommand{\Up}[1]{\uppersets#1}
%:nomenc-exclude:

%%%%%%%%%%%%%%%%%%%%%%%%% Categories
%:section:categories: Categories

%:section:colors: Color
\definecolor{functionality}{rgb}{0.094869,0.500000,0.000000}
\definecolor{requirements}{rgb}{0.555789,0.000000,0.000000}
\definecolor{implementations}{RGB}{214,120,5}
\definecolor{functionalitylight}{rgb}{0.094869,0.500000,0.000000}
\definecolor{requirementslight}{rgb}{0.555789,0.000000,0.000000}
\definecolor{implementationslight}{RGB}{214,120,5}

\definecolor{setcolor}{named}{brown}
\definecolor{catcolor}{rgb}{0.68, 0.74, 0.74}
\definecolor{darkgreen}{rgb}{0.0, 0.5, 0.0}
\definecolor{darkred}{rgb}{0.5, 0.0, 0.0}
\definecolor{brightpink}{rgb}{1.0, 0.65, 0.79}
\definecolor{applegreen}{rgb}{0.55, 0.71, 0.0}
\definecolor{custompurple}{rgb}{0.18,0.46,0.74}
\definecolor{custompink}{rgb}{0.73,0.83,0.91}
\definecolor{funcname}{named}{blue}
\definecolor{classname}{named}{blue}

\def\dpgreen{functionality}
\def\dpred{requirements}

\definecolor{darkblue}{rgb}{0.0, 0.0, 0.55}
\definecolor{shadecolor}{named}{LightBlue}
\definecolor{transmuter}{RGB}{46,117,189}
\definecolor{morphisms}{RGB}{46,117,189}
\definecolor{transmuted}{RGB}{213,87,96}
\definecolor{naturaltransformations}{rgb}{0.1,0.5,0.0}
\definecolor{functors}{rgb}{0.5,0,0.5}

\definecolor{instructors}{RGB}{214,120,5}
\definecolor{upcolor}{named}{Purple}
\definecolor{downcolor}{named}{Orange}
\definecolor{antichaincolor}{named}{Brown}
\definecolor{staincola}{RGB}{181,23,0}
\definecolor{staincolb}{RGB}{255,150,141}
\definecolor{staincolc}{RGB}{181,23,30}
\definecolor{staincold}{RGB}{241,219,200}
\definecolor{staincole}{RGB}{92,24,13}
\definecolor{staincolf}{RGB}{29,177,0}
\definecolor{ropecola}{RGB}{187,127,14}
\definecolor{ropecolb}{RGB}{97,216,54}
% color commands
\newcommand{\colR}{\color{requirements}}
%:nomenc-exclude:
%:example: $\colR X$
\newcommand{\colF}{\color{functionality}}
%:nomenc-exclude:
%:example: $\colF X$
\newcommand{\colI}{\color{implementations}}
%:nomenc-exclude:
%:example: $\colI X$
\newcommand{\colH}{\color[rgb]{0.000000,0.400000,1.000000}}
%:nomenc-exclude:
%:example: $\colH X$
\newcommand{\colU}{\color{purple}}
%:nomenc-exclude:
%:example: $\colU X$
\newcommand{\colL}{\color{orange}}
%:nomenc-exclude:
%:example: $\colL X$
\newcommand{\colUp}{\color{upcolor}}
%:nomenc-exclude:
%:example: $\colUp X$
\newcommand{\colAnti}{\color{antichaincolor}}
%:nomenc-exclude:
%:example: $\colAnti X$
\newcommand{\colDown}{\color{downcolor}}
%:nomenc-exclude:
%:example: $\colDown X$
\newcommand{\colTransmuter}{\color{transmuter}}
%:nomenc-exclude:
%:example: $\colTransmuter X$
\newcommand{\colTransmuted}{\color{transmuted}}
%:nomenc-exclude:
%:example: $\colTransmuted X$
\newcommand{\blue}[1]{\textcolor{blue}{#1}}
%:nomenc-exclude:
\newcommand{\F}[1]{{\color{\dpgreen}#1}}
%:nomenc-exclude:
\newcommand{\Rdia}[1]{\color{\dpred}#1}
%:nomenc-exclude:
\newcommand{\R}[1]{{\color{\dpred}#1}}
\newcommand{\Rtext}[1]{{\text{\color{\dpred}#1}}}
\newcommand{\Ftext}[1]{{\text{\color{\dpgreen}#1}}}
%:nomenc-exclude:
\newcommand{\Fdia}[1]{\color{\dpgreen}#1}
%:nomenc-exclude:

\newcommand{\Rcol}[1]{{\colR #1}}
%:nomenc-exclude:
%:example: \Rcol{X}
\newcommand{\Fcol}[1]{{\colF #1}}
%:nomenc-exclude:
%:example: \Fcol{X}
\newcommand{\Icol}[1]{{\colI #1}}
%:nomenc-exclude:
%:example: \Icol{X}
\newcommand{\gray}[1]{{\color{gray}#1}}
%:nomenc-exclude:
%:example: \gray{X}
\newcommand{\bchanges}{\color[rgb]{0,0.3,0}}
%:nomenc-exclude:
%:example: \bchanges{X}
\newcommand{\changes}[1]{{\color[rgb]{0,0.3,0}#1}}
%:nomenc-exclude:
%:example: \changes{X}
\newcommand{\echanges}{\color[rgb]{0,0,0}}
%:example: \echanges{X}
%:nomenc-exclude:

\newcommand{\styleobj}[1]{{\colTransmuted #1}}
\newcommand{\stylemorph}[1]{{\color{morphisms} #1}}
\newcommand{\stylefunctors}[1]{{\color{functors}#1}}
\newcommand{\stylemonads}[1]{{\color{functors}\mathcal{#1}}}
\newcommand{\stylenat}[1]{{\color{naturaltransformations}#1}}
% \newcommand{\arrowmorphism}{{-Triangle, draw=morphisms, line width=1pt}}
%:section:comments: Personal comments
\newcommand{\AC}[1]{{\color{blue}AC: #1}}
\newcommand{\GZ}[1]{{\color{green}GZ: #1}}
\newcommand{\DS}[1]{{\color{blue!50!red}DS says: #1}}
\newcommand{\JT}[1]{{\color{blue!30!green!30!black}JT: #1}}
%:example: \JT{blah}
\newcommand{\JL}[1]{{\color{magenta}JL: #1}}
%:example: \JL{blah}
%:section:markers: stuff missing
\newcommand{\XXX}{{\color{red}XXX}\xspace}
%:nomenc-exclude:
\newcommand{\citeXXX}{{\color{red}[cite]}\xspace}
%:nomenc-exclude:
\newcommand{\todographics}[1]{\todo[color=red!70]{#1}}
%:example: \todographics{blah}
%:no-inline:
%:nomenc-exclude:
\newcommand{\todotext}[1]{\todo[color=red!50]{#1}}
%:example: \todotext{blah}
%:no-inline:
%:nomenc-exclude:
\newcommand{\todostructure}[1]{\todo[color=red!10]{#1}}
%:example: \todostructure{blah}
%:no-inline:
%:nomenc-exclude:

%:section:status: Status markers
\newcommand{\readytoreview}[1]{#1 \texorpdfstring{\color{darkgreen} [ready] }{}}
%:example: \readytoreview{blah}
\newcommand{\statusdraft}[1]{#1 \texorpdfstring{\color{orange} [draft]}{}}
%:example: \statusdraft{blah}
\newcommand{\statusmissing}[1]{#1 \texorpdfstring{\color{darkred} [missing]}{}}
%:example: \statusdraft{blah}

%\renewcommand{\readytoreview}[1]{#1}
%\renewcommand{\statusdraft}[1]{#1}
%
% Not sure why it does not work
%\newcommand{\readytoreview}[1]{%
%  \ifthenelse{\boolean{statuscolors}}{%
%      {\color{darkgreen}#1}%
%  }{%
%    #1%
%  }%
%}
%\newcommand{\statusdraft}[1]{%
%  \ifthenelse{\boolean{statuscolors}}{%
%      {\color{orange} #1  [draft] }%
%  }{%
%    #1%
%  }%
%}

% \DeclareFontFamily{U}{mathx}{\hyphenchar\font45}
% \DeclareFontShape{U}{mathx}{m}{n}{
%       <5> <6> <7> <8> <9> <10>
%       <10.95> <12> <14.4> <17.28> <20.74> <24.88>
%       mathx10
%       }{}
% \DeclareSymbolFont{mathx}{U}{mathx}{m}{n}
% \DeclareFontSubstitution{U}{mathx}{m}{n}
% \DeclareMathAccent{\widecheck}{0}{mathx}{"71}


%:section:categories/basic: Basic

\newcommand{\after}{\mathrel{\circ}}
%:nomenc:b \after a:``$b$ after $a$''

\newcommand{\then}{\mathrel{\fatsemi}} % Composition (in general)
%:nomenc:a \then b:``$a$ then $b$''
\newcommand{\mthen}{\mathrel{\stylemorph{\fatsemi}}} % Composition for morphisms
%:nomenc:\mora \mthen \morb: Composition of morphisms
\newcommand{\fthen}{\mathrel{\stylefunctors{\fatsemi}}} % Composition for functors
%:nomenc:\funa \fthen \funb: Composition of functors
\newcommand{\nthen}{\mathrel{\stylenat{\fatsemi}}} % Compositions for natural transformations
%:nomenc:\ntrafoa \nthen \ntrafob: Composition of natural transformations


\newcommand{\Ob}{\styleobj{\operatorname{Ob}}}
%:nomenc:\Ob_\CatA:Objects of the category $\CatA$.
%:def:def:categorymain

\newcommand{\catid}{\operatorname{\stylemorph{Id}}} % identity for category
%:nomenc:\Unit\Obja:Identity morphism for the object $\Obja$
%:def:def:categorymain

\newcommand{\Unit}[1]{\catid_{#1}} % identity for an object



\newcommand{\Cat}[1]{\ensuremath{\mathbf{#1}}\xspace}
%:nomenc-exclude:

\newcommand{\Hom}{\stylemorph{\operatorname{Hom}}}
%:nomenc:\HomSet{\CatA}{\Obja}{\Objb}: Hom-set between $\Obja$ and $\Objb$.
%:def:def:categorymain
\newcommand{\HomSet}[3]{\Hom_{#1}\left({#2};{#3}\right)}

\newcommand{\comma}[2]{
\begin{tikzpicture}
\node at (0,0) {#1 \Huge{,}#2};
\end{tikzpicture}}

%:section:categories/generic: Generic names
\newcommand{\CatA}{\Cat{A}}
%:nomenc:\CatA,\CatB,\CatC, \CatD \dots: Symbols for categories
\newcommand{\CatB}{\Cat{B}}
%:nomenc-exclude:
\newcommand{\CatC}{\Cat{C}}
%:nomenc-exclude:
\newcommand{\CatD}{\Cat{D}}
%:nomenc-exclude:
\newcommand{\CatV}{\Cat{V}}
%:nomenc:\CatV: Symbol for enrichment categories.


\newcommand{\ObC}{\Ob_{\CatC}}
%:nomenc-exclude:
\newcommand{\ObD}{\Ob_{\CatD}}
%:nomenc-exclude:


\newcommand{\Obja}{\styleobj{X}}
%:nomenc:\Obja, \Objb, \Objc, \Objd: generic objects
\newcommand{\Objb}{\styleobj{Y}}
%:nomenc-exclude:
\newcommand{\Objc}{\styleobj{Z}}
%:nomenc-exclude:
\newcommand{\Objd}{\styleobj{W}}
%:nomenc-exclude:


\newcommand{\mora}{\stylemorph{f}}
%:nomenc:\mora,\morb,\morc,\mord: Generic morphisms
%:def:def:categorymain
\newcommand{\morb}{\stylemorph{g}}
%:nomenc-exclude:
\newcommand{\morc}{\stylemorph{h}}
%:nomenc-exclude:
\newcommand{\mord}{\stylemorph{i}}
%:nomenc-exclude:

\newcommand{\funa}{\stylefunctors{F}}
%:nomenc:\funa,\funb,\func,\fund: Generic functors
%:def:def:functor
\newcommand{\funb}{\stylefunctors{G}}
%:nomenc-exclude:
\newcommand{\func}{\stylefunctors{H}}
%:nomenc-exclude:
\newcommand{\fund}{\stylefunctors{I}}
%:nomenc-exclude:

\newcommand{\funid}{\stylefunctors{\operatorname{Id}}} % identity functor
%:nomenc:\funid_{\CatA}: Identity functor for category $\CatA$

\newcommand{\natid}{\stylenat{\operatorname{Id}}} % identity natural transformation
%:nomenc:\natid_{\funa}: Identity natural transformation for functor $\fun$

\newcommand{\ntrafoa}{\stylenat{\alpha}}
%:nomenc:\ntrafoa,\ntrafob,\ntrafoc,\ntrafod: Generic natural transformations
%:def:def:natural-transformation
\newcommand{\ntrafob}{\stylenat{\beta}}
%:nomenc-exclude:
\newcommand{\ntrafoc}{\stylenat{\gamma}}
%:nomenc-exclude:
\newcommand{\ntrafod}{\stylenat{\delta}}
%:nomenc-exclude:

%:section:transmutation: Transmutation

\newcommand{\transmuter}[1]{\textbf{\colTransmuter #1}\xspace}
%:nomenc-exclude:
\newcommand{\transmuted}[1]{\textbf{\colTransmuted #1}\xspace}
%:nomenc-exclude:
\newcommand{\technology}[1]{\mathsf{#1}}
%:nomenc-exclude:
\newcommand{\motor}{\transmuter{motor}}
%:nomenc-exclude:
\newcommand{\move}{\transmuter{move}}
%:nomenc-exclude:
\newcommand{\dynamo}{\transmuter{dynamo}}
%:nomenc-exclude:
\newcommand{\wheels}{\transmuter{wheels}}
%:nomenc-exclude:
\newcommand{\electricpower}{\transmuted{electricity}}
%:nomenc-exclude:
\newcommand{\rotationalmotion}{\transmuted{rotation}}
%:nomenc-exclude:
\newcommand{\translationalmotion}{\transmuted{translation}}
%:nomenc-exclude:


%:section:categories/monads: Monads
\newcommand{\monA}{\stylefunctors{M}}
%:nomenc:\monA,\monB: Generic monads.
%:def:def:monad
\newcommand{\monAA}{{\monA\fthen\monA}}
%:nomenc-exclude:
\newcommand{\monAAA}{{\monA\fthen\monA\fthen\monA}}
%:nomenc-exclude:
\newcommand{\monB}{\stylefunctors{N}}
%:nomenc-exclude:

\newcommand{\monunit}{\stylenat{\operatorname{un}}} % Monad unit
%:def:def:monad
\newcommand{\moncomp}{\stylenat{\operatorname{mu}}} % Monad identity
%:def:def:monad


\newcommand{\Lendo}{\stylefunctors{L}} % lower-set endofunctor
\newcommand{\Lmon}{\stylemonads{L}} % lower-set monad

\newcommand{\Uendo}{\stylefunctors{U}} % upper-set endofunctor
%:def:def:Uendo
\newcommand{\Umon}{\stylemonads{U}} % upper-set monad
%:def:def:Umon

%:section:codesign-spaces: Co-design spaces
\newcommand{\LF}{\lowersets\funsp}
%:nomenc-exclude:
\newcommand{\UR}{\Up\ressp}
%:nomenc-exclude:
\newcommand{\Aressp}{{\antichains\ressp}}
%:nomenc-exclude:
\newcommand{\Uressp}{\UR}
%:nomenc-exclude:


%:section:sets/well-known-functions: Well-known functions

\newcommand{\mapid}{\operatornamewithlimits{}{Id}} % Identity map
%:nomenc:\mapid_{\setA}:Identity map on $\setA$

\newcommand{\ceil}[1]{\left \lceil #1 \right \rceil}
%:nomenc:\ceil{x}:Rounding of $x$ to the next integer

\newcommand{\funceil}{\ensuremath{\operatorname{ceil}}}
\newcommand{\funfloor}{\ensuremath{\operatorname{floor}}}
%:def:ex:rounding-functions

\newcommand{\rtntte}{\textsf{rtntte}} % Round to nearest, ties to even
%:def:ex:rounding-functions




%:section:categories/companion: Companion/conjoints
\newcommand{\companion}[1]{\hat{#1}}
\newcommand{\comp}[1]{\widehat{#1}}
\newcommand{\conjoint}[1]{\check{#1}}
\newcommand{\conj}[1]{\widecheck{#1}}

\newcommand{\colim}{\operatorname{colim\;}}
\newcommand{\Coll}{\operatorname{Col}}

%:section:misc: Misc

\newcommand{\col}[1]{\mathrm{col(#1)}}

\newcommand{\coprodMap}[2]{{#1}\mathbf{+}{#2}}
\newcommand{\cP}{P}
\newcommand{\cQ}{Q}
\newcommand{\cR}{R}


%:section:categories/operations: Operations
\newcommand{\Ctimes}{\pmb{\times}} % Product in a category
%:def:def:categorical-product

\newcommand{\Cplus}{\pmb{+}} % Co-product in a category
%:def:def:catcoproduct

%\newcommand{\Ctimes}{ \tikz[baseline=-.55ex] \node [inner sep=0pt,cross out,draw,line width=1pt,minimum size=1ex] (a) {};}


%:section:categories/constructors: Constructors

\newcommand{\twisted}[1]{\mathrm{Tw}\left(#1\right)}
%:nomenc:\twisted{\CatA}:Twisted arrow construction on category $\CatA$.
%:def:def:twisted-arrow

\newcommand{\op}{^{\mathrm{op}}}


%\newcommand{\feasibleset}[1]{F_{#1}}
%\newcommand{\fix}{\text{fix}}

%\newcommand{\fupd}{f^\mathrm{upd}}
%\newcommand{\frdt}{f^\mathrm{rdt}}


%:section:categories/semigroups: Semigroups
\newcommand{\sgrpA}{\stylesets{S}} % Generic semigroup names
%:nomenc:\sgrpA, \sgrpB:Generic semigroup names.
\newcommand{\sgrpB}{\stylesets{T}} % Generic semigroup
%:nomenc-exclude:
\newcommand{\sgrpelA}{\styleelements{s}} % Generic semigroup element
%:nomenc:\sgrpelA:Generic semigroup elements.
\newcommand{\sgrpelAa}{\sgrpelA_1} % Generic semigroup element
%:nomenc-exclude:
\newcommand{\sgrpelAb}{\sgrpelA_2} % Generic semigroup element
%:nomenc-exclude:
\newcommand{\sgrpmorA}{\stylefunctors{f}} % Generic semigroup morphism
%:nomenc:\sgrpmorA,\sgrpmorB:Generic semigroup morphisms.
\newcommand{\sgrpmorB}{\stylefunctors{g}} % Generic semigroup morphism
%:nomenc-exclude:
%:section:categories/monoids: Monoids


\newcommand{\idmon}{{\color{blue}1}} % identity for monoid
%:def:def:monoid
\newcommand{\monoidA}{\stylesets{M}} % Generic monoid names
%:nomenc:\monoidA, \monoidB: Generic monoid names
%:def:def:monoid
\newcommand{\monoidB}{\stylesets{N}} % Generic monoid
%:nomenc-exclude:
%:def:def:monoid
\newcommand{\monelA}{\styleelements{m}} % Generic monoid element
%:nomenc:\monelA, \monelB: Generic monoid elements
%:def:def:monoid
\newcommand{\monelAa}{\monelA_1} % Generic monoid element
%:nomenc-exclude:
%:def:def:monoid
\newcommand{\monelAb}{\monelB_2} % Generic monoid element
%:nomenc-exclude:
%:def:def:monoid
\newcommand{\monelB}{\styleelements{n}} % Generic monoid element
%:nomenc-exclude:
%:def:def:monoid
\newcommand{\monelBa}{\monelB_1} % Generic monoid element
%:nomenc-exclude:
%:def:def:monoid
\newcommand{\monelBb}{\monelB_2} % Generic monoid element
%:nomenc-exclude:
%:def:def:monoid
\newcommand{\mtimes}{\mathrel{\pmb{\otimes}}} % Monoid operation
%:def:def:monoid


%:section:categories/monoidal: Monoidal categories

\newcommand{\mtimescat}{\mathrel{\stylefunctors{\pmb{\otimes}}}} % functor
%:nomenc:\mtimescat_{\CatA}:Monoidal operation for category $\CatA$.
%:def:def:monoidal_cat
\newcommand{\idmoncat}{\styleobj{\mathbf{1}}} % Identity object for monoidal operation
%:def:def:monoidal_cat
\newcommand{\mtimesD}{\mathrel{\mtimescat_{\CatD}}}
%:nomenc-exclude:
\newcommand{\mtimesC}{\mathrel{\mtimescat_{\CatC}}}
%:nomenc-exclude:


\newcommand{\ginv}{\operatorname{inv}} % group inverse

% \newcommand{\leftunitor}{\stylenat{\lambda}} % Left unitor
%\newcommand{\rightunitor}{\stylenat{\rho}} % Right unitor
% \newcommand{\associator}{\stylenat{\alpha}} % Associator

\newcommand{\leftunitor}{\stylenat{\operatorname{lu}}} % Left unitor
%:def:def:monoidal_cat
\newcommand{\rightunitor}{\stylenat{\operatorname{ru}}} % Right unitor
%:def:def:monoidal_cat
\newcommand{\associator}{\stylenat{\operatorname{as}}} % Associator
%:def:def:monoidal_cat

\newcommand{\braiding}{\stylenat{\operatorname{br}}} % Braiding
%:def:def:monoidal_cat


\newcommand{\strongmu}{\stylenat{\mu}} % Isomorphism for strong monoidal functor
%:def:def:strong-monoidal-functor
\newcommand{\strongeps}{\stylemorph{\operatorname{iso}}}  % Isomorphism for strong monoidal functor
%:def:def:strong-monoidal-functor

%:section:categories/adjunctions: Adjunctions

\newcommand{\funl}{\stylefunctors{L}} % left adjunct functor
%:def:def:cat-adjunction-v1
\newcommand{\funr}{\stylefunctors{R}} % right adjunct functor
%:def:def:cat-adjunction-v1
\newcommand{\adjunction}{\dashv} % Adjunction
%:nomenc:\funl \adjunction \funr: $\funl$ and $\funr$ are adjoint functors.
%:def:def:cat-adjunction-v1

\newcommand{\adjtau}{\stylenat{\tau}}

\newcommand{\equivunit}{\stylenat{\operatorname{un}}} % Unit
\newcommand{\equivcounit}{\stylenat{\operatorname{co}}} % Co-unit
%:def:def:cat-equivalence

%:section:categories/traced: Traced monoidal categories



\newcommand{\Tr}{\operatorname{Tr}} % Trace operator
%:def:def:traced-monoidal-category

\newcommand{\Conw}{\aword{Conw}} % Conway operator



\newcommand{\para}{\text{par}}


\newcommand{\prodMap}[2]{{#1}\times{#2}}
%:nomenc-exclude:

\newcommand{\qqand}{\qquad\text{and}\qquad}
%:nomenc-exclude:


% \newcommand{\snack}[1]{\mathsf{#1}}





\newcommand{\textF}[1]{\text{\F{#1}}}
\newcommand{\textR}[1]{\text{\R{#1}}}
% \newcommand{\thing}[1]{\text{#1}}





\newcommand{\ubar}[1]{\underaccent{\bar}{#1}}


\newcommand{\unc}{\mathsf{Unc}}

%:section:tuples: Tuples
\newcommand{\tup}[1]{\left\langle#1\right\rangle}
%:example: $\tup{\posA, \posAleq}$
\newcommand{\tupp}[1]{\langle#1\rangle}
%:example: $\tupp{\posA, \posAleq}$
\newcommand{\emptytuple}{\left\langle \right\rangle} % zero-size tuple

% The one below supposedly would allow splitting tuples over newlines,
% but I could not make it work - AC
%\makeatletter
%\newcommand\tup[1]{%
%  \@tempcnta=0
%  \left\langle
%  \@for\@ii:=#1\do{%
%    \@insertbreakingcomma
%    \@ii
%  }%
%  \right\rangle
%}
%\def\@insertbreakingcomma{%
%  \ifnum \@tempcnta = 0 \else\,,\ \linebreak[1] \fi
%  \advance\@tempcnta\@ne
%}
%\makeatother

%:section:booleans: Booleans

\newcommand{\true}{\top}
\newcommand{\false}{\bot}
\newcommand{\booland}{\pmb{\wedge}}
\newcommand{\boolor}{\pmb{\vee}}
%:section:arrows: Arrows

\newcommand{\sto}{\mathrel{\color{darkblue}\rightarrow}} % Set arrow
\newcommand{\mto}{\mathrel{\pmb{\stylemorph{\to}}}} % Morphism arrow
\newcommand{\fto}{\mathrel{\stylefunctors{\to}}} % Functors arrow
\newcommand{\ftolong}{\mathrel{\stylefunctors{\longrightarrow}}} % Functors arrow, longer
\newcommand{\nto}{\mathrel{\stylenat{\Rightarrow}}} % Natural transformation arrow
\newcommand{\ntolong}{\mathrel{\stylenat{\Longrightarrow}}} % Natural transformation arrow, longer
\newcommand{\nfromlong}{\mathrel{\stylenat{\Longleftarrow}}} % Inverted natural transformation arrow


\newcommand{\To}[1]{\xrightarrow{#1}}
\newcommand{\ntoiso}{\mathrel{\stylenat{\xrightarrow{\cong}}}}
\newcommand{\mtoiso}{\mathrel{\stylemorph{\xrightarrow{\cong}}}}
%:example: $a \To f b$





%\newcommand{\slashedrightarrow}{\relbar\joinrel\relbar\joinrel\mapstochar\joinrel\rightarrow}
\newcommand{\slashedrightarrow}{\relbar\joinrel\mapsto}
%:nomenc-exclude:
\newcommand{\profto}{\mathrel{\slashedrightarrow}} % profunctor arrow
%:def:def:profunctor

\newcommand{\toinPos}{\mto_{\Pos}}
\newcommand{\toiso}{\overset{\sim}{\to}}
\newcommand\too{\longrightarrow}


\newcommand{\Imp}{\Rightarrow} % Implies

\newcommand{\embedsin}{\hookrightarrow}
%:nomenc:\CatA \embedsin \CatB:$\CatA$ embeds in $\CatB$.


%:section:dp: DP
%:section:dp/formalization: Formalization
\newcommand{\fun}{\ensuremath{{\colF f}}\xspace} % A generic functionality in $\funsp$.
%:def:def:DPI
\newcommand{\res}{\ensuremath{{\colR r}}\xspace} % A generic cost in $\ressp$.
%:def:def:DPI
\newcommand{\imp}{\ensuremath{{\colI i}}\xspace} % A generic implementation in in $\impsp$.
%:def:def:DPI
\newcommand{\funsp}{\ensuremath{{\colF F}}\xspace} % Functionality space
%:def:def:DPI
\newcommand{\ressp}{\ensuremath{{\colR R}}\xspace} % Requirements space
%:def:def:DPI
\newcommand{\impsp}{\ensuremath{{\colI I}}\xspace} % Implementation space
%:def:def:DPI
\newcommand{\prov}{{\colF\aword{prov}}} % unctionality of an implementation
%:def:def:DPI
%:nomenc:\prov \colon \impsp\sto\funsp: functionality of an implementation
\newcommand{\req}{{\colR\aword{req}}} %
%:nomenc:\req  \colon \impsp\sto\ressp: requirements of an implementation
%:def:def:DPI
%:section:dp/top-bottom: Top and bottom
\newcommand{\restop}{\top_{\ressp}}
%:nomenc-exclude:
\newcommand{\resbot}{\bot_{\ressp}}
%:nomenc-exclude:
\newcommand{\funtop}{\top_{\funsp}}
%:nomenc-exclude:
\newcommand{\funbot}{\bot_{\funsp}}
%:nomenc-exclude:

\newcommand{\funleq}{\posleq_{\funsp}}
%:nomenc-exclude:
\newcommand{\resleq}{\posleq_{\ressp}}
%:nomenc-exclude:
\newcommand{\fungeq}{\posgeq_{\funsp}}
%:nomenc-exclude:
\newcommand{\resgeq}{\posgeq_{\ressp}}
%:nomenc-exclude:

%:section:posets/symbols: Symbols

\newcommand{\posleq}{\preceq}
%:nomenc:\posAleq:Order relation associated to the poset $\posA$
\newcommand{\postop}{\top} % Top of a poset
%:nomenc:\top_{\posA}:Top of poset $\posA$
%:def:def:top
\newcommand{\posbot}{\bot} % Bottom of a poset
%:nomenc:\bot_{\posA}:Bottom of poset $\posA$
%:def:def:bot

\newcommand{\poslt}{\prec}
%:nomenc-exclude:
\newcommand{\ordleq}{\preceq}
%:nomenc-exclude:
\newcommand{\ordgeq}{\succeq}
%:nomenc-exclude:
\newcommand{\posgeq}{\succeq}
%:nomenc-exclude:
%\newcommand{\leqP}{\posleq_{\cP}}
%\newcommand{\leqQ}{\posleq_{\cQ}}



\newcommand{\resMin}{{\Min_{\resleq}}}
%:nomenc-exclude:

%:section:abbrevs: Abbreviations

\renewcommand{\etal}{{et\,al.}\xspace}
%:nomenc-exclude:

\renewcommand{\eg}{\textbf{\color{red} e.g.}\xspace}% avoid!
%:nomenc-exclude:

\renewcommand{\etc}{{etc.}\xspace}% avoid!
%:nomenc-exclude:

\renewcommand{\ie}{\textbf{\color{red} i.e.}\xspace}% avoid!
%:nomenc-exclude:

\newcommand{\subto}{\text{s.t.}} % ``subject to'' (used in optimization problems)
%:nomenc-exclude:

\newcommand{\with}{\text{using}} % used in optimization problem
%:nomenc-exclude:

%:section:paper1: Original paper



\newcommand{\cdpiN}{\mathcal{V}}
\newcommand{\cdpin}{v}
\newcommand{\cdpinA}{v_1}
\newcommand{\cdpinB}{v_2}
\newcommand{\cdpiresind}{i}
\newcommand{\cdpifunind}{j}
\newcommand{\cdpiresindA}{i_1}
\newcommand{\cdpifunindB}{j_2}
\newcommand{\dpinumf}{\textrm{n}_f}
\newcommand{\dpinumr}{\textrm{n}_r}
\newcommand{\cdpinnumf}{{\dpinumf}_{\cdpin}}
\newcommand{\cdpinnumr}{{\dpinumr}_{\cdpin}}
\newcommand{\cdpiE}{\mathcal{E}}

\newcommand{\RR}{{\colR \alpha}} % A specific antichain, used in a proof.
%:nomenc-exclude:




\newcommand{\unconnectedfun}{\mathsf{UF}} % XXX: not a good choice
\newcommand{\unconnectedres}{\mathsf{UR}}

%:section:dp/computational:Computational representation
\newcommand{\rtof}{\ensuremath{{\colH \varphi}}\xspace}
%:def:def:rtof
\newcommand{\ftor}{\ensuremath{{\colH h}}\xspace}
%:def:def:ftor
\newcommand{\ftorL}{\ensuremath{{\colL h_L}}\xspace}
\newcommand{\ftorU}{\ensuremath{{\colU h_U}}\xspace}
\newcommand{\ftoR}{\ensuremath{{\colH H}}\xspace}



\newcommand{\scottcontinuous}{Scott continuous\xspace}
%:nomenc-exclude:
\newcommand{\scottcontinuity}{Scott continuity\xspace}
%:nomenc-exclude:

%:section:uncertainty: Uncertainty paper

\newcommand{\ufloor}{{\colL\aword{floor}}}
\newcommand{\uceil}{{\colU\aword{ceil}}}


\newcommand{\udpa}{\boldsymbol{u}_a}
\newcommand{\udpb}{\boldsymbol{u}_b}
\newcommand{\udpL}{{\colL \boldsymbol{\mathsf{L}}}}
\newcommand{\udpU}{{\colU \boldsymbol{\mathsf{U}}}}
\newcommand{\udpsp}{\UDP}
\newcommand{\udpleq}{\posleq_{\udpsp}}

\newcommand{\dpsp}{\DP}
\newcommand{\dpleq}{\posleq_\dpsp}
%:nomenc-exclude:

%:section:units: Currencies
\newcommand{\currency}[1]{\textbf{\colTransmuted #1}\xspace}
\newcommand{\USD}{\currency{USD}}
\newcommand{\usd}{USD\xspace}
\newcommand{\SGD}{\currency{SGD}}
\newcommand{\sgd}{SGD\xspace}
\newcommand{\CHF}{\currency{CHF}}
\newcommand{\chf}{CHF\xspace}
\newcommand{\EUR}{\currency{EUR}}
\newcommand{\eur}{EUR\xspace}

\def\bitcoinA{%
  \leavevmode
  \vtop{\offinterlineskip %\bfseries
  \setbox0=\hbox{B}%
  \setbox2=\hbox to\wd0{\hfil\hskip-.03em
    \vrule height .3ex width .15ex\hskip .08em
    \vrule height .3ex width .15ex\hfil}
    \vbox{\copy2\box0}\box2}}
\newcommand{\stdcurr}{\bitcoinA{}\xspace} % Generic currency


%:section:dp/symbols: DP


\newcommand{\adp}{\mathbf{d}} % generic design problem as a profunctor


\newcommand{\dprob}{\aword{dp}}
\newcommand{\dpseries}{\aword{series}}
\newcommand{\dppar}{\aword{par}}
\newcommand{\dploop}{\aword{loop}}

\newcommand{\dploopb}{\aword{loopb}}
\newcommand{\terms}{\aword{Terms}}
%
\newcommand{\udpsem}{\Phi}
%
\newcommand{\dpsem}{\varphi}
%
\newcommand{\atoms}{\mathcal{A}}
\newcommand{\atree}{\boldsymbol{\aword{T}}}
\newcommand{\val}{\boldsymbol{v}}
%
\newcommand{\ops}{\aword{ops}}
%
%
\newcommand{\acprod}{\mathbin{\boldsymbol{\times}}}
%
\newcommand{\oploop}{\dagger}
\newcommand{\opseries}{\mathbin{\varocircle}}
\newcommand{\oppar}{\mathbin{\varotimes}}
\newcommand{\opcoprod}{\mathbin{\varovee}}
%
\newcommand{\UId}{\aword{UId}}
\newcommand{\vdc}{\aword{vdc}}


%:section:posets/attributes: Attributes
\newcommand{\posetwidth}{\aword{width}}
%:nomenc:\posetwidth(\posA): Width of the poset $\posA$.
%:def:def:poset-width

\newcommand{\posetheight}{\aword{height}}
%:nomenc:\posetheight(\posA): Height of the poset $\posA$.
%:def:def:poset-height

%:section:posets/domain: Domain theory

\newcommand{\lfp}{\operatorname{lfp}} % Least fixed point

\newcommand{\CPO}{\textsf{CPO}\xspace} % Complete partial order
%:def:def:cpo

\newcommand{\DCPO}{\textsf{DCPO}\xspace} % Directed-complete partial order
%:def:def:cpo




%:section:dp/queries: Queries in $DP$

\newcommand{\Feasibility}{\aword{Feasibility}}
%:def:prob:Feasibility
\newcommand{\FeasibleImp}{\aword{FeasibleImp}}
%:def:prob:FeasibleImp
\newcommand{\FixFunMinReq}{\aword{FixFunMinReq}}
%:def:prob:FixFunMinReq
\newcommand{\FixResMinFun}{\aword{FixResMinFun}}
%:def:prob:FixResMinFun


%:section:categories/named: Named categories

\newcommand{\DP}{\Cat{DP}} % Category of design problems
%:def:def:DP
\newcommand{\UDP}{\Cat{UDP}}
\newcommand{\DPI}{\Cat{DPI}}
\newcommand{\Draw}{\Cat{Draw}} % Category of drawings
%:def:def:Draw
\newcommand{\Bool}{\Cat{Bool}} % Booleans (\XXX: see how?)



\newcommand{\Category}{\Cat{Cat}} % Category of small categories
%:def:def:Category
\newcommand{\Vect}{\Cat{Vect}} % Category of vector spaces
%:def:ex:Vect
\newcommand{\FinVect}{\Cat{FinVect}} % Category of finite-dimensional vector spaces
%:def:sub:trace-linear
\newcommand{\Rel}{\Cat{Rel}} % Category of sets and relations
%:def:def:Rel
\newcommand{\FinSet}{\Cat{FinSet}} % Category of finite sets and functions
%:def:ex:FinSet
\newcommand{\Set}{\Cat{Set}} % Category of sets and functions
%:def:def:Set
\newcommand{\Prof}{\mathbb{P}\Cat{rof}}
\newcommand{\Pos}{\Cat{Pos}} % Category of posets and monotone maps
%:def:def:Pos
\newcommand{\Injset}{\Cat{InjSet}} % Category of sets and injective functions
%:def:ex:Injset
\newcommand{\Graph}{\Cat{Grph}}
%:def:def:Graph
\newcommand{\graph}{\mathcal{G}}
\newcommand{\source}{\stylemaps{s}}
\newcommand{\target}{\stylemaps{t}}
\newcommand{\vertices}{\stylesets{V}}
\newcommand{\vertexa}{\styleelements{v}}
\newcommand{\vertexb}{\styleelements{w}}
\newcommand{\arcs}{\stylesets{A}}
\newcommand{\arc}{\styleelements{a}}
\newcommand{\arca}{\styleelements{a}}
\newcommand{\arcb}{\styleelements{b}}



\newcommand{\LTI}{\Cat{LTI}}
\newcommand{\Free}{\Cat{Free}} % free construction
\newcommand{\feas}{\Cat{Feas}}


%:section:chapters: Symbols used in particular chapters

%:section:chapters/composition: \cref{ch:composition}

%:section:chapters/sameness: \cref{ch:sameness}

\newcommand{\Si}{\text{S\`i}}
%:nomenc-exclude:

\newcommand{\strain}{\varepsilon}
\newcommand{\force}{F}
\newcommand{\deformation}{\Delta x}
\newcommand{\springconst}{k}
\newcommand{\youngmod}{E}

%:section:chapters/transmutation: \cref{ch:transmutation}

\newcommand{\Curr}{\Cat{Curr}} % Currency category
%:def:def:Curr

%:section:chapters/connection: \cref{ch:connection}


\newcommand{\Berg}{\Cat{Berg}}
%:nomenc:\Berg:The category of Swiss mountains
%:def:def:Berg
\newcommand{\Bergama}{\Cat{BergAma}}
%:def:sec:subcat_berg
\newcommand{\Berglazy}{\Cat{BergLazy}}
%:def:sec:subcat_berg

\newcommand{\Intermodal}{\Cat{Intermodal}}
\newcommand{\Car}{\Cat{Car}}
\newcommand{\Flight}{\Cat{Flight}}
\newcommand{\Board}{\Cat{Board}}

\newcommand{\LIN}{\transmuted{LIN}}
%:nomenc-exclude:
\newcommand{\FCO}{\transmuted{FCO}}
\newcommand{\FCOf}{\transmuted{FCO}_\transmuted{f }}
%:nomenc-exclude:
\newcommand{\MPXf}{\transmuted{MPX}_\transmuted{f }}
%:nomenc-exclude:
\newcommand{\ZRH}{\transmuted{ZRH}}
\newcommand{\ZRHf}{\transmuted{ZRH}_\transmuted{f }}
%:nomenc-exclude:

\newcommand{\FCOc}{\transmuted{FCO}_\transmuted{c }}
%:nomenc-exclude:
\newcommand{\MPXc}{\transmuted{MPX}_\transmuted{c }}
%:nomenc-exclude:
\newcommand{\ZRHc}{\transmuted{ZRH}_\transmuted{c }}
%:nomenc-exclude:

\newcommand{\FCOoff}{\transmuter{FCO} \ \transmuter{offboard }}
%:nomenc-exclude:
\newcommand{\MPXoff}{\transmuter{MPX} \ \transmuter{offboard }}
%:nomenc-exclude:
\newcommand{\ZRHoff}{\transmuter{ZRH} \ \transmuter{offboard }}
%:nomenc-exclude:
\newcommand{\FCOon}{\transmuter{FCO} \ \transmuter{onboard }}
%:nomenc-exclude:
\newcommand{\MPXon}{\transmuter{MPX} \ \transmuter{onboard }}
%:nomenc-exclude:
\newcommand{\ZRHon}{\transmuter{ZRH} \ \transmuter{onboard }}
%:nomenc-exclude:

\newcommand{\mplane}{\mbox{\smaller[2]\raisebox{-1pt}{\Plane}}}
\newcommand{\mplanerot}{\rotatebox{180}{\mbox{\smaller[2]\raisebox{-8pt}{\Plane}}}}
%:nomenc-exclude:
\definecolor{alitalia}{named}{darkgreen}
\definecolor{swiss}{named}{darkblue}
\definecolor{ewings}{named}{darkred}
\newcommand{\alitaliaN}{{\color{alitalia}\mplanerot}\ \transmuter{Alitalia}\ \transmuter{N }}
\newcommand{\alitaliaNrot}{{\color{alitalia}\mplane}\ \transmuter{Alitalia}\ \transmuter{N }}
%:nomenc-exclude:
\newcommand{\alitaliaS}{{\color{alitalia}\mplane}\ \transmuter{Alitalia}\ \transmuter{S }}
\newcommand{\alitaliaSrot}{{\color{alitalia}\mplanerot}\ \transmuter{Alitalia}\ \transmuter{S }}
%:nomenc-exclude:
\newcommand{\swissN}{{\color{swiss}\mplane}\ \transmuter{Swiss}\ \transmuter{N }}
%:nomenc-exclude:
\newcommand{\swissS}{{\color{swiss}\mplanerot}\ \transmuter{Swiss}\ \transmuter{S }}
%:nomenc-exclude:
\newcommand{\ewingsN}{{\color{ewings}\mplanerot}\ \transmuter{Ewings}\ \transmuter{N }}
\newcommand{\ewingsNrot}{{\color{ewings}\mplane}\ \transmuter{Ewings}\ \transmuter{N }}
%:nomenc-exclude:
\newcommand{\ewingsS}{{\color{ewings}\mplanerot}\ \transmuter{Ewings}\ \transmuter{S }}
%:nomenc-exclude:
\newcommand{\thefourohfive}{\includegraphics[width=4mm]{int80}} % change number to 405
%:nomenc-exclude:


%:section:chapters/mapping: \cref{ch:mapping}
\newcommand{\Company}{\transmuter{Company}}
%:nomenc-exclude:
\newcommand{\SanyoDenki}{\text{Sanyo Denki}}
%:nomenc-exclude:
\newcommand{\Soyo}{\text{Soyo}}
%:nomenc-exclude:
\newcommand{\Price}{\transmuter{Price}}
%:nomenc-exclude:
\newcommand{\Volume}{\transmuter{Volume}}
%:nomenc-exclude:
\newcommand{\Size}{\transmuter{Size}}
%:nomenc-exclude:
\newcommand{\Multiply}{\transmuter{Multiply}}
%:nomenc-exclude:
\newcommand{\Database}{\Cat{Database}}


%:section:chapters/combinations: \cref{ch:combination}

\newcommand{\sbanana}{\raisebox{-2pt}{\includegraphics[height=10pt]{banana}}}
%:nomenc-exclude:
\newcommand{\sapple}{\raisebox{-2pt}{\includegraphics[height=10pt]{red-apple}}}
%:nomenc-exclude:
\newcommand{\scarrot}{\raisebox{-2pt}{\includegraphics[height=10pt]{carrot}}}
%:nomenc-exclude:
\newcommand{\Snacks}{\stylesets{\operatorname{Snacks}}}
%:nomenc-exclude:
\newcommand{\Drinks}{\stylesets{\operatorname{Drinks}}}
%:nomenc-exclude:
\newcommand{\Participants}{\stylesets{\operatorname{Participants}}}
%:nomenc-exclude:
\newcommand{\swater}{\raisebox{-2pt}{\includegraphics[height=10pt]{water}}} % need to find better icon
%:nomenc-exclude:
\newcommand{\stea}{\raisebox{-2pt}{\includegraphics[height=10pt]{beer}}}
%:nomenc-exclude:
\newcommand{\eats}{\stylemaps{\operatorname{eats}}}
%:nomenc-exclude:
\newcommand{\drinks}{\stylemaps{\operatorname{drinks}}}
%:nomenc-exclude:
\newcommand{\meal}{\stylemaps{\operatorname{meal}}}
%:nomenc-exclude:


%:section:chapters/translation: \cref{ch:translation}

\newcommand{\Plans}{\Cat{Plans}}
%:def:ex:planning-as-search-functor



%:section:miscellanea: To categorize

\newcommand{\constit}{\underline{Constituents}}
\newcommand{\condit}{\underline{Conditions}}

\newcommand{\MS}[1]{{\color{blue}#1}}


\newcommand{\One}{\mathbf{1}} % \XXX not sure if category or set




\newcommand{\iindex}[1]{#1\index{#1}}
%:nomenc-exclude:
\newcommand{\aword}[1]{\ensuremath{\mathsf{#1}}\xspace}
%:nomenc-exclude:


%\newcommand{\Lo}[1]{\mathsf{L}#1}


%\newcommand{\myvee}{\textstyle{\bigvee}}
%\newcommand{\mycup}{\textstyle{\bigcup}}
%\newcommand{\M}{\mathsf{M}}
%\newcommand{\T}{\mathsf{T}}


\newcommand{\definedas}{\doteq} % ``defined as''



% \newcommand{\low}{\aword{Low}} % to remove?
% \newcommand{\upp}{\mathsf{Upp}} % to remove?

\newcommand{\Lax}{\operatorname{Lax}}


\DeclareMathOperator{\MOb}{\lvert\mspace{2mu}\cdot\mspace{2mu}\rvert}

\newcommand{\triv}{\mathsf{Triv}}


\newcommand{\dvert}{\operatorname{Vert}}



\newcommand{\marker}{{\bullet}}
%:nomenc-exclude:

\newcommand{\exname}[1]{\texttt{\footnotesize #1}} % Name of exercise
%:nomenc-exclude:

\newcommand{\dd}{\textrm{d}}
%:section:deprecated:Deprecated

\newcommand{\id}{\operatorname{id}}


%:section:chapters/epluribus:Epluribus


\newcommand{\alphabeta}{{\color{blue}\square}}
\newcommand{\alphabetb}{{\color{red}\square}}
\newcommand{\sprout}{\text{sprout}}
\newcommand{\yng}{\text{young}}
\newcommand{\dead}{\text{dead}}
\newcommand{\mature}{\text{mature}}
\newcommand{\old}{\text{old}}

%:section:chapters/epluribus/morse:Morse code
\newcommand{\morsedot}{\ensuremath{\bullet}} % Morse dot
\newcommand{\morsedash}{\ensuremath{\pmb{-}}} % Morse dash
\newcommand{\morsedsp}{\ensuremath{\,s_1\,}} % Silence between dots and dashes
\newcommand{\morselsp}{\ensuremath{\,s_3\,}} % Silence between letters
\newcommand{\morsewsp}{\ensuremath{\,s_7\,}} % Silence between words


\newcommand{\Morsedot}{\ensuremath{{\color{black}\rule{\mb}{\ml}}}} % Beep of $\ell$
\newcommand{\Morsedash}{\ensuremath{{\color{black}\rule{3\mb}{\ml}}}}% Beep of $3\ell$
\newcommand{\Morsedsp}{\ensuremath{{\color{gray}\rule{\mb}{\ml}}}}% Silence of $\ell$
\newcommand{\Morselsp}{\ensuremath{{\color{gray}\rule{3\mb}{\ml}}}}% Silence of $3\ell$
\newcommand{\Morsewsp}{\ensuremath{{\color{gray}\rule{7\mb}{\ml}}}}% Silence of $7\ell$

%:section:chapters/epluribus/morse/internal: internal
%:nomenc-exclude:
\newcommand{\mst}[1]{\textbf{#1}} % Formatting for morse code table
\newcommand{\lettersp}{\,{\color{red}\rule{3pt}{8pt}}\,}
\newcommand{\wordsp}{\,{\color{green}\rule{3pt}{8pt}}\,}

\newcommand{\morseA}{\morsedot\morsedsp  \morsedash}
\newcommand{\morseE}{\morsedot}
\newcommand{\morseI}{\morsedot\morsedsp  \morsedot}
\newcommand{\morseL}{\morsedash}
\newcommand{\morseM}{\morsedash\morsedsp \morsedash}
\newcommand{\morseW}{\morsedot\morsedsp \morsedash\morsedsp \morsedash}
\newlength{\ml}
\setlength{\ml}{7pt}
\newlength{\mb}
\setlength{\mb}{4pt}

\newcommand{\MorseA}{\Morsedot\Morsedsp  \Morsedash}
\newcommand{\MorseE}{\Morsedot}
\newcommand{\MorseI}{\Morsedot\Morsedsp  \Morsedot}
\newcommand{\MorseL}{\Morsedash}
\newcommand{\MorseM}{\Morsedash\Morsedsp \Morsedash}
\newcommand{\MorseW}{\Morsedot\Morsedsp \Morsedash\Morsedsp \Morsedash}


%:section:chapters/epluribus/ASCII:ASCII example
\newcommand{\alphanums}{\texttt{A}}
\newcommand{\asciienc}{\operatorname{ASCII}}

%:section:text: Frequently mispelled words
\newcommand{\whomo}{\text{homomorphism}\xspace}
\newcommand{\whomos}{\text{homomorphisms}\xspace}
\newcommand{\wHomo}{\text{Homomorphism}\xspace}
\newcommand{\wHomos}{\text{Homomorphisms}\xspace}





%:section:python:Python code
%:nomenc-exclude:
\newcommand{\funcnamestyle}[1]{{  \normalsize \ttfamily \color{funcname} #1}} % Style for function names
%:example: \funcnamestyle{function}
\newcommand{\classnamestyle}[1]{{\bfseries \normalsize \ttfamily \color{classname} #1}} % Style for class names
%:example: \classnamestyle{Class}

%\newcommand{\funcname}[1]{{\funcnamestyle{#1()}}\xspace}
%:example: \funcname{function}

\newcommand{\classname}[1]{\classnamestyle{#1}\xspace}
%:example: \classname{Class}
\newcommand{\Setoid}{\text{\classname{Setoid}}}
\newcommand{\EnumerableSet}{\text{\classname{EnumerableSet}}}
\newcommand{\FiniteSet}{\text{\classname{FiniteSet}}}


%\newcommand{\classname}[1]{{\bfseries\color{classname}\mintinline{python}{#1}}\xspace}
\newcommand{\funcname}[1]{{\color{funcname}\mintinline{python}{#1()}}\xspace}
