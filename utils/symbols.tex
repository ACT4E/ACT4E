% !TEX root = ../ACT4E-devel-fast.tex

%:section:gen: Generic symbols
%:section:gen/sets: Sets
%:section:gen/relations: Relations
%:section:gen/posets: Posets
%:section:gen/semigroups: Semigroups

%:section:vol1: Specific use in Volume 1

%:section:vol1/05_pluribus:\partrefplus{part:pluribus}

%:section:vol1/07_orders:\partrefplus{part:order}

%:section:vol1/10_arrows:\partrefplus{part:first}

%:section:vol1/22_operations:\partrefplus{part:combination}

%:section:vol1/25_translation:\partrefplus{part:thicker-arrows}

%:section:vol1/30_design:\partrefplus{part:co-design}

%:section:vol1/40_computation:\iflabelexists{part:computation}{\partrefplus{part:computation}}

%:section:vol1/45_enrichap:\partrefplus{part:enrichment}

%:section:vol1/50_opera:\partrefplus{part:opera}

%%%%%%%%%%%%%%%%%%%%%%%%% Sets

%:section:gen/sets/styles: Sets
\newcommand{\stylesets}[1]{{\ColorIfNowBlack{formulasetcolor}{\mathbf{#1}}}}
%:nomenc-exclude:
\newcommand{\styleLsets}[1]{{\color{formulasetLcolor}\mathbf{#1}}}
%:nomenc-exclude:
\newcommand{\styleelements}[1]{{\color{elementscolor}{#1}}}
%:nomenc-exclude:
\newcommand{\stylemaps}[1]{{\stylemorph{#1}}}
%:nomenc-exclude:
\newcommand{\EM}[1]{\ensuremath{#1}\xspace}
%:nomenc-exclude:

\newcommand{\makesymbolset}[1]{\EM{\stylesets{#1}}}
\newcommand{\makesymbolLset}[1]{\EM{\styleLsets{#1}}}

\newcommand{\setAneut}{\mathbf{A}}
%:nomenc-exclude:
\newcommand{\setBneut}{\mathbf{B}}
%:nomenc-exclude:
\newcommand{\setCneut}{\mathbf{C}}
%:nomenc-exclude:
%:section:gen/sets/generic: Generic sets and elements
\newcommand{\setA}{\makesymbolset{A}}
%:nomenc:\setA,\setB,\setC,\setD,\dots:Generic names for sets.
\newcommand{\setAn}[1]{\setA_{\color{formulasetcolor}{#1}}}
%:nomenc-exclude:
\newcommand{\setLA}{\makesymbolset{A}}
%:nomenc-exclude:
\newcommand{\setAprime}{\makesymbolset{A'}}
%:nomenc-exclude:
\newcommand{\setBprime}{\makesymbolset{B'}}
%:nomenc-exclude:
\newcommand{\setB}{\makesymbolset{B}}
%:nomenc-exclude:
\newcommand{\setBn}[1]{\setB_{\color{formulasetcolor}{#1}}}
%:nomenc-exclude:
\newcommand{\setLB}{\styleLsets{B}}
%:nomenc-exclude:
\newcommand{\setC}{\makesymbolset{C}}
%:nomenc-exclude:
\newcommand{\setCn}[1]{\makesymbolset{C}_{\color{formulasetcolor}{#1}}}
%:nomenc-exclude:
\newcommand{\setLC}{\styleLsets{C}}
%:nomenc-exclude:
\newcommand{\setCprime}{\makesymbolset{C'}}
%:nomenc-exclude:
\newcommand{\setD}{\makesymbolset{D}}
%:nomenc-exclude:
\newcommand{\setDn}[1]{\makesymbolset{D}_{\color{formulasetcolor}{#1}}}
%:nomenc-exclude:
\newcommand{\setLD}{\styleLsets{D}}
%:nomenc-exclude:
\newcommand{\setE}{\makesymbolset{E}}
%:nomenc-exclude:
\newcommand{\setEn}[1]{\makesymbolset{E}_{\color{formulasetcolor}{#1}}}
%:nomenc-exclude:
\newcommand{\setLE}{\styleLsets{E}}
%:nomenc-exclude:
\newcommand{\setF}{\makesymbolset{F}}
%:nomenc-exclude:
\newcommand{\setFn}[1]{\setF_{\color{formulasetcolor}{#1}}}
%:nomenc-exclude:
\newcommand{\setLF}{\styleLsets{F}}
%:nomenc-exclude:
\newcommand{\setG}{\makesymbolset{G}}
%:nomenc-exclude:
\newcommand{\setH}{\makesymbolset{H}}
%:nomenc-exclude:
\newcommand{\setS}{\makesymbolset{S}}
%:nomenc-exclude:
\newcommand{\setI}{\makesymbolset{I}}
%:nomenc-exclude:
\newcommand{\setIel}{\styleelements{i}}
%:nomenc-exclude:
\newcommand{\setJ}{\makesymbolset{J}}
%:nomenc-exclude:
\newcommand{\setJel}{\styleelements{j}}
%:nomenc-exclude:
\newcommand{\setK}{\makesymbolset{K}}
%:nomenc-exclude:
\newcommand{\setKel}{\styleelements{k}}
%:nomenc-exclude:
\newcommand{\setAel}{\styleelements{a}}
%:nomenc:\setAel,\setBel,\setCel:Generic names for elements of sets.
%:nomenc-exclude:

\newcommand{\setAeln}[1]{\styleelements{a_{#1}}}
%:nomenc-exclude:

\newcommand{\setBel}{\styleelements{b}}
%:nomenc-exclude:
\newcommand{\setBeln}[1]{\styleelements{b_{#1}}}
%:nomenc-exclude:
\newcommand{\setCel}{\styleelements{c}}
%:nomenc-exclude:
\newcommand{\setCeln}[1]{\styleelements{c_{#1}}}
%:nomenc-exclude:
\newcommand{\setDel}{\styleelements{d}}
%:nomenc-exclude:
\newcommand{\setDeln}[1]{\styleelements{d_{#1}}}
%:nomenc-exclude:
\newcommand{\setEel}{\styleelements{e}}
%:nomenc-exclude:
\newcommand{\setEeln}[1]{\styleelements{e_{#1}}}
%:nomenc-exclude:
\newcommand{\setFel}{\styleelements{f}}
%:nomenc-exclude:
\newcommand{\setFeln}[1]{\styleelements{f_{#1}}}
%:nomenc-exclude:
\newcommand{\setGel}{\styleelements{g}}
%:nomenc-exclude:
\newcommand{\setGeln}[1]{\styleelements{g_{#1}}}
%:nomenc-exclude:
\newcommand{\setHel}{\styleelements{h}}
%:nomenc-exclude:
\newcommand{\setHeln}[1]{\styleelements{h_{#1}}}
%:nomenc-exclude:

\newcommand{\subA}{\makesymbolset{S}}
%:nomenc:\subA,\subB:Generic names for subsets.
\newcommand{\subAn}[1]{\subA_{\color{formulasetcolor}{#1}}}
%:nomenc-exclude:
\newcommand{\subB}{\makesymbolset{T}}
%:nomenc-exclude:
\newcommand{\subBn}[1]{\subB_{\color{formulasetcolor}{#1}}}
%:nomenc-exclude:
\newcommand{\subC}{\makesymbolset{U}}
%:nomenc-exclude:
\newcommand{\subCn}[1]{\makesymbolset{U}_{\color{formulasetcolor}{#1}}}
%:nomenc-exclude:
\newcommand{\compl}[1]{#1^\mathrm{c}}

\newcommand{\subD}{\makesymbolset{V}}

\newcommand{\setnor}[1]{\makesymbolset{S}_{#1}} % XXX: what is this?
%:nomenc-exclude:
\newcommand{\setstrnor}[1]{\makesymbolset{T}_{#1}}  % XXX: what is this?
%:nomenc-exclude:
\newcommand{\norm}[1]{\vert #1 \vert} % norm of vector
%:nomenc:\norm x: norm of vector $x$
\newcommand{\normset}[1]{[#1]} % ???
%:nomenc-exclude:
\newcommand{\ela}{\styleelements{x}}
%:nomenc:\ela,\elb,\elc,\eld, \ele:Generic names for elements of sets.
\newcommand{\elna}[1]{\styleelements{x_{#1}}}
%:nomenc-exclude:
\newcommand{\elb}{\styleelements{y}}
%:nomenc-exclude:
\newcommand{\elnb}[1]{\styleelements{y_{#1}}}
%:nomenc-exclude:
\newcommand{\elc}{\styleelements{z}}
%:nomenc-exclude:
\newcommand{\elnc}[1]{\styleelements{z_{#1}}}
%:nomenc-exclude:
\newcommand{\elan}{x}
%:nomenc:\elan,\elbn,\elcn,\eld:Generic names for elements of generic sets (no color).
\newcommand{\elbn}{y}
%:nomenc-exclude:
\newcommand{\elcn}{z}
%:nomenc-exclude:
\newcommand{\eld}{\styleelements{u}}
%:nomenc-exclude:
\newcommand{\ele}{\styleelements{v}}
%:nomenc-exclude:
\newcommand{\mapa}{\stylemaps{f}}
%:nomenc:\mapa,\mapb,\mapc,\mapd,\mape:Generic names for maps between sets.
\newcommand{\mapaindex}[1]{\stylemaps{f}_{#1}}
%:nomenc-exclude:
\newcommand{\mapb}{\stylemaps{g}}
%:nomenc-exclude:
\newcommand{\mapbindex}[1]{\stylemaps{g}_{#1}}
%:nomenc-exclude:
\newcommand{\mapc}{\stylemaps{h}}
%:nomenc-exclude:
\newcommand{\mapd}{\stylemaps{k}}
%:nomenc-exclude:
\newcommand{\mape}{\stylemaps{m}}
%:nomenc-exclude:
\newcommand{\mapf}{\stylemaps{p}}
%:nomenc-exclude:
\newcommand{\mapg}{\stylemaps{q}}
%:nomenc-exclude:
\newcommand{\maph}{\stylemaps{r}}
%:nomenc-exclude:
\newcommand{\cod}{\operatorname{cod}}
%:nomenc: \cod f: Codomain of function $f$
%:def:sec:functions
\newcommand{\dom}{\operatorname{dom}}
%:nomenc: \dom f: Domain of function $f$
%:def:sec:functions
\newcommand{\diffun}{C^1}

\newcommand{\setofsetsA}{\mathcal{S}}

\newcommand{\setintersection}{\mathbin{{\color{formulasetcolor}\cap}}} % Set intersection
\newcommand{\setunion}{\mathbin{{\color{formulasetcolor}\cup}}} % Set union
\newcommand{\bigsetunion}{\mathbin{{\color{formulasetcolor}\bigcup}}} % Big Set union
%:nomenc-exclude:
\newcommand{\bigsetintersection}{\mathbin{{\color{formulasetcolor}\bigcap}}} % Big Set intersection
%:nomenc-exclude:
\newcommand{\setin}{\linebreak[0]\mathrel{{\color{formulasetcolor}\in}}\linebreak[0]} % Belongs to
%:nomenc:\ela\setin\setA:The element $\ela$ belongs to the set $\setA$.
\newcommand{\notsetin}{\mathrel{{\color{formulasetcolor}\not\in}}} % Does not belong to
%:nomenc:\ela\notsetin\setA:The element $\ela$ does not belong to the set $\setA$.

\newcommand{\setsubseteq}{\mathrel{\color{formulasetcolor}{\subseteq}}} % subset
%:nomenc:\setA\setsubseteq\setB:The set $\setA$ is a subset of $\setB$.

\newcommand{\setsupseteq}{\mathrel{\color{formulasetcolor}{\supseteq}}} % superset
%:nomenc:\setA\setsupseteq\setB:The set $\setA$ is a superset of $\setB$.


%:section:gen/sets/known: Well-known sets.

\newcommand{\Emptyset}{\stylesets{\emptyset}} % Empty set
\newcommand{\cnumbers}{\EM{\stylesets{\mathbb{C}}}} % Complex numbers
\newcommand{\reals}{\EM{\stylesets{\mathbb{R}}}} % Real numbers
\newcommand{\natnumbers}{\EM{\stylesets{\mathbb{N}}}} % Natural numbers: $0, 1, 2, \dots$
\newcommand{\wnumbers}{\EM{\stylesets{\mathbb{Z}}}} % Integers: $0, 1, -1, 2, -2, \dots$
\newcommand{\ratnumbers}{\EM{\stylesets{\mathbb{Q}}}} % Rational numbers
\newcommand{\posReals}{\EM{\reals_{>0}}} % Positive real numbers
\newcommand{\nonNegReals}{\EM{\reals_{\geq0}}} % Non-negative real numbers
\newcommand{\nonNegRealsComp}{\overline{\reals}_{\geq0}} % Completion of non-negative real numbers.

\newcommand{\Rcomp}{\nonNegRealsComp} % Not sure - need to check
%:nomenc-exclude:
\newcommand{\singleton}{\makeset{\!\singletonel\!}} % Singleton set, containing the element $\singletonel$.
\newcommand{\singletonel}{\bullet} % element of the singleton
%:nomenc-exclude:
\newcommand{\singletonobj}{\makeset{\styleobj{\singletonel}}} % ???
%:nomenc-exclude:

\newcommand{\Runit}[1]{\reals^{\!\!\textrm{[#1]}}} % R with units
%:nomenc:\Runit m: A copy of $\reals$ with units of meters

%:section:gen/sets/constructors: Constructors
\newcommand{\powerset}{\powersetmap\,}
\newcommand{\powersetmap}{\stylesets{\aword{Pow}}}
\newcommand{\powersetfunctor}{\stylefunctors{\aword{Pow}}} % powerset seen as a functor
%:nomenc:\powerset \setA: Power set of $\setA$.
%:def:def:power-set

%:section:gen/sets/operations: Operations

%%%% \newcommand{\cartprod}{\SYNinner{cartesian product}{\mathbin{\stylesets{\times}}}} % cartesian product
\newcommand{\cartprod}{\mathbin{\stylesets{\times}}} % cartesian product
%:nomenc:\setA\cartprod\setB:Cartesian product of two sets.
%:def:def:cartesian-product

\newcommand{\funcprod}{\mathbin{\textcolor{morphisms}{\times}}} % maps product
%:nomenc:\mora\funcprod\morb:Product of two functions

\newcommand{\funcsum}{\mathbin{\textcolor{morphisms}{\mathbf{+}}}} % maps direct sum
%:nomenc:\mora\funcsum\morb:Direct sum of two functions


% \newcommand{\setdisunion}{\SYNinner{disjoint union of sets}{\mathbin{\stylesets{+}}}} % Disjoint union
\newcommand{\setdisunion}{\mathbin{\stylesets{+}}} % Disjoint union
%:nomenc:\setA\setdisunion\setB:Disjoint union of two sets.
%:def:def:disjoint-union

\newcommand{\disunionA}[1]{{\tup{1, #1}}}
%:nomenc:\disunionA{a}, \disunionA{b}:Decorated elements of disjoint union
%:def:def:disjoint-union
\newcommand{\disunionB}[1]{{\tup{2, #1}}}
%:nomenc-exclude:
\newcommand{\disunionI}[1]{{\tup{i, #1}}}
\newcommand{\disunionJ}[1]{{\tup{j, #1}}}
\newcommand{\disunionK}[1]{{\tup{k, #1}}}
\newcommand{\inj}{\stylemorph{\aword{in}}}


\newcommand{\injA}{\stylemorph{\aword{in}}_\stylemorph{1}}
%:nomenc:\injA,\injB:Injections into $\setA \setdisunion \setB$.
%:def:def:disjoint-union

\newcommand{\injB}{\stylemorph{\aword{in}}_\stylemorph{2}}
%:nomenc-exclude:

\newcommand{\projA}{\stylemaps{\aword{pr}}_\stylemorph{1}}
\newcommand{\projB}{\stylemaps{\aword{pr}}_\stylemorph{2}}
\newcommand{\proj}{\stylemorph{\aword{pr}}}
\newcommand{\projstart}{\stylemorph{\aword{pr}_\aword{s}}}
\newcommand{\projend}{\stylemorph{\aword{pr}_\aword{e}}}


%%%%%%%%%%%%%%%%%%%%%%%%% Linear Algebra
%:section:gen/linear algebra: Linear Algebra


\newcommand{\field}{\textbf{k}}
\newcommand{\idmat}{\mathbb{1}} % Identity matrix

\newcommand{\trace}{\aword{Tr}\,}
%:nomenc:\trace A: trace of a linear map
\newcommand{\deter}{\aword{det}\,}
%:nomenc:\deter A: determinant of a linear map
\newcommand{\mdet}{\stylefunctors{\aword{det}}\,} % Matrix determinant
%:nomenc:\mdet A: matrix determinant (as functor)

\newcommand{\mattransp}{^{\intercal}}
%:nomenc:\mat{A}\mattransp:Transpose of a matrix $\mat{A}$.

\newcommand{\posdef}{\mathcal{P}^+} % Positive-definite matrices
% \newcommand{\possemidef}{\mathcal{P}} % Positive-semidefinite matrices

\newcommand{\vecspA}{U}
%:nomenc:\vecspA, \vecspB, \vecspC: generic vector spaces
\newcommand{\vecspB}{V}
%:nomenc-exclude:
\newcommand{\vecspC}{W}
%:nomenc-exclude:
\newcommand{\vecspAa}{U_1}
%:nomenc-exclude:
\newcommand{\vecspAb}{U_2}
%:nomenc-exclude:
\newcommand{\vecspAc}{U_3}
%:nomenc-exclude:
\newcommand{\vecspBa}{V_1}
%:nomenc-exclude:
\newcommand{\vecspBb}{V_2}
%:nomenc-exclude:
\newcommand{\vecspBc}{V_3}
%:nomenc-exclude:
\newcommand{\vecspCa}{W_1}
%:nomenc-exclude:
\newcommand{\vecspCb}{W_2}
%:nomenc-exclude:
\newcommand{\vecspCc}{W_3}
%:nomenc-exclude:

%%%%%%%%%%%%%%%%%%%%%%%%% Relations
%:section:gen/relations/styles:
\newcommand{\relstyle}[1]{\stylemorph{#1}} % style for relation (temporary)

\newcommand{\makesymbolrel}[1]{\EM{\relstyle{#1}}}
\newcommand{\relA}{\makesymbolrel{R}}
%:nomenc:\relA, \relB:Generic relation names.
\newcommand{\relB}{\makesymbolrel{S}}
%:nomenc-exclude:
\newcommand{\relC}{\makesymbolrel{T}}
%:nomenc-exclude:
\newcommand{\relD}{\makesymbolrel{U}}
%:nomenc-exclude:

\newcommand{\transclos}{^{\relstyle{+}}}
\newcommand{\traclos}[1]{{#1}^{\relstyle{+}}}

\newcommand{\reltransp}{^{\intercal}}
%:nomenc:\relA\reltransp:Transpose of a relation $\relA$.

%%%%%%%%%%%%%%%%%%%%%%%%% Posets
%:section:gen/posets/generic:Generic poset names

\newcommand{\stylepos}[1]{{\textcolor{colorposetsymbol}{#1}}}
%:nomenc-exclude:
\newcommand{\posAlet}{P}
%:nomenc-exclude:
\newcommand{\posAeln}{p}
%:nomenc-exclude:
\newcommand{\posBlet}{Q}
%:nomenc-exclude:
\newcommand{\posBeln}{q}
%:nomenc-exclude:
\newcommand{\posClet}{R}
%:nomenc-exclude:
\newcommand{\posCeln}{r}
%:nomenc-exclude:
\newcommand{\posDlet}{S}
%:nomenc-exclude:
\newcommand{\posDeln}{s}
%:nomenc-exclude:
\newcommand{\posElet}{T}
%:nomenc-exclude:
\newcommand{\posEeln}{t}
%:nomenc-exclude:
\newcommand{\posFlet}{U}
%:nomenc-exclude:
\newcommand{\posFeln}{u}
%:nomenc-exclude:

\newcommand{\preaugm}[1]{\mathscr{C}(#1)}
\newcommand{\freechoice}[2]{\mathbin{\F{\bigsqcap_{#1\setin #2}}}}
\newcommand{\forcedchoice}[2]{\mathbin{\R{\bigsqcup_{#1\setin #2}}}}
\newcommand{\freechoicesmall}{\mathbin{\F{\sqcap}}}
\newcommand{\forcedchoicesmall}{\mathbin{\R{\sqcup}}}
\newcommand{\makesymbolpos}[1]{\EM{\stylepos{\mathbf{#1}}}}

\newcommand{\posA}{\makesymbolpos{\posAlet}}
\newcommand{\subposA}{\makesymbolpos{\posDlet}}
%:nomenc: \posA, \posB, \posC: Generic posets
%:def:def:poset
\newcommand{\posAset}{\makesymbolset{\posAlet}} % The underlying set for poset $\posA$.
%:nomenc:\posAset, \posBset, \posCset: Underlying set for the posets
%:def:def:poset
\newcommand{\posAsetn}[1]{\makesymbolset{\posAlet}_{\color{formulasetcolor}{#1}}}
%:nomenc-exclude:
\newcommand{\posgenA}{\mathbf{\posAlet}}
%:nomenc-exclude:
\newcommand{\posAel}{\styleelements{\posAeln}}
%:nomenc-exclude:
\newcommand{\posAnel}[1]{\styleelements{\posAeln_{#1}}}
%:nomenc-exclude:
\newcommand{\posAopel}{\styleelements{\posAeln\opel}}
%:nomenc:\posAel, \posBel, \posCel: Generic elements of posets
%:nomenc-exclude:
\newcommand{\posgenAel}{\posAeln}
%:nomenc-exclude:
\newcommand{\posB}{\makesymbolpos{\posBlet}}
%:nomenc-exclude:
\newcommand{\posBset}{\stylesets{\posBlet}}
%:nomenc-exclude:
\newcommand{\posBsetn}[1]{\stylesets{\posBlet}_{\color{formulasetcolor}{#1}}}
%:nomenc-exclude:
\newcommand{\posgenB}{\mathbf{\posBlet}}
%:nomenc-exclude:
\newcommand{\posBel}{\styleelements{\posBeln}}
%:nomenc-exclude:
\newcommand{\posBnel}[1]{\styleelements{\posBeln_{#1}}}
%:nomenc-exclude:
\newcommand{\posBopel}{\styleelements{\posBeln\opel}}
%:nomenc-exclude:
\newcommand{\posgenBel}{\posBeln}
%:nomenc-exclude:

\newcommand{\posC}{\makesymbolpos{\posClet}}
%:nomenc-exclude:
\newcommand{\posCset}{\stylesets{\posClet}}
%:nomenc-exclude:
\newcommand{\posCsetn}[1]{\stylesets{\posClet}_{\color{formulasetcolor}{#1}}}
%:nomenc-exclude:
\newcommand{\posgenC}{\mathbf{\posClet}}
%:nomenc-exclude:
\newcommand{\posCel}{\styleelements{\posCeln}}
%:nomenc-exclude:
\newcommand{\posCnel}[1]{\styleelements{\posCeln_{#1}}}
%:nomenc-exclude:
\newcommand{\posCopel}{\styleelements{\posCeln\opel}}
%:nomenc-exclude:
\newcommand{\posgenCel}{\posCeln}
%:nomenc-exclude:
\newcommand{\posD}{\makesymbolpos{\posDlet}}
%:nomenc-exclude:
\newcommand{\posDset}{\stylesets{\posDlet}}
%:nomenc-exclude:
\newcommand{\posDsetn}[1]{\stylesets{\posDlet}_{\color{formulasetcolor}{#1}}}
%:nomenc-exclude:
\newcommand{\posgenD}{\mathbf{\posDlet}}
%:nomenc-exclude:
\newcommand{\posDel}{\styleelements{\posDeln}}
%:nomenc-exclude:
\newcommand{\posDnel}[1]{\styleelements{\posDeln_{#1}}}
%:nomenc-exclude:
\newcommand{\posgenDel}{\posDeln}
%:nomenc-exclude:
\newcommand{\posE}{\makesymbolpos{\posElet}}
%:nomenc-exclude:
\newcommand{\posEnel}[1]{\styleelements{\posEeln_{#1}}}
%:nomenc-exclude:
\newcommand{\posF}{\makesymbolpos{\posFlet}}
%:nomenc-exclude:
\newcommand{\posFnel}[1]{\styleelements{\posFeln_{#1}}}
%:nomenc-exclude:
\newcommand{\posela}{\ela}
%:nomenc:\posela, \poselb, \poselc, \poseld: Generic elements of posets
\newcommand{\poselna}[1]{\elna{#1}}
%:nomenc-exclude:
\newcommand{\poselb}{\elb}
%:nomenc-exclude:
\newcommand{\poselnb}[1]{\elnb{#1}}
%:nomenc-exclude:
\newcommand{\poselc}{\elc}
%:nomenc-exclude:
\newcommand{\poselnc}[1]{\elnc{#1}}
%:nomenc-exclude:
\newcommand{\poselan}{\elan}
%:nomenc-exclude:
\newcommand{\poselbn}{\elbn}
%:nomenc-exclude:
\newcommand{\poselcn}{\elcn}
%:nomenc-exclude:
\newcommand{\poseld}{\eld}
%:nomenc-exclude:
\newcommand{\funposA}{\F{\posgenA}}
%:nomenc-exclude:
\newcommand{\funposB}{\F{\posgenB}}
%:nomenc-exclude:
\newcommand{\funposC}{\F{\posgenC}}
%:nomenc-exclude:
\newcommand{\funposD}{\F{\posgenD}}
%:nomenc-exclude:

\newcommand{\funposAel}{\F{\posAeln}}
%:nomenc-exclude:
\newcommand{\funposAopel}{\F{\posAeln\opel}}
%:nomenc-exclude:
\newcommand{\funposBel}{\F{\posBeln}}
%:nomenc-exclude:
\newcommand{\funposBopel}{\F{\posBeln\opel}}
%:nomenc-exclude:
\newcommand{\funposCel}{\F{\posCeln}}
%:nomenc-exclude:
\newcommand{\funposCopel}{\F{\posCeln\opel}}
%:nomenc-exclude:
\newcommand{\funposDel}{\F{\posDeln}}
%:nomenc-exclude:
\newcommand{\funposDopel}{\F{\posDeln\opel}}
%:nomenc-exclude:

\newcommand{\RB}[1]{\R{\mathbf{#1}}}
\newcommand{\FB}[1]{\F{\mathbf{#1}}}

\newcommand{\resposA}{\R{\posgenA}}
%:nomenc-exclude:
\newcommand{\resposB}{\R{\posgenB}}
%:nomenc-exclude:
\newcommand{\resposC}{\R{\posgenC}}
%:nomenc-exclude:
\newcommand{\resposD}{\R{\posgenD}}
%:nomenc-exclude:

\newcommand{\resposAel}{\R{\posAeln}}
%:nomenc-exclude:
\newcommand{\resposBel}{\R{\posBeln}}
%:nomenc-exclude:
\newcommand{\resposCel}{\R{\posCeln}}
%:nomenc-exclude:
\newcommand{\resposDel}{\R{\posDeln}}
%:nomenc-exclude:

\newcommand{\posgenX}{\mathbf{X}}
%:nomenc-exclude:
\newcommand{\posX}{\makesymbolpos{X}}
%:nomenc-exclude:
\newcommand{\posY}{\makesymbolpos{Y}}
%:nomenc-exclude:
\newcommand{\posXel}{\styleelements{x}}
%:nomenc-exclude:
\newcommand{\posYel}{\styleelements{y}}
%:nomenc-exclude:
\newcommand{\posgenXel}{x}
%:nomenc-exclude:
\newcommand{\posgenY}{\mathbf{Y}}
%:nomenc-exclude:
\newcommand{\posgenYel}{y}
%:nomenc-exclude:

% \newcommand{\interval}[1]{\left[#1\right]}
\newcommand{\interv}[2]{{\colL[}#1,\smallspace#2{\colU]}}
%:nomenc:\interv{a}{b}: Interval with boundaries $a$ and $b$.
\newcommand{\posAleq}{\posleqof\posA} % Order on poset $\posA$
\newcommand{\posAleqop}{\posleqopof\posA} % Order on poset $\posA$
%:nomenc-exclude:

\newcommand{\posDPleq}{\posleqof\DP}
%:nomenc-exclude:
\newcommand{\posAgeq}{\posgeqof\posA}
%:nomenc-exclude:
\newcommand{\posBleq}{\posleqof\posB}
%:nomenc-exclude:
\newcommand{\posCleq}{\posleqof\posC}
%:nomenc-exclude:
\newcommand{\posDleq}{\posleqof\posD}
%:nomenc-exclude:
\newcommand{\posAMin}{\mathop{{\Min_{\posAleq}}}}
%:nomenc-exclude:
\newcommand{\posAmin}{\mathop{{\min_{\posAleq}}}}
%:nomenc-exclude:
\newcommand{\posAmax}{\mathop{{\max_{\posAleq}}}}
%:nomenc-exclude:
\newcommand{\posAA}{\antichains\posA}
%:nomenc-exclude:
\newcommand{\posPA}{\powerset\posA}
%:nomenc-exclude:
\newcommand{\upperb}{\stylesets{\aword{UppB}}\,} % Upper bound
%:nomenc-exclude:
\newcommand{\lowerb}{\stylesets{\aword{LowB}}\,} % Lower bound
%:nomenc-exclude:
\newcommand{\posUA}{\uppersets\posA} % Upper sets of poset $\posA$
%:nomenc-exclude:
\newcommand{\posUB}{\uppersets\posB}
%:nomenc-exclude:
\newcommand{\upperposet}[1]{\tup{\setOfUppersets (\stylesets{#1}) \com \supseteq}}
\newcommand{\upperposetpar}[1]{\tup{\uppersets (\stylesets{#1}) \com  \supseteq}}
\newcommand{\upperposetop}[1]{\tup{\setOfUppersets(\stylesets{#1}) \com \subseteq}}
\newcommand{\lowerposet}[1]{\tup{\setOfLowersets(\stylesets{#1})\com \subseteq}}
\newcommand{\lowerposetpar}[1]{\tup{\setOfLowersets(\stylesets{#1})\com \subseteq}}
\newcommand{\lowerposetop}[1]{\tup{\setOfLowersets(\stylesets{#1})\com \supseteq}}
\newcommand{\upperlowerposet}[1]{\tup{\setOfUppersets (\setOfLowersets(\stylesets{#1})) \com \supseteq}}
\newcommand{\lowerupperposet}[1]{\tup{\setOfLowersets (\setOfUppersets(\stylesets{#1})) \com \subseteq}}

\newcommand{\embeddinggen}{\stylemaps{i}}
\newcommand{\embeddingup}{\stylemaps{i}_{\stylemaps{\mathrm{U}}}}
\newcommand{\embeddinglow}{\stylemaps{i}_{\stylemaps{\mathrm{L}}}}
%:nomenc-exclude:
\newcommand{\posLA}{\lowersets\posA} % Lower sets of poset $\posA$
%:nomenc-exclude:
\newcommand{\posLB}{\lowersets\posB}
%:nomenc-exclude:
\newcommand{\posAAleq}{\posleqof{\posAA}}
%:nomenc-exclude:
\newcommand{\posint}[1]{\PosTwistedArrow#1}
\newcommand{\posintbis}[1]{\PosArrow#1}

\newcommand{\powerposet}{\stylefunctors{\aword{Pow}}\,}
%:nomenc:\powerposet \setA: Power poset of $\setA$.
%:def:def:power-poset


%:section:gen/posets/operations-sets:Operations on sets
\newcommand{\Min}{\operatorname*{Min}}
%:nomenc:\Min_{\posAleq} \subA : Minimal elements of the subset $\subA$.
%:def:def:Min
\newcommand{\Max}{\operatorname*{Max}}
%:nomenc:\Max_{\posAleq} \subA : Maximal elements of the subset $\subA$.
%:def:def:Max

\newcommand{\Inf}{\aword{Inf}\,}
%:nomenc:\Inf_{\posAleq} \subA : Infimum element of the subset $\subA$.
%:def:def:greatest-lower-bound
\newcommand{\Sup}{\aword{Sup}\,}
%:nomenc:\Sup_{\posAleq} \subA : Supremum element of the subset $\subA$.
%:def:def:least-upper-bound

% \newcommand{\low}{\mathsf{L}}
% \newcommand{\upp}{\mathsf{U}}

\newcommand{\upit}{{\colUp \pmb{\uparrow}}} % upper closure
%:nomenc:\upit \subA: Upper closure of $\subA$.
%:def:def:upperclosure

\newcommand{\downit}{{\colDown \pmb{\downarrow}}} % lower closure
%:nomenc:\downit \subA: Lower closure of $\subA$.
%:def:def:lowerclosure

%:section:gen/posets/operations-elements:Operations on elements

\newcommand{\join}{\mathbin{\pmb{\vee}}} % Join
%:nomenc:\ela \join \elb: Join of two elements $\ela$, $\elb$
%:def:def:lattice

\newcommand{\meet}{\mathbin{\pmb{\wedge}}} % Meet
%:nomenc:\ela \meet \elb: Meet of two elements $\ela$, $\elb$
%:def:def:lattice

%:section:gen/posets/constructors: Constructors

\newcommand{\antichains}{{\colAnti \text{Anti}}\,} % antichain symbols
%:nomenc:\antichains \posA: The set of antichains of a poset $\posA$.
%:def:def:antichain
\newcommand{\fantichains}{{\colAnti \aword{Anti}}_\mathrm{f}\,} % finite antichain
%:nomenc:\fantichains \posA: The set of finite antichains of a poset $\posA$.

\newcommand{\setOfUppersets}{{\colU\aword{USets}}\,}
%:nomenc:\setOfUppersets \posA: The set of upper sets of a poset $\posA$.
\newcommand{\setOfLowersets}{{\colL\aword{LSets}}\,}
%:nomenc:\setOfLowersets \posA: The set of lower sets of a poset $\posA$.
\newcommand{\setOfDCUppersets}{\overline{\colU\aword{UpSets}}\,}
%:nomenc:\setOfDCUppersets \posA : $\subset \setOfUppersets \posA$. The set of downward-closed upper sets of poset $\posA$.
\newcommand{\setOfUCLowersets}{\overline{\colL\aword{LowSets}}\,}
%:nomenc:\setOfUCLowersets \posA : $\subset \setOfLowersets \posA$. The set of upward-closed upper sets of poset $\posA$.

\newcommand{\uppersets}{\mathscr{\colUp U}}
%:nomenc:\uppersets \posA : $= \tup{\setOfUppersets \posA, \supseteq}$. Poset of downward-closed uppersets of $\posA$ ordered by $\supseteq$.
\newcommand{\lowersets}{\mathscr{\colDown L}}
%:nomenc:\lowersets \posA : $=\tup{\setOfUppersets \posA, \subseteq}$ Poset of upward-closed lowerset of $\posA$ ordered by $\subseteq$.
%:def:def:lowerset
\newcommand{\dcuppersets}{\overline{\uppersets}} % closed upper sets
%:nomenc:\dcuppersets \posA : $= \tup{\setOfDCUppersets \posA, \supseteq}$ Poset of downward-closed uppersets of $\posA$ ordered by $\supseteq$.
\newcommand{\uclowersets}{\overline{\lowersets}}
%:nomenc:\uclowersets \posA : $= \tup{\setOfUCLowersets \posA, \subseteq}$ Poset of upward-closed lowersets of $\posA$.

\newcommand{\fuppersets}{\overline{\uppersets}_\mathrm{f}}
%:nomenc:\fuppersets \posA: $ \subseteq \dcuppersets  \posA $ Poset of finitely-supported upper sets of a poset $\posA$.
%:def:def:fuppersets
\newcommand{\flowersets}{\overline{\lowersets}_\mathrm{f}}
%:nomenc:\flowersets \posA : $\subseteq \uclowersets  \posA $ Poset of finitely-supported lower sets  of a poset $\posA$.

\newcommand{\Rleq}{\mathrel{\stylemorph{\leq}}} % $\leq$ for $\reals$ (making sure to use right one)
%:nomenc-exclude:
\newcommand{\Rgeq}{\mathrel{\stylemorph{\geq}}} % $\leq$ for $\reals$ (making sure to use right one)
%:nomenc-exclude:
\newcommand{\Nleq}{\mathrel{\stylemorph{\leq}}} % $\leq$ for $\natnumbers$ (making sure to use right one)
%:nomenc-exclude:
\newcommand{\Up}[1]{\uppersets#1}
%:nomenc-exclude:
\newcommand{\Lo}[1]{\lowersets#1}
%:nomenc-exclude:

%%%%%%%%%%%%%%%%%%%%%%%%% Categories
%:section:gen/categories: Categories

%:section:gen/colors: Color
\definecolor{functionality}{rgb}{0.094869,0.500000,0.000000}
\definecolor{requirements}{rgb}{0.555789,0.000000,0.000000}
\definecolor{implementations}{RGB}{214,120,5}
\definecolor{functionalitylight}{rgb}{0.094869,0.800000,0.000000}
\definecolor{requirementslight}{rgb}{0.8,0.000000,0.000000}
\definecolor{implementationslight}{RGB}{214,120,5}
\definecolor{listcolor}{named}{black}
\definecolor{setcolor}{RGB}{243, 180, 126} % for background colors in pictures
\definecolor{setcolorbord}{RGB}{253, 143, 47}
\definecolor{posetcolor}{rgb}{0.725, 0.875, 0.992} % for background colors in pictures
\definecolor{posetcolorbord}{RGB}{81, 171, 252} % for background colors in pictures
% \definecolor{graphcolor}{rgb}{0.75, 0.75, 0.7} % for background colors in graphss
\definecolor{graphcolor}{RGB}{252, 246, 207}% for background colors in graphss
\definecolor{graphcolorbord}{RGB}{219, 213, 173} % for background colors in pictures
\definecolor{formulasetcolor}{named}{brown} % for formulas
\definecolor{formulasetLcolor}{RGB}{213,87,96} % for formulas
\definecolor{elementscolor}{rgb}{0.1,0,0.4}
\definecolor{catcolor}{rgb}{0.78, 0.79, 0.79}
\definecolor{darkgreen}{rgb}{0.0, 0.5, 0.0}
\definecolor{darkred}{rgb}{0.5, 0.0, 0.0}
\definecolor{brightpink}{rgb}{1.0, 0.65, 0.79}
\definecolor{applegreen}{rgb}{0.55, 0.71, 0.0}
\definecolor{custompurple}{rgb}{0.18,0.46,0.74}
\definecolor{custompink}{rgb}{0.73,0.83,0.91}
\definecolor{funcname}{named}{blue}
\definecolor{classname}{named}{blue}
\definecolor{fieldname}{named}{darkgreen}
\definecolor{parula1}{RGB}{53,42,135}
\definecolor{parula2}{RGB}{15,92,221}
\definecolor{parula3}{RGB}{18,125,216}
\definecolor{parula4}{RGB}{7,156,207}
\definecolor{parula5}{RGB}{21,177,180}
\definecolor{parula6}{RGB}{21,177,180}
\definecolor{parula7}{RGB}{89,189,140}
\definecolor{parula8}{RGB}{165,190,107}
\definecolor{parula9}{RGB}{225,185,82}

\def\dpgreen{functionality}
\def\dpred{requirements}

\definecolor{darkblue}{rgb}{0.0, 0.0, 0.55}
\definecolor{shadecolor}{named}{LightBlue}
\definecolor{transmuter}{RGB}{46,117,189}
\definecolor{objects}{RGB}{213,87,96}
\definecolor{morphisms}{RGB}{46,117,189}
\definecolor{colorposetsymbol}{RGB}{90,90,90}

\definecolor{transmuted}{RGB}{213,87,96}
\definecolor{naturaltransformations}{rgb}{0.1,0.5,0.0}
\definecolor{functors}{rgb}{0.5,0,0.5}

\definecolor{instructors}{RGB}{214,120,5}
\definecolor{upcolor}{RGB}{252, 3, 107}
\definecolor{downcolor}{named}{Orange}
\definecolor{antichaincolor}{named}{Brown}
\definecolor{staincola}{RGB}{181,23,0}
\definecolor{staincolb}{RGB}{255,150,141}
\definecolor{staincolc}{RGB}{181,23,30}
\definecolor{staincold}{RGB}{241,219,200}
\definecolor{staincole}{RGB}{92,24,13}
\definecolor{staincolf}{RGB}{29,177,0}
\definecolor{ropecola}{RGB}{187,127,14}
\definecolor{ropecolb}{RGB}{97,216,54}
% color commands
\newcommand{\colR}{\color{requirements}}
%:nomenc-exclude:
%:example: $\colR X$
\newcommand{\colF}{\color{functionality}}
%:nomenc-exclude:
%:example: $\colF X$
\newcommand{\colI}{\color{implementations}}
%:nomenc-exclude:
%:example: $\colI X$
\newcommand{\colH}{\color[rgb]{0.000000,0.400000,1.000000}}
%:nomenc-exclude:
%:example: $\colH X$
\newcommand{\colU}{\color{upcolor}}
%:nomenc-exclude:
%:example: $\colU X$
\newcommand{\colL}{\color{downcolor}}
%:nomenc-exclude:
%:example: $\colL X$
\newcommand{\colUp}{\color{upcolor}}
%:nomenc-exclude:
%:example: $\colUp X$
\newcommand{\colAnti}{\color{antichaincolor}}
%:nomenc-exclude:
%:example: $\colAnti X$
\newcommand{\colDown}{\color{downcolor}}
%:nomenc-exclude:
%:example: $\colDown X$
\newcommand{\colTransmuter}{\color{transmuter}}
%:nomenc-exclude:
%:example: $\colTransmuter X$
\newcommand{\colTransmuted}{\color{transmuted}}
%:nomenc-exclude:
%:example: $\colTransmuted X$
\newcommand{\blue}[1]{\textcolor{blue}{#1}}
%:nomenc-exclude:
\newcommand{\F}[1]{{\textcolor{\dpgreen}{#1}}}
%:nomenc-exclude:
\newcommand{\Rdia}[1]{{\textcolor{\dpred}{#1}}}
%:nomenc-exclude:
\newcommand{\R}[1]{{\textcolor{\dpred}{#1}}}
\newcommand{\Rtext}[1]{\text{\textcolor{\dpred}{#1}}}
\newcommand{\Ftext}[1]{\text{\textcolor{\dpgreen}{#1}}}
%:nomenc-exclude:
\newcommand{\Fdia}[1]{\color{\dpgreen}#1}
%:nomenc-exclude:

\newcommand{\Rcol}[1]{{\colR #1}}
%:nomenc-exclude:
%:example: \Rcol{X}
\newcommand{\Fcol}[1]{{\colF #1}}
%:nomenc-exclude:
%:example: \Fcol{X}
\newcommand{\Icol}[1]{{\colI #1}}
%:nomenc-exclude:
%:example: \Icol{X}
\newcommand{\gray}[1]{{\color{gray}#1}}
%:nomenc-exclude:
%:example: \gray{X}
\newcommand{\bchanges}{\color[rgb]{0,0.3,0}}
%:nomenc-exclude:
%:example: \bchanges{X}
\newcommand{\changes}[1]{{\color[rgb]{0,0.3,0}#1}}
%:nomenc-exclude:
%:example: \changes{X}
\newcommand{\echanges}{\color[rgb]{0,0,0}}
%:example: \echanges{X}
%:nomenc-exclude:

%:section:gen/comments: Personal comments
%:nomenc-exclude:
\newcommand{\AC}[1]{{\color{blue}AC: #1}}
\newcommand{\GZ}[1]{{\color{green}GZ: #1}}
\newcommand{\DS}[1]{{\color{blue!50!red} DS: #1}}
\newcommand{\JT}[1]{{\color{blue!30!green!30!black}JT: #1}}
%:example: \JT{blah}
\newcommand{\JL}[1]{{\color{magenta}JL: #1}}
%:example: \JL{blah}

%:section:gen/categories/styles: Styles
\newcommand{\styleobj}[1]{{\colTransmuted{#1}}}
%:nomenc-exclude:
\newcommand{\stylemorph}[1]{{\textcolor{morphisms}{#1}}}
%:nomenc-exclude:
\newcommand{\stylefunctors}[1]{{\textcolor{functors}{#1}}}
%:nomenc-exclude:
\newcommand{\stylemonads}[1]{{\textcolor{functors}{\mathcal{#1}}}}
%:nomenc-exclude:
\newcommand{\stylenat}[1]{{\textcolor{naturaltransformations}{#1}}}
%:nomenc-exclude:
\newcommand{\styledp}[1]{\stylefunctors{#1}}

%:section:gen/categories/composition: Composition

\newcommand{\after}{\mathbin{\circ}}
%:nomenc:b \after a:``$b$ after $a$''

% \newcommand{\then}{\mathbin{\fatsemi}} % Composition (in general)
%%%:nomenc:a \then b:``$a$ then $b$''

\newcommand{\mthen}{\mathbin{\stylemorph{\fatsemi}}\linebreak[0]} % Composition for morphisms
%:nomenc:\mora \mthen \morb: Composition of morphisms
\newcommand{\mthenof}[1]{\mathbin{\stylemorph{\fatsemi}_{#1}}\linebreak[0]} % Composition for morphisms
%:nomenc-exclude:

\newcommand{\fthen}{\mathbin{\stylefunctors{\fatsemi}}} % Composition for functors
\newcommand{\fthenof}[1]{\mathbin{\stylefunctors{\fatsemi}_{#1}}\linebreak[0]} % Composition for morphisms
%:nomenc:\funa \fthen \funb: Composition of functors

% \newcommand{\nthen}{\mathbin{\stylenat{\fatsemi}}} % Compositions for natural transformations
%  :nomenc:\ntrafoa \nthen \ntrafob: Composition of natural transformations

\newcommand{\nunit}{\eta}
\newcommand{\ncounit}{\varepsilon}

\newcommand{\Ob}{\styleobj{\operatorname{Ob}}}
%:nomenc:\Ob_\CatA: Objects of the category $\CatA$.
%:def:def:categorymain

\newcommand{\Mor}{\stylemorph{\operatorname{Mor}}}
%:nomenc:\Mor_\CatA: Collection of all morphisms of the category $\CatA$.
%:def:def:categorymain

\newcommand{\catid}{\stylemorph{\idname}} % identity for category
%:nomenc:\catidof\Obja:Identity morphism for the object $\Obja$
%:def:def:categorymain

\newcommand{\catidof}[1]{\catid_{#1}} % Identity for an object
%:nomenc-exclude:
\newcommand{\catidat}[1]{\catid_{#1}} % Identity for an object

\newcommand{\catidofat}[2]{\catid^{#1}_{#2}} % Identity for an object

% \newcommand{\Cat}[1]{\EM{{\index{#1@\textbf{#1}}\mathbf{#1}}}}
% \newcommand{\CatLinkConcept}[2]{\EM{{\SYNinner{#2}{\CatStyle{#1}}}}}
\newcommand{\CatLinkConcept}[2]{{\SYNinner{#2}{\ensuremath{\CatStyle{#1}}}}\xspace}
\newcommand{\LinkConcept}[2]{{\SYNinner{#2}{\ensuremath{#1}}}\xspace}
%:example:-
%:nomenc-exclude:
\newcommand{\cCatLinkConcept}[2]{{\SYNinner{#2}{\ensuremath{\cCatStyle{#1}}}}\xspace}
%:example:-
%:nomenc-exclude:
\newcommand{\cCatStyle}[1]{{\fatstart\mathbf{#1}\fatend}}%
%:nomenc-exclude:
\newcommand{\CatStyle}[1]{\EM{\mathbf{#1}}}
%:nomenc-exclude:

\newcommand{\Hom}{\stylefunctors{\operatorname{Hom}}}
%:nomenc:\HomSet{\CatA}{\Obja}{\Objb}: Hom-set between $\Obja$ and $\Objb$.
%:def:def:categorymain

\newcommand{\HomSet}[3]{\Hom_{#1}\left({#2};\linebreak[0]{#3}\right)}
%:nomenc-exclude:

%:section:gen/categories/generic: Generic names
\newcommand{\CatA}{\CatStyle{A}}
%:nomenc:\CatA,\CatB,\CatC, \CatD, \dots: Symbols for categories
\newcommand{\CatB}{\CatStyle{B}}
%:nomenc-exclude:
\newcommand{\CatC}{\CatStyle{C}}
%:nomenc-exclude:
\newcommand{\CatD}{\CatStyle{D}}
%:nomenc-exclude:
\newcommand{\CatE}{\CatStyle{E}}
%:nomenc-exclude:
\newcommand{\CatV}{\CatStyle{V}}
%:nomenc:\CatV: Symbol for enrichment categories.
\newcommand{\CatI}{\CatStyle{I}}
%:nomenc:\CatI, \CatJ, \CatK: Symbol for a diagram shape category.
\newcommand{\CatJ}{\CatStyle{J}}
%:nomenc:\CatJ: Symbol for a diagram shape category.
%:nomenc-exclude:
\newcommand{\CatK}{\CatStyle{K}}
%:nomenc:\CatK: Symbol for a diagram shape category.
%:nomenc-exclude:

\newcommand{\Obof}[1]{\Ob_{#1}}
\newcommand{\ObC}{\Obof\CatC}
%:nomenc-exclude:
\newcommand{\ObD}{\Obof\CatD}
%:nomenc-exclude:
\newcommand{\twoCat}{\CatStyle{2}}
%:nomenc-exclude:

\newcommand{\Obja}{\styleobj{X}}
%:nomenc:\Obja, \Objb, \Objc, \Objd: Generic objects in a category

\newcommand{\Objan}[1]{\styleobj{X_{#1}}}
%:nomenc-exclude:

\newcommand{\Objaset}{\makesymbolset{X}} % ???
%:nomenc-exclude:

\newcommand{\Objael}{\styleobj{x}}
%:nomenc-exclude:

\newcommand{\Objaela}{\styleobj{x_1}}
%:nomenc-exclude:

\newcommand{\Objaelb}{\styleobj{x_2}}
%:nomenc-exclude:

\newcommand{\Objb}{\styleobj{Y}}
%:nomenc-exclude:

\newcommand{\Objbn}[1]{\styleobj{Y_{#1}}}
%:nomenc-exclude:

\newcommand{\Objbset}{\makesymbolset{Y}}
%:nomenc-exclude:

\newcommand{\Objbel}{\styleobj{y}}
%:nomenc-exclude:

\newcommand{\Objbela}{\styleobj{y_1}}
%:nomenc-exclude:

\newcommand{\Objbelb}{\styleobj{y_2}}
%:nomenc-exclude:

\newcommand{\Objc}{\styleobj{Z}}
%:nomenc-exclude:

\newcommand{\Objcn}[1]{\styleobj{Z_{#1}}}
%:nomenc-exclude:

\newcommand{\Objcset}{\makesymbolset{Z}}
%:nomenc-exclude:

\newcommand{\Objcel}{\styleobj{z}}
%:nomenc-exclude:
\newcommand{\Objcela}{\styleobj{z_1}}
%:nomenc-exclude:
\newcommand{\Objcelb}{\styleobj{z_2}}
%:nomenc-exclude:
\newcommand{\Objd}{\styleobj{U}}
%:nomenc-exclude:
\newcommand{\Objdset}{\makesymbolset{U}}
%:nomenc-exclude:
\newcommand{\Objdel}{\styleobj{u}}
%:nomenc-exclude:
\newcommand{\Obje}{\styleobj{V}}
%:nomenc-exclude:
\newcommand{\Objeset}{\makesymbolset{V}}
%:nomenc-exclude:
\newcommand{\Objf}{\styleobj{W}}
%:nomenc-exclude:
\newcommand{\Objfset}{\makesymbolset{W}}
%:nomenc-exclude:

\newcommand{\mora}{\stylemorph{f}}
%:nomenc:\mora,\morb,\morc,\mord: Generic morphisms
%:def:def:categorymain
\newcommand{\moran}[1]{\stylemorph{f_{#1}}}
%:nomenc-exclude:
\newcommand{\morb}{\stylemorph{g}}
%:nomenc-exclude:
\newcommand{\morbn}[1]{\stylemorph{g_{#1}}}
%:nomenc-exclude:
\newcommand{\morc}{\stylemorph{h}}
%:nomenc-exclude:
\newcommand{\morcn}[1]{\stylemorph{h_{#1}}}
%:nomenc-exclude:
\newcommand{\mord}{\stylemorph{i}}
%:nomenc-exclude:
\newcommand{\more}{\stylemorph{j}}
%:nomenc-exclude:
\newcommand{\morf}{\stylemorph{k}}
%:nomenc-exclude:
\newcommand{\morg}{\stylemorph{l}}
%:nomenc-exclude:
\newcommand{\morh}{\stylemorph{m}}
%:nomenc-exclude:
\newcommand{\mori}{\stylemorph{n}}
%:nomenc-exclude:
\newcommand{\cohm}{\stylemorph{\text{coh}}}
%:nomenc-exclude:
\newcommand{\cohmbis}{\stylemorph{\text{coh}'}}
%:nomenc-exclude:


\newcommand{\isomorphic}{\simeq}



\newcommand{\duplicator}{\stylemorph{d}}
\newcommand{\discarder}{\stylemorph{e}}
\newcommand{\terminator}{\stylemorph{t}}

\newcommand{\dtsysa}{\mora}
\newcommand{\dtsysb}{\morb}
\newcommand{\dtsysc}{\morc}

\newcommand{\cofun}{\stylefunctors{\Pi_\F{f}}}
\newcommand{\confun}{\stylefunctors{\Pi_\R{r}}}
\newcommand{\funeqa}{\swarrow}
\newcommand{\funeqb}{\nearrow}
\newcommand{\funobspace}[1]{\stylefunctors{#1}_{\!\styleobj{\bullet}}}
\newcommand{\funmorspace}[1]{\stylefunctors{#1}_{\!\stylemorph{\shortto}}}
% \newcommand{\funobspace}[1]{\styleobj{#1}_{\!\styleobj{\bullet}}}
% \newcommand{\funmorspace}[1]{\stylemorph{#1}_{\!\stylemorph{\shortto}}}

\newcommand{\funamor}{\funmorspace{F}}
\newcommand{\funaob}{\funobspace{F}}
\newcommand{\funbmor}{\funmorspace{G}}
\newcommand{\funbob}{\funobspace{G}}

\newcommand{\polyfun}{\stylefunctors{P}}

\newcommand{\dpitodp}{\stylefunctors{F}}
\newcommand{\dpitodpmor}{\funmorspace{F}}
\newcommand{\dpitodpob}{\funobspace{F}}
\newcommand{\dptodpi}{\stylefunctors{G}}
\newcommand{\dptodpimor}{\funmorspace{G}}
\newcommand{\dptodpiob}{\funobspace{G}}

\newcommand{\funa}{\stylefunctors{F}}
%:nomenc:\funa,\funb,\func,\fund: Generic functors
%:def:def:functor
\newcommand{\funb}{\stylefunctors{G}}
%:nomenc-exclude:
\newcommand{\func}{\stylefunctors{H}}
%:nomenc-exclude:
\newcommand{\fund}{\stylefunctors{K}}
%:nomenc-exclude:

\newcommand{\funaA}{\stylefunctors{F_1}}
%:nomenc-exclude:
\newcommand{\funaB}{\stylefunctors{F_2}}
%:nomenc-exclude:
\newcommand{\funaC}{\stylefunctors{F_3}}
%:nomenc-exclude:
\newcommand{\funbA}{\stylefunctors{G_1}}
%:nomenc-exclude:
\newcommand{\funbB}{\stylefunctors{G_2}}
%:nomenc-exclude:
\newcommand{\funbC}{\stylefunctors{G_3}}
%:nomenc-exclude:
\newcommand{\funcA}{\stylefunctors{H_1}}
%:nomenc-exclude:
\newcommand{\funcB}{\stylefunctors{H_2}}
%:nomenc-exclude:
\newcommand{\funcC}{\stylefunctors{H_3}}
%:nomenc-exclude:

\newcommand{\funaMap}{\stylemaps{F}}
%:nomenc-exclude:
\newcommand{\funbMap}{\stylemaps{G}}
%:nomenc-exclude:
\newcommand{\funcMap}{\stylemaps{H}}
%:nomenc-exclude:
\newcommand{\fundMap}{\stylemaps{K}}
%:nomenc-exclude:

\newcommand{\idname}{\operatorname{id}}
% \newcommand{\id}{\idname}


\newcommand{\funid}{\stylefunctors{\idname}} % identity functor
%:nomenc:\funid_{\CatA}: Identity functor for category $\CatA$
\newcommand{\funidof}[1]{\funid_{#1}}
%:nomenc-exclude:
\newcommand{\funidat}[1]{\funid_{#1}}
%:nomenc-exclude:
\newcommand{\ntrafoa}{\stylenat{\alpha}}
%:nomenc:\ntrafoa,\ntrafob,\ntrafoc,\ntrafod: Generic natural transformations
%:def:def:natural-transformation
\newcommand{\ntrafob}{\stylenat{\beta}}
%:nomenc-exclude:
\newcommand{\ntrafoc}{\stylenat{\gamma}}
%:nomenc-exclude:
\newcommand{\ntrafod}{\stylenat{\delta}}
%:nomenc-exclude:


\newcommand{\natid}{\stylenat{\idname}} % identity natural transformation
%:nomenc:\natid_{\funa}: Identity natural transformation for functor $\funa$


%:section:vol1/40_computation/30_monads:\iflabelexists{chap:monads}{\chrefplus{chap:monads}}
\newcommand{\monA}{\stylefunctors{M}}
%:nomenc:\monA,\monB: Generic monads.
%:def:def:monad
\newcommand{\monAA}{{\monA\fthen\monA}}
%:nomenc-exclude:
\newcommand{\monAAA}{{\monA\fthen\monA\fthen\monA}}
%:nomenc-exclude:
\newcommand{\monB}{\stylefunctors{N}}
%:nomenc-exclude:

\newcommand{\monunit}{\stylenat{\aword{un}}} % Monad unit
%:def:def:monad
\newcommand{\moncomp}{\stylenat{\aword{mu}}} % Monad identity
%:def:def:monad
\newcommand{\monadfunction}[1]{\EM{\stylemorph{\text{#1}}}}
\newcommand{\fish}{\monadfunction{fish}}
%:def:def:monad-computer-science
\newcommand{\lift}{\monadfunction{lift}}
%:def:def:monad-computer-science
\newcommand{\mjoin}{\monadfunction{join}}
%:def:def:monad-computer-science
\newcommand{\bind}{\monadfunction{bind}}
%:def:def:monad-computer-science
\newcommand{\fmap}{\monadfunction{fmap}}
%:def:def:monad-computer-science
\newcommand{\return}{\monadfunction{return}}
%:def:def:monad-computer-science

\newcommand{\Uendo}{\stylefunctors{U}} % upper-set endofunctor
%:def:def:Uendo
\newcommand{\Umon}{\stylemonads{U}} % upper-set monad
%:def:def:Umon

\newcommand{\Lendo}{\stylefunctors{L}} % lower-set endofunctor
\newcommand{\Lmon}{\stylemonads{L}} % lower-set monad

% \newcommand{\Int}{\stylemonads{\mathbf{Int}}} % Interval monad
% \newcommand{\Intb}{\stylemonads{\mathbf{Intb}}} % other one

%:section:gen/dp/codesign-spaces: Co-design spaces
\newcommand{\LF}{\lowersets\funsp}
%:nomenc-exclude:
\newcommand{\UR}{\Up\ressp}
%:nomenc-exclude:
\newcommand{\Aressp}{{\antichains\ressp}}
\newcommand{\fAressp}{{\fantichains\ressp}}
%:nomenc-exclude:
\newcommand{\Afunsp}{{\antichains\funsp}}
%:nomenc-exclude:
\newcommand{\Uressp}{\UR}
\newcommand{\Lfunsp}{\LF}
%:nomenc-exclude:

%:section:gen/sets/well-known-functions: Well-known functions

\newcommand{\relid}{\stylemorph{\idname}} % Identity map
\newcommand{\mapid}{\stylemorph{\idname}} % Identity map
%:nomenc:\mapid_{\setA}:Identity map on $\setA$
\newcommand{\mapidat}[1]{\mapid_{#1}}

% \newcommand{\ceil}[1]{\left \lceil #1 \right \rceil}
%  :nomenc:\ceil{x}:Rounding of $x$ to the next integer

\newcommand{\funceil}{\EM{\stylemorph{{\aword{ceil}}}}} % ceiling function
%:def:ex:rounding-functions
\newcommand{\funfloor}{\EM{\stylemorph{\aword{floor}}}} % floor function
%:def:ex:rounding-functions

\newcommand{\rtntte}{\stylemorph{\aword{rtntte}}} % Round to nearest, ties to even
%:def:ex:rounding-functions

%:section:gen/categories/companion: Companion/conjoints
\newcommand{\compmap}{\stylefunctors{\aword{comp}}}
\newcommand{\comp}[1]{\compmap(#1)} % Companion
%:def:def:comp_conj
\newcommand{\conjmap}{\stylefunctors{\aword{conj}}}
\newcommand{\conj}[1]{\conjmap(#1)} % Conjoint
%:def:def:comp_conj

%:section:playground: Playground

\newcommand{\posid}{\makesymbolset{I}}
\newcommand{\posidel}{\styleelements{\iota}}

%:section:misc: Misc

\newcommand{\col}[1]{\mathrm{col(#1)}}

\newcommand{\prodMapob}[2]{{#1}\mathbin{\styleobj{\times}}{#2}}
\newcommand{\prodMapmor}[2]{{#1}\mathbin{\stylemorph{\times}}{#2}}
\newcommand{\coprodMapmor}[2]{{#1}\mathbin{\stylemorph{+}}{#2}}
\newcommand{\coprodMapob}[2]{{#1}\mathbin{\styleobj{+}}{#2}}
\newcommand{\cP}{P}
\newcommand{\cQ}{Q}
\newcommand{\cR}{R}

%:section:gen/categories/operations: Operations
\newcommand{\Ptimes}{\mathbin{\color{colorposetsymbol}\boldtimes}} % Product of posets
\newcommand{\Ctimes}{\boldtimes} % Product in a category
%:def:def:categorical-product

\newcommand{\Pplus}{\mathbin{\color{colorposetsymbol}\boldplus}} % Disjoint union of posets
\newcommand{\Cplus}{\boldplus} % Co-product in a category
%:def:def:catcoproduct

%\newcommand{\Ctimes}{ \tikz[baseline=-.55ex] \node [inner sep=0pt,cross out,draw,line width=1pt,minimum size=1ex] (a) {};}

%:section:gen/categories/constructors: Constructors
\newcommand{\Free}{\CatLinkConcept{Free}{free category construction on graphs}} % free construction
%:def:def:free-category
%:nomenc:\Free(G):Free category constructed from the graph G

% \newcommand{\twisted}[1]{\TwistedArrow#1}
%  :nomenc:\twisted{\CatA}:Twisted arrow construction on category $\CatA$.

\newcommand{\PosArrow}{\textcolor{colorposetsymbol}{\mathbf{Arr}}\,} % Arrow poset
\newcommand{\Arrow}{\ArrowFunctor\,} % Arrow category
\newcommand{\Arrowof}{\ArrowFunctor\,}
\newcommand{\ArrowFunctor}{\stylefunctors{\mathbf{Arr}}}
%:nomenc:\Arrow{\CatA}:Arrow construction on category $\CatA$.
%:def:def:arrow_category
\newcommand{\PosTwistedArrow}{\textcolor{colorposetsymbol}{\mathbf{Tw}}\,} % Twisted Arrow poset

\newcommand{\TwistedArrow}{\stylefunctors{\mathbf{Tw}}\,} % Twisted Arrow category
%:nomenc:\TwistedArrow{\CatA}:Twisted arrow construction on category $\CatA$.
%:def:def:twisted-arrow


\newcommand{\op}{^{\mathrm{op}}}
%:nomenc-exclude:
\newcommand{\opel}{^*}
%:nomenc-exclude:
\newcommand{\Fop}{^{\F{*}}}
%:nomenc-exclude:

\newcommand{\funcbetween}{\aword{between}}
%:nomenc-exclude:

%\newcommand{\fupd}{f^\mathrm{upd}}
%\newcommand{\frdt}{f^\mathrm{rdt}}

%:section:gen/semigroups/names: Semigroups
\newcommand{\sgrpA}{\CatStyle{S}} % Generic semigroup names
%:nomenc:\sgrpA, \sgrpB, \sgrpC:Generic semigroup names.
\newcommand{\sgrpAset}{\makesymbolset{S}} % Generic semigroup names
%:nomenc:\sgrpAset, \sgrpBset, \sgrpCset:Generic names for underlying carrier set of a semigroup.
\newcommand{\sgrpB}{\CatStyle{T}} % Generic semigroup
%:nomenc-exclude:
\newcommand{\sgrpC}{\CatStyle{U}} % Generic semigroup
%:nomenc-exclude:
\newcommand{\sgrpBset}{\makesymbolset{T}} % Generic semigroup
%:nomenc-exclude:
\newcommand{\sgrpCset}{\makesymbolset{U}} % Generic semigroup
%:nomenc-exclude:
\newcommand{\mlog}{\stylefunctors{\log}} % Matrix logarithm
\newcommand{\mexp}{\stylefunctors{\exp}} % Matrix exponential

\newcommand{\sgrpela}{\stylemorph{x}} % Generic semigroup element
%:nomenc:\sgrpela,\sgrpelb,\sgrpelc:Generic semigroup elements
\newcommand{\sgrpelb}{\stylemorph{y}} % Generic semigroup element
%:nomenc-exclude:
\newcommand{\sgrpelc}{\stylemorph{z}} % Generic semigroup element
%:nomenc-exclude:
\newcommand{\sgrpelA}{\stylemorph{s}} % Generic semigroup element of $\sgrpA$.
%:nomenc-exclude:
\newcommand{\sgrpelAa}{\sgrpelA_1} % Generic semigroup element
%:nomenc-exclude:
\newcommand{\sgrpelAb}{\sgrpelA_2} % Generic semigroup element
%:nomenc-exclude:
\newcommand{\sgrpmorA}{\stylefunctors{F}} % Generic semigroup morphism
%:nomenc:\sgrpmorA,\sgrpmorB:Generic semigroup morphisms.
\newcommand{\sgrpmorB}{\stylefunctors{G}} % Generic semigroup morphism
%:nomenc-exclude:

%:section:gen/monoids: Monoids
\newcommand{\idmon}{\stylemorph{\idname}} % identity for monoid
%:def:def:monoid
\newcommand{\monoidA}{\CatStyle{M}} % Generic monoid names
%:nomenc:\monoidA, \monoidB, \dots: Generic monoid names
%:def:def:monoid
\newcommand{\monoidB}{\CatStyle{N}} % Generic monoid
%:nomenc-exclude:
\newcommand{\monoidAset}{\makesymbolset{M}} % Underlying set of $\monoidA$.
%:nomenc:\monoidAset, \monoidBset, \dots: Underlying sets for the monoids.
%:def:def:monoid
\newcommand{\monoidBset}{\makesymbolset{N}} % Underlying set of $\monoidB$.
%:nomenc-exclude:


\newcommand{\monela}{\stylemorph{x}} % Generic monoid element
%:nomenc:\monela, \monelb, \monelc: Generic monoid elements
%:def:def:monoid
\newcommand{\monelb}{\stylemorph{y}} % Generic monoid element
%:nomenc-exclude:
\newcommand{\monelc}{\stylemorph{z}} % Generic monoid element
%:nomenc-exclude:

\newcommand{\monelA}{\stylemorph{m}} % Generic monoid element
%:nomenc:\monelA, \monelB: Generic monoid elements
%:def:def:monoid
\newcommand{\monelAa}{\monelA_1} % Generic monoid element
%:nomenc-exclude:
\newcommand{\monelAb}{\monelB_2} % Generic monoid element
%:nomenc-exclude:
\newcommand{\monelB}{\stylemorph{n}} % Generic monoid element
%:nomenc-exclude:
\newcommand{\monelBa}{\monelB_1} % Generic monoid element
%:nomenc-exclude:
\newcommand{\monelBb}{\monelB_2} % Generic monoid element
%:nomenc-exclude:
\newcommand{\mtimes}{\mthen} % Monoid operation
%:def:def:monoid

\newcommand{\mtimesof}[1]{\mthenof{#1}}

%:section:gen/groups: Groups
\newcommand{\idgrp}{\stylemorph{\idname}} % identity for group
%:def:def:group
\newcommand{\grpA}{\CatStyle{G}} % Generic group names
%:nomenc:\grpA, \grpB: Generic group names
%:def:def:group
\newcommand{\grpB}{\CatStyle{H}}
%:nomenc-exclude:
\newcommand{\grpAset}{\makesymbolset{G}} % Underlying set of $\grpA$.
%:nomenc:\grpAset, \grpBset: Underlying set of  groups.
%:def:def:monoid
\newcommand{\grpBset}{\makesymbolset{H}} % Underlying set of $\grpB$.
%:nomenc-exclude:
\newcommand{\ginv}{\stylefunctors{\aword{inv}}} % Group inverse
%:def:def:group

\newcommand{\grpela}{\stylemorph{x}} % Generic group element
%:nomenc:\grpela, \grpelb, \grpelc: Generic group elements
\newcommand{\grpelb}{\stylemorph{y}}
%:nomenc-exclude:
\newcommand{\grpelc}{\stylemorph{z}}
%:nomenc-exclude:
\newcommand{\grpelA}{\stylemorph{g}} % Generic group element
%:nomenc-exclude:
\newcommand{\grpelAa}{\grpelA_1} % Generic group element
%:nomenc-exclude:
\newcommand{\grpelAb}{\grpelB_2} % Generic group element
%:nomenc-exclude:

\newcommand{\grpelB}{\stylemorph{n}} % Generic group element
%:nomenc-exclude:

\newcommand{\grpelBa}{\grpelB_1} % Generic group element
%:nomenc-exclude:

\newcommand{\grpelBb}{\grpelB_2} % Generic group element
%:nomenc-exclude:

\newcommand{\gtimes}{\mthen} % Group composition operation
%:def:def:group

%:section:gen/categories/monoidal: Monoidal categories
\newcommand{\Moncat}{\CatStyle{MonCat}}

\newcommand{\monpostimes}{\mtimescatob} % Monoidal poset operation
\newcommand{\mtimescatob}{\mathbin{\styleobj{\pmb{\otimes}}}} % Stacking category objects semigroup operation
%:def:def:stacking-semi-cat
\newcommand{\mtimescatmor}{\mathbin{\stylemorph{\pmb{\otimes}}}} % Stacking category morphisms semigroup operation
%:def:def:stacking-semi-cat
\newcommand{\mtimescat}{\mathbin{\stylefunctors{\pmb{\otimes}}}} % Monoidal product
%:nomenc:\mtimescatof{\CatA}:Monoidal operation for category $\CatA$.
%:def:def:monoidal-cat
\newcommand{\mtimescatof}[1]{\mathbin{\stylefunctors{\pmb{\otimes}}_{#1}}}


\newcommand{\idmoncat}{\styleobj{\mathbf{1}}} % Identity object for monoidal operation
%:def:def:monoidal-cat
\newcommand{\mtimesD}{\mathbin{\mtimescatof{\CatD}}}
%:nomenc-exclude:
\newcommand{\mtimesC}{\mathbin{\mtimescatof{\CatC}}}
%:nomenc-exclude:

\newcommand{\sourceperm}[1]{{#1}_{s}}
%:nomenc-exclude:

\newcommand{\targetperm}[1]{{#1}_{t}}
%:nomenc-exclude:

\newcommand{\prcat}{\CatStyle{PR}}
%:nomenc-exclude:


% \newcommand{\leftunitor}{\stylenat{\lambda}} % Left unitor
%\newcommand{\rightunitor}{\stylenat{\rho}} % Right unitor
% \newcommand{\associator}{\stylenat{\alpha}} % Associator

\newcommand{\leftunitor}{\stylenat{\operatorname{lu}}} % Left unitor
%:def:def:monoidal-cat
\newcommand{\rightunitor}{\stylenat{\operatorname{ru}}} % Right unitor
%:def:def:monoidal-cat
\newcommand{\associator}{\stylenat{\operatorname{as}}} % Associator
%:def:def:monoidal-cat

\newcommand{\braiding}{\stylenat{\operatorname{br}}} % Braiding
%:def:def:monoidal-cat

\newcommand{\strongmu}{\stylenat{\mu}} % Isomorphism for strong monoidal functor
%:def:def:strong-monoidal-functor
\newcommand{\strongeps}{\stylemorph{\operatorname{iso}}}  % Isomorphism for strong monoidal functor
%:def:def:strong-monoidal-functor

\newcommand{\ev}{\stylemorph{\operatorname{ev}}} % evaluation map for dualizable objects
\newcommand{\coev}{\stylemorph{\operatorname{coev}}}  % coevaluation map for dualizable objects

%:section:gen/categories/adjunctions: Adjunctions

\newcommand{\funl}{\stylefunctors{L}} % Left adjunct functor
%:def:def:cat-adjunction-v1
\newcommand{\funr}{\stylefunctors{R}} % Right adjunct functor
%:def:def:cat-adjunction-v1
\newcommand{\adjunction}{\dashv} % Adjunction
%:nomenc:\funl \adjunction \funr: $\funl$ and $\funr$ are adjoint functors.
%:def:def:cat-adjunction-v1

\newcommand{\adjtau}{\stylenat{\tau}} % Adjunction isomorphism
%:def:def:adj-iso

\newcommand{\equivunit}{\stylenat{\operatorname{un}}} % Unit
%:def:def:adj-counit

\newcommand{\equivcounit}{\stylenat{\operatorname{co}}} % Co-unit
%:def:def:adj-counit

%:section:gen/categories/traced: Traced monoidal categories

\newcommand{\Tr}{\operatorname{Tr}} % Trace operator
%:def:def:traced-monoidal-category

\newcommand{\Conw}{\aword{Conw}} % Conway operator

\newcommand{\para}{\text{par}}
%:nomenc-exclude:


% :nomenc-exclude:

\newcommand{\qqand}{\qquad\text{and}\qquad}
%:nomenc-exclude:

% \newcommand{\snack}[1]{\mathsf{#1}}

\newcommand{\textF}[1]{\text{\F{#1}}}
\newcommand{\textR}[1]{\text{\R{#1}}}
% \newcommand{\thing}[1]{\text{#1}}

\newcommand{\ubar}[1]{\underaccent{\bar}{#1}}

\newcommand{\unc}{\mathsf{Unc}}
\newcommand{\uncmon}{\mathcal{U}}

%:section:gen/tuples: Tuples


\newcommand{\emptytuple}{\tup{\,}} % zero-size tuple
\newcommand{\emptylist}{\makelist{\,}} % zero-size list
\newcommand{\emptycobj}{\cObj{\,}}
% The one below supposedly would allow splitting tuples over newlines,
% but I could not make it work - AC
%\makeatletter
%\newcommand\tup[1]{%
%  \@tempcnta=0
%  \left\langle
%  \@for\@ii:=#1\do{%
%    \@insertbreakingcomma
%    \@ii
%  }%
%  \right\rangle
%}
%\def\@insertbreakingcomma{%
%  \ifnum \@tempcnta = 0 \else\,,\ \linebreak[1] \fi
%  \advance\@tempcnta\@ne
%}
%\makeatother

%:section:gen/booleans: Booleans

\newcommand{\true}{\top} % True
\newcommand{\false}{\bot} % False
\newcommand{\booland}{\mathbin{\wedge}} % Boolean and
\newcommand{\boolor}{\mathbin{\vee}} % Boolean or

\newcommand{\Imp}{\Rightarrow} % Implies
\newcommand{\Eqv}{\Leftrightarrow} % Equivalence

%:section:gen/categories/arrows: Arrows

\newcommand{\sto}{\mathrel{\textcolor{darkblue}{\rightarrow}}} % Set arrow
\newcommand{\stolong}{\mathrel{\textcolor{darkblue}{\longrightarrow}}} % Set arrow
%:nomenc-exclude:
\newcommand{\mto}{\mathrel{{\stylemorph{\to}}}\linebreak[0]} % Morphism arrow
\newcommand{\mtoin}[1]{\mathrel{{\stylemorph{\to}}_{#1}}\linebreak[0]}
%:nomenc-exclude:

\newcommand{\norto}{\mathrel{\stylemorph{\dasharrow}}}
\newcommand{\mfrom}{\mathrel{{\stylemorph{\leftarrow}}}} % Morphism arrow
\newcommand{\fto}{\mathrel{\stylefunctors{\to}}} % Functors arrow
\newcommand{\ftoin}[1]{\mathrel{{\stylefunctors{\to}}_{#1}}\linebreak[0]}
\newcommand{\ftolong}{\mathrel{\stylefunctors{\longrightarrow}}} % Functors arrow, longer
%:nomenc-exclude:
\newcommand{\nto}{\mathrel{\stylenat{\Rightarrow}}} % Natural transformation arrow
\newcommand{\ntolong}{\mathrel{\stylenat{\Longrightarrow}}} % Natural transformation arrow, longer
%:nomenc-exclude:
\newcommand{\nfromlong}{\mathrel{\stylenat{\Longleftarrow}}} % Inverted natural transformation arrow
%:nomenc-exclude:
%\newcommand{\longmapsto}{\xmapsto{\phantom{mm}}}

\newcommand{\To}[1]{\xrightarrow{#1}}
\newcommand{\ntoiso}{\mathrel{\stylenat{\xrightarrow{\cong}}}}
\newcommand{\mtoiso}{\mathrel{\stylemorph{\xrightarrow{\cong}}}}
%:example: $a \To f b$

%\newcommand{\slashedrightarrow}{\relbar\joinrel\relbar\joinrel\mapstochar\joinrel\rightarrow}
\newcommand{\slashedrightarrow}{\relbar\joinrel\mapsto}
%:nomenc-exclude:
\newcommand{\profto}{\mathrel{\slashedrightarrow}\linebreak[0]} % profunctor arrow
%:def:def:profunctor

\newcommand{\toMonCat}{\mathrel{\to_\Moncat}}
%:nomenc-exclude:

\newcommand{\toDP}{{\mtoin\DP}}
%:nomenc-exclude:

\newcommand{\toDPres}[1]{\overset{#1}{\toDP}}
%:nomenc-exclude:

\newcommand{\toSet}{\mtoin\Set}
%:nomenc-exclude:

\newcommand{\toRel}{\mtoin\Rel}
%:nomenc-exclude:

\newcommand{\toinPos}{\mtoin\Pos}
%:nomenc-exclude:

\newcommand{\toinCat}{\mtoin\Category}
%:nomenc-exclude:

\newcommand{\toiso}{\mathrel{\overset{\sim}{\to}}}
%:nomenc-exclude:

\newcommand{\too}{\longrightarrow}
%:nomenc-exclude:


\newcommand{\embedsin}{\hookrightarrow}
%:nomenc:\CatA \embedsin \CatB:$\CatA$ embeds in $\CatB$.

%:section:gen/dp: DP
%:section:gen/dp/formalization: Formalization
\newcommand{\fun}{\EM{{\colF f}}} % A generic functionality in $\funsp$.
%:def:def:DPI
\newcommand{\res}{\EM{{\colR r}}} % A generic cost in $\ressp$.

\newcommand{\resan}[1]{\res_\R{#1}}
\newcommand{\funan}[1]{\fun_\F{#1}}
%:def:def:DPI
\newcommand{\imp}{\EM{{\colI i}}} % A generic implementation in in $\impsp$.
%:def:def:DPI

\newcommand{\impn}[1]{\EM{{\colI i_{#1}}}}
\newcommand{\funsp}{\EM{{\colF \mathbf{F}}}} % Functionality space
\newcommand{\ulfunsp}{\uppersets \lowersets \funsp}
\newcommand{\ulressp}{\uppersets \lowersets \ressp}
\newcommand{\funspn}[1]{\EM{{\colF \mathbf{F}_{#1}}}} % Functionality space
\newcommand{\funspa}{\EM{{\colF \mathbf{A}}}} % Functionality space
%:nomenc-exclude:
\newcommand{\funspb}{\EM{{\colF \mathbf{B}}}} % Functionality space
%:nomenc-exclude:
\newcommand{\funspc}{\EM{{\colF \mathbf{C}}}} % Functionality space
%:nomenc-exclude:

\newcommand{\neutra}{\EM{\mathbf{A}}}
\newcommand{\neutrb}{\EM{\mathbf{B}}}
\newcommand{\neutrc}{\EM{\mathbf{C}}}
\newcommand{\neutrd}{\EM{\mathbf{D}}}

\newcommand{\ressp}{\EM{{\colR\mathbf{R}}}} % Requirements space
%:def:def:DPI
\newcommand{\resspn}[1]{\EM{{\colR\mathbf{R}_{#1}}}}

\newcommand{\resspa}{\EM{{\colR\mathbf{A}}}} % Requirements space
%:nomenc-exclude:
\newcommand{\resspb}{\EM{{\colR\mathbf{B}}}} % Requirements space
%:nomenc-exclude:
\newcommand{\resspc}{\EM{{\colR\mathbf{C}}}} % Requirements space
%:nomenc-exclude:
\newcommand{\resspd}{\EM{{\colR\mathbf{D}}}} % Requirements space
%:nomenc-exclude:

\newcommand{\impsp}{\EM{{\colI\mathbf{I}}}} % Implementation space
%:def:def:DPI

\newcommand{\impspn}[1]{\EM{{\colI\mathbf{I}_{#1}}}}
\newcommand{\prov}{{\colF\aword{prov}}} % unctionality of an implementation
%:def:def:DPI
%:nomenc:\prov \colon \impsp\sto\funsp: functionality of an implementation

\newcommand{\provn}[1]{{\colF\aword{prov}_{#1}}}
\newcommand{\req}{{\colR\aword{req}}} %
%:nomenc:\req  \colon \impsp\sto\ressp: requirements of an implementation
%:def:def:DPI
%:section:gen/dp/top-bottom: Top and bottom

\newcommand{\reqn}[1]{{\colR\aword{req}_{#1}}}
\newcommand{\restop}{\top_{\ressp}}
%:nomenc-exclude:
\newcommand{\resbot}{\bot_{\ressp}}
%:nomenc-exclude:
\newcommand{\funtop}{\top_{\funsp}}
%:nomenc-exclude:
\newcommand{\funbot}{\bot_{\funsp}}
%:nomenc-exclude:

\newcommand{\funleq}{\posleqof{\funsp}}
%:nomenc-exclude:
\newcommand{\resleq}{\posleqof{\ressp}}
%:nomenc-exclude:
\newcommand{\fungeq}{\posgeqof{\funsp}}
%:nomenc-exclude:
\newcommand{\resgeq}{\posgeqof{\ressp}}
%:nomenc-exclude:

%:section:gen/posets/symbols: Symbols

\newcommand{\poscat}[1]{\mathbf{Ctgr}\,#1}
%:nomenc-exclude:
% \newcommand{\poscat}[1]{#1^{\dagger}}

\newcommand{\posleq}{\mathrel{\stylemorph{\preceq}}}
%:nomenc:\posAleq:Order relation associated to the poset $\posA$
%:nomenc-exclude:
\newcommand{\posleqof}[1]{\mathrel{{\stylemorph{\preceq}}_{#1}}}
%:nomenc-exclude:
\newcommand{\posleqopof}[1]{\mathrel{{\stylemorph{\preceq}}_{#1}\op}}
%:nomenc-exclude:
\newcommand{\posgeq}{\mathrel{\stylemorph{\succeq}}}
%:nomenc-exclude:
\newcommand{\posgeqof}[1]{\mathrel{{\stylemorph{\succeq}}_{#1}}}

\newcommand{\postop}{\top} % Top of a poset
%:nomenc:\top_{\posA}:Top of poset $\posA$
%:def:def:top
\newcommand{\posbot}{\bot} % Bottom of a poset
%:nomenc:\bot_{\posA}:Bottom of poset $\posA$
%:def:def:bot

\newcommand{\poslt}{\mathrel{\stylemorph{\prec}}}
\newcommand{\posltof}[1]{\mathrel{\stylemorph{\prec}_{#1}}}
%:nomenc-exclude:
\newcommand{\ordleq}{\preceq}
%:nomenc-exclude:
\newcommand{\ordgeq}{\succeq}
%:nomenc-exclude:

\newcommand{\resMin}{{\Min_{\resleq}}}
%:nomenc-exclude:
\newcommand{\funMax}{{\Max_{\fungeq}}}
%:nomenc-exclude:

%:section:typing: Typing utils

%:section:typing/abbrevs: Abbreviations

\renewcommand{\etal}{{et\,al.}\xspace}
%:nomenc-exclude:

\renewcommand{\eg}{e.g.\xspace}% avoid!
%:nomenc-exclude:

\renewcommand{\etc}{{etc.}\xspace}% avoid!
%:nomenc-exclude:

\renewcommand{\ie}{\text{i.e.}\xspace}% avoid!
%:nomenc-exclude:

\newcommand{\subto}{\text{s.t.}} % ``subject to'' (used in optimization problems)
%:nomenc-exclude:

\newcommand{\with}{\text{using}} % used in optimization problem
%:nomenc-exclude:

%:section:paper1: Original paper

\newcommand{\cdpiN}{\mathcal{V}}
\newcommand{\cdpiE}{\mathcal{E}}
\newcommand{\cdpin}{v}
\newcommand{\cdpie}{e}
\newcommand{\cdpinA}{v_1}
\newcommand{\cdpinB}{v_2}
\newcommand{\cdpiresind}{i}
\newcommand{\cdpifunind}{j}
\newcommand{\cdpiresindA}{i_1}
\newcommand{\cdpifunindB}{j_2}
\newcommand{\dpinumf}{\textrm{n}_f}
\newcommand{\dpinumr}{\textrm{n}_r}
\newcommand{\cdpinnumf}{{\dpinumf}_{\cdpin}}
\newcommand{\cdpinnumr}{{\dpinumr}_{\cdpin}}

\newcommand{\RR}{{\colR \alpha}} % A specific antichain, used in a proof.
%:nomenc-exclude:

\newcommand{\unconnectedfun}{\mathsf{UF}}
%:nomenc-exclude:
\newcommand{\unconnectedres}{\mathsf{UR}}
%:nomenc-exclude:

%:section:gen/dp/computational:Computational representation
\newcommand{\rtof}{\EM{{\colH k}}}
%:def:def:rtof
\newcommand{\rtoF}{\EM{{\colH K}}}
%:def:def:rtoF
\newcommand{\ftor}{\EM{{\colH h}}}
%:def:def:ftor
\newcommand{\ftoR}{\EM{{\colH H}}}
%:def:def:ftoR

\newcommand{\ftorL}{\EM{{\colL h_L}}}
\newcommand{\ftorU}{\EM{{\colU h_U}}}


%:section:uncertainty: Uncertainty paper

\newcommand{\ufloor}{{\colL\aword{floor}}}
\newcommand{\uceil}{{\colU\aword{ceil}}}

\newcommand{\udpa}{u_a}
\newcommand{\udpb}{u_b}
\newcommand{\udpL}{{\colL \boldsymbol{\mathsf{L}}}}
\newcommand{\udpU}{{\colU \boldsymbol{\mathsf{U}}}}
\newcommand{\udpsp}{\UDP}
\newcommand{\udpleq}{\posleqof{\udpsp}}

\newcommand{\dpsp}{\DP}
\newcommand{\dpleq}{\posleqof\dpsp}
%:nomenc-exclude:

%:section:gen/units: Currencies
\newcommand{\currency}[1]{\text{\fontseries{bf}\colTransmuted #1}\xspace}
\newcommand{\USD}{\currency{USD}}
\newcommand{\usd}{USD\xspace}
\newcommand{\SGD}{\currency{SGD}}
\newcommand{\sgd}{SGD\xspace}
\newcommand{\CHF}{\currency{CHF}}
\newcommand{\CHFneutral}{CHF}
\newcommand{\chf}{CHF\xspace}
\newcommand{\EUR}{\currency{EUR}}
\newcommand{\eur}{EUR\xspace}

\def\bitcoinA{%
    \leavevmode
    \vtop{\offinterlineskip %\bfseries
    \setbox0=\hbox{B}%
    \setbox2=\hbox to\wd0{\hfil\hskip-.03em
    \vrule height .3ex width .15ex\hskip .08em
    \vrule height .3ex width .15ex\hfil}
        \vbox{\copy2\box0}\box2}}
\newcommand{\satoshiA}{\text{satoshi} \xspace}
\newcommand{\stdcurr}{\bitcoinA{}\xspace} % Generic currency
\newcommand{\standardcurrency}{CHF}

%:section:gen/dp/symbols: DP

% \newcommand{\adp}{\stylefunctors{\mathbf{d}}} % generic design problem as a profunctor
%\newcommand{\bdp}{\stylefunctors{\mathbf{e}}} % generic design problem as a profunctor
%\newcommand{\adp}{\stylefunctors{\mathbf{d}}} % generic design problem as a profunctor
\newcommand{\adp}{\adpa} % generic design problem as a profunctor
%:nomenc-exclude:
\newcommand{\adpa}{\stylefunctors{\mathbf{d}}} % generic design problem as a profunctor
%:nomenc:\adpa,\adpb,\adpc:Generic design problems as profunctors
\newcommand{\adpb}{\stylefunctors{\mathbf{e}}} % generic design problem as a profunctor
%:nomenc-exclude:
\newcommand{\adpc}{\stylefunctors{\mathbf{g}}} %generic design problem as a profunctor
%:nomenc-exclude:
\newcommand{\adpd}{\stylefunctors{\mathbf{h}}} % generic design problem as a profunctor
%:nomenc-exclude:
\newcommand{\adpab}{(\adpa\dpthen\adpb)} %
%:nomenc-exclude:

\newcommand{\adpan}[1]{\stylefunctors{\mathbf{d}_{#1}}}

\newcommand{\dpseries}{\aword{series}}
\newcommand{\dppar}{\aword{par}}
\newcommand{\dploop}{\aword{loop}}

\newcommand{\dploopb}{\aword{loopb}}
\newcommand{\terms}{\aword{Terms}}
%
\newcommand{\udpsem}{\Phi}
%
\newcommand{\dpsem}{\varphi}
%
\newcommand{\atoms}{\mathcal{A}}
\newcommand{\atree}{\boldsymbol{\aword{T}}}
\newcommand{\val}{\boldsymbol{v}}
%
\newcommand{\ops}{\aword{ops}}
%
%
% \newcommand{\acprod}{\mathbin{\boldsymbol{\times}}}
%
\newcommand{\oploop}{\dagger}
\newcommand{\opseries}{\mathbin{\varocircle}}
\newcommand{\oppar}{\mathbin{\varotimes}}
\newcommand{\opcoprod}{\mathbin{\varovee}}
%
\newcommand{\UId}{\catid^{\UDP}}
\newcommand{\vdc}{\aword{vdc}} % Van Der Corput sequence
%:def:sec:van-der-corput

%:section:gen/posets/attributes: Attributes
\newcommand{\posetwidth}{\aword{width}}
%:nomenc:\posetwidth(\posA): Width of the poset $\posA$.
%:def:def:poset-width

\newcommand{\posetheight}{\aword{height}}
%:nomenc:\posetheight(\posA): Height of the poset $\posA$.
%:def:def:poset-height

%:section:gen/posets/domain: Domain theory

\newcommand{\lfp}{\textsf{lfp}} % Least fixed point

\newcommand{\CPO}{\textsf{CPO}\xspace} % Complete partial order
%:def:def:cpo

\newcommand{\DCPO}{\textsf{DCPO}\xspace} % Directed-complete partial order
%:def:def:cpo

%:section:gen/dp/queries: Queries in $DP$

\newcommand{\Feasibility}{\aword{Feasibility}}
%:def:prob:Feasibility
\newcommand{\FeasibleImp}{\aword{FeasibleImp}}
%:def:prob:FeasibleImp
\newcommand{\FixFunMinRes}{\EM{\stylefunctors{\aword{Fix\-Fun\-Min\-Res}}}}
%:def:prob:FixFunMinRes
\newcommand{\FixResMaxFun}{\EM{\stylefunctors{\aword{Fix\-Res\-Max\-Fun}}}}
%:def:prob:FixResMaxFun
\newcommand{\FixFunMinResBack}{\stylefunctors{\aword{Fix\-Fun\-Min\-Res\-Back}}}
%:def:prob:FixFunMinRes
%:nomenc-exclude:
\newcommand{\FixResMaxFunBack}{\stylefunctors{\aword{Fix\-Res\-Max\-Fun\-Back}}}
%:def:prob:FixResMaxFun
%:nomenc-exclude:
%:section:gen/categories/named: Well-known categories

\newcommand{\DP}{\CatLinkConcept{DP}{category of design problems}} % Category of design problems
%:def:def:DP

\newcommand{\DPfinite}{{\EM{\Pos_{\fuppersets}}}}
\newcommand{\feas}{\CatStyle{Feas}} % Synonym of $\DP$
%:def:def:DP

\newcommand{\UDP}{\CatLinkConcept{UDP}{category of uncertain DPs}}
%:def:def:UDP
\newcommand{\DPI}{\CatLinkConcept{DPI}{semicategory of design problems with implementation}} % Semi-category $\DPI$
%:def:def:DPI

\newcommand{\Bool}{\CatLinkConcept{Bool}{posets of booleans}} % Posets/category of Booleans

\newcommand{\CatMon}{\CatLinkConcept{Mon}{category of monoids and morphisms}} % Category of Monoids

\newcommand{\Category}{\CatLinkConcept{Cat}{category of small categories}} % Category of small categories
%:def:def:Category
\newcommand{\Vect}{\CatLinkConcept{Vect}{category of vector spaces}} % Category of vector spaces
%:def:ex:Vect
\newcommand{\FinVect}{\CatLinkConcept{FinVect}{category of finite-dimensional vector spaces}} % Category of finite-dimensional vector spaces
%:def:sub:trace-linear
\newcommand{\Rel}{\CatLinkConcept{Rel}{category of sets and relations}} % Category of sets and relations
%:def:def:Rel


\newcommand{\inrel}[3]{#1\mkern 2mu #2 \mkern 2mu #3} % To be used xRy at the moment
%:def:def:inrel
\newcommand{\FinSet}{\CatLinkConcept{FinSet}{category of finite sets and functions}} % Category of finite sets and functions
%:def:ex:FinSet
\newcommand{\Set}{\CatLinkConcept{Set}{category of sets and functions}} % Category of sets and functions
%:def:def:Set
\newcommand{\Injset}{\CatLinkConcept{InjSet}{category of sets and injective functions}} % Category of sets and injective functions
%:def:ex:Injset



\newcommand{\Prof}{\mathbb{P}\CatStyle{rof}} % Category of profunctors
\newcommand{\Pos}{\CatLinkConcept{Pos}{category of posets and monotone maps}} % Category of posets and monotone maps
%:def:def:Pos
\newcommand{\Lat}{\CatLinkConcept{Lat}{category of lattices and lattices homomorphisms}} % Category of lattices and lattice homomorphisms
%:def:def:Lat
\newcommand{\BoundedLat}{\CatLinkConcept{BoundedLat}{category of bounded lattices and lattices homomorphisms}} % Category of lattices and lattice homomorphisms
%:def:def:BoundedLat
\newcommand{\Grph}{\CatLinkConcept{Grph}{category of graphs and graphs homomorphisms}} % Category of directed graphs
%:def:def:Grph


\newcommand{\Effects}{\CatLinkConcept{Eff}{category of sets and functions with side effects}} % The effects category
%:def:def:Effects




\newcommand{\UPos}{{\EM{\Pos_{\uppersets}}}}
\newcommand{\fUPos}{{\EM{\Pos_{\fuppersets}}}}
\newcommand{\LPos}{{\EM{\Pos_{\lowersets}}}} %
\newcommand{\ulposmap}[1]{{#1}^\star}
\newcommand{\Graph}{\CatStyle{Grph}} % Category of graphs
%:def:def:Graph

\newcommand{\LTI}{\CatLinkConcept{LTI}{category of finite-dimensional linear time-invariant systems}}
%:def:def:LTICat
\newcommand{\Beh}{\CatStyle{Beh}}
%:def:def:Beh
\newcommand{\LBeh}{\CatStyle{LBeh}}
%:def:def:linear-io-behavior

\newcommand{\RelL}{\cCatLinkConcept{Rel}{category of tuple-sets and relations}} % category of tuple-sets and relations
%:def:def:RelL
\newcommand{\SetL}{\cCatLinkConcept{Set}{category of tuple-sets and functions}} % category of tuple-sets and functions
%:def:def:SetL
\newcommand{\PosL}{\cCatLinkConcept{Pos}{category of tuple-posets and monotone maps}} % category of tuple-posets and monotone maps
%:def:def:PosL
\newcommand{\DPL}{\cCatLinkConcept{DP}{category of tuple-posets and design problems}} % Category of lists of DPs
%:def:def:DPL

\newcommand{\genericlti}[1]{\tupp{\prstart_{#1},\mat{A}_{#1},\mat{B}_{#1},\mat{C}_{#1},\mat{D}_{#1}}}
%:nomenc-exclude:

\newcommand{\genericplti}[1]{\tupp{\prstart_{#1},\mat{A}_{#1},\mat{B}_{#1},\mat{C}_{#1}}}
%:nomenc-exclude:


%:section:vol1/10_arrows/80_graphs:

\newcommand{\graph}{\mathcal{G}}
\newcommand{\evaluation}{\mathsf{evaluation}}
\newcommand{\deploy}[1]{\mathsf{deploy}_{#1}}
\newcommand{\resApp}{\Psi}
\newcommand{\resGraph}{\mathcal{G}_\mathrm{R}}
\newcommand{\resGraphSet}{\mathbf{G}_\mathrm{R}}
\newcommand{\compGraph}{\mathcal{G}_\mathrm{C}}
\newcommand{\compGraphSet}{\mathbf{G}_\mathrm{C}}
\newcommand{\benchmarking}{\mathsf{benchmarking}}
\newcommand{\source}{\stylemaps{\operatorname{src}}}
\newcommand{\sourcen}[1]{\stylemaps{\operatorname{src}_{#1}}}
\newcommand{\target}{\stylemaps{\operatorname{tgt}}}
\newcommand{\targetn}[1]{\stylemaps{\operatorname{tgt}_{#1}}}
\newcommand{\vertices}{\makesymbolset{V}}
\newcommand{\verticesn}[1]{\vertices_{\color{formulasetcolor}#1}}
\newcommand{\resVertices}{\vertices_\mathrm{R}}
\newcommand{\compVertices}{\vertices_\mathrm{C}}
\newcommand{\vertexa}{\styleelements{v}}
\newcommand{\vertexan}[1]{\styleelements{v_{#1}}}
\newcommand{\vertexb}{\styleelements{w}}
\newcommand{\vertexbn}[1]{\styleelements{w_{#1}}}
\newcommand{\vertexc}{\styleelements{x}}
\newcommand{\arcs}{\makesymbolset{E}}
\newcommand{\arcsn}[1]{\arcs_{\color{formulasetcolor}#1}}
\newcommand{\resArcs}{\arcs_\mathrm{R}}
\newcommand{\compArcs}{\arcs_\mathrm{C}}
\newcommand{\arc}{\styleelements{a}}
\newcommand{\arca}{\styleelements{a}}
\newcommand{\arcan}[1]{\styleelements{a_{#1}}}
\newcommand{\arcb}{\styleelements{b}}
\newcommand{\arcbn}[1]{\styleelements{b_{#1}}}
\newcommand{\arcc}{\styleelements{c}}
\newcommand{\arccn}[1]{\styleelements{c_{#1}}}
\newcommand{\colorArcsComp}{\stylemaps{t}}
\newcommand{\colorVerticesComp}{\stylemaps{s}}
\newcommand{\colorArcsRes}{\stylemaps{t}}
\newcommand{\colorVerticesRes}{\stylemaps{s}}
\newcommand{\resType}{\styleelements{t}}
\newcommand{\resTypes}{\stylesets{T}}
\newcommand{\resCap}{\styleelements{c}}
\newcommand{\resCapacity}{\stylesets{C}}
\newcommand{\resLatency}{\stylesets{L}}
\newcommand{\resLat}{\styleelements{l}}
\newcommand{\resBandwidth}{\stylesets{B}}
\newcommand{\resBand}{\styleelements{b}}
\newcommand{\nodeCmd}{\styleelements{\text{cmd}}}
\newcommand{\nodeObs}{\styleelements{\text{obs}}}
\newcommand{\compOps}{\stylemaps{\text{ops}}}
\newcommand{\compSize}{\stylesets{S}}
\newcommand{\inp}{\styleelements{\text{input}}}
\newcommand{\outp}{\styleelements{\text{output}}}
\newcommand{\applications}{\stylesets{\text{Apps}}}
\newcommand{\application}{\stylesets{\text{app}}}
\newcommand{\applicationfeas}{\stylesets{\text{app}_{\circ}}}
\newcommand{\resfeas}{\res_{\circ}}

%:section:vol1/22_operations/40_sameness: \chrefplus{chap:sameness}

\newcommand{\Si}{\text{S\`i}}
%:nomenc-exclude:

\newcommand{\strain}{\varepsilon}
\newcommand{\force}{F}
\newcommand{\deformation}{\Delta x}
\newcommand{\springconst}{k}
\newcommand{\youngmod}{E}

%:section:vol1/10_arrows/30_transmutation:

\newcommand{\Curr}{\CatStyle{Curr}} % Currency category
%:def:def:Curr

\newcommand{\Temp}{\CatStyle{Temp}} % Temperature category
%:def:ex:temperatures

\newcommand{\transmuter}[1]{\text{\fontseries{bf}\colTransmuter #1}\xspace}
%:nomenc-exclude:
\newcommand{\transmuted}[1]{\text{\fontseries{bf}\colTransmuted #1}\xspace}
%:nomenc-exclude:
\newcommand{\technology}[1]{\mathsf{#1}}
%:nomenc-exclude:
\newcommand{\motor}{\transmuter{motor}}
%:nomenc-exclude:
\newcommand{\move}{\transmuter{move}}
%:nomenc-exclude:
\newcommand{\dynamo}{\transmuter{dynamo}}
%:nomenc-exclude:
\newcommand{\wheels}{\transmuter{wheels}}
%:nomenc-exclude:
\newcommand{\electricpower}{\transmuted{electricity}}
%:nomenc-exclude:
\newcommand{\rotationalmotion}{\transmuted{rotation}}
%:nomenc-exclude:
\newcommand{\translationalmotion}{\transmuted{translation}}
%:nomenc-exclude:


\newcommand{\Berg}{\CatStyle{Berg}}
%:nomenc:\Berg:The category of Swiss mountains
%:def:def:Berg
\newcommand{\Bergama}{\CatStyle{BergAma}}
%:def:sec:subcat_berg
\newcommand{\Berglazy}{\CatStyle{BergLazy}}
%:def:sec:subcat_berg
\newcommand{\subcat}[1]{\CatStyle{SubCat}(#1)}

\newcommand{\Intermodal}{\CatStyle{Intermodal}}
\newcommand{\Car}{\CatStyle{Car}}
\newcommand{\Flight}{\CatStyle{Flight}}
\newcommand{\Board}{\CatStyle{Board}}

\newcommand{\LIN}{\transmuted{LIN}}
%:nomenc-exclude:
\newcommand{\FCO}{\transmuted{FCO}}
\newcommand{\FCOf}{\transmuted{FCO}_\transmuted{f }}
%:nomenc-exclude:
\newcommand{\MPXf}{\transmuted{MPX}_\transmuted{f }}
%:nomenc-exclude:
\newcommand{\ZRH}{\transmuted{ZRH}}
\newcommand{\ZRHf}{\transmuted{ZRH}_\transmuted{f }}
%:nomenc-exclude:

\newcommand{\FCOc}{\transmuted{FCO}_\transmuted{c }}
%:nomenc-exclude:
\newcommand{\MPXc}{\transmuted{MPX}_\transmuted{c }}
%:nomenc-exclude:
\newcommand{\ZRHc}{\transmuted{ZRH}_\transmuted{c }}
%:nomenc-exclude:

\newcommand{\FCOoff}{\transmuter{FCO} \ \transmuter{offboard }}
%:nomenc-exclude:
\newcommand{\MPXoff}{\transmuter{MPX} \ \transmuter{offboard }}
%:nomenc-exclude:
\newcommand{\ZRHoff}{\transmuter{ZRH} \ \transmuter{offboard }}
%:nomenc-exclude:
\newcommand{\FCOon}{\transmuter{FCO} \ \transmuter{onboard }}
%:nomenc-exclude:
\newcommand{\MPXon}{\transmuter{MPX} \ \transmuter{onboard }}
%:nomenc-exclude:
\newcommand{\ZRHon}{\transmuter{ZRH} \ \transmuter{onboard }}
%:nomenc-exclude:

\newcommand{\mplane}{\mbox{\smaller[2]\raisebox{-1pt}{\Plane}}}
\newcommand{\mplanerot}{\rotatebox{180}{\mbox{\smaller[2]\raisebox{-8pt}{\Plane}}}}
%:nomenc-exclude:
\definecolor{alitalia}{named}{darkgreen}
\definecolor{swiss}{named}{darkblue}
\definecolor{ewings}{named}{darkred}
\newcommand{\alitaliaN}{{\color{alitalia}\mplanerot}\ \transmuter{Ita}\ \transmuter{N }}
\newcommand{\alitaliaNrot}{{\color{alitalia}\mplane}\ \transmuter{Ita}\ \transmuter{N }}
%:nomenc-exclude:
\newcommand{\alitaliaS}{{\color{alitalia}\mplane}\ \transmuter{Ita}\ \transmuter{S }}
\newcommand{\alitaliaSrot}{{\color{alitalia}\mplanerot}\ \transmuter{Ita}\ \transmuter{S }}
%:nomenc-exclude:
\newcommand{\swissN}{{\color{swiss}\mplane}\ \transmuter{Swiss}\ \transmuter{N }}
%:nomenc-exclude:
\newcommand{\swissS}{{\color{swiss}\mplanerot}\ \transmuter{Swiss}\ \transmuter{S }}
%:nomenc-exclude:
\newcommand{\ewingsN}{{\color{ewings}\mplanerot}\ \transmuter{Ewings}\ \transmuter{N }}
\newcommand{\ewingsNrot}{{\color{ewings}\mplane}\ \transmuter{Ewings}\ \transmuter{N }}
%:nomenc-exclude:
\newcommand{\ewingsS}{{\color{ewings}\mplanerot}\ \transmuter{Ewings}\ \transmuter{S }}
%:nomenc-exclude:
\newcommand{\thefourohfive}{\includegraphics[width=4mm]{int80}} % change number to 405
%:nomenc-exclude:

%:section:vol1/10_arrows/70_mapping:
\newcommand{\Company}{\transmuter{Company}}
%:nomenc-exclude:
\newcommand{\SanyoDenki}{\text{Sanyo Denki}}
%:nomenc-exclude:
\newcommand{\Soyo}{\text{Soyo}}
%:nomenc-exclude:
\newcommand{\Price}{\transmuter{Price}}
%:nomenc-exclude:
\newcommand{\Volume}{\transmuter{Volume}}
%:nomenc-exclude:
\newcommand{\Size}{\transmuter{Size}}
%:nomenc-exclude:
\newcommand{\Multiply}{\transmuter{Multiply}}
%:nomenc-exclude:
\newcommand{\Database}{\CatStyle{Database}}

%:section:vol1/22_operations/20_combinations: \chrefplus{chap:combination}
\newcommand{\globe}{\raisebox{-1pt}{\includegraphics[height=8pt]{globe}}}
\newcommand{\sbanana}{\raisebox{-2pt}{\includegraphics[height=8pt]{banana}}}
%:nomenc-exclude:
\newcommand{\sapple}{\raisebox{-2pt}{\includegraphics[height=10pt]{red-apple}}}
%:nomenc-exclude:
\newcommand{\sapplebw}{\raisebox{-2pt}{\includegraphics[height=10pt]{red-apple_bw}}}
%:nomenc-exclude:
\newcommand{\scarrot}{\raisebox{-2pt}{\includegraphics[height=10pt]{carrot}}}
%:nomenc-exclude:
\newcommand{\scarrotbw}{\raisebox{-2pt}{\includegraphics[height=10pt]{carrot_bw}}}
%:nomenc-exclude:
\newcommand{\sgrapes}{\raisebox{-2pt}{\includegraphics[height=10pt]{grapes}}}
%:nomenc-exclude:
\newcommand{\sgrapesbw}{\raisebox{-2pt}{\includegraphics[height=10pt]{grapes_bw}}}
%:nomenc-exclude:
\newcommand{\sbretzel}{\raisebox{-2pt}{\includegraphics[height=10pt]{bretzel}}}
%:nomenc-exclude:
\newcommand{\sbretzelbw}{\raisebox{-2pt}{\includegraphics[height=10pt]{bretzel_bw}}}
%:nomenc-exclude:
\newcommand{\sfondue}{\raisebox{-2pt}{\includegraphics[height=10pt]{fondue}}}
%:nomenc-exclude:
\newcommand{\sfonduebw}{\raisebox{-2pt}{\includegraphics[height=10pt]{fondue_bw}}}
%:nomenc-exclude:
\newcommand{\schoco}{\raisebox{-2pt}{\includegraphics[height=10pt]{chocolate}}}
%:nomenc-exclude:
\newcommand{\scheese}{\raisebox{-2pt}{\includegraphics[height=10pt]{cheese}}}
%:nomenc-exclude:
\newcommand{\scheesebw}{\raisebox{-2pt}{\includegraphics[height=10pt]{cheese_bw}}}
%:nomenc-exclude:
\newcommand{\sburger}{\raisebox{-2pt}{\includegraphics[height=10pt]{hamburger}}}
%:nomenc-exclude:
\newcommand{\sburgerbw}{\raisebox{-2pt}{\includegraphics[height=10pt]{hamburger_bw}}}
%:nomenc-exclude:
\newcommand{\Snacks}{\stylesets{\aword{Snacks}}}
%:nomenc-exclude:
\newcommand{\Drinks}{\stylesets{\aword{Drinks}}}
%:nomenc-exclude:
\newcommand{\Participants}{\stylesets{\aword{People}}}
%:nomenc-exclude:
\newcommand{\swater}{\raisebox{-2pt}{\includegraphics[height=10pt]{water}}} % need to find better icon
%:nomenc-exclude:
\newcommand{\stea}{\raisebox{-2pt}{\includegraphics[height=10pt]{beer}}}
%:nomenc-exclude:
\newcommand{\sbubtea}{\raisebox{-2pt}{\includegraphics[height=10pt]{bubble-tea}}}
%:nomenc-exclude:
\newcommand{\smilk}{\raisebox{-2pt}{\includegraphics[height=10pt]{milk}}}
%:nomenc-exclude:
\newcommand{\swine}{\raisebox{-2pt}{\includegraphics[height=10pt]{wine}}}
%:nomenc-exclude:
\newcommand{\sdrink}{\raisebox{-2pt}{\includegraphics[height=10pt]{drink}}}
%:nomenc-exclude:
\newcommand{\smate}{\raisebox{-2pt}{\includegraphics[height=10pt]{mate}}}
%:nomenc-exclude:
\newcommand{\spritz}{\raisebox{-2pt}{\includegraphics[height=10pt]{spritz}}}
%:nomenc-exclude:
\newcommand{\scoffee}{\raisebox{-2pt}{\includegraphics[height=10pt]{coffee}}}
%:nomenc-exclude:
\newcommand{\eats}{\stylemaps{\operatorname{eats}}}
%:nomenc-exclude:
\newcommand{\drinks}{\stylemaps{\operatorname{drinks}}}
%:nomenc-exclude:
\newcommand{\meal}{\stylemaps{\operatorname{meal}}}
%:nomenc-exclude:

%:section:vol1/25_translation/50_functors: \chrefplus{chap:functors}

\newcommand{\Plans}{\CatStyle{Plans}}
%:def:ex:planning-as-search-functor

%:section:miscellanea: To categorize

\newcommand{\constit}{\underline{Constituents}}
%:nomenc-exclude:
\newcommand{\condit}{\underline{Conditions}}
%:nomenc-exclude:

\newcommand{\MS}[1]{{\color{blue}#1}}
%:nomenc-exclude:

\newcommand{\One}{\mathbf{1}}

% XXX: not sure if category or set

\newcommand{\mat}[1]{\mathbf{#1}} % Style for matrices
%:nomenc-exclude:
\newcommand{\posdefmatset}[1]{\aword{PDM}(#1)}
\newcommand{\vect}[1]{\mathbf{#1}} % Style for vectors
%:nomenc-exclude:

\newcommand{\iindex}[1]{\index{#1}#1\xspace}
%:nomenc-exclude:
\newcommand{\aword}[1]{\EM{\mathrm{#1}}}
%:nomenc-exclude:

%\newcommand{\Lo}[1]{\mathsf{L}#1}

%\newcommand{\myvee}{\textstyle{\bigvee}}
%\newcommand{\mycup}{\textstyle{\bigcup}}
%\newcommand{\M}{\mathsf{M}}
%\newcommand{\T}{\mathsf{T}}

\newcommand{\definedas}{\mathrel{\coloneqq}} % ``defined as''

\newcommand{\Lax}{\operatorname{Lax}}

\DeclareMathOperator{\MOb}{\lvert\mspace{2mu}\cdot\mspace{2mu}\rvert}

\newcommand{\triv}{\mathsf{Triv}}

\newcommand{\dvert}{\operatorname{Vert}}

\newcommand{\marker}{{\bullet}}
%:nomenc-exclude:

\newcommand{\objmarker}{{\styleobj{\bullet}}}
%:nomenc-exclude:

\newcommand{\exname}[1]{\texttt{\footnotesize #1}} % Name of exercise
%:nomenc-exclude:

\newcommand{\dd}{\textrm{d}}
%:section:deprecated:Deprecated

\newcommand{\resel}{\stylesets{r}}
\newcommand{\setel}{\styleelements{x}}
%:section:vol1/05_pluribus/misc:

% TO PRODUCE ORIGINALS
\newcommand{\cancericon}{\includegraphics[height=2.5mm]{cancer}}
%:nomenc-exclude:
\newcommand{\aquariusicon}{\includegraphics[height=2.5mm]{aquarius}}
%:nomenc-exclude:
\newcommand{\ariesicon}{\includegraphics[height=2.5mm]{aries}}
%:nomenc-exclude:

\newcommand{\alphabeta}{ {\color{blue}\bullet} }
\newcommand{\alphabetb}{ {\color{red}\bullet} }
\newcommand{\alphabetasymba}{\cancericon}
\newcommand{\alphabetasymbb}{\aquariusicon}
\newcommand{\alphabetasymbc}{\ariesicon}
\newcommand{\sprout}{\text{sprout}}
\newcommand{\yng}{\text{young}}
\newcommand{\dead}{\text{dead}}
\newcommand{\mature}{\text{mature}}
\newcommand{\old}{\text{old}}
\newcommand{\alive}{\text{alive}}

%:section:vol1/05_pluribus/50_morphisms:\chrefplus{chap:morphisms}
%:section:vol1/05_pluribus/50_morphisms/morse:Morse code
\newcommand{\morsedot}{\EM{\bullet}} % Morse dot
\newcommand{\morsedash}{\EM{\pmb{-}}} % Morse dash
\newcommand{\morsedsp}{\EM{\,s_1\,}} % Silence between dots and dashes
\newcommand{\morselsp}{\EM{\,s_3\,}} % Silence between letters
\newcommand{\morsewsp}{\EM{\,s_7\,}} % Silence between words

\newcommand{\Morsedot}{\EM{{\color{black}\rule{\mb}{\ml}}}} % Beep of $\ell$
\newcommand{\Morsedash}{\EM{{\color{black}\rule{3\mb}{\ml}}}}% Beep of $3\ell$
\newcommand{\Morsedsp}{\EM{{\color{gray}\rule{\mb}{\ml}}}}% Silence of $\ell$
\newcommand{\Morselsp}{\EM{{\color{gray}\rule{3\mb}{\ml}}}}% Silence of $3\ell$
\newcommand{\Morsewsp}{\EM{{\color{gray}\rule{7\mb}{\ml}}}}% Silence of $7\ell$

%:section:vol1/05_pluribus/50_morphisms/morse/internal: internal
%:nomenc-exclude:
\newcommand{\mst}[1]{\textbf{#1}} % Formatting for morse code table
\newcommand{\lettersp}{\raisebox{-1pt}{\,{\color{red}\rule{3pt}{8pt}}\,}}
\newcommand{\wordsp}{\raisebox{-1pt}{\,{\color{green}\rule{3pt}{8pt}}\,}}

\newcommand{\morseA}{\morsedot\morsedsp  \morsedash}
\newcommand{\morseE}{\morsedot}
\newcommand{\morseI}{\morsedot\morsedsp  \morsedot}
\newcommand{\morseL}{\morsedash}
\newcommand{\morseM}{\morsedash\morsedsp \morsedash}
\newcommand{\morseX}{\morsedash\morsedsp \morsedot\morsedsp \morsedot\morsedsp \morsedash}
\newcommand{\morseW}{\morsedot\morsedsp \morsedash\morsedsp \morsedash}
\newlength{\ml}
\setlength{\ml}{7pt}
\newlength{\mb}
\setlength{\mb}{4pt}

\newcommand{\worda}{\makelist{\alphabeta, \alphabeta, \alphabetb, \alphabeta, \alphabetb, \alphabetb, \alphabetb, \alphabeta} }
\newcommand{\wordb}{ \makelist{\alphabetb, \alphabetb, \alphabeta, \alphabetb}}

\newcommand{\MorseA}{\Morsedot\Morsedsp  \Morsedash}
\newcommand{\MorseE}{\Morsedot}
\newcommand{\MorseI}{\Morsedot\Morsedsp  \Morsedot}
\newcommand{\MorseL}{\Morsedash}
\newcommand{\MorseM}{\Morsedash\Morsedsp \Morsedash}
\newcommand{\MorseX}{\Morsedash\Morsedsp \Morsedot\Morsedsp \Morsedot\Morsedsp \Morsedash}
\newcommand{\MorseW}{\Morsedot\Morsedsp \Morsedash\Morsedsp \Morsedash}

%:section:vol1/05_pluribus/50_morphisms/ASCII:ASCII example
\newcommand{\alphanums}{\stylesets{\texttt{char}}}
\newcommand{\morsesymbols}{\stylesets{\texttt{mchar}}}
\newcommand{\morseesymbols}{\stylesets{\texttt{emchar}}} % includes spaces
\newcommand{\asciienc}{\stylefunctors{\asciifunc}}
\newcommand{\asciifunc}{\operatorname{ASCII}}
\newcommand{\dual}{\stylefunctors{\text{Dual}}}
\newcommand{\nnot}{\stylefunctors{\text{Not}}}
\newcommand{\workshop}[1]{\text{Workshop}(#1)}
\newcommand{\closwor}{c_\mathrm{W}}

\newcommand{\morseset}{\stylesets{M}}
\newcommand{\morsemorph}{\stylefunctors{\operatorname{morse}}}

%:section:text: Frequently mispelled words
%:nomenc-exclude:


\newcommand{\morsemap}{\operatorname{morse}}

%:section:python:Python code
%:nomenc-exclude:
\newcommand{\funcnamestyle}[1]{{\smaller[1]\ttfamily\color{funcname}#1}} % Style for function names
%:example: \text{\funcnamestyle{function}}
\newcommand{\classnamestyle}[1]{{\smaller[1]\bfseries\ttfamily\color{classname}#1}} % Style
%:example: \text{\classnamestyle{Class}}
\newcommand{\fieldnamestyle}[1]{{\smaller[1]\bfseries\ttfamily\color{fieldname}#1}} % Style for class names
%:example: \text{\fieldnamestyle{field}}

% \newcommand{\funcname}[1]{{\color{funcname}\mintinline{python}{#1()}}\xspace}



\newcommand{\TypeVar}[1]{\text{\classnamestyle{#1}}\xspace}
\newcommand{\Setoid}{\text{\classnamestyle{Setoid}}\xspace}
\newcommand{\SetDisjointUnion}{\text{\classnamestyle{SetDisjointUnion}}\xspace}
\newcommand{\EnumerableSet}{\text{\classnamestyle{EnumerableSet}}\xspace}
\newcommand{\FiniteSet}{\text{\classnamestyle{FiniteSet}}\xspace}
\newcommand{\Semigroup}{\text{\classnamestyle{Semigroup}}\xspace}
\newcommand{\Group}{\text{\classnamestyle{Group}}\xspace}
\newcommand{\FiniteGroup}{\text{\classnamestyle{FiniteGroup}}\xspace}
\newcommand{\Monoid}{\text{\classnamestyle{Monoid}}\xspace}
\newcommand{\Relation}{\text{\classnamestyle{Relation}}\xspace}
\newcommand{\FiniteMonoid}{\text{\classnamestyle{FiniteMonoid}}\xspace}
% \newcommand{\FiniteMapping}{\text{\classnamestyle{FiniteMapping}}\xspace}
\newcommand{\FiniteSemigroup}{\text{\classnamestyle{FiniteSemigroup}}\xspace}
\newcommand{\FreeSemigroup}{\text{\classnamestyle{FreeSemigroup}}\xspace}
% \newcommand{\Element}{\text{\classnamestyle{Element}}\xspace}
\newcommand{\Morphism}{\text{\classnamestyle{Morphism}}\xspace}
\newcommand{\ConcreteRepr}{\text{\classnamestyle{ConcreteRepr}}\xspace}
\newcommand{\FiniteMap}{\text{\classnamestyle{FiniteMap}}\xspace}
\newcommand{\Object}{\text{\classnamestyle{Object}}\xspace}
\newcommand{\FiniteSetProperties}{\text{\classnamestyle{FiniteSetProperties}}\xspace}
\newcommand{\Iterator}{\text{\classnamestyle{Iterator}}\xspace}
\newcommand{\FiniteSetRepresentation}{\text{\classnamestyle{FiniteSetRepresentation}}\xspace}
\newcommand{\MakeSetUnion}{\text{\classnamestyle{MakeSetUnion}}\xspace}
\newcommand{\SetUnion}{\text{\classnamestyle{SetUnion}}\xspace}
\newcommand{\FiniteSetUnion}{\text{\classnamestyle{FiniteSetUnion}}\xspace}
\newcommand{\SetProduct}{\text{\classnamestyle{SetProduct}}\xspace}
\newcommand{\FiniteMakeSetProduct}{\text{\classnamestyle{FiniteMakeSetProduct}}\xspace}
\newcommand{\EnumerableSetUnion}{\text{\classnamestyle{EnumerableSetUnion}}\xspace}

\newcommand{\Mapping}{\text{\classnamestyle{Mapping}}\xspace}

\newcommand{\ABC}{\text{\classnamestyle{ABC}}\xspace}
%\newcommand{\classname}[1]{{\bfseries\color{classname}\mintinline{python}{#1}}\xspace}
%
% \newcommand{\Felements}{\fieldname{elements}}
% %:example: The \Felements
% \newcommand{\Funion}{\fieldname{union}}
% %:example: The \Funion
% \newcommand{\Fproduct}{\fieldname{product}}
% %:example: The \Fproduct
\newcommand{\hardexercise}{\texorpdfstring{$\star$}{*}}
%:example: Dummy  \hardexercise

\newcommand{\flength}{\stylefunctors{\operatorname{length}}}
% \newcommand{\listsof}[1]{{#1}^{\color{blue}\ast}}
\newcommand{\listsof}[1]{\listfun\,#1}

\newcommand{\streamsof}[1]{\streamfun\,#1}
% \newcommand{\streamsof}[1]{{#1}^{\natnumbers}}
\newcommand{\listfun}{\stylefunctors{\mathsf{List}}}
\newcommand{\streamfun}{\stylefunctors{\mathsf{Stream}}}

% \newcommand{\streamsof}[1]{{#1}^{\color{purple}\star}}

%:section:vol1/10_arrows/75_processes:\chrefplus{chap:processes}
\newcommand{\prgen}{\stylesets{A}}
\newcommand{\prin}{\stylesets{U}}
\newcommand{\prinel}{\styleelements{u}}
\newcommand{\prineln}[1]{\styleelements{u_{#1}}}

\newcommand{\prinL}{\styleLsets{U}}
\newcommand{\prout}{\stylesets{Y}}
\newcommand{\proutel}{\styleelements{y}}
\newcommand{\prouteln}[1]{\styleelements{y_{#1}}}
\newcommand{\proutL}{\styleLsets{Y}}
\newcommand{\prsol}{\stylesets{S}}
\newcommand{\prsolel}{\styleelements{s}}
\newcommand{\prsoleln}[1]{\styleelements{s_{#1}}}
\newcommand{\prst}{\stylesets{X}}
\newcommand{\prstel}{\styleelements{x}}
\newcommand{\prsteln}[1]{\styleelements{x_{#1}}}
\newcommand{\prstL}{\styleLsets{X}}
\newcommand{\prdyn}{\stylemorph{\operatorname{dyn}}}
\newcommand{\prreadout}{\stylemorph{\operatorname{ro}}}
\newcommand{\prObja}{\styleLsets{A}}
\newcommand{\prObjb}{\styleLsets{B}}
\newcommand{\prObjc}{\styleLsets{C}}
\newcommand{\prObjd}{\styleLsets{D}}
\newcommand{\prObje}{\styleLsets{E}}
\newcommand{\prObjt}{\styleLsets{T}}
\newcommand{\prObjv}{\styleLsets{V}}
\newcommand{\prObjw}{\styleLsets{W}}

\newcommand{\tangbundle}[1]{\mathcal{T}#1} % Tangent bundle


\newcommand{\pbacka}{\mapa^\sharp}
\newcommand{\pforwa}{\mapa}
\newcommand{\pbackb}{\mapb^\sharp}
\newcommand{\pforwb}{\mapb}
\newcommand{\pbackc}{\mapc^\sharp}
\newcommand{\pforwc}{\mapc}
\newcommand{\setlensap}{\setA^+}
\newcommand{\setlensam}{\setA^-}
\newcommand{\setlensbp}{\setB^+}
\newcommand{\setlensbm}{\setB^-}
\newcommand{\setlenscp}{\setC^+}
\newcommand{\setlenscm}{\setC^-}
\newcommand{\setlensdp}{\setD^+}
\newcommand{\setlensdm}{\setD^-}

\newcommand{\End}{{\mathbf{End}}}
%:nomenc:\Endof\setA:Endomorphisms of $\setA$
\newcommand{\Endof}[1]{\End(#1)}
%:nomenc-exclude:

\newcommand{\Aut}{{\mathbf{Aut}}}
%:nomenc:\Autof\setA:Automorphisms of $\setA$
\newcommand{\Autof}[1]{\Aut(#1)}
%:nomenc-exclude:

%:section:vol1/10_arrows/75_processes/procedures: Procedures
\newcommand{\ExecTime}{\CatStyle{ProcTime}} % Procedures with execution time
%:def:def:ExecTime
\newcommand{\sizefun}{\aword{size}} % Size of datatype
%:def:def:ExecTime
\newcommand{\ProcSize}{\CatStyle{ProcSize}} % Procedures with sized sets
%:def:def:ProcSize
\newcommand{\sizetran}{\sigma}
%:def:def:ProcSize
\newcommand{\ProcSizeTime}{\CatStyle{ProcSizeTime}} % Procedures with size-dependent durations
%:def:def:ProcSizeTime
\newcommand{\timefun}{\aword{dur}}
%:def:def:ProcSizeTime

\newcommand{\TimeMonoidal}{\CatStyle{T}}
\newcommand{\SpaceMonoidal}{\CatStyle{S}}
\newcommand{\SpaceTime}{\SpaceMonoidal\TimeMonoidal}
\newcommand{\ProcMod}{\CatStyle{ProcSizeTime}_{\TimeMonoidal}}
\newcommand{\ProcSTMod}{\CatStyle{ProcSizeTime}_{\SpaceTime}}
\newcommand{\spacefun}{\aword{space}}


%:section:vol1/05_pluribus/80_actions/matrix-groups:Matrix groups
\newcommand{\mgOn}{\EM{\mathrm{O}(n, \reals)}} % Orthogonal group
%:def:def:general-orthogonal-group
\newcommand{\mgSOn}{\EM{\mathrm{SO}(n, \reals)}} % Special orthogonal group
%:def:def:special-orthogonal-group
\newcommand{\mgGLn}{\EM{\mathrm{GL}(n, \reals)}} % General linear group
%:def:def:general-linear-group
\newcommand{\mgSLn}{\EM{\mathrm{SL}(n, \reals)}} % Special linear group
%:def:def:special-linear-group
\newcommand{\mgEn}{\EM{\mathrm{E}(n, \reals)}} % Euclidean group
%:def:def:general-euclidean-group
\newcommand{\mgSEn}{\EM{\mathrm{SE}(n, \reals^n)}} % Special euclidean group
%:def:def:special-euclidean-group

\newcommand{\mgSOtwo}{\EM{\mathrm{SO}(2, \reals)}}
%:nomenc-exclude:
\newcommand{\mgSOthree}{\EM{\mathrm{SO}(3, \reals)}}
%:nomenc-exclude:
\newcommand{\mgSEtwo}{\EM{\mathrm{SE}(2, \reals)}}
%:nomenc-exclude:
\newcommand{\mgSEthree}{\EM{\mathrm{SE}(3, \reals)}}
%:nomenc-exclude:

\newcommand{\Ematrix}[2]{
    \begin{bmatrix}
        #1 & #2 \\
        % \stylemorph{\vect{0}} & \stylemorph{1}
        \vect{0} & 1
    \end{bmatrix}
}

\newcommand{\Epoint}[1]{
    \begin{bmatrix}
        % #1 \\ \styleobj{1}
        #1 \\ 1
    \end{bmatrix}
}

%:section:vol1/05_pluribus/80_actions:\chrefplus{chap:actions}
\newcommand{\act}{\stylefunctors{\aword{act}}}
\newcommand{\lact}{\stylefunctors{\aword{Covact}}}
\newcommand{\ract}{\stylefunctors{\aword{Contravact}}}
\newcommand{\mgact}{\stylefunctors{\aword{apply}}}


\newcommand{\defmap}[6]{
    \begin{aligned}
        #1 \colon & #2 & \!\!#3        & \  #4 \\
        & #5 & \!\!\mapsto   & \  #6
    \end{aligned}
}

\newcommand{\defmapperiod}[6]{
    \begin{aligned}
        #1 \colon & #2 & \!\!#3        & \  #4, \\
        & #5 & \!\!\mapsto   & \  #6.
    \end{aligned}
}

\newcommand{\defmapcomma}[6]{
    \begin{aligned}
        #1 \colon & #2 & \!\!#3        & \  #4, \\
        & #5 & \!\!\mapsto   & \  #6,
    \end{aligned}
}

\newcommand{\defmapunnamed}[5]{
    \begin{aligned}
        & #1 & \!\!#2        & \  #3 \\
        & #4 & \!\!\mapsto   & \  #5
    \end{aligned}
}


\newcommand{\defmapset}[5]{%
    \defmap{#1}{#2}{\sto}{#3}{#4}{#5}%
}
\newcommand{\defmapperiodset}[5]{%
    \defmapperiod{#1}{#2}{\sto}{#3}{#4}{#5}%
}
\newcommand{\defmapcommaset}[5]{%
    \defmapcomma{#1}{#2}{\sto}{#3}{#4}{#5}%
}
\newcommand{\defmapunnset}[5]{%
    \defmapunnamed{#1}{#2}{\sto}{#3}{#4}%
}

% \newcommand{\defmappos}[5]{%
%   \defmap{#1}{#2}{\mto_{\Pos}}{#3}{#4}{#5}%
% }
% \newcommand{\defmapperiodpos}[5]{%
%   \defmapperiod{#1}{#2}{\mto_{\Pos}}{#3}{#4}{#5}%
% }
% \newcommand{\defmapcommapos}[5]{%
%   \defmapcomma{#1}{#2}{\mto_{\Pos}}{#3}{#4}{#5}%
% }

%
% \newcommand{\definemap}[5]{
%     \begin{aligned}
%         #1 \colon & #2 & \xrightarrow{\phantom{mi}} & \  #3 \\
%                   & #4 & \xmapsto{\phantom{mi}}     & \  #5
%     \end{aligned}
% }
%


\newcommand{\defmappos}[5]{
    \begin{aligned}
        #1 \colon & #2 & \pmb{\stylemorph{\rightarrow}}_\Pos\ \  & \   #3 \\
        & #4 & \xmapsto{\phantom{mm}} \                               & \ #5
    \end{aligned}
}

\newcommand{\defmapcommapos}[5]{
    \begin{aligned}
        #1 \colon & #2 & \pmb{\stylemorph{\rightarrow}}_\Pos\ \  & \   #3, \\
        & #4 & \xmapsto{\phantom{mm}} \                               & \ #5,
    \end{aligned}
}

\newcommand{\defmapperiodpos}[5]{
    \begin{aligned}
        #1 \colon & #2 & \pmb{\stylemorph{\rightarrow}}_\Pos\ \  & \   #3, \\
        & #4 & \xmapsto{\phantom{mm}} \                               & \ #5.
    \end{aligned}
}




\newcommand{\matmor}[1]{\stylemorph{\mat{#1}}}
\newcommand{\vectmor}[1]{\stylemorph{\vect{#1}}}
\newcommand{\vectob}[1]{\styleobj{\vect{#1}}}

\newcommand{\prstart}{\styleelements{\aword{st}}}
\definecolor{bcolor}{rgb}{0.6,0.8,0.3}

% \newcommand{\slimstart}{{\color{elementscolor}{\langle}}}
% \newcommand{\slimend}{{\color{elementscolor}{\rangle}}}
\newcommand{\elconcat}{\mathbin{\color{elementscolor},}}

% \newcommand{\morstart}{{\color{morphisms}{\langle}}}
% \newcommand{\morend}{{\color{morphisms}{\rangle}}}
% \newcommand{\morconcat}{\mathbin{\color{morphisms},}}

% could use \llbracket or \rrbracket
% \newcommand{\fatstart}{\textbf{\color{listcolor}{[}}}
% \newcommand{\fatend}{\textbf{\color{listcolor}{]}}}
\newcommand{\fatstart}{\llangle}
\newcommand{\fatend}{\rrangle}


\newcommand{\blackcomma}{\mathbin{\color{black},}}


% strict cartesian category construction
\newcommand{\cCat}[1]{%
    \EM{{\fatstart\mathbf{#1}\fatend}}%
}




\newcommand{\cMor}[1]{%
    \fatstart%
    #1%
    \fatend%
}
% \newcommand{\cprodmor}{\mathbin{\mathbin{\star}}} % cCat product morphisms

% tuples and lists operations
\newcommand{\tupentry}[2]{%
    #1%
    [%
    #2%
    ]%
}

\newcommand{\listentry}[2]{%
    #1%
    [%
    #2%
    ]%
}




\newcommand{\istype}[2]{#1\colon #2}

\newcommand{\Moore}{\CatStyle{Moo}} % Category of Moore machines
%:def:def:moore

\newcommand{\More}{\CatStyle{Mor}} % the Mor category
%:def:def:more

\newcommand{\signals}{\stylefunctors{S}}
\newcommand{\signalsin}[1]{\signals(#1)}



\newcommand{\vectorfield}{\CatStyle{VF}}

\newcommand{\tmpEB}{\CatStyle{EB}}
\newcommand{\tmpDS}{\CatStyle{DS}}

\newcommand{\catacOb}{\stylefunctors{\varphi}}
\newcommand{\catacMor}{\stylefunctors{\gamma}}

\newcommand{\monact}{\stylemorph{a}}

\newcommand{\pralpha}{\stylemorph{\alpha}}
\newcommand{\prbeta}{\stylemorph{\beta}}
\newcommand{\prc}{\styleobj{C}}
\newcommand{\prcop}{\styleobj{C}^{\ast}}
\newcommand{\prd}{\styleobj{D}}
\newcommand{\prdop}{\styleobj{D}^{\ast}}
\newcommand{\pre}{\styleobj{E}}
\newcommand{\funab}{(\funa\fthen\funb)}
\newcommand{\funabob}{{\funab}_{\text{ob}}}
\newcommand{\funabmor}{{\funab}_{\text{mor}}}
\newcommand{\prs}{\styleelements{s}}
\newcommand{\prt}{\styleelements{t}}


%\newcommand*\productop{\mathbin{\Pi}}
\newcommand*\productop{\amalg}
%effort and tracking
\newcommand{\effort}{P_\mathrm{effort}} % effort
\newcommand{\track}{P_\mathrm{track}} % tracking

%:section:vol1/exam2021:Exam 2021
%:nomenc-exclude:
\newcommand{\fast}{\text{fast}}
\newcommand{\slow}{\text{slow}}
\newcommand{\llarge}{\text{large}}
\newcommand{\ssmall}{\text{small}}
\newcommand{\llong}{\text{long}}
\newcommand{\sshort}{\text{short}}
\newcommand{\speed}{\textbf{Speed}}
\newcommand{\size}{\textbf{Size}}
\newcommand{\money}{\textbf{Money}}
\newcommand{\ttime}{\textbf{Time}}
\newcommand{\cheap}{\text{20K}}
\newcommand{\midprice}{\text{40K}}
\newcommand{\expensive}{\text{120K}}
\newcommand{\energy}{\textbf{Energy}}
\newcommand{\mass}{\textbf{Mass}}
\newcommand{\wdp}{\stylefunctors{\mathbf{w}}} % (Exam) generic design problem for wirings
\newcommand{\cdp}{\stylefunctors{\mathbf{c}}} % (Exam) generic fixed design problem as a profunctor

\newcommand{\poscheap}{\text{cheap}}
\newcommand{\posexpensive}{\text{expensive}}
\newcommand{\posmidrange}{\text{midrange}}
\newcommand{\poslight}{\text{light}}
\newcommand{\posheavy}{\text{heavy}}

%:section:vol1/50_opera/operads:\chrefplus{chap:operads}

\newcommand{\operada}{\mathcal{O}}
%:nomenc:\operada, \operadb: Generic operads names
\newcommand{\operadb}{\mathcal{P}}
%:nomenc-exclude:
%:section:vol1/25_translation/55_specialization: \chrefplus{chap:specialization}
%:section:vol1/25_translation/55_specialization/drawings: drawings
\newcommand{\Draw}{\CatStyle{Draw}} % Category of drawings
%:def:def:Draw

\newcommand{\scaling}{\text{sc}}
\newcommand{\rotation}{\text{rot}}
\newcommand{\translation}{\text{tra}}
\newcommand{\rototrans}{\text{rotra}}
\newcommand{\affinetrafo}{\text{aff}}


%
%:section:vol1/30_design/80_construct_dp: Construction
%
\newcommand{\diag}{\aword{diag}} % Diagonal function
%:nomenc:\diag_P:Diagonal function
%:def:def:diagonal-function
%:if:devel:
\newcommand{\codiag}{\aword{codiag}} % Diagonal function
%:nomenc:\codiag_P:Co-diagonal function
%:def:def:codiagonal-function
%:if:devel:

%:section:compositional-theories: Compositional theories
%:nomenc-exclude:
\newcommand{\statement}{\stylesets{P}}
\newcommand{\sol}{\stylesets{S}}
\newcommand{\solver}{\stylemaps{\sigma}}
\newcommand{\prob}{\stylemaps{f}}
\newcommand{\Lagset}{\CatStyle{LagSet}}
\newcommand{\Lagrel}{\CatStyle{LagRel}}
\newcommand{\LagCat}{\CatStyle{LagCat}}
\newcommand{\LagMonCat}{\CatStyle{LagMonCat}}
\newcommand{\LagDP}{\CatStyle{LagDP}}
\newcommand{\LagDPerf}{\CatStyle{LagDPerf}}
\newcommand{\NRel}{\CatStyle{NRel}}
\newcommand{\NSet}{\CatStyle{NSet}}
\newcommand{\DPerf}{\CatStyle{DPerf}}


%:section:dynamical-systems: Dynamical Systems
\newcommand{\Time}{\stylesets{T}}


%:section:negative: Negative design

\newcommand{\Neg}{\text{Neg}}

%:section:timedp: time dp
\newcommand{\dpmode}{\stylemaps{\varphi}}
\newcommand{\setindex}{\stylesets{I}}
\newcommand{\setindexel}{\styleelements{i}}
\newcommand{\setindexelb}{\styleelements{j}}

\newcommand{\setTime}{\stylesets{T}}
\newcommand{\setTimeel}{\styleelements{t}}


%:section:tosort: To sort
%:nomenc-exclude:

\newcommand{\repreff}[1]{\aword{rep}(#1)}

%\newcommand{\EfW}{\styleobj{\mathbf{W}}} % the world object
\newcommand{\EfW}{\globe} % the world object
%:def:def:effects
\newcommand{\morab}{\mora\mthen\morb}

\newcommand{\perm}{\mathbf{p}} % permutation
\newcommand{\Perms}{\mathbf{Perms}} % set of permutations
%:nomenc:\Perms_n: Group of permutations of $n$ elements.
\newcommand{\MaxSteepness}{\aword{MaxSteep}}


\newcommand{\checkyes}{{\color{darkgreen}\cmark}}
\newcommand{\checkno}{{\color{black}\xmark}}

\newcommand{\boolset}{\stylesets{\text{\textbf{Bool}}}}


\newcommand{\nlbigvee}{\bigvee\nolimits} % Bigvee with no limits
%:nomenc-exclude:



\newcommand{\colim}{\operatorname{colim\;}}
\newcommand{\Coll}{\operatorname{Col}}


\newcommand{\feasibleset}[1]{\stylesets{K}_{#1}}
\newcommand{\feasiblesetprojfun}{\stylesets{P}}
%\newcommand{\fix}{\text{fix}}

\newcommand{\sumres}{\mathbin{\boxplus}}

% \newcommand{\cproddp}{\mathbin{\circledstar}} % cCat product
% \newcommand{\cprod}{\mathbin{\circledast}} % cCat product
% \newcommand{\listconcat}{\mathbin{\color{black}\ast}}
% \newcommand{\tupconcat}{\mathbin{\color{black}\ast}}

% \newcommand{\cprod}{\mathbin{\mthen_{\hspace{-1pt}\stylemorph{\fatstart\hspace{-1.5pt}\fatend}}}}
% \newcommand{\listconcat}{\mathbin{\mthen_{\hspace{-0.4pt}\stylemorph{[\hspace{-1.1pt}]}}}}
% \newcommand{\tupconcat}{\mathbin{\mthen_{\hspace{-0.4pt}\stylemorph{\langle\hspace{-1.1pt}\rangle}}}}
\newcommand{\cprod}{\mathbin{\mthen_{\hspace{-1pt}\stylemorph{\fatstart}}}}
\newcommand{\listconcat}{\mathbin{\mthen_{\hspace{-0.4pt}\stylemorph{[}}}}
\newcommand{\tupconcat}{\mathbin{\mthen_{\hspace{-0.4pt}\stylemorph{\langle}}}}


\newcommand{\cproddp}{\cprod}


\newcommand{\Cont}{\CatStyle{Cont}}
\newcommand{\Surj}{\CatStyle{Surj}}

\newcommand{\dpid}{\funid} % Identity for DPs
\newcommand{\dpidat}[1]{\dpid_{#1}}
\newcommand{\dpthen}{\fthen} % Concatenation of DPs
%:def:def:dp-series

\newcommand{\dpsigma}[2]{\styledp{\Sigma}^{#1}_{#2}}


\newcommand{\asciichar}{{[0,127]}}


\newcommand{\Nom}{\stylefunctors{\operatorname{Nom}}}
%:nomenc:\NomSet{\CatA}{\Obja}{\Objb}: Nom-set between $\Obja$ and $\Objb$.

\newcommand{\NomSet}[3]{\Nom_{#1}\left({#2};\linebreak[0]{#3}\right)}
%:nomenc-exclude:
\newcommand{\prcatb}{\CatStyle{PR0}}
\newcommand{\engamma}{\styleobj{\aword{eo}}}
\newcommand{\engammaof}[2]{\engamma_{#1,#2}}
\newcommand{\enid}{\stylemorph{\aword{em}}}
\newcommand{\enidof}[1]{\enid_{#1}}
\newcommand{\enm}{\stylemorph{\aword{ei}}}
\newcommand{\enmof}[3]{\enm_{#1,#2,#3}}

\newcommand{\cardmap}{\stylemaps{\aword{card}}} % cardinality map
\newcommand{\cardof}[1]{\cardmap\left(#1\right)}
\newcommand{\absmap}{\stylemaps{\aword{abs}}} % absolute value map
\newcommand{\absvalueof}[1]{\absmap\left(#1\right)}
\newcommand{\normmap}{\stylemaps{\aword{norm}}} % norm
\newcommand{\normof}[1]{\normmap\left(#1\right)}

\newcommand{\divides}[2]{#1\vert#2}
\newcommand{\CatCop}{\CatC\op}
\newcommand{\funidC}{\funid_\CatC}
\newcommand{\dualof}[1]{{#1}^\vee}
\newcommand{\Objadual}{\dualof\Obja}
\newcommand{\matdim}[2]{{#1 \times #2}}
\newcommand{\ntimesn}{\matdim{n}{n}}
\newcommand{\realsnn}{\reals^{\ntimesn}}
\newcommand{\SGrp}{\CatLinkConcept{SGrp}{category of semigroups and morphisms}} % category of semigroups and morphisms
\newcommand{\Mon}{\CatLinkConcept{Mon}{category of monoids and morphisms}} % category of monoids and morphisms
\newcommand{\Grp}{\CatLinkConcept{Grp}{category of groups and morphisms}} % category of groups and morphisms
\newcommand{\UMoore}{\CatStyle{UMoore}}
\newcommand{\RMoore}{\CatStyle{RMoore}}

\newcommand{\setTwo}{\stylesets{2}} % The set with two elements
\newcommand{\posAdefinition}{\posA=\tupp{\posAset, {{\posAleq}}}}
\newcommand{\posBdefinition}{\posB=\tupp{\posBset, {{\posBleq}}}}

\newcommand{\realswithleq}{\tup{\reals,{{\Rleq}}}}
\newcommand{\natswithleq}{\tup{\natnumbers,{{\Nleq}}}}
\newcommand{\catcoprodpsi}{\stylemorph{\psi}}
\newcommand{\catprodphi}{\stylemorph{\phi}}
\newcommand{\realswithplusmonoid}{\tup{\reals,{{+}}}}

\newcommand{\VectRs}{\CatStyle{Vect}_{\mathbb{R}}}
\newcommand{\MatRs}{\CatStyle{Mat}_{\mathbb{R}}}
\newcommand{\VectR}{\LinkConcept{\VectRs}{category of real vector spaces}}
\newcommand{\MatR}{\LinkConcept{\MatRs}{category of real matrices}}
