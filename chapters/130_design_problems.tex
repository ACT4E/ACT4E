% !TEX root = ../CategoricalCoDesign.tex
\subsection{Creating design problems from catalogues}
\begin{shaded*}
\begin{definition}[Span]
Given a category~$\Cat{C}$, a \emph{span} from an object~$x$ to an object~$y$ is a diagram of the form
\begin{equation}
\includesag{51_span}
\end{equation}
where~$z$ is some other object of~$\Cat{C}$. 
\end{definition}
\end{shaded*}
\begin{example}
Consider the category~$\mathbf{trek}$, introduced in \cref{sec:trekking}. An example of span in this category is reported in~\cref{fig:exmountains}.
\begin{figure}[h!]
\begin{center}
\includesag{130_mountains}
\end{center}
\caption{Swiss peaks can be thought of as a span in~$\mathbf{trek}$. \label{fig:exmountains}}
\end{figure}
Recall that \textsf{Matterhorn Peak}, \textsf{Jungfrau Peak}, and \textsf{Pilatus Peak} are objects of~$\mathbf{trek}$, and the arrows are morphisms in~$\mathbf{trek}$ (paths from one location to the other).
\end{example}

\begin{definition}[Catalogue] \label{def:catalogue}
A \emph{catalogue} is a span in~$\Pos$.
It thus consists of 3 posets~$I$,~$\F{F}$,~$\R{R}$.
We call them implementation space, functionality space, and requirements space, respectively. We need to define two map~$\prov \colon I \to \F{F}$ (an implementation \textbf{prov}ides a functionality) and~$\req\colon I \to \R{R}$ (an implementation \textbf{req}uires resources). 
\begin{equation}
\includesag{130_catalogue}
\end{equation}
\end{definition}

\begin{definition}[Design problem induced by a catalogue]
Every catalogue~$\tup{I,\prov,\req}$ \emph{induces} a design problem of the form~$d\colon \F{F}\tickar \R{R}$, with
\begin{equation}
    \begin{aligned}
    d\colon \F{F}\op \times \R{R}&\to \Bool\\
    \tup{\F{f}^*,\R{r}}&\mapsto \bigvee_{i\in I}\left(\prov(i)\ordgeq_{\F{F}}\F{f} \right)\wedge \left( \req(i)\ordleq_{\R{R}}\R{r}\right)
    \end{aligned}
\end{equation}
\end{definition}

\begin{example}
We now want to revisit the leading example of \cref{sec:attributes_sameness} with the newly introduced co-design perspective. Let's consider a list of electrical motors as in~\cref{tab:electric_motors_2}.
\begin{table}[h]
    \centering
    \adjustbox{max width=\textwidth}{%
    \begin{tabular}{c|c|c|c|c|c}
         Motor ID & Company& $\unit[\text{Torque}]{[kg\cdot cm]}$ & \unit[Weight]{[g]} & \unit[Max Power]{[W]} & \unit[Cost]{[USD]} \\
         \hline
         \textsf{1204}&\textsf{SOYO} & 0.18& 60.0 &2.34 &19.95\\
         \textsf{1206}&\textsf{SOYO} &0.95& 140.0 &3.00 &19.95\\
         \textsf{1207}&\textsf{SOYO} &0.65& 130.0 &2.07 &12.95\\
         \textsf{2267}&\textsf{SOYO} &3.7& 285.0 &4.76 &16.95\\
         \textsf{2279}&\textsf{Sanyo Denki} &1.9& 165.0  &5.40 & 164.95\\
        \textsf{1478}&\textsf{SOYO} & 19.0& 1,000 & 8.96&49.95\\
        \textsf{2299}&\textsf{Sanyo Denki} & 2.2& 150.0 &5.90&59.95
    \end{tabular}%
    }
    \caption{A simplified catalogue of motors.}
    \label{tab:electric_motors_2}
\end{table}

We can think of this as a catalogue of electric motors~$\tup{I_\mathrm{EM},\prov_\mathrm{EM},\req_\mathrm{EM}}$. In particular, the set of implementations collects all the motor models, which we can specify using the motor IDs:
\begin{equation}
    I_\mathrm{EM}=\{1204,1206,1207,2267,2279,1478,2299 \}.
\end{equation}
We now have to think about \R{resources} and \F{functionalities}. Each motor \R{requires} some \R{weight} (in \unit[]{g}), \R{power} (in \unit[]{W}), and has some \R{cost} (in USD), and \F{provides} some \F{torque} (in~$\unit[]{kg\cdot cm}$). The correspondences are given by the details in \cref{tab:electric_motors_2}. For instance, we have~$\prov_\mathrm{EM}(1204)=0.18$ and $\req_\mathrm{EM}(1204)=\{\unit[60]{g},\unit[2.34]{W},\unit[19.95]{USD}\}$. The catalogue induces a design problem $d_\mathrm{EM}$ with diagrammatic form as in \cref{fig:dp_em}. In particular, we can query the design problem for combinations of \F{functionalities} and \R{resources}. For instance:
\begin{equation}
    d\left(\unit[0.2]{kg\cdot cm}, \tup{\unit[50.0]{g},\unit[2.0]{W},\unit[15.0]{USD}} \right)=\false,
\end{equation}
since no listed model can provide \unit[0.2]{kg\cdot cm} \F{torque} by requiring the set of resour

\begin{figure}[tbh]
\begin{center}
    \begin{tikzpicture}[DP]
    \node[dp={1}{3}] (mot) {Electric Motor};
    \draw[runconn, runame={\R{weight}}] (mot_res1){};
    \draw[runconn, runame={\R{max. power}}] (mot_res2){};
    \draw[runconn, runame={\R{cost}}] (mot_res3){};
    \draw[funconn, funame={\F{torque}}] (mot_fun1){};
\end{tikzpicture}
\end{center}
\caption{The battery design problem.\label{fig:dp_em}}
\end{figure}

\GZ{maybe need to introduce completed version of positive reals, what do you think JL?}

\begin{comment}
\begin{figure}[tbh]
\begin{center}
    \begin{tikzpicture}[DP]
    \node[dp={2}{3}] (bat) {Battery};
    \draw[runconn, runame={$c_\mathrm{b}$}] (bat_res1){};
    \draw[runconn, runame={$m_\mathrm{b}$}] (bat_res2){};
    \draw[runconn, runame={$R_\mathrm{b}$}] (bat_res3){};
    \draw[funconn, funame={$C_\mathsf{prov}$}] (bat_fun1){};
    \draw[funconn, funame={$M_\mathsf{b}$}] (bat_fun2){};
\end{tikzpicture}
\end{center}
\caption{The battery design problem.\label{fig:dp_em}}
\end{figure}

\begin{table}[tbh]
\begin{center}
\begin{tabular}{cccc}
Technology&Specific energy [\unitfrac[]{J}{kg}]&Specific cost [\unitfrac[]{J}{\$}]&Life [\# cycles]\\
\hline
$\mathsf{NiMH}$&100.0&3.41&500\\
$\mathsf{NiH2}$&45.0&10.5&20,000\\
$\mathsf{LCO}$&195.0&2.84&750\\
$\mathsf{LMO}$&150.0&2.84&500\\
$\mathsf{NiCad}$&30.0&7.50&500\\
$\mathsf{SLA}$&30.0&7.00&500\\
$\mathsf{LiPo}$&250.0&2.50&600\\
$\mathsf{LFP}$&90.0&1.50&1,500
\end{tabular}
\end{center}
\caption{Specifications of common battery technologies~\cite{censi2015}. \label{tab:battery}}
\end{table}
\todo{finish with design problem description}
\end{comment}
\end{example}

