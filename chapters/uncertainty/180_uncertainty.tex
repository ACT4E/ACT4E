% !TEX root = standalone.tex
\todo{Add introduction on uncertainty}


\todo{Give two different forms of interval
orders}

\todo{Give interpretation as categorical constructs}


\begin{ctdefinition}[Twisted arrow category]
    Given a category \CatC, we denote its \emph{twisted arrow category} by $\twisted{\CatC}$. This is a category which is composed of:
    \begin{compactenum}
        \item \emph{Objects:} Arrows (morphisms) in \CatC.
        \item \emph{Morphisms:} A morphism between two arrows $f\colon A\to B$, $g\colon C\to D$ is given by the pair of arrows $\tup{p,q}$ such that the following diagram commutes:
        \begin{equation}
            \includesag{180_twistedarrow}
        \end{equation}
    \end{compactenum}
\end{ctdefinition}

\begin{example}[Intervals]
    Consider a poset~$P$. The twisted arrow category~$\twisted{P}$ is isomorphic to the set of nonempty \emph{intervals}~$[a,b]=\{p \in P\mid a\leqP p \leqP b\}$. Note that~$\twisted{P}$ is a poset as well, ordered by inclusion.
\end{example}
\begin{remark}
    Recall \cref{sec:posetsarecats} and note that the map which sends a poset (a category) to its twisted arrow category is a functor, which sends objects of the poset
\end{remark}




\section{Using monads to understand uncertainty}

Take the $\mathsf{Unc}$ functor $\mathsf{Unc}\colon \DP\to \DP$ which
\begin{compactenum}
    \item Maps an object $P$ in \DP (poset) to its twisted arrow category $\twisted{P}$, representing a poset interval.
    \item Maps a morphism in \DP $d\colon \F{F}\profto \R{R}$ to $\tup{\low d,\upp d}$, where
    \begin{equation}
        \begin{aligned}
            \low d\colon \F{F_{\low}}&\profto \R{R_{\low}},\\
            \upp d\colon \F{F_{\upp}}&\profto \R{R_{\upp}},
        \end{aligned}
    \end{equation}
    and $\tup{\low d,\upp d}$ is a boolean profunctor of the form
    \begin{equation}
        \begin{aligned}
            &\tup{\low d,\upp d}\colon \left(\F{F_{\low}}\times \F{F_{\upp}} \right)\op \times \left(\R{R_{\low}}\times \R{R_{\upp}} \right)\toinPos \Bool\\
            &\tup{\tup{\F{f_{\low}},\F{f_{\upp}}}^*,\tup{\R{r_{\low}},\R{r_{\upp}}}}\mapsto \low d(\F{f_{\low}}^*,\R{r_{\low}})\wedge \upp d(\F{f_{\upp}}^*,\R{r_{\upp}})
        \end{aligned}
    \end{equation}
\end{compactenum}

\todo{write down better}
Is this a functor?

\begin{proof}
    Consider two design problems $f\colon \F{A}\profto \R{B}$ and $g\colon \F{B}\profto \R{C}$.

    We know that
    \begin{equation}
        \begin{aligned}
            \unc(f)&\colon \left( \F{A_{\low}}\times \F{A_{\upp}}\right)\op \times \left( \R{B_{\low}}\times \R{B_{\upp}}\right)\toinPos \Bool\\
            \unc(g)&\colon \left( \F{B_{\low}}\times \F{B_{\upp}}\right)\op \times \left( \R{C_{\low}}\times \R{C_{\upp}}\right)\toinPos \Bool.
        \end{aligned}
    \end{equation}

    We have
    \begin{equation}
        \begin{aligned}
            &\left(\unc(f)\then \unc(g)\right) (\tup{\ubar{a},\bar{a}}^*,\tup{\ubar{c},\bar{c}})\\
            &=\bigvee_{\tup{\ubar{b},\bar{b}}\in B_{\low}\times B_{\upp}} \unc(f)(\tup{\ubar{a},\bar{a}}^*,\tup{\ubar{b},\bar{b}})\wedge \unc(g)(\tup{\ubar{b},\bar{b}}^*,\tup{\ubar{c},\bar{c}})\\
            &= \bigvee_{\tup{\ubar{b},\bar{b}}\in B_{\low}\times B_{\upp}} \low f(\ubar{a}^*,\ubar{b})\wedge \upp f(\bar{a}^*,\bar{b})\wedge \low g(\ubar{b}^*,\ubar{c})\wedge \upp g(\bar{b}^*,\bar{c})\\
            &=(\low f\then \low g)(\ubar{a}^*,\ubar{c})\wedge (\upp f\then \upp g)(\bar{a}^*,\bar{c})\\
            &=\low f\then g (\ubar{a}^*,\ubar{c})\wedge \upp f\then g(\bar{a}^*,\bar{c})\\
            &=\unc(f\then g)(\tup{\ubar{a},\bar{a}}^*,\tup{\ubar{c},\bar{c}}).
        \end{aligned}
    \end{equation}
\end{proof}


\section{L and U monads}
In this section we propose another example of monads related to posets and design problems. We start by defining the \emph{$U$ endofunctor}.
\begin{definition}[$U$ endofunctor]
    \label{def:ufunctor}
    The \emph{$U$ endofunctor} has the form $U\colon \Pos \to \Pos$ and acts on objects and morphisms as follows:
    \begin{compactenum}
        \item \emph{On objects}: Given a poset $P\in \Ob_\Pos$, $U$ maps $P$ to its upper set\footnote{Recall that in \cref{lem:u_bounded_lat} we proved that the upper set is itself an object of \Pos.}.
        \item \emph{On morphisms}: Given posets $P,Q$, and a monotone map $f\colon P\to Q$, the $U$ endofunctor acts as:
        \begin{equation}
            \begin{aligned}
                U(f)\colon \Up P&\to \Up Q\\
                P'&\mapsto \upit \left( \bigcup_{p\in P'} \{f(p)\}\right).
            \end{aligned}
        \end{equation}
    \end{compactenum}
\end{definition}
We now want to prove that the $U$ endofunctor is an endofunctor, and the proof requires the following two facts.
\begin{lemma}
    \label{lem:unpack_u_functor}
    Given posets $P,Q$, a monotone map $f\colon P \to Q$, and a family of singleton sets $\{S_i\}_{i\in I}$, with $S_i=\{s_i\}$, $s_i\in P$, the following equality holds:
    \begin{equation}
        \label{eq:lemma_unpack}
        \upit\left( \bigcup_{p\in \upit \bigcup_{i\in I}S_i}\{f(p)\}\right)= \upit \left( \bigcup_{i\in I} \{f(s_i)\}\right).
    \end{equation}
\end{lemma}
\begin{proof}
    We first want to show that:
    \begin{equation}
        \label{eq:unpack_1}
        \underbrace{\upit \left(\bigcup_{p\in \upit \bigcup_{i\in I}S_i}\{f(p)\} \right)}_{\star}\subseteq \upit \underbrace{\left( \bigcup_{i\in I}\{f(s_i)\}\right)}_{\diamond}.
    \end{equation}
    Let's take a
    \begin{equation}
        q\in \upit\left( \bigcup_{p\in \upit \bigcup_{i\in I}S_i}\{f(p)\}\right).
    \end{equation}
    If we have such a $q$, it means that there exists a
    \begin{equation}
        q'\in \bigcup_{p\in \upit\bigcup_{i\in I}S_i}\{f(p)\}
    \end{equation}
    such that~$q'\ordleq_Q q$, and hence there is a~$p'\in \upit \bigcup_{i\in I} S_i$ such that $q'=f(p')$. Consequently, there must exist an $i'\in I$ such that $s_{i'}\ordleq_P p'$. The monotonicity of $f$ implies:
    \begin{equation}
        f(s_{i'})\ordleq_Q f(p')=q'\ordleq_Q q.
    \end{equation}
    We know that $s_{i'}\in \diamond$ and any $q^*\in Q$ satisfying $f(s_{i'})\ordleq_Q q^*$ belongs to $\upit \diamond$. Therefore,~$\star\subseteq \upit \diamond$, which proves the validity of \cref{eq:unpack_1}.

    We now want to show that:
    \begin{equation}
        \label{eq:unpack_2}
        \upit \left(\bigcup_{p\in \upit \bigcup_{i\in I}S_i}\{f(p)\} \right)\supseteq \upit \left( \bigcup_{i\in I}\{f(s_i)\}\right).
    \end{equation}
    By now taking a
    \begin{equation}
        q\in \upit \left( \bigcup_{i\in I}\{f(s_i)\}\right),
    \end{equation}
    we know that there is a $i'\in I$ such that $f(s_{i'})\ordleq_Q q$. Furthermore, we know that $f(s_{i'})\in \diamond$. Therefore, any $q^*\ordgeq_Q f(s_{i'})$ must be in $\upit \diamond$, meaning that $q\in \star$, and proving the validity of \cref{eq:unpack_2}.

    The validity of \cref{eq:unpack_1} and \cref{eq:unpack_2} implies \cref{eq:lemma_unpack}.
\end{proof}

\begin{lemma}
    \label{lem:unpack_part_2}
    Given posets $P,Q$ and a monotone map $f\colon P\to Q$, we have:
    \begin{equation}
        \upit \left( \bigcup_{p'\in \upit \{p\}} \{f(p')\}\right)=\upit \{f(p)\}.
    \end{equation}
\end{lemma}
\begin{proof}
    The proof follows from \cref{lem:unpack_u_functor}, by considering a family of singleton sets consisting solely of the set $\{p\}$.
\end{proof}
We can now show that the $U$ endofunctor is indeed a functor.
\begin{lemma}
    The $U$ endofunctor is indeed a functor.
\end{lemma}
\begin{proof}
    $U$ has a valid form and given a poset $P$, maps $\id_P$ to $\id_{\Up P}$. We now need to show that $U$ fulfills morphism composition. Consider maps $f\colon P \to Q$ and $g\colon Q \to R$. We have:
    \begin{equation}
        \label{eq:ufunctor_1}
        \begin{aligned}
            U(f\then g)\colon \Pos &\to \Pos\\
            P'&\mapsto \upit \left( \bigcup_{p\in P'} \{g(f(p))\}\right).
        \end{aligned}
    \end{equation}
    On the other hand, we have:
    \begin{equation}
        \begin{aligned}
            U(f)\colon \Pos &\to \Pos\\
            P'&\mapsto \upit \left( \bigcup_{p\in P'}\{f(p)\}\right),
        \end{aligned}
    \end{equation}
    and
    \begin{equation}
        \begin{aligned}
            U(g)\colon \Pos &\to \Pos\\
            Q'&\mapsto \upit \left( \bigcup_{q\in Q'}\{g(q)\}\right),
        \end{aligned}
    \end{equation}
    leading to
    \begin{equation}
        \label{eq:ufunctor_2}
        \begin{aligned}
            U(f)\then U(g)\colon \Pos &\to \Pos\\
            P'&\mapsto \upit\left( \bigcup_{q\in \upit \bigcup_{p\in P'}\{f(p)\}}\{g(q)\}\right)\\
            &\mapsto \upit \left(\bigcup_{p\in P'} \{ g(f(p))\}\right). \qquad \qquad (\cref{lem:unpack_u_functor})
        \end{aligned}
    \end{equation}
    Since \cref{eq:ufunctor_1} and \cref{eq:ufunctor_2} are equivalent, $U$ is a functor.
\end{proof}
Having proven that $U$ is a valid functor, we are now ready to define the $U$ monad.
\begin{definition}[$U$ monad]
    The \emph{$U$ monad} on \Pos consists of:
    \begin{compactenum}
        \item The $U$ endofunctor (\cref{def:ufunctor}).
        \item The unit natural transformation $\eta_U \colon \id_{\Pos}\Rightarrow U$, which associates to every object $P\in \Ob_\Pos$ a morphisms in \Pos given by:
        \begin{equation}
            \begin{aligned}
                \eta_{U}^P\colon P &\to \Up P\\
                p&\mapsto \upit \{p\}.
            \end{aligned}
        \end{equation}
        \item The compositiona natural transformation $\mu_U\colon U\then U\Rightarrow U$, which associates to every $P\in \Ob_\Pos$ the morphism in \Pos given by:
        \begin{equation}
            \begin{aligned}
                \mu{_U}^P\colon \Up{(\Up{P})}&\to \Up P\\
                P''&\mapsto \bigcup_{P'\in P''}P'.
            \end{aligned}
        \end{equation}
    \end{compactenum}
\end{definition}

\begin{lemma}
    The $U$ monad is indeed a monad.
\end{lemma}
\begin{proof}
    To show that $U$ is indeed a monad, we need to show the following:
    \begin{compactenum}
        \item $\eta_U$ is a natural transformation;
        \item $\mu_U$ is a natural transformation;
        \item left unitality holds;
        \item right unitality holds;
        \item associativity holds;
    \end{compactenum}
    We prove them in order.

    \emph{1)~$\eta_U$ is a natural transformation}: We need to show that for any~$f\in \hom_{\Pos}(P,Q)$, we have:
    \begin{equation}
        \id_{\Pos}(f)\then \eta_{U}^Q=\eta_{U}^P\then U(f).
    \end{equation}
    By expanding the left-hand side, we obtain:
    \begin{equation}
        \left[\id_{\Pos}(f)\then \eta_{U}^Q\right](p)=\upit \{f(p)\}.
    \end{equation}
    By expanding the right-hand side, we get:
    \begin{equation}
        \begin{aligned}
            \eta_{U}^P\colon \Pos &\to \Pos\\
            p&\mapsto \upit \{p\}. \qquad \qquad (\cref{lem:unpack_part_2})
        \end{aligned}
    \end{equation}
    and
    \begin{equation}
        \begin{aligned}
            U(f)\colon \Pos &\to \Pos\\
            P'&\mapsto \upit \bigcup_{p'\in P'}\{f(p')\},
        \end{aligned}
    \end{equation}
    and hence
    \begin{equation}
        \begin{aligned}
            \left[\eta_{U}^P\then U(f)\right](p)&=\upit \left( \bigcup_{p'\in \upit \{p\}} \{f(p')\}\right)\\
            &=\upit \{f(p)\}.
        \end{aligned}
    \end{equation}

    \emph{2)~$\mu_U$ is a natural transformation}:
    \todo{finish/continue}
\end{proof}
