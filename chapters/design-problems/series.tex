\section{Interconnection of DPIs}

\todo{Do here the interconnection of two DPIs and their semantics.}
Graphically, one is allowed to connect only edges of different
color, and of the same type. This interconnection is indicated with the symbol~``$\posleq$''
in a rounded box~(\cref{fig:connection}).

\begin{figure}[h]
  \centering
  \includegraphics[scale=0.33]{unc_connection}
  \caption{\label{fig:connection}}
\end{figure}

%\captionsideleft{\label{fig:connection}}{\includegraphics[scale=0.33]{papers/uncertainty/unc_connection.pdf}}

\noindent The semantics of the interconnection is that the resources
required by the first DPI are provided by the second DPI. This is
a partial order inequality constraint of the type~$\res_{1}\posleq\fun_{2}$.


\todo{Show that composition so defined does not make DPI a category, but it makes it a bicategory (associative
  up to isomorphism)}


\todo{discussion about \emph{decomposition is not decoupling}}
