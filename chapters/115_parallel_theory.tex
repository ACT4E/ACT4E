% !TEX root = ../ACT4E-full.tex
% \section{Parallel composition}
So far, we have described a single way to compose morphisms of a category: the~$\then$ operation. However, category theory allows to define other ways of composing morphisms, adding structure to the basic category defined in~\cref{def:categorymain}.
\begin{ctdefinition}[Monoidal category]\label{def:monoidal_cat}
A \emph{monoidal structure} on a category~$\CatC$ consists of:
\begin{compactenum}
    \item An object~$I\in \Ob_\CatC$ called the \emph{monoidal unit}.
    \item A functor~$\otimes \colon \CatC \times \CatC\to \CatC$, called the \emph{monoidal product}.
\end{compactenum}
The two constituents are subject to the natural isomorphisms:
\begin{compactenum}
    \item[a)] $\lambda_c \colon I\otimes c \cong c$ for every~$c\in \Ob_\CatC$,
    \item[b)] $\rho_c \colon c\otimes I \cong c$ for every~$c\in \Ob_\CatC$,
    \item[c)] $\alpha_{c,d,e}\colon (c\otimes d)\otimes e \cong c\otimes (d\otimes e)$ for every~$c,d,e\in \Ob_\CatC$.
\end{compactenum}
These isomorphisms are themselves required to satisfy the triangle identity
\begin{equation}
\includesag{30_triangle_identity}
\end{equation}
and the pentagon identity
\begin{equation}
\includesag{30_pentagon_identity}
\end{equation}
for $a,b,c,d\in \Ob(\CatC)$.
\noindent A category equipped with a monoidal structure is called a \emph{monoidal category}. Note that in the case in which the isomorphisms in a), b), and c) are equivalences, one calls the category \emph{strict} monoidal.
\end{ctdefinition}

\begin{example}
Let's digest the definition of monoidal category with an explanatory example. We consider the structure~$\tup{\Set,\times,\singleton}$ and show that it indeed forms a monoidal category. First of all, we specify how the monoidal product (cartesian product here) acts on objects and morphisms in \Set (it is a functor). Given~$A,B\in \Ob_{\Set}$,~$A\times B$ is the usual cartesian product of sets, and given~$f\colon A\to A'$, $g\colon B\to B'$, we have:
\begin{equation*}
    \begin{aligned}
    (f\times g)\colon A\times B&\toiso A'\times B'\\
    \tup{a,b}&\mapsto \tup{f(a),g(b)}.
    \end{aligned}
\end{equation*}
Furthermore, given any~$A,B,C\in \Ob_{\Set}$, we specify the unitor~$\alpha_{A,B,C}$:
\begin{equation*}
    \begin{aligned}
    \alpha_{A,B,C}\colon (A\times B)\times C&\to A\times (B\times C)\\
    \tup{\tup{a,b},c}&\mapsto \tup{a,\tup{b,c}}
    \end{aligned}
\end{equation*}
This defines an isomorphism (I can ``back and forth'', by switching the tuple separation). We now need to check whether~$\alpha$ is natural. We check this graphically, using the commutative diagram in \cref{fig:monoidal_set_ass_nat}.

\begin{figure}[h!]
\begin{center}
\includesag{115_set_mon_ass}
\end{center}
\caption{\label{fig:monoidal_set_ass_nat}}
\end{figure}

Given an object~$A\in \Ob_\Set$, the unitor~$\lambda_A$ is given by:
\begin{equation*}
    \begin{aligned}
    \lambda_A\colon \singleton \times A &\toiso A\\
    \tup{\ast,a}&\mapsto a.
    \end{aligned}
\end{equation*}

Again, this defines an isomorphism. Consider a morphism~$f\colon A\to A'$. We now prove naturality graphically (\cref{fig:monoidal_set_unit_nat}).

\begin{figure}[h!]
\begin{center}
    \begin{tikzcd}
    \singleton \times A \arrow{r}{\lambda_A}&A\arrow{d}{f}\\
    \singleton \times A'\arrow{r}[swap]{\lambda_{A'}}
    \arrow{d}[swap]{\id_{\singleton}\times f}&A'
    \end{tikzcd}
\end{center}
\end{figure}
\end{example}


\begin{example}
\label{ex:robot}
Consider $\mathbb{R}^2$, discretized as a two-dimensional grid, representing locations (cells) which a robot can reach. The configuration space of the robot is $\mathbb{R}^2\times \Theta$, where $\Theta=[0,2\pi)$. A specific configuration $\tup{x,y,\theta}$ is characterized at each time by the position of the robot $x,y\in \mathbb{R}$ and its orientation $\theta \in \Theta$. The action space of the robot is $\mathcal{A}=\{\mathsf{stay},\leftarrow, \rightarrow, \uparrow, \downarrow\}$. This is a category, where each configuration of the robot is an object, and morphisms are robot actions which change configurations. Each configuration has the identity morphism which does not change it ($\mathsf{stay}$). Composition of morphisms is given by concatenation of actions (\cref{fig:robotcategory}). Assuming the existence of multiple robots $r_i=\tup{x_i,y_i,\theta_i}$, it is possible to define a product $r_i\otimes r_j$, which is to be intended as ``we have a robot at configuration $r_i$ and another one at configuration $r_j$''. However, this cannot be a proper monoidal product, because two robots cannot have the same configuration (physically, they cannot lie on each other), and hence $r_i\otimes r_i$ does not exist. By assuming that two robots could share the same configuration, this would be a valid monoidal product.
\begin{figure}[tbh]
\begin{center}
\includesag{120_robotcategory}
\end{center}
\caption{Example of the robot category. \label{fig:robotcategory}}
\end{figure}
\end{example}


\begin{ctdefinition}[Symmetric monoidal category]
Let~$\tup{\CatC,\otimes,I}$ be a monoidal category (\cref{def:monoidal_cat}). A \emph{symmetric structure} on it consists of one component: For any objects~$c,d\in\Ob_\CatC$ an isomorphism~$\sigma_{c,d}\colon (c\otimes d)\To{\cong}(d\otimes c)$, called the \emph{braiding}. The braiding must satisfy:
\begin{enumerate}
	\item \emph{Naturality:} Given any morphisms~$f_1\colon c_1\to d_1$ and~$f_2\colon c_2\to d_2$, the following diagram must commute:
	\begin{equation}
	\includesag{50_sym_1}
	\end{equation}
	\item \emph{Triangle identities:} Given any objects~$c,d\in\Ob_\CatC$, the following diagrams must commute:
\begin{equation}
	\includesag{50_sym_2}
\end{equation}
\item \emph{Hexagon identity:} Given any objects~$c,d,e\in \Ob_\CatC$, the following diagram must commute:
\begin{equation}
    \includesag{50_sym_3}
\end{equation}
\end{enumerate}
\end{ctdefinition}
