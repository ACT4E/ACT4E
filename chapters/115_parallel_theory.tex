% !TEX root = ../CategoricalCoDesign.tex
% \section{Parallel composition}
So far, we have described a single way to compose morphisms of a category: the~$\then$ operation. However, category theory allows to define other ways of composing morphisms, adding structure to the basic category defined in~\cref{def:categorymain}.
\begin{ctdefinition}[Monoidal category]\label{def:monoidal_cat}
A \emph{monoidal structure} on a category~$\CatC$ consists of:
\begin{compactenum}
    \item An object~$I\in \Ob(\CatC)$ called the \emph{monoidal unit}.
    \item A functor~$\otimes \colon \CatC \times \CatC\to \CatC$, called the \emph{monoidal product}.
\end{compactenum}
The two constituents are subject to the natural isomorphisms:
\begin{compactenum}
    \item[a)] $\lambda_c \colon I\otimes c \cong c$ for every~$c\in \Ob(\CatC)$,
    \item[b)] $\rho_c \colon c\otimes I \cong c$ for every~$c\in \Ob(\CatC)$,
    \item[c)] $\alpha_{c,d,e}\colon (c\otimes d)\otimes e \cong c\otimes (d\otimes e)$ for every for every~$c,d,e\in \Ob(\CatC)$.
\end{compactenum}
These isomorphisms are themselves required to satisfy the triangle identity
\begin{equation}
\includesag{30_triangle_identity}
\end{equation}
and the pentagon identity
\begin{equation}
\includesag{30_pentagon_identity}
\end{equation}
for $a,b,c,d\in \Ob(\CatC)$.
\noindent A category equipped with a monoidal structure is called a \emph{monoidal category}.
\end{ctdefinition}

\GZ{Positive example first}
\begin{example}
\label{ex:robot}
Consider $\mathbb{R}^2$, discretized as a two-dimensional grid, representing locations (cells) which a robot can reach. The configuration space of the robot is $\mathbb{R}^2\times \Theta$, where $\Theta=[0,2\pi)$. A specific configuration $\tup{x,y,\theta}$ is characterized at each time by the position of the robot $x,y\in \mathbb{R}$ and its orientation $\theta \in \Theta$. The action space of the robot is $\mathcal{A}=\{\mathsf{stay},\leftarrow, \rightarrow, \uparrow, \downarrow\}$. This is a category, where each configuration of the robot is an object, and morphisms are robot actions which change configurations. Each configuration has the identity morphism which does not change it ($\mathsf{stay}$). Composition of morphisms is given by concatenation of actions (\cref{fig:robotcategory}). Assuming the existence of multiple robots $r_i=\tup{x_i,y_i,\theta_i}$, it is possible to define a product $r_i\otimes r_j$, which is to be intended as ``we have a robot at configuration $r_i$ and another one at configuration $r_j$''. However, this cannot be a proper monoidal product, because two robots cannot have the same configuration (physically, they cannot lie on each other), and hence $r_i\otimes r_i$ does not exist. By assuming that two robots could share the same configuration, this would be a valid monoidal product.
\begin{figure}[tbh]
\begin{center}
\includesag{120_robotcategory}
\end{center}
\caption{Example of the robot category. \label{fig:robotcategory}}
\end{figure}
\end{example}


\begin{ctdefinition}[Symmetric monoidal category]
Let~$\tup{\CatC,\otimes,\singleton}$ be a monoidal category (\cref{def:monoidal_cat}). A \emph{symmetric structure} on it consists of one component: For any objects~$c,d\in\Ob_\CatC$ an isomorphism~$\sigma_{c,d}\colon (c\otimes d)\To{\cong}(d\otimes c)$, called the \emph{braiding}. The braiding must satisfy:
\begin{enumerate}
	\item \emph{Naturality:} Given any morphisms~$f_1\colon c_1\to d_1$ and~$f_2\colon c_2\to d_2$, the following diagram must commute:
	\begin{equation}
	\includesag{50_sym_1}
	\end{equation}
	\item \emph{Triangle identities:} Given any objects~$c,d\in\Ob_\CatC$, the following diagrams must commute:
\begin{equation}
	\includesag{50_sym_2}
\end{equation}
\item \emph{Hexagon identity:} Given any objects~$c,d,e\in \Ob_\CatC$, the following diagram must commute:
\begin{equation}
    \includesag{50_sym_3}
\end{equation}
\end{enumerate}
\end{ctdefinition} 

\begin{example}
\label{ex:robot_2}
Consider \cref{ex:robot}. By assuming that two robots can share the same configuration and that robots are indistinguishable (i.e.,~$r_i\otimes r_j=r_j\otimes r_i$ for any robot indey~$i,j$), one would have a symmetric monoidal category.
\end{example}

