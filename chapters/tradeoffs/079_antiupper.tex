\section{From antichains to uppersets, and viceversa}
\begin{definition}[Upper closure operator]
    \label{def:upperclosure}
    The \emph{upper closure operator} $\uparrow$ maps a subset to the smallest upper set that includes it, i.e.:
    \begin{equation}
        \begin{aligned}
            \uparrow \colon \powerset(P)&\to \mathsf{U}P\\
            S&\mapsto \{y\in P \mid \exists x\in S \colon x\ordleq y\}.
        \end{aligned}
    \end{equation}
\end{definition}
\begin{remark}
    Note that, by definition, an upper set is closed to upper closure.
\end{remark}
\begin{remark}
    For any~$S\in \powerset(P)$,~$\uparrow S$ is in fact an upper set.
    \begin{proof}
        Suppose~$y\in \uparrow S$ and~$z\in P$, and suppose $y\ordleq z$. By definition~$\exists x$ s.t.~$x\ordleq y$, meaning that~$x\ordleq z$. Thus,~$z\in \uparrow S$, as was to be shown.
    \end{proof}
\end{remark}

\begin{lemma}
    The upper closure operator~$\uparrow$ is a monotone map.
\end{lemma}
\begin{proof}
    Consider the posets~$\tup{\powerset(P),\subseteq}$ and~$\tup{\mathsf{U}P,\supseteq}$, and~$S_1,S_2\in \powerset(P)$. It is clear that given~$S_1\subseteq S_2$, one has
    \begin{equation*}
        \{y\in P\mid \exists x\in S_1\colon x\ordleq y\} \supseteq \{y\in P\mid \exists x\in S_2\colon x\ordleq y\}.
    \end{equation*}
    Therefore,~$\uparrow S_1\supseteq \ \uparrow S_2$, satisfing the monotonicity property for~$\uparrow$.
\end{proof}

\begin{lemma}
    \label{up-cl-inj-antichains}
    Let $A$ and $B$ be subsets of $P$ that are antichains. Then
    \begin{equation*}
        \uparrow A = \ \uparrow B \quad \Rightarrow \quad A = B.
    \end{equation*}
\end{lemma}

\begin{proof}
    First, let's fix an $a \in A$. From $\uparrow A = \ \uparrow B$ we know that in particular $A \subseteq \ \uparrow B$. This means that for our fixed $a \in A$ there exists $b \in B$ such that $b \leq a$. From $\uparrow A = \ \uparrow B$ it also follows that $B \subseteq \ \uparrow A$, so to the  $b \in B$ given above, there exists $a' \in A$ such that $a' \leq b$. In total, we have $a' \leq b \leq a$, and since $A$ is an antichain, we must have $a' = a$. This implies that $a' = b = a$. In particular, we have $a \in B$.

    The above shows that $A \subseteq B$. To show $B \subseteq A$, we can fix any $b \in B$ and repeat the above argumentation, now with the roles of $A$ and $B$ exchanged.
\end{proof}

In the example of the pizza recipes, first, consider the upper set of a single element of the poset, e.g.~$p_1=\tup{\unit[1]{\USD},\unit[2]{h}}$  (\cref{fig:upperclosure_1}).
\begin{figure}[h!]
    \begin{center}
        \includesag{70_upper_closure_1}
    \end{center}
    \caption{The upper closure of a singleton set of pizza recipes. \label{fig:upperclosure_1}}
\end{figure}
Then, consider the case of two elements, with~$p_2=\tup{\unit[2]{\USD},\unit[1]{h}}$ (\cref{fig:upperclosure_2}).

\begin{figure}[h!]
    \begin{center}
        \includesag{70_upper_closure_2}
    \end{center}
    \caption{The upper closure of a set of pizza recipes. \label{fig:upperclosure_2}}
\end{figure}
Note that the upper set of the subset formed by the two elements is the union of the upper sets of the single elements.

\begin{definition}[Lower closure operator]
    The \emph{lower closure operator} $\downarrow$ maps a subset to the smallest lower set that includes it, i.e.
    \begin{equation*}
        \begin{aligned}
            \downarrow\colon \powerset(P)&\to \mathsf{L}P\\
            S&\mapsto \{ y\in P \mid \exists x\in S \colon y\ordleq x\}.
        \end{aligned}
    \end{equation*}
\end{definition}

\begin{lemma}
    The lower closure operator $\downarrow$ is a monotone map.
\end{lemma}

\JL{The following proof is a bit redundant... we can say ``analogous to the case of the upper closure operation'' and/or invoke the principle of duality for posets.}
\begin{proof}
    Consider the posets~$\tup{\powerset(P),\subseteq}$ and~$\tup{\mathsf{L}P,\subseteq}$, and let~$S_1,S_2\in \powerset(P)$. It is clear that given~$S_1\subseteq S_2$, one has
    \begin{equation}
        \{y\in P\mid \exists x\in S_1\colon y\ordleq x\} \subseteq \{y\in P\mid \exists x\in S_2\colon y\ordleq x\}.
    \end{equation}
    Therefore,~$\uparrow S_1\subseteq \ \uparrow S_2$, satisfing the monotonicity property for~$\downarrow$.
\end{proof}



\begin{example}
    Consider the battery example of~\cref{ex:battery}, and the antichain given by the battery models~$a=\tup{\unit[10]{\USD},\unit[1]{kg}}$,~$b=\tup{\unit[20]{\USD},\unit[0.5]{kg}}$, and~$c=\tup{\unit[30]{\USD},\unit[0.25]{kg}}$ (\cref{fig:examplebatt}).
    The lower closure uperator~$\downarrow\{a,b,c\}$ represents all the battery models which, if existing, would dominate~$\{a,b,c\}$.

\end{example}
\begin{figure}[h!]
    \begin{center}
        \includesag{70_battery_1}
    \end{center}
    \caption{Battery example. From the left: antichain, upper closure, and lower closure.
    \label{fig:examplebatt}}
\end{figure}


\begin{definition}[Min]
    \label{def:Min}
    $\Min \colon \powerset(P) \to \antichains P$ is the map that sends a subset~$S$ of a poset to the minimal elements of that subset, i.e., those elements~$a \in S$ such that~$a \ordleq b$ for all~$b \in S$. In formulas:
    \begin{equation*}
        \begin{aligned}
            \Min \colon \powerset(P) &\to \antichains P\\
            S&\mapsto \{ x\in S\colon (y\in S)\wedge(y\ordleq x)\Rightarrow (x=y)\}.
        \end{aligned}
    \end{equation*}
    Note that~$\Min(S)$ could be empty.
\end{definition}

\begin{definition}[Max]
    \label{def:Max}
    $\Max \colon \powerset(P) \to \antichains P$ is the map that sends a subset~$S$ of a poset to the maximal elements of that subset, i.e., those elements~$a \in S$ such that~$a \ordgeq b$ for all~$b \in S$. In formulas:
    \begin{equation*}
        \begin{aligned}
            \Max \colon \powerset(P) &\to \antichains P\\
            S&\mapsto \{ x\in S\colon (y\in S)\wedge(y\ordgeq x)\Rightarrow (x=y)\}.
        \end{aligned}
    \end{equation*}
    Note that~$\Max(S)$ could be empty.
\end{definition}

\todo{This is a remnant of older times. To remove. }

\begin{lemma}
    Given a poset~$\tup{P,\ordleq}$,~$\tup{\antichains P,\ordleq_{\antichains P}}$ is a poset with
    \begin{equation}
        \label{eq:orderantichain}
        A\ordleq_{\antichains P} B \text{ if and only if } \uparrow A \supseteq \ \uparrow B.
    \end{equation}
    Furthermore, it is bounded by the top~$\top_{\antichains P}=\emptyset$ and the bottom~$\bot_{\antichains P}=\{\bot_P\}$.
\end{lemma}

\begin{proof}
    We need to show the poset properties (\cref{def:poset}).
    We can prove the following:

    \

    \begin{compactitem}
        \item \emph{Reflexivity}: From~$\tup{P,\ordleq}$ being a poset we know that
        \begin{equation}
            \begin{aligned}
                \{y\in P \mid \exists x\in A \colon x\ordleq y\} &\supseteq \{y\in P \mid \exists x\in A \colon x\ordleq y\},\\
                \uparrow A =\ \uparrow A
            \end{aligned}
        \end{equation}
        and hence~$A\ordleq_{\antichains P}A$.

        \

        \item \emph{Antisymmetry}: One has
        \begin{equation}
            \begin{aligned}
                \left(A\ordleq_{\antichains P} B\right) \wedge \left(B\ordleq_{\antichains P} A\right)
                &\Leftrightarrow \left(\uparrow A \supseteq \ \uparrow \ B\right) \wedge \left( \uparrow  B\supseteq \ \uparrow \ A\right)\\
                &\Leftrightarrow \ \uparrow A= \ \uparrow B \\
                & \Rightarrow A = B.
            \end{aligned}
        \end{equation}
        The last implication is by  \cref{lem:up-cl-inj-antichains}.

        \


        \item \emph{Transitivity}: One has
        \begin{equation}
            \begin{aligned}
                \left(A\ordleq_{\antichains P} B\right) \wedge \left(B\ordleq_{\antichains P} C\right)&\Leftrightarrow  \left(\uparrow A \supseteq \ \uparrow \ B\right) \wedge \left( \uparrow  B\supseteq \ \uparrow C\right)\\
                &\Imp \ \uparrow A\supseteq \ \uparrow C\\
                &\Imp A\ordleq_{\antichains P}C.
            \end{aligned}
        \end{equation}
        In order to find the top, we need to find the smallest set~$\top_{\antichains P}$ such that~$A\ordleq_{\antichains P} \top_{\antichains P}$ for all~$A\in \antichains P$. In other words, such that~$\uparrow A\supseteq \ \uparrow \top_{\antichains P}$ for all~$A\in \antichains P$. This is clearly~$\emptyset$, since~$\uparrow \emptyset = \emptyset$. Similarly, in order to find the bottom, we need to find the set~$\bot_{\antichains P}$ such that~$\bot_{\antichains P} \ordleq_{\antichains P} A$ for all~$A\in \antichains P$. In other words, such that~$\uparrow \bot_{\antichains P} \supseteq \ \uparrow A$ for all~$A\in \antichains P$. We obtain a bottom if we set $\bot_{\antichains P} := \top_P$, since $\top_P \supseteq A$ for all $A \subseteq P$, and hence, by monotonicity of $\uparrow$, we have in particular $\uparrow \top_P \supseteq \ \uparrow A$ for all antichains $A$.
    \end{compactitem}
\end{proof}



\begin{definition}[Downward closed set]
    \label{def:downward-closed-upperset}
    An upper set~$S$ is \emph{downward-closed} in a poset~$P$ if
    \begin{equation}
        S =\, \uparrow \Min S.
    \end{equation}
\end{definition}

\begin{remark}
    The set of downward-closed upper sets of~$P$ is denoted~$\underline{\mathsf{U}}P$.
\end{remark}

