% !TEX root = standalone.tex
\section{Mobility}

\todotext{We didn't define what is a graph yet.}
For a specific mode of transportation, say a car, we can define a graph
\begin{equation*}
    G_\mathrm{c}=\tup{V_\mathrm{c},A_\mathrm{c},s_\mathrm{c},t_\mathrm{c}},
\end{equation*}
where~$V_\mathrm{c}$ represents geographical locations which the car can reach and~$A_\mathrm{c}$ represents the paths it can take (e.g. roads). Similarly, we consider a graph~$G_\mathrm{s}=\tup{V_\mathrm{s},A_\mathrm{s},s_\mathrm{s},t_\mathrm{s}}$, representing the subway system of a city, with stations~$V_\mathrm{s}$ and subway lines going through paths~$A_\mathrm{s}$, and a graph $G_\mathrm{b}=\tup{V_\mathrm{b},A_\mathrm{b},s_\mathrm{b},t_\mathrm{b}}$, representing onboarding and offboarding at airports. In the following, we want to express intermodality: the phenomenon that someone might travel to a certain intermediate location in a car and then take the subway to reach their final destination.

By considering the graph~$G=(V,A,s,t)$ with~$V=V_\mathrm{c}\cup V_\mathrm{s}\cup V_\mathrm{b}$ and~$A=A_\mathrm{c}\cup A_\mathrm{s}\cup A_\mathrm{b}$, we obtain the desired intermodality graph. Graph~$G$ can be seen as a new category, with objects~$V$ and morphisms~$A$.
\begin{example}
    Consider the \Car category, describing your road trip in California, with
    \begin{equation*}
        V_\mathrm{c}=\{\textsf{SFO}_\mathrm{c},\textsf{S. Mateo},\textsf{Half Moon Bay},\textsf{SBP}_\mathrm{c},\textsf{Lake Balboa},\textsf{LAX}_\mathrm{c}\},
    \end{equation*}
    and arrows as in~\cref{fig:carcat}. The nodes represent typical touristic road-trip checkpoints in California and the arrows represent famous highways connecting them.

    \begin{figure}[h!]
        \begin{center}
            \includesag{30_carcategory}
        \end{center}
        \caption{The \Car category. \todographics{Change the layout so that it fits. Use points?}}
         \label{fig:carcat}
    \end{figure}

    Furthermore, consider the $\Flight$ category with $V_\mathrm{f}=\{\textsf{SFO}_\mathrm{f}, \textsf{SJC}, \textsf{SBP}_\mathrm{f}, \textsf{LAX}_\mathrm{f}\}$ and arrows as in~\cref{fig:flight}. The nodes represent airports in california and the arrows represent connections, offerend by specific flight companies.

    \begin{figure}[h!]
        \begin{center}
            \includesag{30_flight}
        \end{center}
        \caption{The $\Flight$ category. \label{fig:flight}}
    \end{figure}

    We then consider the $\Cat{board}$ category, with nodes
    \begin{equation*}
        V_\mathrm{b}=\{\textsf{SFO}_\mathrm{f},\textsf{SFO}_\mathrm{c},\textsf{SBP}_\mathrm{f},\textsf{SBP}_\mathrm{c},\textsf{LAX}_\mathrm{f},\textsf{LAX}_\mathrm{c}\}
    \end{equation*}
    and arrows as in~\cref{fig:boarding}. Nodes represent airports and airport parkings, and arrows represent the onboarding and offboarding paths one has to walk to get from the parkings to the airport and vice-versa.

    \begin{figure}[h!]
        \begin{center}
            \includesag{30_boarding}
        \end{center}
        \caption{The $\Cat{board}$ category. \label{fig:boarding}}
    \end{figure}

    The combination of the three, which we call the \emph{intermodal graph}, can be represented as a graph, with \textcolor{red}{red} arrows for the car network, \textcolor{blue}{blue} arrows for the flight network, \textcolor{darkgreen}{green} arrows for the boarding network, and black dashed arrows for intermodal morphisms, arising from composition of morphisms involving multiple modes (\cref{fig:intermodal}). Imagine that you are in the parking lot of \textsf{LAX} airport and you want to reach \textsf{S. Mateo}. From there, you will e.g. onboard to a \textsf{United} flight to \textsf{SFO}$_\mathrm{f}$, will then offboard reaching the parking lot \textsf{SFO}$_\mathrm{c}$, and drive on highway \textsf{US-101} reaching \textsf{S. Mateo}. This is intermodality.

    \begin{figure}[h!]
        \begin{center}
            \adjustbox{max width=\textwidth}
            {\includesag{30_intermodal}}
        \end{center}
        \caption{Intermodal graph. The dashed arrows represent intermodal morphisms, and we depict just one of them for simplicity. \label{fig:intermodal}
        }
    \end{figure}
\end{example}

The intermodal network category $\Cat{intermodal}$ is the free category on the graph illustrated in \cref{fig:intermodal}.

