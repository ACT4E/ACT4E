
\subsection{Thesis}

The thesis of this book is that most engineering fields would benefit 
from knowing and using the language of applied category theory to
address the design and analysis of complex systems.


\subsection{Systems and components}
\label{sec:systems-and-components}

What is a ``system''?

Here is a great quote\footnote{
    This quote is by Howard Aiken (1900-1973), creator of the MARK I computer,
    as quoted by Kenneth E. Iverson (1920-2004), creator of programming language APL,
    as quoted in~\cite{McIntyre1999Role}, but ultimately sourceless and probably apocryphal.
}:

\begin{quote}
    A system is composed of components;  \\
    a component is something you understand.
\end{quote}
 
The first part of the quote, ``A system is \emph{composed} of \emph{components}'', is plain as day as much as it is tautological. We could equally say: ``A system is \emph{partitioned} in \emph{parts}''.

The second part, ``a component is something you understand'', is where the insight lies: we call ``system'' what is too complex to be understood naturally by a human.

Haiken referred to computer engineering, but we find exactly the same sentiment expressed in other fields. In systems engineering, Leveson puts it as ``complexity can be defined as intellectual unmanageability''~\cite{leveson12engineering}.

We will be content of this anthropocentric and slightly circular definition of systems and
complexity: ``systems'' are ``complex'' and ``components'' are ``simple''.

Whether something is a complex system also depends on the task that we need to do with it. One
way to visualize this is to imagine a ``phenomenon'' as a high-dimensional object that we can see
from different angles~(\figref{aspects}). For each task, we have a different
projection. The decomposition of the system in components can be different according to the
task. For example, a system that might be easy to simulate could be very difficult to control.

\begin{figure}[h]
    \centering
\includegraphics[width=6cm]{plato.png}
\caption{\label{fig:aspects}
    Engineers live in a Plato's cave with multiple light sources.
    A certain phenomenon can be illuminated from different angles
    and look very different, very simple or complex.
}
\end{figure} 

\todo{Make better figure}

\subsection{What ACT can do for you}



\todo[inline]{describe how ACT can help}

\subsection{What ACT cannot do}

\todo[inline]{describe limitations of ACT}

\subsection{Book organization}

\todo[inline]{To write.}
