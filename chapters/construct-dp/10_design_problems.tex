% !TEX root = standalone.tex


\section{Creating design problems from catalogues [to change]}
\label{sec:spans}

\todo{We don't need this anymore because we define DPIs like this. However we can make this as a section that
says, DPIs are spans, and DPI serial composition is like composition of spans. }
\begin{ctdefinition}[Span]
    \label{def:span}
    Given a category~\CatC, a \emph{\iindex{span}} from an object~$x$ to an object~$y$ is a diagram of the form
    \begin{equation}
        \includesag{51_span}
    \end{equation}
    where~$z$ is some other object of~\CatC.
\end{ctdefinition}

\begin{example}
    Consider the category~\Berg, introduced in \cref{sec:trekking}. An example of span in this category is reported in~\cref{fig:exmountains}.
    \begin{figure}[h!]
        \begin{center}
            \includesag{130_mountains}
        \end{center}
        \caption{Swiss peaks can be thought of as a span in~\Trek. \label{fig:exmountains}}
    \end{figure}
    Recall that \textsf{Matterhorn Peak}, \textsf{Jungfrau Peak}, and \textsf{Pilatus Peak} are objects of~\Trek, and the arrows are morphisms in~\Trek (paths from one location to the other).
\end{example}

\begin{definition}[Catalogue]\label{def:catalogue}
    A \emph{catalogue} is a span in~\Pos.
    It thus consists of 3 posets~$\impsp$,~$\F{F}$,~$\R{R}$.
    We call them implementation space, functionality space, and requirements space, respectively. We need to define two maps~$\prov \colon I \to \F{F}$ (an implementation \textbf{prov}ides a functionality) and~$\req\colon I \to \R{R}$ (an implementation \textbf{req}uires resources):
    \begin{equation*}
        \includesag{130_catalogue}
    \end{equation*}
\end{definition}

\begin{definition}[Design problem induced by a catalogue]
    Every catalogue~$\tup{\impsp,\prov,\req}$ \emph{induces} a design problem of the form~$d\colon \F{F}\profto \R{R}$, with
    \begin{equation*}
        \begin{aligned}
            d\colon \F{F}\op \times \R{R}&\to \Bool\\
            \tup{\F{f}^*,\R{r}}&\mapsto \bigvee_{\imp\in \impsp}\left(\prov(\imp)\ordgeq_{\F{F}}\F{f} \right)\wedge \left( \req(i)\ordleq_{\R{R}}\R{r}\right)
        \end{aligned}
    \end{equation*}
\end{definition}
