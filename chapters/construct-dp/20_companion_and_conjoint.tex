% !TEX root = ../../ACT4E-full.tex


\section{Companions and conjoint}
We round out our discussion of~\DP by introducing two formulae for transforming monotone maps in~\Pos into design problems in~\DP. Each monotone map~$f$ can be transformed into two design problems, called its \emph{companion}~$\comp{f}$ and \emph{conjoint}~$\conj{f}$. Many of the design problems that we have introduced can be realized as companions and conjoints of appropriate monotone maps.

\begin{definition}[Companion and conjoint]
    \label{def:comp_conj}
    Let~$\cP$ and~$\cQ $ be posets, and suppose that~$f\colon\cP \toinPos \cQ $ is a monotone map. We define its \emph{companion} in~\DP, denoted~$\comp{f}\colon \F{\cP} \profto \R{\cQ}$,
    and its \emph{conjoint}, denoted~$\conj{f}\colon \F{\cQ} \profto \R{\cP}$ as
    \begin{equation}
        \comp{f}(\F{p}^*,\R{q})\coloneqq f(\F{p}) \ordleq_\cQ \R{q}
        \qquad\text{and}\qquad
        \conj{f}(\F{q}^*,\R{p})\coloneqq \F{q} \ordleq_\cQ f(\R{p}).
    \end{equation}
\end{definition}

\begin{lemma}
    \label{prop:comp_conj}
    Both the companion and conjoint constructions from \cref{def:comp_conj} are functorial from~\Pos to~\DP: they preserve identities and composition.
\end{lemma}
\begin{proof}
    We will show that the companion and conjoint are functors of the following forms:
    \begin{equation}
        \comp{(\cdot)}\colon\Pos\to\DP
        \qquad\text{and}\qquad
        \conj{(\cdot)}\colon\Pos\to\DP\op.
    \end{equation}
    First, we see that they send the identity monotone map~$f(p)=p$ to the unit~$\Unit{\cP }$ for any poset~$\cP $, because
    \begin{equation}
        \begin{aligned}
            \comp{\id}(\F{p_1}^*,\R{p_2})&= (\F{p_1} \ordleq_{\cP} \R{p_2})\\
            &=\conj{\id}(\F{p_1}^*,\R{p_2}).
        \end{aligned}
    \end{equation}
    Now suppose that~$f\colon  \cP \toinPos \cQ $ and~$g\colon \cQ \toinPos \cR$ are given. We first show that $\conj{g}\then\conj{f}=\conj{f\then g}$.
    For any $p\in P$ and $r\in R$, one has
    \begin{equation}
        \begin{aligned}
            \left(\conj{g}\then \conj{f}\right)(\F{r}^*,\R{p})
            &=\bigvee_{q\in Q} \conj{g}(\F{r}^*,\R{q})\wedge\conj{f}(\F{q}^*,\R{p})\\
            &=\bigvee_{q\in Q} (\F{r}\ordleq_R g(\R{q})) \wedge (\F{q}\ordleq_Q f(\R{p})) \\
            &= \F{r}\ordleq_R g(f(\R{p}))\\
            &=\left(\conj{f\then g}\right)(\F{r}^*,\R{p}).
        \end{aligned}
    \end{equation}
    Similarly, we can prove that~$\comp{f}\then \comp{g}=\comp{f\then g}$:
    \begin{equation}
        \begin{aligned}
            \left(\comp{f}\then \comp{g}\right)(\F{p}^*,\R{r})&=\bigvee_{q\in Q} \comp{f}(\F{p}^*,\R{q})\wedge\comp{g}(\F{q}^*,\R{r})\\
            &=\bigvee_{q\in Q} (f(\F{p})\ordleq_Q \R{q})\wedge (g(\F{q})\ordleq_R \R{r})\\
            &=g(f(\F{p}))\ordleq_R \R{r}\\
            &=\left(\comp{f\then g}\right)(\F{p}^*,\R{r}).
        \end{aligned}
    \end{equation}
\end{proof}

\begin{example}
    The identity design problem~$\id_A\colon \F{A} \profto \R{A}$ is the companion (and the conjoint) of the identity map~$\id_A'\colon A \toinPos A$. This is easy to check, as
    \begin{equation}
        \begin{aligned}
            \comp{\id}_A'(\F{a_1}^*,\R{a_2})&=\id_A'(\F{a_1})\ordleq \R{a_2}\\
            &=\F{a_1}\ordleq \R{a_2}\\
            &=\id_A(\F{a_1}^*,\R{a_2}).
        \end{aligned}
    \end{equation}
\end{example}

\begin{example}
    The coproduct injections~$\iota_A, \iota_B$ for design problems are the companions of the coproduct injections for the disjoint union.
\end{example}

\begin{example}
    The product projections~$\pi_A, \pi_B$ for design problems are the conjoints of the coproduct injections for the disjoint union.
\end{example}

\paragraph{Interesting implications}
Consider a poset $A$, which can be thought of as a map~$f\colon 1\to A$. By taking the companion of~$f$ one gets
\begin{equation}
    \begin{aligned}
        \comp{f}\colon \F{1}&\profto \R{A}\\
        \tup{\F{1},\R{a}}&\mapsto f(1)\ordleq \R{a}.
    \end{aligned}
\end{equation}
By taking the conjoint, one gets
\begin{equation}
    \begin{aligned}
        \conj{f}\colon \F{A}&\profto \R{1}\\
        \tup{\F{a}^*,\R{1}}&\mapsto \F{a}\ordleq f(\R{1}).
    \end{aligned}
\end{equation}
These two cases represent design problems with either \emph{constant} resources or constant, functionalities, respectively.


\section{Sum and intersection with companion and conjoints}

We can also re-define the sum~$\vee$ and intersection~$\wedge$ using companions and conjoints, which allows us to introduce some useful constructions.

\begin{definition}[Diagonal function]
    Define the \emph{diagonal function}~$\Delta_P\colon P \to P \times P$:
    \begin{equation}
        \begin{aligned}
            \Delta_P \colon P & \to P \times P, \\
            p & \mapsto \tup{p, p}.
        \end{aligned}
    \end{equation}
\end{definition}

\begin{definition}[Codiagonal function]
    Define the \emph{codiagonal function}~$\Diamond_P\colon P+P \to P $:
    \begin{equation}
        \begin{aligned}
            \Diamond_P \colon P + P & \to P,  \\
            \tup{1,p} & \mapsto p, \\
            \tup{2,p} & \mapsto p.
        \end{aligned}
    \end{equation}
\end{definition}

\noindent Using the diagonal function, \cref{lem:intersection} can be rewritten as the following lemma.

\begin{lemma}
    Given~$f, g\colon \F{A} \profto \R{B}$, we have:
    \begin{equation}
        f \vee g =  \conj{\Diamond}_A \then (f + g)\then \comp{\Diamond}_B.
    \end{equation}
\end{lemma}

\begin{proof}
    First of all, note that
    \begin{equation}
        \begin{aligned}
            \conj{\Diamond}_A\colon \F{A}&\profto \R{A}+\R{A}\\
            \tup{\F{a_1}^*,\tup{1,\R{a_2}}}&\mapsto \F{a_1}\ordleq \R{a_2}\\
            \tup{\F{a_1}^*,\tup{1,\R{a_3}}}&\mapsto \F{a_1}\ordleq \R{a_3}
        \end{aligned}
    \end{equation}
    and
    \begin{equation}
        \begin{aligned}
            \comp{\Diamond}_B\colon \F{B}+\F{B}&\profto \R{B}\\
            \tup{\tup{1,\F{b_1}}^*,\R{b_3}}&\mapsto \F{b_1}\ordleq \R{b_3}\\
            \tup{\tup{2,\F{b_2}}^*,\R{b_3}}&\mapsto \F{b_2}\ordleq \R{b_3}
        \end{aligned}
    \end{equation}
    We start by looking at~$\underbrace{\conj{\Diamond}_A\then (f+g)}_{\star}\colon \F{A} \profto \R{B}+\R{B}$.
    \begin{equation}
        \begin{aligned}
            \star (\tup{\F{a}^*,\R{b}})&=\bigvee_{a'\in A+A} \conj{\Diamond}_A(\tup{\F{a}^*,\R{a'}})\wedge (f+g)(\tup{\F{a'}^*,\R{b}})\\
            &=\left( \bigvee_{\tup{1,a'}\in A+A} (\F{a}\ordleq \R{a'})\wedge f(\F{a'}^*,\R{b}) \right)\vee \left( \bigvee_{\tup{2,a'}\in A+A} (\F{a}\ordleq \R{a'})\wedge g(\F{a'}^*,\R{b}) \right)\\
            &=f(\F{a}^*,\R{b}) \vee g(\F{a}^*,\R{b}).
        \end{aligned}
    \end{equation}

    Let's now look at~$\star \then \comp{\Diamond}_B\colon \F{A} \profto \R{B}$:
    \begin{equation}
        \begin{aligned}
            &(\star \then \comp{\Diamond}_B)(\F{a}^*,\R{b'})\\
            &=\bigvee_{b\in B+B} \star(\F{a}^*,\R{b})\wedge \comp{\Diamond}_B(\F{b}^*,\R{b'}) \\
            &=\left(\bigvee_{\tup{1,b}\in B+B} f(\F{a}^*,\R{b}) \wedge (\F{b}\ordleq \R{b'})\right) \vee
            \left(\bigvee_{\tup{2,b}\in B+B} g(\F{a}^*,\R{b}) \wedge (\F{b}\ordleq \R{b'})\right)\\
            &=f(\F{a}^*,\R{b'})\vee g(\F{a}^*,\R{b'}).
        \end{aligned}
    \end{equation}
\end{proof}

Similarly, using the codiagonal function, one can prove the following.
\begin{lemma}
    Given~$f, g\colon \F{A} \profto \R{B}$, we have:
    \begin{equation}
        f \wedge g = \comp{\Delta}_A \then(f + g) \then \conj{\Delta}_B.
    \end{equation}
\end{lemma}
\begin{proof}
    First, note that
    \begin{equation}
        \begin{aligned}
            \comp{\Delta}_A \colon \F{A}&\profto \R{A}\times \R{A}\\
            \tup{\F{a_1}^*,\tup{\R{a_2},\R{a_3}}}&\mapsto \Delta_A(\F{a_1})\leq \tup{\R{a_2},\R{a_3}}\\
            &= \tup{\F{a_1},\F{a_1}}\leq \tup{\R{a_2},\R{a_3}}\\
            &= (\F{a_1}\leq \R{a_2}) \wedge (\F{a_1}\leq \R{a_3}).
        \end{aligned}
    \end{equation}
    and
    \begin{equation}
        \begin{aligned}
            \conj{\Delta}_B \colon \F{B}\times \F{B}&\profto \R{B}\\
            \tup{\tup{\F{b_1},\F{b_2}}^*,\R{b_3}}&\mapsto \tup{\F{b_1},\F{b_2}}\leq \Delta_B(\R{b_3})\\
            &= (\F{b_1}\leq \R{b_3}) \wedge (\F{b_2}\leq \R{b_3}).
        \end{aligned}
    \end{equation}
    We start by looking at $\comp{\Delta}_A \then (f+g) \colon \F{A}\profto \R{B}+\R{B}$:
    \begin{equation}
        \begin{aligned}
            \left(\comp{\Delta}_A\then (f+g)\right)\left(\tup{\F{a}^*,\R{b}}\right)&=\bigvee_{}
        \end{aligned}
    \end{equation}
    \todo{Adjust signatures, have to find a good way to write it down}
\end{proof}
Unlike $\conj{\Diamond} = \mathsf{split}$ and $\comp{\Diamond} = \mathsf{fuse}$, $\comp{\Delta}$ and $\conj{\Delta}$ do not have an intuitive diagrammatic representation.
