% !TEX root = ../CategoricalCoDesign.tex
\todo{Add introduction on uncertainty}



\todo{Give two different forms of interval
orders}

\todo{Give interpretation as categorical constructs}


\begin{shaded*}
\begin{definition}[Twisted arrow category]
Given a category $\CatC$, we denote its \emph{twisted arrow category} by $\twisted{\CatC}$. This is a category which is composed of:
\begin{compactenum}
    \item \emph{Objects:} Arrows (morphisms) in $\CatC$.
    \item \emph{Morphisms:} A morphism between two arrows $f\colon A\to B$, $g\colon C\to D$ is given by the pair of arrows $\tup{p,q}$ such that the following diagram commutes:
    \begin{equation}
    \includesag{180_twistedarrow}
    \end{equation}
\end{compactenum} 
\end{definition}
\end{shaded*}

\begin{example}[Intervals]
Consider a poset $P$. The twisted arrow category $\twisted{P}$ is isomorphic to the set of nonempty \emph{intervals} $[a,b]=\{p \in P\mid a\leq_P p \leq_P b\}$. Note that $\twisted{P}$ is a poset as well, ordered by inclusion.
\end{example}
\begin{remark}
Recall \cref{sec:posetsarecats} and note that the map which sends a poset (a category) to its twisted arrow category is a functor, which sends objects of the poset
\end{remark}

\section{Monads}

\begin{shaded*}
\begin{definition}[Monad]
Let $\CatC$ be a category. A \emph{monad} on $\CatC$ consists of:
\begin{compactenum}
    \item A functor $T \colon \CatC \to \CatC$.
    \item A natural transformation $\eta \colon \id_\CatC \Rightarrow T$ called \emph{unit}.
    \item A natural transformation $\mu\colon TT\Imp T$ called \emph{composition} or \emph{multiplication}.
\end{compactenum}
The constituents must satisfy \emph{left and right unitality}
\begin{equation}
\includesag{55_monad_1}
\end{equation}
and \emph{associativity}
\begin{equation}
\includesag{55_monad_2}
\end{equation}
\end{definition}
\end{shaded*}


\todo{Move below to somewhere else}


\begin{shaded*}
\begin{definition}[Natural transformation]
Let $\Cat{C}$ and $\Cat{D}$ be categories, and let $F,G\colon \Cat{C}\to \Cat{D}$ be functors. To specify a \emph{natural transformation} $\alpha\colon F\to G$
\begin{equation}
\includesag{55_natural_1}
\end{equation}
one specifies for each obect $c\in \CatC$ a morphism $\alpha_c\colon F(c)\to G(c)$ in $\Cat{D}$, called the $c$\emph{-component} of $\alpha$. For every morphism $f\colon c\to d$ in $\Cat{C}$, these components must satisfy the \emph{naturality condition}:
\begin{equation}
    F(f)\then \alpha_d = \alpha_c\then G(f),
\end{equation}
i.e. the following diagram must commute:
\begin{equation}
\includesag{55_natural_2}
\end{equation}
\end{definition}

\begin{remark}[Natural isomorphism]
A natural transformation $\alpha\colon F\to G$ is called a \emph{natural isomorphism} if each component $\alpha_c$ is an isomorphism in $\CatD$.
\end{remark}
\end{shaded*}


\section{Using monads to understand uncertainty}

Take the $\mathsf{Unc}$ functor $\mathsf{Unc}\colon \DP\to \DP$ which 
\begin{compactenum}
    \item Maps an object $P$ in $\DP$ (poset) to its twisted arrow category $\twisted{P}$, representing a poset interval.
    \item Maps a morphism in $\DP$ $d\colon \F{F}\tickar \R{R}$ to $\tup{\low d,\upp d}$, where
    \begin{equation}
    \begin{aligned}
        \low d\colon \F{F_{\low}}&\tickar \R{R_{\low}},\\
        \upp d\colon \F{F_{\upp}}&\tickar \R{R_{\upp}},
    \end{aligned}
    \end{equation}
    and $\tup{\low d,\upp d}$ is a boolean profunctor (i.e., a morphism in $\DP$) of the form
    \begin{equation}
    \begin{aligned}
        \tup{\low d,\upp d}\colon \left(\F{F_{\low}}\times \F{F_{\upp}} \right)\op \times \left(\R{R_{\low}}\times \R{R_{\upp}} \right)&\toinPos \Bool\\
        \tup{\tup{\F{f_{\low}},\F{f_{\upp}}}^*,\tup{\R{r_{\low}},\R{r_{\upp}}}}&\mapsto \low d(\F{f_{\low}}^*,\R{r_{\low}})\wedge \upp d(\F{f_{\upp}}^*,\R{r_{\upp}})
    \end{aligned}
    \end{equation}
\end{compactenum}

\todo{write down better}
Is this a functor?

\begin{proof}
Consider two design problems $f\colon \F{A}\tickar \R{B}$ and $g\colon \F{B}\tickar \R{C}$. 

We know that
\begin{equation}
    \begin{aligned}
    \unc(f)&\colon \left( A_{\low}\times A_{\upp}\right)\op \times \left( B_{\low}\times B_{\upp}\right)\toinPos \Bool\\
    \unc(g)&\colon \left( B_{\low}\times B_{\upp}\right)\op \times \left( C_{\low}\times C_{\upp}\right)\toinPos \Bool.
    \end{aligned}
\end{equation}

We have
\begin{equation}
    \begin{aligned}
    &\left(\unc(f)\then \unc(g)\right) (\tup{\ubar{a},\bar{a}}^*,\tup{\ubar{c},\bar{c}})\\
    &=\bigvee_{\tup{\ubar{b},\bar{b}}\in B_{\low}\times B_{\upp}} \unc(f)(\tup{\ubar{a},\bar{a}}^*,\tup{\ubar{b},\bar{b}})\wedge \unc(g)(\tup{\ubar{b},\bar{b}}^*,\tup{\ubar{c},\bar{c}})\\
    &= \bigvee_{\tup{\ubar{b},\bar{b}}\in B_{\low}\times B_{\upp}} \low f(\ubar{a}^*,\ubar{b})\wedge \upp f(\bar{a}^*,\bar{b})\wedge \low g(\ubar{b}^*,\ubar{c})\wedge \upp g(\bar{b}^*,\bar{c})\\
    &=(\low f\then \low g)(\ubar{a}^*,\ubar{c})\wedge (\upp f\then \upp g)(\bar{a}^*,\bar{c})\\
    &=\low f\then g (\ubar{a}^*,\ubar{c})\wedge \upp f\then g(\bar{a}^*,\bar{c})\\
    &=\unc(f\then g)(\tup{\ubar{a},\bar{a}}^*,\tup{\ubar{c},\bar{c}}).
    \end{aligned}
\end{equation}
\end{proof}
