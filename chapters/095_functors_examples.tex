\section{Other examples of functors}

\subsection{The list functor}

\todo{to write}

\subsection{Planning as the search of a functor}
\begin{example}
  Recall the category \Berg introduced in \cref{sec:trekking} and define a category \Plans where objects are specific areas of the mountain and morphisms describing visiting order constraints, illustrated in \cref{fig:visiting_order_constraints}.

  \todo{Add more details, e.g. what is exactly a morphism? how to define composition?
  This simple transcription is not really understandable without looking at the lecture.}

  \begin{figure}[h]
    \begin{center}
      \includesag{095_plans}
    \end{center}
    \caption{Example of visiting order constraints on a mountain}
    \label{fig:visiting_order_constraints}
  \end{figure}
  A plan is a morphism in \Plans. When we talk about \textbf{planning} in this context, we refer to the action of finding a functor from \Plans to \Berg. Let's look at this in more detail. The objects of \Berg are tuples~$\tup{p,v}$, where~$p$ represent coordinates of a specific location and $v\in \mathbb{R}^3$ represents velocities. Morphisms in \Berg are paths that connect locations. For the sake of our planning, we can identify areas of the mountain as sets of locations. Such areas are, for instance, the \transmuted{lodge}, \transmuted{panoramic lake}, and the \transmuted{peak} (note that the \transmuted{peak} is a single object). Given some plans as in \cref{fig:visiting_order_constraints}, we want to find a map~$P$ which maps each object in \Plans (area of the mountain) to an object of \Berg (specific location and velocity). Similarly, it must map each morphism in \Plans (visiting order constraints) to a morphism in \Berg (specific paths). This is illustrated in \cref{fig:plans_functor}.
  \todo{add illustration}
\end{example}

\begin{figure}
  \centering
  \caption{\todo{to create}}
  \label{fig:plans_functor}
\end{figure}

