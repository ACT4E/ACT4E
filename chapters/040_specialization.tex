% !TEX root = ../CategoricalCoDesign.tex
\label{sec:specialization}


\subsection{Relational Databases}

A \emph{relational database} like PostgreSQL, MySQL, etc. presents 
the data to the user as relations. This does not necessarily mean
that the data is stored as tuples, as in the mathematical model, but 
rather that what the user can do is query and manipulate relations.
This conceptual model is now 50 years old.

\todo{add citation to paper \url{https://dl.acm.org/doi/10.1145/362384.362685}}.




Suppose we want to buy an electric stepper motor for a robot that we are building, and for this we consult a catalogue of electric stepper motors\footnote{See \href{https://www.pololu.com/category/87/stepper-motors}{pololu.com} for a standard catalogue of electric stepper motors.}. The catalogue might be organized as a large table, where on the left-hand side there is a column listing all available motors (identified with a model name), and the remaining columns correspond to different attributes that each of the models of motor might have, such as the name of the company that manufactures the motor, the size dimensions, the weight, the maximum power, the price, etc. A simple illustration is provided in \cref{tab:electric_motors}.
\begin{table}[h]
    \centering
    \adjustbox{max width=\textwidth}{%
    \begin{tabular}{c|c|c|c|c|c}
         Motor ID & Company& $\unit[\text{Size}]{[mm^3]}$ & \unit[Weight]{[g]} & \unit[Max Power]{[W]} & \unit[Cost]{[USD]} \\
         \hline
         \textsf{1204}&\textsf{SOYO} & 20 x 20 x 30& 60.0 &2.34 &19.95\\
         \textsf{1206}&\textsf{SOYO} &28 x 28 x 45& 140.0 &3.00 &19.95\\
         \textsf{1207}&\textsf{SOYO} &35 x 35 x 26& 130.0 &2.07 &12.95\\
         \textsf{2267}&\textsf{SOYO} &42 x 42 x 38& 285.0 &4.76 &16.95\\
         \textsf{2279}&\textsf{Sanyo Denki} &42 x 42 x 31.5& 165.0  &5.40 & 164.95\\
        \textsf{1478}&\textsf{SOYO} & 56.4 x 56.4 x 76& 1,000 & 8.96&49.95\\
        \textsf{2299}&\textsf{Sanyo Denki} & 50 x 50 x 16& 150.0 &5.90&59.95
    \end{tabular}%
    }
    \caption{A simplified catalogue of motors.}
    \label{tab:electric_motors}
\end{table}

\begin{comment}
\begin{table}[h]
    \centering
    \begin{tabular}{c|c|c|c|c|c}
         Motor ID & Company& Size & \unit[Weight]{[g]} & \unit[Max Power]{[W]} & \unit[price]{[USD]} \\
         \hline
         $\textsf{Model1}$&Company $\textsf{B}$ & 2 x 3 x 4& 10 & &259\\
         $\textsf{Model2}$&Company $\textsf{A}$ &2 x 3 x 4& 20 & &109\\
         $\textsf{Model3}$&Company $\textsf{B}$ &2 x 3 x 4& 5 & &124\\
         $\textsf{Model4}$&Company $\textsf{C}$ &2 x 3 x 4& 30 & &399\\
         $\textsf{Model5}$&Company $\textsf{A}$ &2 x 3 x 4& 45 & &245  \\
        $\textsf{Model6}$&Company $\textsf{D}$ & 2 x 3 x 4& 20 & &89\\
        $\textsf{Model7}$&Company $\textsf{B}$ & 2 x 3 x 4& 15 &&130
    \end{tabular}
    \caption{A simplified catalogue of motors.}
    \label{tab:electric_motors}
\end{table}
\end{comment}
One useful way to think of tables of data is in terms of sets and functions. In this example, we can consider the set
\begin{equation*}
M \coloneqq \{ \textsf{1204}, \textsf{1206}, \textsf{1207}, \textsf{2267}, \textsf{2279}, \textsf{1478}, \textsf{2299} \},
\end{equation*}
of models of motors, as well as the set~$C \coloneqq \{ \textsf{SOYO}, \textsf{Sanyo Denki}\}$ of manufacturing companies, the set $S$ of possible motor sizes, the set~$W$ of possible weights, the set $J$ of possible maximal powers, and the set~$P$ of possible prices. Each attribute of a motor may be thought of as a function from the set~$M$ to set of possible values for the given attribute. For example, there is a function~$\textsf{Company}\colon M \to C$ which maps each model to the corresponding company that manufactures it. So, according to Table \ref{tab:electric_motors}, we have \text{e.g.}~$\textsf{Company}(\textsf{1204}) = \textsf{SOYO}$, and~$\textsf{Company}(\textsf{2279}) = \textsf{Sanyo Denki}$, etc.

Note that in ``real life'', the catalogue of motors might not have seven entries, as in \cref{tab:electric_motors}, but has in fact hundreds of entries, and is implemented digitally as a database, \text{i.e.} a collection of interrelated tables. In this case, we will want to be able to search and filter the data based on various criteria. Many natural operations on tables and databases may be described simply in terms of operations with functions. We will use this setting as a way to introduce compositional aspects of working with sets and functions, and a preview of how this might be useful for thinking, in particular, about databases.

Sticking with \cref{tab:electric_motors}, suppose, for instance, that we want to consider only motors from Company~$\textsf{Sanyo Denki}$. In terms of the function
\begin{equation*}
\textsf{Company}\colon M \to C
\end{equation*}
this corresponds to the preimage~$\textsf{Company}^{-1}(\{ \textsf{Sanyo Denki} \}) = \{ \textsf{2279}, \textsf{2299}\}$, which is a subset of the set~$M$. Or, we may want to consider only motors which cost between 40 and 200 USD. In terms of the obvious function
\begin{equation*}
\textsf{Price}\colon M \to P,
\end{equation*}
this means we wish to restrict ourselves to the preimage
\begin{equation*}
\textsf{Price}^{-1}(\{ 49.95, 59.95, 164.95\}) = \{ \textsf{1478}, \textsf{2299}, \textsf{2279} \} \subseteq M.
\end{equation*}

Now suppose we wish to add a column to our table for ``volume'', because we may want to only consider motors that have, at most, a certain volume. For this we define a set~$V$ of possible volumes (let's take~$V = \mathbb{R}_{\geq 0}$, the non-negative real numbers), and define a function
\begin{equation*}
\begin{aligned}
\textsf{Multiply}\colon S &\to V\\
\tup{l, w, h} &\mapsto l \cdot w \cdot h,
\end{aligned}
\end{equation*}
which maps any size of motor to its corresponding volume by multiplying together the given numbers for length, width, and heigth.  Now we can compose this function with the function
\begin{equation*}
\textsf{Size}\colon M \to S
\end{equation*}
to obtain a function
\begin{equation*}
\textsf{Volume}\colon M \to V,
\end{equation*}
which defines a new column in our table. The composition of functions is usually written as~$\textsf{Volume} = \textsf{Multiply} \ \circ \ \textsf{Size}$, however we stick to our convention of writing~$\textsf{Volume} = \textsf{Size} \then \textsf{Multiply}$. Schematically, we can represent what we did as a diagram (\cref{fig:diagram_functions}).

%We call such a diagram \textbf{commutative}, because the composition $\mathsf{Size} \then \mathsf{Multiply}$ is equal to the function $\mathsf{Volume}$ (in fact, we defined $\textsf{Volume}$ this way).


\begin{figure}[h!]
\begin{center}
\includesag{40_dpcatfig_data_comm_diag}
\end{center}
\caption{A diagram of functions. \label{fig:diagram_functions}}
\end{figure}

We can interpret the arrows in this diagram as being part of a category, one where $M$,~$S$, and~$V$ are among the objects, and where the functions~$\textsf{Size}$, $\textsf{Multiply}$ and~$\textsf{Volume}$ are morphisms. We probably want to consider the other sets associated with our database as also part of this category, and the other functions which we defined so far, too. One idea might be to just include all the sets and functions that we've defined so far, as well as all possible compositions of those functions, and obtain a category (maybe call it $\textsf{Database}$?) in a way that is similar to how one can build a category from a graph (\cref{sec:catsfromgraphs}). This would be an option. However, we may want soon to add new sets and functions to our database framework, or think about new kinds of functions between them that we had not considered before. And we might not want to re-think each time precisely which category we are working with.


\subsection{The Category Set}

A helpful concept here is to think of our specific sets and functions as living in a very (very) large category which contains all possible sets as its objects and all possible functions as its morphisms. This category is know as the category of sets, and it is an important protagonist in category theory. We will denote it by $\textsf{Set}$. It is a short exercise to check that the following does indeed define a category.

\begin{shaded*}
\begin{definition}[Category of sets]
The category of sets~$\Set$ is defined by:
    \begin{compactenum}
    \item \emph{Objects}: all sets.
    \item \emph{Morphisms}: given sets~$X$ and~$Y$, the homset~$\Set(X,Y)$ is the set of all functions from~$X$ to~$Y$.
    \item \emph{Identity morphism}: given a set~$X$, its identity morphism~$\text{id}_X$ is
    is the identity function~$X \to X, \ \text{id}_X(x) = x$.
    \item \emph{Composition operation}: the composition operation is the usual composition of functions.
    \end{compactenum}
\end{definition}
\end{shaded*}

\todo{HOM NOTATION EVERYWHERE}
\todo{Put another definition, from category theory for the sciences (channel)}

We did say above, however, that we could build a category (let's call it \textsf{Database}) that only involves the sets that we are using for our database, and the functions between them that we are working with. What we would need for \textsf{Database} to be a category is that if any function is in \textsf{Database}, then also its sources and target sets are, and we would need that any composition of functions in \textsf{Database} is again in \textsf{Database}. (Also, we define the identity morphism for any set in \textsf{Database} to be the identity function on that set.) If these conditions are met, \textsf{Database} is what is called a \emph{subcategory} of the category of Sets. Here is the general definition. 


\subsection{Notion of subcategory}

\begin{shaded*}
\begin{definition}[Subcategory]
\label{def:subcategory}
	Given a category~$\Cat{C}$, a \emph{subcategory}~$\Cat{B}$ consists of a subcollection of the collection of objects and morphisms of~$\Cat{C}$ such that:
	\begin{enumerate}[(i)]
	\item If a morphism~$f \colon x\to y$ is in $\Cat{B}$, then so are the objects~$x$ and~$y$.
	\item If the morphisms~$f\colon x\to y$ and~$g\colon y\to z$ are in~$\Cat{B}$, then so is their composite~$f\then g\colon x\to z$.
	\item If~$x$ is in~$\Cat{B}$, then so is the identity morphism~$\text{id}_x$.
	\end{enumerate}
\end{definition}
\end{shaded*}


[INSERT SOME BRIDGING TEXT HERE]


\begin{lemma}
    The category~$\Cat{Set}$ is a subcategory (\cref{def:subcategory}) of~$\Cat{Rel}$.
\end{lemma}
\begin{proof}
	We need to prove the conditions presented in \cref{def:subcategory}.
	\begin{enumerate}[(i)]
	\item If a morphism of~$\Rel$ $f \colon X\to Y$ is in~$\Set$, then so are the objects~$X$ and~$Y$. Both~$\Rel$ and~$\Set$ have sets as objects. If a morphism~$f\colon X\to Y$ of~$\Rel$ is in~$\Set$, then a relation~$F\subseteq X\times Y$ between set~$X$ and set~$Y$ exists. This relation can be expressed in~$\Set$ as~$f\colon X\to Y$ and hence the objects~$X$ and~$Y$ exist.
	\item Two morphisms~$f\colon X\to Y$ and~$g\colon Y\to Z$ in~$\Rel$ are relations~$F\subseteq X\times Y$,~$G\subseteq Y\times Z$. If they are in~$\Set$, they can be written as functions~$f\colon X\to Y$ and~$g\colon Y\to Z$ and their composition is in~$\Set$ as well. 
	\item If an object of~$\Rel$ $X$ is in~$\Set$, then so is the identity morphism~$\text{Id}_X$. This is true: for every object~$X$ of~$\Set$ there exists the identity morphism~$\text{Id}_X\colon X\to X$. The diagonal morphism can be expressed as~$\text{Id}_X\colon X\to X$.
	\end{enumerate}
\end{proof}

\todo{FinSet is a subcategory of Set}

\subsection{Drawings}


Category of Infinite Drawings

Objects are maps from R2 -> Bool 

Morphisms are invertible transformations of the plane. 

Among these we can define subclasses:

0) invertible transforations
1) Affine invertible transformations 

x = Ax + b

A invertible

2) Rototranslations

3) Scalings 
3) Translations
3) Rotations




\subsection{Other examples of subcategories in engineering}

\AC{In enginedering very common to look at special types of functions.
And in most cases the properties are preserved by function composition. 
Thus, they form a category}


% R^n \to R^m


\todo{injective functions}
\todo{Continuous functions (topologies)}
\todo{Differentiable functions: Set to Manifolds}

\todo{Lipschitz bounded}
\todo{smooth}
\todo{cont diff, composition is compdiff}
%\begin{exercise}
%Check that $\Set$, as specified above, does in fact define a category.
%\end{exercise}


\subsubsection{Generalization outside of R}
Generalization to more general spaces. 
We used the fact thath R is:
\begin{itemize}
\item For defining continuous functions, we used the fact that R is topological space (minimum needed for defining a continuous function.). In fact, the real definitoin of continuous function is: 

ADD: continuous function

\item To define Lipschitz we needed the fact that R is a metric space 
\item For differentiable, smooth, you define this on Manifolds. Exists tangent space. 
\end{itemize}
     
Hence in general, the objects of these are different, so it's not really a relation of subcategory, that requires
the objects to be the same. However we will see later that you can generalize this notion using functors. 

\todo{functor F:C→D that is both injective on objects and a faithful functor.}




\subsection{Some counterexamples}

C: Endomorphisms of the plane

f: R^2 -> R^2 



D: Funzioni modulo quadrato


la composizione di due funzioni e' definita come

  f;g =  f( g(x) ) modulo Square 
  
  
  f; (g;h) =  f(g;h(x)) mod S) mod S
           =  f(g(h(x mod S) mod S)) mod S) mod S
           =  f(g(h(x mod S)))

  (f;g); h =   (f;g)( h(x) mod S) mod S
           =    f( g( h(x) mod S) mod S) mod S
           =    f( g( h(x) mod S)
           
           
 Z 