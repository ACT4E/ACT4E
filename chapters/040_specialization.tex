% !TEX root = ../ACT4E-full.tex
% !TEX root = ../ACT4E-full.tex
\label{sec:specialization}

\section{Databases, sets, functions}

\AC{I'm not sure about this example. I feel that if one is happy
with relational databases as a trivial example, they already know Intuitively
functions as morphisms (maybe because they know types).
If you don't know databases, then that topic cannot be used to introduce a simple
concept such as databases.
}

\JL{In the second sentence above, is ``that topic'' referring to ``functions as morphisms''? I'm also not sure if I understand the overall comment. In any case, if it's helpful: the original idea behind this database example was to do things in a way that is compatible with how databases are treated in Seven Sketches. (I'm not necessarily attached to keeping this example, this is just to explain the idea/intention.) I don't know anything about databases anyway : ) I just thought it probably works well if it is in line with the FQL way of seeing things.}

\AC{
I propose to have here the \textbf{Spreadsheet} category: objects are cells,
morphisms are the formulas. There is an object $1$ that can be the domain
for the constant value of cells.

}

We continue the discussion of \cref{subsec:relational-databases}.

In the particular case of tables with primary keys,
things are easier. In relational databases,
a table column is a primary key if the values of that column are guaranteed to be unique.

If the values in the column are unique, the column serves as the name of the row. In the table
above the motor ID serves as the primary key.

Consider a table with columns $P \times X_1 \times X_2 \times \cdots X_n$,
where $P$ is the primary key column. Then, given a key $p \in P$,
we can obtain the value in the other columns. We first find the unique
row with the key $p$, and then we read out the values.

Therefore, a table with columns $P \times X_1 \times X_2 \times \cdots X_n$
can be seen as a tuple $\tup{S, \{f_i\}_i}$:
\begin{itemize}
    \item A subset $A \subseteq P$ that gives us the available keys.
    \item $n$ read-out functions $f_i: P \to X_i$, each giving the
    corresponding value of the $i$-th attribute.
\end{itemize}

\JL{I don't see yet how the set $A$ comes into play.}
\begin{comment}
\begin{table}[h]
    \centering
    \begin{tabular}{c|c|c|c|c|c}
         Motor ID & Company& Size & \unit[Weight]{[g]} & \unit[Max Power]{[W]} & \unit[price]{[USD]} \\
         \hline
         $\textsf{Model1}$&Company $\textsf{B}$ & 2 x 3 x 4& 10 & &259\\
         $\textsf{Model2}$&Company $\textsf{A}$ &2 x 3 x 4& 20 & &109\\
         $\textsf{Model3}$&Company $\textsf{B}$ &2 x 3 x 4& 5 & &124\\
         $\textsf{Model4}$&Company $\textsf{C}$ &2 x 3 x 4& 30 & &399\\
         $\textsf{Model5}$&Company $\textsf{A}$ &2 x 3 x 4& 45 & &245  \\
        $\textsf{Model6}$&Company $\textsf{D}$ & 2 x 3 x 4& 20 & &89\\
        $\textsf{Model7}$&Company $\textsf{B}$ & 2 x 3 x 4& 15 &&130
    \end{tabular}
    \caption{A simplified catalogue of motors.}
    \label{tab:electric_motors}
\end{table}
\end{comment}


In this example, we can consider the primary key to be the set
\begin{equation*}
M \coloneqq \{ \transmuted{1204}, \transmuted{1206}, \transmuted{1207}, \transmuted{2267}, \transmuted{2279}, \transmuted{1478}, \transmuted{2299} \},
\end{equation*}
of models of motors. The other colums are given by the set
\begin{equation*}
    C \coloneqq \{ \transmuted{SOYO}, \transmuted{Sanyo Denki}\}
\end{equation*}
of manufacturing companies, the set~$S$ of possible motor sizes, the set~$W$ of possible weights, the set~$J$ of possible maximal powers, and the set~$P$ of possible prices. Each attribute of a motor may be thought of as a function from the set~$M$ to set of possible values for the given attribute. For example, there is a function~$\transmuted{Company}\colon M \to C$ which maps each model to the corresponding company that manufactures it. So, according to \cref{tab:electric_motors}, we have \text{e.g.}~$\transmuted{Company}(\transmuted{1204}) = \transmuted{SOYO}$, and~$\transmuted{Company}(\transmuted{2279}) = \transmuted{Sanyo Denki}$, etc.

Note that in ``real life'', the catalogue of motors might not have seven entries, as in \cref{tab:electric_motors}, but has in fact hundreds of entries, and is implemented digitally as a database, \text{i.e.} a collection of interrelated tables. In this case, we will want to be able to search and filter the data based on various criteria. Many natural operations on tables and databases may be described simply in terms of operations with functions. We will use this setting as a way to introduce compositional aspects of working with sets and functions, and a preview of how this might be useful for thinking, in particular, about databases.

Sticking with \cref{tab:electric_motors}, suppose, for instance, that we want to consider only motors from Company~$\transmuted{Sanyo Denki}$. In terms of the function
\begin{equation*}
\transmuted{Company}\colon M \to C
\end{equation*}
this corresponds to the preimage~$\transmuted{Company}^{-1}(\{ \transmuted{Sanyo Denki} \}) = \{ \transmuted{2279}, \transmuted{2299}\}$, which is a subset of the set~$M$. Or, we may want to consider only motors which cost between 40 and 200 USD. In terms of the obvious function
\begin{equation*}
\transmuted{Price}\colon M \to P,
\end{equation*}
this means we wish to restrict ourselves to the preimage
\begin{equation*}
\transmuted{Price}^{-1}(\{ 49.95, 59.95, 164.95\}) = \{ \transmuted{1478}, \transmuted{2299}, \transmuted{2279} \} \subseteq M.
\end{equation*}

Now suppose we wish to add a column to our table for ``volume'', because we may want to only consider motors that have, at most, a certain volume. For this we define a set~$V$ of possible volumes (let's take~$V = \mathbb{R}_{\geq 0}$, the non-negative real numbers), and define a function
\begin{equation*}
\begin{aligned}
\transmuted{Multiply}\colon S &\to V\\
\tup{l, w, h} &\mapsto l \cdot w \cdot h,
\end{aligned}
\end{equation*}
which maps any size of motor to its corresponding volume by multiplying together the given numbers for length, width, and heigth.  Now we can compose this function with the function
\begin{equation*}
\transmuted{Size}\colon M \to S
\end{equation*}
to obtain a function
\begin{equation*}
\transmuted{Volume}\colon M \to V,
\end{equation*}
which defines a new column in our table. The composition of functions is usually written as~$\transmuted{Volume} = \transmuted{Multiply} \circ \transmuted{Size}$, however we stick to our convention of writing~$\transmuted{Volume} = \transmuted{Size} \then \transmuted{Multiply}$. Schematically, we can represent what we did as a diagram (\cref{fig:diagram_functions}).

%We call such a diagram \textbf{commutative}, because the composition $\mathsf{Size} \then \mathsf{Multiply}$ is equal to the function $\mathsf{Volume}$ (in fact, we defined $\textsf{Volume}$ this way).


\begin{figure}[h!]
\begin{center}
\includesag{40_dpcatfig_data_comm_diag}
\end{center}
\caption{A diagram of functions. \label{fig:diagram_functions}}
\end{figure}

We can interpret the arrows in this diagram as being part of a category, where~$M$,~$S$, and~$V$ are among the objects, and where the functions~$\transmuted{Size}$, $\transmuted{Multiply}$ and~$\transmuted{Volume}$ are morphisms. We probably want to consider the other sets associated with our database as also part of this category, and the other functions which we defined so far, too. One idea might be to just include all the sets and functions that we've defined so far, as well as all possible compositions of those functions, and obtain a category, which we call $\Cat{Database}$, in a way that is similar to how one can build a category from a graph (\cref{sec:catsfromgraphs}). This would be an option. However, we may want soon to add new sets and functions to our database framework, or think about new kinds of functions between them that we had not considered before. And we might not want to re-think each time precisely which category we are working with.

\section{The Category \Set}

A helpful concept here is to think of our specific sets and functions as living in a very (very) large category which contains all possible sets as its objects and all possible functions as its morphisms. This category is known as the category of sets, and it is an important protagonist in category theory. We will denote it by \Set. It is a short exercise to check that the following does indeed define a category.

\begin{ctdefinition}[Category of sets]
The category of sets \Set is defined by:
    \begin{compactenum}
    \item \emph{Objects}: all sets.
    \item \emph{Morphisms}: given sets~$\Obja$ and~$\Objb$, the homset~$\Hom_\Set(\Obja,\Objb)$ is the set of all functions from~$\Obja$ to~$\Objb$.
    \item \emph{Identity morphism}: given a set~$\Obja$, its identity morphism~$\text{id}_\Obja$ is the identity function~$\Obja \to \Obja, \ \text{id}_\Obja(\obja) = \obja$.
    \item \emph{Composition operation}: the composition operation is the usual composition of functions.
    \end{compactenum}
\end{ctdefinition}

We did say above, however, that we could build a category~$\Cat{Database}$ which only involves the sets that we are using for our database, and the functions between them that we are working with. What we would need for~$\Cat{Database}$ to be a category is that if any function is in~$\Cat{Database}$, then also its sources and target sets are, and we would need that any composition of functions in~$\Cat{Database}$ is again in~$\Cat{Database}$. (Also, we define the identity morphism for any set in~$\Cat{Database}$ to be the identity function on that set.) If these conditions are met,~$\Cat{Database}$ is what is called a \emph{subcategory} of~\Set.


\section{Notion of subcategory}
\index{subcategory}
\begin{ctdefinition}[Subcategory]
\label{def:subcategory}
A subcategory~\CatD of a category~\CatC is a category for which:
\begin{compactenum}
    \item All the objects in $\Ob_\CatD$ are in $\Ob_\CatC$;
    \item For any objects $\Obja,\Objb\in \Ob_\CatD$, $\Hom_\CatD(\Obja,\Objb)\subseteq \Hom_\CatC(\Obja,\Objb)$;
    \item If $\Obja\in \Ob_\CatD$, then $\id_\Obja \in \Hom_\CatC(\Obja,\Obja)$ is in $\Hom_\CatD(\Obja,\Obja)$ and acts as its identity morphism;
    \item If~$\mora \colon \Obja \to \Objb$ and~$\morb\colon \Objb\to \Objc$ in $\CatD$, then the composite~$\mora \then \morb$ in $\CatC$ is in~$\CatD$ and represents the composite in~$\CatD$.
\end{compactenum}
\end{ctdefinition}

Two important examples of subcategory are the following.

\begin{example}[Finite Sets]
\FinSet is the category of finite sets and all functions between them. It is a subcategory of the category \Set of sets and functions. While an object $\Obja \in \Ob_\Set$ is a set with arbitrary cardinality, $\Ob_{\FinSet}$ only includes sets which have finitely many elements. Objects of \FinSet are in \Set, but the converse is not true. Furthermore, given $\Obja,\Objb\in \Ob_\FinSet$, we take $\Hom_{\FinSet}(\Obja,\Objb)=\Hom_{\Set}(\Obja,\Objb)$.
\end{example}


\begin{example}[\Set and \Rel]
    The category \Set is a subcategory of \Rel. To show this, we need to prove the conditions presented in \cref{def:subcategory}.
	\begin{enumerate}
	\item In both \Rel and \Set, the collection of objects is all sets.
	\item Given~$\Obja,\Objb\in \Ob_{\Set}$, we know that~$\Hom_{\Set}(\Obja,\Objb)\subseteq \Hom_{\Rel}(\Obja,\Objb)$, i.e., that all functions between sets~$\Obja,\Objb$ are a particular subset of all relations between~$\Obja,\Objb$.
	\item For each~$\Obja \in \Ob_{\Set}$, the identity relation~$\id_\Obja=\{\tup{\obja,\obja'}\in \Obja\times \Obja \mid \obja=\obja'\}$ corresponds to the identity function~$\id_\Obja \colon \Obja \to \Obja$ in \Set.
	\item Let~$R\subseteq \Obja\times \Objb$ and $S\subseteq \Objb \times \Objc$ be relations which are functions. We need to show that their composition in \Rel, expressed as~$R\then S\subseteq \Obja\times \Objc$, is again a function. This was proven in \cref{lemma:comprelfun}.
	\end{enumerate}

\end{example}


\section{Drawings}

\index{Drawings}
\begin{definition}[Drawings]
There exists a category~\Draw in which:
\begin{compactenum}
\item An object in~$\alpha\in \Ob_\Draw$ is a black-and-white drawing,
that is a function~$\alpha \colon \mathbb{R}^2 \to \Bool$.
\item A morphism in~$\Hom_{\Draw}(\alpha, \beta)$ between two drawings~$\alpha$ and~$\beta$ is an invertible map~$f\colon \mathbb{R}^2 \to \mathbb{R}^2$ such that~$\alpha(x) = \beta(f(x))$.
\item The identity function at any object~$\alpha$ is the identity map
on~$\mathbb{R}^2$.
\item Composition is given by function composition.
\end{compactenum}
\end{definition}

\begin{exercise}
Check whether just considering
    \begin{itemize}
        \item affine invertible transformations, or
        \item rototranslations, or
        \item scalings, or
        \item translations, or
        \item rotations,
    \end{itemize}
as morphisms forms a subcategory of~\Draw.
 \end{exercise}

\book{
\todo{Add a few figures here.}

We can now think about the different types of transformations.
\begin{compactitem}
    \item \textbf{Scalings.} Let $s,t\in \mathbb{R}$. Scalings can be represented as functions of the form
    \begin{equation*}
        \begin{aligned}
        f_{s,t}\colon \mathbb{R}^2&\to \mathbb{R}^2\\
        \tup{x,y}&\mapsto \tup{s x,t y},
        \end{aligned}
    \end{equation*}
    \item \textbf{Translations.}  Let $s,t\in \mathbb{R}$. Translations are functions of the form
    \begin{equation*}
        \begin{aligned}
        f_{s,t}\colon \mathbb{R}^2&\to \mathbb{R}^2\\
        \tup{x,y}&\mapsto \tup{x+s,y+t}.
        \end{aligned}
    \end{equation*}
    \item \textbf{Rotations.} Let $\theta \in [0,2\pi)$. Rotations are functions of the form
    \begin{equation*}
        \begin{aligned}
        f_{\theta}\colon \mathbb{R}^2&\to \mathbb{R}^2\\
        \tup{x,y}&\mapsto \tup{x\cos(\theta)+y\sin(\theta), y\cos(\theta)-x\sin(\theta)}.
        \end{aligned}
    \end{equation*}
    \item ...

\end{compactitem}}

\todo{Finish the above, show which ones are subcategories, etc.}


\section{Other examples of subcategories in engineering}

In engineering it is very common to look at specific types of functions; in many cases, the properties of a certain type of function are preserved by function composition, and so they form a category.





\subsubsection{\Injset forms a subcategory of \Set}
\begin{definition}[Injective  function]
\label{def:injective-function}
Let $\mora \colon \Obja\to \Objb$ be a function. The function $\mora$ is \emph{injective} if, for all $\obja,\obja'\in \Obja$ holds: $\mora(\obja)=\mora(\obja')\implies \obja=\obja'$.
\end{definition}


\begin{example}
We can define a category \Injset which has the same objects as \Set but restricts the morphisms to be \emph{injective functions}.
We want to show that \Injset is a subcategory of \Set. Composition and identity morphisms are defined as in \Set.

Since~$\Ob_{\Injset}=\Ob_{\Set}$, the first condition of \cref{def:subcategory} is satisfied. Injective functions are a particular type of functions: this satisfies the second condition. Given~$\Obja\in \Ob_{\Injset}$, the identity morphism~$\id_\Obja\in \Hom_{\Set}(\Obja,\Obja)$ corresponds to the identity morphism in~$\Hom_{\Injset}(\Obja,\Obja)$, i.e., the identity function is injective. This proves the third condition. To check the fourth condition, consider two morphisms~$\mora \in \Hom_\Set(\Obja,\Objb)$,~$\morb \in \Hom_\Set(\Objb,\Objc)$ such that~$\mora \in \Hom_\Injset(\Obja,\Objb)$ and~$\morb\in \Hom_\Injset(\Objb,\Objc)$. From the injectivity of~$\mora,\morb$, we know that given~$\obja,\obja'\in \Obja$, $\mora(\obja)=\mora(\obja') \Leftrightarrow \obja=\obja'$ and~$\objb,\objb'\in \Obja$,~$\morb(\objb)=\morb(\objb') \Leftrightarrow \objb=\objb'$. Furthermore, we have:
\begin{equation*}
    \begin{aligned}
    (\mora\then \morb)(\obja)=(\mora \then \morb)(\obja')&\implies \mora(\obja)=\mora(\obja')\\
    &\implies \obja=\obja',
    \end{aligned}
\end{equation*}
which proves the fourth condition of \cref{def:subcategory}, i.e. that the composition of injective functions is injective.
\end{example}



\book{
\begin{definition}[Continuous functions]
Let $f\colon \mathbb{R}\to \mathbb{R}$ be a function. We call $f$ \emph{continuous} at $c\in \mathbb{R}$ if $\lim_{x\to c}f(x)=f(c)$; $f$ is continuous over $\mathbb{R}$ if the condition is satisfied for all $c\in \mathbb{R}$.

\begin{example}
We can define a category $\Cat{Cont}$ which $\Ob_\Cat{Cont}=\mathbb{R}$ and in which the morphisms are given by continuous functions. Composition and identity are as in~\Set.  We want to show that $\Cat{Cont}$ is a subcategory of~\Set.
\todo{write down formally and use that composition of continuous is continuous}
\end{example}
\end{definition}
\todo{Continuous functions (topologies)}
\todo{Differentiable functions: Set to Manifolds}

\begin{definition}[Differentiable functions]
A function $f\colon U\subset \mathbb{R}\to \mathbb{R}$, defined on an open set $U$, is \emph{differentiable} at $a\in U$ if the derivative
\begin{equation}
    f'(a)=\lim_{h\to 0} \frac{f(a+h)-f(a)}{h}
\end{equation}
exists; $f$ is differentiable on $U$ if it is differentiable at every point of $U$.
\end{definition}

\begin{example}
the composition of differentiable functions is differentiable
\end{example}


\todo{Lipschitz bounded}

\begin{definition}[Lipschitz continuous function]
A real valued function $f\colon \mathbb{R}\to \mathbb{R}$ is called \emph{Lipschitz} continuous if there exists a positive real constant $\kappa$ such that, for all $x_1,x_2\in \mathbb{R}$:
\begin{equation}
    \vert f(x_1)-f(x_2)\vert \leq \kappa \vert x_1-x_2\vert.
\end{equation}
\end{definition}

\begin{example}
the composition of differentiable functions is differentiable
\end{example}

\todo{smooth}
\todo{cont diff, composition is compdiff}
%\begin{exercise}
%Check that \Set, as specified above, does in fact define a category.
%\end{exercise}


\subsubsection{Generalization outside of R}
Generalization to more general spaces.
We used the fact thath R is:
\begin{itemize}
\item For defining continuous functions, we used the fact that R is topological space (minimum needed for defining a continuous function.). In fact, the real definitoin of continuous function is:

\todo{ADD: continuous function}

\item To define Lipschitz we needed the fact that R is a metric space
\item For differentiable, smooth, you define this on Manifolds. Exists tangent space.
\end{itemize}

Hence in general, the objects of these are different, so it's not really a relation of subcategory, that requires
the objects to be the same. However we will see later that you can generalize this notion using functors.

\todo{functor F:C→D that is both injective on objects and a faithful functor.}
}


\section{Subcategories of Berg}
\label{sec:subcat_berg}
Recall the category \Berg presented in \cref{sec:trekking}. In the following, we want to give both a positive and a negative example of subcategories related to \Berg.


We first start our discussion by introducing an \emph{amateur} version of \Berg, called \Bergama, which only considers paths (morphisms) in \Berg, whose steepness does not exceed a critical value, say 1/2. Is \Bergama a subcategory of \Berg? Let's check the different conditions:
\begin{enumerate}
    \item The constraint on the maximum steepness restricts the objects which are acceptable in \Bergama via the identity morphisms of \Berg. Indeed, recall that given an object~$\tup{p,v}\in \Ob_{\Berg}$, the identity morphism is defined as~$1_{\tup{p,v}}=\tup{\gamma,0}$, with~$\gamma(0)=p$ and~$\dot{\gamma}(0)=v$. The steepness is computed via $v$. In particular, \Bergama will only contain objects whose identity morphisms do not exceed the steepness constraint, i.e.~$\Ob_\Bergama \subseteq \Ob_\Berg$.
    \item For~$A,B\in \Ob_{\Bergama}$, we know that paths satisfying the steepness constraint are specific paths in \Berg, i.e.~$\Hom_{\Bergama}\subseteq \Hom_{\Berg}$.
    \item The identity morphisms in \Berg which satisfy the steepness constraint are, by definition, in \Bergama and they act as identities there.
    \item Given two morphisms~$\mora,\morb$ which can be composed in \Bergama, the maximum steepness of their composition~$\mora \then \morb$ is given by:
    \begin{equation*}
        \text{MaxSteepness}(\mora \then \morb)=\max \ \{\text{MaxSteepness}(\mora),\text{MaxSteepness}(\morb)\}<1/2.
    \end{equation*}
\end{enumerate}

This shows that \Bergama is a subcategory of \Berg. What would an example of non-subcategory of \Berg be? Let's define a new category \Berglazy, which now discriminates morphisms based on the lengths of the paths they represent. For instance, assume that as amateur hikers, we don't want to consider morphisms which are more than \unit[1]{km} long. By concatenating two paths (morphisms) of length \unit[0.6]{km} in \Berglazy, the resulting composition will be \unit[1.2]{km}, violating the posed constraint and hence not being in \Berglazy. This violates the fourth property of \cref{def:subcategory}.



% \book{
% \section{Some counterexamples}

% C: Endomorphisms of the plane

% % f: R^2 -> R^2



% functions modulo squared


% composition of functions is given by

% %   f;g =  f( g(x) ) modulo Square


% %   f; (g;h) =  f(g;h(x)) mod S) mod S
% %           =  f(g(h(x mod S) mod S)) mod S) mod S
% %           =  f(g(h(x mod S)))

% %   (f;g); h =   (f;g)( h(x) mod S) mod S
% %           =    f( g( h(x) mod S) mod S) mod S
% %           =    f( g( h(x) mod S)


% %  Z
% }
