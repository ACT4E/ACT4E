% !TEX root = ../ACT4E-ready.tex
% \section{Posets with more structure}
In the following, we add some structure to the definition of a poset, by introducing \emph{monoidal posets} and \emph{lattices}.
\subsection{Monoidal posets}
\begin{definition}[Monoidal poset]
\label{def:monoidal_poset}
A \emph{monoidal structure} on a poset~$\tup{P,\ordleq}$ consists of:
\begin{compactenum}
    \item An element~$I\in P$, called \emph{monoidal unit}, and
    \item a function~$\otimes\colon P\times P\to P$, called the \emph{monoidal product}. Note that we write
    \begin{equation}
        \otimes(p_1,p_2)=p_1\otimes p_2, \quad p_1,p_2\in P.
    \end{equation}
\end{compactenum}
The constituents must satisfy the following properties:
\begin{compactenum}[(a)]
    \item \emph{Monotonicity}: For all~$p_1,p_2,q_1,q_2\in P$, if~$p_1\ordleq q_1$ and~$p_2\ordleq q_2$, then
    \begin{equation*}
    p_1\otimes p_2\ordleq q_1\otimes q_2.
    \end{equation*}
    \item \emph{Unitality}: For all~$p\in P$,~$I\otimes p=p$ and~$p\otimes I=p$.
    \item \emph{Associativity}: For all~$p,q,r\in P$, $(p\otimes q)\otimes r=p\otimes (q\otimes r)$.
\end{compactenum}
A poset equipped with a monoidal structure $\tup{P,\ordleq,I,\otimes}$ is called a \emph{monoidal poset}.
\end{definition}

\begin{example}
Consider the real numbers~$\mathbb{R}$ with the poset structure given the usual ordering. Consider 0 as monoidal unit and the operation~$+\colon \mathbb{R}\times \mathbb{R}\to \mathbb{R}$ as mononidal product. It is easy to see that the conditions of~\cref{def:monoidal_poset} are satisfied:
\begin{compactenum}[(a)]
    \item If~$p_1\ordleq p_2$ and $q_1\ordleq q_2$, it is true that~$p_1+p_2\ordleq q_1+q_2$,~$\forall p_1,p_2,q_1,q_2\in \mathbb{R}$.
    \item $0+p=p+0=0$,~$\forall p\in \mathbb{R}$.
    \item $(p+q)+r=p+(q+r)$,~$\forall p,q,r\in \mathbb{R}$.
\end{compactenum}
\end{example}
\subsection{Lattices}
\begin{definition}[Lattice]
\label{def:lattice}
A \emph{lattice} is a poset~$\tup{P, \ordleq}$ with some additional properties:
\begin{compactenum}
    \item Given two points~$p, q \in P$, it is always possible to define their least upper bound, called \emph{join}, and indicated as~$p \vee q$.
    \item Given two points~$p, q \in P$, it is always possible to define their greatest lower bound, called \emph{meet}, and indicated as~$p \wedge q$.
\end{compactenum}
\end{definition}

\begin{remark}[Bounded lattices]
If there is a least upper bound for the entire lattice~$A$, it is called
the \emph{top} ($\top$). If a greatest lower bound exists it is called the \emph{bottom} ($\bot$). If both a top and a bottom exist, we call the lattice \emph{bounded}, and denote it by~$\tup{A,\ordleq,\vee,\wedge,\bot,\top}$.
\end{remark}

\begin{example}
    In \cref{ex:hasseinclusion} we presented the poset arising from the power set~$\powerset(A)$ of a set~$A$ and ordered via subset inclusion. This is a lattice, bounded by~$A$ and by the empty set~$\emptyset$. Note that this lattice possesses two (dual) monoidal structures~$\tup{\powerset(A),\subseteq,\emptyset,\cup}$ and~$\tup{\powerset(A),\subseteq,A,\cap}$.
\end{example}

\begin{example}
Consider the set~$\{1,2,3,6\}$ ordered by divisibility. For instance, since 2 divides by 6, we have $2\ordleq 6$. This is a lattice. However, the set~$\{1,2,3\}$ ordered by divisibility is not, since 2 and 3 lack a meet (\cref{fig:exlattice}).
\begin{figure}[h!]
\begin{center}
\includesag{40_dpcatfig_exlattice}
\end{center}
\caption{Examples of a lattice and a non-lattice. \label{fig:exlattice}}
\end{figure}
\end{example}

\begin{lemma}
\label{lemma:u_bounded_lat}
$\UR$ is a bounded lattice (\cref{def:lattice}) with
\begin{equation}
    \{\UR,\leq_{\UR},\bot_{\UR},\top_{\UR},\vee_{\UR},\wedge_{\UR}\}=\{\UR,\supseteq,R,\emptyset,\cap,\cup\}.
\end{equation}
\begin{proof}
Consider the poset~$\tup{\UR,\supseteq}$ and~$P,Q\in \UR$. 
\begin{itemize}
    \item First, we need to show that~$P\cap Q\in \UR$. One has~$P \subseteq \UR$ and $Q\subseteq \UR$, meaning that by definition, if~$x\in P\cap Q$, we have~$x\in P \wedge x\in Q$. It follows that~$x\in \UR$ for all~$x\in P\cap Q$. Furthermore, we need to show that~$P\cap Q$ is the least upper bound of $P,Q$. Assume this is not true, i.e. there exists a~$T\in \UR$,~$T\neq P\cap Q$, such that~$P\supseteq T\supseteq P\cap Q$ and~$Q\supseteq T\supseteq P\cap Q$. Using the fact that intersection preserves inclusions, one has
\begin{equation}
\begin{aligned}
    P\cap Q &\supseteq T\cap T \supseteq P\cap Q\\
    P\cap Q &\supseteq T \supseteq P\cap Q\\
    T&= P\cap Q,
\end{aligned}
\end{equation}
which contradicts the assumption. Therefore,~$P\cap Q$ is the least upper bound of~$P,Q$.
\item Second, we need to show that~$P\cup Q\in \LF$. One has~$P\subseteq \UR$ and~$Q\subseteq \UR$, meaning that by definition, if~$x\in P\cup Q$, we have either~$x\in P$ or~$x\in Q$. If~$x\in P$, then~$x\in \UR$. If~$x\in Q$, then~$x\in \UR$. It follows that~$x\in \UR$ for all~$x\in P\cup Q$.  Furthermore, we need to show that~$P\cup Q$ is the greatest lower bound of~$P,Q$. Assume this is not true, i.e. there exists a~$T\in \UR$,~$T\neq P\cup Q$, such that~$P\cup Q\supseteq T\supseteq P$ and~$P\cup Q\supseteq T\supseteq Q$. Using the fact that union preserves inclusions, one has
\begin{equation}
    \begin{aligned}
    (P\cup Q)\cup (P\cup Q) &\supseteq T \cup T \supseteq P\cup Q\\
    P\cup Q &\supseteq T\supseteq P\cup Q\\
    T&=P\cup Q,
    \end{aligned}
\end{equation}
which contradicts the assumption.  Therefore,~$P\cup Q$ is the greatest lower bound of~$P,Q$.
\end{itemize}
We have therefore proved that~$\tup{\UR,\supseteq}$ is a lattice. To show that it is bounded, we notice that~$\emptyset \subseteq T$ for any~$T\in \UR$, meaning that~$\emptyset$ is the top. Furthermore, we notice that~$T\subseteq R$ for any~$T\in \UR$, meaning that~$R$ is a bottom. Therefore, the lattice is bounded.
\end{proof}
\end{lemma}

\begin{lemma}
$\LF$ is a bounded lattice (\cref{def:lattice}) with 
\begin{equation}
    \{\LF,\leq_{\LF},\bot_{\LF},\top_{\LF},\vee_{\LF},\wedge_{\LF}\}=\{\LF,\subseteq,\emptyset,F,\cup,\cap\}.
\end{equation}
\end{lemma}
\begin{proof}
Consider the poset~$\tup{\LF,\subseteq}$ and~$P,Q\in \LF$.
\begin{itemize}
    \item First, we need to show that~$P\cup Q\in \LF$. One has~$P \subseteq \LF$ and~$Q\subseteq \LF$, meaning that by definition, if~$x\in P\cup Q$, either~$x\in P$ or~$x\in Q$. If~$x\in P$, then~$x\in \LF$. If~$x\in Q$, then~$x\in \LF$. It follows that~$x\in \LF$ for all~$x\in P\cup Q$. Furthermore, we need to show that~$P\cup Q$ is the least upper bound of~$P,Q$. Assume this is not true, i.e. there exists a~$T\in \LF$,~$T\neq P\cup Q$, such that~$P\subseteq T\subseteq P\cup Q$ and~$Q\subseteq T\subseteq P\cup Q$. Using the fact that union preserves inclusions, one has
\begin{equation}
\begin{aligned}
    P\cup Q &\subseteq T\cup T \subseteq P\cup Q\\
    P\cup Q &\subseteq T \subseteq P\cup Q\\
    T&= P\cup Q,
\end{aligned}
\end{equation}
which contradicts the assumption. Therefore,~$P\cup Q$ is the least upper bound of~$P,Q$.
\item Second, we need to show that~$P\cap Q\in \LF$. One has~$P\subseteq \LF$ and~$Q\subseteq \LF$, meaning that by definition, if~$x\in P\cap Q$, we have~$x\in P\wedge x\in Q$, i.e.~$x\in \LF$, for all~$x\in P\cap Q$. Furthermore, we need to show that~$P\cap Q$ is the greatest lower bound of~$P,Q$. Assume this is not true, i.e. there exists a~$T\in \LF$,~$T\neq P\cap Q$, such that~$P\cap Q\subseteq T\subseteq P$ and~$P\cap Q\subseteq T\subseteq Q$. Using the fact that intersection preserves inclusions, oen has
\begin{equation}
    \begin{aligned}
    (P\cap Q)\cap (P\cap Q) &\subseteq T \cap T \subseteq P\cap Q\\
    P\cap Q &\subseteq T\subseteq P\cap Q\\
    T&=P\cap Q,
    \end{aligned}
\end{equation}
which contradicts the assumption.  Therefore,~$P\cap Q$ is the greatest lower bound of~$P,Q$.
\end{itemize}
We have therefore proved that~$\tup{\LF,\subseteq}$ is a lattice. To show that it is bounded, we notice that~$\emptyset \subseteq T$ for any~$T\in \LF$, meaning that~$\emptyset$ is the bottom. Furthermore, we notice that~$T\subseteq F$ for any~$T\in \LF$, meaning that~$F$ is a top. Therefore, the lattice is bounded. 
\end{proof}