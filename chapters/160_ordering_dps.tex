% !TEX root = ../CategoricalCoDesign.tex
We claimed that category theory is an efficient language for talking about \emph{structure}, and showed how the category~$\DP$ could accommodate all the basic operations required by a theory of formal engineering design. Here, we illustrate some of the applications and advantages of~$\DP$ in reasoning about and solving design problems, starting with the fact that that~$\DP$ is compact closed, which allows us to compose and reason about ``design problems of design problems''.

\begin{definition}[Order on~$\DP$]
\label{def:DP_loc_pos}

Suppose that~$A$ and~$B$ are posets, and that~$f,g \colon \F{A} \tickar \R{B}$ are design problems. We say that~$f$ \emph{implies}~$g$, denoted~$f \ordgeq_{\DP} g$, if~$f(\F{a}^*,\R{b}) \ordleq g(\F{a}^*,\R{b} )$ in~$\Bool$, for all~$\F{a} \in \F{A}$
and~$\R{b} \in \R{B}$. In other words, if the fact that~$f$ is feasible implies that~$g$ is feasible. We diagrammatically represent the relation~$f \ordgeq_{\DP} g$ as in~\cref{fig:dpimplies}.

\begin{figure}[h!]
\begin{center}
\includesag{60_relation}
\end{center}
\caption{The design problem~$f$ implies the design problem $g$. \label{fig:dpimplies}}
\end{figure}
\end{definition}

\begin{remark}
For any functionality-resource pair~$\F{A},\R{B}$, we denote by~$1_{\F{A},\R{B}}$ the design problem which is always feasible. We denote by~$0_{\F{A},\R{B}}$ the design problem which is never feasible, for any functionality-resource pair~$\F{A},\R{B}$.
\end{remark}
\begin{lemma}
\label{lemma:dpboundedlattice}
$\Hom_\DP(\F{A},\R{B})$ is a bounded lattice with union~$\vee$ as meet, intersection$\wedge$ as join, least upper bound~$1_{\F{A},\R{B}}$ and greatest lower bound~$0_{\F{A},\R{B}}$.
\end{lemma}

\begin{proof}
First of all, we need to prove that~$\Hom_\DP(\F{A},\R{B})$ is a poset. To prove this, we check the following:

\begin{itemize}
    \item \emph{Reflexivity}: Given~$f\in \Hom_\DP(\F{A},\R{B})$,~$f\ordgeq_\DP f$ is always true.
    \item \emph{Antisymmetry}: Given~$f,g\in \Hom_\DP(\F{A},\R{B})$, if~$f\ordgeq_\DP g$ and~$g\ordgeq_\DP f$, then~$f=g$.
    \item \emph{Transitivity}: Given~$f,g,h\in \Hom_\DP(\F{A},\R{B})$,~$f\ordgeq_\DP g$, and~$g\ordgeq_\DP h$, then~$f\ordgeq_\DP h$.
\end{itemize}
Therefore,~$\Hom_\DP$ is a poset. Furthermore, consider two design problems~$f,g\in \Hom_\DP(\F{A},\R{B})$. Their least upper bound (join) is~$f\wedge g$, since it is the least design problem implying both~$f$ and~$g$. Their greatest lower bound (meet), instead, is~$f\vee g$, since it is the greatest design problem implied by both~$f$ and~$g$. This proves that~$\Hom_\DP$ is a lattice. To prove that it is bounded, we identify the top element as~$1_{\F{A},\R{B}}$ (it implies all other design problems) and the bottom element as~$0_{\F{A},\R{B}}$ (it is implied by all the other design problems).
\end{proof}

