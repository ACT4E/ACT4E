% !TEX root = standalone.tex
\section{Monoidal functors}
\label{sec:monoidal-functors}
\begin{ctdefinition}[\iindex{Strong monoidal functor}]
  \label{def:strong-monoidal-functor}

  Let~$\tup{\CatC,\otimes_\CatC,I_\CatC}$ and~$\tup{\CatD,\otimes_\CatD,I_\CatD}$ be two monoidal categories. A \emph{strong monoidal functor} between \CatC and \CatD is given by:
  \begin{enumerate}
    \item A functor
    \begin{equation}
      F\colon \CatC\to \CatD;
    \end{equation}
    \item An isomorphism
    \begin{equation}
      \epsilon\colon I_\CatD\to F(I_\CatC);
    \end{equation}
    \item A natural isomorphism
    \begin{equation}
      \mu_{x,y}\colon F(x)\otimes_\CatD F(y) \to F(x\otimes_\CatC y),\quad \forall x,y\in \CatC,
    \end{equation}
  \end{enumerate}
  satisfying the following conditions:
  \begin{enumerate}
    \item[a)] \emph{Associativity}: For all objects~$x,y,z\in \CatC$, the following diagram commutes.
    \begin{equation}
      \includesag{120_natural_associativity}
    \end{equation}
    where~$a^\CatC$ and~$a^\CatD$ are called \emph{associators}.
    \item[b)] \emph{Unitality}: For all~$x\in \CatC$, the following diagrams commute:
    \begin{equation}
      \includesag{120_natural_unitality}
    \end{equation}
    where~$l^\CatC$ and~$r^\CatC$ represent the left and right \emph{unitors}.
  \end{enumerate}
\end{ctdefinition}


\todo{Do example here of a compiler having to be a functor to a pre-monoidal category.}
