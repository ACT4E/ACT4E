% !TEX root = ../ACT4E-full.tex


\section{The functionality-to-minimal-requirements map $\ftor$}

It is useful to also describe a design problem as a map from functionality
to resources or viceversa that abstracts over implementations.

A useful analogy is the state space representation \emph{vs} the
transfer function representation of a linear time-invariant system:
the state space representation is richer, but we only need the transfer
function to characterize the input-output response.

\todo{define the map from functionality to upper sets }

\begin{definition}
    \label{def:ftor}Given a DPI $\tup{ \funsp,\ressp,\impsp,\exc,\eval} $,
    define the map~$\ftor:\funsp\rightarrow\Aressp$ that associates
    to each functionality~$\funn$ the objective function of~\cref{prob:problem1},
    which is the set of minimal resources necessary to realize~$\funn$:
    \begin{eqnarray*}
        \ftor:\funsp & \rightarrow & \Aressp,\\
        \funn& \mapsto & \resMin\{\eval(\imp)\mid\left(\imp\in\impsp\right)\,\wedge\,\left(\funn\posleq\exc(\imp)\right)\}.
    \end{eqnarray*}
    If a certain functionality~$\funn$ is infeasible, then $\ftor(\funn)=\emptyset$.
\end{definition}
\captionsideleft{\label{fig:setup_h-1}}{\includegraphics[scale=0.33]{gmcdp_setup_h}}

\todo{add remark of when this might fail (antichains do not capture the uppersets)}


\begin{example}
    In the case of the motor design problem, the map~$\ftor$ assigns
    to each pair of $\left\langle \F{\text{speed}},\F{\text{torque}}\right\rangle $
    the achievable trade-off of \R{cost}, \R{mass}, and other resources~(\cref{fig:motor-trade-offs}).
    The antichains are depicted as continuous curves, but they could also
    be composed by a finite set of points.

    \captionsideleft{\label{fig:motor-trade-offs}}{\includegraphics[scale=0.33]{papers/arxiv_submission_v6/gmcdp_motor_tradeoffs.pdf}}
\end{example}

By construction, $\ftor$ is monotone (\defref{monotone}), which
means that
\[
    \funn_{1}\funleq\funn_{2}\quad\Rightarrow\quad\ftor(\funn_{1})\posleq_{\Aressp}\ftor(\funn_{2}),
\]
where~$\posleq_{\Aressp}$ is the order on antichains defined in
\cref{lem:antichains-are-poset}. Monotonicity of~$\ftor$ means that
if the functionality~$\funn$ is increased the antichain of resources
will go ``up'' in the poset of antichains~$\Aressp$, and at some
point it might reach the top of~$\Aressp$, which is the empty set,
meaning that the problem is not feasible.



\todo{describe equally the map $\rtof$}

