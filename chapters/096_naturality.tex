\begin{ctdefinition}[Natural transformation]
Let $\Cat{C}$ and $\Cat{D}$ be categories, and let $F,G\colon \Cat{C}\to \Cat{D}$ be functors. To specify a \emph{natural transformation} $\alpha\colon F\to G$
\begin{equation}
\includesag{55_natural_1}
\end{equation}
one specifies for each obect $c\in \CatC$ a morphism $\alpha_c\colon F(c)\to G(c)$ in $\Cat{D}$, called the $c$\emph{-component} of $\alpha$. For every morphism $f\colon c\to d$ in $\Cat{C}$, these components must satisfy the \emph{naturality condition}:
\begin{equation}
    F(f)\then \alpha_d = \alpha_c\then G(f),
\end{equation}
i.e. the following diagram must commute:
\begin{equation}
\includesag{55_natural_2}
\end{equation}
\end{ctdefinition}

\begin{ctdefinition}[Natural isomorphism]
\label{def:nat_iso}
A natural transformation $\alpha\colon F\to G$ is called a \emph{natural isomorphism} if each component $\alpha_c$ is an isomorphism in $\CatD$.
\end{ctdefinition}