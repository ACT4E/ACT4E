% !TEX root = ../ACT4E-full.tex

We have seen that functors are ``morphisms between categories''. Indeed, categories may be assembled into a category \CatC where the objects are categories and morphisms are functors. It turns out that there is an important third layer to this world of categories: there are also kinds of morphisms \emph{between} functors, and these are known as ``natural transformations''. To represent the three layers of structure involved in the world of categories, we will often draw diagrams like this, 
\begin{equation}
\includesag{3-layer-diagrams}
\end{equation}


where the points represent categories, the single arrows represent functors, and the double arrows represent natural transformations. 


\begin{ctdefinition}[Natural transformation]
Let $\Cat{C}$ and $\Cat{D}$ be categories, and let $F,G\colon \Cat{C}\to \Cat{D}$ be functors. A \emph{natural transformation} $\alpha\colon F\to G$ 
\begin{equation}
\includesag{55_natural_1}
\end{equation}
is defined by the following constituent data, satisfying the following condition. 

\underline{Data:} 
\begin{compactenum}
\item For each object $c\in \CatC$, a morphism $\alpha_c\colon F(c)\to G(c)$ in $\Cat{D}$, called the $c$\emph{-component} of $\alpha$. 
\end{compactenum}
\underline{Condition:} 
\begin{compactenum}
\item For every morphism $f\colon c\to d$ in $\Cat{C}$, the components of $\alpha$ must satisfy the \emph{naturality condition}
\begin{equation}
    F(f)\then \alpha_d = \alpha_c\then G(f),
\end{equation}
i.e. the following diagram must commute:
\begin{equation}
\includesag{55_natural_2}
\end{equation}
\end{compactenum}
\end{ctdefinition}

\begin{ctdefinition}[Natural isomorphism]
\label{def:nat_iso}
A natural transformation $\alpha\colon F\to G$ is called a \emph{natural isomorphism} if each component $\alpha_c$ is an isomorphism in $\CatD$.
\end{ctdefinition}

\begin{example}
Consider the category $\Cat{Vect}_{\mathbb{R}}$ whose objects are real vector spaces and whose morphisms are linear maps. (For convenience, in the following we sometimes omit reference to the ground field.) Recall that the \emph{dual} of a vector space $V$ is the vector space
$$
V^* := \Hom_{\Cat{Vect}}(V, \mathbb{R}), 
$$
\text{i.e.} the space of all linear maps from $V$ to $\mathbb{R}$. Also, recall the if $f: V \rightarrow W$ is a linear map, then its dual is a linear map $f^*: W^* \rightarrow V^*$. 

Applying the above duality construction twice to a vector space or a linear map gives their double dual. It turns out that this is a functorial operation. That is, there is a functor $\text{Double dual}: \Cat{Vect} \rightarrow \Cat{Vect}$ that maps every vector space and every linear map to its double dual. 

Furthermore, for any vector space $V$, there is a ``canonical'' or ``natural'' map $\alpha_V : V \rightarrow V^{**}$ defined by 
$$
\alpha_V(v)(l) = l(v) \quad \quad  v \in V, l \in V^*. 
$$
These form the components of a natural transformation from the identity functor on $\Cat{Vect}$ to the double dual functor. 
\begin{equation}
\includesag{nat-trafo-ddual}
\end{equation}



\end{example}



