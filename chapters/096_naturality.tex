% !TEX root = ../ACT4E-full.tex

We have seen that functors are ``morphisms between categories''. Indeed, categories may be assembled into a category \CatC where the objects are categories and morphisms are functors. It turns out that there is an important third layer to this world of categories: there are also kinds of morphisms \emph{between} functors, and these are known as ``natural transformations''. To represent the three layers of structure involved in the world of categories, we will often draw diagrams like this, 
\begin{equation}
\includesag{3-layer-diagrams}
%% https://q.uiver.app/?q=WzAsMyxbMCwwLCJcXGJ1bGxldCJdLFszLDAsIlxcYnVsbGV0Il0sWzYsMCwiXFxidWxsZXQiXSxbMCwxLCIiLDAseyJvZmZzZXQiOi0xLCJjdXJ2ZSI6LTV9XSxbMCwxLCIiLDIseyJvZmZzZXQiOjEsImN1cnZlIjo1fV0sWzAsMV0sWzEsMiwiIiwwLHsiY3VydmUiOi00fV0sWzEsMiwiIiwwLHsiY3VydmUiOjR9XSxbNSw0LCIiLDAseyJzaG9ydGVuIjp7InNvdXJjZSI6MjAsInRhcmdldCI6MjB9fV0sWzYsNywiIiwwLHsib2Zmc2V0IjotNCwiY3VydmUiOi0xLCJzaG9ydGVuIjp7InNvdXJjZSI6MjAsInRhcmdldCI6MjB9fV0sWzYsNywiIiwyLHsib2Zmc2V0Ijo0LCJjdXJ2ZSI6MSwic2hvcnRlbiI6eyJzb3VyY2UiOjIwLCJ0YXJnZXQiOjIwfX1dLFszLDUsIiIsMCx7Im9mZnNldCI6MiwiY3VydmUiOjEsInNob3J0ZW4iOnsic291cmNlIjoyMCwidGFyZ2V0IjoyMH19XSxbMyw1LCIiLDAseyJvZmZzZXQiOi0yLCJjdXJ2ZSI6LTEsInNob3J0ZW4iOnsic291cmNlIjoyMCwidGFyZ2V0IjoyMH19XSxbNiw3LCIiLDEseyJzaG9ydGVuIjp7InNvdXJjZSI6MjAsInRhcmdldCI6MjB9fV1d
%\begin{tikzcd}
%	\bullet &&& \bullet &&& \bullet
%	\arrow[""{name=0, anchor=center, inner sep=0}, shift left=1, bend left=40, from=1-1, to=1-4]
%	\arrow[""{name=1, anchor=center, inner sep=0}, shift right=1, bend right=40, from=1-1, to=1-4]
%	\arrow[""{name=2, anchor=center, inner sep=0}, from=1-1, to=1-4]
%	\arrow[""{name=3, anchor=center, inner sep=0}, bend left=40, from=1-4, to=1-7]
%	\arrow[""{name=4, anchor=center, inner sep=0}, bend right=40, from=1-4, to=1-7]
%	\arrow[shorten <=4pt, shorten >=4pt, Rightarrow, from=2, to=1]
%	\arrow[shift left=4, bend left=20, shorten <=7pt, shorten >=7pt, Rightarrow, from=3, to=4]
%	\arrow[shift right=4, bend right=20, shorten <=7pt, shorten >=7pt, Rightarrow, from=3, to=4]
%	\arrow[shift right=2, bend right=20, shorten <=4pt, shorten >=4pt, Rightarrow, from=0, to=2]
%	\arrow[shift left=2, bend left=20, shorten <=4pt, shorten >=4pt, Rightarrow, from=0, to=2]
%	\arrow[shorten <=6pt, shorten >=6pt, Rightarrow, from=3, to=4]
%\end{tikzcd}
\end{equation}


where the points represent categories, the single arrows represent functors, and the double arrows represent natural transformations. 


\begin{ctdefinition}[Natural transformation]
Let $\Cat{C}$ and $\Cat{D}$ be categories, and let $F,G\colon \Cat{C}\to \Cat{D}$ be functors. A \emph{natural transformation} $\alpha\colon F\to G$ 
\begin{equation}
\includesag{55_natural_1}
\end{equation}
is defined by the following constituent data, satisfying the following condition. 

\underline{Data:} 
\begin{compactenum}
\item For each object $c\in \CatC$ a morphism $\alpha_c\colon F(c)\to G(c)$ in $\Cat{D}$, called the $c$\emph{-component} of $\alpha$. 
\end{compactenum}
\underline{Condition:} 
\begin{compactenum}
\item For every morphism $f\colon c\to d$ in $\Cat{C}$, these components must satisfy the \emph{naturality condition}
\begin{equation}
    F(f)\then \alpha_d = \alpha_c\then G(f),
\end{equation}
i.e. the following diagram must commute:
\begin{equation}
\includesag{55_natural_2}
\end{equation}
\end{compactenum}
\end{ctdefinition}

\begin{ctdefinition}[Natural isomorphism]
\label{def:nat_iso}
A natural transformation $\alpha\colon F\to G$ is called a \emph{natural isomorphism} if each component $\alpha_c$ is an isomorphism in $\CatD$.
\end{ctdefinition}

\begin{example}
Consider the category $\Cat{Vect}_{\mathbb{R}}$ whose objects are real vector spaces and whose morphisms are linear maps. For convenience, in the following we sometimes omit reference to the ground field. Recall that the \emph{dual} of a vector space $V$ is the vector space
$$
V^* := \Hom_{\Cat{Vect}}(V, \mathbb{R}), 
$$
\text{i.e.} the space of all linear maps from $V$ to $\mathbb{R}$. 
\end{example}



