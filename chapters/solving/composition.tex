% !TEX root = standalone.tex
\section{Composition operators for design problems}

This section defines a handful of composition operators for design
problems. Later, \cref{sec:Decomposing2} will prove that any co-design
problem can be described in terms of a subset of these operators.

\label{sec:threeoperators}
\begin{definition}[$\dpseries$]
    \label{def:series-composition}The series composition of two DPIs
    $\dprob_{1}=\langle\funsp_{1},\ressp_{1},\impsp_{1},\prov_{1},\req_{1}\rangle$
    and $\dprob_{2}=\langle\funsp_{2},\ressp_{2},\impsp_{2},\prov_{2},$
    $\req_{2}\rangle$, for which~$\funsp_{2}=\ressp_{1}$, is
    \[
        \dpseries(\dprob_{1},\dprob_{2})\doteq\left\langle \funsp_{1},\ressp_{2},\impsp,\prov,\req\right\rangle ,
    \]
    where:
    \begin{eqnarray*}
        \impsp & = & \{\left\langle \imp_{1},\imp_{2}\right\rangle \in\impsp_{1}\times\impsp_{2}\mid\req_{1}(\imp_{1})\posleq_{\ressp_{1}}\prov_{2}(\imp_{2})\},\\
        \prov & : & \left\langle \imp_{1},\imp_{2}\right\rangle \mapsto\prov_{1}(\imp_{1}),\\
        \req & : & \left\langle \imp_{1},\imp_{2}\right\rangle \mapsto\req_{2}(\imp_{2}).
    \end{eqnarray*}
\end{definition}
\captionsideleft{\label{fig:composition-2}}{
    \includegraphics[scale=0.33]{gmcdp_series3}
}
\begin{definition}[$\dppar$]
    \label{def:parallel}
    The parallel composition of two DPIs $\dprob_{1}=\left\langle \funsp_{1},\ressp_{1},\impsp_{1},\prov_{1},\req_{1}\right\rangle $
    and $\dprob_{2}=\langle\funsp_{2},\ressp_{2},\impsp_{2},\prov_{2},$
    $\req_{2}\rangle$ is
    \[
        \dppar(\dprob_{1},\dprob_{2})\doteq\left\langle \funsp_{1}\times\funsp_{2},\ressp_{1}\times\ressp_{2},\impsp_{1}\times\impsp_{2},\prov,\req\right\rangle ,
    \]
    where:
    \begin{eqnarray}
        \prov & : & \left\langle \imp_{1},\imp_{2}\right\rangle \mapsto\left\langle \prov_{1}(\imp_{1}),\prov_{2}(\imp_{2})\right\rangle ,\label{eq:dppar-exec}\\
        \req & : & \left\langle \imp_{1},\imp_{2}\right\rangle \mapsto\left\langle \req_{1}(\imp_{1}),\req_{2}(\imp_{2})\right\rangle .\nonumber
    \end{eqnarray}

\end{definition}
\captionsideleft{\label{fig:dppar}}{
    \includegraphics[scale=0.33]{gmcdp_parallel2}
}


\begin{definition}[$\dploop$]
    \label{def:dp_loop}Suppose~$\dprob$ is a DPI with factored functionality
    space~$\funsp_{1}\times\ressp$:
    \[
        \dprob=\left\langle \funsp_{1}\times\ressp,\ressp,\impsp,\left\langle \prov_{1},\prov_{2}\right\rangle ,\req\right\rangle.
    \]
    Then we can define the DPI~$\dploop(\dprob)$ as
    \[
        \dploop(\dprob)\doteq\left\langle \funsp_{1},\ressp,\impsp',\prov_{1},\req\right\rangle ,
    \]
    where~$\impsp'\subseteq\impsp$ limits the implementations to those
    that respect the additional constraint~$\req(\imp)\posleq\prov_{2}(\imp)$:
    \[
        \impsp'=\{\imp\in\impsp:\req(\imp)\posleq\prov_{2}(\imp)\}.
    \]
    This is equivalent to ``closing a loop'' around~$\dprob$ with
    the constraint~$\fun_{2}\posgeq\res$~(\cref{fig:sloop}).
\end{definition}

\captionsideleft{\label{fig:sloop}\label{fig:sloop2}}{
    \includegraphics[scale=0.33]{gmcdp_sloop2}
}

The operator~$\dploop$ is asymmetric because it acts on a design
problem with 2 functionalities and 1 resources. We can define a symmetric
feedback operator $\dploopb$ as in \cref{fig:loop_general}, which
can be rewritten in terms of $\dploop$, using the construction in~\cref{fig:loop_general2}\emph{.}

\begin{figure}[h]
    \hspace*{\fill}
    \subfloat[\label{fig:loop_general}]{
        \includegraphics[scale=0.33]{gmcdp_loop_general}

    }
    \hspace*{\fill}
    \subfloat[\label{fig:loop_general2}]{
        \includegraphics[scale=0.33]{gmcdp_loop_general2}
    }
    \hspace*{\fill}
    \caption{A symmetric operator $\dploopb$ can be defined in terms of $\dploop$.}
\end{figure}

A ``co-product'' (see,~\eg\,\cite[Section 2.4]{spivak14category})
of two design problems is a design problem with the implementation
space~$\impsp=\impsp_{1}\sqcup\impsp_{2}$, and it represents the
exclusive choice between two possible alternative families of designs.
\begin{definition}[Co-product]
    \label{def:parallel-1}Given two DPIs with same functionality and
    resources $\dprob_{1}=\left\langle \funsp,\ressp,\impsp_{1},\prov_{1},\req_{1}\right\rangle $
    and $\dprob_{2}=\left\langle \funsp,\ressp,,\impsp_{2},\prov_{2},\req_{2}\right\rangle ,$
    define their co-product as
    \[
        \dprob_{1}\sqcup\dprob_{2}\doteq\left\langle \funsp,\ressp,\impsp_{1}\sqcup\impsp_{2},\prov,\req\right\rangle ,
    \]
    where
    \begin{eqnarray}
        \prov & : & \imp\mapsto\begin{cases}
                                   \prov_{1}(\imp), & \text{if }\imp\in\impsp_{1},\\
                                   \prov_{2}(\imp), & \text{if }\imp\in\impsp_{2},
        \end{cases}\label{eq:dppar-exec-1}\\
        \req & : & \imp\mapsto\begin{cases}
                                  \req_{1}(\imp), & \text{if }\imp\in\impsp_{1},\\
                                  \req_{2}(\imp), & \text{if }\imp\in\impsp_{2}.
        \end{cases}\nonumber
    \end{eqnarray}

\end{definition}


\captionsideleft{\label{fig:dpcoproduct}}{
    \includegraphics[scale=0.33]{gmcdp_coproduct}
}
