% !TEX root = ../ACT4E-ready.tex
\label{sec:productset}
\subsection{Products}

\book{


Is ``$X \times Y$'' the same as ``$Y \times X$''?
It depends on the context. Intuitively, for the categories of resources treated in this work, we would not make a distinction between ``having~$X$ and~$Y$'' and ``having~$Y$ and~$X$''.
However, there are contexts in which this is not valid. For example, if we are
using~$X \times Y$ to mean that we will have the resource~$X$ today, and the
resource~$Y$ tomorrow, then~$Y \times X$ would not be the same as~$X \times Y$. 

In the contexts in which this symmetry holds, we call the category ``symmetric monoidal'': we will talk about this more in detail later in the book.

The word ``symmetric'' is well-suited, because to say that the two objects are equivalent, or ``isomorphic'', we can postulate that we always have a way to get
``$X \times Y$'' starting from ``$Y \times X$'' and viceversa. Diagrammatically, this is depicted by the two arrows that connect them~(\cref{fig:e17}).

\begin{figure}[h!]
    \centering
    \includesag{30_dpcatfig_e17}
    \caption{Objects equivalence in the symmetric case. \label{fig:e17}}
\end{figure}


\JL{This opening bit seems confusing/misleading to me. I can edit/correct later; also I would move it to the section on monoidal products.}
}

We'll start off by recalling a familiar way of combining two sets $\obja$ and $\objb$.


\begin{definition}[Cartesian product of sets]
\label{def:cartesian-product}
   Given two sets~$A,B$, their \emph{cartesian product} is denoted~$A\times  B$
   and defined as 
   \begin{equation}
       A\times  B=\{ \tup{a,b}\mid a\in A\text{ and } b\in B\}.
   \end{equation}
\end{definition}

\begin{example}
Consider the sets $\obja = \{1, 2, 3, 4\}$ and $\objb = \{ \ast, \dagger \}$. 
We have
$$\obja \times \objb = \{ \tup{1, \dagger}, \tup{2, \dagger}, \tup{3, \dagger}, \tup{4, \dagger}, \tup{1, *}, \tup{2, *}, \tup{3, *}, \tup{4, *} \}.  $$
We can, however, also represent $\obja \times \objb$ in a way which highlights its structure more:
% https://q.uiver.app/?q=WzAsMTEsWzEsMSwiXFxsYW5nbGUgMSwgKiBcXHJhbmdsZSJdLFsyLDEsIlxcbGFuZ2xlIDIsICogXFxyYW5nbGUiXSxbMywxLCJcXGxhbmdsZSAzLCAqIFxccmFuZ2xlIl0sWzQsMSwiXFxsYW5nbGUgNCwgKiBcXHJhbmdsZSJdLFsyLDIsIlxcbGFuZ2xlIDIsIFxcZGFnZ2VyIFxccmFuZ2xlIl0sWzEsMiwiXFxsYW5nbGUgMSwgXFxkYWdnZXIgXFxyYW5nbGUiXSxbMywyLCJcXGxhbmdsZSAzLCBcXGRhZ2dlciBcXHJhbmdsZSJdLFs0LDIsIlxcbGFuZ2xlIDQsIFxcZGFnZ2VyIFxccmFuZ2xlIl0sWzAsM10sWzAsMF0sWzUsM10sWzgsOSwiWSIsMCx7InN0eWxlIjp7ImhlYWQiOnsibmFtZSI6Im5vbmUifX19XSxbOCwxMCwiWCIsMix7InN0eWxlIjp7ImhlYWQiOnsibmFtZSI6Im5vbmUifX19XV0=
\[\begin{tikzcd}
	{} \\
	& {\langle 1, * \rangle} & {\langle 2, * \rangle} & {\langle 3, * \rangle} & {\langle 4, * \rangle} \\
	& {\langle 1, \dagger \rangle} & {\langle 2, \dagger \rangle} & {\langle 3, \dagger \rangle} & {\langle 4, \dagger \rangle} \\
	{} &&&&& {}
	\arrow["Y", no head, from=4-1, to=1-1]
	\arrow["X"', no head, from=4-1, to=4-6]
\end{tikzcd}\]
In particular, the cartesian product comes naturally equipped with two projection maps $\pi_1$ and $\pi_2$ which map an element of
$\obja \times \objb$ to its first and second coordinate, respectively:  
$$ \pi_1(\tup{x,y}) =  x \quad \text{ and } \quad   \pi_2(\tup{x,y}) = y. $$

We will often depict the situation like this: 
% https://q.uiver.app/?q=WzAsNixbMCwwLCJYIl0sWzQsMCwiWSJdLFsyLDAsIlggXFx0aW1lcyBZIl0sWzIsMSwiXFxsYW5nbGUgeCwgeSBcXHJhbmdsZSJdLFs0LDEsInkiXSxbMCwxLCJ4Il0sWzIsMCwiXFxwaV8xIiwyXSxbMiwxLCJcXHBpXzIiXSxbMyw0LCIiLDIseyJzdHlsZSI6eyJ0YWlsIjp7Im5hbWUiOiJtYXBzIHRvIn19fV0sWzMsNSwiIiwwLHsic3R5bGUiOnsidGFpbCI6eyJuYW1lIjoibWFwcyB0byJ9fX1dXQ==
\[\begin{tikzcd}
	X && {X \times Y} && Y \\
	x && {\langle x, y \rangle} && y
	\arrow["{\pi_1}"', from=1-3, to=1-1]
	\arrow["{\pi_2}", from=1-3, to=1-5]
	\arrow[maps to, from=2-3, to=2-5]
	\arrow[maps to, from=2-3, to=2-1]
\end{tikzcd}\]

\end{example}

%\begin{example}
%Consider the sets~$\{\diamond,\star\}$ and $\{\dagger, \ast\}$. Their product can be represented as in \cref{fig:cartesian-product}.
%\begin{figure}[h!]
%    \centering
%    \includesag{50_cartesian_product}
%    \caption{Example of cartesian product of two sets.\label{fig:cartesian-product}}
%\end{figure}
%\end{example}

Given the Cartesian product of two sets, we can define two \emph{projection maps}~$\pi_1$ and~$\pi_2$, which, given an element of the product, will return the first or the second element, respectively:
\begin{equation}
\begin{aligned}
    \pi_1(\tup{a,b}) &\coloneqq a,\\
    \pi_2(\tup{a,b}) &\coloneqq b.
\end{aligned}
\end{equation}

In our case, we would be able to say that, given both~$\transmuted{resource}_1$ and~$\textsf{resource}_2$ together, we can recover~$\transmuted{resource}_1$ and~$\transmuted{resource}_2$ separately~(\cref{fig:resource-product}).

\begin{figure}[h!]
    \centering
    \includesag{30_recover}
    \caption{Two projection maps. \label{fig:resource-product}}
\end{figure}


Later, we will see other examples of products, such as products of partially ordered sets.
All these products are an instance of a more general concept of product that exists in category
theory.
 

\begin{ctdefinition}[Categorical Product]
Let~$\CatC$ be a category and let~$A,B\in \Ob_{\CatC}$ be objects. The \emph{product} of~$A$ and~$B$ is:
\begin{compactenum}
    \item an object~$A \times B \in \Ob_{\CatC}$, together with
    \item two \emph{projection morphisms}~$\pi_A \colon A \times B \to A$ and~$\pi_B \colon A \times B \to B$,
\end{compactenum} 
such that, given any~$X \in \Ob_{\CatC}$ and morphisms~$f \colon X \to A, g \colon X \to B$, there exists a \emph{unique} morphism~$(\prodMap{f}{g}) \colon X \to A \times B$ such that~$f = (\prodMap{f}{g})\then \pi_A$ and~$g=(\prodMap{f}{g})\then \pi_B$. Diagrammatically:
\begin{equation}
\includesag{50_defproduct}
\end{equation}
\end{ctdefinition} 

The formulation, slightly technical, hints at uniqueness properties or ``universality''. The definition implies that if a product exists, then it is unique, in the sense that all the admissible products can be derived from each other. Therefore, when presented with a new category, one can guess what construction might be the product, and check the definition to make sure that it is indeed \emph{the} product.

\begin{example}
Suppose you are at an engineering conference in Switzerland, and there will be a hike as a group outing. The organizers have prepared snacks to go. Each participant can choose a food from~$X=\{a,b,c\}$ (think: apple, banana, carrot) and a drink from~$Y=\{w,t\}$ (think: water, tea). Let~$P$ denote the set of participants. The choice of snacks could be organized as depicted in \cref{fig:snacks_1}, i.e., each participant chooses a food, and chooses a drink. This can be described via maps~$f\colon P\to X$ and~$g\colon P\to Y$.

\begin{figure}[h!]
\begin{center}
\includesag{50_snacks_1}
\end{center}
\caption{Each participant chooses a food and a drink. \label{fig:snacks_1}}
\end{figure}


Alternatively, snacks could be prepackaged in all possible combinations of food and drink choices, represented as~$X\times Y$ (\cref{fig:snacks_2}).

\begin{figure}[h!]
\begin{center}
\includesag{50_snacks_2}
\end{center}
\caption{Each participant chooses a combination of food and a drink. \label{fig:snacks_2}}
\end{figure}

This can be described via the map~$\phi\colon P\to X\times Y$. Thanks to the definition of product, we can notice that the two approaches are essentially the same. The situation is reported in \cref{fig:snacks_3}.

\begin{figure}[h!]
\begin{center}
\includesag{50_snacks_3}
\end{center}
\caption{Choosing food and drink separately is essentially the same as choosing a combination of the two. \label{fig:snacks_3}}
\end{figure}

Indeed, given the maps~$f$ and~$g$, we can build a map~$\phi_{f,g}$:
\begin{equation*}
    \begin{aligned}
    \phi_{f,g}\colon P&\to X\times Y\\
    p&\mapsto \tup{f(p),g(p)}.
    \end{aligned}
\end{equation*}
Furthermore, given~$\phi_{f,g}$, one can recover~$f$ and~$g$:
\begin{equation*}
    f=\phi_{f,g}\then \pi_1 \text{ and }g=\phi_{f,g}\then \pi_2.
\end{equation*}
In other words, this means that the diagram in \cref{fig:snacks_3} is commutative.
\end{example}

\begin{example}
Let $m,n\in \mathbb{N}$, and draw an arrow $m\to n$ if $m$ divides $n$. For instance, 6 divides 12 and hence there is an arrow $6\to 12$. The product between any two $m,n\in \mathbb{N}$ in this category is given by the greatest common divisor. 
\end{example}

\begin{example}
Let's consider the ordered set~$\tup{\mathbb{R},\leq}$, where given $x_1,x_2\in \mathbb{R}$ we can draw an arrow~$x_1\to x_2$ if~$x_1\leq x_2$. By following the products's commutative diagram, we know that the product of~$x_1$ and~$x_2$ is a $z\in \mathbb{R}$ such that
\begin{compactitem}
\item $z\leq x_1$;
\item $z\leq x_2$;
\item For all~$x\in \mathbb{R}$ with~$x\leq x_1$ and~$x\leq x_2$, we have~$x\leq z$.
\end{compactitem}
In other words, the product of $x_1,x_2\in \mathbb{R}$ is given by~$\min\{x_1,x_2\}$, and is also called \emph{meet}.
\end{example}

\begin{example}
Let $S$ be a set, and $X,Y\subseteq S$ subsets. We can draw an arrow $X\to Y$ if $X\subseteq Y$. By following the product's commutative diagram, it is easy to see that the product of~$X$ and~$Y$ is given by~$X\cap Y$.
\end{example}


\begin{example}
Suppose that we are designing a vehicle, and we are thinking about choices of engine. Both electric engines and internal combustion engins can produce \transmuted{motion}, but each from a different source of energy. The electric engine uses \transmuted{electric energy}; the internal combustion engine uses \transmuted{gasoline}. The situation is depicted in \cref{fig:e16a}, using the interpretation of the arrows that we have introduced for engineering design components. Namely, the arrow from motion to gasoline represents the internal combustion engine, and its direction is to be read as follows: given the desired functionality $\mathsf{motion}$, $\mathsf{internal \ combustion \ engine}$ provides a way of getting it using $\mathsf{gasoline}$. The other arrow in the figure represents the component \transmuted{electric \ engine}, and is interpreted in a similar way. 


\begin{figure}[h!]
    \centering
    \includesag{30_dpcatfig_e14}
    \caption{Alternative ways to generate \transmuted{motion}. \label{fig:e16a}}
\end{figure}

We could also consider building a hybrid vehicle, where we can obtain \transmuted{motion} from \textbf{either} \transmuted{gasoline} \textbf{or} \transmuted{electric energy} (\cref{fig:e16b}).

\begin{figure}[h!]
    \centering
    \includesag{30_dpcatfig_e15}
    \caption{We can generate \transmuted{motion} from either \transmuted{gasoline} or \transmuted{electric} \transmuted{energy}. \label{fig:e16b}}
\end{figure}
\end{example}


\begin{example}
Suppose that as a manufacturer, you want to label your products with
\begin{compactitem}
\item A production date (8-digit code), and
\item a model number (4-digit code).
\end{compactitem}
Instead of two separate labels, you can make one:
\begin{equation*}
    \underbrace{20210115}_{\text{date}}\underbrace{5900}_{\text{model number}}
\end{equation*}
We call this single label the \emph{product code}. We can see this diagrammatically, by considering the set~$Z$ of all product codes:
\begin{center}
    \begin{tikzcd}[column sep=large]
    &P\arrow{dr}\arrow{dl}\arrow[d,dashed]&\\
    X&Z\arrow{l}{\text{first 8}}[swap]{\pi_1}\arrow{r}{\pi_2}[swap]{\text{ last 4}}&Y
    \end{tikzcd}
\end{center}
The set~$Z$, together with the maps~$\pi_1\colon Z\to X$, and~$\pi_2\colon Z\to Y$, will satisfy the definition of ``product of~$X$ and~$Y$''. However,~$Z$ is not precisely the cartesian product~
$X\times Y$: elements of~$Z$ are 12-digit codes, while elements of~$X\times Y$ are pairs~$\tup{x,y}$ where~$x$ is a 8-digit code and~$y$ is a 4-digit code. Since both~$Z$ and~$X\times Y$ satisfy the definition of product, they are isomorphic. In fact, there is a unique way such that the following diagram commutes:
\begin{center}
    \begin{tikzcd}[column sep=large, row sep=large]
    &Z\arrow{dl}[swap]{\text{first 8}}\arrow{dr}{\text{last 4}}\arrow[d,"\phi_{\text{first 8},\text{last 4}}" description]&\\
    X&X\times Y\arrow{r}\arrow{l}&Y
    \end{tikzcd}
\end{center}
where~$\phi_{\text{first 8},\text{last 4}}$ is given by:
\begin{equation*}
\begin{aligned}
    \phi_{\text{first 8},\text{last 4}}\colon Z&\to X\times Y\\
    202101155900&\mapsto \tup{20210115,5900}.
\end{aligned}
\end{equation*}
\end{example}


\book{

\begin{ctdefinition}[Product category]
Given two categories~$\CatC$ and~$\CatD$, one defines the \emph{product category}~$\CatC \times \CatD$ to be the category specified as follows:
\begin{compactenum}
    \item \emph{Objects}: Objects are pairs~$\tup{c,d}$, with~$c\in \CatC$ and~$d\in \CatD$.
    \item \emph{Morphisms}: Morphisms are pairs of morphisms~$\tup{f,g}\colon \tup{c,d}\to \tup{c',d'}$, with~$f\colon c\to c'$,~$g\colon d\to d'$.
    \item \emph{Composition of morphisms}: The composition of morphisms is given by composing each component of the pair separately, i.e.~$\tup{f,g}\then \tup{f',g'}=\tup{f\then f',g\then g'}$. 
\end{compactenum}
\end{ctdefinition}


\begin{example}
Consider two posets~$P,Q$ as categories. The product poset~$P\times Q$ (\cref{def:productposet}) is the product category of the two posetal categories.
\end{example}}

