% !TEX root = ../ACT4E-full.tex
\label{sec:productset}
\section{Products}

\book{


Is ``$X \times Y$'' the same as ``$Y \times X$''?
It depends on the context. Intuitively, for the categories of resources treated in this work, we would not make a distinction between ``having~$X$ and~$Y$'' and ``having~$Y$ and~$X$''.
However, there are contexts in which this is not valid. For example, if we are
using~$X \times Y$ to mean that we will have the resource~$X$ today, and the
resource~$Y$ tomorrow, then~$Y \times X$ would not be the same as~$X \times Y$.

In the contexts in which this symmetry holds, we call the category ``symmetric monoidal'': we will talk about this more in detail later in the book.

The word ``symmetric'' is well-suited, because to say that the two objects are equivalent, or ``isomorphic'', we can postulate that we always have a way to get
``$X \times Y$'' starting from ``$Y \times X$'' and viceversa. Diagrammatically, this is depicted by the two arrows that connect them~(\cref{fig:e17}).

\begin{figure}[h!]
    \centering
    \includesag{30_dpcatfig_e17}
    \caption{Objects equivalence in the symmetric case. \label{fig:e17}}
\end{figure}


\JL{This opening bit seems confusing/misleading to me. I can edit/correct later; also I would move it to the section on monoidal products.}
}

\AC{yes, move to the part on monoidal products, when talking about symmetry}
We'll start off by recalling a familiar way of combining two sets, $\Obja$ and $\Objb$.


\begin{definition}[Cartesian product of sets]
\label{def:cartesian-product}
   Given two sets~$\Obja,\Objb$, their \emph{cartesian product} is denoted~$\Obja \times  \Objb$
   and defined as
   \begin{equation}
       \Obja \times \Objb =\{ \tup{\obja,\objb}\mid \obja \in \Obja \text{ and } \objb \in \Objb\}.
   \end{equation}
\end{definition}

\begin{example}
Consider the sets $\Obja = \{1, 2, 3, 4\}$ and $\Objb = \{ \ast, \dagger \}$.
We have
$$\Obja \times \Objb = \{ \tup{1, \dagger}, \tup{2, \dagger}, \tup{3, \dagger}, \tup{4, \dagger}, \tup{1, *}, \tup{2, *}, \tup{3, *}, \tup{4, *} \}.  $$
We can, however, also represent $\Obja \times \Objb$ in a way which highlights its structure more:
% https://q.uiver.app/?q=WzAsMTEsWzEsMSwiXFxsYW5nbGUgMSwgKiBcXHJhbmdsZSJdLFsyLDEsIlxcbGFuZ2xlIDIsICogXFxyYW5nbGUiXSxbMywxLCJcXGxhbmdsZSAzLCAqIFxccmFuZ2xlIl0sWzQsMSwiXFxsYW5nbGUgNCwgKiBcXHJhbmdsZSJdLFsyLDIsIlxcbGFuZ2xlIDIsIFxcZGFnZ2VyIFxccmFuZ2xlIl0sWzEsMiwiXFxsYW5nbGUgMSwgXFxkYWdnZXIgXFxyYW5nbGUiXSxbMywyLCJcXGxhbmdsZSAzLCBcXGRhZ2dlciBcXHJhbmdsZSJdLFs0LDIsIlxcbGFuZ2xlIDQsIFxcZGFnZ2VyIFxccmFuZ2xlIl0sWzAsM10sWzAsMF0sWzUsM10sWzgsOSwiWSIsMCx7InN0eWxlIjp7ImhlYWQiOnsibmFtZSI6Im5vbmUifX19XSxbOCwxMCwiWCIsMix7InN0eWxlIjp7ImhlYWQiOnsibmFtZSI6Im5vbmUifX19XV0=
\[\begin{tikzcd}
	{} \\
	& {\langle 1, * \rangle} & {\langle 2, * \rangle} & {\langle 3, * \rangle} & {\langle 4, * \rangle} \\
	& {\langle 1, \dagger \rangle} & {\langle 2, \dagger \rangle} & {\langle 3, \dagger \rangle} & {\langle 4, \dagger \rangle} \\
	{} &&&&& {}
	\arrow["Y", no head, from=4-1, to=1-1]
	\arrow["X"', no head, from=4-1, to=4-6]
\end{tikzcd}\]
In particular, the cartesian product comes naturally equipped with two projection maps $\pi_1$ and $\pi_2$ which map an element of
$\Obja \times \Objb$ to its first and second coordinate, respectively:
$$ \pi_1(\tup{x,y}) =  x \quad \text{ and } \quad   \pi_2(\tup{x,y}) = y. $$

We will often depict the situation like this:
% https://q.uiver.app/?q=WzAsNixbMCwwLCJYIl0sWzQsMCwiWSJdLFsyLDAsIlggXFx0aW1lcyBZIl0sWzIsMSwiXFxsYW5nbGUgeCwgeSBcXHJhbmdsZSJdLFs0LDEsInkiXSxbMCwxLCJ4Il0sWzIsMCwiXFxwaV8xIiwyXSxbMiwxLCJcXHBpXzIiXSxbMyw0LCIiLDIseyJzdHlsZSI6eyJ0YWlsIjp7Im5hbWUiOiJtYXBzIHRvIn19fV0sWzMsNSwiIiwwLHsic3R5bGUiOnsidGFpbCI6eyJuYW1lIjoibWFwcyB0byJ9fX1dXQ==
\[\begin{tikzcd}
	X && {X \times Y} && Y \\
	x && {\langle x, y \rangle} && y
	\arrow["{\pi_1}"', from=1-3, to=1-1]
	\arrow["{\pi_2}", from=1-3, to=1-5]
	\arrow[maps to, from=2-3, to=2-5]
	\arrow[maps to, from=2-3, to=2-1]
\end{tikzcd}\]

\end{example}

%\begin{example}
%Consider the sets~$\{\diamond,\star\}$ and $\{\dagger, \ast\}$. Their product can be represented as in \cref{fig:cartesian-product}.
%\begin{figure}[h!]
%    \centering
%    \includesag{50_cartesian_product}
%    \caption{Example of cartesian product of two sets.\label{fig:cartesian-product}}
%\end{figure}
%\end{example}


%In our case, we would be able to say that, given both~$\transmuted{resource}_1$ and~$\textsf{resource}_2$ together, we can recover~$\transmuted{resource}_1$ and~$\transmuted{resource}_2$ separately~(\cref{fig:resource-product}).
%
%\begin{figure}[h!]
%    \centering
%    \includesag{30_recover}
%    \caption{Two projection maps. \label{fig:resource-product}}f
%\end{figure}


In this section, we will introduce the ``categorical product''. This notion generalizes the definition of a cartesian product of sets. We will see that the cartesian product exemplifies the categorical product when we are working within the category \Set, but that in other other categories, the categorical product can show itself quite differently!

To give an introduction to the ideas, we'll work through an example which involves the familiar cartesian product of sets, but we'll view  this example through eyeglasses which will highlight the ``categorical product'' essence of the cartesian product.

Our introductory example is as follows. Suppose you are at an engineering conference in Switzerland, and there will be a hike as a group outing. The organizers have prepared snacks to go. Each participant can choose a food from~$X=\{a,b,c\}$ (think: apple, banana, carrot) and a drink from~$Y=\{w,t\}$ (think: water, tea). Let~$T$ denote the set of participants. The choice of snacks could be organized as depicted in \cref{fig:snacks_1}, i.e., each participant chooses a food, and chooses a drink. This can be described via functions~$f\colon T \to X$ and~$g\colon  T \to Y$.

\begin{figure}[h!]
\begin{center}
\includesag{50_snacks_1}
\end{center}
\caption{Each participant chooses a food and a drink. \label{fig:snacks_1}}
\end{figure}



Alternatively, snacks could be pre-packaged in such a way as to allow all possible combinations of food and drink choices. This corresponds to~$X\times Y$. Then the choice participants make of which lunch package they'd like is described by a single function~$\phi\colon T\to X\times Y$, see \cref{fig:snacks_2}).

\begin{figure}[h!]
\begin{center}
\includesag{50_snacks_2}
\end{center}
\caption{Each participant chooses a combination of food and a drink. \label{fig:snacks_2}}
\end{figure}

Intuitively, the two situations (two choices separately, or one choice of a pre-packaged snack) are ``the same'' in a certain sense. In our models, we can make this precise. Specifically, if we start with the functions~$f$ and~$g$, we can use them to build the following function:
\begin{equation*}
    \begin{aligned}
    \phi_{f,g}\colon T&\to X\times Y\\
    s&\mapsto \tup{f(s),g(s)}.
    \end{aligned}
\end{equation*}
Furthermore, given~$\phi_{f,g}$, one can recover~$f$ and~$g$:
\begin{equation*}
    f=\phi_{f,g}\then \pi_1 \quad  \text{ and } \quad g=\phi_{f,g}\then \pi_2.
\end{equation*}
These two equations say that the diagram in \cref{fig:snacks_3} is commutative. The whole situation can be summarized thus: given a set $T$ and functions $f: T \rightarrow \Obja$ and $g: T \rightarrow \Objb$ as in \cref{fig:snacks_2}, there is a unique function $\phi_{f,g} : T \rightarrow \Obja \times \Objb$ such that the diagram \cref{fig:snacks_3} commutes. This is the general pattern for the definition of the categorical product, which we state now. It is probably helpful to read the definition together with the clarifying remarks that follow it.



\begin{figure}[h!]
\begin{center}
\includesag{50_snacks_3}
\end{center}
\caption{Choosing food and drink separately is essentially the same as choosing a combination of the two. \label{fig:snacks_3}}
\end{figure}






\begin{ctdefinition}[Categorical Product]
Let~\CatC be a category and let~$\Obja, \Objb \in \ObC$ be objects. The \emph{product} of~$\Obja$ and~$\Objb$ is defined by the following consituent data, satisfying the following condition. \\
\underline{Data:}
\begin{compactenum}
    \item an object~$\Objc \in \ObC$ (this is ``the product'' of $\Obja$  and $\Objb$);
    \item \emph{projection morphisms}~$\pi_1 \colon \Objc \to \Obja$ and~$\pi_2 \colon \Objc \to \Objb$,
\end{compactenum}
\underline{Condition:}
\begin{compactenum}
    \item For any~$T \in \ObC$ and any morphisms~$\mora \colon T \to \Obja, \morb \colon T \to \Objb$, there exists a \emph{unique} morphism~$\phi_{\mora,\morb} \colon T \to \Objc$ such that~$\mora = (\phi_{\mora,
\morb})\then \pi_1$ and~$\morb=(\phi_{\mora, \morb})\then \pi_2$.
\end{compactenum}
\end{ctdefinition}

\begin{remark}
Diagrammatically, the condition above states that the diagrams of this form commute:
\begin{equation}
\includesag{50_defproduct}
\end{equation}
\end{remark}

\begin{remark}\label{prod unique up to iso}
In the above definition, technically both $\Objc$ \emph{and} the projection morphisms  constitute the data of ``the product of $\Obja$ and $\Objb$''. However, for simplicity, we usually refer only to $Z$ as ``the product''. Furthermore, we will usually use the notation $\Obja \times \Objb$ to denote the product of $\Obja$ and $\Objb$, in place of $Z$. Similarly, we will usually write $\mora \times \morb$ in place of $\phi_{\mora, \morb}$. The reason we do not do this directly in the definition itself is the following. In general, for fixed~$\Obja$ and~$\Objb$, there may be several different objects~$\Objc$ (together with projection morphisms) that satisfy the definition of being ``the product of~$\Obja$ and~$\Objb$''. Thus, there is, technically, no such thing as ``\emph{the}'' (unique) product of~$\Obja$ and~$\Objb$. However, one can prove that any two candidates which satisfy the definition of being ``the product of~$\Obja$ and~$\Objb$'' will necessarily be isomorphic in a canonical manner. Thus, for simplicity, we will sometimes be slightly sloppy and speak of ``the product of~$\Obja$ and~$\Objb$'' as if it were unique. In many categories there is also indeed a choice for ``the product of~$\Obja$ and~$\Objb$'' that we are used to. For example, in the category \Set, given sets~$\Obja$ and~$\Objb$, the familiar choice for ``the product of~$\Obja$ and~$\Objb$'' is the cartesian product~$\Obja \times \Objb$''. However, other representatives of the product of~$\Obja$ and~$\Objb$ are possible! \cref{ex univ prop prod} illustrates this.
\end{remark}

\begin{remark}
The condition in the definition of the categorical product is know as the ``universal property of the product''. We will attempt to explain this naming. The stated condition involves the product~$\Objc$ of~$\Obja$ and~$\Objb$ \emph{interacting} with every possible choice of object~$T$ and every possible choice of morphisms~$\mora : T \to \Obja$ and~$\morb: T \to \Objb$. We think of the ambient category~\CatC as ``the universe'' (or the ``context''), and this condition states how the product must interact ``with the whole universe''.
We choose the letter ``$T$'' because we think of this as a ``test object'' (similar e.g. to how, in electrodynamics, a ``test charge'' is used to probe an electromagnetic field).
\end{remark}


\begin{example}\label{ex univ prop prod}
Suppose that as a manufacturer, you want to label your products with
\begin{compactitem}
\item A production date (8-digit code), and
\item a model number (4-digit code).
\end{compactitem}
Instead of two separate labels, you can make one
\begin{equation*}
  202101155900
\end{equation*}
where the first 8 digits represent a date, and the last 4 digits are a model number.
We call this single label the \emph{product code}. Let~$Z$ denote the set of all product codes, and consider the maps~$\pi_1\colon Z\to X$, and~$\pi_2\colon Z\to Y$ which, respectively, map a 12-digit product code to its first 8 digits and its last 4 digits. One may check that $Z$, together with the map $\pi_1$ and $\pi_2$, will satisfy the definition of ``the product of~$X$ and~$Y$''.


\begin{center}
    \begin{tikzcd}[column sep=large]
    &T\arrow{dr}\arrow{dl}\arrow[d,dashed]&\\
    X&Z\arrow{l}{\text{first 8}}[swap]{\pi_1}\arrow{r}{\pi_2}[swap]{\text{ last 4}}&Y
    \end{tikzcd}
\end{center}

However,~$Z$ is not precisely the cartesian product of $X$ and $Y$ (which we will call $X\times Y$). The elements of~$Z$ are 12-digit codes, while elements of~$X\times Y$ are pairs~$\tup{x,y}$ where~$x$ is a 8-digit code and~$y$ is a 4-digit code. Since both~$Z$ and~$X\times Y$ satisfy the definition of categorical product, they must, by \cref{prod unique up to iso}, be isomorphic.
To see concretely what this isomorphism between them looks like, note that there is a unique map $\phi_{\text{first 8},\text{last 4}}$ making the following diagram commute:
\begin{center}
    \begin{tikzcd}[column sep=large, row sep=large]
    &Z\arrow{dl}[swap]{\text{first 8}}\arrow{dr}{\text{last 4}}\arrow[d,"\phi_{\text{first 8},\text{last 4}}" description]&\\
    X&X\times Y\arrow{r}\arrow{l}&Y
    \end{tikzcd}
\end{center}
Concretely,~$\phi_{\text{first 8},\text{last 4}}: Z \to X\times Y$ maps for instance
\begin{equation*}
\begin{aligned}
    202101155900&\mapsto \tup{20210115,5900}.
\end{aligned}
\end{equation*}
One can readily show that $\phi_{\text{first 8},\text{last 4}}$ is an isomorphism.
\end{example}


Now we will take a small tour to see some examples of how the categorical product may look in categories different than the category \Set of sets and functions.

\begin{example}
Let $m,n\in \natnumbers$, and draw an arrow $m\to n$ if $m$ divides $n$. For instance, 6 divides 12 and hence there is an arrow $6\to 12$. The product between any two $m,n\in \natnumbers$ in this category is given by the greatest common divisor.
\end{example}

\begin{example}
Let's consider the ordered set~$\tup{\reals,\leq}$, where given $x_1,x_2\in \reals$ we can draw an arrow~$x_1\to x_2$ if~$x_1\leq x_2$. By following the products's commutative diagram, we know that the product of~$x_1$ and~$x_2$ is a $z\in \reals$ such that
\begin{compactitem}
\item $z\leq x_1$;
\item $z\leq x_2$;
\item For all~$x\in \reals$ with~$x\leq x_1$ and~$x\leq x_2$, we have~$x\leq z$.
\end{compactitem}
In other words, the product of $x_1,x_2\in \reals$ is given by~$\min\{x_1,x_2\}$, and is also called \emph{meet}.
\end{example}

\begin{example}
\label{ex:subset_prod}
Let $S$ be a set, and $X,Y\subseteq S$ subsets. We can draw an arrow $X\to Y$ if $X\subseteq Y$. By following the product's commutative diagram, it is easy to see that the product of~$X$ and~$Y$ is given by~$X\cap Y$.
\end{example}


\begin{example}
Suppose that we are designing a vehicle, and we are thinking about choices of engine. Both electric engines and internal combustion engins can produce \transmuted{motion}, but each from a different source of energy. The electric engine uses \transmuted{electric energy}; the internal combustion engine uses \transmuted{gasoline}. The situation is depicted in \cref{fig:e16a}, using the interpretation of the arrows that we have introduced for engineering design components. Namely, the arrow from motion to gasoline represents the internal combustion engine, and its direction is to be read as follows: given the desired functionality $\mathsf{motion}$, $\mathsf{internal \ combustion \ engine}$ provides a way of getting it using $\mathsf{gasoline}$. The other arrow in the figure represents the component \transmuted{electric \ engine}, and is interpreted in a similar way.


\begin{figure}[h!]
    \centering
    \includesag{30_dpcatfig_e14}
    \caption{Alternative ways to generate \transmuted{motion}. \label{fig:e14}}
\end{figure}

We could also consider building a hybrid vehicle, where we can obtain \transmuted{motion} from \textbf{either} \transmuted{gasoline} \textbf{or} \transmuted{electric energy} (\cref{fig:e15}).

\begin{figure}[h!]
    \centering
    \includesag{30_dpcatfig_e15}
    \caption{We can generate \transmuted{motion} from either \transmuted{gasoline} or \transmuted{electric} \transmuted{energy}. \label{fig:e15}}
\end{figure}
\end{example}





\book{

\begin{ctdefinition}[Product category]
Given two categories~\CatC and~\CatD, one defines the \emph{product category}~$\CatC \times \CatD$ to be the category specified as follows:
\begin{compactenum}
    \item \emph{Objects}: Objects are pairs~$\tup{c,d}$, with~$c\in \CatC$ and~$d\in \CatD$.
    \item \emph{Morphisms}: Morphisms are pairs of morphisms~$\tup{f,g}\colon \tup{c,d}\to \tup{c',d'}$, with~$f\colon c\to c'$,~$g\colon d\to d'$.
    \item \emph{Composition of morphisms}: The composition of morphisms is given by composing each component of the pair separately, i.e.~$\tup{f,g}\then \tup{f',g'}=\tup{f\then f',g\then g'}$.
\end{compactenum}
\end{ctdefinition}


\begin{example}
Consider two posets~$P,Q$ as categories. The product poset~$P\times Q$ (\cref{def:productposet}) is the product category of the two posetal categories.
\end{example}}
