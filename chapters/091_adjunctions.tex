% !TEX root = ../ACT4E-jonathan.tex

\section{An example}

\JL{An example of a Galois connection between functionalities and resources could be treated here to motivate the adjunction discussion}





\section{Adjunctions: hom-set description}

In this section we give a definition of adjunction which can be viewed as an analogy with the following situation in linear algebra. Suppose $V$ and $W$ are finite-dimensional real vector spaces, equipped with inner products $(-, -)_V$ and $(-, -)_W$, respectively. The adjoint of a linear map $F: V \rightarrow W$ is a linear map $F^*: W \rightarrow V$ such that 
$$
(Fv, w)_W = (v, F^*w)_V \quad \quad \quad \forall \ v \in V, \ w \in W. 
$$




\begin{ctdefinition}[Adjunction, Version 1]\label{def-adj-iso}
  \label{def:cat-adjunction}
  Let \CatC and \CatD be categories. An \emph{\iindex{adjunction}} from \CatC to \CatD is given by the following data, satisfying the following conditions.

  \underline{Data:}
  \begin{compactenum}
    \item A functor~$F\colon \CatC \to \CatD$ (the \emph{left adjoint});
    \item A functor~$G\colon \CatD \to \CatC$ (the \emph{right adjoint});
    \item A natural isomorphism~$\tau : \Hom_{\CatD}(F - , - ) \Longrightarrow \Hom_{\CatC}(- , G - )$
  \end{compactenum}
\end{ctdefinition}

\begin{remark}
Note that $\tau$ is a natural isomorphism between functors of the form
$$
\CatC^{op} \times \CatD \longrightarrow \Cat{Set}
$$
\end{remark}



\section{Adjunctions: (co)unit description}

Recall: in a category \CatC, a morphism~$\mora \colon \obja \to \objb$ is an isomorphism if there exists a~$\morb \colon \obja \rightarrow \objb$ such that
\begin{equation*}
  f \then g = \id_\obja \text{ and } g \then f = \id{id}_y
\end{equation*}

In this chapter we will look at the category \Category of categories, and at two weakenings of the notion of isomorphism in this setting.
Given functors
\begin{center}
\includesag{091_functors_adjunctions}
\end{center}
instead of requiring the equations
\begin{equation*}
  F \then G = \id_\CatC  \text{ and } G \then F = \id_\CatD
\end{equation*}
we will replace the equality symbols with 2-morphisms:

\begin{equation*}
  F \then G \overset{\eta}{\Longleftarrow} \id_\CatC, \quad  G \then F \overset{\epsilon}{\Longrightarrow} \id_\CatD
\end{equation*}
The last two relationships can also be depicted in the following more geometric manner:
\begin{center}
\includesag{091_2_mor_a}
  \hspace{1cm}
\includesag{091_2_mor_b}
\end{center}

\begin{ctdefinition}[Equivalence of categories]
  \label{def:cat-equivalence}
  Let \CatC and \CatD be categories. An \emph{equivalence} between \CatC and \CatD is the following data:
  \begin{compactenum}
    \item A functor~$F\colon  \CatC \to \CatD$;
    \item A functor~$G\colon \CatD \to \CatC$;
    \item Natural isomorphisms~$\eta \colon \id_\CatC \Rightarrow F \then G$ and~$\epsilon\colon  G \then F \Rightarrow \id_\CatD$.
  \end{compactenum}
\end{ctdefinition}


\begin{ctdefinition}[Adjunction, Version 2]\label{def-adj-counit}
  \label{def:cat-adjunction}
 Let \CatC and \CatD be categories. An \emph{\iindex{adjunction}} from \CatC to \CatD is given by the following data, satisfying the following conditions.

  \underline{Data:}
  \begin{compactenum}
    \item A functor~$F\colon \CatC \to \CatD$ (the \emph{left adjoint});
    \item A functor~$G\colon \CatD \to \CatC$ (the \emph{right adjoint});
    \item Natural transformations~$\eta \colon \id_\CatC \Rightarrow F \then G$ and~$\epsilon \colon G \then F \Rightarrow \id_\CatD$
  \end{compactenum}

  \underline{Conditions:}
  \begin{compactenum}
    \item For all objects~$\obja$ of \CatC, it holds that
    \begin{equation*}
      F\eta_\obja \then \epsilon_{F\obja} = \id_{F\obja} \text{ and }  \eta_{G\objb} \then G\epsilon_\objb = \id_{G\objb}
    \end{equation*}
    \text{i.e.} that the following diagrams commute:

\begin{center}
\includesag{091_adjunction_def}
\end{center}
  \end{compactenum}
  We use the notation $F \dashv G$ to indicate that $F$ and $G$ form an adjunction, with $F$ the left adjoint and $G$ the right adjoint. 
  The 2-morphisms~$\eta$ and~$\epsilon$ are called the \emph{unit} and \emph{counit} of the adjunction.
  An adjunction is called an \emph{adjoint equivalence} if the unit and counit are natural isomorphisms.
\end{ctdefinition}

\begin{remark}
The conditions (triangle identities) from \cref{def-adj-counit} are ``hidden'' in \cref{def-adj-iso} in the condition that $\tau$ be an natural isomorphism. In \cref{relate-adj-defs} we spell out how the two definitions are related. 
\end{remark}

\section{More examples}



\section{Relating the two definitions}\label{relate-adj-defs}



