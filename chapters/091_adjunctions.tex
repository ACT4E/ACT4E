% !TEX root = ../ACT4E-full.tex

Recall: in a category \CatC, a morphism $f: x \rightarrow y$ is an isomorphism if there exists $g : x \rightarrow y$ such that
\[
  f \then g = \text{id}_x \quad \quad \text{ and } \quad \quad g \then f = \text{id}_y
\]


In this chapter we will look at the category \textbf{Cat} of categories, and at two weakenings of the notion of isomorphism in this setting.
Given functors

% https://q.uiver.app/?q=WzAsMixbMCwwLCJ4Il0sWzIsMCwieSJdLFswLDEsImYiLDAseyJjdXJ2ZSI6LTN9XSxbMSwwLCJnIiwwLHsiY3VydmUiOi0zfV1d
\[\begin{tikzcd}
    C && D
    \arrow["F", bend left=30, from=1-1, to=1-3]
    \arrow["G", bend left=30, from=1-3, to=1-1]
\end{tikzcd}\]

instead of requiring the equations
\[
  F \then G = \text{id}_C  \quad \quad \quad  \text{ and } \quad \quad \quad G \then F = \text{id}_D
\]
we will replace the equality symbols with 2-morphisms:
\

\[
  F \then G \overset{\eta}{\Longleftarrow} \text{id}_C  \quad \quad \quad \quad \quad \quad G \then F \overset{\epsilon}{\Longrightarrow} \text{id}_D
\]


\[
  \quad
  \begin{tikzcd}
    & {\mathbf{D}} \\
    {\mathbf{C}} && {\mathbf{C}}
    \arrow["F", bend left=20, from=2-1, to=1-2]
    \arrow["G", bend left=20, from=1-2, to=2-3]
    \arrow[""{name=0, anchor=center, inner sep=0}, "{\text{id}_C}"', bend right=20, from=2-1, to=2-3]
    \arrow["\eta", shorten <=7pt, shorten >=10pt, Leftarrow, from=1-2, to=0]
  \end{tikzcd}
  \quad \quad \quad
  \begin{tikzcd}
    & {\mathbf{C}} \\
    {\mathbf{D}} && {\mathbf{D}}
    \arrow["G", bend left=20, from=2-1, to=1-2]
    \arrow["F", bend left=20, from=1-2, to=2-3]
    \arrow[""{name=0, anchor=center, inner sep=0}, "{\text{id}_D}"', bend right=20, from=2-1, to=2-3]
    \arrow["\epsilon", shorten <=7pt, shorten >=10pt, Rightarrow, from=1-2, to=0]
  \end{tikzcd}
\]

\begin{ctdefinition}[]
  \label{def:cat-equivalence}
  Let \CatC and \CatD be categories. An \emph{equivalence} between \CatC and \CatD is the following data:
  \begin{compactenum}
    \item A functor $F:  \CatC \rightarrow \CatD$;
    \item A functor $G: \CatD \rightarrow \CatC $;
    \item Natural isomorphisms $\eta : \id_\CatC \Rightarrow F \then G$ and $\epsilon : G \then F \Rightarrow \id_\CatD$
  \end{compactenum}
\end{ctdefinition}


\begin{ctdefinition}[]
  \label{def:cat-adjunction}
  Let \CatC and \CatD be categories. An \emph{adjunction} between \CatC and \CatD is given by the following data, satisfying the following conditions.

  \underline{Data:}
  \begin{compactenum}
    \item A functor $F: \CatC \rightarrow \CatD$ (the \emph{left adjoint});
    \item A functor $G: \CatD \rightarrow \CatD$(the \emph{right adjoint});
    \item Natural transformations $\eta : \id_\CatC \Rightarrow F \then G$ and $\epsilon : G \then F \Rightarrow \id_\CatD$
  \end{compactenum}

  \underline{Conditions:}
  \begin{compactenum}
    \item For all objects $c$ of \CatC, it holds that
    \[
      F\eta_c \then \epsilon_{Fc} = \text{id}_{Fc} \quad \text{and}  \quad  \eta_{Gd} \then G\epsilon_d = \text{id}_{Gd}
    \]
    \text{i.e.} that the following diagrams commute:
% https://q.uiver.app/?q=WzAsNixbMCwwLCJGYyJdLFsyLDAsIkZHRmMiXSxbMiwyLCJGYyJdLFs0LDAsIkdkIl0sWzYsMCwiR0ZHZCJdLFs2LDIsIkdkIl0sWzAsMSwiRlxcZXRhX2MiXSxbMSwyLCJcXGVwc2lsb25fe0ZjfSJdLFswLDIsIlxcdGV4dHtpZH1fe0ZjfSIsMl0sWzMsNSwiXFx0ZXh0e2lkfV97R2R9IiwyXSxbMyw0LCJcXGV0YV97R2R9Il0sWzQsNSwiR1xcZXBzaWxvbl9kIl1d
    \[\begin{tikzcd}
        Fc && FGFc && Gd && GFGd \\
        \\
        && Fc &&&& Gd
        \arrow["{F\eta_c}", from=1-1, to=1-3]
        \arrow["{\epsilon_{Fc}}", from=1-3, to=3-3]
        \arrow["{\text{id}_{Fc}}"', from=1-1, to=3-3]
        \arrow["{\text{id}_{Gd}}"', from=1-5, to=3-7]
        \arrow["{\eta_{Gd}}", from=1-5, to=1-7]
        \arrow["{G\epsilon_d}", from=1-7, to=3-7]
    \end{tikzcd}\]
  \end{compactenum}
  The 2-morphisms $\eta$ and $\epsilon$ are called the \emph{unit} and \emph{counit} of the adjunction.
  An adjunction is called an \emph{adjoint equivalence} if the unit and counit are natural isomorphisms.
\end{ctdefinition}






