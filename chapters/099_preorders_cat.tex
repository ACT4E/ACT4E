

\section{A poset as a category}
\label{sec:posetsarecats}
A single poset~$\tup{P, \ordleq}$ can be described as a category, in which each point~$\obja\in P$ is an object, and there is a morphism between~$\obja$ and~$\objb$ if and only if~$\obja \ordleq \objb$. This is a ``thin'' category, wich means that there is at most one morphism
between two objects: For any~$\obja,\objb\in P$, there exist only one relation~$\obja\ordleq \objb$ in~$P$ (\cref{def:poset}). The identity morphism is given by the reflexivity property of posets, i.e. for any~$\obja\in P$, we have~$\obja \ordleq \obja$. Furthermore, composition is given by the transitivity property of posets, i.e. for~$\obja,\objb,\objc \in P$,~$\obja\ordleq \objb$ and~$\objb\ordleq \objc$ implies~$\obja\ordleq \objc$.

\begin{example}
Let's revisit \cref{ex:hasseinclusion}, in which we had a poset~$\powerset\left(\{a,b,c\}\right)$ with order given by inclusion (\cref{fig:posetascat}).
\begin{figure}[h!]
\begin{center}
\includesag{40_dpcatfig_power}
\end{center}
\caption{Power set~$\powerset{\{a,b,c\}}$ as a category. \label{fig:posetascat}}
\end{figure}
This is a category~$\CatC$, with~$\ObC=\powerset\left(\{a,b,c\}\right)$, and morphisms given by the inclusions. Note that we omit to draw self-arrows for the identity morphisms. Composition is given by the transitivity law of posets. For instance, since~$\{a\}\subseteq \{a,b\}$ and~$\{a,b\} \subseteq \{a,b,c\}$, we can say that~$\{a\}\subseteq \{a,b,c\}$.
\end{example}
