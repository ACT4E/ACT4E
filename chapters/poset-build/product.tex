
\subsection{Product of posets}
We can think of the product of posets.

\begin{definition}[Product of posets]
    \label{def:productposet}
    Given two posets~$\tup{\Obja, \ordleq_{\Obja}}$
    and~$\tup{\Objb, \ordleq_{\Objb}}$, the \emph{product poset} is~$\tup{\Obja \times \Objb, \ordleq_{\Obja\times \Objb}}$, where~$\Obja \times \Objb$ is the Cartesian product of two sets (\cref{def:cartesian-product}) and the order~$\ordleq_{\Obja \times \Objb}$ is given by:
    \begin{equation}
        \tup{\obja_1, \objb_1}
        \ordleq_{A\times B}
        \tup{\obja_2, \objb_2}
        \quad
        \Leftrightarrow
        \quad
        (\obja_1 \ordleq_\Obja \obja_2) \wedge
        (\objb_1 \ordleq_\Objb \objb_2).
    \end{equation}
\end{definition}
Recalling the pizza recipes example, we have the two posets representing time and money. Given that we want to minimize both time and costs, by considering the money poset containing elements \unit[1]{\USD}, \unit[2]{\USD}, and \unit[3]{\USD}, and the time poset containing elements \unit[1]{h}, and \unit[2]{h}, one can represent the product as in~\cref{fig:productpizza}.

\begin{figure}[h!]
    \begin{center}
        \includesag{70_hasse_pizza_product}
    \end{center}
    \caption{Product poset of time and cost for pizza recipes.\label{fig:productpizza}}
\end{figure}


\begin{example}
    Consider now the two posets given in~\cref{fig:composing_posets_1}.
    \begin{figure}[h!]
        \begin{center}
            \includesag{40_exposet_1}
        \end{center}
        \caption{Two posets. \label{fig:composing_posets_1}}
    \end{figure}
    Their product is depicted in~\cref{fig:composing_posets_2}.
    \begin{figure}[h!]
        \begin{center}
            \includesag{40_exposet_2}
        \end{center}
        \caption{Product of two posets. \label{fig:composing_posets_2}}
    \end{figure}
\end{example}
