% !TEX root = ../ACT4E-full.tex
One of the challenges of modern engineering is designing complex systems out
of physical or logical components belonging to diverse heterogenous domains.
For example, designing a system such as as a fleet of autonomous taxis
requires thinking about sensing, perception, actuation, planning, control,
energetics, communication, etc. The feasibility of the system depends on all
these aspects in a way that cannot be decoupled.

These complex systems to be engineered are just too complicated to
fit into one human mind, because the solution space to explore is enormous,
and the dependencies between different components can be very complex.
Therefore, there is a need for a formal, ``computable'' \emph{theory of design} to
support or replace human decisions in the design of large complex systems.
The theory must provide a language applicable across different disciplines;
it must support interconnection of systems, and hierarchical composition/decomposition of systems of systems. Finally,
it must be able to lead to tractable optimization problems.

Censi~\cite{Censi2017} began work towards a mathematical theory of co-design,
an axiomatic formalization in which a \emph{design problem} is characterized
as a relation between functionality provided and resources required. Design
problems can be composed in directed graphs, in which the edges represent
the dependencies, or co-design constraints, between the different components
of a system. Under certain assumptions, it is possible to solve the induced
optimization problems of finding the best design, in terms of maximizing
functionality or minimizing resources.

This paper presents a formalization and a large generalization based on
category theory. The category theory perspective allows one to cleanly
formalize the various ideas of composition and refer to well-known
constructs that are easily generalized in a variety of directions. In particular, we are able to capture many notions of composition
of systems.


\begin{figure}[h!]
    \begin{equation}
        \label{eqn:chassis}
        \includesag{20_chassis}
    \end{equation}
    \caption{
        We will be able to derive a formal semantics for
        diagrams such as this, which describes the design of a robotic
        chassis in terms of its components and the \emph{co-design constraints} among
        them. In particular, we are able to elegantly account for the
        \emph{recursive} co-design constraints that make the optimization problems
        coupled; in this case, the chassis requires a motor to move, and the motor
        requires the chassis to support it.
    }
\end{figure}


% \AC{I am going to gray out the following, maybe to put in intro,
% but in any case to de-emphasize}\gray{

% We will proceed slowly from basic definitions to more advanced higher
% category theory. We make the paper self-contained by providing all relevant
% definitions as we go, so that it will make sense to readers without previous
% exposure to category theory. At all steps we give an interpretation of the
% mathematical structures in the context of engineering design.


This paper is completely self-contained: We do not assume that the readers
have any background in category theory, and we will hold hands from very
basic definitions to advanced higher-order category constructs. We hope that
the subject is compelling enough that it can be a ``gateway drug'' for
engineers to learn a little category theory.


%We dare write for two very different audiences:
%
%\begin{enumerate}
%    \item We write for \textbf{engineers}: we want them to appreciate the power of category
%    theory as a formal language for formalizing and solving very complex design problems.
%    Our hope is that... \XXX \JT{We recommend you read Sections~\ref{?}.}
%    \item We write for \textbf{mathematicians}: we want to explain how certain higher-order
%    constructs of category theory
%    have an intuitive semantics when interpreted in the context of engineering design.
%    We believe that sometimes it is good to know that something is \emph{useful}
%    in addition to being \emph{beautiful}. \JT{Read this paper starting from Section~\ref{?}.}
%\end{enumerate}


% Systems thinking breaks down into three parts: 1) functional decomposition, understanding the solution space, and optimization over the solution space. Equilibrium in systems dynamics as a special case of optimization.

% Engineering goes like this: given a functional specification, build the optimal thing which provides that functionality. Optimal can mean cheapest, fastest, most reliable, etc., so the `optimal thing' could be the cheapest building, the fastest robot, or the most reliable manufacturing process. To put it another way: engineers solve \emph{design problems} that provide desired functionalities using a set of minimal resources.

% We wish to address this paper not only to engineers but also to category theorists, especially those who are contemplating higher category theory. A pedagogical motivation for higher category theory, if we need it: complex numbers $x + iy$ are useful because we can add, subtract, multiply, and divide them. The set of all complex numbers, $\mathbb{C}$, is useful because we can describe transformations on sets of complex numbers. The category of Hilbert spaces, $\Hilb$, is useful because we analyze structures and operations on not just on $\mathbb{C}$, but $\mathbb{C}^n$ for all $n \geq 1$. And finally, the 2-category of Hilbert spaces is useful for describing and manipulating correlations between quantum systems.
%
% Again, all of the above categorical mumbo-jumbo is a complete description to category-theorists, so it is not only written for show. Our goal in the remainder of the paper is two-fold: to explain the engineering application to category theorists, and to unpack the category theory for everyone else.

% Our approach will be to start by giving the category-theoretic description of this category---it is compact closed, locally posetal, etc.---and slowly unpacking all of these abstract terms into very concrete, set-theoretic mathematical statements and their proofs. We will trace a running example throughout the paper.
%

\subsection{Related work}


\gray{To write. I thought here we could give some pointers to other papers
in the same style that describe some structure used by engineers or physicisists in a categorical way.

Then we can say two nice things about this paper:
    \begin{itemize}
        \item In addition, our paper is also of a tutorial nature, for people
        who do not already know category theory.
        \item Moreover, this theory is not only ``descriptive'', but also
        ``prescriptive'', in the sense that we can use these results
        to create optimization algorithms (see ``Computation'' section.)
    \end{itemize}
}

\subsection{Notation}

\label{sec:notn_and_background}


We assume the reader to be familiar with everything in this section.

The empty set is denoted $\emptyset$.

We write $\pair{a}{b}$ to denote an ordered tuple with two elements.

We write $\{*\}$ to denote any fixed choice of one-element set.

We denote a map from a set~$A$ to a set~$B$ by~$f\colon A\to B$.

We denote the identity map $A\to A$ by $\id_A$.

If we are considering a finite set with four elements, say, we will often depict this situation graphically by

[INSERT FIGURE].

In other words, the line enclosing the dots is meant to indicate that the four dots are assembled together as the elements of a set.


% \AC{The following is very confusing if you don't already know what one is referring to.}
% \DS{It's fine with me if you erase it. I think I added it because of \cref{ex:category_of_sets}.}
% A set is called \emph{small} if it is contained in some fixed universe; this issue will not concern us. Anyone who is interested can check out \cite{Shulman--sets}. The only thing that matters to us is that there is a set of small sets; it is not small.
