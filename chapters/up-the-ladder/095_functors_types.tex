% !TEX root = standalone.tex


\section{Full and faithful functors}
\begin{ctdefinition}[Full and faithful functors]
    \label{def:functorfullfaith}
    A functor~$F\colon \CatC \to \CatD$ is \emph{full} (respectively \emph{faithful}) if for each pair of objects~$\obja,\objb\in \CatC$, the function
    \begin{equation}
        F\colon \Hom_\CatC(\obja,\objb)\to \Hom_\CatD(F(\obja),F(\objb))
    \end{equation}
    is surjective (respectively injective).
\end{ctdefinition}

\begin{example}
    Let~$\CatC$ be the category depicted in \cref{fig:ex_full_faithful_1}. Let~$E\colon \CatC\to \CatC$ be the endofunctor which maps object~$\Obja\in \Ob_{\CatC}$ to~$\Obja$ and object~$\Objb\in \Ob_{\CatC}$ to~$\Obja$. This functor is full and faithful. Note that the map of objects and the map of morphisms are neither surjective nor injective.

    \begin{figure}[h!]
        \centering
        \includesag{095_full_faithful_1}
        \caption{\label{fig:ex_full_faithful_1}}
    \end{figure}
\end{example}

\begin{example}
    Consider a functor which maps a category \CatC to its preorder structure \CatD, i.e. to a category with the same objects as \CatC and a \emph{unique} morphism from~$x\in \Ob_\CatD$ to~$y\in \Ob_\CatD$ if and if only if there ist at least one morphism from~$x$ to~$y$ in \CatC.
\end{example}
\todo{OK, so what? examples? why is it useful?}


\section{Forgetful functor}

\AC{What is the definition of a forgetful functor?}

\JL{There is not a formal definition of a ``forgetful functor''; I think this lemma doesn't make sense as stated. We could instead state/explain how ``forgetting structure'' is often functorial, and include,  among other things, the Pos --> Set example. }
\begin{example}
    The functor~$\Pos \to \Set$ is a forgetful functor. This functor maps every poset to the set which has the same elements, but no notion of order. Furthermore, it maps each monotone map between posets to the corresponding function between sets. This is a forgetful functor in the sense that it forgets the notion of order and monotone maps.
\end{example}

\todo{Add more examples}
