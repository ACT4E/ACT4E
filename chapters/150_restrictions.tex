% !TEX root = ../CategoricalCoDesign.tex
% \section{Thinking about restrictions and alternatives}
\subsection{Union of Design Problems}
Let $f\colon \F{A}\tickar \R{B}$ and $g\colon \F{A}\tickar \R{B}$ be design problems. We define the \emph{union} $f \vee g$ to be the design problem which is feasible whenever \emph{either} $f$ or $g$ is feasible. This models $f$ and $g$ as interchangeable technologies: Either one can replace the other.

\begin{definition}[Union of design problems]
Given the two design problems $f \colon \F{A} \tickar \R{B}$ and $g \colon \F{A} \tickar \R{B}$, their \emph{union} $f \vee g \colon \F{A} \tickar \R{B}$ is defined by
\begin{equation}
\begin{aligned}
(f \vee g)\colon \F{A}\op \times \R{B} & \toinPos \Bool \\
\tup{\F{a}^*, \R{b}} & \mapsto f(\F{a}^*, \R{b}) \vee g(\F{a}^*, \R{b}),
\end{aligned}
\end{equation}
and represented as in~\cref{fig:uniondp}.
\begin{figure}[h!]
\begin{center}
    \includesag{52_union}
\end{center}
\caption{Diagrammatic representation of the union of design problems. \label{fig:uniondp}}
\end{figure}
\end{definition}

\begin{example}
Jeb's Spaceship Parts is locked in a deadly rivalry with Starshow Bob to supply engines for the new X103 space orbiter. Neither knows the exact operational scenario that the X103 will encounter, but have provided a range of performance benchmarks for their engines (\cref{fig:exunion_1}).
\begin{figure}[h!]
\begin{center}
\includesag{50_rival_jeb_bob}
\end{center}
\caption{Example of two engine producers. \label{fig:exunion_1}}
\end{figure}
Back at NASA headquarters, Beau has uploaded Jeb and Bob's data in order to construct the design problem reported in~\cref{fig:exunion_2}.
\begin{figure}[h!]
\begin{center}
\includesag{50_rival_beau}
\end{center}
\caption{Example of the union of the engine design problems. \label{fig:exunion_2}}
\end{figure}
\end{example}

\subsection{Intersection of Design Problems}
Given two design problems $f, g \colon \F{A} \tickar \R{B}$, we can define a design problem $f \wedge g$ that is feasible if only if $f$ and $g$ are both feasible. We call $f \wedge g$ the \emph{intersection} of $f$ and $g$. One interpretation of $f \wedge g$ is that $f$ and $g$ are two slightly different models of the same process, and we want to make sure that the design is conservatively feasible for both models.

\begin{definition}[Intersection of design problems]
Given design problems $f\colon \F{A} \tickar \R{B}$ and $g\colon \F{A} \tickar \R{B}$,
their \emph{intersection} is denoted $(f \wedge g)\colon \F{A} \tickar \R{B}$, defined by
\begin{equation}
	\begin{aligned}
		(f \wedge g)\colon \F{A}\op \times \R{B} & \toinPos \Bool \\
		\tup{\F{a}^*, \R{b}} & \mapsto f(\F{a}^*, \R{b}) \wedge  g(\F{a}^*, \R{b}),
	\end{aligned}
\end{equation}
and represented as in~\cref{fig:intersectiondp}.

\end{definition}
\begin{figure}[h!]
\begin{center}
    \includesag{52_intersection}
\end{center}
\caption{Diagrammatic representation of the intersection of design problems. \label{fig:intersectiondp}}
\end{figure}

We can directly generalize the intersection $f \wedge g$ by allowing $f$ and $g$ to have different domain and codomains, $f \colon \F{A} \tickar \R{B}$ and $g \colon \F{C} \tickar \R{D}$. We call this putting two design problems ``in parallel''.

