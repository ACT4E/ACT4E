% !TEX root = ../ACT4E-ready.tex
% \section{Thinking about alternatives}
\label{sec:coproductset}
\subsection{Coproduct}

There exists a ``dual'' notion to ``product'' that is called ``coproduct''. Just like the notion of categorical product generalized the definition of the cartesian product of two sets, the categorical coproduct generalizes the definition of the \emph{disjoint union} of two sets. 
%We'll start by illustrating the idea behind coproducts using an example. Suppose that we are considering two vending machines: from the first one can buy food and from the second one a drink. We can think of this situation in terms of resources, by saying that having \transmuted{money} is enough to get \transmuted{food} via the first machine, or to get a \transmuted{drink} via the second machine (\cref{fig:vending_1}).
%
%\begin{figure}[h!]
%    \centering
%    \includesag{60_vending_1}
%    \caption{Alternative ways to use \transmuted{money}. \label{fig:vending_1}}
%\end{figure}
%
%From this we would like to conclude that we can use \transmuted{money} to buy \textbf{either} \transmuted{food} \textbf{or} a \transmuted{drink} (\cref{fig:vending_2}).
%
%\begin{figure}[h!]
%    \centering
%    \includesag{60_vending_2}
%    \caption{We can use \transmuted{money} to buy either \transmuted{food} or a \transmuted{drink}.\label{fig:vending_2}}
%\end{figure}
\begin{comment}
Suppose that we are considering a hybrid car that contains two engines: an
electric engine and an internal combustion engine. Both can produce \transmuted{motion}, but each from a different source of energy. The electric engine uses \transmuted{electric energy}; the internal combustion engine uses \transmuted{gasoline}. The situation is as in \cref{fig:e16a}.

\begin{figure}[h!]
    \centering
    \includesag{30_dpcatfig_e14}
    \caption{Alternative ways to generate $\mathsf{motion}$. \label{fig:e16a}}
\end{figure}

From this we would like to conclude that we can obtain \textsf{motion} from \textbf{either} \textsf{gasoline} \textbf{or} \textsf{electric energy} (\cref{fig:e16b}).

\begin{figure}[h!]
    \centering
    \includesag{30_dpcatfig_e15}
    \caption{We can generate $\mathsf{motion}$ from either $\mathsf{gasoline}$ or $\mathsf{electric} \ \mathsf{energy}$. \label{fig:e16b}}
\end{figure}

To define the idea of ``\textbf{either} \transmuted{food} \textbf{or} \transmuted{drink}'' we can refer to the idea of disjoint union of sets (\cref{def:disjoint-union}). 
\end{comment}

Recall that given sets~$\Obja$ and~$\Objb$, their disjoint union~$\Obja + \Objb$ is a set that contains a distinct copy of~$\Obja$ and~$\Objb$ each. If an element is contained in both~$\Obja$ and~$\Objb$, then there will be two distinct copies of it in the disjoint union~$\Obja +\Objb$.

\begin{definition}[Disjoint union of sets]
\label{def:disjoint-union}
    The \emph{disjoint union} (or sum) of two sets~$\Obja$ and~$\Objb$ is
    \begin{equation}
        \Obja + \Objb=\{\tup{1,\obja}\mid \obja\in \Obja\}\cup\{\tup{2,\objb}\mid \objb\in \Objb\}.
    \end{equation}
\end{definition}

\begin{example}
Consider the sets~$\{\star, \diamond\}$ and~$\{\ast, \dagger\}$. Their disjoint union can be represented as in \cref{fig:disjoint}.
\begin{figure}[h!]
    \begin{center}
    \includesag{30_disjoint_union}
    \end{center}
    \caption{Example of a disjoint union of sets. \label{fig:disjoint}}
\end{figure}
\end{example}

We can define the disjoint union of a set with itself; this corresponds to having two distinct copies of the set~(\cref{fig:disjointself}).

\begin{figure}[h!]
\begin{center}
\includesag{30_disjoint_union_self}
    \caption{Disjoint union of a set with itself \label{fig:disjointself}.}
\end{center}
\end{figure}

As mentioned above, the disjoint union is a particular instance -- in the category \Set  --  of the notion of ``coproduct''.
We will now give the definition of a coproduct in an arbitrary category. Note that it is very similar to the definition that we gave, in the previous section, for the product -- but with a few twists. Analogous remarks to those we gave following the definition of the product apply here!
 
\begin{ctdefinition}[Coproduct]
\label{def:catcoproduct}
Let~$\CatC$ be a category and let~$\Obja, \Objb \in \Ob_{\CatC}$ be objects. The \emph{coproduct} of~$\Obja$ and~$\Objb$ consists of the following constituent data, satisfying the following condition. \\
\underline{Data:}
\begin{compactenum}
\item an object~$Z \in \Ob_{\CatC}$ \ (``the coproduct'' of $\Obja$ and $\Objb$)
\item \emph{injection morphisms}~$\iota_1 \colon \Obja \to Z $ and~$\iota_2 \colon \Objb \to Z$
\end{compactenum}
\underline{Condition:} 
\begin{compactenum}
\item For any~$T \in \Ob_{\CatC}$ and any morphisms $\mora \colon  \Obja \to T, \morb \colon \Objb \to T$, there exists a \emph{unique} morphism~$\psi_{\mora, \morb} \colon Z \to T$ such that~$\mora = \iota_1\then \psi_{\mora, \morb}$ and~$\morb = \iota_2 \then \psi_{\mora,\morb}$.
\end{compactenum}
\end{ctdefinition} 


\begin{remark}
Diagrammatically, the condition above states that diagrams of this form commute: 
\begin{equation}
\includesag{60_defcoproduct}
\end{equation}
\end{remark}

\begin{remark}
Similarly as was the case with the categorical product, ``the coproduct'' of $\Obja$ and $\Objb$ is unique only ``up to isomorphism''. Nevertheless, we will usually simply write $\coprodMap{\Obja}{\Objb}$ for ``the'' coproduct (in place of $Z$ above), and we will usually write $\coprodMap{\mora}{\morb}$ in place of $\psi_{\mora, \morb}$.
\end{remark}

%
%Note that~$\coprodMap{X}{Y}$ is different from~$\coprodMap{Y}{X}$, but the two are isomorphic~(\cref{fig:e16}).
%
%\begin{figure}[h!]
%    \centering
%    \includesag{30_dpcatfig_e16}
%    \caption{$\coprodMap{X}{Y}$ and~$\coprodMap{Y}{X}$ are isomorphic. \label{fig:e16}}
%\end{figure}


%For the case of vending machines, the inclusion maps are as in \cref{fig:inclusionvending}.
%
%\begin{figure}[h!]
%    \centering
%    \includesag{60_inclusion_vending}
%    \caption{Inclusion maps for the vending machine example. \label{fig:inclusionvending}}
%\end{figure}

\begin{example}
Let's consider two battery producers, each producing specific battery technologies. The first producer is able to produce a set~$A=\{\transmuted{LiPo}, \transmuted{LCO},\transmuted{NiH2}\}$ of technologies, and the second one a set~$B=\{\transmuted{LFP},\transmuted{LMO},\transmuted{LiPo}\}$. Each technology has a specific price, belonging to a set of prices~$P=\{\unit[50]{USD},\unit[60]{USD},\unit[70]{USD},\unit[80]{USD}\}$. We specify the price mappings for different technologies via the functions~$f\colon A\to P$ and~$g\colon B\to P$. A battery vendor wants to sell batteries from both producers and wants to create a battery catalogue, which needs to take into account which technology comes from which producer, to be able to distribute the earnings from the sales fairly. To this end, the disjoint union of the sets of technology is considered:
\begin{equation*}
    A+B=\{\tup{1,\transmuted{LiPo}},\tup{1,\transmuted{LCO}},\tup{1,\transmuted{NiH2}},\tup{2,\transmuted{LFP}},\tup{2,\transmuted{LMO}},\tup{2,\transmuted{LiPo}}\}.
\end{equation*}
It is possible to map each technology in~$A,B$ to its own representative in~$A+B$ via the so-called injection maps:
\begin{equation*}
    \begin{aligned}
        \iota_A\colon A&\to A+B\\
        a&\mapsto \tup{1,a},\\
        \iota_B\colon B&\to A+B\\
        b&\mapsto \tup{2,a}.
    \end{aligned}
\end{equation*}
This situation is graphically represented in \cref{fig:coprod_batteries_1}, and mimics the coproduct diagram presented in \cref{def:catcoproduct}.

\begin{figure}[h!]
    \centering
    \scalebox{0.75}{
    \includesag{60_coprod_batt_1}}
    \caption{Battery technologies, companies, prices, and a catalogue.}
    \label{fig:coprod_batteries_1}
\end{figure}


Here, the universal property says that there is a \textbf{unique} function~$\coprodMap{f}{g}\colon A+B\to P$ such that 
\begin{equation*}
    \iota_A\then (f+ g)=f \text{ and }\iota_B\then (f+ g)=g.
\end{equation*}
If we take a~$x\in A+B$ is either ``from~$A$ or from~$B$'':
\begin{equation*}
    \text{either } \exists a\in A:x=\iota_A(a) \text{ or }\exists b\in B:x=\iota_B(b).
\end{equation*}
From this, we can deduce that the desired map~$f+g$ is:
\begin{equation*}
\begin{aligned}
    \coprodMap{f}{g} \colon  A + B &\to P \\
    x &   \mapsto
    \begin{cases}
        f(x),& \text{if } x=\iota_A(a),\quad a \in A, \\
        g(x),& \text{if } x=\iota_B(b),\quad b \in B.
    \end{cases}
\end{aligned}
\end{equation*}
This is a specific example of \Set/\FinSet, in which the coproduct is a generalization of the concept of disjoint union. Now, we could spontaneously ask ourselves: why does the union not ``suffice'' for the coproduct definition in \Set? To see this, let's consider the same situation as before, but now having the catalogue of technologies given by~$A\cup B$ (\cref{fig:coprod_batteries_2}). The interpretation of maps $f,g$ does not change, and injections work as depicted. Note, however, that when asked for a map from the technology~$\transmuted{LiPo}\in A\cup B$, we have no notion of the company which produces it, and we are therefore unsure whether to assign it to~$f(\transmuted{LiPo})=\unit[50]{USD}$ or~$g(\transmuted{LiPo})=\unit[80]{USD}$. Indeed, the unique map~$f+g$ required by the universal property of the coproduct cannot exist, since in case $A\cap B\neq \emptyset$, any element~$x\in A\cap B$ should be simultaneously sent to $f(x)$ and $g(x)$.

\begin{figure}[h!]
    \centering
    \scalebox{0.75}{
    \includesag{60_coprod_batt_2}}
    \caption{Example of why the union is not the coproduct in \Set.}
    \label{fig:coprod_batteries_2}
\end{figure}
\end{example}


\begin{example}
Given~$X,Y\in \Ob_\Rel$ (so $X$ and $Y$ are sets) their coproduct is the disjoint union~$X+Y$. The disjoint union of sets comes equipped with inclusion functions~$\iota_X\colon X\to X+Y$ and~$\iota_Y\colon Y\to X+Y$. If we turn these functions into relations
\begin{equation*}
\begin{aligned}
    R_{\iota_X}&\subseteq X\times (X+Y)\\
    R_{\iota_Y}&\subseteq Y\times (X+Y).
\end{aligned}
\end{equation*}
then these are the injection morphisms for the coproduct in $\Rel$. 
As an aside, we note that in \Rel products and coproducts are \emph{both} given by the disjoint union of sets. We will see later why this is might be expected. 
\end{example}

\begin{example}
Let $m,n\in \mathbb{N}$, and draw an arrow $m\to n$ if $m$ divides $n$. For instance, 6 divides 12 and hence there is an arrow $6\to 12$. The coproduct between any two $m,n\in \mathbb{N}$ in this category is given by the least common multiple. 
\end{example}

\begin{example}
Let's consider the ordered set~$\tup{\mathbb{R},\leq}$, where given $x_1,x_2\in \mathbb{R}$ we can draw an arrow~$x_1\to x_2$ if~$x_1\leq x_2$. By following the coproduct's commutative diagram, we know that the coproduct of~$x_1$ and~$x_2$ is a $z\in \mathbb{R}$ such that
\begin{compactitem}
\item $x_1\leq z$;
\item $x_2\leq z$;
\item For all~$x\in \mathbb{R}$ with~$x_1\leq x$ and~$x_2\leq x$, we have~$z\leq x$.
\end{compactitem}
In other words, the coproduct of $x_1,x_2\in \mathbb{R}$ is given by~$\max\{x_1,x_2\}$, and is also called \emph{join}.
\end{example}

\begin{example}
Let $S$ be a set, and $X,Y\subseteq S$ subsets. We can draw an arrow $X\to Y$ if $X\subseteq Y$. By following the coproduct's commutative diagram, it is easy to see that the coproduct of~$X$ and~$Y$ is given by~$X\cup Y$, i.e., the ``smallest'' set containing both~$X$ and~$Y$.
\end{example}

\begin{example}
We can define a category \Vect in, composed of: 
\begin{compactitem}
\item \emph{Objects}: vector spaces;
\item \emph{Morphisms}: linear maps;
\item \emph{Identity morphisms}: identity maps;
\item \emph{Composition}: composition of linear maps.
\end{compactitem}
It is a good exercise to prove that \Vect really forms a category. In the following, we want to look at the coproduct in \Vect. It is given by the \emph{direct sum} of vector spaces. Recall that given vector spaces~$V$ and~$W$, their direct sum is the set
\begin{equation*}
    V\oplus W\coloneqq \{\tup{v,w}\mid v\in V,w\in W\},
\end{equation*}
equipped with a notion of addition and scalar multiplication derived component-wise from $V$ and $W$. For addition, this means that given vectors~$\tup{v_1,w_1},\tup{v_2,w_2}\in V\oplus W$, their sum in $V \oplus W$ is
\begin{equation*}
    \tup{v_1,w_1}+\tup{v_2,w_2}\coloneqq \tup{v_1+v_2,w_1+w_2}.
\end{equation*}
The injections are given by:
\begin{equation*}
    \begin{aligned}
    \iota_V\colon V&\to V\oplus W\\
    v&\mapsto \tup{v,0_W},\\
    \iota_W\colon W&\to V\oplus W\\
    w&\mapsto \tup{0_V,w},
    \end{aligned}
\end{equation*}
where~$0_V$ and~$0_W$ represent the zero vectors in~$V$ and~$W$. Let's now look at the universal property in this case, by considering any vector space $U\in \Ob_\Vect$, and linear maps~$S\colon V\to U$,~$T\colon W\to U$. The universal property says that we need a unique linear map~$S+T\colon V\oplus W \to U$ such that~$S=\iota_V\then h$ and~$T=\iota_W\then g$. By taking any $\tup{v,w}\in V\oplus W$, we can write:
\begin{equation*}
    \begin{aligned}
    h(\tup{v,w})&=h(\tup{v,0_W}+\tup{0_V,w})\\
    &=h(\iota_V(v)+\iota_W(w)) \\
    &=h(\iota_V(v))+h(\iota_W(v)) \qquad \qquad (h \text{ is linear})\\
    &=(\iota_V\then h)(v)+(\iota_W\then h)(w)\\
    &\overset{!}{=}Sv+Tw.
    \end{aligned}
\end{equation*}
We can hence write the map $S+T$ as
\begin{equation*}
    \begin{aligned}
    S+T\colon V\oplus W&\to U\\
    \tup{v,w}&\mapsto Sv+Tw.
    \end{aligned}
\end{equation*}
\end{example}

\begin{example}[Adapted from~\cite{spivak2014category}]
We can define a category of graphs \Graph. Objects of this category are graphs~$G=\tup{V,A,s,t}$, composed of:
\begin{compactitem}
\item A set of \emph{vertices} $V$;
\item A set of \emph{arrows} $A$;
\item A \emph{source} function $s\colon A\to V$, mapping each arrow to its source vertex;
\item A \emph{target} function $t\colon A\to V$, mapping each arrow to its target vertex.
\end{compactitem}
Morphisms are \emph{graph homomorphisms}. Given graphs~$G=\tup{V,A,s,t}$ and~$G'=\tup{V',A',s',t'}$, a graph homomorphism~$f\colon G\to G'$ is given by maps~$f_0\colon V\to V'$ and~$f_1\colon A\to A'$, such that the following diagrams commute:
\begin{equation*}
\includesag{60_graph_homomorphism}
\end{equation*}
Intuitively, all this is saying is that ``arrows are bound to their vertices'', meaning that if a vertex~$v_1$ is connected to~$v_2$ via an arrow~$a$, the vertices~$f_0(v_1)$ and~$f_0(v_2)$ have to be connected via an arrow~$f_1(a)$. We are now ready to define the coproduct in \Graph. Given two graphs~$G=\tup{V,A,s,t}$ and~$G'=\tup{V',A',s',t'}$, their coproduct is a graph
        \begin{equation*}
            G+G'=\tup{V+V',A+A',s+s',t+t'}.
        \end{equation*}
In~$G+G'$, an arrow connects $v_1$ to $v_2$ if:
\begin{compactitem}
\item $v_1,v_2\in V$ or $v_1,v_2\in V'$, i.e., if both vertices belong to the same graph, and
\item an arrow between from~$v_1$ to~$v_2$ exists in~$G$ or~$G'$.
\end{compactitem}
Given~$s\colon A\to V$ and~$s'\colon A'\to V'$, we have:
\begin{equation*}
        \begin{aligned}
            s+ s'\colon A+ A'&\to V+ V'\\
            x&\mapsto
                \begin{cases}
                   s(x)& \text{if } x\in A\\
                    s'(x)&\text{if } x\in A'.
                    \end{cases}
        \end{aligned}
        \end{equation*}
and
\begin{equation*}
        \begin{aligned}
            t+t'\colon A+ A'&\to V+ V'\\
            x&\mapsto
                \begin{cases}
                   t(x)& \text{if } x\in A\\
                    t'(x)&\text{if } x\in A'.
                    \end{cases}
        \end{aligned}
\end{equation*}
This is nicely graphically representable. Consider two graphs as in \cref{fig:graphs_1}.

\begin{figure}[h!]
    \centering
    \includesag{60_graph_1}
    \caption{Example of graphs for which we want to consider the coproduct. \label{fig:graphs_1}}
\end{figure}

Their coproduct is a graph including the ``disjoint union'' of both original graphs, without connecting them (\cref{fig:graphs_2}).



\begin{figure}[h!]
    \centering
    \includesag{60_graph_2}
    \caption{Example of coproduct of graphs. \label{fig:graphs_2}}
\end{figure}

\end{example}