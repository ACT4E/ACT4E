% !TEX root = ../../ACT4E-full.tex

\subsection{to put back up}


\begin{ctdefinition}[\iindex{Strong monoidal functor}]
    Let~$\tup{\CatC,\otimes_\CatC,I_\CatC}$ and~$\tup{\CatD,\otimes_\CatD,I_\CatD}$ be two monoidal categories. A \emph{strong monoidal functor} between \CatC and \CatD is given by:
    \begin{enumerate}
        \item A functor
        \begin{equation}
            F\colon \CatC\to \CatD;
        \end{equation}
        \item An isomorphism
        \begin{equation}
            \epsilon\colon I_\CatD\to F(I_\CatC);
        \end{equation}
        \item A natural isomorphism
        \begin{equation}
            \mu_{x,y}\colon F(x)\otimes_\CatD F(y) \to F(x\otimes_\CatC y),\quad \forall x,y\in \CatC,
        \end{equation}
    \end{enumerate}
    satisfying the following conditions:
    \begin{enumerate}
        \item[a)] \emph{Associativity}: For all objects~$x,y,z\in \CatC$, the following diagram commutes.
        \begin{equation}
            \includesag{120_natural_associativity}
        \end{equation}
        where~$a^\CatC$ and~$a^\CatD$ are called \emph{associators}.
        \item[b)] \emph{Unitality}: For all~$x\in \CatC$, the following diagrams commute:
        \begin{equation}
            \includesag{120_natural_unitality}
        \end{equation}
        where~$l^\CatC$ and~$r^\CatC$ represent the left and right \emph{unitors}.
    \end{enumerate}
\end{ctdefinition}


\begin{figure}[h!]
    \centering
    \begin{subfigure}{0.2\textwidth}
        \centering
        \includesag{50_sum_series}
        \caption{Series:~$(f \then g)$.}
    \end{subfigure}
    \hspace{10mm} % add space between figures
    \begin{subfigure}{0.2\textwidth}
        \centering
        \includesag{50_sum_parallel}
        \caption{Parallel: $f \otimes g$.}
    \end{subfigure}
    \hspace{10mm} % add space between figures
    \begin{subfigure}{0.2\textwidth}
        \centering
        \includesag{50_sum_biproduct}
        \caption{Biproduct: $f + g$.}
    \end{subfigure}
    \hspace{10mm} % add space between figures
    \begin{subfigure}{0.2\textwidth}
        \centering
        \includesag{50_sum_loop}
        \caption{Loop: $\Tr f$.}
    \end{subfigure}
    \label{fig:diagrams}
\end{figure}

\begin{table}[t!]
    \centering
    \begin{tabular}{c|c|c|crl}
        series &
        $f:A\profto B$ &
        $g:B\profto C$ &
        $f\then g:$ & $A$ & $\profto C$ \\
        %
        sum &
        $f:A\profto B$ &
        $g:A\profto B$ &
        $f\vee g:$ & $A$ & $\profto B$ \\
        %
        intersection &
        $f:A\profto B$ &
        $g:A\profto B$ &
        $f\wedge g:$ & $A$ & $\profto B$ \\
        %
        monoidal product &
        $f:A\profto C$ &
        $g:B\profto D$ &
        $f\otimes g:$ & $A\times B$ & $\profto C \times D$ \\
        %
        product &
        $f:A\profto C$ &
        $g:A\profto D$ &
        $f\times g:$ & $A $ & $\profto C + D$ \\
        %
        coproduct &
        $f:A\profto C$ &
        $g:B\profto C$ &
        $f\sqcup g:$ & $A + B $ & $\profto C$ \\
        %
        biproduct &
        $f:A\profto B$ &
        $g:A\profto B$ &
        $f+ g:$ & $A + A$ & $\profto B + B$ \\
        %
        trace &
        $f: C \times A \profto C \times B$ &
        - &
        $\Tr_{A,B}^C(f) :$ & $A$ & $\profto B$
    \end{tabular}
    \caption{Various composition operations on design problems (i.e. morphisms) in \DP.}
\end{table}
