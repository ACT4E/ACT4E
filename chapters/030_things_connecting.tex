% !TEX root = ../CategoricalCoDesign.tex

%\section{Thinking about how things connect to each other}
Currency categories illustrated how one can use category theory to think about things transforming into each other. In this section, we want to think about how things connect to each other.

\subsection{Distribution networks}

Consider the type of networks that arise for example in the context of electrical power grids. In a simplified model for a certain region or country, we may have the following kinds of components: power plants (places where electrical power is produced), high voltage transmission lines and nodes, transistor stations, low voltage transmission lines and nodes, and consumers (e.g. homes and businesses). 

[INSERT FIGURE]

To model the connectivity between the components of the power grid, we now draw arrows between components that are connected. We set the direction of the arrows to flow from energy production, via transmission components, to energy consumption. 

[INSERT FIGURE]


A possible question one asks about such a power distribution network is: which consumers are serviced by which power sources? For example, some power sources, such as a solar power plant, may fluctuate due to weather conditions, while other power sources, such as a nuclear power plant, may shut down every once in a while due to maintenance work. We also will want to know the connectivity structure of transmission lines. For example, some lines may go down during a storm, and we want to ensure enough redundancy in our system. 

To see which consumers are connected to which power plants, we can following paths traced by sequences of arrows:

[INSERT FIGURE]

In the above example, we see that there are in fact two subsets of energy consumers who are connected to two seperate sets of power sources. (Perhaps this is the power grid of a region which has a large mountain range in the middle, or is divided into two islands!)

Below we show a schematic diagram of a power grid that is taken from [INSERT REFERENCE] \footnote{See \url{https://en.wikipedia.org/wiki/Electrical_grid} and click on the image.} Note that power plants and consumers are not depicted. 


\begin{figure}[h!]
\centering
 \includegraphics[width=0.7\linewidth]{pics/power_dist_network.png}
 \caption{A schematic view of a power grid.}
\label{fig:power_schema}
\end{figure}


We could re-arrange our model example above to resemble this image as follows:

[INSERT FIGURE]

A very basic mathematical notion which underlies the diagram [REF] is the notion of a \textbf{binary relation}. 

[INSERT DEF OF RELATIONS]





\subsection{Path planning}
\label{sec:trekking}


Supposed we are tasked with managing a scientific mission for a Mars rover. In particular, we need need to plan the route that the rover will take in order to travel from its landing position ``$a$'' to a target destination ``$b$''. We have a map of the relevant region on Mars, complete with elevation data, but, of course, only has a certain degree of accuracy. 

To model the landscape, we divide it into a grid of~$1 \times 1$ meter squares, and the center of each is labeled with an~$\tup{x,y}$ coordinate. In total, our model has~$1000 \times 1000$ squares, and we let~$L$ denote the set of coordinates of the centers of the squares. We think of these coordinate labels as the objects of a category~$\cat{C}$ which are the possible (approximate) locations that the rover might be. If the rover is at a given location~$l = \tup{x,y}$, then in our model there are eight possible directions that the rover can move:

\

[INSERT FIGURE]

\


If we draw such arrows for each location label $c \in \cat{C}$, then we obtain a directed graph which looks like this: 

\

[INSERT FIGURE]

\

To model the possible paths the rover might potential travel, we take the free category on this graph. That is, we let the morphisms in our category $\Cat{C}$ be all possible paths obtained by concatenating directed arrows from the above graph [REF]. In particular, given the locations~$a$ and~$b$, we have 
\begin{equation}
\Cat{C}(a, b) = \{ \text{paths from } a \text{ to } b \}.
\end{equation}

\

[INSERT FIGURE]

\

In [REF Figure], some possible paths are drawn in green. Note that by allowing \emph{all} possible paths we are also allowing ones of infinite length (\text{e.g.} where the rover moves around indefinitely long). Since we want our rover to reach its destination in a finite amount of time, we will subsequently take $\Cat{C}$ to be the category where the morphisms are only the paths of \emph{finite length}. 

Next, we include the elevation information in our model, in order to start to optimize the planning of which path we wish the rover to take. We encode elevation data as a function~$h \colon L \to \mathbb{R}$ which assigns a real number to each location label. If~$l$ and~$l'$ are two locations, the absolute elevation difference between~$l$ and $~l'$ is~$\vert h(l') - h(l) \vert$. For our rover, we only want to consider paths such that the absolute elevation difference between any two adjacent locations along that path is less than a given threshold (if the path is too steep, the rover might tip over!). This is one kind of constraint which determines certain paths in our category~$\Cat{C}$ to be infeasible. 

Among the feasible paths, we wish to optimize the path taken by the rover so that it uses the least amount of energy, say.  For any path~$p$ of finite length, we model what the energy cost~$E(p)$ of that path for the rover would be. (We skip the precise details of how we might model this; surely the energy cost of a path will be related to the length of the path, for example.) This defines a function on each of the homsets of our category $\Cat{C}$, 
\begin{equation}
E \colon \Cat{C}(l, l') \longrightarrow \mathbb{R}, 
\end{equation}
which, to each path from~$l$ to~$l'$, assigns the corresponding energy cost of that path. Thus our specific optimization problem is to find those feasible paths which minimize the function~$E\colon \Cat{C}(a,b) \to \mathbb{R}$ defined on paths from the starting position~$a$ to the target position~$b$. 







\

\gray{
Consider a geographical region whose locations are expressed through coordinates~$(x,y)\in \mathbb{R}^2$, \text{e.g.} as given by a map of that region. Furthermore, consider a function~$\mathsf{alt}: \mathbb{R}^2 \to \mathbb{R}_{\geq 0}$ which, for a known location, returns its altitude.

We can think about this situation using a category, call it~$\mathbf{trek}$, where objects are geographical locations~$\tup{x,y}\in \mathbb{R}^2$ and morphisms are continuous paths between them. The identity morphism for each location consists of the trivial path (i.e., not moving), and composition is given by concatenation of paths.
\JL{We need to be more precise about what ``continuous path'' means here! The typical mathematical definition of paths from topology is as a function of a (``time'') parameter, and leads to a well-known situation where concatenation is not an associative operation on the nose... and/or there is also issue that there are crazy kinds of continuous paths, such as space-filling curves... perhaps this example can be modified a bit to capture the basic idea, but avoid the math issues...}
\GZ{Agree, here we just want the connectivity and the filtering of paths which have too large inclinations}

Suppose that a human can only traverse trails which have a maximum inclination of $\alpha>0$ when going uphill and $\beta>0$ when going downhill.
We can now think of the aforementioned human, wanting to go from a location~$\tup{x,y}$ to a location~$\tup{v,w}$. Finding a path consists of finding at least a morphism in~$\Hom_\mathbf{trek}(\tup{x,y},\tup{v,w})$ satisfying the condition on the maximum inclinations~$\alpha$ and~$\beta$.


Using the terminology from \cref{sec:catsfromgraphs}, we can see that~$\mathbf{trek}$ is the free category on a graph with vertices given by geographical locations~$\tup{x,y}\in \mathbb{R}^2$ and arrows given by paths between them. In particular, a valid path~$p\colon \tup{x,y}\mapsto \tup{v,w}$ for the human to be able to reach a destination, has not to exceed the maximum inclination~$\alpha$ when climbin and the maximum inclination~$\beta$ when descending.
}

\subsection{Mobility}

For a specific mode of transportation, say a car, we can define a graph~$G_\mathrm{c}=\tup{V_\mathrm{c},A_\mathrm{c},s_\mathrm{c},t_\mathrm{c}}$, where~$V_\mathrm{c}$ represents geographical locations which the car can reach and~$A_\mathrm{c}$ represents the paths it can take (e.g. roads). Similarly, we consider a graph~$G_\mathrm{s}=(V_\mathrm{s},A_\mathrm{s},s_\mathrm{s},t_\mathrm{s})$, representing the subway system of a city, with stations~$V_\mathrm{s}$ and subway lines going through paths~$A_\mathrm{s}$, and a graph $G_\mathrm{b}=\tup{V_\mathrm{b},A_\mathrm{b},s_\mathrm{b},t_\mathrm{b}}$, representing onboarding and offboarding at airports. In the following, we want to express intermodality: the phenomenon that someone might travel to a certain intermediate location in a car and then take the subway to reach their final destination.

By considering the graph~$G=(V,A,s,t)$, with~$V=V_\mathrm{c}\cup V_\mathrm{s}\cup V_\mathrm{b}$ and~$A=A_\mathrm{c}\cup A_\mathrm{s}\cup A_\mathrm{b}$, we obtain the desired intermodality graph. Graph~$G$ can be seen as a new category, with objects~$V$ and morphisms~$A$.
\begin{example}
Consider the $\Cat{car}$ category, describing your road trip in California, with
\begin{equation*}
    V_\mathrm{c}=\{\textsf{SFO}_\mathrm{c},\textsf{S. Mateo},\textsf{Half Moon Bay},\textsf{SBP}_\mathrm{c},\textsf{Lake Balboa},\textsf{LAX}_\mathrm{c}\},
\end{equation*}
and arrows as in~\cref{fig:carcat}. The nodes represent typical touristic road trip checkpoints in California and the arrows famous highways connecting them.

\begin{figure}[h!]
\begin{center}
\includesag{30_carcategory}
\end{center}
\caption{The $\Cat{car}$ category. \label{fig:carcat}}
\end{figure}

Furthermore, consider the $\Cat{flight}$ category with $V_\mathrm{f}=\{\textsf{SFO}_\mathrm{f}, \textsf{SJC}, \textsf{SBP}_\mathrm{f}, \textsf{LAX}_\mathrm{f}\}$ and arrows as in~\cref{fig:flight}. The nodes represent airports in california and the arrows represent connections, offerend by specific flight companies.

\begin{figure}[h!]
\begin{center}
\includesag{30_flight}
\end{center}
\caption{The $\Cat{flight}$ category. \label{fig:flight}}
\end{figure}

We then consider the $\Cat{board}$ category, with nodes
\begin{equation*}
    V_\mathrm{b}=\{\textsf{SFO}_\mathrm{f},\textsf{SFO}_\mathrm{c},\textsf{SBP}_\mathrm{f},\textsf{SBP}_\mathrm{c},\textsf{LAX}_\mathrm{f},\textsf{LAX}_\mathrm{c}\}
\end{equation*}
and arrows as in~\cref{fig:boarding}. Nodes represent airports and airport parkings, and arrows represent the onboarding and offboarding paths one has to walk to get from the parkings to the airport and vice-versa.

\begin{figure}[h!]
\begin{center}
\includesag{30_boarding}
\end{center}
\caption{The $\Cat{board}$ category. \label{fig:boarding}}
\end{figure}

The combination of the three, which we call the \emph{intermodal graph}, can be represented as a graph, with \textcolor{red}{red} arrows for the car network, \textcolor{blue}{blue} arrows for the flight network, \textcolor{green}{green} arrows for the boarding network, and black dashed arrows for intermodal morphisms, arising from composition of morphisms involving multiple modes (\cref{fig:intermodal}). Imagine that you are in the parking of \textsf{LAX} airport and you want to reach \textsf{S. Mateo}. From there, you will e.g. onboard to a \textsf{United} flight to \textsf{SFO}$_\mathrm{f}$, will then offboard reaching the parking lot \textsf{SFO}$_\mathrm{c}$, and drive on highway \textsf{US-101} reaching \textsf{S. Mateo}. This is intermodality.

\begin{figure}[h!]
\begin{center}
\includesag{30_intermodal}
\end{center}
\caption{Intermodal graph. The dashed arrows represent intermodal morphisms, and we depict just one of them for simplicity. \label{fig:intermodal}}
\end{figure}
\end{example}

The intermodal network category $\Cat{intermodal}$ is the free category on the graph illustrated in \cref{fig:intermodal}.

\GZ{Maybe not needed? If so, rewrite}
\JL{I think this is nice to include! But then maybe for clarity we would want to actually remove the dashed arrow in Figure 18, since that is not part of the graph we are thinking to take the free category of. Technically we could also leave it in, but that seems to confuse the situation, no?}

\subsection{The category $\Rel$}

\begin{remark}[Opposite relations]
\label{remark:oppositerel}
Any relation~$R\colon A\to B$ induces an opposite relation (or transpose relation, reverse relation) \begin{equation}
    R^\intercal =\{\tup{b,a}\in B\times A\colon \tup{a,b}\in R\} \subseteq B\times A.
\end{equation}
Note tha~ $\left( R^\intercal\right)^\intercal = R$
\end{remark}

\begin{definition}[Category~$\Rel$]
    The category~$\Rel$ is defined by:
    \begin{enumerate}
    \item \emph{Objects}: The objects of this category are all sets.
    \item \emph{Morphisms}: The morphisms between any pair of sets~$X, Y$
    are relations~$R\subseteq X\times Y$.
    \item \emph{Identity morphism}: The identity morphism for the set~$X$
    is the diagonal morphism~$\delta_X \colon X\to X\times X$.
    \item \emph{Composition of morphisms}: The composition of two relations~$R \colon X\to Y$,~$S\colon Y\to Z$ is given by
    \begin{equation}
    R \then S \coloneqq \{\tup{x,z} \equiv  \exists y \in Y, \ \left(\tup{x,y} \in R\right) \wedge \left(\tup{y,z} \in S\right)\}.
    \end{equation}
    \end{enumerate}
\end{definition}
\begin{remark}
The opposite category (\cref{def:oppositecat}) of~$\Rel$ has the same objects of~$\Rel$ and as morphisms its opposite relations (\cref{remark:oppositerel}).
\end{remark}

\begin{lemma}
    The category~$\Cat{Set}$ is a subcategory (\cref{def:subcategory}) of~$\Cat{Rel}$.
\end{lemma}
\begin{proof}
	We need to prove the conditions presented in \cref{def:subcategory}.
	\begin{enumerate}[(i)]
	\item If a morphism of~$\Rel$ $f \colon X\to Y$ is in~$\Set$, then so are the objects~$X$ and~$Y$. Both~$\Rel$ and~$\Set$ have sets as objects. If a morphism~$f\colon X\to Y$ of~$\Rel$ is in~$\Set$, then a relation~$F\subseteq X\times Y$ between set~$X$ and set~$Y$ exists. This relation can be expressed in~$\Set$ as~$f\colon X\to Y$ and hence the objects~$X$ and~$Y$ exist.
	\item Two morphisms~$f\colon X\to Y$ and~$g\colon Y\to Z$ in~$\Rel$ are relations~$F\subseteq X\times Y$,~$G\subseteq Y\times Z$. If they are in~$\Set$, they can be written as functions~$f\colon X\to Y$ and~$g\colon Y\to Z$ and their composition is in~$\Set$ as well. 
	\item If an object of~$\Rel$ $X$ is in~$\Set$, then so is the identity morphism~$\text{Id}_X$. This is true: for every object~$X$ of~$\Set$ there exists the identity morphism~$\text{Id}_X\colon X\to X$. The diagonal morphism can be expressed as~$\text{Id}_X\colon X\to X$.
	\end{enumerate}
\end{proof}

\begin{comment}
\begin{shaded*}
\begin{definition}[Union of categories]
Given two categories $\CatC,\CatD$, one can create the \emph{union} $\Cat{E}$ of the two, which is composed of:
\begin{compactenum}
\item \emph{Objects:} $\Ob_\Cat{E}=\Ob_\CatC \cup \Ob_\CatD$.
\item \emph{Morphisms:} A morphism $f$ is given by considering the followng. If $f\in \Hom_\CatC(X,Y)$, then $f\in \Hom_\Cat{E}(X,Y)$. If $f\in \Hom_\CatD(X,Y)$, then $f\in \Hom_\Cat{E}(X,Y)$.
\item \emph{Identity morphism:} For any morphism in $\Cat{E}$, the identity morphism remais the same as in the original category.
\item \emph{Composition operation}: The composition of morphisms remains the same.
\end{compactenum}
\end{definition}
\end{shaded*}
\end{comment}

%\subsection{Generating categories from graphs}
%What we sketched is the previous sections has deeper roots. In the following, we will introduce the concept of \emph{free categories on graphs}.
%
%\begin{shaded*}
%\begin{definition}[Free category on a graph]
%Consider any graph $G=(V,A,s,t)$. We can define a category $\mathbf{Free}(G)$, called the \emph{free category on $G$}. Its objects are the vertices $V$, and given vertices $a\in V$ and $b\in V$, the morphisms $\mathbf{Free}(G)(a,b)$ are the paths from $a$ to $b$.
%%A path is a sequence of ``consecutive'' edges, \text{i.e.} the source of a subsequent edge is equal to the target of its predecessor. We also formally allow for ``empty paths'', \text{i.e.} a sequence of "zero"-many edges which starts and ends at the same vertex.
%The identity morphism $id_a$ is defined to be the empty path starting and ending at a vertex $a \in V$, and composition of morphisms is given by concatenation of paths.
%\end{definition}
%\end{shaded*}
%
%
%With these two new definitions, we can see that $\mathbf{trek}$ is the free category on a graph with vertices given by geographical locations $\tup{x,y}\in \mathbb{R}^2$ and arrows given by paths between them. In particular, a valid path $p\colon \tup{x,y}\mapsto \tup{v,w}$ for the human to be able to reach a destination, has not to exceed the maximum inclination $\alpha$ when going up and the maximum inclination $\beta$ when going down.
