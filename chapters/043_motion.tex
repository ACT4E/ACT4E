% !TEX root = ../ACT4E-ready.tex
\subsection{Dynamical systems and monoids}

\JL{inserting this here as an un-baked idea for a subsection. maybe it could be the first subsection of this chapter; that way idendity laws and associative laws can be introduced before talking about categories}

What are the simplest kinds of mathematical models of a dynamical system that we can think of? 

One possible answer is something like this: we can describe a dynamical system as a set $S$ of possible states, together with a description of how states change over time. For the latter, consider time to be labeled by distinct ``points in time''. Then, we can just think in terms of time-steps, e.g. seconds, or we can think of points in time where e.g. an action is triggered and the system passes to a new state. 

One thing we want to describe is how the state of our system changes over time, and in particular from one moment in time to the next. For any time step, we will not assume that we know what specific state the system is in, but rather we will describe, at once, all possible evolutions during that time step, i.e. we consider all possible initial conditions at once. Given two consecutive moments in time, we might describe the possible changes in the system by a function $T : S \rightarrow S$, which maps each state $s \in S$ to a next state $T(s) \in S$. This is a deterministic change of state: given $s$, the function $T$ determines the next state $T(s)$. The function $T$ is like a rule. Let's call $T$ an ``evolution operator'', because it describes how the system states might evolve over a time step. 

We might want to consider various possible evolution operators. We could consider functions $T_a$, $T_b$, $T_c$, etc. We can also compose these functions: given $T_a$ and $T_b$, we might have, over the course of two time steps, the change described by $T_a \circ T_b$. For simplicity, let's suppose we work with three evolution operations $T_a$, $T_b$, and $T_c$.

\

-> introduce semigroups (implicitly or explicitly)

\


-> introduce monoids 






