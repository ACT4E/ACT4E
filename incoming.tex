\part{Homework}
%\partfirstb{}{placeholder}{}{}



\subsection{Operations}


\begin{gradedexercise}[Composition]
  Implement the composition operations by implementing the interface in \cref{lst:FiniteMapOperations}.
\end{gradedexercise}

\classlisting{FiniteMapOperations}


\begin{gradedexercise}[Obtain relation]
  Given a FiniteMap, obtain the relation.

  \methodsource{FiniteMapOperations}{as_relation}{}

\end{gradedexercise}
\subsection{Disjoint union}
\begin{gradedexercise}
  Given two finite sets, compute the disjoint union.

\end{gradedexercise}














\section{Enriched categories}

\subsection*{Interface}

\classsource{FiniteEnrichedCategory}{}

\subsection*{Representation}



\begin{comment}
\chapter{Legos}
\label{ch:exercises-legos}

\section{Legos}

\subsection*{Representation}


The format is shown in \cref{fig:parts1}.

\datafilefig{parts1}{parts1.parts.yaml}{fig:parts1}

\subsection{Assemblies}

\begin{gradedexercise}
  Given a set of Lego blocks, find out if they are rigidly connected.
\end{gradedexercise}

\begin{gradedexercise}
  Given a set of Lego blocks, find out if they are stable.
\end{gradedexercise}


\begin{gradedexercise}
  Given a set of Lego blocks, find a stable assembly of them.
\end{gradedexercise}

\begin{gradedexercise}
  Given a set of Lego blocks, find a plan to assembled them using 1 robot hand.
\end{gradedexercise}

\begin{gradedexercise}
  Given a set of Lego blocks, find a plan to assemble them using n robot hands.
\end{gradedexercise}

\subsection{Planning}

\todotext{planning problems with legos}


\end{comment}
